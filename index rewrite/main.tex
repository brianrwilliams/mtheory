\documentclass[reprint,
%superscriptaddress,
%groupedaddress,
%unsortedaddress,
%runinaddress,
%frontmatterverbose, 
%preprint,
%preprintnumbers,
%nofootinbib,
%nobibnotes,
%bibnotes,
amsmath,
amssymb,
aps,
%prl,
pra,
%prb,
%rmp,
%prstab,
%prstper,
%floatfix,
]{revtex4-2}
\usepackage{fullpage}
\usepackage{amsmath}
\usepackage{amssymb}
\usepackage{amsfonts}
\usepackage{amsthm}
\usepackage{enumerate}
\usepackage[titletoc]{appendix}
\usepackage{hyperref}

%\usepackage{tikz} 	  % Replaced with tikz-cd, but am
\usepackage{tikz-cd}  % willing to learn tikz if necessary
	\usetikzlibrary{decorations.pathmorphing}

\usepackage[all]{xy}
\usepackage{stmaryrd}
\usepackage{indentfirst}
\usepackage{graphicx}
\usepackage{tabularx}

\usepackage{macros}

\renewcommand\tabularxcolumn[1]{m{#1}}
\newcolumntype{Y}{>{\centering\arraybackslash}X}


\begin{document}

\title{A holographic approach to the six-dimensional superconformal index}

\author{Surya Raghavendran}
\email{}

\maketitle

\section{Introduction}
Given a superconformal field theory, its superconformal index is a generating function for counts of BPS local operators; as a protected quantity, it can be computed exactly at weak-coupling. In this letter, we revisit the problem of describing the superconformal index of the six dimensional $\mc N= (2,0)$ theories of type $A_{N-1}$ from its holographic description.
%%HOLOGRAPHIC INDICES
The superconformal index is defined through a trace over a radially quantized BPS Hilbert space. Specializing to the $\frac{1}{16}$-BPS sector, 
\begin{equation}
\mc I (q, y_a, u) = \operatorname{tr}_{\mc H} \left ( (-1)^F  x^\Delta q^{H + \frac13 (J_{12} + J_{34} + J_{56} ) } y_1^{J_{12}} y_2^{J_{34}}y_3^{J_{56}} u^{R_{12}-R_{34}}  \right )
\end{equation}

In the above, $Q$ denotes any of the supertranslations preserved by the 6d $\mc N=(2,0)$ theory, and the ... denote generators of the Cartan subalgebra of the maximal bosonic subalgebra of the complexified 6d $\mc N=(2,0)$ superconformal algebra $\mf {so}(8) \oplus \mf{sp}(2) \subset \mf {osp}(8|4)$. 


%%TWISTED HOLOGRAPHY


%%



%\input{content\}

\begin{thebibliography}{99}
 
\end{thebibliography}

\end{document}
