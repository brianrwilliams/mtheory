\section{The non-minimal twist}\label{s:nonmin}

We have provided numerous consistency checks that the 11-dimensional theory defined on a manifold with $SU(5)$ holonomy is a twist of supergravity. 
We have referred to this theory as minimal as it depends on the complex structure in the maximal number of directions. 
In this section we characterize a further twist of 11-dimensional supergravity from the lens of the holomorphic theory.  
This further twist is invariant for the group $G_2 \times SU(2)$ and is topological along seven directions, as opposed to just a single direction as in the $SU(5)$ twist. 

On flat space, the further twist is essentially unique and renders the theory topological in seven directions, rather than just one as in the holomorphic twist. 
We will show that it is equivalent to a theory on $\CC^2 \times \RR^7$ that we call ``Poisson'' Chern--Simons theory. 
In the BV formalism, the theory $\ZZ/2$ graded and has fields given by
\[
A \in \Pi \Omega^{0,\bu}(\CC^2) \hotimes \Omega^\bu(\RR^7) ,
\]
where $\Pi$, as always, denotes parity shift.
The equations of motion are of the form
\[
\dbar A + \d_{\RR^7} A + \partial_{z_1} A \wedge \partial_{z_2} A = 0 .
\]
The action functional depends on the holomorphic symplectic structure on $\CC^2$ through the Poisson bracket on the algebra of holomorphic functions.
We give a precise definition below. 

The main result of this section is the following.

\begin{thm}
\label{thm:nonmin}
The non-minimal twist of the 11-dimensional theory is equivalent to Poisson Chern--Simons theory on 
\[
\CC^2 \times \RR^7 .
\]
\end{thm}

From the point of view of the untwisted theory, the non-minimal twist is defined by working in a background where the fermionic ghost in the physical theory is equal to a supertranslation of the form
\[
Q + Q_{nm} 
\]
where $Q$ is the supertranslation which defines the minimal twist, see \S \ref{sec:mintwist}.
The minimal twist of supergravity is obtained by setting a fermionic ghost equal to $Q$. 

In the language of the minimal twist, the supercharge $Q_{nm}$ determines a square-zero element in the $Q$-cohomology of the original supersymmetry algebra (which we will denote by the same letter). 
The characterization of this cohomology in Proposition \ref{prop:susycoh} implies that $Q_{nm}$ is an element 
\[
Q_{nm} \in \wedge^2 L 
\]
where $L \cong \CC^5$ is the defining $SU(5)$ representation. 
In other words, $Q$ is a translation invariant holomorphic two-form on $\CC^5$. 
The condition that $[Q_{nm}, Q_{nm}] = 0$ simply says that $Q_{nm}\wedge Q_{nm} = 0$ as a translation invariant four-form on $\CC^5$. 
By a linear change of coordinates, all such two-forms $Q$ are of the form $Q_{nm} = \d z_i \wedge \d z_j$ where $i,j=1,\ldots, 5$.

From hereon in this section we will rename coordinates by
\[
\CC^5 \times \RR = \CC^2_{z_i} \times \CC^3_{w_a} \times \RR
\]
which is most natural from the point of view of the non-minimal twist. 
We will fix the non-minimal supercharge 
\[
Q_{nm} = \d z_1 \wedge \d z_2 .
\]
Notice that this choice of supercharge breaks the holonomy of the 11-dimensional theory from $SU(5)$ to $SU(3) \times SU(2)$. 

\subsection{Index matching}
\label{sec:indexcheck}

As a first consistency check, we can compare deformation invariants attached to the holomorphic twist and the nonminimal twist. We will find that the local character of the latter agrees with a specialization of the local character computed in \ref{sec:locchar}

\begin{prop}
The  local character of the nonminimal twist of 11d supergravity on flat space is given by
\[
\prod _{(n_1,n_2)\in \Z^2_{\geq 0}} \frac{1}{1-q^{-n_1+n_2}}.
\] 
This agrees with the specializaiton of the local character computed in proposition \ref{prop:locchar}.
\end{prop}
\begin{proof}
The space of solutions to linearized equations of motion is parametrized by a holomorphic function $A$ on $\C^2_{w_j}$. The corresponding linear local operators are labeled by $(n_1,n_2)\in \Z^2_{\geq 0}$  and are given by 
\[
A_{(n_1,n_2)} : A \mapsto \partial_{w_1}^{n_1}\partial_{w_2}^{n_2} A (0).
\]

The character of the linear span of these is given by the geometric series
\beqn\label{nonmin:singleparticle}
\sum _{(n_1,n_2)\in \Z^2_{\geq 0}} q^{-n_1+n_2}
\eeqn
with plethystic exponential given by 
\[
\prod _{(n_1,n_2)\in \Z^2_{\geq 0}} \frac{1}{1-q^{-n_1+n_2}}.
\]

For the last part, it suffices to observe the specialization at the level of single particle indices. A natural choice of fugacities for $SU(3)\times SU(2)$ is given in terms of the fugacities $q_i$ for $SU(5)$ chosen in \ref{sec:locchar} by requiring the additional constraints \[q_1q_2q_3 = 1, \ \ \ q_4q_5=1.\]
After imposing the above constraints, the single particle index \ref{singleparticleindex} is
\[
i(q) = \frac{1}{(1-q)(1-q^{-1})}
\]
where $q = q_4=q_5^{-1}$. This is exactly the sum of the geometric series \ref{nonmin:singleparticle}

\end{proof}

%Notice that changing the values of $i,j$ just has the affect of permuting the holomorphic copies of $\CC^2$ leftover in the further twist. 

%The Lie algebra of gauge symmetries of this model on flat space is 
%\[
%\Omega^{0,\bu} (\CC_i \times \CC_j) \hotimes \Omega^\bu(\RR^7) 
%\]
%which is quasi-isomorphic to $\cO^{hol}(\CC^2)$ equipped with the Poisson bracket $\{-,-\}_{pb}$. 
%
%More generally, the twist can be defined \brian{finish}

\subsection{The non-minimal global symmetry algebra}

We constructed an embedding of the $Q$-cohomology of the supersymmetry algebra into the fields of our 11-dimensional theory on $\CC^5 \times \RR$. 
The further twist is obtained by working in a background where a certain field on $\CC^5 \times \RR$ takes nonzero value $Q_{nm}$. 
Explicitly, the element $Q_{nm} \in \wedge^2 L$ corresponds to the image under $\del$ of a $\gamma$-field of type $\Omega^{1,0}(\CC^5) \otimes \Omega^0(\RR)$. 
According to the embedding in \S \ref{s:residual} this is the $\gamma$-field 
\beqn\label{eqn:gammanm}
\gamma_{nm} = \frac12 (z_1 \d z_2 - z_2 \d z_1) \in \Omega^{1,0}(\CC^5) \otimes \Omega^0(\RR) 
\eeqn
Notice that $\del \gamma_{nm} = \d z_1 \wedge \d z_2$ as desired.

\parsec[sec:nmsymmetry]
Before proceeding to the proof of the theorem above, we perform a simple calculation of the global symmetry algebra present in the $Q_{nm}$-twisted theory. 

Recall that up to a copy of constant functions, the global symmetry algebra of the holomorphic twist of the 11-dimensional theory is the super Lie algebra $E(5,10)$.
From this point of view, the global symmetry algebra of the $Q_{nm}$-twisted theory is given by deformation of this super Lie algebra by the Maurer--Cartan element 
\[
\d z_1 \wedge \d z_2 \in \Omega^{2}_{cl}(\CC^5) .
\]
We recall that the space of closed two-forms on $\CC^5$ is precisely the odd part of the super Lie algebra $E(5,10)$. 

We compute the cohomology of $E(5,10)$ with respect to the differential which is bracketing with this Maurer--Cartan element. 
Recall that we are using the holomorphic coordinates $(z_1,z_2,w_1,w_2,w_3)$ on $\CC^5$. 

There are the following brackets in the super Lie algebra $E(5,10)$ 
\begin{align*}
[f_l \partial_{z_l} , \d z_1 \wedge \d z_2] & = \del f_i \wedge \d z_j - \del f_j \wedge \d z_i \\
[g_a \partial_{w_a} , \d z_1 \wedge \d z_2] & = 0 \\
[h^{ab} \d w_a \wedge \d w_b , \d z_1 \wedge \d z_2 ] & = \ep_{abc} h^{ab} \partial_{w_c} .
\end{align*}
where $f_l \partial_{z_l}$, $g_a \partial_{w_a}$ are divergence-free vector fields on $\CC^5$ and $h^{ab} \d w_a \wedge \d w_b$ is a closed two-form. 

From these relations, we see that the following elements are in the kernel of $[\d z_1 \wedge \d z_2, -]$:
\begin{itemize}
\item $f(z_i) \partial_{z_1} + g(z_i) \partial_{z_2}$ for holomorphic functions $f,g$ on $\CC_{z_1} \times \CC_{z_2}$ which satisfy 
\[
\del_{z_1} f + \del_{z_2} g = 0 .
\]
In other words, this is a divergence-free vector field on $\CC_{z_1} \times \CC_{z_2}$. 
\item $f_b(z_i, w_a) \partial_{w_b}$ for $f_b$ a holomorphic function on $\CC^5$ where $b=1,2,3$. 
\end{itemize}
It is immediate to check that these are the only nonzero elements in the kernel. 
Further, any element of the second type is clearly exact by the closed two-form $\ep^{ijklm} f \d z_l \d z_m$. 

Thus, the cohomology is the (purely bosonic) Lie algebra of divergence-free vector fields on $\CC^2 = \CC_i \times \CC_j$
\[
H^\bu\big(E(5,10), [\d z_1 \wedge \d z_2, -] \big) \simeq \Vect_0(\CC^2) .
\]

We proved in Theorem \ref{thm:global} that the global symmetry algebra of the 11-dimensional theory on $\CC^5 \times \RR$ is equivalent to a central extension $\Hat{E(5,10)}$ of the super Lie algebra~$E(5,10)$. 

The Lie algebra of divergence-free vector fields on $\CC^2$ also admits a central extension:
\beqn\label{eqn:centralvect}
0 \to \CC \to \cO (\CC^2) \to \Vect_0 (\CC^2) \to 0
\eeqn
where $\cO(\CC^2)$ is equipped with the Poisson bracket with respect to the symplectic form~$\d z_1 \wedge \d z_2$.
These two central extension are compatible. 

%Thus, the Lie algebra of gauge symmetries of Poisson Chern--Simons theory on $\CC_i \times \CC_j \times \RR^7$ is a trivial one-dimensional central extension of the cohomology of $E(5,10)$. 

\begin{prop}
Let $\Hat{E(5,10)}$ be the central extension of $E(5,10)$ which is equivalent to the global symmetry algebra of the 11-dimensional theory on $\CC^5 \times \RR$. 
Then, there is an isomorphism of Lie algebras 
\[
H^\bu \big(\Hat{E(5,10)} , [\d z_1 \wedge \d z_2, -] \big) \simeq \cO(\CC^2) .
\]
\end{prop}
\begin{proof}
The only thing to check is that, in cohomology, the cocycle defining the central extension of $E(5,10)$ is the cocycle exhibiting $\cO(\CC^2)$ as the central extension of divergence-free vector fields. 
Recall that the formula \eqref{eqn:cocycle} for the cocycle is 
\[
\varphi(\mu, \mu', \alpha) = \<\mu \wedge \mu', \alpha\>|_{z=0}.
\]

In cohomology, we obtain the cocycle for divergence-free vector fields by plugging in $\alpha = \d z_1 \wedge \d z_2$. 
This gives the cocycle on $\Vect_0(\CC^2)$ 
\[
(f_i \del_{z_i}, g_j \del_{z_j}) \mapsto (f_1 g_2 - f_2 g_1)(z_1=z_2=0) .
\]
This is the cocycle defining \eqref{eqn:centralvect}, as desired. 
\end{proof}.

This proposition implies that the global symmetry algebra of the non-minimal twist of 11-dimensional supergravity is the Lie algebra $\cO(\CC^2)$. 
We will see that this is compatible with the calculation of the non-minimal twist of the full BV theory. 

\subsection{The non-minimal twist of the 11-dimensional theory}

Now, we turn to deducing the action function of the non-minimal twist and hence the proof of Theorem \ref{thm:nonmin}. 
We will show that the eleven-dimensional theory on $\CC^5 \times \RR$ placed in the background where the $(1,0)$ component of $\gamma$ takes the value $\gamma_{nm}$ \eqref{eqn:gammanm} is equivalent to a theory with a purely Chern--Simons-like action functional that we referred to in the introduction to this section. 

Poisson Chern--Siimons theory is defined on any manifold of the form
\[
Z \times M
\]
where $Z$ is a hyper K\"ahler surface and $M$ is a smooth manifold of real dimension seven. 
The fundamental field of the theory is  
\[
\alpha \in \Pi \Omega^{0,\bu}(Z) \; \Hat{\otimes} \; \Omega^\bu(M)  .
\]
Just in our original 11-dimensional theory, this theory is also only $\ZZ/2$ graded. 

The holomorphic symplectic form $\omega_Z^{2,0}$ on $Z$ induces a Poisson bracket define on all Dolbeault forms $\Omega^{0,\bu}(Z)$ which we denote by $\{-,-\}_{pb}$. 
In local Darboux coordinates $(z_1,z_2)$, this bracket reads
\[
\{\alpha^I (z,\zbar) \d \zbar_I , \alpha'^J (z,\zbar) \d \zbar_J\}_{pb} = (\partial_{z_1} \alpha^I \partial_{z_2} \alpha^J \pm \partial_{z_2} \alpha^I \partial_{z_1} \alpha^J) \d \zbar_I \wedge \d \zbar_J . 
\]
The action functional of Poisson Chern--Simons theory is 
\beqn\label{eqn:pcsaction}
    \frac12 \int_{Z \times M} (\alpha \wedge \d\alpha) \wedge \omega^{2,0}_Z  + \frac16 \int_{Z\times M} \alpha \wedge \{\alpha, \alpha\}_{pb} \wedge \omega^{2,0}_Z
\eeqn
where $\{-,-\}$ is the Poisson bracket induced from the symplectic form $\omega_Z$ on $Z$. 

For simplicity, we will work only on flat space $\CC^5 \times \RR = \CC^2_z \times (\CC^3_w \times \RR)$, where we view $Z = \CC^2_z$ as a hyper K\"ahler manifold with its standard holomorphic symplectic form $\omega^{2,0} = \d^2 z$.

We will decompose the fields according to these coordinates. 
For example, we decompose the $\mu$-field as $\mu = \mu_z + \mu_w$ where
\begin{align*}
\mu_z  &\in \PV^{1,\bu}(\CC^2_z) \otimes \PV^{0,\bu}(\CC^3_w) \otimes \Omega^\bu(\RR) \\
\mu_w & \in \PV^{0,\bu}(\CC^2_z) \otimes \PV^{1,\bu}(\CC^3_w) \otimes \Omega^\bu(\RR)  .
\end{align*}
and similarly $\gamma = \gamma_z + \gamma_w$. 
We will also use the notation $\del^z$ for the holomorphic de Rham differential along $\CC_z^2$ and similarly $\del^w$ for the holomorphic de Rham differential along $\CC^3_w$. 

To twist, we expand near the background where the field $\gamma_z$ takes value $\gamma_{nm}$ as in \eqref{eqn:gammanm}. 
This will generate new kinetic and interacting terms. 
%which we can extract by inserting a formal parameter $\delta$ and expressing the action functional in terms of the deformed field $\Tilde{\gamma} = \gamma + \delta \gamma^{1,0}$.

There are two types of interactions in the original theory.
The first is
\begin{equation}\label{eqn:int1}
  \frac12 \int_{\CC^2 \times \CC^3 \times \RR} \frac{1}{1-\nu} \left(\del \gamma \vee \mu^2 \right) \wedge (\d^2 z \wedge \d^3 w)
\end{equation}
and the second is
\begin{equation} \label{eqn:int2}
  \frac16\int_{\CC^2 \times \CC^3 \times \RR} \gamma \partial \gamma \partial \gamma .
\end{equation}

%We can integrate Equation (\ref{eqn:int1}) by parts to put it in the form $\frac12 \int_{X \times Z \times L} \left[(\partial \gamma) \vee (\mu \wedge \mu) \right]$ where $\mu \wedge \mu$ is the wedge product of polvector fields.\brian{there might be some factors I'm being sloppy with here}
Expanding \eqref{eqn:int1} around the background where $\gamma$ takes value $\gamma_{nm}$, we obtain,
%\[
%  \frac12 \int \frac{1}{1-\nu} \left(\del \gamma \vee \mu^2 \right) \wedge (\omega_Z \wedge \Omega_W) + \frac{\delta}{2} \int \frac{1}{1-\nu} \left(\omega_Z \vee \mu^2 \right) \wedge (\omega_Z \wedge \Omega_W) .
%\]
%Here, we have used the equation of motion $\partial \gamma^{1,0} = \Omega_Z$.
%It will be convenient to further expand this into the components $\mu_W, \mu_Y, \gamma_W, \gamma_Y$:
\begin{multline}
 \int \frac{1}{1-\nu} \left(\frac12 \del^w \gamma_w \vee \mu_w^2  + \del^z \gamma_w \vee \mu_w \mu_z + \del^w \gamma_z \vee \mu_w\mu_z + \frac12 \del^z \gamma_z \vee \mu_z^2 \right) \wedge (\d^2 z \wedge \d^3 w) 
 \\
  + \frac{1}{2} \int \frac{1}{1-\nu} \left(\d^2 z \vee \mu_z^2 \right) \wedge (\d^2 z \wedge \d^3 w) .
  \label{eqn:delta1}
\end{multline}

We similarly expand (\ref{eqn:int2}),
%\[
%  \frac16 \int \gamma \partial \gamma \partial \gamma + \frac{\delta}{2} \int \left(\gamma \partial \gamma\right) \wedge \omega_Z .
%\]
%Notice that there are no $\delta^2$ terms since $\partial \gamma^{1,0} \partial \gamma^{1,0} = 0$.
%Again, we further expand this into holomorphic components along $X,Z$:
\beqn
\frac16 \int \left(\gamma_w \partial^z \gamma_w \partial^z \gamma_w +\gamma_w \partial^w \gamma_w \partial^z \gamma_z +  \gamma_w \partial^w \gamma_z \partial^w \gamma_z \right) + \frac{1}{2} \int \left(\gamma_w \partial^w \gamma_w \right) \wedge \d^2 z
\label{eqn:delta2}
\eeqn

The new terms in the non-minimally twisted linearized BRST differential arise from the quadratic terms in the action in Equations \eqref{eqn:delta1} and \eqref{eqn:delta2}:
\begin{equation}\label{eqn:newterms}
  \frac{1}{2} \int (\d^2 z \vee \mu_z^2) \wedge (\d^2 z \wedge \d^3 w) + \frac{1}{2} \int \left(\gamma_w \wedge \partial^w \gamma_w \right) \wedge \d^2 z .
\end{equation}
The non-minimally twisted linear BRST complex thus takes the form
\[
  \begin{tikzcd}
  & \PV^{1,\bu}_Z \hotimes \PV^{0,\bu}_W \ar[dr, "\div^z"] \ar[dashed, rounded corners, to path={ -- ([yshift=-2ex]\tikztostart.west) |- ([xshift=-1.5ex]\tikztotarget.west) -- (\tikztotarget)}, dddddr]\\
  & & \PV^{0,\bu}_Z \hotimes \PV^{0,\bu}_W \\
 & \PV^{0,\bu}_Z \hotimes \PV^{1,\bu}_W \ar[ur, "\div^w"'] & \\
\;_{\cong}  & & \Omega^{0,\bu}_Z \hotimes \Omega^{1,\bu}_W \ar[ul, dashed, bend left = 10, "\Omega^{-1}_W \partial^w"]\\
 & \Omega^{0,\bu}_Z \hotimes \Omega^{0,\bu}_W \ar[ur, "\partial^w"] \ar[dr,"\partial^z"'] \\
  & & \Omega^{1,\bu}_Z \hotimes \Omega^{0,\bu}_W
  %\ar[uuuuul, start anchor =  {[yshift = 0ex, xshift=0ex]}, end anchor = {[yshift=1.0ex, xshift=-5ex]}, bend left = 90, dotted] .
  \end{tikzcd}
\]
Here, we write $Z = \CC^2_z$ and $X = \CC^3_w$ for notational simplicity. 

Here, the dashed arrow along the outside of the diagram corresponds to the BV antibracket with the first term in (\ref{eqn:newterms}).
It is given by the isomorphism 
\[
\Omega^{1,\bu}_Z \hotimes \Omega^{0,\bu}_W \xto{\omega^{2,0}_Z \otimes \id} \PV^{1,\bu}_Z \hotimes \PV^{0,\bu}_W
\]
induced holomorphic symplectic form on $Z$. 
The other dashed arrow corresponds to the BV antibracket with the second term in (\ref{eqn:newterms}).
It is given by the composition
\[
\Omega^{0,\bu}_Z \hotimes \Omega^{1,\bu}_W \xto{\id \otimes \del^w} \Omega^{0,\bu}_Z \hotimes \Omega^{2,\bu}_W \xto{\id \otimes \Omega_W} \PV^{0,\bu}_Z \hotimes \PV^{1,\bu}_W
\]
given by applying the holomorphic de Rham operator along $X$ followed by contracting with the inverse holomorphic volume form along $X$. 
%\brian{introduce PCS}
%Explicitly, if $f,g$ are holomorphic functions on $\CC^2$ then
%\[
%\{f(z_1,z_2) , g(z_1,z_2)\}_{pb} = \partial_{z_i} f \partial_{z_j} g - \partial_{z_j} f \partial_{z_i} g .
%\]

We replace this linear BRST complex, up to quasi-isomorphism, with a smaller BRST complex. 
Consider the complex
\beqn
\Omega^{0,\bu}_Z \hotimes \Omega^{\bu,\bu}_W \hotimes \Omega^\bu_L = \oplus_{k =0}^3 \Omega^{0,\bu}_Z \hotimes \Omega^{k,\bu}_W \hotimes \Omega^\bu_L 
%[-k] 
\eeqn
which is equipped with the differential $\dbar^z + \dbar^w + \del^w + \d_{\RR}$. 
Write $\alpha = \alpha^0 + \cdots + \alpha^3$ for a field in this complex, using the decomposition on the right hand side. 

There is a map of linear BRST complexes from this one to the original one defined by the following equations 
\begin{multline}
\mu_z = (\del_{z_1} \wedge \del_{z_2}) \vee \del^z \alpha^0, \quad \mu_w = (\del_{w_1} \wedge \del_{w_2} \wedge \del_{w_3}) \vee \alpha^2, \quad \nu = \til{\alpha}^3 \\
\beta = \alpha^0 , \quad \gamma_w = \alpha^1 , \quad \gamma_z = 0 .
\label{eqn:g2map}
\end{multline}
In the above equation we have introduced the notation $\til{\alpha}^3 = \Omega_X^{-1} \vee \alpha^3$. 

The restriction of the kinetic terms $\int \gamma (\dbar + \d_{\RR}) \mu + \beta (\dbar + \d_{\RR}) \nu$ along \eqref{eqn:g2map} is
\beqn\label{eqn:kin1}
\int \sum_{k=0}^3 \alpha^k (\dbar + \d_{\RR}) \alpha^{3-k} 
\eeqn
The restriction of the kinetic term $\int \beta \div \mu$ along \eqref{eqn:g2map} is
\beqn\label{eqn:kin2}
\int \alpha^0 \del^w \alpha^2 . 
\eeqn
Finally, the restriction of the kinetic term $\gamma \del^w \gamma$ along \eqref{eqn:g2map} is 
\beqn\label{eqn:kin3} 
\int \frac12 \alpha^1 \del^w \alpha^1 . 
\eeqn
Together, \eqref{eqn:kin1}--\eqref{eqn:kin3} give the kinetic term in Poisson Chern--Simons theory. 

This shows that \eqref{eqn:g2map} is a map of linear BRST complexes.
Applying the obvious contracting homotopy, we see that this map is a quasi-isomorphism.
We will show that the full non-linear map intertwines the action functionals up to cohomologically exact terms, and hence defines an equivalence of BV theories.

We first substitute the values for the fields in \eqref{eqn:g2map} into the original 11-dimensional action. 
The terms which are at least cubic in the action in \eqref{eqn:delta1} become
\beqn
\int \frac{1}{1-\til\alpha^3} \left(\frac12 \del^w \alpha^1 (\til \alpha^2)^2 \d^2 z + \alpha^2 \del^z \alpha^1 \del^z \alpha^0 + \alpha^3 \del^z \alpha^0 \del^z \alpha^0 \right) .
\eeqn
Here, $\til \alpha^2$ denotes the element of $\Omega^{0,\bu}_Z \hotimes \PV^{1,\bu}_W \hotimes \Omega^\bu_L$ corresponding to $\alpha^2$ determined by the Calabi--Yau form $\d^3 w$. 
There is only one cubic term left in \eqref{eqn:delta2} when we substitute the fields according to \eqref{eqn:g2map}.
It is
\beqn
\frac16 \int \alpha^1 \del^z \alpha^1 \del^z \alpha^1. 
\eeqn

Notice that the total action is of the form 
\beqn\label{eqn:pcs1}
S_{pCS} (\alpha) + \int \frac{1}{1-\til\alpha^3} \left(\frac12 \del^w \alpha^1 (\til \alpha^2)^2 \d^2 z + \til\alpha^3 \alpha^2 \del^z \alpha^1 \del^z \alpha^0 + \til\alpha^3 \alpha^3 \del^z \alpha^0 \del^z \alpha^0 \right)
\eeqn
where $S_{pCS}$ is the Poisson Chern--Simons action in \eqref{eqn:pcsaction}.

We will show that the terms not appearing in $S_{pCS}(\alpha)$ are cohomologically trivial. 
First, consider the odd local functional
\beqn\label{eqn:triv1}
\int \frac{1}{1 - \til \alpha^3} \alpha^2 (\til \alpha^2)^2 .
\eeqn
Applying the linearized BRST operator (in the non-minimal twist) this becomes 
\[
\int \frac{1}{1 - \til \alpha^3} \del^w \alpha^1 (\til \alpha^2)^2 +  \int \frac{1}{1 - \til \alpha^3} \del^w (\alpha^2) \alpha^2 (\til \alpha^2)^2 .
\]
The first term in this expression agrees the first term in parentheses in \eqref{eqn:pcs1}.
The second term is zero for symmetry reasons. 
Thus, \eqref{eqn:triv1} trivializes the first term in parentheses in \eqref{eqn:pcs1}. 

It remains to trivialize the remaining terms in parentheses. 
To do this, we fix a primitive $\eta$ of the holomorphic volume form $\d^3 w$. 
That is, $\eta$ is a holomorphic two-form on $W$ satisfying $\del \eta = \d^3 w$. 
Consider the (non-translation invariant) odd local functional
\[
\int \frac{1}{1-\til \alpha^3} \eta (\til \alpha^3)^2 \del^z \alpha^1 \del^z \alpha^0   .
\]
Using $\del^w \eta = \d^3 w$, we see that this functional trivializes the remaining terms in parentheses in \eqref{eqn:pcs1}.
This completes the argument. 

%
%Restricting the action \eqref{eqn:delta1} along the map \eqref{eqn:g2map} we obtain
%\begin{multline}
%\frac12 \alpha^1 (\alpha^2)^2 \del^X \til{\alpha}^3 + \frac12 \del^X \alpha^1 (\alpha^2)^2 
%\end{multline}
%
%\begin{multline}
%\int \frac{\omega_Z}{1-\Omega^{-1}_X \alpha^3} \left(\frac12 \del^X \alpha^1  [ \alpha^2 \vee (\alpha^2 \vee \Omega_X^{-1})] + \alpha^2 \del^Z \alpha^1 \del^Z \alpha^0\right) \\
%+ \frac16 \int \frac{\omega_Z}{1-\Omega^{-1}_X \alpha^3} \alpha^1 \partial^Z \alpha^1 \partial^Z \alpha^1 \\ + \frac{\delta}{2} \int \frac{\omega_Z}{1-\Omega^{-1}_X \alpha^3} \alpha^3 \partial^Z \alpha^0 \partial^Z \alpha^0 + \frac{\delta}{2} \int \frac{\omega_Z}{1-\Omega^{-1}_X \alpha^3} \partial^X \alpha^1 (\alpha^1 \Omega^{-1}_X \alpha^3) .
%\end{multline}
%
%More invariantly, we can write the total BV action as
%\[
%\Tilde{I} = I_{CS} + I' 
%\]
%where $I_{CS} = \frac16 \int \alpha\{\alpha,\alpha\}$ and 
%\beqn
%I' = \frac16 \int \frac{\omega_Z}{1-\Omega^{-1}_X \alpha} \partial^X \alpha [\alpha \vee (\alpha \vee \Omega_X^{-1})] + \frac16 \int \frac{\omega_Z}{1-\Omega^{-1}_X \alpha} [\alpha \vee (\alpha \vee \Omega^{-1}_Z)] \{\alpha,\alpha\} .
%\label{eqn:Iprime}
%\eeqn
%
%Consider the odd functional
%\[
%K = \frac16 \int \frac{\omega_Z}{1-\Omega^{-1}_X \alpha} \alpha \wedge [\alpha \vee (\alpha \vee \Omega_X^{-1})] .
%\]
%
%\begin{lem}
%$Q K + \{I_{CS}, K\} = I'$ .
%\end{lem}
%
%The linear differential applied to $K$ is 
%\[
%\frac16 \int \frac{\omega_Z}{1-\Omega^{-1}_X \alpha} \partial^X \alpha \wedge [\alpha \vee (\alpha \vee \Omega_X^{-1})] \omega_Z .
%\]
%This is the first term in \eqref{eqn:Iprime}. 
%
%Next, we compute
%\[
%\{I_{CS}, K\} = \frac16 \int \frac{\omega_Z}{1-\Omega^{-1}_X \alpha} [\alpha \vee (\alpha \vee \Omega^{-1}_X)] \{\alpha,\alpha\} 
%\]
%which is the second term in \eqref{eqn:Iprime}. 
