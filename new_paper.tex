\documentclass[11pt]{amsart}

\usepackage{macros-mtheory}

%% Bib stuff

\usepackage[utf8]{inputenc}
\usepackage[T1]{fontenc}
\usepackage[backend=biber,
            isbn=false,
            doi=false,
            maxbibnames=5,
            %giveninits=true,
            style=alphabetic,
            citestyle=alphabetic]{biblatex}
\usepackage{url}
\setcounter{biburllcpenalty}{7000}
\setcounter{biburlucpenalty}{8000}
\renewbibmacro{in:}{}
\bibliography{references.bib}
\renewcommand*{\bibfont}{\footnotesize}

\usepackage{xcolor}
\definecolor{e-mail}{rgb}{0,.40,.80}
\definecolor{reference}{rgb}{.40,.60,.2}
\definecolor{citation}{rgb}{0,.25,.5}

\usepackage[colorlinks=true,
            linkcolor=reference,
            citecolor=citation,
            urlcolor=e-mail]{hyperref}
\usepackage{cleveref}


%% Formatting. We can change this in the final version, just makes it easier for me to read in the meantime.

\usepackage[top=1in, bottom=1.25in, left=1.3in, right=1.3in]{geometry}
\setlength{\parskip}{8pt}
\setlength\parindent{0pt}
\setcounter{tocdepth}{2}
\numberwithin{equation}{section}
\let\emptyset\varnothing


%% Comments

\def\brian{\textcolor{violet}{BW: }\textcolor{violet}}
\def\surya{\textcolor{blue}{SR: }\textcolor{blue}}
\def\ingmar{\textcolor{magenta}{IS: }\textcolor{magenta}}

%% Random macros

\def\vectdiv{\fL}
\def\PV{{\rm PV}}
\def\div{\partial_\Omega}

%\usepackage[style = alphabetic]{biblatex}
%\addbibresource{references.bib}
%\AtBeginBibliography{\scriptsize}


\begin{document}

\title{Something about classical M-theory}


%\author{Brian R. Williams}
%\address{School of Mathematics\\
%University of Edinburgh \\ 
%Edinburgh \\ 
%UK}
%\email{brian.williams@ed.ac.uk}


\begin{abstract}
something here
\end{abstract}

\maketitle
\thispagestyle{empty}
\setcounter{tocdepth}{1}
%\tableofcontents

\section{Introduction}

\section{The $L_\infty$ algebras at play}

\subsection{Divergence-free vector fields}

Let $X$ be a Calabi--Yau manifold of dimension $n$ with holomorphic volume form $\Omega \in \Omega^n_X$. 
Let $\vectdiv_X$ be the sheaf of holomorphic divergence-free vector fields, which is of course a sub Lie algebra of $\PV^1_X$. 
Here is a resolution of holomorphic divergence-free vector fields that will be of interest to us.

\begin{dfn}
Let $\cL_X$ be the sheaf of dg Lie algebras which as a sheaf of cochain complexes is 
\[
\PV^1_X \xto{\div} \PV^0_X [-1] 
\]
concentrated in degree zero and one. 
The brackets read:
\begin{align*}
[\mu, \mu'] & = L_\mu(\mu') \\
[\mu, \nu] & = L_\mu(\nu)
\end{align*}
for $\mu,\mu' \in \PV^1_X$, $\nu \in \PV^0_X$. 
\end{dfn}

There is an obvious map of sheaves of dg Lie algebras $\vectdiv_X \to \cL_X$, which exhibits $\cL_X$ as a resolution of divergence-free vector fields. 
%\[
%\begin{tikzcd}
%\ul{0} & \ul{1} \\
%\PV^1_X \ar[r, "\partial_\Omega"] & \PV^0_X . 
%\end{tikzcd}
%\]

The dg Lie algebra $\cL_X$ is a sub dg Lie algebra of a bigger dg Lie algebra, consisting of {\em all} polyvector fields on $X$. 
The Schouten bracket 
\[
\{\cdot, \cdot\} \colon \PV^{i}_X \times \PV^{j}_X \to \PV^{i+j-1}_X 
\]
endows the sheaf of cochain complexes
\[
\left(\PV^{-\bu}_X [-1] , \div \right) \define \bigg( \PV^n_X [n-1] \xto{\div} \PV^{n-1}_X [n-2] \to \cdots \to \PV^1_X \xto{\div} \PV^0_X[-1] \bigg)
\] 
with the structure of a sheaf of dg Lie algebras.
The obvious embedding 
\[
\cL_X \hookrightarrow \PV^{-\bu}_X [-1]
\] 
is a map of dg Lie algebras. 

Consider, instead, the following sub complex of $\PV^{-\bu}_X$ 
\[
\left(\cS^\bu , \div\right) \define \bigg(\PV^{n-2}_X [n-3] \to \cdots \to \PV^1_X \xto{\div} \PV^0_X[-1] \bigg) .
\]
Notice that this is {\em not} a sub dg Lie algebra. 
Nevertheless, it is naturally a module for $\cL_X$, and we can define the quotient dg module.

\begin{dfn}
Let 
\[
\cM^\bu \define \PV^{-\bu}_X [-1] \; / \; \cS^\bu  .
\]
be the quotient dg $\cL_X$-module.
Explicitly, as a sheaf of cochain complexes $\cM_X$ is given by the two-term complex
\[
\PV^n_X [n-1] \xto{\div} \PV^{n-1}_X [n-2] .
\]
The action of $\cL_X$ on $\cM_X$ is through the Schouten bracket $\{\cdot,\cdot\}$. 
\end{dfn}

We will denote sections of this dg module by $\beta \in \PV^n_X$ and $\gamma \in \PV^{n-1}_X$. 

\begin{dfn}
Let $\fg_X$ be the sheaf of dg Lie algebras obtained by the semi-direct product of $\cL_X$ with $\cM_X$
\[
\fg_X \define \cL_X \ltimes \cM_X .
\]
\end{dfn}

\subsection{An $L_\infty$-deformation}

Suppose now that $X$ is a Calabi--Yau manifold of odd dimension $n=2k+1$. 
In this section it is also necessary to forget $\ZZ$-gradings to $\ZZ/2$-gradings. 
So, for instance, the sheaf of $\ZZ$-graded dg Lie algebras becomes the sheaf of $\ZZ/2$-graded dg Lie algebras which as a sheaf of graded vector spaces is
\[
\fg_X = \bigg(\PV^1_X \oplus \Pi \PV^0_X \bigg) \oplus \bigg(\PV^{2k+1} \oplus \Pi \PV^{2k}_X \bigg) .
\]

The first piece of business is to define a $k$-linear map which we will use to define an $L_\infty$ deformation of the dg Lie algebra $\fg_X$. 
To define it, notice that contraction with the holomorphic volume form determines an isomorphism 
\[
\Omega \vee \colon \PV^{\ell}_X \cong \Omega^{2k +1 - \ell}_X ,
\]
for any $\ell$. 
In particular, any section $\gamma \in \PV^{2k}_X$ determines a one-form $\Omega \vee \gamma \in \Omega^1_X$. 
Also, if $\omega \in \Omega^{\ell}_X$ is an $\ell$-form, let $\omega \vee \Omega^{-1} \in \PV^{2k+1-\ell}_X$ be the polyvector field satisfying
\[
\Omega \vee (\omega \vee \Omega^{-1}) = \omega .
\]


\begin{dfn}
Let
\[
\mu_k \colon \PV^{2k}_X \times \cdots \times \PV^{2k}_X \to \PV^1_X 
\]
be the $k$-linear map of sheaves defined by the formula
\[
\mu_k (\gamma_1,\ldots, \gamma_k) \define \bigg[\partial \left(\Omega \vee \gamma_1\right)\cdots \partial \left(\Omega \vee \gamma_k\right)\bigg] \vee \Omega^{-1} .
\]
\end{dfn}

\begin{lem}
For any $\gamma_1,\ldots, \gamma_k$, the vector field $\mu_k(\gamma_1,\ldots, \gamma_k)$ is divergence-free. 
\end{lem}

\begin{dfn}
\label{dfn:def}
Suppose $X$ is a Calabi--Yau manifold of dimension $2k+1$. 
Define the multinear maps of sheaves
\[
[\cdot, \ldots, \cdot]_j \colon \fg_X \times \cdots \times \fg_X \to \Pi^j \fg_X
\]
for $j=1,2,k,k+1$ by the formulas
\begin{itemize}
\item $[\cdot]_1 = \partial_\Omega$, the divergence operator;
\item $[\cdot,\cdot]_2 = \{\cdot,\cdot\}$, the Schouten bracket;
\item the only nonzero component of the $k$-ary bracket is
\[
[\gamma_1,\ldots, \gamma_k]_k = \mu_k(\gamma_1,\ldots, \gamma_k) \in \PV^1_X .
\]
\item there are two nonzero components of the $(k+1)$-ary bracket:
\[
[\gamma_1,\ldots, \gamma_k, \gamma_{k+1}]_{k+1} = \sum_{i_1 < \cdots < i_{k+1}} \gamma_{i_1} \wedge \mu_k(\gamma_{i_2},\ldots, \gamma_{i_{k+1}}) \in \PV^{2k+1}_X .
\]
and 
\[
[\nu, \gamma_1, \ldots, \gamma_k]_{k+1} = \# \nu \mu_k (\gamma_1,\ldots, \gamma_k) + \sum_{i=1}^k \bigg[\left(\Omega \vee \{\nu, \gamma_1\}\right)\cdots \Hat{\partial \left(\Omega \vee \gamma_k\right)} \cdots \partial \left(\Omega \vee \gamma_k\right) \bigg] \vee \Omega^{-1} \in \PV^{1}_X .
\]
\end{itemize}
\end{dfn}

\begin{thm}
The operations of Definition \ref{dfn:def} endow $\fg_X$ with the structure of a sheaf of $L_\infty$-algebras.
\end{thm}

\begin{proof}
Before turning on the brackets built from $\mu_k$ we know that $\fg_X$ defines the structure of a dg Lie algebra.
Thus, it suffices to check that the new $(k+1)$-ary and $(k+2)$-ary higher Jacobi relations involving $\mu_k$ are satisfied. 

The $(k+1)$-ary relations we must check are:
\begin{itemize}
\item[(1)] $\div [\gamma_1,\ldots, \gamma_k,\gamma_{k+1}]_{k+1} = \sum \{\gamma_{i_1}, [\gamma_{i_2},\ldots, \gamma_{i_{k+1}}]\}$. 
\item[(2)] $[\div \mu, \gamma_1, \ldots, \gamma_k]_{k+1} = \{\mu, [\gamma_1,\ldots, \gamma_k]_k\} + \sum [\{\mu, \gamma_{i_1}\}, \gamma_{i_2}, \ldots, \gamma_{i_{k}}]_{k}$.
\item[(3)] $\div [\nu,\gamma_1,\ldots, \gamma_k]_{k+1} = \{\nu, [\gamma_1,\ldots, \gamma_{k}]_k\}$.
\end{itemize}

\ul{(1)}:
We apply the BV relation to the left hand side.
Since $\mu_k(\gamma_1,\ldots,\gamma_k)$ is divergence-free, the left hand side is
\[
\sum \left(\div \gamma_{i_1} \wedge \mu_k(\gamma_{i_2}, \ldots, \gamma_{i_{k+1}}) + \{\gamma_{i_1}, \mu_k(\gamma_{i_2},\ldots, \gamma_{i_{k+1}})\} \right) .
\]
The second terms in the sum cancels with the right hand side of the Jacobi identity.
Thus, it suffices to show that 
\[
\sum \div \gamma_{i_1} \wedge \mu_k(\gamma_{i_2}, \ldots, \gamma_{i_{k+1}}) = 0 .
\]
Notice that this expression only depends on the divergences of the polyvector fields $\gamma_1,\ldots, \gamma_{k+1}$. 
Moreover, it is totally symmetric in the inputs. 
Thus, we can write it as $L(\div \gamma_1,\ldots, \div \gamma_{k+1})$ where
\[
L \colon \Sym^{k+1} \left(\wedge^{2k-2} V \right) \to V
\]
is some ${\rm SL}(2k+1)$-invariant map. 
Here $V$ denotes the defining ${\rm SL}(2k+1)$ representation.
As such $L= 0$ \brian{justify}.

\ul{(2)}: 
The left hand side of the Jacobi identity reads
\begin{equation}\label{eqn:3jacobi2a}
(\div \mu) \mu_k (\gamma_1,\ldots, \gamma_k) + \sum_{i=1}^k \{\div \mu, \gamma_i\} \vee \bigg[\left(\Omega \vee \gamma_1\right) \cdots \Hat{\partial \left(\Omega \vee \gamma_i\right)} \cdots \partial \left(\Omega \vee \gamma_k\right) \bigg]
\end{equation}
Consider the first term on the right hand side of the Jacobi identity. 
Expanding out, we can write this as
\begin{equation}\label{eqn:3jacobi2b}
\sum_{i=1}^k \{\mu, \div \gamma_i\} \vee \bigg[\left(\Omega \vee \gamma_1\right) \cdots \Hat{\partial \left(\Omega \vee \gamma_i\right)} \cdots \partial \left(\Omega \vee \gamma_k\right) \bigg] + \# (\div \mu) \mu_k(\gamma_1,\ldots, \gamma_k) .
\end{equation}
The final term cancels with the first term of \eqref{eqn:3jacobi2a}. 
The remaining terms on the right hand side of the Jacobi relation read
\[
\sum_{i=1}^k \partial \{\mu, \gamma_i\} \vee \bigg[\left(\Omega \vee \gamma_1\right) \cdots \Hat{\partial \left(\Omega \vee \gamma_i\right)} \cdots \partial \left(\Omega \vee \gamma_k\right) \bigg] .
\]
This cancels with the sum of the second term in \eqref{eqn:3jacobi2a} and the first term in \eqref{eqn:3jacobi2b}.
\end{proof}
\end{document}
