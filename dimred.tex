In this section we wish to demonstrate that our proposal for the interactions for twisted 11-dimensional supergravity agree with conjectural descriptions of twisted type IIA and type I supergravities due to \cite{CLSugra}. We begin by introducing certain variants of Kodaira-Spencer theory which will feature in the descriptions of twists of type IIA and type I.

\parsec[sec:SU(4)twist]

Let $X$ be a Calabi-Yau manifold. Recall the $\Z/2$ graded Poisson BV theory denoted $\cE_{KS}$ defined in section \surya{}. Let $\cE_{{mKS}}\subset \cE_{KS}$ denote the smallest subcomplex containing those fields with nonzero pairing under the Poisson tensor. Explicitly, we have that $cE_{mKS}$ is the sheaf of cochain complexes
\beqn
\begin{tikzcd}
                        &                          &                                     & {\PV^{0,\bu}}    \\
                        &                          & {\PV^{1,\bu}} \arrow[r, "u\div"]    & {u\PV^{0,\bu}}   \\
                        & {\PV^{2,\bu}} \arrow[r]  & {u\PV^{1,\bu}} \arrow[r, "u\div"]   & {u^2\PV^{0,\bu}} \\
{\PV^{3,\bu}} \arrow[r] & {u\PV^{2,\bu}} \arrow[r] & {u^2\PV^{1,\bu}} \arrow[r, "u\div"] & {u^3\PV^{0,\bu}}
\end{tikzcd}.
\eeqn

Clearly, the inclusion is a Poisson map, so restricting $I_{BCOV}$ to this subcomplex defines another $\Z/2$ graded Poisson BV theory referred to as minimal Kodaira-Spencer theory. Under the identification of Kodaira-Spencer theory as the closed string sector of the B-model, minimal Kodaira-Spencer theory should be thought of as consisting of those closed string fields present in the supergravity approximation.

With this in hand the conjecture of \cite{CLSugra} takes the following form.

\begin{conj}
  Let $M$ be a real symplectic surface and $X$ a Calabi-Yau fourfold. The $SU(4)$-invariant twist of type IIA supergravity on $M\times X$ is the $\Z/2$-graded Poisson BV theory with fields \[\Omega^{\bu}(M)\otimes \cE_{mKS}.\] The $L_{\infty}$ structure is the natural one on the tensor product of the cdga $\Omega^{\bu}(M)$ with the $L_{\infty}$-algebra $\cE_{mKS}$.
\end{conj}

\parsec[sec:IIApot]

Our goal is to compare the dimensional reduction of our 11d theory $\cE$ on $\R\times \C^{\times}\times X$ where $X$ is a Calabi-Yau 4-fold with the $SU(4)$ invariant twist of type IIA on $\R^{2}\times X$. Doing so will require some additional modifications to the above conjectural description. Indeed, recall that in the physical theory, the components of the C-field in 11d that are not supported along the M-theory circle become the components of the Ramond-Ramond 2-form of type IIA. However, as noted in \cite{CLSugra} components of Ramond-Ramond fields do not appear as fields in Kodaira-Spencer theory; rather it is components of their field strengths that appear. Since components of the C-field become components of $\gamma$ in $\cE$ \surya{hopefully we can see this in the component fields section}, this suggests that we must modify our description of the twist of type IIA to include potentials for certain fields.

Minimal Kodaira-Spencer theory with potentials on a Calabi-Yau fourfold $X$ is defined to be the $\Z/2$-graded BV theory whose space of fields is the sheaf of cochain complexes $\cE_{pot}$
\beqn
\begin{tikzcd}
& {\PV^{0,\bu}}_\eta  \\
{\PV^{1,\bu}}_\mu \arrow[r, "u\div"] & {u\PV^{0,\bu}}_\nu \\
{u^{-1}\PV^{4,\bu}}_\beta \arrow[r, "u\div"] & {\PV^{3,\bu}}_\gamma  \\
{\PV^{4,\bu}}_\theta &
\end{tikzcd}
\eeqn.

The interaction is given by \[I_{pot} = \int \frac{1}{1-\nu}(\mu\mu\del\gamma + \eta\del\gamma\del\gamma + \eta\mu\del\theta).\]


Note that there is a natural map of cochain complexes $\partial \cE_{pot}\to \cE_{mKS}$ given by the dotted arrows below:
\beqn
\begin{tikzcd}
                                       & {\PV^{0,\bu}} \arrow[rrr, dotted, "\id"] &                          &                                     & {\PV^{0,\bu}}    \\
{\PV^{1,\bu}} \arrow[r, "u\div"]       & {u\PV^{0,\bu}} \arrow[rr, dotted, "\id"] &                          & {\PV^{1,\bu}} \arrow[r, "u\div"]    & {u\PV^{0,\bu}}   \\
{u^{-1}\PV^{4,\bu}} \arrow[r, "u\div"] & {\PV^{3,\bu}} \arrow[r, dotted, "\div"]  & {\PV^{2,\bu}} \arrow[r]  & {u\PV^{1,\bu}} \arrow[r, "u\div"]   & {u^2\PV^{0,\bu}} \\
{\PV^{4,\bu}} \arrow[r, dotted, "\div"]        & {\PV^{3,\bu}} \arrow[r]          & {u\PV^{2,\bu}} \arrow[r] & {u^2\PV^{1,\bu}} \arrow[r, "u\div"] & {u^3\PV^{0,\bu}}
\end{tikzcd}.
\eeqn
This is easily seen to be a Poisson map. We have that $I_{pot} = \partial^{*}I_{BCOV}$, so we see that $I_{pot}$ satisfies the classical master equation. Therefore, $I_{pot}$ determines an $L_{\infty}$ structure on $\cE_{pot}$, and hence one on $\Omega^{\bu}(\R^{2})\otimes \cE_{pot}$. In what follows, the $SU(4)$-invariant twist of IIA will refer to the BV theory $\Omega^{\bu}(\R^{2})\otimes \cE_{pot}$.

\parsec[sec:dimred]
\begin{prop}
The $SU(4)$ invariant twist of type IIA on $\R^{2}\times X$ is the dimensional reduction of the 11d theory $\cE$ on $\R\times \C^{\times}\times X.$
\end{prop}
\begin{proof}
We use the notion of dimensional reduction as defined in \cite{}.
Let $\pi : \R\times \C^{\times}\times X \to \R^{2}\times X$ be the projection with fiber $S^{1}\subset \C^{\times}$. Note that there is an isomorphism \[\int: \pi^{*}(\Omega^{\bullet}(\R^{2})\otimes \cE_{pot})\to \cE\] given by \[(\eta, \mu, \nu, \beta,\gamma,\theta)\mapsto (\nu = \nu, \mu = \mu + \theta \vee \Omega_{X} \wedge\del_{z}, \beta = \beta, \gamma = \gamma\vee \Omega_{X} \eta dz ).\] It is clear that this isomorphism presrves the BV pairings; we need only check that $\int^{*} I$ is cohomologous to $I_{pot}$ in the deformation-obstruction complex of the free limit of $\Omega^{\bu}(\R^{2})\otimes \cE_{pot}$.

We readily compute:
\[\int^{*}I = \]
\end{proof}
