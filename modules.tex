% Created 2020-10-14 Wed 14:51
% Intended LaTeX compiler: pdflatex
\documentclass[11pt]{amsart}
\usepackage[utf8]{inputenc}
\usepackage[T1]{fontenc}
\usepackage{graphicx}
\usepackage{grffile}
\usepackage{longtable}
\usepackage{wrapfig}
\usepackage{rotating}
\usepackage[normalem]{ulem}
\usepackage{amsmath}
\usepackage{textcomp}
\usepackage{amssymb}
\usepackage{capt-of}
\usepackage{hyperref}
\usepackage{amsfonts}
\usepackage{macros-mtheory}

\date{\today}
\title{Twisted \(M\)-theory and its perturbative quantization.}
\hypersetup{
 pdfauthor={Surya and Brian},
 pdftitle={Twisted \(M\)-theory and its perturbative quantization.},
 pdfkeywords={},
 pdfsubject={},
 pdfcreator={Emacs 27.1 (Org mode 9.4)}, 
 pdflang={English}}

\def\pv{{\rm PV}}
\def\PV{{\rm PV}}
\def\T{{\rm T}}
\def\del{\partial}
\def\Vect{{\rm Vect}}

\begin{document}

\section{Modules}
Suppose $X$ is a Calabi--Yau manifold of complex dimension $n$. 
Consider the cochain complex
\[
\begin{tikzcd}
\ul{0} & \ul{1} \\
\Vect(X) \ar[r, "\partial_\Omega"] & \cO (X) 
\end{tikzcd}
\]
which we will refer to as $\cL$ in this section.
Here, $\Vect(X)$ is the space of holomorphic vector fields and $\cO(X)$ is the space of holomorphic functions.
The differential is the divergence operator with respect to the holomorphic volume form.
There is the natural structure of a dg Lie algebra on this cochain complex defined by
\begin{align*}
[\mu, \mu'] & = L_\mu \mu' \\
[\mu, \nu] & = L_\mu \nu 
\end{align*}
for $\mu,\mu' \in \Vect(X)$ and $\nu \in \cO(X)$. 
To see that this is a dg Lie algebra we observe that the divergence operator acts as a derivation on the Lie bracket of vector fields. 

In the next three subsections we will describe three equivalent descriptions of the same $\cL$-module.

\subsection{Differential form description}

Next, consider the following cochain complex 
\[
\begin{tikzcd}
\ul{0} & \ul{1} \\
\cO(X) \ar[r, "\partial"] & \Omega^1 (X) 
\end{tikzcd}
\]
that we denote by $\cM$ in this section. 
This cochain complex consists of holomorphic functions and one-forms together with the holomorphic de Rham differential between them. 
We denote by $\beta \in \cO(X)$ a holomorphic function and $\gamma \in \Omega^1(X)$ a holomorphic one-form.

\begin{dfn}
Define the following dg $\cL$-module structure $\rho$ on $\cM$:
\[
\begin{array}{cclcc}
\rho (\mu) \beta & = & L_\mu(\beta) + (\partial_\Omega \mu) \beta & \in &\cO(X) \\
\rho (\mu) \gamma & = & L_\mu (\gamma) + (\partial_\Omega \mu) \gamma & \in &\Omega^1(X) \\ 
\rho(\nu) \beta & = & (\partial \nu) \beta & \in & \Omega^1(X) .
\end{array}
\]
\end{dfn}

\begin{lem}
$\rho$ defines the structure of a dg $\cL$-module.
\end{lem}
\begin{proof}
First, we must show that $\rho$ is compatible with the differentials.
This amounts to checking the following relations:
\begin{itemize}
\item $\partial (\rho(\mu) \beta) = \rho(\partial_\Omega \mu) \beta + \rho(\mu) \partial \beta$.
\item $\rho(\partial_\Omega \mu) \gamma = 0$. 
\item $\rho (\nu) \partial \beta = 0$. 
\end{itemize}
The last two relations follow immediately from the definition of $\rho$.
The left-hand side of the first item is
\begin{equation}\label{eqn:lhs1}
\partial L_\mu( \beta) + \partial ((\partial_\Omega \mu) \beta) =  L_\mu( \partial \beta) + (\partial \partial_\Omega \mu) \beta + (\partial_\Omega \mu) \partial \beta .
\end{equation}
The right-hand side of the first item is
\begin{equation}\label{eqn:rhs1}
(\partial \partial_\Omega \mu) \beta + L_\mu (\partial \beta) + (\partial_\Omega \mu) \partial \beta 
\end{equation}
which clearly agrees with \eqref{eqn:lhs1}. 

Next, we show that $\rho$ is compatible with the Lie bracket. 
This amounts to showing the following relations:
\begin{itemize}
\item $[\rho(\mu), \rho(\mu')] \beta = \rho(L_\mu \mu')\beta$.
\item $[\rho(\mu), \rho(\mu')] \gamma = \rho(L_\mu \mu')\gamma$.
\item $[\rho(\mu),\rho(\nu)] \beta = \rho(L_\mu \nu) \beta$. 
\end{itemize}
\end{proof}

\subsection{Schouten bracket description}

Consider the dg Lie algebra of {\em all} polyvector fields on $X$:
\[
\left(\PV^{-\bu} (X) [-1] \; , \; \partial_\Omega \; , \; \{\cdot, \cdot\} \right)
\]
where the differential \brian{finish}

The Lie algebra $\cL$ resolving divergence free vector fields is clearly a sub Lie algebra of $\PV^{-\bu} (X) [-1]$. 
In particular, $\cL$ acts on $\PV^{-\bu} (X) [k]$ for any shift $k \in \ZZ$. 

Consider the submodule 
\[
\PV^{- \bu\leq n-2} (X) [k] = \bigg(\PV^{n-2}(X) [k+n-2] \xto{\partial_\Omega} \PV^{n-3}(X) [k+n-3] \to \cdots \to \PV^{0} [k] \bigg) 
\]
of $\PV^{-\bu}(X)[k]$. 
Its quotient is the $\cL$-module $\Tilde{\cM}$ which is given by the two-term complex
\[
\Tilde{\cM} = \bigg(\PV^{n}(X) [k+n] \xto{\partial_\Omega}  \PV^{n-1}(X) [k+n-1] \bigg) .
\]
The action of $\cL$ on $\Tilde{\cM}$ is simply the restriction of the Schouten bracket. 

The following lemma is a direct calculation.

\begin{lem}
Contraction with the holomorphic volume form $\Omega$ defines an isomorphism of $\cL$-modules
\[
\Omega \vee (-) \colon \Tilde{\cM} \to \cM .
\] 
\brian{fix shifts}
\end{lem}

\subsection{Contragradient description} 

Define the following cochain complex
\[
\begin{tikzcd}
\ul{0} & \ul{1} \\
\Omega^{n} (X) \ar[r, "\partial^\vee_\Omega"] & \Omega^{n}(X) \otimes_{\cO} \Omega^1 (X) 
\end{tikzcd}
\]
which is isomorphic to the contragradient bundle $\cL^!$ \brian{shifts}. 
Here, the differential is defined by the formula
\[
\partial^\vee_\Omega (\Omega f) = \Omega \otimes (\partial f) .
\]
The cochain complex has the natural structure of an $\cL$-module defined by the formulas
\[
\begin{array}{cclcc}
\Tilde{\rho}(\mu) \Tilde{\beta} & = & L_\mu(\Tilde{\beta}) & \in &\Omega^n \\
\Tilde{\rho} (\mu) \Tilde{\gamma} & = & L_\mu (\Tilde{\gamma}) & \in &\Omega^n \otimes \Omega^1 \\ 
\rho(\nu) \Tilde{\beta} & = & \Tilde{\beta} (\partial \nu)  & \in & \Omega^n \otimes \Omega^1 .
\end{array}
\]

\begin{lem}
The map $\Omega (-) \colon \cM \to \Tilde{\cM}$ is an isomorphism of $\cL$-modules.
\end{lem}
%\[
%\<\partial^\vee_\Omega (\Tilde{\beta}) , \mu\> = \<\Tilde{\beta}, \partial_\Omega \mu\> .
%\]

%$\cL$ is the sub Lie algebra of the dg Lie algebra consisting of all holomorphic polyvector fields $\PV^{\bu}(X)[1]$.  


\section{$L_\infty$ deformations}
We will describe a family of deformations of the dg Lie algebra $\cL \ltimes \Pi \cM$ defined when the dimension of the Calabi--Yau manifold is odd.

\subsection{$n=2$} 
Consider the case $\dim_{\CC}(X) = 5$. 
Introduce the new brackets
\begin{align*}
[\gamma_1, \gamma_2]_2 &= (\partial \gamma_1 \partial \gamma_2)\vee \Omega \in \PV^1(X) \\
[\gamma_1, \gamma_2, \gamma_3]_3 & = (\gamma_1 \partial \gamma_2 \partial \gamma_2)\vee \Omega \in \Omega^0(X) \\
[\gamma_1, \gamma_2, \nu]_3 & = \nu (\partial \gamma_1 \partial \gamma_2)\vee \Omega \in \PV^1(X) .
\end{align*}

The fact that the the differential is a biderivation for the new $2$-ary brackets is obvious. 
The $3$-ary relations we must check are
\begin{itemize}
\item[(1)] $[[\gamma_1, \gamma_2] , \gamma_3] + [[\gamma_2, \gamma_3], \gamma_1] + [[\gamma_3, \gamma_1], \gamma_2] = 0 .$
\item[(2)] $[[\mu, \gamma_1], \gamma_2] + [[\gamma_1, \gamma_2], \mu] + [[\gamma_2, \mu], \gamma_1] = [\gamma_1, \gamma_2, \partial_\Omega \mu]_3$.
\end{itemize}

Let us first check (1). 



\end{document}