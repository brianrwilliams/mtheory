%\section{Twisted matrix model}
\section{Infinite-dimensional symmetry in flat backgrounds}
\label{s:E(5,10)}

\subsection{Global symmetry algebra}
\label{sec:global}

In any field theory, the cohomology classes of states of odd ghost number have the structure of a Lie algebra. 
More generally, after shifting the cohomological degree by one, the full cohomology of states with respect to the linear BRST operator is naturally a graded Lie algebra. 
If we forget the grading to a $\ZZ/2$ grading, then this global symmetry algebra has the structure of a super Lie algebra. 

In general, taking cohomology loses information. 
If the dg Lie (or $L_\infty$) algebra we start with is not formal, then there exist higher-order operations on the linearized BRST cohomology. 
We will refer to this $L_\infty$ algebra as the global symmetry algebra of the theory.

Before taking cohomology with respect to the linear BRST operator, we described the super $L_\infty$ structure on the parity shift of the eleven-dimensional fields in the previous section. 
This is encoded by the full BV action of the eleven-dimensional theory.
The cubic component of the full BV action induces the super Lie algebra structure present in the linearized BRST cohomology. 

Our main result is to relate the global symmetry algebra of the minimal twist of eleven-dimensional supergravity on $\CC^5 \times \RR$ to a certain infinite-dimensional exceptional super Lie algebra studied by Kac \cite{KacBible,KacE510} called $E(5,10)$.
We recall the definition below. 

\begin{thm}\label{thm:global}
Let $\Pi\cE(\CC^5 \times \RR)$ be the parity shift of the fields of eleven-dimensional supergravity on $\CC^5 \times \RR$ and denote by $\delta^{(1)}$ the linearized BRST operator. 
\begin{enumerate}
\item 
As a super Lie algebra, the $\delta^{(1)}$-cohomology of $\Pi\cE(\CC^5 \times \RR)$ is isomorphic to the trivial one-dimensional central extension of the super Lie algebra $E(5,10)$.
\item 
The global symmetry algebra is equivalent, as a super $L_\infty$ algebra, to the non-trivial central extension of $E(5,10)$ determined by the even cocycle defined in~\eqref{eqn:cocycle}. 
\end{enumerate}
\end{thm}

This result implies that the action functional $S_{BF, \infty} + J$ of the eleven-dimensional theory is invariant for the infinite-dimensional Lie algebra $E(5,10)$. 

%defined by the even cocycle
%\begin{align*}
%E(5,10) \times E(5,10) \times E(5,10) & \to \CC \\
%(\mu,\mu',\alpha) & \mapsto \<\mu \wedge \mu', \alpha\>|_{z=0} .
%\end{align*} 

%The parity of the functional in the theorem is odd, but it is also trilinear. 
%Thus, as a cocycle in the Chevalley--Eilenberg complex of $E(5,10)$ it is of total even parity.

%The super Lie algebra $E(5,10)$ is very closely related to the super Lie algebra $E(5,10)$ studied by Kac; there is a dense embedding of super Lie algebras $E(5,10) \hookrightarrow E(5,10)$. 

\subsection{Linearized BRST cohomology} 

We compute the linearized BRST cohomology of eleven-dimensional supergravity.
Then we will describe the induced structure of a super Lie algebra present in the parity shift of the cohomology, proving part (1) of Theorem~\ref{thm:global}.

\parsec[]

First we recall the definition of the exceptional simple super Lie algebra $E(5,10)$. 
Recall that $\Vect_0 (\CC^5)$ is the Lie algebra of divergence-free holomorphic vector fields on $\CC^5$.
Let $\Omega^{2}_{cl} (\CC^5)$ be the module of holomorphic $2$-forms that are closed for the holomorphic de Rham operator $\del$.

The even part of the super Lie algebra $E(5,10)$ is the Lie algebra
\[
E(5,10)_+ = \Vect_0(\CC^5)
\]
of divergence-free vector fields on $\CC^5$,
whose elements we continue to denote by $\mu$. 
The odd piece is the module 
\[
E(5,10)_- = \Omega^{2}_{cl} (\CC^5),
\]
whose elements we denote by $\alpha$. 
Besides the natural module structure, there is odd bracket $ E(5,10)_-\otimes E(5,10)_\to E(5,10)_+$
The bracket uses the isomorphism $\Omega^{-1} \vee (-) \colon \Omega^{4} \cong \Vect (\CC^5)$ induced by the standard Calabi--Yau form $\d^5 z$, and is defined by
\beqn\label{eqn:e510}
[\alpha, \alpha'] = \Omega^{-1} \vee (\alpha \wedge \alpha') .
\eeqn
Since both $\alpha, \alpha'$ are closed two-forms,  the resulting vector field on the right hand side is divergence free. 
In coordinates, if $f^{ij} \d z_i \wedge \d z_j$ and $g^{kl} \d z_k \wedge \d z_l$ are two closed two-forms, their bracket is the vector field $\ep_{ijklm} f^{ij}g^{kl} \partial_{z_m}$. 

To be precise, Kac studied a more algebraic version of the algebra we have just introduced, where holomorphic functions are replaced by holomorphic polynomials.
As such, the simple super Lie algebra that appears in the classification in~\cite{KacBible} is a dense sub Lie algebra of what we call $E(5,10)$, consisting of those vector fields and two-forms that have polynomial coefficients.

\parsec[]

If $\cE$ is the space of fields of any theory in the BV or BRST formalism, the shift $\cL = \cE[-1]$ has the structure of a Lie, possibly $L_\infty$ algebra. 
In the $\ZZ/2$ graded world, the parity shifted object $\cL = \Pi \cE$ has the structure of a super $L_\infty$ algebra. 

In this section, we use the description of the eleven-dimensional theory as the deformation of the BF action $S_{BF,\infty}$ by the functional $J$ of Theorem \ref{thm:dfn}. 
We set the coupling $g = 1$. For any other nonzero value of $g$, we will obtain an isomorphic super $L_\infty$ algebra as explained above.
We would also obtain equivalent  results if we used the other model of the eleven-dimensional theory explained in~\S\ref{s:altdfn}. 
%We remark on this below. %in~\S\ref{s:altglobal}. 

The full differential on the cochain complex of observables of the theory is given by the BV bracket with the BV action. 
For us, this~is 
\[
\delta = \{S_{BF,\infty} + J, -\} .
\]
The linear BRST operator (dual to the differential on the cochain complex of fields) comes only from the quadratic summands in $S_{BF,\infty}$, and is of the form
\beqn\label{eqn:linearBRST}
\delta^{(1)} = \dbar + \d_{\RR} + \div |_{\mu \to \nu} + \del |_{\beta \to \gamma} .
\eeqn

To compute the cohomology with respect to $\delta^{(1)}$ we can use a spectral sequence, first taking the cohomology with respect to $\dbar + \d_{\RR}$ and then with respect to $\div$. 
By the $\dbar$ and de Rham Poincar\'e lemmas, the cohomology of the space of fields of the eleven-dimensional theory on $\CC^5 \times \RR$ with respect $\dbar + \d_{\RR}$ results in the cochain complex
\begin{equation}
  \label{eq:lin1} 
  \begin{tikzcd}[row sep = 1 ex]
    - & + \\ \hline
    \Vect(\CC^5) \ar[r, "\div"] & \cO(\CC^5) \\ 
     \cO(\CC^5) \ar[r, "\del"] & \Omega^{1}(\CC^5).
\end{tikzcd}
\end{equation}
Recall that $\Vect(\CC^5), \cO(\CC^5)$, and $\Omega^1(\CC^5)$ denote the space of holomorphic vector fields, functions, and one-forms, respectively.

The cohomology with respect to the remaining linearized BRST operator consists of the space of triples $(\mu, [\gamma], b)$ where:
\begin{itemize}
\item $\mu$ is a divergence-free holomorphic vector field on $\CC^5$, which is constant along $\RR$
\[
\mu = \mu \otimes 1 \in \Pi \Vect_0(\CC^5) \otimes \Omega^0(\RR) .
\]
Note that $\mu$ is a ghost in the $\ZZ/2$ graded theory. 
\item $[\gamma]$ is an equivalence class of a holomorphic one-form modulo exact holomorphic one-forms along $\CC^5$, which are also constant along $\RR$
\[
[\gamma] = [\gamma] \otimes 1 \in \left(\Omega^{1}(\CC^5) / \d \cO(\CC^5) \right) \otimes \Omega^0(\RR) .
\]
\item A constant function $b \in \Pi \CC$ on $\CC^5 \times \RR$.
This is a $\beta$-type field in the eleven-dimensional theory, any constant function is closed for the de Rham differential. 
This element is also a ghost in the $\ZZ/2$-graded theory. 
\end{itemize}

\parsec[]

After parity shifting, we've identified the solutions to the linear equations of motion with triples
\[
(\mu, [\gamma], b) \in \Vect_0(\CC^5) \oplus \Pi \Omega^{1}(\CC^5) / \del \cO(\CC^5) \oplus \CC .
\]
The bracket induced by the cubic component of $S_{BF, \infty}$ in the classical BV action is the usual bracket on divergence-free vector fields together with the module structure on holomorphic one-forms by Lie derivative.
Notice that the Lie derivative commutes with the $\del$ operator, so this action descends to equivalence classes as above. 
The elements $b$ are central. 

The final term in the BV action $J = \frac16\int \gamma \wedge \del \gamma \wedge \del \gamma$ induces the following Lie bracket on the solutions to the linearized equations of motion
\beqn\label{eqn:eqb}
\big[[\gamma], [\gamma'] \big] = \Omega^{-1} \vee (\del \gamma \wedge \del \gamma') \in \Vect_0(\CC^5) .
\eeqn
where $\Omega^{-1}$ denotes the section of $\PV^{5,hol}(\CC^5)$ which is inverse to the Calabi--Yau form $\Omega$ on $\CC^5$. 
Notice that this bracket is well-defined as it does not depend on the particular equivalence classes and that the resulting vector field is automatically divergence-free.

\parsec[]

Having described the linearized BRST cohomology as a super vector space, we turn to the proof of Theorem \ref{thm:global}.

\begin{proof}[Proof of Theorem \ref{thm:global}]
For the first part, we write down an explicit map between the cohomology computed above and the algebra $E(5,10)$. 

The relationship of the $\mu$-elements in $E(5,10)$ and the eleven-dimensional theory is apparent.

Next, we need to relate the equivalence classes $[\gamma]$ with the closed two-forms $\alpha$ in $E(5,10)$. 
On flat space, any closed differential form is exact (this is a holomorphic version of the Poincar\'e lemma). 
In other words, there is an isomorphism
\[
\del \colon \Omega^1 (\CC^5) / \d \cO(\CC^5) \xto{\cong} \Omega^{2}_{cl}(\CC^5)
\]
induced by the holomorphic de Rham differential.
This gives the relationship between the equivalence class $[\gamma]$ in the eleven-dimensional theory and a closed two-form in $E(5,10)$ by $\alpha = \del \gamma$. 
It is clear from Equations \eqref{eqn:e510} and \eqref{eqn:eqb} that this assignment intertwines the Lie brackets in $E(5,10)$ and the twist of eleven-dimensional supergravity. 
This completes the proof of part (1).

For part (2), we first produce the following homotopy data:
\begin{equation}
\begin{tikzcd}
\arrow[loop left]{l}{K}(\Pi \cE , \delta^{(1)})\arrow[r, shift left, "q"] &(E(5,10) \oplus \CC_b \, , \, 0)\arrow[l, shift left, "i"] \: ,
\end{tikzcd}
\end{equation}

\begin{itemize}
\item On the $\nu$'s we take $K$ to be any operator $K \colon \cO \to \Vect$ such that $\div K \nu = \nu$. 
On the $\gamma$'s we take $K$ to be any operator $K \colon \Omega^1 \to \Omega^0$ which satisfies the homotopy relation
\beqn\label{eqn:htpy1}
\til{K} \del \gamma + \del K \gamma = \gamma
\eeqn
for some auxiliary operator $\til{K} \colon \Omega^2_{cl} \to \Omega^1$.

The precise form of each of these operators will not be needed.
The existence of such operators is guaranteed by the holomorphic Poincar\'e lemma.
The operator $K$ annihilates fields $\beta$ and $\mu$. 
\item 
The map $q$ is described as follows. 
First $q(\mu) = \mu - K \div (\mu)$.
Notice that $q(\mu)$ is automatically divergence-free.
Next, $q(\gamma) = \del \gamma$.
If $\beta$ is a holomorphic function, then $q(\beta) = \beta (z=0)$.
\item 
The map $i$ embeds $\mu$ and $b$ in the obvious way. On a closed two form $\alpha$, we have that $i(\alpha) = \til{K}\gamma$.
\end{itemize}

It is straightforward to check that this comprises well-defined homotopy data, the only nontrivial thing to check is the relation $\id - i \circ q = \delta^{(1)} K - K \delta^{(1)}$. 
Plugging in the field $\gamma$ we see that we must check that
\[
\gamma - \til{K} \del \gamma = \del K \gamma 
\]
which is precisely \eqref{eqn:htpy1}. 

Given this homotopy data, we can compute the homotopy transferred $L_\infty$ structure on the linearized BRST cohomology. 
Since $\nu$ does not survive to cohomology and the fact that there are no nontrivial Lie brackets involving the field $\beta$, this transferred structure is easy to compute. 

There is a single diagram which contributes to the transferred structure, it is given by
\begin{equation}
\begin{tikzpicture}
\begin{feynman}
%\vertex at (-2,0) {$\mu'_3 \ = $};
\vertex(a) at (-1,1) {$i(\mu)$};
\vertex(b) at (-1,0) {$i([\gamma])$};
\vertex(c) at (-1,-1) {$i(\mu')$};
\vertex(d) at (0,0.5);
\vertex(e) at (1,0);
\vertex(f) at (2,0) {$q$};
\diagram* {(a)--(d), (b)--(d), (d)--[edge label = $K$](e), (c)--(e), (f)--(e)};
\end{feynman}
\end{tikzpicture}
\end{equation}
together with a similar diagram with the $\mu$ and $\mu'$ flipped. 

This diagram leads to a new $3$-ary bracket on $E(5,10) \oplus \CC_b$
\[
\big[\mu,\mu',[\gamma]\big]_3 = \varphi(\mu,\mu',[\gamma])
\]
where $\varphi \in \clie^\text{even} (E(5,10))$ is the even Lie algebra cocycle defined by the formula
\beqn
\begin{array}{rclr}
\varphi \colon E(5,10) \times E(5,10) \times E(5,10) & \to & \CC_b \\
\varphi(\mu,\mu',\alpha) & = & \<\mu \wedge \mu', \alpha\>|_{z=0} .
\label{eqn:cocycle}
\end{array}
\eeqn
Since $b$ is central, this cocycle defines a central extension of $E(5,10)$.
\end{proof}

\parsec[]
We briefly remark on Lie algebra cohomology for super Lie algebras.
The Lie algebra cohomology $\clie^{\bu,\bu}(\cL)$ of any super Lie algebra $\cL$ is graded by $\ZZ \times \ZZ/2$. 
The first grading is by the symmetric degree in the Chevalley--Eilenberg complex.
The second grading is the internal parity of the super Lie algebra $\cL$. 
The Chevalley--Eilenberg differential is degree $(1,+)$. 

The cocycle $\varphi$ has homogenous bigrading $(3,-)$.
In the above discussion we forgot the bigrading to a totalized $\ZZ/2$ grading where 
\begin{align*}
\clie^\text{even} (\cL) & = \clie^{2\bu , +} (\cL) \oplus \clie^{2\bu+1, -}(\cL) \\
\clie^\text{odd} (\cL) & = \clie^{2\bu , -} (\cL) \oplus \clie^{2\bu+1, +}(\cL) .
\end{align*}
With this totalization, $\varphi$ is an even cocycle and hence determines a super $L_\infty$ central extension by the one-dimensional even vector space $\CC$. 

%\parsec[s:altglobal]

%In \S \ref{s:altdfn} we gave an equivalent description of the eleven-dimensional theory as a deformation of the BF action $S_{BF,0}$ by the functional $\til J$. 

%\subsection{Twisted matrix model}
