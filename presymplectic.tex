\documentclass[11pt]{amsart}

\pdfoutput = 1

\usepackage{macros,slashed,amsaddr}

\linespread{1.5}

\setcounter{tocdepth}{2}
\numberwithin{equation}{section}
\newcommand{\nocontentsline}[3]{}
\newcommand{\tocless}[2]{\bgroup\let\addcontentsline=\nocontentsline#1{#2}\egroup}
\newcommand{\changelocaltocdepth}[1]{%
  \addtocontents{toc}{\protect\setcounter{tocdepth}{#1}}%
  \setcounter{tocdepth}{#1}%
}
\setcounter{tocdepth}{1}

\def\brian{\textcolor{ForestGreen}{BW: }\textcolor{ForestGreen}}

\newcommand{\defterm}[1]{\textbf{\emph{#1}}}

\def\PV{{\rm PV}}

\begin{document}

\title{An interacting presymplectic BV theory}

\maketitle

\tableofcontents

Here was my understanding of the theory you told me. 
It's a $\ZZ/2$-graded theory with two sets of fields
\begin{align*}
\mu & \in \Omega^\bu (\RR) \otimes \bigg(\PV^{1,\bu}(\CC^5) \xto{\partial} \PV^{0,\bu}(\CC^5) \bigg) \\
\gamma & \in \Omega^{\bu}(\RR) \otimes \Omega^{1,\bu}(\CC^5)
\end{align*}

There is a pairing of the form
\[
\omega(\gamma, \mu) = \int_{\RR \times \CC^5} \gamma \wedge (\mu \vdash \Omega) .
\]
Here $(-) \vdash \Omega$ is the isomorphism $\PV^{k, \bu} \cong \Omega^{5-k, \bu}$.
Note that the $\PV^{0,\bu}(\CC^5)$ component of $\mu$ does not appear in the pairing. 

If we get the shifts right 
\[
\Omega^\bu (\RR) \otimes \bigg(\PV^{1,\bu}(\CC^5) \xto{\partial} \PV^{0,\bu}(\CC^5) \bigg)
\] 
forms a dg Lie algebra defined by the Schouten bracket. 
Then the $\gamma$-fields form a module for this.

There is a sort of ``modified BF theory" whose action is of the form
\[
\omega\left(\gamma, \d \mu + \frac{1}{2} \{\mu, \mu\} \right) = \int \gamma \bigg(\d \mu + \frac12 \{\mu, \mu\} \bigg) \vdash \Omega .
\]

We consider a deformation of this action by the functional
\[
\int \gamma \partial \gamma \partial \gamma .
\]
\end{document}
