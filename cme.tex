\documentclass[11pt]{amsart}

\usepackage{macros-mtheory}

%% Bib stuff

\usepackage[utf8]{inputenc}
\usepackage[T1]{fontenc}
\usepackage[backend=biber,
            isbn=false,
            doi=false,
            maxbibnames=5,
            %giveninits=true,
            style=alphabetic,
            citestyle=alphabetic]{biblatex}
\usepackage{url}
\setcounter{biburllcpenalty}{7000}
\setcounter{biburlucpenalty}{8000}
\renewbibmacro{in:}{}
\bibliography{references.bib}
\renewcommand*{\bibfont}{\footnotesize}

\usepackage{xcolor}
\definecolor{e-mail}{rgb}{0,.40,.80}
\definecolor{reference}{rgb}{.40,.60,.2}
\definecolor{citation}{rgb}{0,.25,.5}

\usepackage[colorlinks=true,
            linkcolor=reference,
            citecolor=citation,
            urlcolor=e-mail]{hyperref}
\usepackage{cleveref}


%% Formatting. We can change this in the final version, just makes it easier for me to read in the meantime.

\usepackage[top=1in, bottom=1.25in, left=1.3in, right=1.3in]{geometry}
\setlength{\parskip}{8pt}
\setlength\parindent{0pt}
\setcounter{tocdepth}{2}
\numberwithin{equation}{section}
\let\emptyset\varnothing


%% Comments

\def\brian{\textcolor{violet}{BW: }\textcolor{violet}}
\def\surya{\textcolor{blue}{SR: }\textcolor{blue}}
\def\ingmar{\textcolor{magenta}{IS: }\textcolor{magenta}}

%% Random macros

\def\vectdiv{\fL}
\def\PV{{\rm PV}}
\def\div{\partial_\Omega}
\def\im{{\rm i}}
\def\til{\Tilde}

%\usepackage[style = alphabetic]{biblatex}
%\addbibresource{references.bib}
%\AtBeginBibliography{\scriptsize}


\begin{document}

\title{The classical master equation}


%\author{Brian R. Williams}
%\address{School of Mathematics\\
%University of Edinburgh \\ 
%Edinburgh \\ 
%UK}
%\email{brian.williams@ed.ac.uk}


\maketitle
\thispagestyle{empty}
\setcounter{tocdepth}{1}
%\tableofcontents

\begin{lem}\label{lem:pi}
Let $\gamma_1,\gamma_2$ be any two elements of $\PV^{4,\bu}(X)$ and define
\[
\Pi(\gamma_1,\gamma_2) = \gamma_1 \vee (\div \gamma_2 \vee \Omega) \in \PV^{2,\bu}(X) .
\]
Then
\[
\div \Pi(\gamma_1,\gamma_2) = \div \gamma_1 \vee (\div \gamma_2 \vee \Omega) .
\]
\end{lem} 

\begin{dfn} \label{dfn:classical}
Let $X$ be a Calabi--Yau five-fold and $L$ a smooth one-dimensional manifold.
{\em Eleven-dimensional twisted supergravity} on $X \times L$ is the $\ZZ/2$-graded BV theory on $X$ whose fields are
\begin{align*}
(\mu, \nu) & \in \big[\Pi \PV^{1,\bu}(X) \; \Hat{\otimes} \;
\Omega^{\bu}(L)\big] \oplus \big[\PV^{0,\bu}(X) \; \Hat{\otimes} \;
\Omega^{\bu}(L)\big] \\
(\beta, \gamma) & \in \big[\Pi \PV^{5,\bu}(X) \; \Hat{\otimes} \;
\Omega^{\bu}(L)\big] \oplus \big[\PV^{4,\bu}(X) \; \Hat{\otimes} \;
\Omega^{\bu}(L)\big]
\end{align*}
The BV pairing is 
\beqn
\int^\Omega \gamma \wedge \mu + \int^\Omega \beta \nu
\eeqn
and the BV action is
\begin{multline}\label{eqn:bvaction}
\int^\Omega \gamma \wedge \dbar \mu + \int^\Omega \beta \dbar \nu + \int^\Omega \beta \div \mu + \frac12 \int^\Omega \frac{1}{1-\nu} \mu \wedge \mu \wedge \div \gamma \\ 
+ \frac{c}{6} \int^\Omega \frac{1}{1-\nu} \gamma \wedge \big[\div \gamma \vee (\div \gamma \vee \Omega) \big] .
\end{multline}
\end{dfn}

Notice that the last term in the action can be written as
\[
\frac{c}{6} \int^\Omega \frac{1}{1-\nu} \gamma \wedge \div \Pi(\gamma, \gamma)
\]
with $\Pi(\gamma, \gamma)$ as in Lemma \ref{lem:pi}. 

In order to see that this definition is sound we must check that the BV action satisfies the classical master equation. 
Before doing that, we remark on a simplification of this action that is bit easier to understand. 

The action takes a relatively simpler form if instead of resolving the constraint that $\mu$ be divergence-free we impose it. 
Imposing the constraint $\div \mu = 0$ has the effect of throwing out the fields $\nu, \beta$ and imposes the gauge condition that $\gamma$ is only defined up to a total divergence. \footnote{Really, the zero-mode part of $\beta$ also survives, but we throw this out by hand since it does not interact with the remaining fields.}
That is, the fields become 
\beqn\label{eqn:bvonshell2}
\mu \in \ker \div \subset \PV^{1,\bu}(X) \; \Hat{\otimes} \;
\Omega^{\bu}(L) , \quad \gamma \in \left(\PV^{4,\bu} (X) / \div \right) \; \Hat{\otimes} \;
\Omega^{\bu}(L) .
\eeqn
Doing this, we see that the action becomes 
\beqn\label{eqn:bvonshell2}
\int^\Omega \gamma \wedge \dbar \mu + \frac12 \int^\Omega \mu \wedge \mu \wedge \div \gamma \\ 
+ \frac{c}{6} \int^\Omega \gamma \wedge \big[\div \gamma \vee (\div \gamma \vee \Omega) \big] .
\eeqn
By integrating by parts with respect to $\dbar$ it follows that the operator $\{\int^\Omega \gamma \wedge \dbar \mu, -\}$ annihilates the second two terms in the above action. 
The middle term in \eqref{eqn:bvonshell} can be rewritten as $\frac12 \int \gamma \wedge \{\mu, \mu\}$. 
From this observation, the equation 
\beqn\label{eqn:cmeos1}
\left\{\frac12 \int^\Omega \mu \wedge \mu \wedge \div \gamma, \frac12 \int^\Omega \mu \wedge \mu \wedge \div \gamma \right\} = 0
\eeqn
follows from the Jacobi identity for the Schouten bracket. 
Next, we have
\begin{multline}
\left\{\frac12 \int^\Omega \mu \wedge \mu \wedge \div \gamma, \frac16 \int^\Omega \gamma \wedge \big[\div \gamma \vee (\div \gamma \vee \Omega) \big] \right\} = \int^\Omega \mu \wedge \div \gamma \wedge \big[\div \gamma \vee (\div \gamma \vee \Omega) \big] \\ + ??
\end{multline}
%Since $\mu$ is divergence-free, the Lagrangian on the right hand side is a total derivative. 
Integrating by parts, we see that this can be written as
\[ 
\int^\Omega \gamma \wedge \{\mu, \big[\div \gamma \vee (\div \gamma \vee \Omega) \big]\} + \int^\Omega \mu \wedge \gamma \wedge \div \big[\div \gamma \vee (\div \gamma \vee \Omega) \big] + ?? .
\]
The second term vanishes by Lemma \ref{lem:divfree}, thus 
\beqn\label{eqn:cmeos2}
\left\{\frac12 \int \div \gamma \mu^2 , \frac16 \int \gamma [\div \gamma \vee (\div \gamma \vee \Omega)]\right\} = 0 .
\eeqn
This piece of the master equation is equivalent to the relations
\begin{align*}
\{\mu, \div \gamma \vee (\div \gamma' \vee \Omega)\} & = \{\mu, \div \gamma\} \vee (\div \gamma' \vee \Omega) \pm \div \gamma \vee (\{\mu, \div \gamma'\} \vee \Omega) \\
0 & = \{\gamma, \div \gamma' \vee (\div \gamma'')\} + perm .
\end{align*}
for $\mu$ divergence-free. 

In conclusion, we see that the simplified BV action \ref{eqn:bvonshell} does satisfy the classical master equation.

\begin{prop}
The full BV action \eqref{eqn:bvaction} satisfies the classical master equation.
\end{prop}
\begin{proof}
The claim follows from the following series of relations
\beqn\label{eqn:cme1}
\left\{\int^\Omega \beta \div \mu,\frac12 \int^\Omega e^\nu \mu \wedge \mu \wedge \div \gamma\right\} + \left\{\frac12 \int^\Omega e^\nu \mu \wedge \mu \wedge \div \gamma, \frac12 \int^\Omega e^\nu \mu \wedge \mu \wedge \div \gamma \right\} = 0
\eeqn
\end{proof}

\section{Kodaira--Spencer theory with potentials}

Let
\[
\cE_{\rm KS} = \PV^{\bu,\bu} [[u]] [2] .
\]
be the full space of fields of Kodaira--Spencer theory. 

Suppose that $d \ne 3$ and consider the subspace $\cE \subset \cE_{\rm KS}$ of Kodaira--Spencer fields
\[
\begin{tikzcd}
\ul{-1} & \ul{0} & \cdots & \ul{d-4} & \ul{d-3} & \cdots &&  \\
\PV^{1,\bu} \ar[r] & u\PV^{0,\bu} & & & & & \subset & \cE_{\rm KS} \\
& & & \PV^{d-2, \bu} \ar[r] & u \PV^{d-3, \bu} \ar[r] & \cdots & &  .
\end{tikzcd}
\]
%We will write the fields in the above decomposition as $(\eta, \mu, u \nu) \in \cE$. 
This is a subcomplex equipped with differential $\dbar - u \div$. 

The trace pairing leads to isomorphisms
\begin{align*}
\cE^\vee_{\rm KS} & \cong u^{-1} \Bar{\PV} [u^{-1}] [2d-4] \\
\Bar{\cE}^\vee_{\rm KS} & \cong u^{-1} \PV [u^{-1}] [2d-4]
\end{align*} 
where we identify the dual of $\CC[[u]]$ with $u^{-1} \CC[u^{-1}]$ by the residue pairing. 
Similarly, for the subspace $\cE \subset \cE_{\rm KS}$ we see that $\cE^\vee$ is 
\[
\begin{tikzcd}
\ul{??} & \ul{??} & \cdots & \ul{??} & \ul{??}  \\
u^{-d+1}\Bar{\PV}^{d,\bu} \ar[r] & u^{-d+2}\PV^{d-1,\bu} \ar[r] & \cdots & & & \\
& & & u^{-2} \Bar{\PV}^{d, \bu} \ar[r] & u^{-1} \PV^{d-1, \bu} & .
\end{tikzcd}
\]
There is a similar presentation for $\Bar{\cE}^\vee$. 
%\begin{align*}
%\cE^\vee & \cong u^{-1} \Bar{\PV}^{d,\bu}[2d-4] \oplus u^{-1} \Bar{\PV}^{d-1,\bu}[2d-3] \oplus u^{-2} \Bar{\PV}^{d,\bu} [2d-2] \\
%\Bar{\cE}^\vee & \cong u^{-1} \PV^{d,\bu}[2d-4] \oplus u^{-1} \PV^{d-1,\bu}[2d-3] \oplus u^{-2} \PV^{d,\bu}[2d-2]
%\end{align*} 

The bracket on $\cO(\Bar{\cE}_{\rm KS})$ is constructed from the pairing on $\Bar{\cE}^\vee_{\rm KS}$ defined by
\beqn\label{eqn:pairing}
f(u) \alpha \otimes g(u) \beta \mapsto \frac1{2\pi \im} {\rm Res}_{u=0} u f(u) g(u) \, \int^\Omega (\div \alpha) \beta .
\eeqn
This induced a Poisson bracket on $\cO(\Bar{\cE}_{\rm KS})$ of degree $2d -5$. 

\begin{lem} \label{lem:pairing}
This pairing descends to a well-defined pairing on $\Bar{\cE}^\vee$ and hence a Poisson bracket on $\cO(\Bar{\cE})$ of degree $2d-5$. 
The map 
\[
\cO(\Bar{\cE}_{\rm KS}) \to \cO(\Bar{\cE})
\]
intertwines the resulting Poisson brackets. 
\end{lem}
\begin{proof}
\end{proof}

The BCOV action $I_{\rm BCOV} \in \oloc(\cE_{\rm KS})$ 
satisfies the classical master equation.
%\[
%Q I_{\rm BCOV} + \frac12 \{I_{\rm BCOV}, I_{\rm BCOV}\} = 0.
%\]
Let $I \in \oloc(\cE)$ denote the restriction of $I_{\rm BCOV}$ to the subspace $\cE$.
By Lemma \ref{lem:pairing} the local functional $I$ satisfies the classical master equation
\beqn\label{eqn:IE}
Q I + \frac12 \{I,I\} = 0 , \qquad {\rm in } \quad \oloc(\cE).
\eeqn 


\subsection{}

Let $\til{\cE}$ denote the cochain complex of vector bundles
\beqn\label{eqn:tilE}
\begin{tikzcd}
\ul{-1} & \ul{0} & \cdots & \ul{d-5} & \ul{d-4} &   \\
\PV^{1,\bu} \ar[r] & u\PV^{0,\bu} & \cdots & & \\
& & & u^{-1} \PV^{d, \bu} \ar[r] & \PV^{d-1, \bu} 
\end{tikzcd}
\eeqn
There is a non-degenerate pairing on this complex of bundles defined by 
\[
(f(u) \alpha, g(u) \beta) \mapsto {\rm Res}_{u=0} \frac{f(u) g(u)}{u} \int^\Omega \alpha \wedge \beta .
\]
This pairing is graded anti-symmetric of degree $-2d+5$ and so induces a Poisson bracket on $\oloc(\til\cE)$ of degree $2d-5$.

Define the map
\[
\Phi\colon \til\cE \to \cE 
\]
as follows:
\begin{itemize}
\item on the first line of \eqref{eqn:tilE} the map $\Phi$ is the identity. 
\item on the second line of \eqref{eqn:tilE} the map is
\[
\Phi(f(u) \otimes \alpha) = {\rm Res}_{u=0} \frac{f(u)}{u} \partial \alpha .
\]
\end{itemize}

\begin{prop}
The map $\Phi$ induces a cochain map
\[
\Phi^* \colon \oloc(\cE) \to \oloc(\til \cE) 
\]
which intertwines the graded Poisson brackets. 
\end{prop}

As a corollary of this result we obtain from the BCOV action a local functional 
\[
\til I = \Phi^* I \in \oloc(\til\cE)
\] 
which solves the classical master equation
\[
Q \til I + \frac12 \{\til I, \til I\} = 0, \qquad {\rm in } \quad \oloc(\til \cE) .
\]

\end{document}
