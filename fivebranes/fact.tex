%\documentclass[11pt]{amsart}
%
%%\usepackage{../macros-master}
%\usepackage{macros-fivebrane}
%
%\begin{document}

\section{Factorization algebras in twisted $M$ theory}
\label{s:fact}

In this section we use the formalism of factorization algebras to give a conjectural description of the space of observables of the universal fivebrane theory after performing the holomorphic twist.
For the case at hand, our expectation for the universal theory can be understood as the worldvolume theory on a large number of fivebranes. 
We will see how this relates to the description of the states in twisted supergravity as defined in the last section.
Later, we see how our interpretation is supported by a relationship of the supergravity index to the fivebrane superconformal index.

Our ansatz for the worldvolume theories in the large $N$ limit relies on a proposal for twisted holography proposed by Costello and Li in \cite{CLsugra} and further developed by Costello, Gaiotto, Paquette in \cite{CostelloM2,CostelloM5,costello2021twisted,CP}.
The core idea is that the algebra of operators on both sides of the duality are Koszul dual. 
Many other examples and support for this twisted holographic principle have been carried out in \cite{Oh:2021wes,Oh:2020hph,Gaiotto:2021xce}.

It is absolutely crucial for this proposal that we work in a derived setting using the Batalin--Vilkovisky (BV) formalism. 
The observables of any theory in the BV formalism have the structure of a factorization algebra on spacetime \cite{CG2}. 
Classically, if a theory in the BV formalism is described by a local $L_\infty$ algebra $\cL$ on spacetime $M$ then to an open set $U \subset M$ the factorization algebra of observables assigns the cochain complex
\[
\clie^\bu(\cL(U))
\]
which computes the Lie algebra cohomology of the $L_\infty$ algebra $\cL(U)$. 
The differential on this cochain complex is precisely the Chevalley--Eilenberg differential associated to the~$L_\infty$ structure maps. 

At the quantum level the factorization algebra of observables is deformed.
Perturbatively, Costello and Gwilliam use the BV formalism to give a systematic way to study quantization within the context of factorization algebras~\cite{CG2}. 
Even in perturbation theory, the full quantum behavior of our proposal for the minimal twist of eleven-dimensional theory is an open question. 
We leave the problem to characterize the quantum algebra structure to future work, and in this paper we do not address anything past tree-level in the bulk eleven-dimensional theory.

We start with the factorization algebra of classical observables of our prototype for the minimal twist of eleven-dimensional supergravity which we recalled in \S \ref{s:twisted}. 
This model is defined on any eleven-manifold $S \times X$ where $X$ is a Calabi--Yau fivefold and $S$ a real oriented one-manifold. 
%In the \S \ref{s:twisted} we recalled a proposal for the description of the minimal twist of eleven-dimensional supergravity on $S \times X$. 
The primary object we used to describe the solutions of the equations of motion was the $\Z/2$ graded local $L_\infty$ algebra $\cL_{sugra}$, see \S \ref{s:Lsugra}. 
%As a graded vector space, observables on an open set $U \subset X \times S$ are thus $\cO(\cL_{sugra}(U)) = \Sym (\Pi \cL_{sugra})^\vee$.
%The differential is encoded by the $L_\infty$ structure which is exactly the Chevalley--Eilenberg differential. 
In terms of this local $L_\infty$ algebra, the $\Z/2$ graded commutative dg algebra of classical observables supported on an open set $U \subset S \times X$ is 
\beqn\label{eqn:sugraobs}
\clie^\bu\left(\cL_{sugra}(U)\right) .
\eeqn
We will denote the classical factorization algebra on $S \times X$ of classical observables by~$\Obs_{sugra}$.


%\parsec[s:membraneobs]
%
%
%\parsec[s:fivebraneobs]
%
%Let $Z$ be a complex threefold. 
%In \ref{s:single} we have described the holomorphic twist of a theory on a single fivebrane within the (degenerate) BV formalism in terms of the (abelian) local $L_\infty$ algebra $\cL_{single}$ on $Z$. 
%The factorization algebra of classical observables of the holomorphic twist of the single fivebrane theory assigns to an open set $U \subset Z$
%the cochain complex
%\[
%\clie^\bu\left(\cL_{single}(U)\right) .
%\]
%We will denote the entire factorization algebra on $Z$ of classical observables by~$\clie^\bu(\cL_{single})$.


\subsection{The fivebrane decomposition of twisted supergravity}\label{s:resm5}

Fix a complex three-fold $Z$, where the fivebranes will be supported, and assume that it is equipped with a sqaure-root of its canonical bundle $K^{1/2}_Z$.
We consider the eleven-dimensional theory that we introduced in \S \ref{s:twisted} on $\R \times X_Z$ where $X_Z$ is the Calabi--Yau fivefold
\beqn
X_Z = \text{Tot}(K_Z^{1/2} \otimes \C^2).
\eeqn
This is the total space of the rank two holomorphic vector bundle $K_Z^{1/2} \otimes \C^2$ over~$Z$. 
%In the remainder of this section we denote by $V \to Z$ this rank two holomorphic vector bundle. 
One can also think about $\R \times X_Z$ as the total space of the {\em real} rank five bundle over~$Z$; although this description obfuscates the geometric structure used to define the eleven-dimensional theory.
Let 
\beqn
\pi \colon \R \times X_Z \to Z 
\eeqn
be the map which projects out the real factor followed by the projection to the base of the total space of the complex rank two bundle $p \colon X_Z \to Z$.

The classical observables of the eleven-dimensional theory form a factorization algebra on $\R \times X_Z$ that we have denoted by $\Obs_{sugra}$.
Like sheaves, factorization algebras can be pushed forward.
Thus, using $\pi$ we obtain a factorization algebra $\pi_* \Obs_{sugra}$ defined on the three-fold~$Z$.
Explicitly, to an open set $U \subset Z$ the factorization assigns the cochain complex
\begin{align*}
(\pi_* \Obs_{sugra}) (U) & = \Obs_{sugra} \left(\pi^{-1} U\right) \\
& = \Obs_{sugra}(\R \times p^{-1}U) .
\end{align*}

This factorization algebra is not the factorization algebra associated to an ordinary sort of field theory on $Z$.
In physics terminology this feature is due to the presence of an infinite tower of so-called Kaluza--Klein modes which means that the space of fields is not given as the sections of a finite rank vector bundle on~$Z$.
Mathematically, this is due to the fact that the map $\pi$ is not proper.
Nevertheless there is a subfactorization algebra $\Bar{\pi}_* \Obs_{sugra}$ which admits a natural grading so that each filtered component can be understood as such. 
This grading is determined by looking at an eigenspace decomposition of a certain compact abelian group acting on the fields of the eleven-dimensional theory on $\R \times X_Z$ which we now describe. 

Like $\Obs_{sugra}$ in equation \eqref{eqn:sugraobs}, the factorization algebra $\Bar{\pi}_* \Obs_{sugra}$ is of the form $\clie^\bu(\cG)$.
Here, $\cG$ is a sheaf of $L_\infty$ algebras on $Z$ which is presented as the $C^\infty$-sections of an $\infty$-dimensional pro vector bundle.
%To get a sense of the construction of this factorization algebra we take a brief excursion to a simpler situation related to branes in five-dimensional Chern--Simons theory. 

%\parsec[]
%
%In~\S \ref{s:twistedsugra} we recalled some features of a five-dimensional Chern--Simons type theory that Costello has shown to be the result of applying the $\Omega$-background to $M$-theory on a Taub--NUT space \cite{CostelloM5}.
%In this setting fivebranes which hit the tip of the Taub--NUT are deformed into chiral defects supported along a Riemann surface~$C$ in the five-manifold. 
%In analogy with our eleven-dimensional setup, we take the five-manifold to be $\R \times {\rm Tot}(K_C)$ with the deformed fivebranes wrapping the zero section.
%Denote by $\Obs_{CS}$ the classical observables\footnote{We treat the theory classically as a perturbative theory on $\R \times {\rm Tot}(K_C)$.
%We note that the theory is defined in terms of two parameters $\ep, \delta$ where $\delta$ is the Chern--Simons coupling and $\ep$ is the non-commutative parameter controlling the deformation quantization of the holomorphic symplectic manifold ${\rm Tot}(K_C)$.
%For convenience we set both $\ep, \delta$ to be one.}
%of the non-commmutative five-dimensional Chern--Simons theory on $\R \times {\rm Tot}(K_C)$.
%Let 
%\beqn
%\pi \colon \R \times {\rm Tot}(K_C) \to C
%\eeqn
%be the natural projection and consider the factorization algebra $\pi_* \Obs_{CS}$ on $C$. 
%
%Notice that there is a $U(1)$ action on the fields of the five-dimensional theory induced by the $U(1)$ action on the five-manifold which scales the fiber of~$K_C$.
%Using this we can look at the subfactorization algebra 
%\beqn\label{eqn:taylor1a}
%\Bar{\pi}_* \Obs_{CS} \subset \pi_* \Obs_{CS}
%\eeqn
%which to an open set $U \subset C$ assigns the complex of observables in $\Obs_{CS} (\pi^{-1} U)$ which are finite sums of integral eigenvectors for this $U(1)$ action.
%
%For the rank one abelian gauge group, the space of fields of the five-dimensional theory on $\R \times {\rm Tot}(K_C)$ is
%\[
%\Omega^\bu(\R) \otimes \Omega^{0,\bu}({\rm Tot}(K_C)) [1] .
%\]
%The interaction is encoded by the symplectic structure on ${\rm Tot}(K_C)$.
%%For simplicity, let us assume that the canonical bundle on $C$ is trivializable. 
%For $U \subset C$, the complex of observables is given in terms of the Lie algebra cohomology
%\[
%(\pi_* \Obs_{CS})(U) \simeq \clie^\bu \left(\Omega^{0,\bu}(p^{-1}U)\right)
%\]
%where $p \colon K_C \to C$.
%The Chevalley--Eilenberg complex is taken with respect to the dg Lie algebra structure coming from the holomorphic Poisson structure on the total space of~$K_C$. 
%If $U$ is a trivializing cover then
%\[
%(\pi_* \Obs_{CS})(U) \simeq \clie^\bu \left(\Omega^{0,\bu}(C) \otimes \cO^{hol}(\C)\right) .
%\]
%On the other hand, we have
%\[
%(\Bar{\pi}_* \Obs_{CS})(U) \simeq \clie^\bu\left(\Omega^{0,\bu}(C) [[w]] \right) .
%\]
%Here, $w$ plays the role of a formal coordinate on the fiber of $K_C$ and $w^{k}$ as parametrizes the weight $k$ eigenspace---locally, $w$ is simply the one-form $\d z$.
%Using this notation, the embedding \eqref{eqn:taylor1a} is induced by the Taylor series expansion map $\cO(\C) \to \C[[w]]$.
%The dg Lie algebra structure on $\Omega^{0,\bu}(C) [[w]]$ is given by the Moyal commutator. 
%This is similar to the non-formal case---it is the natural bracket if we understand $\cO(C)[[w]]$ as functions on the formal neighborhood of the zero section in~$K_C$.

%The differential is simply $\dbar$ in the direction of the curve $C$. 
%If we write an arbitrary element of this complex as $w^k \alpha$, where $\alpha \in \Omega^{0,\bu}(C)$, then the Lie bracket is
%\[
%[w^k \alpha, w^l \beta] 

\parsec[]

We are in the situation of a complex three-fold $Z$ embedded as a submanifold of the eleven-manifold~$\R \times \text{Tot}(K_{Z}^{1/2} \otimes \C^2)$. 
%The construction of the restricted factorization algebra $\Bar{\pi}_* \Obs_{sugra}$ is completely similar to the simple case of five-dimensional Chern--Simons theory as we just discussed.
Consider the group $U(1) \times U(1)$ rotating the fibers of the rank two bundle $K_Z^{1/2} \otimes \C^2$. 
Using this we can define the subfactorization algebra 
\beqn\label{eqn:taylor1b}
\Bar{\pi}_* \Obs_{sugra} \subset \pi_* \Obs_{sugra}
\eeqn
which to an open set~$U \subset Z$ assigns the subcomplex of observables in~$\Obs_{sugra} (\pi^{-1} U)$ which are finite sums of integral eigenvectors for this $U(1) \times U(1)$ action.

By construction, the restricted factorization algebra is of the form
\[
\Bar{\pi}_*\Obs_{sugra} = \clie^\bu(\cG_Z) 
\]
where $\cG_Z$ is an infinite-rank local $L_\infty$ algebra on $Z$. 
The embedding of factorization algebras \eqref{eqn:taylor1b} is induced by a partial Taylor expansion map
\[
\pi_* \cL_{sugra} \to \cG_Z .
\]

\parsec[s:cstarfive]

We now consider a particular decomposition of the factorization algebra $\Bar{\pi}_* \Obs_{sugra}$. 
In a local chart $U \subset Z$ we can write a generic field of the eleven-dimensional theory on $\pi^{-1} U$ in coordinates as $f(t;w,z)$ where $t \in \R$ is the real coordinate, $w=(w_1,w_2)$ is the holomorphic fiber coordinate of the rank two bundle $X_Z$ over $Z$, and $z = (z_1,z_2,z_3)$ is a local holomorphic coordinate for $U \subset Z$. 

%For simplicity let us consider the eleven-dimensional theory on flat space $\R \times \C^5$ with some number of fivebranes supported on
%\[
%0 \times 0 \times \C^3 \subset \C^5 .
%\]
%is that (after taking into account the backreaction) the observables on a large number of fivebranes is Koszul dual to the factorization algebra 
%
%\parsec[s:flatdecomp]
%
%%In the simple case where $Z = \C^3$ and we identify the total space of $K^{1/2}_Z \otimes \C^2$ with $\C^5$ then the manifold obtained by removing the locus of the brane is homeomorphic to
%%\[
%%\C^3 \times (\R \times \C^2 - 0) .
%%\]
%%
%%Let $\pi \colon \R \times \C^5 - (0 \times \C^3) \to \C^3 \times \R_{>0}$ be the projection map whose fibers are homeomorphic to the sphere $S^4$ which links the location of the fivebranes.
%%We restrict the factorization algebra of the eleven dimensional theory $\Obs_{sugra}$ to the open set obtained by removing the locus of the brane.
%
%In the simple case that $Z = \C^3$ and we identify the total space of $K^{1/2}_Z \otimes \C^2$ with $\C^5$ there is a more direct construction of the factorization algebra $\Obs_{sugra}|_{Z}$. 
%
%Let $\pi \colon \R \times \C^5 \to \C^3$ be the projection map.
%Then, via $\pi$ we can pushforward the factorization algebra associated to the eleven-dimensional theory to obtain a factorization algebra
%\[
%\pi_* \Obs_{sugra} 
%\]
%on $\C^3$.
%This factorization algebra is not the factorization algebra associated to an ordinary sort of field theory on $\C^3$. 
%Nevertheless there is a subfactorization algebra which admits a natural grading so that each filtered component can be understood as such.

With this notation in place, we introduce the following $\C^\times$ action on the fields of the eleven-dimensional theory:
\begin{itemize}
\item On the fields $\mu(t;w,z) \in \Omega^\bu(\R) \otimes \PV^{1,\bu}(\C^2 \times U)$ the action is
\[
\lambda \cdot \mu(t;w,z) = \mu(\lambda t;\lambda w , z).
\]
\item On the fields $\nu(t;w,z) \in \Omega^\bu(\R) \otimes \PV^{0,\bu}(\C^2 \times U)$ the action is
\[
\lambda \cdot \nu(t;w,z) = \nu(\lambda t;\lambda w , z).
\]
\item On the fields $\beta(t;w,z) \in \Omega^\bu(\R) \otimes \Omega^{0,\bu}(\C^2 \times U)$ the action is
\[
\lambda \cdot \beta(t;w,z) = \lambda^{-1} \beta(\lambda t;\lambda w , z).
\]
\item On the fields $\gamma(t;w,z) \in \Omega^\bu(\R) \otimes \Omega^{1,\bu}(\C^2 \times U)$ the action is
\[
\lambda \cdot \gamma(t;w,z) = \lambda^{-1} \gamma(\lambda t;\lambda w , z).
\]
\end{itemize}

The following proposition summarizes that this $\C^\times$-action is compatible with the $L_\infty$ structure on $\cL_{sugra}$.
Its proof is an immediate computation. 

\begin{prop}
The $L_\infty$ structure on $\cL_{sugra}$ equivariant for this $\C^\times$ action. 
The odd symplectic form on $\cL_{sugra}$ is weight $-1$ for this $\C^\times$ action. 
\end{prop}

%\begin{proof}
%The only nontrivial bracket to check compatibility with the $\C^\times$ action is the one which takes two $\gamma$-type sections to a $\mu$-type section of the form
%\[
%[\gamma_1, \gamma_2] = \Omega^{-1} \vee (\del \gamma_1 \wedge \del \gamma_2) 
%\]
%where $\Omega \vee (-)$ is the operation which takes a four-form 
%
%It suffices to check equivariance in local coordinates; there are three cases.
%Suppose $\gamma_1(t;w,z) = f^i(t;w,z) \d z_i$ and $\gamma_2(t;w,z) = g^j(t;w,z) \d z_j$where $f^i,g^j$ are de Rham--Dolbeault forms for $i,j=1,2,3$. 
%Then $\lambda \cdot \gamma_1 = \lambda^{-1} f^i(\lambda t;\lambda w, z) \d z_i$ and $\lambda \cdot \gamma_2 = \lambda^{-1} g^j(\lambda t;\lambda w, z) \d z_j$. 
%Thus, if the $\mu$-type field is expanded as $[\gamma_1,\gamma_2] = F^k(t;w,z) \del_{z_k} + G^a (t;w,z) \del_{w_a}$ for some de Rham--Dolbeault forms $F^k,G^a$, $k=1,2,3,a=1,2$, it suffices to show that 
%\beqn
%\begin{array}{llll}
%\label{eqn:lambda1} \lambda \cdot \left(F^k(t;w,z) \del_{z_k} \right) & = \lambda^{-2} F^k(\lambda t;\lambda w,z) \del_{z_k} \\
%\lambda \cdot \left(G^a(t;w,z) \del_{w_a} \right) & = \lambda^{-2} G^a(\lambda t;\lambda w,z) \del_{w_a} .
%\end{array}
%\eeqn
%We expand $\del \gamma_1 \wedge \del \gamma_2$ to three terms
%\begin{multline}
%\del_{w_a} f^i(t;w,z) \del_{z_l} g^j(t;w,z) \d w_a \d z_i \d z_l \d z_j + \del_{z_k} f^i(t;w,z) \del_{w_b} g^j(t;w,z) \d z_k \d z_i \d w_b \d z_j \\  + \del_{w_a} f^i(t;w,z) \del_{w_b} g^j(t;w,z) \d w_a \d z_i \d w_b \d z_j .
%\end{multline}
%Thus in the notation above we have 
%\begin{align*}
%F^k (t;w,z) & = \ep_{ab} \ep_{ijk} \del_{w_a} f^i(t;w,z) \del_{w_b} g^j(t;w,z) \\
%G^a(t;w,z) & = \ep_{ilj} \ep_{ab} \del_{w_a} f^i(t;w,z) \del_{z_l} g^j(t;w,z)
%+ \ep_{kij} \ep_{ab} \del_{z_k} f^i(t;w,z) \del_{w_a} g^j(t;w,z)  .
%\end{align*}
%We compute the action of $\lambda \in \C^\times$ on $\mu = F^k \del_{z_k}$
%\[
%\lambda \cdot \left(F^k \del_{z_k} \right) = \ep_{ab} \ep_{ijk} \lambda^{-1} \del_{w_a} f^i(\lambda t;\lambda w,z) \lambda^{-1} \del_{w_b} g^j(\lambda t;\lambda w,z) \del_{z_k} .
%\]
%Thus the first part of \eqref{eqn:lambda1} is satisfied. 
%Similarly we observe that $\lambda \cdot \left(G^a(t;w,z) \del_{w_a} \right)  = \lambda^{-2} G^a(\lambda t;\lambda w,z) \del_{w_a}$. 
%
%The next case is when $\gamma_1(t;w,z) = f^i(t;w,z) \d z_i$ and $\gamma_2(t;w,z) = g^a(t;w,z) \d w_a$where $f^i,g^a$ are de Rham--Dolbeault forms for $i=1,2,3$ and $a=1,2$. 
%Then $\lambda \cdot \gamma_1 = \lambda^{-1} f^i(\lambda t;\lambda w, z) \d z_i$ and $\lambda \cdot \gamma_2 = g^a(\lambda t;\lambda w, z) \d w_a$. 
%Thus, if the $\mu$-type field is expanded as $[\gamma_1,\gamma_2] = F^k(t;w,z) \del_{z_k} + G^a (t;w,z) \del_{w_a}$ for some de Rham--Dolbeault forms $F^k,G^a$, $k=1,2,3,a=1,2$, it suffices to show that 
%\beqn
%\begin{array}{llll}
%\label{eqn:lambda2} \lambda \cdot \left(F^k(t;w,z) \del_{z_k} \right) & = \lambda^{-1} F^k(\lambda t;\lambda w,z) \del_{z_k} \\
%\lambda \cdot \left(G^a(t;w,z) \del_{w_a} \right) & = \lambda^{-1} G^a(\lambda t;\lambda w,z) \del_{w_a} .
%\end{array}
%\eeqn
%We expand $\del \gamma_1 \wedge \del \gamma_2$ to three terms
%\begin{multline}
%\del_{w_b} f^i(t;w,z) \del_{z_j} g^a(t;w,z) \d w_b \d z_i \d z_j \d w_a + \del_{z_k} f^i(t;w,z) \del_{w_b} g^a(t;w,z) \d z_k \d z_i \d w_b \d w_a \\  + \del_{z_j} f^i(t;w,z) \del_{z_k} g^a(t;w,z) \d z_j \d z_i \d z_k \d w_a .
%\end{multline}
%Thus in the notation above we have 
%\begin{align*}
%F^k (t;w,z) & = \ep_{ab} \ep_{ijk} \del_{w_a} f^i(t;w,z) \del_{z_l} g^b(t;w,z) + \ep_{ab} \ep_{kij} \del_{z_k} f^i(t;w,z) \del_{w_a} g^b(t;w,z) \\
%G^a(t;w,z) & = \ep_{ab} \ep_{jik} \del_{z_j} f^i(t;w,z) \del_{z_k} g^b(t;w,z) .
%\end{align*}
%We compute the action of $\lambda \in \C^\times$ on $\mu = G^a \del_{z_a}$
%\[
%\lambda \cdot \left(G^a \del_{w_a} \right) = \ep_{ab} \ep_{jik} \del_{z_j} f^i(t;w,z) \del_{z_k} g^b(t;w,z) \lambda^{-1} \del_{w_a} .
%\]
%Thus the second part of \eqref{eqn:lambda2} is satisfied. 
%Similarly we observe that $\lambda \cdot \left(F^k(t;w,z) \del_{z_k} \right)  = \lambda^{-2} F^k(\lambda t;\lambda w,z) \del_{z_k}$. 
%
%The last case is when $\gamma_1(t;w,z) = f^a(t;w,z) \d w_a$ and $\gamma_2(t;w,z) = g^b(t;w,z) \d w_b$where $f^a,g^b$ are de Rham--Dolbeault forms for and $a,b=1,2$.
%The argument is nearly identical so we omit the details. 
%\end{proof}

We consider this decomposition at the level of the local $L_\infty$ algebra $\cG_Z$.
Recall that the Chevalley--Eilenberg cochains of this local $L_\infty$ algebra is the restricted factorization algebra~$\Bar{\pi}_* \Obs_{sugra} = \clie^\bu(\cG_Z)$. 
For each $n \in \Z$ and open set $U \subset Z$, let 
\[
\cG_Z(U)^{(n)}\subset \cG_Z(U)
\]
be the weight $n$ eigenspace with respect to this $\C^\times$ action.
As a corollary of the proposition, we get a product decomposition 
\beqn
\label{eqn:Gdecomp}
\cG_Z = \prod_{n \geq -1} \cG^{(n)} .
\eeqn
In particular, we see that $\cG^{(0)}_Z$ is itself a local Lie algebra (that we will soon describe). 
Moreover, every $\cG^{(n)}$, $n \geq -1$ is a (local) module for this local Lie algebra.

%The $\C^\times$ action is compatible with the $\Z/2$ graded $L_\infty$ structure on $\cL_{sugra}$. 
%Since the $n$th eigenspace is trivial when $n < -1$, we see that the product
%\[
%\cG_Z (U)  \prod_{n \geq -1} (\pi_*\cL_{sugra})(U)^{(n)}
%\]
%is equipped with the structure of a $\Z/2$ graded $L_\infty$ algebra.
%In this way, the assignment 
%\[
%\Bar{\pi}_* \cL_{sugra} \colon U \mapsto (\Bar{\pi}_* \cL_{sugra})(U) 
%\]
%defines a sheaf of $\Z/2$ graded $L_\infty$ algebras on $Z$. 

%\begin{lem}
%\label{lem:technical}
%For each $n$, $\cG
%\[
%U \mapsto \Omega^{0,\bu}(U, \cV_{fivebrane}^{(n)})
%\]
%for some finite rank super holomorphic vector bundle $\cV_{fivebrane}^{(n)}$ on $Z$.
%For each $n$, the differential is of the form $\dbar + Q^{hol}$.
%In particular, this endows $\Bar{\pi}_* \cL_{sugra}$ with the structure of a pro-vector bundle on $Z$. 
%\end{lem}
%
%\begin{proof}
%In \S \ref{s:Lsugra} we introduced a holomorphic vector bundle $L_X$ defined on any Calabi--Yau five-fold $X$ whose sheaf of holomorphic sections was equipped with a $\Z/2$ graded $L_\infty$ structure. 
%Here, we consider the Calabi--Yau fivefold $X = X_Z$ as defined above.
%For any open subset $U \subset Z$ of the threefold the value of $\cL_{sugra}(\pi^{-1}U)$ as a sheaf of super vector spaces is 
%\[
%\Omega^\bu(\R) \otimes \Omega^{0,\bu}(p^{-1} U, L_{X_Z}),
%\]
%where $p \colon X_Z \to Z$ is the rank two bundle over $Z$. 
%The one-ary structure map, or differential, is of the form $\d_{dR} + \dbar + Q^{hol}$ where $Q^{hol}$ is some odd holomorphic differential operator acting on sections of $L_{X_Z}$. 
%
%Since the de Rham complex of $\R$ is contractible we have a quasi-isomorphism of $L_\infty$ algebras
%\[
%\cL_{sugra}(\pi^{-1}U) \simeq \Omega^{0,\bu}(p^{-1} U, L_{X_Z})
%\]
%for every $U \subset Z$. 
%Without loss of generality, we can assume that $U$ is a coordinate chart for the vector bundle $X_Z$ so that $X_Z|_U \cong U \times \C^2$ and where the bundle $L_{X_Z}$ splits as 
%\[
%L_{X_Z}|_{\pi^{-1} U} \cong L' \boxtimes L'' 
%\]
%where $L'$ is a holomorphic super vector bundle on $U$ and $L''$ is.a holomorphic vector on the fiber $\C^2$.
%Then, we can further identify this with a Dolbeault complex of the form
%\[
%\Omega^{0,\bu} \left(U, L' \otimes \Omega^{0,\bu}(\C^2, L'') \right) . 
%\]
%By the Dolbeault Poincar\'e lemma applied to the holomorphic vector bundle $L''$ over $\C^2$,
%\[
%\Omega^{0,\bu} \left(U, L' \otimes M\right) 
%\]
%where we now interpret $L' \otimes M$ as an infinite rank bundle over $U \subset Z$---
%the remaining differential is $\dbar_U + Q^{hol}$ where $Q^{hol}$ is a holomorphic differential operator.
%
%With these simplifications, it suffices to show that the weight $n$ eigenspace of $L' \otimes M$ is finite rank over $U$ \brian{almost finished}
%%we can fix a quasi-isomorphism $M \simeq \Omega^{0,\bu}(\C^2, L'')$ where $M$ is a 
%\end{proof}


\parsec[s:weight-1]

The first non trivial case is the weight $(-1)$ piece $\cG_Z^{(-1)}$.
We will show that
$\cG_Z^{(-1)}$ is equivalent as an abelian local Lie algebra to one of the form $\Omega^{0,\bu}(Z, \cV^{(-1)})$ where $\cV^{(-1)}$ is the holomorphic vector bundle
\[
\cV^{(-1)} = K^{1/2}_Z \otimes \C^2 \oplus \cO_Z \oplus \Pi \T^*_Z .
\]
This Dolbeault complex is equipped with the natural $\dbar$ operator and the $\del$ operator which goes from $\Omega^{0,\bu}(Z)$ to $\Omega^{1,\bu}(Z)$. 

Explicitly weight $(-1)$ summand $\cV^{(-1)}$ consists of:
\begin{itemize}
\item 
Vector fields which are locally of the form $f^a(z) \del_{w_a}$ where $f^{a}(z)$ is a holomorphic function on $Z$.
Notice that these vector fields are automatically divergence-free.
Since $\del_{w_a}$ is treated as a section of $K^{1/2}_Z$ we see that these are local sections of $K^{1/2}_Z \otimes \C^2$. 
\item 
Next there are holomorphic functions (type $\beta$) and holomorphic $one$-forms (type $\gamma$) which do not depend on $w_a$. 
Since $\del$ is weight zero for the decomposition these forms combine to form the $\Z/2$ graded complex of sheaves
\[
\cO_Z \xto{\del} \Pi \Omega^{1,hol}_Z .
\]
\end{itemize}

\parsec[s:weight0]

The weight zero summand $\cG_Z^{(0)}$ is special because it carries the induced structure of a local $L_\infty$ algebra on $Z$ inherited from the $L_\infty$ algebra $\cL_{sugra}$.
As a local Lie algebra it is equivalent to one of the form $\Omega^{0,\bu}(Z, \cV^{(0)})$.
We will prove that it is equivalent to a local Lie algebra version of the exceptional super Lie algebra $E(3|6)$. 

By local version, we mean a version of $E(3|6)$ which exists as a sheaf of super Lie algebras of the form $\Omega^{0,\bu}(Z, \cV^{(0)})$ on any complex threefold $Z$ equipped with a square-root of its canonical bundle.
The even part of this sheaf is simply a semi-direct product 
\beqn
\Vect^{hol}(Z) \oplus \sl(2) \otimes \cO_Z .
\eeqn
Notice the similarity with the even part of $E(3|6)$ in equation \eqref{eqn:evenE36}. 
The odd part is 
\beqn
\Omega^{1,hol}(Z, K^{-1/2}_Z) \otimes \C^2 
\eeqn
Again, we observe the similarities with the odd part of $E(3|6)$ given in equation \eqref{eqn:oddE36}. 
The Lie bracket on this sheaf of super vector spaces is defined analogously to the bracket on $E(3|6)$. 
In particular, the bracket operations only involve holomorphic differential operators and hence extends to a local dg super Lie algebra structure on the Dolbeault resolution of the above holomorphic vector bundles. 
We denote this local dg super Lie algebra by $\cE(3|6)$.

When $Z = \C^3$ the $\infty$-jets at $0 \in \C^3$ of the local Lie algebra $\cE(3|6)$ is equivalent to the exceptional super Lie algebra $E(3|6)$.

\begin{prop}\label{prop:v0}
As a $\Z/2$ graded local Lie algebra on $Z$, the weight zero summand~$\cG^{(0)}_Z$ is equivalent to $\cE(3|6)$. 
\end{prop}

%Since this is the weight zero component, the sheaf of sections is endowed with the structure of a Lie algebra inherited from the local $L_\infty$ algebra $\cL_{sugra}$.
%More precisely he weight zero piece is a local Lie algebra enhancement of this exceptional Lie algebra.

To see that $\cG^{(0)}$ and $\cE(3|6)$ are equivalent as super vector bundles we must show that
\[
\cV^{(0)} = \T_Z \oplus \lie{sl}(2) \otimes \cO_Z \oplus \Pi \left( \T^*_Z \otimes K^{-1/2}_Z \otimes \C^2 \right).
\]
The local $L_\infty$ structure of $\cL_{sugra}$ endows the Dolbeault complex of this holomorphic vector bundle with a local Lie algebra structure.
The differential turns out to be the natural $\dbar$ operator.
We describe the bracket below.

Explicitly, we enumerate all holomorphic sections of $\cV^{(0)}$:
\begin{itemize}
\item Vector fields which are locally of the form 
\[
\mu = f^i(z) \del_{z_i} - \frac12 \del_{z_i} f^i(z) w_a \del_{w_a}
\]
and vector fields which are locally of the form
\[
\mu = g(z) A^{ab} w_a \del_{w_b}.
\]
The condition that $\mu$ be divergence-free implies that $(A^{ab}) \in \lie{sl}(2)$.
We identify the first such vector fields with sections of $\T_Z$ and the second such vector fields with sections of $\lie{sl}(2) \otimes \cO_Z$. 
\item Holomorphic functions of the form $\beta = w_a f(z)$ and holomorphic one-forms of the form $\gamma = g^{a}(z) \d w_a + h^{a} (z) w_a \d z$. 
Since $\del$ is weight zero, such sections are equipped with a differential 
\[
\Gamma^{hol}(Z, K^{-1/2}\otimes \C^2) \xto{D} \Pi \Gamma^{hol}(Z, K^{-1/2} \otimes \C^2) \oplus \Pi\Omega^{1,hol}(Z, K^{-1/2}\otimes \C^2) 
\]
which sends $w_a f(z) \mapsto (f(z) , w_a \del f (z))$. 
This $\Z/2$ graded complex of sheaves is equivalent to $\Pi\Omega^{1,hol}(Z, K^{-1/2}\otimes \C^2)$. 
\end{itemize}

The Lie bracket on $\cG_Z^{(0)}$ is defined from a Lie algebra structure on holomorphic sections of $\cV^{(0)}$ inherited from the $L_\infty$ algebra $\cL_{sugra}$. 
In fact, there is only a two-ary bracket, and an immediate calculation shows that it agrees with the Lie bracket on $\cE(3|6)$. 

%\begin{rmk}
%In \cite{KacClass} an embedding of super Lie algebras from $E(3|6)$ into $E(5|10)$ is constructed.
%Recall that in the case $Z = \C^3$, the $\infty$-jets at $0 \in \C^3$ of the local Lie algebra $\cE(3|6)$ is precisely the exceptional super Lie algebra $E(3|6)$.
%Similarly, the $\infty$-jets at $0 \in \C^3$ of the local $L_\infty$ algebra $\cG_{\C^3}$ is $\Hat{E(5|10)}$.
%The embedding of local $L_\infty$ algebras on $Z$ from $\cE(3|6)$ into $\cG_Z$ that we just described agrees with the embedding of \cite{KacClass} upon taking $\infty$-jets.
%(Note that the central term in $\Hat{E(5|10)}$ plays no role herem since it sits in $\C^\times$-weight $-1$.)
%\end{rmk}

%\begin{itemize}
%\item For $\Vect^{hol}(Z)$ there is the standard commutator of holomorphic vector fields. 
%This acts on the sections of $\lie{sl}(2) \otimes \cO_Z$ and $\Pi\Omega^{1,hol}(Z, K^{-1/2}_Z \otimes \C^2)$ by Lie derivative. 
%\item On sections of $\lie{sl}(2) \otimes \cO_Z$ there is the matrix commutator. 
%This acts on the odd part $\Pi\Omega^{1,hol}(Z, K^{-1/2}_Z \otimes \C^2)$ where we view $\C^2$ as the fundamental $\lie{sl}(2)$ representation. 
%\item Finally, and most interestingly, there is a bracket of the form
%\[
%\Pi\Omega^{1,hol}(Z, K^{-1/2}_Z \otimes \C^2) \times \Pi\Omega^{1,hol}(Z, K^{-1/2}_Z \otimes \C^2) \to \Vect^{hol}(Z) \oplus \lie{sl}(2) \otimes \cO_Z 
%\]
%given in coordinates by
%\begin{multline}
%[f^{ai}(z) w_a \d z_i, g^{bj} (z) w_b \d z_j] = \ep_{ijk} \ep_{ab} f^{ai} (z) g^{bj}(z) \del_{z_k} \\ + \ep_{ijk} \ep_{bc}  \del_{z_k} f^{ai}(z) g^{bj}(z) w_a \del_{w_c} + \ep_{ijk} \ep_{ac} f^{ai}(z) \del_{z_k} g^{bj}(z) w_b \del_{w_c} .
%\end{multline}
%The top line is a holomorphic vector field on~$Z$ and the bottom line is a $\lie{sl}(2)$-valued holomorphic function on~$Z$.
%\end{itemize}


%\parsec[s:mainfact]
%
%Since $\cG_Z$ is a pro-vector bundle with a compatible $L_\infty$ structure, the assignment 
%\[
%U\subset Z \mapsto \clie^\bu\left(\Bar{\pi}_* \cL_{sugra}(U)\right) 
%\]
%has the structure of a factorization algebra on the three-fold $Z$. 
%We denote this factorization algebra on $Z$ by $\Obs_{sugra}|_Z$. 

%\begin{prop}
%Let $Z$ be a Calabi--Yau three-fold and $\pi \colon Z \times \C^2 \times \R \to Z$ be the projection. 
%There is a pro local Lie algebra $\Bar{\pi}_*\cL_{sugra}$ and a factorization algebra $\Bar{\pi}_* \Obs_{sugra} = \clie^\bu(\Bar{\pi}_* \cL_{sugra})$ on $Z$ such that:
%\begin{itemize}
%\item[(1)] there is a natural inclusion of of factorization algebras on $Z$
%\[
%\Bar{\pi}_* \Obs_{sugra} \hookrightarrow \pi_* \Obs_{sugra}
%\]
%which is dense at the level of cohomology. 
%\item[(2)] there is a weight grading on $\cL_{\pi,sugra}$ which is concentrated in degrees $\geq -1$ and gives rise to a decomposition of vector bundles
%\beqn\label{eqn:decomp3}
%\Bar{\pi}_{*} \cL_{sugra} = \prod_{n \geq -1} \cV_{n} 
%\eeqn
%\item[(3)]
%In weight zero, there is an equivalence of local Lie algebras on $Z$ 
%\[
%\cV_0 \simeq \cE(3|6)|Z 
%\]
%where $\cE(3|6)|Z$ is a local Lie algebra enhancement of the exceptional simple super Lie algebra $E(3|6)$. 
%\end{itemize}
%\end{prop}

%We want to argue that $\Obs_{sugra} |_{\C^3} \cong \Bar{\pi}_* \Obs_{sugra}$. 

\parsec[s:general_decomp]

We move on to give the following general description of the weight $j$ component $\cG^{(j)}$.
Since we have already described $j = -1,0$ we focus on $j \geq 1$.

\begin{prop}
\label{prop:Vj}
Let $j \geq 1$. 
The complex of vector bundles $\cG^{(j)}$ is quasi-isomorphic to
\beqn
\Omega^{0,\bu}(Z, \cV^{(j)}) 
\eeqn
where $\cV^{(j)}$ is the super holomorphic vector bundle 
\beqn
\label{eqn:Vj}
\begin{tikzcd}
\ul{even} & \ul{odd} \\
S^{j}(\C^2) \otimes \T_Z \otimes K^{-j/2}_Z & S^{j-1}(\C^2) \otimes K^{-(j+1)/2}_Z \\
S^{j+2}(\C^2) \otimes K^{-j/2}_Z & S^{j+1}(\C^2) \otimes \T^*_Z \otimes K^{-(j+1)/2}_Z .
\end{tikzcd}
\eeqn
\end{prop}

\begin{proof}
As above, let $\C^2$ stand for the fundamental $\lie{sl}(2)$ representation and so the irreducible highest weight $j$ representation is $S^j(\C^2)$.

The weight $j \geq 1$ complex of vector bundles $\cG^{(j)}$ is readily seen to be of the following form
\beqn
\begin{tikzcd}
\ul{even} & \ul{odd} \\
\Omega^{0,\bu}(Z , \T_Z \otimes K^{-j/2}_Z) \otimes S^j(\C^2) \ar[dr, "D_1"] \\ & \Omega^{0,\bu}(Z, K^{-j/2}_Z) \otimes S^j(\C^2) \\ 
\Omega^{0,\bu}(Z , K^{-j/2}_Z) \otimes S^{j+1}(\C^2) \otimes \C^2 \ar[ur, "D_2"'] & \\
& \Omega^{1,\bu}(Z, K^{-(j+1)/2}_Z) \otimes S^{j+1}(\C^2) \\ 
\Omega^{0,\bu}(Z, K^{-(j+1)/2}_Z) \otimes S^{j+1}(\C^2) \ar[ur, "D_3"] \ar[dr, "D_4"'] \\
& \Omega^{0,\bu}(Z, K^{-(j+1)/2}_Z) \otimes S^j(\C^2) \otimes \C^2 . 
\end{tikzcd}
\eeqn

Here, as usual, the $\dbar$ operator is left implicit.
Let us describe the differentials $D_1,D_2,D_3,D_4$.
Recall that the five-fold $X = \text{Tot}(K_Z^{1/2} \otimes \C^2)$ is equipped with a non-vanishing holomorphic volume form. 
The corresponding divergence operator $\div$ restricted to the weight $j$ subspace is $\div = D_1 + D_2$. 
The differential $D_2$ is given by the identity on $\Omega^{0,\bu}(Z, K^{-j/2}_Z)$ tensored with the natural $\lie{sl}(2)$-equivariant projection
\beqn
S^{j+1}(\C^2) \otimes \C^2 \cong S^{j+2}(\C^2) \oplus S^{j}(\C^2) \twoheadrightarrow S^{j}(\C^2) .
\eeqn

The holomorphic de Rham operator on $X$ is $\del = D_3+D_4$. 
The differential $D_4$ is the identity on $\Omega^{0,\bu}(Z, K^{-(j+1)/2}$ tensored with the natural $\lie{sl}(2)$-equivariant inclusion
\beqn
S^{j+1}(\C^2) \hookrightarrow S^{j-1}(\C^2) \oplus S^{j+1}(\C^2) \cong S^j(\C^2) \otimes \C^2 .
\eeqn

There is a spectral sequence whose first term is computed by the $D_2,D_4$ cohomology. 
This term in the spectral sequence is isomorphic the complex $\Omega^{0,\bu}(Z, \cV^{(j)})$ where $\cV^{(j)}$ is as in equation \eqref{eqn:Vj}.
There are no further terms in the spectral sequence, so the result follows. 
\end{proof}

One can write down an explicit quasi-isomorphism
\beqn
\Phi \colon \Omega^{0,\bu}(Z, \cV^{(j)}) \to \cG^{(j)} 
\eeqn 
as follows.
\begin{itemize}
\item First we describe what the map looks like on the even summands.
Locally a section of $\Omega^{0,\bu}(Z, S^j(\C^2) \otimes T_Z \otimes K^{-j/2}_Z)$ is of the form
\beqn
f(w) \otimes g_i(z) \del_{z_i} 
\eeqn
where $f(w) \in \C[w_1,w_2]_j$ is a homogenous degree $j$ polynomial in the variables $w_1,w_2$ and the $g_i(z)$'s are Dolbeault forms on $Z$.
Define the divergence-free $\mu$-type field $\Phi(f(w) \otimes g_i(z) \del_{z_i})$ by the expression
\beqn
f(w) g(z) \del_{z_i} - \frac{1}{j+2} \left(\del_{z_i} g_i(z)\right) f(w) E_w
\eeqn
where $E_w = w_a \del_{w_a}$ is the Euler vector field in the direction transverse to the brane. 
\item Locally a section of $\Omega^{0,\bu}(Z, S^{j+2} \otimes K^{-j/2}_Z)$ is of the form
\beqn
f(w) \otimes g(z)  
\eeqn
where $f(w) \in \C[w_1,w_2]_{j+2}$ is a homogenous degree $j$ polynomial in the variables $w_1,w_2$ and $g(z)$ is a Dolbeault form on $Z$.
Define the divergence-free $\mu$-type field $\Phi(f(w) \otimes g(z))$ by the expression
\beqn
g(z)(\del_{w_1} f (w) \del_{w_2} - \del_{w_2} f(w) \del_{w_1}) 
\eeqn
\item Next for the odd summands.
Locally a section of $\Omega^{0,\bu}(Z, S^{j-1} \otimes K^{-(j+1)/2}_Z)$ is of the form
\beqn
f(w) \otimes g(z)  
\eeqn
where $f(w) \in \C[w_1,w_2]_{j-1}$ is a homogenous degree $j-1$ polynomial in the variables $w_1,w_2$ and $g(z)$ is a Dolbeault form on $Z$.
Define the $\gamma$-field $\Phi(f(w) \otimes g(z))$ by the expression
\beqn
\frac12 g(z)f(w)(w_1 \d w_2 - w_2 \d w_1) .
\eeqn
\item
Locally a section of $\Omega^{0,\bu}(Z, S^{j+1} \otimes T^*_Z \otimes K^{-(j+1)/2}_Z)$ is of the form
\beqn
f(w) \otimes g^i(z)  \d z_i
\eeqn
where $f(w) \in \C[w_1,w_2]_{j+1}$ is a homogenous degree $j+1$ polynomial in the variables $w_1,w_2$ and the $g^i(z)$'s are Dolbeault forms on $Z$.
Define the $\gamma$-field $\Phi(f(w) \otimes g^i(z) \d z_i)$ by the expression
\beqn
f(w) g^i (z) \d z_i .
\eeqn
\end{itemize}

The complex $\cG^{(j)}$ is manifestly a local module for the local Lie algebra $\cG^{(0)}$. 
With the identification $\cG^{(0)} \simeq \cE(3|6)$ and $\cG^{(j)} \simeq \Omega^{0,\bu}(Z, \cV^{(j)})$ we can read off this module structure completely explicitly. 

Recall that as a complex of vector bundles $\cE(3|6) = \Omega^{0,\bu}(Z, \cV^{(0)})$ where 
\beqn
\cV^{(0)} = \T_Z \oplus \lie{sl}(2) \otimes \cO_Z \oplus \Pi \left( \T^*_Z \otimes K^{-1/2}_Z \otimes \C^2 \right) .
\eeqn
The local Lie algebra structure on $\cE(3|6)$ arises from a Lie algebra structure on the sheaf of holomorphic sections of $\cV^{(0)}$.
Likewise, the $\cE(3|6)$-module structure on $\cG^{(j)} \simeq \Omega^{0,\bu}(Z, \cV^{(j)})$ arises from a holomorphic $\cV^{(0)}$-module structure on the holomorphic sections of $\cV^{(j)}$. 
By holomorphic, we mean that the structure maps are all holomorphic differential operators. 

We describe this $\cV^{(0)}$-module structure on $\cV^{(j)}$ explicitly.
Holomorphic sections of $\T_Z$ will act by Lie derivative on $\cV^{(j)}$.
Holomorphic $\sl(2)$-valued functions will also act in the natural way since each component of $\cV^{(j)}$ is labeled by an irreducible $\lie{sl}(2)$ representation. 

Finally, we need to explain how holomorphic sections of the odd component $\T^*_Z \otimes K^{-1/2}_Z \otimes \C^2$ of $\cV^{(0)}$ act.
We first give a global description of this action, and then we will write down the explicit formula in local coordinates.
For the local coordinate description recall that a general section of $\T^*_Z \otimes K^{-1/2}_Z \otimes \C^2$ has the form $w_a g^i(z) \d z_i$ where $g^i(z)$ is a holomorphic function. 
\begin{itemize}
\item The odd part of $\cV^{(0)}$ acts on the component $S^{j}(\C^2) \otimes \T_Z \otimes K^{-j/2}_Z$ through the composition
\beqn
\begin{tikzcd}
\left(\T^*_Z \otimes K^{-1/2}_Z \otimes \C^2\right) \otimes \left(S^{j}(\C^2) \otimes \T_Z \otimes K^{-j/2}_Z\right) \ar[r,"\cong"] & \left(S^{j-1}(\C^2) \oplus S^{j+1}(\C^2)\right) \otimes \left(\T_Z \otimes \T^*_Z \otimes K^{-(j+1)/2}_Z\right) \ar[dl] \ar[d] \\
S^{j-1}(\C^2) \otimes K^{-(j+1)/2}_Z & S^{j+1}(\C^2) \otimes \T^*_Z \otimes K^{-(j+1)/2}_Z 
\end{tikzcd}
\eeqn
Here, the leftmost downward arrow is the evident $\lie{sl}(2)$ projection together with the canonical pairing between sections of $\T_Z$ and $\T^*_Z$.
The rightmost downward arrow is the other $\lie{sl}(2)$ projection together with the Lie derivative of holomorphic one-forms.  
Given a local section $f(w) \otimes h_i(z) \del_{z_i}$ of $S^j(\C^2) \otimes \T_Z \otimes K_Z^{-j/2}$ an explicit formula for this action is
\begin{align*}
(w_a g^i(z) \d z_i) \cdot (f(w) \otimes h_k(z) \del_{z_k}) & = \ep_{ab} (\del_{w_b} f(w)) (g^i h_i)(z)  \\ & + w_a f(w) L_{h_k \del_{z_k}} (g^i \d z_i) .
\end{align*}
\item The odd part of $\cV^{(0)}$ acts on the component $S^{j+2}(\C^2) \otimes K^{-j/2}_Z$ through the composition
\beqn
\begin{tikzcd}
\left(\T^*_Z \otimes K^{-1/2}_Z \otimes \C^2\right) \otimes \left(S^{j+2}(\C^2) \otimes K^{-j/2}_Z\right) \ar[r,"\cong"] & \left(S^{j+1}(\C^2) \oplus S^{j+3}(\C^2)\right) \otimes \left(\T^*_Z \otimes K^{-(j+1)/2}_Z\right) \ar[d] \\
& S^{j+1}(\C^2) \otimes \T^*_Z \otimes K^{-(j+1)/2}_Z
\end{tikzcd}
\eeqn
where the downward arrow is induced by the evident $\lie{sl}(2)$ projection.
\item The odd part of $\cV^{(0)}$ acts on the component $S^{j-1}(\C^2) \otimes K^{-(j+1)/2}_Z$ through the composition
\beqn
\begin{tikzcd}
\left(\T^*_Z \otimes K^{-1/2}_Z \otimes \C^2\right) \otimes \left(S^{j-1}(\C^2) \otimes K^{-j/2}_Z\right) \ar[r,"\cong"] & \left(S^{j-2}(\C^2) \oplus S^{j}(\C^2)\right) \otimes \left(\T^*_Z \otimes K^{-1}_Z K^{-j/2}_Z\right) \ar[d] \\
& S^{j}(\C^2) \otimes \T_Z \otimes K^{-j/2}_Z
\end{tikzcd}
\eeqn
where the downward arrow is induced by the evident $\lie{sl}(2)$ projection together with the holomorphic de Rham operator taking holomorphic one-forms to holomorphic two-forms. 
\item Finally, the odd part of $\cV^{(0)}$ acts on the component $S^{j+1}(\C^2) \otimes \T^*_Z \otimes K^{-(j+1)/2}_Z$ through the composition
\beqn
\begin{tikzcd}
\left(\T^*_Z \otimes K^{-1/2}_Z \otimes \C^2\right) \otimes \left(S^{j+1}(\C^2) \otimes \T^*_Z \otimes K^{-j/2}_Z\right) \ar[r,"\cong"] & \left(S^{j}(\C^2) \oplus S^{j+2}(\C^2)\right) \otimes \left(\T^*_Z \otimes \T^*_Z \otimes K^{-1}_Z K^{-j/2}_Z\right) \ar[dl] \ar[d] \\
S^j (\C^2) \otimes \T_Z \otimes K_Z^{-j/2} & S^{j+2}(\C^2)  \otimes K^{-j/2}_Z
\end{tikzcd}
\eeqn
where the leftmost downward arrow is induced by the evident $\lie{sl}(2)$ projection.
The rightmost downward arrow is induced by the remaining $\lie{sl}(2)$ projection together with the holomorphic de Rham operator taking holomorphic two-forms to holomorphic three-forms.
\end{itemize}

\parsec[s:kacrelation]

In the case $Z = \C^3$, the product decomposition of $\cG = \cG_{\C^3}$ in equation \eqref{eqn:Gdecomp} is closely related to a decomposition of the exceptional simple super Lie algebra $E(5|10)$ studied in \cite{KR2}. 
Here, the eleven-manifold bulk is just 
\beqn
\R \times {\rm Tot}(K_{\C^3}^{1/2} \otimes \C^2) \simeq \R \times \C^5 .
\eeqn

In \S \ref{s:e510} we recalled the result from \cite{RSW} that the $\infty$-jets of $\cL_{sugra}$ at $0 \in \R \times \C^5$ is quasi-isomorphic to $\Hat{E(5|10)}$, a certain central extension of $E(5|10)$.
It follows that the $\infty$-jets of the local Lie algebra $\cG$ at $0 \in \C^3$ is also quasi-isomorphic to $\Hat{E(5|10)}$. 

In \cite{KR2} the following weight decomposition of $E(5|10)$ is constructed.
Denote by $\{z_i\}$ the local coordinates along the three-fold $Z=\C^3$ that the fivebrane wraps and $\{w_a\}$ for the transverse holomorphic coordinates to the zero section in the five-fold ${\rm Tot}(K_{\C^3}^{1/2} \otimes \C^2)$.
Assign the following weights to the super Lie algebra $E(5|10)$. 
\begin{itemize} 
\item the coordinate $z_i$ has weight zero, $\wt(z_i) = 0$. 
\item the coordinate $w_a$ has weight $+1$, $\wt(w_a) = +1$. 
\item the parity of an element carries an additional weight of $-1$. 
Thus, for example, the odd element $[\d w_1 \d z_1] \in \Omega^{2,cl}(\Hat{D}^5)$ carries weight $+1 - 1 = 0$. 
(If we think about the odd part as the space of closed two-forms then equivalently this grading translates to the one-form symbol $\d(-)$ as carrying weight $-1/2$.)
\end{itemize} 
The weight grading is concentrated in degrees $\geq -1$. 
In particular, there is a decomposition of super vector spaces
\beqn\label{eqn:decomp1}
E(5|10) = \til V_{-1} \times \prod_{n \geq 0} V_n 
\eeqn
with $\til V_{-1}$ being the weight $-1$ subspace and $V_n$ being the weight $n$ subspace for $n \geq 0$.  
It is straightforward to verify that this weight grading is compatible with the super Lie algebra structure on $E(5|10)$.      

      v n                  
At the level of $\infty$-jets the the decomposition in equation \eqref{eqn:Gdecomp} induces a weight grading of $\Hat{E(5|10)}$ which extends the one on $E(5|10)$ that we just described by declaring that the central term have weight $-1$.
In this way, we get a related decomposition of super $L_\infty$ algebras
\beqn\label{eqn:decomp2}
\Hat{E(5|10)} = \prod_{n               \geq -1} V_n            
\eeqn                      
We will refer to this as the \textit{fivebrane decomposition} of $\Hat{E(5|10)}$.
Here $V_{-1}$ is a $\C$-extension of $\til V_{-1}$ defined in the decomposition \eqref{eqn:decomp1}.
Notice that for $n \geq 0$ the $V_n$'s are the same as in the non centrally extended case.

The decomposition in equation \eqref{eqn:decomp2} has the property that $V_0$ is isomorphic to $E(3|6)$ as super Lie algebras.
In particular, for each $n$, $V_n$ is a module for $V_0 = E(3|6)$.
In fact, each $V_n$ is an irreducible $E(3|6)$-module \cite{KR2}.

This is compatible with our description in proposition~\ref{prop:v0} where we showed that as local Lie algebras $\cG^{(0)} \simeq \cE(3|6)$. 
Indeed, the $\infty$-jets of the local Lie algebra $\cE(3|6)$ at $0 \in \C^3$ is quasi-isomorphic to $E(3|6)$. 
By the compatibility of our decomposition with \cite{KR2} we see that $V_n$ is quasi-isomorphic to the $\infty$-jets of $\cG^{(n)}$ at $0 \in \C^3$ as a module for $E(3|6)$.

\subsection{Koszul duality for factorization algebras: an ansatz}
\label{s:noether}

In quantum field theory Koszul duality naturally appears in the problem of coupling topological line operators to some ambient bulk theory. 
More generally, for higher dimensional topological defects, this problem is encoded by Koszul duality for the theory of $\EE_n$ algebras \cite{FrancisGaitsgory} \cite[\S 5.2]{LurieHA}.

More generally, we anticipate a general theory of Koszul duality for factorization algebras which should encode the problem of coupling arbitrary defects (without the condition of being topological).
Even for factorization algebras of holomorphic-topological nature this theory has not been studied in mathematics. 
Nevertheless, we will emphasize features that we expect this general form of Koszul duality to possess which will allow us to nail down its behavior on a rather general class of factorization algebras. 

In this first part of this subsection we briefly recall how Koszul duality enters in the problem of coupling line operators. 
We refer~\cite[\S 6]{CP1},~\cite[\S 8]{CG1}, or the review~\cite{PWkoszul} for more details. 
Then, we give an ansatz for Koszul duality for factorization algebras of the form $\clie^\bu(\cL)$ where $\cL$ is some local Lie algebra. 
From the point of view of the perturbative BV formalism this is not much of a restriction, all such factorization algebras of classical observables can be cast in this form. 

\parsec[s:lines]
Suppose that we have a bulk theory living on a spacetime of the form 
\[
\R \times M 
\]
where $M$ is some smooth manifold. 
Denote by $\cA$ the corresponding factorization algebra on $\R \times M$. 
The theory could have arbitrary behavior along $M$, but we assume that the theory is topological along $\R$. 
This means that when viewed as a factorization algebra on $\R$ that $\cA$ is locally constant and is hence equivalent to the data of an $\EE_1$ or $A_\infty$ algebra.

Next, assume that $\cB$ is another $\EE_1$ algebra, which we think of as being associated to some quantum mechanical system along the real line.
This is a local model for the desired line operator that we are attempting to couple to the bulk theory.
Koszul duality enters in the problem of coupling the two quantum mechanical systems $\cA$ and $\cB$---where we view $\cA$ simply as an $\EE_1$ algebra. 

A coupling of the two systems is Maurer--Cartan element in the algebra
\[
\alpha \in \cA \otimes \cB .
\]
That is, $\alpha$ is an element of ghost degree one which satisfies the Maurer--Cartan equation
\[
\delta \alpha + \alpha \star \alpha = 0 .
\]
Given such an $\alpha$ we can deform the algebra $\cA \otimes \cB$ by adding the term $[\alpha,-]$ to the differential. 
In other words, at the cochain level only the differential, not the product structure, is modified. 

In principle, there are more general ways to `couple' two $\EE_1$ algebras; generally this is controlled by the Hochschild cohomology which governs algebra deformations of $\cA \otimes \cB$. 
we will elaborate further on this definition. 
In \cite{CG1} (see also \cite{PWkoszul}) it is shown how this notion relates to the physicists description of coupling in terms of local Lagrangians.

To see Koszul duality, the key observation is that the data of the Maurer--Cartan element $\alpha$ is equivalent to the data of a map of $\EE_1$ algebras
\[
\alpha \colon \cA^! \to \cB 
\]
where $\cA^!$ is Koszul dual to the algebra $\cA$. 
We then have the following slogan: the Koszul dual of the algebra of observables of the bulk theory $\cA^!$ is the algebra of operators on the {\em universal} line defect supported on $\RR \times \{x\}$, where $x \in M$.  

\parsec[s:celine]

Before moving towards our definition of Koszul duality for a general class of factorization algebras, we briefly recast the case of duality for $\EE_1$ algebras in terms of factorization algebras. 

We will focus on a slight generalization of the standard Koszul duality between the exterior and symmetric algebras.

\begin{prop}
Let $\lie{g}$ be a Lie algebra and equip the filtered associative (and commutative) dg algebra $\clie^\bu(\fg)$ with the augmentation induced by the tautological homomorphism $0 \to \lie{g}$.
The Koszul dual of $\clie^\bu(\lie{g})$ with respect to this augmentation is equivalent to the universal enveloping algebra $U \fg$. 
\end{prop}

There are explicit models for the associative dg algebras $\clie^\bu(\lie{g})$ and $U \lie{g}$ as locally constant factorization algebras on $\R$.
First, observe that we can tensor $\fg$ with the commutative dg algebra of de Rham forms to obtain a dg Lie algebra $\fg \otimes \Omega^\bu(\R)$. 
This has a natural enhancement to a local dg Lie algebra as this is simply the smooth sections of the bundle of Lie algebras $\fg \otimes \wedge^\bu \T^*_\RR$ equipped with the de Rham operator.

Using this local Lie algebra, we obtain a model for the associative (and commutative) dg algebra $\clie^\bu(\fg)$ as the factorization algebra
\[
\clie^\bu(\fg \otimes \Omega^\bu_\R).
\]
To an open set $U \subset \R$ this produces the Chevellay--Eilenberg complex computing the Lie algebra {\em cohomology} of the dg Lie algebra $\fg \otimes \Omega^\bu(U)$---the $\fg$-valued de Rham forms on $U$.
 
Similarly, a model for $U \fg$ is the locally constant factorization algebra
\[
\clie_\bu(\fg \otimes \Omega^\bu_{\R,c}),
\]
see \cite[\S 3.4]{CG1}.
To an open set $U \subset \R$ this produces the Chevellay--Eilenberg complex computing the Lie algebra {\em homology} of the dg Lie algebra $\fg \otimes \Omega^\bu_c(U)$---the $\fg$-valued compactly supported de Rham forms on $U$.

\parsec[s:generalkoszul]

In analogy with the case of $\EE_1$, or locally constant factorization, algebras above we make the following definition. 

\begin{dfn}
Let $\cL$ be a local $L_\infty$ algebra on a manifold $M$ and consider the factorization algebra $\clie^\bu(\cL)$ which assigns to an open set $U \subset M$ the cochain complex $\clie^\bu(\cL(U))$. 
The \defterm{$!$-dual factorization algebra} is 
\[
\clie^\bu(\cL)^! \define \clie_\bu (\cL_{c}) 
\]
where $\clie_\bu(\cL_c)$ assigns to an open set $U \subset M$ the cochain complex $\clie_\bu(\cL_c(U))$.
In other words, the $!$-dual factorization algebra of $\clie^\bu(\cL)$ is the (untwisted) factorization enveloping algebra of the local Lie algebra $\cL$. 
\end{dfn} 

There are many things lacking in this definition. 
First, we do not define the $!$-dual for an arbitrary factorization algebra, only for ones of the form $\clie^\bu(\cL)$ where $\cL$ is a local Lie algebra. 
Also, we will not prove that $!$-dual satisfies any Koszul duality axioms. 
From the discussion above we see that $!$-dual does agree with Koszul duality in the case of associative algebras.\footnote{It is not difficult to see that $!$-duality for locally constant factorization algebras on $\R^n$ agrees with $\EE_n$ Koszul duality between $\clie^\bu(\fg)$, viewed as an $\EE_n$ algebra, and the $\EE_n$ enveloping algebra $U_{\EE_n} \fg$ \cite{Knudsen, Lurie}}

\parsec[s:noether]

While we don't prove that the factorization algebra $\clie^\bu(\cL)^!$ satisfies any sort of categorical duality, we point out a universality that is satisfied with reference to couplings.
In \cite[Part 3]{CG2} a factorization algebra enhancement of Noether's theorem is formulated. 
The general context is the following: 
\begin{itemize}
\item $\cB$ is the factorization algebra on spacetime $M$ of classical observables for some auxiliary theory in the BV formalism.
\item $\cL$ is a local Lie algebra on $M$ which acts on the theory by local symmetries. 
\end{itemize}
Then, the classical version of Noether's theorem for factorization algebras produces a map of factorization algebras 
\[
\clie_\bu(\cL_c) \to \cB .
\]

In our context, we imagine that the local Lie algebra $\cL$ describes a theory in the BV formalism.
The factorization algebra of classical observables is $\Obs = \clie^\bu(\cL)$. 
A natural way to couple the factorization algebras $\Obs$ and $\cB$ is to ask that $\cL$ act on $\cB$ as above.
Then, this result produces a map of factorization algebras $\Obs^! \to \cB$.
In this sense, $\Obs^!$ is the universal factorization algebra which couples to $\Obs$.

\subsection{Finite $N$ factorization algebras}
\label{sec:factsummary}

We summarize the key points of this section which will lead to a general conjecture for the factorization algebra of observables for the worldvolume theory on a finite number of fivebranes in the holomorphic twist.
In the beginning of this section we defined a factorization algebra $\Bar{\pi}_*\Obs_{sugra}$ which we think about as being the factorization algebra of observables of twisted supergravity restricted to the worldvolume of the fivebrane or membrane.
This factorization algebra is of the form 
\beqn
\label{eqn:factgrad}
\Bar{\pi}_*\Obs_{sugra} = \clie^\bu(\cG)
\eeqn
where $\cG = \cG_Z$ is a sheaf of $L_\infty$ algebras on the worldvolume~$Z$ of the fivebrane.

We have seen that $\cG$ is given as the sheaf of sections of a pro vector bundle. 
In fact, the local $L_\infty$ algebra $\cG$ came with a natural decomposition 
\[
\cG = \prod_{k \geq -1} \cG^{(k)} .
\]
There is a related local $L_\infty$ algebra 
\beqn
\til \cG \define \prod_{k \geq 0} \cG^{(k)}
\eeqn
where we simply forget the weight $-1$ component. 
These product decompositions also hold at the level of compactly supported sections. 

This weight grading on $\cG$ induces a filtration on the factorization algebra \eqref{eqn:factgrad} and on its $!$-dual 
\beqn
\left(\Bar{\pi}_*\Obs_{sugra}\right)^! = \clie_\bu(\cG_{c}).
\eeqn
We will focus just on the $!$-dual factorization algebra.
To construct the filtration, first consider the natural filtration on the local Lie algebra $\cG$ induced by the grading
\beqn
\cG = F^0 \cG \supset F^1 \cG \supset \cdots 
\eeqn
where for $N \geq 1$ we have
\beqn
F^N \cG = \cG^{(\geq N-2)} = \prod_{j \geq N-2} \cG^{(j)} ,
\eeqn
and similarly for $\til \cG$. 
If we define 
\beqn
\cG_1 \define \cG / \cG^{(\geq 0)} 
\eeqn
and for $N > 1$
\beqn\label{eqn:gN}
\cG_N \define \cG / \cG^{(\geq N-1)} , \quad \til \cG_N \define \til \cG / \cG^{(\geq N-1)} ,
\eeqn
then the associated graded local Lie algebras $\op{Gr} \cG$ and $\op{Gr} \til{\cG}$ satisfy $\op{Gr} \cG = \op{colim} \cG_N$ and $\op{Gr} \til \cG = \op{colim} \til \cG_N$.
These filtrations also hold at the level of compactly supported sections. 

The filtration on $\cG$ induces a filtration of the factorization algebra $\clie_\bu (\cG_{c})$ 
\beqn
\clie_\bu (\cG_c) = F^0 \clie_\bu (\cG_c) \supset F^1 \clie_\bu (\cG_c) \supset \cdots 
\eeqn
where $F^N \clie_\bu(\cG_c) = \clie_\bu(F^N \cG_c)$. 
Similarly, we have a filtration on the factorization algebra $\clie_\bu(\til \cG_c)$. 
At the level of the associated graded, we have
\beqn
\label{eqn:lim}
\op{Gr} \clie_\bu (\cG_{c}) = \op{colim}_{N \geq 1} \clie_\bu(\cG_{N,c}) .
\eeqn
Similarly $\op{Gr} \clie_\bu (\til \cG_{c}) = \op{colim}_{N \geq 2} \clie_\bu(\til \cG_{N,c})$.
%The subalgebra $\cG^{(\geq k)} = \oplus_{j \geq k} \cG^{(j)}$ is an ideal for every $k$. 
%This sequence of ideals induces a limit diagram of vector bundles
%\beqn\label{eqn:lim}
%0 = \cG / \cG^{(\geq -1)} \leftarrow \cG / \cG^{(\geq 0)} \leftarrow \cG / \cG^{(\geq 1)} \leftarrow \cdots ,
%\eeqn
%where $0$ is the zero vector bundle.
%For each $k$, we point out that $\cG / \cG^{(\geq k)}$ is genuinely a finite rank vector bundle on the worldvolume.
%The resulting filtration of the $!$-dual
%\beqn
%\left(\Bar{\pi}_*\Obs_{sugra}\right)^! = \clie_\bu(\cG_{c})
%\eeqn 
%of the factorization algebra \eqref{eqn:factgrad} is of the form
%\beqn\label{eqn:fil1}
%\clie_\bu (\cG_{1,c}) \subset \clie_\bu (\cG_{2,c}) \subset \cdots \subset \clie_\bu(\cG_c) .
%\eeqn
%Similarly, there is a filtration on the factorization algebra $\clie_\bu (\til \cG_c)$ of the form
%\beqn
%\label{eqn:fil2}
%\clie_\bu(\til \cG_{2,c}) \subset \clie_\bu(\til \cG_{3,c}) \subset \cdots \subset \clie_\bu(\til \cG_c).
%\eeqn

The first term in the limit \eqref{eqn:lim} is 
\[
\clie_\bu(\cG_{1,c})
\]
where $\cG_1$ is the abelian local Lie algebra $\cG_{1} \cong \cG^{(-1)}$---this is just the weight $(-1)$ piece of the decomposition of $\cG$. 
In \cite{SWtensor}, Saberi and the second author have given an explicit description of the holomorphic twist of the worldvolume theory on a single fivebrane, that is, the six-dimensional superconformal theory associated to the abelian Lie algebra~$\lie{gl}(1)$.
We denote the corresponding factorization algebra of classical observables on the three-fold~$Z$ by~$\Obs^{cl}_1$.

We recollected the description of the holomorphic twist of the theory on a single fivebrane in~\S\ref{s:single}. 
This is a free theory and the underlying $\Z \times \Z/2$ graded cochain complex of fields $\cE_{\lie{gl(1)}}$ with linear BRST differential is
\beqn
\begin{tikzcd}
\ul{-1} & \ul{0} \\
\Omega^{2,\bu}(Z) \ar[r,"\del"] & \Omega^{3,\bu}(Z) \\
\Pi \Omega^{0,\bu}(Z, K_{Z}^{1/2} \otimes \C^2) . 
\end{tikzcd} 
\eeqn
Here we recall in the $\Z \times \Z/2$ bigrading the differential has bidegree $(1,0)$. 
The factorization algebra $\Obs_1$ is given by~$\cO(\cE_1)$.

\begin{prop}
\label{prop:factabelian}
There is a quasi-isomorphism of factorization algebras valued in $\Z/2$ graded commutative dg algebras on the three-fold~$Z$
\[
\clie_\bu(\cG_{1,c}) \xto{\simeq} \Obs^{cl}_{1} .
\]
\end{prop}

\begin{proof}
In weight $(-1)$ the abelian local Lie algebra takes the form
\[
\cG_Z^{(-1)} \simeq \Omega^{0,\bu}(Z,\cV^{(-1)}) 
\]
where $\cV^{(-1)}$ was defined in \ref{s:weight-1}.
As a sheaf of $\Z/2$ gradedcochain complexes the factorization algebra $\clie_\bu(\cG_Z^{(-1)})$ assigns to an open set $U\subset Z$ the graded symmetric algebra on the complex
\beqn\label{eqn:weight-1a}
\begin{tikzcd}
\ul{odd} & \ul{even}\\
\Omega_c^{0,\bu}(U, K_Z^{1/2} \otimes \C^2) & \\
\Omega_c^{0,\bu}(U) \ar[r,"\del"] & \Omega_c^{1,\bu}(U) .
\end{tikzcd}
\eeqn
On the other hand, as a $\Z/2$ graded cochain complex, the factorization algebra of observables of the theory on a single fivebrane is of the form 
\[
\cO(\cE_{1}(U)) = \Sym(\Pi \overline{\cE}_{1,c}^!(U)) .
\]
It is immediate to see that as a $\Z/2$ graded cochain complex $\cE_c^!(U)$ is exactly \eqref{eqn:weight-1a}.
The result then follows by applying ellipticity.
\end{proof}

We remark that the Chevalley--Eilenberg complex of an $L_\infty$ algebra $\clie_\bu(\lie{g})$ does not have the structure of a commutative dg algebra.
However, when $\lie{g}$ is abelian (so, a cochain complex) we can identify this complex with the symmetric algebra on the cochain complex $\fg^*[-1]$.
In order to see the quantum observables on a single fivebrane from our holographic analysis we must include effects from the backreaction, which we do not do here.

We now formulate an expectation about the worldvolume theory on a stack of holomorphically twisted fivebranes.
Evidence for this description will be given in the remaining parts of this paper, and we will formulate a more refined conjecture in the next section at the level of local operators.

Recall that the holomorphic, or minimal, twist of a theory with six-dimensional $\cN=(2,0)$ supersymmetry exists on any complex three-fold $Z$ equipped with $K_Z^{1/2}$.
For $N \geq 1$, let $\Obs_{N}$ be the factorization algebra of observables of the holomorphic twist of the worldvolume theory on a stack of $N$ fivebranes wrapping a threefold~$Z$. 
For each open set $U \subset Z$ our expectation is that there is an isomorphism of vector spaces
\beqn
H^\bu \left(\Obs_N(U) \right) \simeq H_\bu (\cG_{N,c}(U)) ,
\eeqn
where $H_\bu$ is the Lie algebra homology.

Similarly, for $N > 1$, let $\til \Obs_N$ be the factorization algebra of classical observables of the holomorphic twist of the worldvolume theory on a stack of $N$ fivebranes with the center of mass degrees of freedom removed.
Then, for each open set $U \subset Z$ we similarly expect
\beqn
H^\bu \left(\til\Obs_N(U) \right) \simeq H_\bu (\til \cG_{N,c}(U)) .
\eeqn
In the language of superconformal $\cN=(2,0)$ theories our conjecture is that the value of the factorization algebra $\clie_\bu(\til \cG_{N,c})$ on an open set $U$ is equivalent to the space of observables of the holomorphic twist of the six-dimensional superconformal field theory associated to the Lie algebra $\lie{sl}(N)$ supported on $U$.

We leave the study of the full factorization algebra structure present in $\Obs_N$ for future work, and emphasize here that we are only making expectations for the space of observables supported an open set.
In the next section we consider local operators, and we will formulate a refined conjecture of the space of local operators as a module for the exceptional super Lie algebra $E(3|6)$. 


%\end{document}
