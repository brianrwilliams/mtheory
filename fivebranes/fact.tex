\documentclass[11pt]{amsart}

%\usepackage{../macros-master}
\usepackage{macros-fivebrane}

\begin{document}

\section{Factorization algebras in twisted $M$ theory}

The observables of any theory in the BV formalism enhance to the structure of a factorization algebra on spacetime \cite{CG2}. 
Classically, if a theory in the BV formalism is described by a local $L_\infty$ algebra $\cL$ on spacetime $M$ then to an open set $U \subset M$ the factorization algebra of classical observables assigns the cochain complex
\[
\clie^\bu(\cL(U))
\]
which computes the Lie algebra cohomology of $\cL(U)$. 
The differential on this cochain complex is precisely the Chevellay--Eilenberg differential associated to the $L_\infty$ structure maps. 

\begin{rmk}
The notion of a factorization algebra captures both the local and non-local operators in a quantum field theory. 
Indeed, from the data of a factorization algebra, one can recover local operators by the following formal construction. 
Let $\Obs$ be a factorization algebra on a smooth manifold $M$.
The classical \defterm{point observables} at $p \in M$ is the limit $\Obs(p) = \lim_{U \ni p} \Obs(U)$ where the limit runs over open sets $U \subset M$ containing $p$.
Generally this limit is difficult to compute, but for certain theories it is possible to give a concise expression which captures the essential features of the theory.
For example, in a holomorphic theory, the algebra of local operators is equivalent to the algebra generated by holomorphic derivatives of fields evaluated at a point \cite{CG2}.
\end{rmk}

At the quantum level this Chevellay--Eilenberg complex is deformed.
The BV formalism ...
Even in perturbation theory the full quantum behavior of the holomorphical-topological eleven-dimensional theory is an open question. 
We leave the problem to characterize the quantization to future work. 
\brian{add discussion}

In this section we will consider the classical factorization algebra of observables of our prototype for the minimal twist of eleven-dimensional supergravity given in \S \ref{s:twisted} which is defined on $S \times X$ where $X$ is a Calabi--Yau fivefold and $S$ a real oriented one-manifold. 
%In the \S \ref{s:twisted} we recalled a proposal for the description of the minimal twist of eleven-dimensional supergravity on $S \times X$. 
The primary object we used to describe the solutions of the equations of motion was the $\Z/2$ graded local $L_\infty$ algebra $\cL_{sugra}$, see \S \ref{s:Lsugra}. 
%As a graded vector space, observables on an open set $U \subset X \times S$ are thus $\cO(\cL_{sugra}(U)) = \Sym (\Pi \cL_{sugra})^\vee$.
%The differential is encoded by the $L_\infty$ structure which is exactly the Chevalley--Eilenberg differential. 

In terms of this local $L_\infty$ algebra, the $\Z/2$ graded commutative dg algebra of classical observables supported on an open set $U \subset S \times X$ is 
\[
\clie^\bu\left(\cL_{sugra}(U)\right) .
\]
We will denote the entire classical factorization algebra on $S \times X$ of classical observables by~$\Obs_{sugra}$.



%\parsec[s:membraneobs]
%
%
%\parsec[s:fivebraneobs]
%
%Let $Z$ be a complex threefold. 
%In \ref{s:single} we have described the holomorphic twist of a theory on a single fivebrane within the (degenerate) BV formalism in terms of the (abelian) local $L_\infty$ algebra $\cL_{single}$ on $Z$. 
%The factorization algebra of classical observables of the holomorphic twist of the single fivebrane theory assigns to an open set $U \subset Z$
%the cochain complex
%\[
%\clie^\bu\left(\cL_{single}(U)\right) .
%\]
%We will denote the entire factorization algebra on $Z$ of classical observables by~$\clie^\bu(\cL_{single})$.




\subsection{The fivebrane decomposition}\label{s:resm5}

With an eye towards holography for fivebranes we fix a complex three-fold $Z$ where the fivebranes will be supported and consider the eleven-dimensional theory on $\R \times X_Z$ where $X_Z$ is the Calabi--Yau fivefold
\[
X_Z = \text{Tot}(K_Z^{1/2} \otimes \C^2),
\]
the total space of the rank two holomorphic vector bundle $K_Z^{1/2} \otimes \C^2$ over $Z$. 
%In the remainder of this section we denote by $V \to Z$ this rank two holomorphic vector bundle. 
One can also think about $\R \times X_Z$ as the total space of the {\em real} rank five bundle over $Z$; although this description obfuscates the geometric structure used to define the eleven-dimensional theory.
Let 
\[
\pi \colon \R \times X_Z \to Z 
\]
be the map which projects out the real factor followed by the projection to the base of the total space of the complex rank two bundle $p \colon X_Z \to Z$.

Like sheaves, factorization algebras can be pushed forward; thus from the factorization algebra $\Obs_{sugra}$ on $\R \times X_Z$ we obtain a factorization algebra $\pi_* \Obs_{sugra}$ on $Z$. 
To an open set $U \subset Z$ the value of this factorization algebra is
\begin{align*}
(\pi_* \Obs_{sugra}) (U) & = \Obs_{sugra} \left(\pi^{-1} U\right) \\
& = \Obs_{sugra}(\R \times p^{-1}U) .
\end{align*}

This factorization algebra is not the factorization algebra associated to an ordinary sort of field theory on $Z$.
Nevertheless there is a subfactorization algebra which admits a natural grading so that each filtered component can be understood as such.
This grading is determined by a $\C^\times$ action on the fields of the eleven-dimensional theory on $\R \times X_Z$ which we now describe. 

In a local chart $U \subset Z$ we can write a generic field of the eleven-dimensional theory on $\pi^{-1} U$ in coordinates as $f(t;w,z)$ where $t \in \R$ is the real coordinate, $w=(w_1,w_2)$ is the holomorphic fiber coordinate of the rank two bundle $X_Z$ over $Z$, and $z = (z_1,z_2,z_3)$ is a local holomorphic coordinate for $U \subset Z$. 

%For simplicity let us consider the eleven-dimensional theory on flat space $\R \times \C^5$ with some number of fivebranes supported on
%\[
%0 \times 0 \times \C^3 \subset \C^5 .
%\]
%is that (after taking into account the backreaction) the observables on a large number of fivebranes is Koszul dual to the factorization algebra 
%
%\parsec[s:flatdecomp]
%
%%In the simple case where $Z = \C^3$ and we identify the total space of $K^{1/2}_Z \otimes \C^2$ with $\C^5$ then the manifold obtained by removing the locus of the brane is homeomorphic to
%%\[
%%\C^3 \times (\R \times \C^2 - 0) .
%%\]
%%
%%Let $\pi \colon \R \times \C^5 - (0 \times \C^3) \to \C^3 \times \R_{>0}$ be the projection map whose fibers are homeomorphic to the sphere $S^4$ which links the location of the fivebranes.
%%We restrict the factorization algebra of the eleven dimensional theory $\Obs_{sugra}$ to the open set obtained by removing the locus of the brane.
%
%In the simple case that $Z = \C^3$ and we identify the total space of $K^{1/2}_Z \otimes \C^2$ with $\C^5$ there is a more direct construction of the factorization algebra $\Obs_{sugra}|_{Z}$. 
%
%Let $\pi \colon \R \times \C^5 \to \C^3$ be the projection map.
%Then, via $\pi$ we can pushforward the factorization algebra associated to the eleven-dimensional theory to obtain a factorization algebra
%\[
%\pi_* \Obs_{sugra} 
%\]
%on $\C^3$.
%This factorization algebra is not the factorization algebra associated to an ordinary sort of field theory on $\C^3$. 
%Nevertheless there is a subfactorization algebra which admits a natural grading so that each filtered component can be understood as such.

With this notation in place, we introduce the following $\C^\times$ action on the fields of the eleven-dimensional theory:
\begin{itemize}
\item On the fields $\mu(t;w,z) \in \Omega^\bu(\R) \otimes \PV^{1,\bu}(\C^2 \times U)$ the action is
\[
\lambda \cdot \mu(t;w,z) = \mu(\lambda t;\lambda w , z).
\]
\item On the fields $\nu(t;w,z) \in \Omega^\bu(\R) \otimes \PV^{0,\bu}(\C^2 \times U)$ the action is
\[
\lambda \cdot \nu(t;w,z) = \nu(\lambda t;\lambda w , z).
\]
\item On the fields $\beta(t;w,z) \in \Omega^\bu(\R) \otimes \Omega^{0,\bu}(\C^2 \times U)$ the action is
\[
\lambda \cdot \beta(t;w,z) = \lambda^{-1} \beta(\lambda t;\lambda w , z).
\]
\item On the fields $\gamma(t;w,z) \in \Omega^\bu(\R) \otimes \Omega^{1,\bu}(\C^2 \times U)$ the action is
\[
\lambda \cdot \gamma(t;w,z) = \lambda^{-1} \gamma(\lambda t;\lambda w , z).
\]
\end{itemize}

The main result of this section is the following. 

\begin{prop}
The $L_\infty$ structure on $\cL_{sugra}$ equivariant for this $\C^\times$ action. 
The odd symplectic form on $\cL_{sugra}$ is weight $-1$ for this $\C^\times$ action. 
\end{prop}
\begin{proof}
The only nontrivial bracket to check compatibility with the $\C^\times$ action is the one which takes two $\gamma$-type sections to a $\mu$-type section of the form
\[
[\gamma_1, \gamma_2] = \Omega^{-1} \vee (\del \gamma_1 \wedge \del \gamma_2) 
\]
where $\Omega \vee (-)$ is the operation which takes a four-form 

It suffices to check equivariance in local coordinates; there are three cases.
Suppose $\gamma_1(t;w,z) = f^i(t;w,z) \d z_i$ and $\gamma_2(t;w,z) = g^j(t;w,z) \d z_j$where $f^i,g^j$ are de Rham--Dolbeault forms for $i,j=1,2,3$. 
Then $\lambda \cdot \gamma_1 = \lambda^{-1} f^i(\lambda t;\lambda w, z) \d z_i$ and $\lambda \cdot \gamma_2 = \lambda^{-1} g^j(\lambda t;\lambda w, z) \d z_j$. 
Thus, if the $\mu$-type field is expanded as $[\gamma_1,\gamma_2] = F^k(t;w,z) \del_{z_k} + G^a (t;w,z) \del_{w_a}$ for some de Rham--Dolbeault forms $F^k,G^a$, $k=1,2,3,a=1,2$, it suffices to show that 
\beqn
\begin{array}{llll}
\label{eqn:lambda1} \lambda \cdot \left(F^k(t;w,z) \del_{z_k} \right) & = \lambda^{-2} F^k(\lambda t;\lambda w,z) \del_{z_k} \\
\lambda \cdot \left(G^a(t;w,z) \del_{w_a} \right) & = \lambda^{-2} G^a(\lambda t;\lambda w,z) \del_{w_a} .
\end{array}
\eeqn
We expand $\del \gamma_1 \wedge \del \gamma_2$ to three terms
\begin{multline}
\del_{w_a} f^i(t;w,z) \del_{z_l} g^j(t;w,z) \d w_a \d z_i \d z_l \d z_j + \del_{z_k} f^i(t;w,z) \del_{w_b} g^j(t;w,z) \d z_k \d z_i \d w_b \d z_j \\  + \del_{w_a} f^i(t;w,z) \del_{w_b} g^j(t;w,z) \d w_a \d z_i \d w_b \d z_j .
\end{multline}
Thus in the notation above we have 
\begin{align*}
F^k (t;w,z) & = \ep_{ab} \ep_{ijk} \del_{w_a} f^i(t;w,z) \del_{w_b} g^j(t;w,z) \\
G^a(t;w,z) & = \ep_{ilj} \ep_{ab} \del_{w_a} f^i(t;w,z) \del_{z_l} g^j(t;w,z)
+ \ep_{kij} \ep_{ab} \del_{z_k} f^i(t;w,z) \del_{w_a} g^j(t;w,z)  .
\end{align*}
We compute the action of $\lambda \in \C^\times$ on $\mu = F^k \del_{z_k}$
\[
\lambda \cdot \left(F^k \del_{z_k} \right) = \ep_{ab} \ep_{ijk} \lambda^{-1} \del_{w_a} f^i(\lambda t;\lambda w,z) \lambda^{-1} \del_{w_b} g^j(\lambda t;\lambda w,z) \del_{z_k} .
\]
Thus the first part of \eqref{eqn:lambda1} is satisfied. 
Similarly we observe that $\lambda \cdot \left(G^a(t;w,z) \del_{w_a} \right)  = \lambda^{-2} G^a(\lambda t;\lambda w,z) \del_{w_a}$. 

The next case is when $\gamma_1(t;w,z) = f^i(t;w,z) \d z_i$ and $\gamma_2(t;w,z) = g^a(t;w,z) \d w_a$where $f^i,g^a$ are de Rham--Dolbeault forms for $i=1,2,3$ and $a=1,2$. 
Then $\lambda \cdot \gamma_1 = \lambda^{-1} f^i(\lambda t;\lambda w, z) \d z_i$ and $\lambda \cdot \gamma_2 = g^a(\lambda t;\lambda w, z) \d w_a$. 
Thus, if the $\mu$-type field is expanded as $[\gamma_1,\gamma_2] = F^k(t;w,z) \del_{z_k} + G^a (t;w,z) \del_{w_a}$ for some de Rham--Dolbeault forms $F^k,G^a$, $k=1,2,3,a=1,2$, it suffices to show that 
\beqn
\begin{array}{llll}
\label{eqn:lambda2} \lambda \cdot \left(F^k(t;w,z) \del_{z_k} \right) & = \lambda^{-1} F^k(\lambda t;\lambda w,z) \del_{z_k} \\
\lambda \cdot \left(G^a(t;w,z) \del_{w_a} \right) & = \lambda^{-1} G^a(\lambda t;\lambda w,z) \del_{w_a} .
\end{array}
\eeqn
We expand $\del \gamma_1 \wedge \del \gamma_2$ to three terms
\begin{multline}
\del_{w_b} f^i(t;w,z) \del_{z_j} g^a(t;w,z) \d w_b \d z_i \d z_j \d w_a + \del_{z_k} f^i(t;w,z) \del_{w_b} g^a(t;w,z) \d z_k \d z_i \d w_b \d w_a \\  + \del_{z_j} f^i(t;w,z) \del_{z_k} g^a(t;w,z) \d z_j \d z_i \d z_k \d w_a .
\end{multline}
Thus in the notation above we have 
\begin{align*}
F^k (t;w,z) & = \ep_{ab} \ep_{ijk} \del_{w_a} f^i(t;w,z) \del_{z_l} g^b(t;w,z) + \ep_{ab} \ep_{kij} \del_{z_k} f^i(t;w,z) \del_{w_a} g^b(t;w,z) \\
G^a(t;w,z) & = \ep_{ab} \ep_{jik} \del_{z_j} f^i(t;w,z) \del_{z_k} g^b(t;w,z) .
\end{align*}
We compute the action of $\lambda \in \C^\times$ on $\mu = G^a \del_{z_a}$
\[
\lambda \cdot \left(G^a \del_{w_a} \right) = \ep_{ab} \ep_{jik} \del_{z_j} f^i(t;w,z) \del_{z_k} g^b(t;w,z) \lambda^{-1} \del_{w_a} .
\]
Thus the second part of \eqref{eqn:lambda2} is satisfied. 
Similarly we observe that $\lambda \cdot \left(F^k(t;w,z) \del_{z_k} \right)  = \lambda^{-2} F^k(\lambda t;\lambda w,z) \del_{z_k}$. 

The last case is when $\gamma_1(t;w,z) = f^a(t;w,z) \d w_a$ and $\gamma_2(t;w,z) = g^b(t;w,z) \d w_b$where $f^a,g^b$ are de Rham--Dolbeault forms for and $a,b=1,2$.
The argument is nearly identical so we omit the details. 
\end{proof}

For each $n \in \Z$ and open set $U \subset Z$, let 
\[
\pi_* \cL_{sugra}(U)^{(n)} \subset \cL_{sugra}(\pi^{-1} U)
\]
be the weight $n$ eigenspace with respect to this $\C^\times$ action. 
%The $\C^\times$ action is compatible with the $\Z/2$ graded $L_\infty$ structure on $\cL_{sugra}$. 
Since the $n$th eigenspace is trivial when $n < -1$, we see that the product
\[
(\Bar{\pi}_* \cL_{sugra})(U) \define \prod_{n \geq -1} (\pi_*\cL_{sugra})(U)^{(n)}
\]
is equipped with the structure of a $\Z/2$ graded $L_\infty$ algebra.
In this way, the assignment 
\[
\Bar{\pi}_* \cL_{sugra} \colon U \mapsto (\Bar{\pi}_* \cL_{sugra})(U) 
\]
defines a sheaf of $\Z/2$ graded $L_\infty$ algebras on $Z$. 

\begin{lem}
\label{lem:technical}
For each $n$, the sheaf of cochain complexes $U \mapsto \Bar{\pi}_* \cL_{sugra}(U)^{(n)}$ is quasi-isomorphic to one of the form
\[
\Omega^{0,\bu}(U, \cV_{fivebrane}^{(n)})
\]
for some finite rank super holomorphic vector bundle $\cV_{fivebrane}^{(n)}$ on $Z$.
For each $n$, the differential is of the form $\dbar + Q^{hol}$.
In particular, this endows $\Bar{\pi}_* \cL_{sugra}$ with the structure of a pro-vector bundle on $Z$. 
\end{lem}

\begin{proof}
In \S \ref{s:Lsugra} we introduced a holomorphic vector bundle $L_X$ defined on any Calabi--Yau five-fold $X$ whose sheaf of holomorphic sections was equipped with a $\Z/2$ graded $L_\infty$ structure. 
Here, we consider the Calabi--Yau fivefold $X = X_Z$ as defined above.
For any open subset $U \subset Z$ of the threefold the value of $\cL_{sugra}(\pi^{-1}U)$ as a sheaf of super vector spaces is 
\[
\Omega^\bu(\R) \otimes \Omega^{0,\bu}(p^{-1} U, L_{X_Z}),
\]
where $p \colon X_Z \to Z$ is the rank two bundle over $Z$. 
The one-ary structure map, or differential, is of the form $\d_{dR} + \dbar + Q^{hol}$ where $Q^{hol}$ is some odd holomorphic differential operator acting on sections of $L_{X_Z}$. 

Since the de Rham complex of $\R$ is contractible we have a quasi-isomorphism of $L_\infty$ algebras
\[
\cL_{sugra}(\pi^{-1}U) \simeq \Omega^{0,\bu}(p^{-1} U, L_{X_Z})
\]
for every $U \subset Z$. 
Without loss of generality, we can assume that $U$ is a coordinate chart for the vector bundle $X_Z$ so that $X_Z|_U \cong U \times \C^2$ and where the bundle $L_{X_Z}$ splits as 
\[
L_{X_Z}|_{\pi^{-1} U} \cong L' \boxtimes L'' 
\]
where $L'$ is a holomorphic super vector bundle on $U$ and $L''$ is.a holomorphic vector on the fiber $\C^2$.
Then, we can further identify this with a Dolbeault complex of the form
\[
\Omega^{0,\bu} \left(U, L' \otimes \Omega^{0,\bu}(\C^2, L'') \right) . 
\]
By the Dolbeault Poincar\'e lemma applied to the holomorphic vector bundle $L''$ over $\C^2$,
\[
\Omega^{0,\bu} \left(U, L' \otimes M\right) 
\]
where we now interpret $L' \otimes M$ as an infinite rank bundle over $U \subset Z$---
the remaining differential is $\dbar_U + Q^{hol}$ where $Q^{hol}$ is a holomorphic differential operator.

With these simplifications, it suffices to show that the weight $n$ eigenspace of $L' \otimes M$ is finite rank over $U$ \brian{almost finished}
%we can fix a quasi-isomorphism $M \simeq \Omega^{0,\bu}(\C^2, L'')$ where $M$ is a 
\end{proof}


We explicitly describe the first few homogenous eigenspaces of $\Bar{\pi}_* \cL_{sugra}$. 

\parsec[s:weight-1]

As an example, let us consider the weight $(-1)$ piece of the decomposition $\Omega^{0,\bu}(Z, \cV_{fivebrane}^{(-1)})$. 
We will show that 
\[
\cV_{fivebrane}^{(-1)} = K^{1/2}_Z \otimes \C^2 \oplus \cO_Z \oplus \Pi \T^*_Z 
\]
whose Dolbeault forms are equipped with the $\dbar$ operator and the $\del$ operator which takes functions to one-forms. 

Explicitly weight zero summand $\cV_{fivebrane}^{(0)}$ consists of:
\begin{itemize}
\item 
Vector fields which are locally of the form $f^a(z) \del_{w_a}$ where $f^{a}(z)$ is a holomorphic function on $\C^3$.
Notice that these vector fields are automatically divergence-free.
Since $\del_{w_a}$ is treated as a section of $K^{1/2}_Z$ we see that these are local sections of $K^{1/2}_Z \otimes \C^2$. 
\item 
Next there are holomorphic functions (type $\beta$) and holomorphic $one$-forms (type $\gamma$) which do not depend on $w_a$. 
Since $\del$ is weight zero for the decomposition these forms combine to form the $\Z/2$ graded complex of sheaves
\[
\cO_Z \xto{\del} \Pi \Omega^{1,hol}_Z .
\]
\end{itemize}

\parsec[s:weight0]

Particularly important for us will be the weight zero part $\Omega^{0,\bu}(Z, \cV_{fivebrane}^{(0)})$ of the $\Z/2$ graded $L_\infty$ algebra $\Bar{\pi}_* \cL_{sugra}$ which inherits the structure of a $\Z/2$ graded $L_\infty$ algebra.
When $Z = \C^3$ we will see that the global sections of this $\Z/2$ graded $L_\infty$ algebra is closely related to the exceptional Lie algebra $E(3|6)$ studied by Kac \cite{KacClass}.
%More precisely he weight zero piece is a local Lie algebra enhancement of this exceptional Lie algebra.

As a holomorphic vector bundle, we will see that $\cV_{fivebrane}^{(0)}$ is 
\[
\cV_{fivebrane}^{(0)} = \T_Z \oplus \lie{sl}(2) \otimes \cO_Z \oplus \Pi \left( \T^*_Z \otimes K^{-1/2}_Z \otimes \C^2 \right).
\]
The $L_\infty$ structure of $\cL_{sugra}$ endows the sheaf of sections of this holomorphic vector bundle with a Lie bracket which we will describe.

Explicitly, we enumerate all weight zero holomorphic sections:
\begin{itemize}
\item Vector fields which are locally of the form 
\[
\mu = f^i(z) \del_{z_i} - \frac12 \del_{z_i} f^i(z) w_a \del_{w_a}
\]
and vector fields which are locally of the form
\[
\mu = g(z) A^{ab} w_a \del_{w_b}.
\]
The condition that $\mu$ be divergence-free implies that $(A^{ab}) \in \lie{sl}(2)$.
We will identify the first such vector fields with sections of $\T_Z$ and the second such vector fields with sections of $\lie{sl}(2) \otimes \cO_Z$. 
\item Holomorphic functions of the form $\beta = w_a f(z)$ and holomorphic one-forms of the form $\gamma = g^{a}(z) \d w_a + h^{a} (z) w_a \d z$. 
Since $\del$ is weight zero, such sections are equipped with a differential 
\[
\Gamma^{hol}(Z, K^{-1/2}) \xto{D} \Pi \Gamma^{hol}(Z, K^{-1/2}) \oplus \Pi\Omega^{1,hol}(Z, K^{-1/2}) 
\]
which sends $w_a f(z) \mapsto (f(z) , w_a \del f (z))$. 
This $\Z/2$ graded complex of sheaves is clearly equivalent to $\Pi\Omega^{1,hol}(Z, K^{-1/2})$. 
\end{itemize}

The $L_\infty$ structure on $\Bar{\pi}_* \cL_{sugra}$ induces the structure of a local $\Z/2$ graded dg Lie algebra on $\Omega^{0,\bu}(Z, \cV_{fivebrane}^{(0)})$. 
The differential is simply $\dbar$. 
The Lie bracket is induced from a Lie algebra structure on holomorphic sections of $\cV_{fivebrane}^{(0)}$ which we can describe as follows:
\begin{itemize}
\item For $\Vect^{hol}(Z)$ there is the standard commutator of holomorphic vector fields. 
This acts on the sections of $\lie{sl}(2) \otimes \cO_Z$ and $\Pi\Omega^{1,hol}(Z, K^{-1/2}_Z \otimes \C^2)$ by Lie derivative. 
\item On sections of $\lie{sl}(2) \otimes \cO_Z$ there is the matrix commutator. 
This acts on the odd part $\Pi\Omega^{1,hol}(Z, K^{-1/2}_Z \otimes \C^2)$ where we view $\C^2$ as the fundamental $\lie{sl}(2)$ representation. 
\item Finally, and most interestingly, there is a bracket of the form
\[
\Pi\Omega^{1,hol}(Z, K^{-1/2}_Z \otimes \C^2) \times \Pi\Omega^{1,hol}(Z, K^{-1/2}_Z \otimes \C^2) \to \Vect^{hol}(Z) \oplus \lie{sl}(2) \otimes \cO_Z 
\]
given in coordinates by
\[
[f^a(z) \d w_a , g^a \d w_b] = \left(??, ??\right) 
\]
\brian{finish}
\end{itemize}


\parsec[s:mainfact]

Since $\Bar{\pi}_* \cL_{sugra}$ is a pro-vector bundle with a compatible $L_\infty$ structure, the assignment 
\[
U\subset Z \mapsto \clie^\bu\left(\Bar{\pi}_* \cL_{sugra}(U)\right) 
\]
has the structure of a factorization algebra on the three-fold $Z$. 
We denote this factorization algebra by $\Bar{\pi}_* \Obs_{sugra}$. 


%\begin{prop}
%Let $Z$ be a Calabi--Yau three-fold and $\pi \colon Z \times \C^2 \times \R \to Z$ be the projection. 
%There is a pro local Lie algebra $\Bar{\pi}_*\cL_{sugra}$ and a factorization algebra $\Bar{\pi}_* \Obs_{sugra} = \clie^\bu(\Bar{\pi}_* \cL_{sugra})$ on $Z$ such that:
%\begin{itemize}
%\item[(1)] there is a natural inclusion of of factorization algebras on $Z$
%\[
%\Bar{\pi}_* \Obs_{sugra} \hookrightarrow \pi_* \Obs_{sugra}
%\]
%which is dense at the level of cohomology. 
%\item[(2)] there is a weight grading on $\cL_{\pi,sugra}$ which is concentrated in degrees $\geq -1$ and gives rise to a decomposition of vector bundles
%\beqn\label{eqn:decomp3}
%\Bar{\pi}_{*} \cL_{sugra} = \prod_{n \geq -1} \cV_{n} 
%\eeqn
%\item[(3)]
%In weight zero, there is an equivalence of local Lie algebras on $Z$ 
%\[
%\cV_0 \simeq \cE(3|6)|Z 
%\]
%where $\cE(3|6)|Z$ is a local Lie algebra enhancement of the exceptional simple super Lie algebra $E(3|6)$. 
%\end{itemize}
%\end{prop}

%We want to argue that $\Obs_{sugra} |_{\C^3} \cong \Bar{\pi}_* \Obs_{sugra}$. 

\subsection{The membrane decomposition}

We briefly go through an analogous decomposition for membranes. 
For this, we fix a Riemann surface $C$ and consider the eleven-dimensional theory on $\R \times X_C$ where $X_C$ is the Calabi--Yau fivefold
\[
X_C = \text{Tot}(K_C^{1/4} \otimes \C^4),
\]
the total space of the rank four bundle $K_C^{1/4} \otimes \C^2$ over $C$.
Denote by 
\[
\pi \colon V \to \R \times C
\] 
the pullback of this vector bundle along the projection $\R \times C \to C$.
We can equivalently think of the eleven-manifold as the total space of the bundle $V$ over $\R \times C$. 
In this description, the twisted membranes will be supported along the zero section of $V$. 
The factorization algebra $\Obs_{sugra}$ on ${\rm Tot}(V)$ can be pushed forward along $\pi$ to a factorization algebra $\pi_* \Obs_{sugra}$ on $\R \times C$. 

As in the fivebrane case, this is not the factorization algebra associated to an ordinary field theory on $\R \times C$.
Nevertheless there is a subfactorization algebra which admits a natural grading so that each filtered component can be understood as such.
This grading is determined by a (different) $\C^\times$ action on the fields of the eleven-dimensional theory on ${\rm Tot}(V) \cong \R \times X_C$ which we now describe. 

In a local THF chart $U \subset \R \times C$ we can write a generic field of the eleven-dimensional theory on $\pi^{-1} U$ in coordinates as $f(t;z,w)$ where $(t;z) \in U \subset\R \times C$ is THF coordinate and $w=(w_1,w_2,w_3,w_4)$ is the holomorphic fiber coordinate of the rank four bundle $V$.

%For simplicity let us consider the eleven-dimensional theory on flat space $\R \times \C^5$ with some number of fivebranes supported on
%\[
%0 \times 0 \times \C^3 \subset \C^5 .
%\]
%is that (after taking into account the backreaction) the observables on a large number of fivebranes is Koszul dual to the factorization algebra 
%
%\parsec[s:flatdecomp]
%
%%In the simple case where $Z = \C^3$ and we identify the total space of $K^{1/2}_Z \otimes \C^2$ with $\C^5$ then the manifold obtained by removing the locus of the brane is homeomorphic to
%%\[
%%\C^3 \times (\R \times \C^2 - 0) .
%%\]
%%
%%Let $\pi \colon \R \times \C^5 - (0 \times \C^3) \to \C^3 \times \R_{>0}$ be the projection map whose fibers are homeomorphic to the sphere $S^4$ which links the location of the fivebranes.
%%We restrict the factorization algebra of the eleven dimensional theory $\Obs_{sugra}$ to the open set obtained by removing the locus of the brane.
%
%In the simple case that $Z = \C^3$ and we identify the total space of $K^{1/2}_Z \otimes \C^2$ with $\C^5$ there is a more direct construction of the factorization algebra $\Obs_{sugra}|_{Z}$. 
%
%Let $\pi \colon \R \times \C^5 \to \C^3$ be the projection map.
%Then, via $\pi$ we can pushforward the factorization algebra associated to the eleven-dimensional theory to obtain a factorization algebra
%\[
%\pi_* \Obs_{sugra} 
%\]
%on $\C^3$.
%This factorization algebra is not the factorization algebra associated to an ordinary sort of field theory on $\C^3$. 
%Nevertheless there is a subfactorization algebra which admits a natural grading so that each filtered component can be understood as such.

With this notation in place, we introduce the following $\C^\times$ action on the fields of the eleven-dimensional theory:
\begin{itemize}
\item On the fields $\mu(t;w,z)$ the action is
\[
\lambda \cdot \mu(t;z,w) = \mu(t; z , \lambda w).
\]
\item On the fields $\nu(t;w,z)$ the action is
\[
\lambda \cdot \nu(t;z,w) = \nu(t;z ,\lambda w).
\]
\item On the fields $\beta(t;w,z)$ the action is
\[
\lambda \cdot \beta(t;z,w) = \lambda^{-1} \beta(t; z, \lambda w).
\]
\item On the fields $\gamma(t;w,z)$ the action is
\[
\lambda \cdot \gamma(t;z,w) = \lambda^{-1} \gamma(t; z ,\lambda w).
\]
\end{itemize}
 

\begin{prop}
The $L_\infty$ structure on $\pi_* \cL_{sugra}$ is equivariant for this $\C^\times$ action. 
%The odd symplectic form on $\cL_{sugra}$ is weight $-1$ for this $\C^\times$ action. 
\end{prop}

As in the case of fivebranes, this proposition allows us to decompose the local Lie algebra $\pi_* \cL_{sugra}$ into eigenspaces. 
Similarly to the proof of Lemma \ref{lem:technical} one can show that the $n$th eigenspace is quasi-isomorphic to a local Lie algebra of the form
\[
\Omega^\bu(\R) \otimes \Omega^{0,\bu}\left(C, \cV_{membrane}^{(n)}\right)
\]
where $\cV^{(n)}_{membrane}$ is a holomorphic super vector bundle on $C$.

We will state explicit descriptions of the first few eigenspaces without details of the proof. 
The first nontrivial eigenspace is, as in the fivebrane case, $n=-1$. 
Here, one can see that 
\[
\cV^{(-1)}_{membrane} \simeq K_{C}^{1/4} \otimes \C^4 \oplus \Pi K^{3/4}_C \otimes \C^4 .
\]
For $n=0$ one has 
\[
\cV^{(0)}_{membrane} \simeq \T_C \oplus \lie{sl}(4) \otimes \cO_C \oplus \Pi \left(K^{-1/2}_C \otimes \wedge^2 \C^4 \oplus K^{-1/2}_C \otimes S^2 \C^4\right) .
\]
\subsection{Koszul duality for factorization algebras: an ansatz} 

In quantum field theory Koszul duality naturally appears in the problem of coupling topological line operators to some ambient bulk theory. 
More generally, for higher dimensional topological defects, this problem is encoded by Koszul duality for the theory of $\EE_n$ algebras \cite{??}.

More generally, we anticipate a general theory of Koszul duality for factorization algebras which should encode the problem of coupling arbitrary defects (without the condition of being topological).
Even for factorization algebras of holomorphic-topological nature this theory has not been studied in mathematics. 
Nevertheless, we will emphasize features that we expect this general form of Koszul duality to possess which will allow us to nail down its behavior on a rather general class of factorization algebras. 

In this first part of this subsection we briefly recall how Koszul duality enters in the problem of coupling line operators. 
We refer~\cite[\S 6]{CP1},~\cite[\S ??]{CG1}, or the review~\cite{PWkoszul} for more details. 
Then, we give an ansatz for Koszul duality for factorization algebras of the form $\clie^\bu(\cL)$ where $\cL$ is some local Lie algebra. 
From the point of view of the BV formalism this is not much of a condition, all such factorization algebras of classical observables can be cast in this form. 

\parsec[s:lines]
Suppose that we have a bulk theory living on a spacetime of the form 
\[
\R \times M 
\]
where $M$ is some smooth manifold. 
Denote by $\cA$ the corresponding factorization algebra on $\R \times M$. 
The theory could have arbitrary behavior along $M$, but we assume that the theory is topological along $\R$. 
This means that when viewed as a factorization algebra on $\R$ that $\cA$ is locally constant and is hence equivalent to the data of an $\EE_1$ or $A_\infty$ algebra.

Next, assume that $\cB$ is another $\EE_1$ algebra, which we think of as being associated to some quantum mechanical system along the real line.
This is a local model for the desired line operator that we are attempting to couple to the bulk theory.
Koszul duality enters in the problem of coupling the two quantum mechanical systems $\cA$ and $\cB$---where we view $\cA$ simply as an $\EE_1$ algebra. 

A coupling of the two systems is Maurer--Cartan element in the algebra
\[
\alpha \in \cA \otimes \cB .
\]
That is, $\alpha$ is an element of ghost degree one which satisfies the Maurer--Cartan equation
\[
\delta \alpha + \alpha \star \alpha = 0 .
\]
Given such an $\alpha$ we can deform the algebra $\cA \otimes \cB$ by adding the term $[\alpha,-]$ to the differential. 
In other words, at the cochain level only the differential, not the product structure, is modified. 

In principle, there are more general ways to `couple' two $\EE_1$ algebras; generally this is controlled by the Hochschild cohomology which governs algebra deformations of $\cA \otimes \cB$. 
we will elaborate further on this definition. 
In \cite{CG1} (see also \cite{PWkoszul}) it is shown how this notion relates to the physicists description of coupling in terms of local Lagrangians.

To see Koszul duality, the key observation is that the data of the Maurer--Cartan element $\alpha$ is equivalent to the data of a map of $\EE_1$ algebras
\[
\alpha \colon \cA^! \to \cB 
\]
where $\cA^!$ is Koszul dual to the algebra $\cA$. 
We then have the following slogan: the Koszul dual of the algebra of observables of the bulk theory $\cA^!$ is the algebra of operators on the {\em universal} line defect supported on $\RR \times \{x\}$, where $x \in M$.  

\parsec[s:celine]

Before moving towards our definition of Koszul duality for a general class of factorization algebras, we briefly recast the case of duality for $\EE_1$ algebras in terms of factorization algebras. 

We will focus on a slight generalization of the standard Koszul duality between the exterior and symmetric algebras.

\begin{prop}
Let $\lie{g}$ be a Lie algebra and equip the associative dg algebra $\clie^\bu(\fg)$ with the augmentation induced by the tautological Lie map $0 \to \lie{g}$.
The Koszul dual of $\clie^\bu(\lie{g})$ with respect to this augmentation is equivalent to the universal enveloping algebra $U \fg$. 
\end{prop}

There are explicit models for the associative dg algebras $\clie^\bu(\lie{g})$ and $U \lie{g}$ as locally constant factorization algebras on $\R$.
First, observe that we can tensor $\fg$ with the commutative dg algebra of de Rham forms to obtain a dg Lie algebra $\fg \otimes \Omega^\bu(\R)$. 
This has a natural enhancement to a local dg Lie algebra as this is simply the smooth sections of the bundle of Lie algebras $\fg \otimes \wedge^\bu \T^*_\RR$ equipped with the de Rham operator.

Using this local Lie algebra, we obtain a model for the associative (and commutative) dg algebra $\clie^\bu(\fg)$ as the factorization algebra
\[
\clie^\bu(\fg \otimes \Omega^\bu_\R).
\]
To an open set $U \subset \R$ this produces the Chevellay--Eilenberg complex computing the Lie algebra {\em cohomology} of the dg Lie algebra $\fg \otimes \Omega^\bu(U)$---the $\fg$-valued de Rham forms on $U$.
 
Similarly, a model for $U \fg$ is the locally constant factorization algebra
\[
\clie_\bu(\fg \otimes \Omega^\bu_{\R,c}).
\]
To an open set $U \subset \R$ this produces the Chevellay--Eilenberg complex computing the Lie algebra {\em homology} of the dg Lie algebra $\fg \otimes \Omega^\bu_c(U)$---the $\fg$-valued compactly supported de Rham forms on $U$.

\parsec[s:generalkoszul]

In analogy with the case of $\EE_1$, or locally constant factorization, algebras above we make the following definition. 

\begin{dfn}
Let $\cL$ be a local $L_\infty$ algebra on a manifold $M$ and consider the factorization algebra $\clie^\bu(\cL)$ which assigns to an open set $U \subset M$ the cochain complex $\clie^\bu(\cL(U))$. 
The \defterm{$!$-dual factorization algebra} is 
\[
\clie^\bu(\cL)^! \define \clie_\bu (\cL_{c}) 
\]
where $\clie_\bu(\cL_c)$ assigns to an open set $U \subset M$ the cochain complex $\clie_\bu(\cL_c(U))$.
In other words, the $!$-dual factorization algebra of $\clie^\bu(\cL)$ is the (untwisted) factorization enveloping algebra of the local Lie algebra $\cL$. 
\end{dfn} 

There are many things lacking in this definition. 
First off we do not define the $!$-dual for an arbitrary factorization algebra, only for ones of the form $\clie^\bu(\cL)$ where $\cL$ is a local Lie algebra. 
Also, we will not prove that $!$-dual satisfies any Koszul duality axioms. 
From the discussion above we see that $!$-dual does agree with Koszul duality in the case of associative algebras.\footnote{It is not difficult to see that $!$-duality for locally constant factorization algebras on $\R^n$ agrees with $\EE_n$ Koszul duality between $\clie^\bu(\fg)$, viewed as an $\EE_n$ algebra, and the $\EE_n$ enveloping algebra $U_{\EE_n} \fg$ \cite{Knudsen, Lurie,...}}

With an eye towards couplings, we do point out a universality that is satisfied by $!$-duality. 
In \cite[Chapter ??]{CG2} a factorization algebra enhancement of Noether's theorem is formulated. 
The general context is the following: 
\begin{itemize}
\item $\Obs$ is the factorization algebra of classical observables of a theory in the BV formalism on a manifold $M$.
\item $\cG$ is a local Lie algebra on $M$ which acs on the theory by local symmetries. 
\end{itemize}
Then, the classical version of Noether's theorem for factorization algebras produces a map of factorization algebras 
\[
\clie_\bu(\cG_c) \to \Obs .
\]

We arrive at $!$-duality in our situation above in the case that $\Obs$ is a factorization algebra of the form $\clie^\bu(\cL)$ where $\cL$ is another local $L_\infty$ algebra on $M$. 
This is always possible in the BV formalism: on an open set $U$ the space of Maurer--Cartan elements in the $L_\infty$ algebra $\cL(U)$ is the space of solutions to the equations of motion on $U$.
Thus, tautologically, the local Lie algebra $\cL$ is a symmetry of the classical theory. 

\end{document}
