%\documentclass[11pt]{amsart}

%\usepackage{../macros-master}
%\usepackage{macros-fivebrane}

%\begin{document}

\section{Factorization algebras in twisted $M$ theory}

The observables of any theory in the BV formalism enhance to the structure of a factorization algebra on spacetime \cite{CG2}. 
Classically, if a theory in the BV formalism is described by a local $L_\infty$ algebra $\cL$ on spacetime $M$ then to an open set $U \subset M$ the factorization algebra of classical observables assigns the cochain complex
\[
\clie^\bu(\cL(U))
\]
which computes the Lie algebra cohomology of $\cL(U)$. 
The differential on this cochain complex is precisely the Chevellay--Eilenberg differential associated to the $L_\infty$ structure maps. 

\begin{rmk}
The notion of a factorization algebra captures both the local and non-local operators in a quantum field theory. 
From the data of a factorization algebra, one can recover local operators by the following formal construction. 

Let $\Obs$ be a factorization algebra on a smooth manifold $M$.
The classical \defterm{point observables} at $p \in M$ is the limit $\Obs(p) = \lim_{U \ni p} \Obs(U)$ where the limit runs over open sets $U \subset M$ containing $p$.

Generally this limit is difficult to compute, but for certain theories it is possible to give a concise expression which captures \brian{difference between limit and homotopy limit}
\end{rmk}

At the quantum level this Chevellay--Eilenberg complex is deformed.
The BV formalism ...
Even in perturbation theory the full quantum behavior of the holomorphical-topological eleven-dimensional theory is an open question. 
We leave the problem to characterize the quantization to future work. 

\subsection{The factorization algebra associated to twisted supergravity}

\parsec[s:sugraobs]

As above, let $X$ be a Calabi--Yau fivefold and $S$ a real oriented one-manifold. 
In the first section we recalled a proposal for the description of the minimal twist of eleven-dimensional supergravity on $X \times S$. 
The primary object we used to describe the solutions of the equations of motion was the $\Z/2$ graded local $L_\infty$ algebra $\cL_{sugra}$, see \S \ref{s:Lsugra}. 
%As a graded vector space, observables on an open set $U \subset X \times S$ are thus $\cO(\cL_{sugra}(U)) = \Sym (\Pi \cL_{sugra})^\vee$.
%The differential is encoded by the $L_\infty$ structure which is exactly the Chevalley--Eilenberg differential. 
The $\Z/2$ graded commutative dg algebra of classical observables supported on an open set $U \subset X \times S$ is 
\[
\clie^\bu\left(\cL_{sugra}(U)\right) .
\]
We will denote the entire classical factorization algebra on $X \times S$ of classical observables by~$\clie^\bu(\cL_{sugra})$.

\parsec[s:membraneobs]


\parsec[s:fivebraneobs]

Let $Z$ be a complex threefold. 
In \ref{s:single} we have described the holomorphic twist of a theory on a single fivebrane within the (degenerate) BV formalism in terms of the (abelian) local $L_\infty$ algebra $\cL_{single}$ on $Z$. 
The factorization algebra of classical observables of the holomorphic twist of the single fivebrane theory assigns to an open set $U \subset Z$
the cochain complex
\[
\clie^\bu\left(\cL_{single}(U)\right) .
\]
We will denote the entire factorization algebra on $Z$ of classical observables by~$\clie^\bu(\cL_{single})$.

\subsection{Koszul duality for factorization algebras: an ansatz} 

In quantum field theory Koszul duality naturally appears in the problem of coupling topological line operators to some ambient bulk theory. 
More generally, for higher dimensional topological defects, this problem is encoded by Koszul duality for the theory of $\EE_n$ algebras \cite{??}.

More generally, we anticipate a general theory of Koszul duality for factorization algebras which should encode the problem of coupling arbitrary defects (without the condition of being topological).
Even for factorization algebras of holomorphic-topological nature this theory has not been studied in mathematics. 
Nevertheless, we will emphasize features that we expect this general form of Koszul duality to possess which will allow us to nail down its behavior on a rather general class of factorization algebras. 

In this first part of this subsection we briefly recall how Koszul duality enters in the problem of coupling line operators. 
We refer~\cite[\S 6]{CP1},~\cite[\S ??]{CG1}, or the review~\cite{PWkoszul} for more details. 
Then, we give an ansatz for Koszul duality for factorization algebras of the form $\clie^\bu(\cL)$ where $\cL$ is some local Lie algebra. 
From the point of view of the BV formalism this is not much of a condition, all such factorization algebras of classical observables can be cast in this form. 

\parsec[s:lines]
Suppose that we have a bulk theory living on a spacetime of the form 
\[
\R \times M 
\]
where $M$ is some smooth manifold. 
Denote by $\cA$ the corresponding factorization algebra on $\R \times M$. 
The theory could have arbitrary behavior along $M$, but we assume that the theory is topological along $\R$. 
This means that when viewed as a factorization algebra on $\R$ that $\cA$ is locally constant and is hence equivalent to the data of an $\EE_1$ or $A_\infty$ algebra.

Next, assume that $\cB$ is another $\EE_1$ algebra, which we think of as being associated to some quantum mechanical system along the real line.
This is a local model for the desired line operator that we are attempting to couple to the bulk theory.
Koszul duality enters in the problem of coupling the two quantum mechanical systems $\cA$ and $\cB$---where we view $\cA$ simply as an $\EE_1$ algebra. 

A coupling of the two systems is Maurer--Cartan element in the algebra
\[
\alpha \in \cA \otimes \cB .
\]
That is, $\alpha$ is an element of ghost degree one which satisfies the Maurer--Cartan equation
\[
\delta \alpha + \alpha \star \alpha = 0 .
\]
Given such an $\alpha$ we can deform the algebra $\cA \otimes \cB$ by adding the term $[\alpha,-]$ to the differential. 
In other words, at the cochain level only the differential, not the product structure, is modified. 

In principle, there are more general ways to `couple' two $\EE_1$ algebras; generally this is controlled by the Hochschild cohomology which governs algebra deformations of $\cA \otimes \cB$. 
we will elaborate further on this definition. 
In \cite{CG1} (see also \cite{PWkoszul}) it is shown how this notion relates to the physicists description of coupling in terms of local Lagrangians.

To see Koszul duality, the key observation is that the data of the Maurer--Cartan element $\alpha$ is equivalent to the data of a map of $\EE_1$ algebras
\[
\alpha \colon \cA^! \to \cB 
\]
where $\cA^!$ is Koszul dual to the algebra $\cA$. 
We then have the following slogan: the Koszul dual of the algebra of observables of the bulk theory $\cA^!$ is the algebra of operators on the {\em universal} line defect supported on $\RR \times \{x\}$, where $x \in M$.  

\parsec[s:celine]

Before moving towards our definition of Koszul duality for a general class of factorization algebras, we briefly recast the case of duality for $\EE_1$ algebras in terms of factorization algebras. 

We will focus on a slight generalization of the standard Koszul duality between the exterior and symmetric algebras.

\begin{prop}
Let $\lie{g}$ be a Lie algebra and equip the associative dg algebra $\clie^\bu(\fg)$ with the augmentation induced by the tautological Lie map $0 \to \lie{g}$.
The Koszul dual of $\clie^\bu(\lie{g})$ with respect to this augmentation is equivalent to the universal enveloping algebra $U \fg$. 
\end{prop}

There are explicit models for the associative dg algebras $\clie^\bu(\lie{g})$ and $U \lie{g}$ as locally constant factorization algebras on $\R$.
First, observe that we can tensor $\fg$ with the commutative dg algebra of de Rham forms to obtain a dg Lie algebra $\fg \otimes \Omega^\bu(\R)$. 
This has a natural enhancement to a local dg Lie algebra as this is simply the smooth sections of the bundle of Lie algebras $\fg \otimes \wedge^\bu \T^*_\RR$ equipped with the de Rham operator.

Using this local Lie algebra, we obtain a model for the associative (and commutative) dg algebra $\clie^\bu(\fg)$ as the factorization algebra
\[
\clie^\bu(\fg \otimes \Omega^\bu_\R).
\]
To an open set $U \subset \R$ this produces the Chevellay--Eilenberg complex computing the Lie algebra {\em cohomology} of the dg Lie algebra $\fg \otimes \Omega^\bu(U)$---the $\fg$-valued de Rham forms on $U$.
 
Similarly, a model for $U \fg$ is the locally constant factorization algebra
\[
\clie_\bu(\fg \otimes \Omega^\bu_{\R,c}).
\]
To an open set $U \subset \R$ this produces the Chevellay--Eilenberg complex computing the Lie algebra {\em homology} of the dg Lie algebra $\fg \otimes \Omega^\bu_c(U)$---the $\fg$-valued compactly supported de Rham forms on $U$.

\parsec[s:generalkoszul]

\begin{dfn}
Let $\cL$ be a local $L_\infty$ algebra on a manifold $M$ and consider the factorization algebra $\clie^\bu(\cL)$ which assigns to an open set $U \subset M$ the cochain complex $\clie^\bu(\cL(U))$. 
The \defterm{Koszul dual factorization algebra} is 
\[
\clie^\bu(\cL(U))^! \define \clie_\bu (\cL_{c}(U)) .
\]
In other words, the Koszul dual factorization algebra of $\clie^\bu(\cL)$ is the (untwisted) factorization enveloping algebra of the local Lie algebra $\cL$. 
\end{dfn} 

\subsection{The restricted factorization algebra}

For simplicity let us consider the eleven-dimensional theory on flat space $\R \times \C^5$ with some number of fivebranes supported on
\[
0 \times 0 \times \C^3 \subset \C^5 .
\]
The general prescription of twisted holography is that (after taking into account the backreaction) the observables on a large number of fivebranes is Koszul dual to the factorization algebra 

\parsec[s:flat]

%In the simple case where $Z = \C^3$ and we identify the total space of $K^{1/2}_Z \otimes \C^2$ with $\C^5$ then the manifold obtained by removing the locus of the brane is homeomorphic to
%\[
%\C^3 \times (\R \times \C^2 - 0) .
%\]
%
%Let $\pi \colon \R \times \C^5 - (0 \times \C^3) \to \C^3 \times \R_{>0}$ be the projection map whose fibers are homeomorphic to the sphere $S^4$ which links the location of the fivebranes.
%We restrict the factorization algebra of the eleven dimensional theory $\Obs_{sugra}$ to the open set obtained by removing the locus of the brane.

In the simple case that $Z = \C^3$ and we identify the total space of $K^{1/2}_Z \otimes \C^2$ with $\C^5$ there is a more direct construction of the factorization algebra $\Obs_{sugra}|_{Z}$. 

Let $\pi \colon \R \times \C^5 \to \C^3$ be the projection map.
Then, via $\pi$ we can pushforward the factorization algebra associated to the eleven-dimensional theory to obtain a factorization algebra
\[
\pi_* \Obs_{sugra} 
\]
on $\C^3$.
This factorization algebra is not the factorization algebra associated to an ordinary sort of field theory on $\C^3$. 
Nevertheless there is a subfactorization algebra which admits a natural grading so that each filtered component can be understood as such.

For any open set $U \subset \C^3$ we can consider the following $\C^\times$ action on the fields of the eleven-dimensional theory supported on $\R \times \C^2 \times U$ defined by
\begin{itemize}
\item On the fields $\mu(t;w,z) \in \Omega^\bu(\R) \otimes \PV^{1,\bu}(\C^2 \times U)$ the action is
\[
\lambda \cdot \mu(t;w,z) = \mu(\lambda t;\lambda w , z).
\]
\item On the fields $\nu(t;w,z) \in \Omega^\bu(\R) \otimes \PV^{0,\bu}(\C^2 \times U)$ the action is
\[
\lambda \cdot \nu(t;w,z) = \nu(\lambda t;\lambda w , z).
\]
\item On the fields $\beta(t;w,z) \in \Omega^\bu(\R) \otimes \Omega^{0,\bu}(\C^2 \times U)$ the action is
\[
\lambda \cdot \beta(t;w,z) = \lambda^{-1} \beta(\lambda t;\lambda w , z).
\]
\item On the fields $\mu(t;w,z) \in \Omega^\bu(\R) \otimes \Omega^{1,\bu}(\C^2 \times U)$ the action is
\[
\lambda \cdot \gamma(t;w,z) = \lambda^{-1} \gamma(\lambda t;\lambda w , z).
\]
\end{itemize}

For each $n \in \ZZ$ and open set $U \subset \C^3$, let 
\[
\pi_* \cL_{sugra}(U)^{(n)} \subset \cL_{sugra}(\R \times \C^2 \times U)
\]
be the weight $n$ eigenspace with respect to this $\C^\times$ action. 
The $\C^\times$ action is compatible with the $\Z/2$ graded $L_\infty$ structure on $\cL_{sugra}$. 
Since the $n$th eigenspace is trivial when $n < -1$, we see that the product
\[
(\Bar{\pi}_* \cL_{sugra})(U) \define \prod_{n \geq -1} (\pi_*\cL_{sugra})(U)^{(n)}
\]
is equipped with the structure of a $\Z/2$ graded $L_\infty$ algebra.
In this way, the assignment 
\[
\Bar{\pi}_* \cL_{sugra} \colon U \mapsto (\Bar{\pi}_* \cL_{sugra})(U) 
\]
defines a sheaf of $\Z/2$ graded $L_\infty$ algebras on $\C^3$. 

\begin{prop}
For each $n$, the sheaf of cochain complexes $U \mapsto \pi_* \cL_{sugra}(U)^{(n)}$ is quasi-isomorphic to one of the form
\[
\Omega^{0,\bu}(U, \cV^{(n)})
\]
for some finite rank super holomorphic vector bundle $\cV^{(n)}$ on $\C^3$. 
In particular, this endows $\Bar{\pi}_* \cL_{sugra}$ with the structure of a pro-vector bundle on $\C^3$. 
\end{prop}

Now, since $\Bar{\pi}_* \cL_{sugra}$ is a pro-vector bundle with a compatible $L_\infty$ structure, the assignment 
\[
U\subset \C^3 \mapsto \clie^\bu\left(\Bar{\pi}_* \cL_{sugra}(U)\right) 
\]
has the structure of a factorization algebra on $\C^3$ which we denote by $\Bar{\pi}_* \Obs_{sugra}$. 


\begin{prop}
Let $Z$ be a Calabi--Yau three-fold and $\pi \colon Z \times \C^2 \times \R \to Z$ be the projection. 
There is a pro local Lie algebra $\cL_{\pi, sugra}$ on $Z$ such that:
\begin{itemize}
\item[(1)] there is a natural inclusion of of factorization algebras on $Z$
\[
\Bar{\pi}_* \Obs_{sugra} \hookrightarrow \pi_* \Obs_{sugra}
\]
which is dense at the level of cohomology. 
\item[(2)] there is a weight grading on $\cL_{\pi,sugra}$ which is concentrated in degrees $\geq -1$ and gives rise to a decomposition of vector bundles
\beqn\label{eqn:decomp3}
\Bar{\pi}_{*} \cL_{sugra} = \prod_{n \geq -1} \cV_{n} 
\eeqn
\item[(3)]
In weight zero, there is an equivalence of local Lie algebras on $Z$ 
\[
\cV_0 \simeq \cE(3|6)|Z .
\]
\end{itemize}
\end{prop}

%We want to argue that $\Obs_{sugra} |_{\C^3} \cong \Bar{\pi}_* \Obs_{sugra}$. 

\parsec

\begin{prop}
On flat space 
\end{prop}

%\end{document}
