\documentclass[11pt]{amsart}

%\usepackage{../macros-master}
\usepackage{macros-fivebrane}

%\linespread{1.2} %for editing

\addbibresource{references.bib}

\begin{document}

\title{Fivebranes}

\section{A single fivebrane}

The holomorphic twist of the fivebrane theory is defined on any complex three-fold $X$ (which is not necessarily Calabi--Yau).
In full generality, we must assume that $X$ is equipped with a square-root of the canonical bundle $K_X^{\frac12}$. 
 
There are four fundamental fields $(\omega, \eta, \phi,\psi)$ which consist of Dolbeault forms of type $(1,1)$ and $(0,2)$ on $X$
\[
\omega \in \Omega^{1,1}(X), \quad \eta \in \Omega^{0,2}(X),
\]
and sections 
\[
\phi \in \Omega^0(X , K^{\frac12}_\Sigma), \quad \psi \in \Omega^{0,2} (X , K^{\frac12}) .
\]
%where $R \cong \CC^2$ is the fundamental $Sp(1)$ representation. 
The equations of motion read
\beqn
\label{eqn:eom}
\begin{split}
\dbar \omega + \del \eta & = 0 \\
\dbar \eta = \dbar \phi = \dbar \psi & = 0 .
\end{split}
\eeqn

There is a system of gauge symmetries for the fields $\omega, \eta, \psi$. 
First, there are ghosts $A^{0,1}, A^{1,0}$ of Dolbeault type $(0,1)$ and $(1,0)$, respectively, which act by
\beqn
\label{eqn:ghost}
\begin{split}
\omega & \mapsto \omega + \dbar A^{1,0} + \del A^{0,1} \\
\eta & \mapsto \eta + \dbar A^{0,1} .
\end{split}
\eeqn
There is also a ghost $a \in \Omega^{0,1}(X, K^{\frac12})$ which acts by $\psi \mapsto \psi + \dbar a$. 
Finally, there are ghosts-for-ghosts which act on the ghosts by 
\beqn
\label{eqn:gghost}
\begin{split}
A^{1,0} & \mapsto A^{1,0} + \del \chi \\
A^{0,1} & \mapsto A^{0,1} + \dbar \chi \\
a & \mapsto a + \dbar \phi ,
\end{split}
\eeqn
where $\chi \in C^\infty(X)$ is a scalar and $\varphi \in \Gamma(X, K^{\frac12}_X)$. 

This theory does not admit a Lagrangian description. 
However, it can still be put in a degenerate form of the BV formalism.
For the sections $\phi$ we simply 
Rather, it is a (free) theory equipped with a presymplectic BV structure defined by the following two-form on the space of fields\brian{...}

\subsection{Local operators}

Upon applying the $\dbar$-equations of motion, the algebra of local operators of the theory are algebraically generated by the following linear local operators and their holomorphic derivatives
\begin{itemize}
\item $\cO_\chi$, $\cO_{\d z_i}$ where
\[
\cO_{\d z_i} \colon A^{1,0} = f^j (z) \d z_j \mapsto f^i (0) 
\]
for $i=1,2,3$. 
Notice that $\cO_\chi$ is of ghost degree $+2$ and $\cO_{\d z_i}$ is of ghost degree $+1$. 
\item $\cO_{\phi}$, $\cO_{\varphi}$. 
The operator $\cO_{\phi}$ is of ghost degree zero and $\cO_{\varphi}$ is of ghost degree $+2$. 
\end{itemize}
The residual classical BRST operator acts by 
\beqn
Q \cO_{\d z_i} = \del_{z_i} \cO_{\chi} 
\eeqn
and annihilates all remaining operators. 

\begin{table}[htbp]
\centering
\begin{tabular}{llrrr}
 & $x$ & $y_1$ & $y_2$ & $z$ \\
$\cO_\chi$ & $x^{12}$ & 1 & 1 & 1\\
$\cO_{\d z_1}$ & $x^8$ & $y_1^{-1}$ & 1 & 1\\
$\cO_{\d z_2}$ & $x^8$ & $y_1$ & $y_2^{-1}$ & 1\\
$\cO_{\d z_3}$ & $x^8$ & $1$ & $y_2$ & 1\\
$\cO_\phi$ & x\textsuperscript{6} & 1 & 1 & $z^{\frac12}$ \\
$\cO_\varphi$ & x\textsuperscript{6} & 1 & 1 & $z^{-\frac12}$ \\
\end{tabular}
\caption{\label{tab:org156c4a5}Spectrum for a single fivebrane in the holomorphic twist}
\end{table}

\begin{prop}
The single particle index for a single fivebrane is 
\beqn
f_{five} (x,y_1,y_2,z) = \frac{x^6(z^{\frac12} + z^{-\frac12}) - x^8 (y_1^{-1} + y_1 y_2^{-1} + y_2) + x^{12}}{(1-x^4 y_1) (1-x^4 y_1^{-1} y_2) (1-x^4 y_2^{-1})} .
\eeqn
\end{prop}

\section{The $E(3,6)$ symmetry}

For $d > 1$ punctured affine space $\CC^d \setminus 0$ is not affine. 
We denote by $\sfA_3$ a particular algebraic model for the derived space of functions on punctured affine space. 
This model admits an inclusion of dg algebras 
\[
\sfA_3 \hookrightarrow \RR\Gamma(\CC^3 \setminus 0, \cO) 
\]
which is dense at the level of cohomology.


\begin{thm}
The operators $J[r,s]$ and their descendants form the dg Lie algebra
\beqn
\sfA_{3} \otimes \cO(\CC^2) 
\eeqn
where the commutator is 
\[
[F, G] = \ep_{ab} \del_{w_a} F \del_{w_b} G + \oint_{z \in S^5} \ep_{ab} \del_{w_a} F \del_{w_b} G |_{w=0}   .
\]
\end{thm}

\textbf{Notation}: 
Recall that the $\mu$-field is a section of $\PV^{1,\bu}(\CC^5)\otimes \Omega^\bu(\RR)$.
We will decompose it as
\[
\mu = \mu_{z_i} \del_{z_i} + \mu_{w_a} \del_{w_a} 
\]
so that the $\mu_{z_i}$ correspond to vector fields that point along the fivebrane and the $\mu_{w_a}$ are transverse. 
In this section we make explicit the Dolbeault-de Rham form dependence on the $\mu$-fields.
We will denote the components of the $\mu_{w_a}$-field, for instance, by
\[
\mu^{k;l}_{w_i} \del_{w_i} \in \PV^{1,k}(\CC^5) \otimes \Omega^{l} (\RR) .
\]

Introduce the differential operator
\[
D_{r,s} \define \frac{1}{r!s!} \del_{w_1}^r \del_{w_2}^r .
\]
Recall that there are bosonic operators $\til{J}_a [r,s]$ which couple to the vector fields $\mu_{w_a}$ which are normal to the fivebrane via
\beqn
\label{eqn:Ja}
\int_{\CC^3} \til{J}_a [r,s] (z) D_{r,s} \mu_{w_a}^{3;0}(z) .
\eeqn
Notice that $J_a[r,s]$ is a local (zero form) operator which couples to the Dolbeault $(0,3)$ part of $\mu_{w_a}$. 

By holomorphic descent we obtain the Dolbeault valued local operator 
\beqn
\til{J}_a^\bu [r,s] (z) = \til{J}_a [r,s] + \til{J}_a^i[r,s] \d \zbar_i + \til{J}^{ij}_{a} [r,s] \d \zbar_i \d \zbar_j + \cdots .
\eeqn
This Dolbeault valued operator couples by the same formula as in \eqref{eqn:Ja}.
In this coupling other Dolbeault components of $\mu_{w_a}$ also appear,
\beqn
\label{eqn:Jb}
\int_{\CC^3} \til{J}_a [r,s] (z) D_{r,s} \mu_{w_a}^{3;0}(z) + \int_{\CC^3} \til{J}^i_a [r,s] (z) D_{r,s} \mu_{w_a}^{2;0}(z) \d \zbar_i + \int_{\CC^3} \til{J}^{ij}_a [r,s] (z) D_{r,s} \mu_{w_a}^{1;0}(z) \d \zbar_i \d \zbar_j .
\eeqn

\subsection{$\til J_1 \til J_1$}

The gauge variation of $\mu_1$ is
%\begin{align*}
%	Q \mu_1^{3;0} & = \dbar \mu^{2;0} + \mu_{w_a} \partial_{w_a} \fc_{w_1} + \mu_{z_i} \partial_{z_i} \fc_{w_1} - \fc_{w_a} \partial_{w_a} \mu_1 - \fc_{z_i} \partial_{z_i} \mu_{w_1} \\
%& + \partial_{w_2} \fc_\gamma \partial_z \alpha -  \partial_z\fc_\gamma \partial_{w_2} \alpha + \partial_{w_2} \fc_\alpha \partial_z \gamma - \partial_z \fc_\alpha \partial_{w_2} \gamma .
%\end{align*}

\subsection{Embedding $E(3|6)$ into twisted supergravity}


\parsec[]
First we explain how $E(3|6)$ embeds inside of $E(5|10)$. 
We follow \cite{KacCheng}. 

A vector field $X = X(z) \in \Vect(\C^3)$ embeds as the vector field
\[
X \mapsto X(z) - \frac12 (\del X)(z) \sum_{i=1}^2 w_a \del_{w_a} .
\]
Here, $\del X$ is the divergence of $X$ with respect to the standard holomorphic volume form on $\C^3$. 
The elements $e,h,f \otimes f(z) \in \lie{sl}(2) \otimes \C[z_i]$ embed as the vector fields
\[
f(z) w_1 \del_{w_2} , \quad f(z) (w_1 \del_{w_2} - w_2 \del_{w_1}, \quad f(z) w_2 \del_{w_1} . 
\]
respectively. 

Next, we explain how the odd summand embeds. 
Recall, a general element in the odd part of $E(3|6)$ can be written as
\[
g(z) \del_{z_i}^{1/2} \otimes r_\pm .
\]
This element embeds into $E(5|10)$ as the following closed two-form
\[
\del \left((w_1-w_2) g(z) \d z_i \right) .
\]
\brian{check this}

\subsection{The local Lie algebra} 

Let $\cE(5|10)$ be the following local Lie algebra on $X \times \C^2 \times \R$. 

Let $\pi \colon X \times \C^2 \times \RR \to X$. 
We will give a model $\cG_{grav}$ for $\pi_* \cE(5|10)$ as a local Lie algebra on $\C^3$. 

First, introduce the holomorphic line bundle 
\[
w_1^k w_2^\ell \cO_{X} \define ??  .
\]
Let $\lie{s}_2$ denote the Lie algebra of polynomial vector fields in the variables $w_1,w_2$ which preserve the holomorphic volume form $\d w_1 \d w_2$. 

\begin{dfn}
Let $\cG_{grav}$ be the following local Lie algebra on the complex three-fold $X$. 
\begin{itemize}
\item The even part is 
\[
\Omega^{0,\bu}(X , \T_X \otimes \cO_X [w_1,w_2]) \oplus \Omega^{0,\bu}(X , \lie{s}_2) .
\]
\item The odd part is 
\[
\Omega^{\geq 2, \bu}(X, \cO_X[w_1,w_2]) \oplus \Omega^{1,\bu}(X, w_1 \cO_X[w_1,w_2]) \oplus \Omega^{1,\bu}(X, w_2 \cO_X[w_1,w_2]) \oplus \cdots
\]
\end{itemize} 
\end{dfn}

\parsec

Consider the following auxiliary grading on the local Lie algebra $\cG_{grav}$. 
The variables $w_a$ are declared to have weight $+1$. 
%The variables $\til w_a$ are declared to have weight $-1$. 
Additionally, parity has weight $-1$.
Thus, for example, the weight of $w_1 \d z_1 \d z_2$ is zero.

\section{Careful index computation}

Let's compute the index of a single fivebrane. 

\begin{table}[htbp]
\centering
\begin{tabular}{llrrr}
 & $t_1$ & $t_2$ & $q$ & $r$ \\
$z_1$ & $1$ & 0 & 0 & $-1$\\
$z_2$ & $-1$ & $1$ & 0 & $-1$\\
$z_3$ & $0$ & $-1$ & 0 & $-1$\\
$w_1$ & $0$ & $0$ & $1$ & $3/2$\\
$w_2$ & 0& 0 & $-1$ & $3/2$ \\
\\
\end{tabular}
\caption{Assignment of fugacities}
\end{table}

\begin{itemize}
\item Single particle operators on the odd copy of holomorphic two-forms $\Pi \Omega^{2,hol}$ contribute
\[
- r^3 \frac{t_1 r^{-1} + t_1^{-1} t_2 r^{-1} + t_2^{-1} r^{-1} }{(1-t_1^{-1}r) (1-t_1 t_2^{-1} r) (1-t_2 r)} 
\]
\item Single particle operators on the even copy of holomorphic three-forms $\Omega^{3,hol}$ contribute
\[
r^3 \frac{1}{(1-t_1^{-1}r) (1-t_1 t_2^{-1} r) (1-t_2 r)} 
\]
\item Single particle operators on $K^{-1/2} \otimes \C^2$ contribute
\[
f_{single} (t_1,t_2,q,r) = r^{\frac32} \frac{q + q^{-1}}{(1-t_1^{-1}r) (1-t_1 t_2^{-1} r) (1-t_2 r)}
\]
\end{itemize}

Putting this all together we get 
\[
f_{tensor} (t_1,t_2,q,r) = \frac{(q + q^{-1})r^{\frac32} - (t_1 + t_1^{-1} t_2 + t_2^{-1} )r^2 + r^3}{(1-t_1^{-1}r) (1-t_1 t_2^{-1} r) (1-t_2 r)} 
\]
\end{document}