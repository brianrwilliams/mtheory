\section{Conjectures for operators on a finite number of fivebranes}

In conjecture \ref{conj:fact} we have formulated a conjectural description of the factorization algebra of classical observables $\Obs_{N}$ associated to the worldvolume theory on a stack of $N$ fivebranes in the holomorphic twist.
In this section we begin to provide some evidence for this description at the level of local operators, which is just a small piece of the factorization algebra.
As we reviewed just in the previous section, the space of local operators is what categorifies the specific superconformal index that we study in this paper.
%The structure of a three-dimensional holomorphic factorization algebra induces algebraic operations on the algebra of local operators

For each $N$ we have constructed a local Lie algebra $\cG_N$ on the worldvolume three-fold $Z$. 
Our conjecture is that the factorization algebra associated to a stack of $N$ holomorphically twisted fivebranes is $\Obs_N \simeq \clie_\bu(\cG_{N,c})$, where $\cG_{N,c}$ denotes the cosheaf of compactly supported sections. 
The main goal of this section is to extract the algebra of local operators from this factorization algebra and to then deduce explicit formulas for the local character, and hence the superconformal index.

For a stack of $N=1$ fivebranes, which corresponds to the abelian six-dimensional superconformal field theory, we find that our local character matches exactly with the expressions in the literature. 
This is not a surprise as we have shown that even at the level of factorization algebras $\clie_\bu(\cG_{1,c})$ is quasi-isomorphic to~$\Obs_1$, see Proposition~\ref{prop:factabelian}.

The main computation of this section is a closed formula for the local character of the factorization algebra $\clie_\bu(\cG_{N,c})$ for $N > 1$, see Theorem \ref{thm:finite}. 
Following conjecture \ref{conj:fact} and the general discussion of \S \ref{sec:sucaindex} we are led to hypothesize a closed formula for the superconformal index for the theory on a finite number of fivebranes (in flat space).
As far as the authors are aware of there is no closed formula for the refined superconformal index (with four independent fugacities) for the theory on a stack of $N > 1$ fivebranes.
For small values of $N$ we expand our closed formulas to low orders in the fugacity $q$ (which roughly counts instanton charge) to match exactly with expressions in the literature. 

\subsection{Operators on a single fivebrane}

We deduce the character of the holomorphic twist of the theory on a single fivebrane and will find an exact match with the index of the six-dimensional superconformal theory associated to the abelian Lie algebra $\lie{gl}(1)$.
By the proposition \ref{prop:factabelian} we can compute this character either from a first principles description of the theory, or holographically by focusing on the weight $(-1)$ part of the decomposition of $\Bar{\pi}_* \Obs_{sugra}$.

\begin{lem}
\label{lem:single}
The $\Z \times \Z/2$ graded algebra of local operators $\Obs_{1}(0)$ of the holomorphic twist of the worldvolume theory of a single fivebrane is quasi-isomorphic to the graded symmetric algebra on the linear dual of the topological vector space
\beqn\label{eqn:localfree}
V_0[[z_1,z_2,z_3]] \simeq \Omega^{2}_{cl} (\Hat{D}^3)[1] \oplus \Pi \Omega^0(\Hat{D}^3, K^{1/2}) \otimes \C^2 [1].
\eeqn
\end{lem}

\begin{proof}
The jet expansion at $0 \in \C^3$ determines a map from the sections of the abelian holomorphic-topological local Lie algebra on $\C^3$ to the cochain complex
\beqn
\begin{tikzcd}
\ul{-1} & \ul{0} \\
\Omega^{2}(\Hat{D}^3) \ar[r,"\del"] & \Omega^{3}(\Hat{D}^3) \\
\Pi \Omega^0(\Hat{D}^3, K^{1/2}_{\Hat{D}^3}\otimes \C^2) . 
\end{tikzcd} 
\eeqn
On the formal disk all closed two-forms are automatically exact, which implies the lemma.
\end{proof}

We present the character of $\Obs_1(0)$ as the plethystic exponential of the character $f_1(t_1,t_2,r,q)$ of the space of linear local operators
\beqn
\chi_{1} (t_1,t_2,r,q) = {\rm PExp} \big[f_1(t_1,t_2,r,q) \big] .
\eeqn
According to the weights listed above and using the description of local operators in Lemma \ref{lem:single} we have the following contributions to the single particle character~$f_{1}(t_1,t_2,r,q)$.

\begin{itemize}
\item Single particle operators on the odd copy of holomorphic two-forms $\Pi \Omega^{2,hol}$ contribute
\[
- q^2 \frac{\chi^{\lie{sl}(3)}_{[0,1]}(t_1,t_2)}{(1-t_1^{-1}q) (1-t_1 t_2^{-1} q) (1-t_2 q)} = - q^2 \frac{t_1  + t_1^{-1} t_2  + t_2^{-1} }{(1-t_1^{-1}q) (1-t_1 t_2^{-1} q) (1-t_2 q)}
\]
where $\chi^{\lie{sl}(3)}_{[1,0]}(t_1,t_2)$ is the $\lie{sl}(3)$ character of highest weight $[1,0]$.
\item Single particle operators on the even copy of holomorphic three-forms $\Omega^{3,hol}$ contribute
\[
q^3 \frac{1}{(1-t_1^{-1}q) (1-t_1 t_2^{-1} q) (1-t_2 q)} 
\]
\item Single particle operators on $K^{1/2} \otimes \C^2$ contribute
\[
q^{3/2} \frac{\chi_{1}^{\lie{sl}(2)} (r)}{(1-t_1^{-1}q) (1-t_1 t_2^{-1} q) (1-t_2 q)} = q^{3/2}\frac{(r + r^{-1})}{(1-t_1^{-1}q) (1-t_1 t_2^{-1} q) (1-t_2 q)}
\]
where $\chi_{1}^{\lie{sl}(2)} (r)$ is the $\lie{sl}(2)$ character of highest weight one.
\end{itemize}

Putting this all together we obtain the following.

\begin{prop}
\label{prop:6done}
The local character $\chi_{1}(t_1,t_2,r,q)$ of the holomorphic twist of the theory on a single fivebrane is given by the plethystic exponential of the single particle character
\beqn\label{eqn:6done}
f_{1} (t_1,t_2,r,q) = \frac{q^{3/2}(r + r^{-1}) - q^2(t_1 + t_1^{-1} t_2 + t_2^{-1} ) + q^3}{(1-t_1^{-1}q) (1-t_1 t_2^{-1} q) (1-t_2 q)} .
\eeqn
\end{prop}

In terms of the parameters $y_1,y_2,y_3,y,q$ this single particle character reads
\beqn
\label{eqn:6done1}
f_{1} (y_i,y,q) = \frac{qy + q^2y^{-1} - q^2(y^{-1}_1+y^{-1}_2+y^{-1}_3) + q^3}{(1-y_1q) (1-y_2 q) (1-y_3 q)} .
\eeqn
The expression matches exactly with the index of the abelian six-dimensional superconformal theory.
For example, compare with \cite[Eq. (3.1)]{Kim:2013nva} or \cite[Eq. (3.35)]{Bhattacharya:2008zy}.
From now on, we will give all formulas for the index in terms of the parameters $y_1,y_2,y_3,y,q$.

Generally speaking, after twisting there are enhancements of symmetries which are present in the original theory. 
In \cite{SW6d} we have shown that at the level of the holomorphic twist the twisted superconformal algebra $\lie{osp}(6|2)$ gets enhanced to the infinite-dimensional exceptional super Lie algebra $E(3|6)$ \cite{KacClass}. 
For the case of the single fivebrane theory, this implies that the local operators $\Obs_{1}(0)$ form a representation for $E(3|6)$. 

This fact also follows from our holographic analysis. 
In \S \ref{s:fact} we have expressed the restriction of the factorization algebra of observables of twisted eleven-dimensional supergravity to the three-fold $Z$ as the Chevalley--Eilenberg cochains of a local $L_\infty$ algebra $\cG$. 
Recall that we have a decomposition of local Lie algebras $\cG = \oplus_{j \geq -1} \cG^{(j)}$ on the three-fold $Z$. 
In Proposition \ref{lem:single} we have shown that $\clie_\bu (\cG_{c}^{(-1)})$ is equivalent to the factorization algebra $\Obs_{1}$.
On $\C^3$, the global sections of the local Lie algebra $\cG^{(0)}$ is closely related to $E(3|6)$---the $\infty$-jets of $\cG^{(0)}$ at $0 \in \C^3$ is quasi-isomorphic to $E(3|6)$.
Combining these facts we see that $\Obs_{1}(0)$ is a module for $E(3|6)$. 

In appendix \S \ref{s:kr} we will describe this module from a purely representation theoretic point of view and compare our expression for the character to the character of a certain irreducible module for the exceptional super Lie algebra $E(3|6)$ considered in~\cite{KR2}.

%\parsec
%
%There are various degenerations, or specializations, of this character which are interesting to consider.
%These specializations involve restricting the character above to a subalgebra of the full Cartan that we considered above.
%
%One degeneration of this character involves specializing $t_1=t_2=1$ which results in the $U(1) \times SU(2)$ character:
%\beqn
%f_{1}(r,q) = \frac{(r+r^{-1})q^{3/2} - 3 q^{2} + q^3}{(1-q)^3} .
%\eeqn
%They compute the absolute (non-super) character of the module $I(0,0;1;-1)$ where they additionally specialize $t_1=t_2=r=1$. 
%In a similar method to the one used in \cite{KR1}, one can compute the specialized (super) character of $I(0,0;1;-1)$ to find
%\[
%\chi_{u(1)} (q,t_1=t_2=r=1) = \frac{2 q^{3/2} - 3 q^2 + q^3}{(1-q)^3} .
%\]

\parsec
There are various degenerations, or specializations, of this character which are interesting to consider.
A particularly meaningful one is related to two different deformations of the theory by elements in the (twisted) superconformal algebra.
Recall that after performing the holomorphic twist the residual superconformal algebra is~$\lie{osp}(6|2)$.
We have recalled in \S\ref{s:global1} how the bosonic part of this algebra is represented by fields of the eleven-dimensional theory. 

There are two types of odd elements of~$\lie{osp}(6|2)$ that also have a natural interpretation in the eleven-dimensional theory.
The odd part of~$\lie{osp}(6|2)$ can be identified with the twelve-dimensional space
\[
\C^3 \otimes \C^2 \oplus \wedge^2(\C^3) \otimes \C^2 
\]
where $\C^3, \C^2$ are the fundamental $\lie{sl}(3)$ and $\lie{sl}(2)$ representations, respectively. 
The $\lie{gl}(1)$ factor in the bosonic part of $\lie{osp}(6|2)$ acts with weight $1/2$ on both summands. 

\begin{itemize}
\item The summand $\C^3 \otimes \C^2$ embeds into the ghosts of twisted supergravity via the $\gamma$-type fields which satisfy
\[
\del \gamma = \d w_a \d z_i .
\]
where $i=1,2,3$ and $a = 1,2$.
Note that $\gamma$ appears to be ambiguous up to a closed holomorphic one-form, but since there is a linear gauge symmetry which sends $\beta \mapsto \del \beta$, it implies that $\gamma$ is unique up to a BRST exact term. 
Since in our model all closed one-forms are rendered trivial in cohomology
\item The summand $\wedge^2(\C^3) \otimes \C^2$ embeds as another $\gamma$-type field which satisfies 
\[
\del \gamma = w_a \d z_i \d z_j .
\]
\end{itemize}

Both deformations break the global Cartan subalgebra down to $\lie{gl}(1) \times \lie{gl}(1)$ according to the specializations
\beqn\label{eqn:special1}
y=1 , \quad y_3 = 1 .
\eeqn
Notice that due to the constraint $y_1y_2y_3=1$ this forces $y_1 = y_2^{-1}$.
As one can easily check, this specialization yields the following single particle index
\[
f_{1}(y_1, y_1^{-1},y_3=1, y=1, q) = \frac{q}{1-q} 
\]
which recovers the single particle index of a single chiral boson on the Riemann surface $\Sigma = \C_{z_1}$. 
Notice that the dependence on the parameter $y_1$ has completely dropped out even though we have not specialized it to any value.
%Notice that although the Cartan subalgebra generated by the vector field $z_1 \del_{z_1} - z_2 \del_{z_2}$ is unbroken by this deformation, the dependence on its fugacity $t_1$ completely drops out of the expression.

\subsection{A conjectural description of operators on a stack of two fivebranes}

In \S\ref{sec:factsummary} we saw that the decomposition of the local $L_\infty$ algebra $\cG = \cG_Z$ on $Z$ induces a filtration of the factorization algebra $\clie_\bu(\cG_c)$. 
\[
\clie_{\bu}(\cG_{1,c}) \subset \clie_{\bu}(\cG_{2,c}) \subset \cdots .
\]
We now turn to the factorization algebra $\clie_{\bu}(\cG_{2,c})$.

Recall that $\cG_{2}$ is the local $L_\infty$ algebra on $Z$ defined as $\cG_{2} = \cG / \cG^{\geq 1}$. 
Since $\cG$ is concentrated in weights $\geq -1$ we see that $\til \cG_{2}$ is of the form
\[
\cG_2 = \til \cG_2 \ltimes \cG_1 
\]
where $\cG_1 = \cG^{(-1)}$ is the weight $(-1)$ piece and $\til \cG_2 = \cG^{\geq 0} / \cG^{\geq 1} = \cG^{(0)}$.  
We focus mostly on the factorization algebra $\clie_\bu(\til \cG_{2,c})$.

We have already characterized the local dg Lie algebra $\til \cG_{2} = \cG^{(0)}$ as the weight zero part of $\cG$ on on any threefold $Z$ in \S\ref{s:weight0}. 
We have also shown that $\cG^{(0)}$ is equivalent to the local Lie algebra $\cE(3|6)$. 
The even part of $\cE(3|6)$ is
\[
\Omega^{0,\bu}(Z, \T_Z) \oplus \Omega^{0,\bu}(Z) \otimes \lie{sl}(2) 
\]
with its natural cohomological grading by Dolbeault form type. 
The odd part of $\cE(3|6)$ is
\[
\Omega^{1,\bu}(Z, K_Z^{-1/2}) \otimes \C^2 .
\]
The differential is $\dbar$ and the Lie bracket has been described in \S\ref{s:weight0}.

%On $Z = \C^3$ this local dg Lie algebra is related to the exceptional simple super Lie algebra $E(3|6)$ classified by Kac \cite{KacClass}. 
%Indeed, one can show (see the forthcoming work \cite{SW6d}) that the fiber of the $\infty$-jet bundle of $\cG_2$ at $0 \in \C^3$ is quasi-isomorphic to $E(3|6)$. 

\parsec

We continue by computing the character of local operators associated to the factorization algebra $\clie_\bu(\cG_{2,c})$ using Lemma~\ref{lem:envelope}.
For simplicity we will use the fugacities $y_i, y, q$.

\begin{itemize}
\item Single particle operators coming from the copy of holomorphic vector fields $\Vect^{hol}(\C^3)$ contribute
\[
q^4 \frac{\chi^{\lie{sl}(3)}_{[1,0]}(y_i)}{(1-y_1q) (1-y_2 q) (1-y_3 q)}  = q^4 \frac{y_1 + y_2 + y_3}{(1-y_1q) (1-y_2 q) (1-y_3 q)} 
\]
\item Single particle operators coming from $\lie{sl}(2)$-valued holomorphic functions $\lie{sl}(2) \otimes \cO^{hol}(\C^3)$ contribute
\[
q^3 \frac{\chi_2^{\lie{sl}(2)} (q^{-1/2}y)}{(1-y_1q) (1-y_2 q) (1-y_3 q)}  = \frac{q^2 y^2 + q^3 + q^4 y^{-2}}{(1-y_1q) (1-y_2 q) (1-y_3 q)} 
\]
%\[
%q^3\frac{r^2 + r^{-2} + 1}{(1-t_1^{-1}q) (1-t_1 t_2^{-1} q) (1-t_2 q)} 
%\]
\item Single particle operators coming from the odd piece of $E(3|6)$ which is $\Omega^{1,hol} \otimes K^{-1/2} \otimes \C^2$ contribute
\[
q^{7/2}\frac{\chi^{\lie{sl}(2)}_{1}(q^{-1/2} y) \, \chi_{[0,1]}^{\lie{sl}(3)} (y_i)}{(1-y_1q) (1-y_2 q) (1-y_3 q)} = q^{3}\frac{(y + q y^{-1})(y_1^{-1} + y_2^{-1} + y_3^{-1})}{(1-y_1q) (1-y_2 q) (1-y_3 q)} 
\]
\end{itemize}

Combining these expressions we obtain the following.

\begin{prop} \label{prop:6dtwo}
The character of local operators of the factorization algebra $\clie_\bu(\til \cG_{2,c})$ on $\C^3$ is given by the plethystic exponential of the following expression
\beqn\label{eqn:6dtwo}
\til f_{2} (y_i,y,q) = \frac{q^4(y_1+y_2+y_3) + q^2 (y^2 + q + q^2 y^{-2}) - q^{3} (y + q y^{-1})(y_1^{-1} + y_2^{-1} + y_3^{-1})}{(1-y_1q) (1-y_2 q) (1-y_3 q)}.
\eeqn
%\beqn\label{eqn:6dtwo}
%f_{2} (t_1,t_2,r,q) = \frac{q^4(t_1^{-1} + t_1 t_2^{-1}  + t_2) + q^3 (r^2 + r^{-2} + 1) - q^{7/2} (r + r^{-1})(t_1 + t_1^{-1} t_2 + t_2^{-1})}{(1-t_1^{-1}q) (1-t_1 t_2^{-1} q) (1-t_2 q)} .
%\eeqn
\end{prop}

Recall that our conjecture for the factorization algebra associated to the holomorphic twist of the six-dimensional worldvolume theory associated to a stack of two fivebranes is $\Obs_2 \simeq \clie_\bu(\cG_{2,c}) \cong \clie_\bu(\cG_{1,c}) \otimes \clie_\bu(\til \cG_{2,c})$. 
And after removing the center of mass degrees of freedom, our conjecture is $\til \Obs_2 \simeq \clie_\bu(\til \cG_{2,c})$.
We can now state a decategorified version of conjecture \ref{conj:fact} at the level of superconformal indices, or local characters.

\begin{conj}\label{conj:6dtwo}
The superconformal index of the six-dimensional superconformal theory associated to the Lie algebra $\lie{sl}(2)$ is given by
\[
\til \chi_{2} (y_i,y,q) = {\rm PExp} \left[\til f_2(y_i,y,q) \right] .
\]
where $\til f_2(y_i,y,q)$ is as in \eqref{eqn:6dtwo}.
\end{conj}

Similarly, the index associated to the $\lie{gl}(2)$ theory, which is the local character of $\clie_\bu(\cG_{2,c})$, is conjectured to be simply the product 
\[
\chi_{2} (y_i,y,q) = \chi_{2} (y_i,y,q) \cdot \chi_{1}(y_i,y,q)
\]
where the character $\chi_{1}$ for the $\lie{gl}(1)$ theory is given in proposition~\ref{prop:6done}
Equivalently, $\chi_2$ is the plethystic exponential of $f_2 = f_1 + \til f_2$. 

%\parsec[]
%
%The specialization of this index $t_1=t_2=r=1$ yields the single particle index
%\[
%\frac{3q^4 + 3 q^3 - 6 q^{7/2}}{(1-q)^3}. 
%\]

\parsec[]

The specialization of this index $y=1, y_3=1$ in \eqref{eqn:special1} yields the plethystic exponential of the following single particle index
\[
\til f_{2}(y_1, y_2, y_3=1, y=1, q) = \frac{q^2}{1-q} 
\]
which is the same as the single particle index of Virasoro vacuum module on the Riemann surface $\Sigma = \C_{z_1}$. 

\subsection{A closed formula for the finite $N$ index}

Before exhibiting the general formula for the local character of the factorization algebra $\clie_\bu(\cG_{N,c})$ on $\C^3$ we set up some notation. 
As above, we let $\chi_k^{\lie{sl}(2)}$ and $\chi^{\lie{sl}(3)}_{[k,l]}$ denote the highest weight $\lie{sl}(2)$ and $\lie{sl}(3)$ characters. 
We also define the following expression which appears in the denominator in all of our characters
\beqn
d(y_i,y,q) = (1-y_1 q)(1-y_2q)(1-y_3q) .
\eeqn 
To simplify formulas, we will temporarily denote the single particle character for the $N=1$ theory $\Obs_1$ by 
\beqn
g_{-1} (y_i,y,q) = f_1(y_i,y,q)
\eeqn
where $f_1(y_i,y,q)$ is as in equation \eqref{eqn:6done1} and also denote by 
\beqn
g_0 (y_i,y,q) = \til f_2(y_i,y,q)
\eeqn
where $\til f_2(y_i,y,q)$ is as in equation \eqref{eqn:6dtwo}. 
Thus $g_2$ is the single particle local character of $\clie_\bu(\til \cG_{2,c}) = \clie_{\bu}(\cG_c^{(0)})$.
Finally, for $k \geq 1$ let
%\begin{align*}
%f_k (y_1,y_2,y_3,y,q) & \define q^{3k/2} \left(q \chi^{\lie{sl}(2)}_{k-2}(q^{-1/2} y)(y_1 + y_2 + y_3) + \chi^{\lie{sl}(2)}_k(q^{-1/2} y) \right. \\
%& \left.  - q \chi^{\lie{sl}(2)}_{k-3}(q^{-1/2}y) - \chi^{\lie{sl}(2)}_{k-1} (q^{-1/2} y) (y_1^{-1} + y_2^{-1} + y_3^{-1} ) \right) .
%\end{align*}
%\begin{align*}
%g_k (y_i,y,q) & \define q^{3} \left(q^{1 + 3 (k-2)/2} \chi^{\lie{sl}(2)}_{k-2}(q^{-1/2} y)(y_1 + y_2 + y_3) + q^{3(k-2)/2} \chi^{\lie{sl}(2)}_k(q^{-1/2} y) \right. \\
%& \frac{\left.  - q^{3(k-1)/2} \chi^{\lie{sl}(2)}_{k-3}(q^{-1/2}y) - q^{-1 + 3(k-1)/2} \chi^{\lie{sl}(2)}_{k-1} (q^{-1/2} y) (y_1^{-1} + y_2^{-1} + y_3^{-1} ) \right)}{d(y_i,y,q)} .
%\end{align*}
\begin{align*}
g_k (y_i,y,q) & \define q^{3} \left(q^{1 + 3 k/2} \chi^{\lie{sl}(2)}_{k}(q^{-1/2} y)(y_1 + y_2 + y_3) + q^{3k/2} \chi^{\lie{sl}(2)}_{k+2}(q^{-1/2} y) \right. \\
& \frac{\left.  - q^{3(k+1)/2} \chi^{\lie{sl}(2)}_{k-1}(q^{-1/2}y) - q^{-1 + 3(k+1)/2} \chi^{\lie{sl}(2)}_{k+1} (q^{-1/2} y) (y_1^{-1} + y_2^{-1} + y_3^{-1} ) \right)}{d(y_i,y,q)} .
\end{align*}
%and hence the conjectural single particle index for the superconformal theory associated to the Lie algebra $\lie{sl}(2)$. 

\begin{thm}
\label{thm:finite}
Let $N \geq 3$. 
The local character of the factorization algebra $\clie_{\bu}(\cG_{N,c})$ is
\beqn
\chi_{N}(y_1,y_2,y_3,y,q) = \text{PExp}\left[\sum_{k=-1}^{N-2} g_k(y_1,y_2,y_3,y,q)\right].
\eeqn
Similarly, the local character of the factorization algebra $\clie_\bu(\til{\cG}_{N,c})$ is 
\beqn
\til{\chi}_{N}(y_1,y_2,y_3,y,q) = \text{PExp}\left[\sum_{k=0}^{N-2} g_k(y_1,y_2,y_3,y,q)\right].
\eeqn
\end{thm}
\begin{proof}
By Lemma~\ref{lem:envelope} the character of $\clie_\bu (\cG_{N,c})$ is given by 
\beqn
\chi_N = \text{PExp} \left[f_N\right]
\eeqn
where $f_N$ is the single particle local character.
Thus, it suffices to show that $f_N = \sum_{k = -1}^{N-2} g_k$.
Recall that from the filtration \eqref{eqn:fil1} we obtain a filtration of the factorization algebra $\clie_\bu(\cG_{N,c})$:
\beqn
\clie_{\bu}(\cG_{1,c}) \subset \clie_\bu(\cG_{2,c}) \subset \cdots \subset \clie_{\bu}(\cG_{N,c}) .
\eeqn
%\beqn
%\clie_{\bu}(\cG_{2,c}) \leftarrow \clie_{\bu}(\cG_{3,c}) \leftarrow \cdots \leftarrow \clie_\bu (\cG_{N,c}) .
%\eeqn
associated graded factorization algebra is of the form
\beqn
\text{Gr} \, \clie_{\bu} (\cG_{N,c}) \simeq \bigoplus_{k=-1}^N \Sym \left(\cG^{(k)}_c\right).
\eeqn

Thus, we must see that $g_k$ is the single particle index of the factorization algebra $\clie_\bu(\cG^{(k)}_c)$, which is a direct observation using the description of $\cG^{(k)}$ we have given in Proposition \ref{prop:Vj}.
\end{proof}

We thus arrive at the following conjecture for the index of the worldvolume theory on a stack of a finite number of fivebranes which we phrase in terms of the six-dimensional superconformal theory associated to the Lie algebra $\lie{sl}(N)$.

\begin{conj} 
The superconformal index of the six-dimensional superconformal theory associated to the Lie algebra $\lie{sl}(N)$ is $\til \chi_{N}(y_1,y_2,y_3,y,q)$. 
\end{conj}

\parsec

It follows from the filtration in \eqref{eqn:fil1} that the large $N$ limit of $\til{\chi}_N$ is precisely the multiparticle supergravity index we computed in proposition~\ref{prop:sugraindex1}. 
Alternatively, we have the following direct proof of this fact. 

\begin{prop}
One has
\beqn
\chi_{sugra}(y_i, y, q) = \lim_{N \to \infty} \chi_N(y_i,y,q)
\eeqn
\end{prop}

\begin{proof}
It suffices to show that at the level of single particle indices one has
\beqn
f_{sugra}(y_i, y, q) = \lim_{N \to \infty} \til{f}_N(y_i,y,q) ,
\eeqn
where $\til{f}_N = \sum_{k = -1}^{N-2} g_k$. 

We will use the following identity 
\beqn
\sum_{k=0}^\infty q^{3k/2} \chi_{k}^{\lie{sl}(2)}(q^{-1/2}y) = \frac{1}{(1-q y)(1-q^2 y^{-1})} .
\eeqn
We will denote this expression by $S(y,q)$.

Using this identity one can directly see that the result reduces to observing that
\begin{multline}
\left(q^4 (y_1+y_2+y_3) + 1 - q^6 - q^2 (y_1^{-1} + y_2^{-1} + y_3^{-1})\right)S(y,q) - 1 + q^3= \\
\left(q^4(y_1+y_2+y_3)-q^2(y_1^{-1} + y_2^{-1} + y_3^{-1})+(1-q^3)(yq + y^{-1} q^2) \right) S(y,q) .
\end{multline}


%\begin{itemize}
%\item $q^4 \sum_{k=0}^\infty q^{3k/2} \chi_{k}^{\lie{sl}(2)}(q^{-1/2} y)(y_1+y_2+y_3) = \frac{q^4(y_1+y_2+y_3)}{(1-q y)(1-q^2 y^{-1})}$. 
%\end{itemize}

\end{proof}

As an immediate corollary we have the following result.
\begin{cor}
For any $N \geq 1$ one has
\beqn
\chi_{sugra}(y_i,y,q) = \til{\chi}_N(y_i,y,q) \mod q^{N+1} .
\eeqn
\end{cor}
\begin{proof}
This follows from observing that at the level of single particle states $f_N$ is of order $q^{N}$.
\end{proof}

\parsec

We can also apply the specialization $y=1, y_3=1$ to $\chi_N$.
\begin{prop}
Upon specializing $y=1,y_3=1$ (so that $y_1 y_2 = 1$) one has the following single particle index
\beqn
f_N (y_1,y_2, y_3=1,y=1,q) = \frac{q^2 + q^3 + \cdots + q^{N}}{1-q} 
\eeqn
\end{prop}
\begin{proof}
By induction it suffices to show that the specialization of the single particle local character $g_k$ of the factorization algebra $\clie_\bu(\cG^{(k)})$ is $q^{k+2} / (1-q)$. 
We have already seen this in the case $k=-1,0$, so it suffices to show this when $k \geq 1$.

First observe that the denominator becomes
\beqn
d(y_1,y_2,y_3=1,y=1,q) = (1-y_1 q)(1-y_2q) (1-q) .
\eeqn

Next, we observe that the numerator of $g_k (y_1,y_2,y_3=1,y=1,q)$ can be factored as
\begin{align*}
q^{3 + 3k/2} \left(q^{-(k+2)/2} + q^{-(k-2)/2} - q^{-k/2} (y_1+y_2) \right) 
& = q^{k+2} (1 + q^2 - q (y_1 + y_2)) \\
& = q^{k+2} (1 - y_1 q) (1-y_2 q) 
\end{align*}
where in the last line we have used $y_1 y_2 = 1$.
The result follows.
\end{proof}


