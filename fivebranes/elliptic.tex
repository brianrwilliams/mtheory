\documentclass[11pt]{amsart}

%\usepackage{../macros-master}
\usepackage{macros-fivebrane}

\begin{document}

\section{The two-dimensional ``free boson''}

\subsection{A model for punctured affine space}

Let $\sfA$ be the graded algebra generated by the degree zero variables $z_1,z_2,\lambda_1,\lambda_2$ and the degree one variable $\omega$ subject to the relation
\beqn
z_1 \lambda_1 + z_2 \lambda_2 = 1 .
\eeqn

Define the linear map 
\beqn
\dbar \colon \sfA \to \sfA[1] 
\eeqn
by the rules
\begin{itemize}
	\item $\dbar(z_i) = 0$.
	\item $\dbar(\lambda_i) = \ep_{ij} z_j \omega$ ,
\end{itemize}
and extend its action to the entire graded algbra $\sfA$ by the rule that it is a graded derivation.
Notice that this differential is well-defined since
\beqn
\dbar(z_1 \lambda_1 + z_2 \lambda_2) = z_1 z_2 \omega + z_2 (-z_1 \omega) = 0 .
\eeqn

The cohomology $H^\bu(\sfA, \dbar)$ is concentrated in degrees zero and one. 
In degree zero, the cohomology is simply polynomials in the variables $z_1,z_2$
\beqn
H^0(\sfA) \cong \C[z_1,z_2].
\eeqn
And in degree one, the cohomology can be identified with
\beqn
H^1(\sfA) \cong \C[\lambda_1,\lambda_2] \omega .
\eeqn

\subsection{The higher residue}

Consider the subalgebra $R$ generated by $z_i, \lambda_i$. 
Define the linear map
\beqn
\Res \colon \sfA \to \sfA/R \cong \C[-1]
\eeqn
which simply projects onto the line spanned by the element $\omega$.
Notice that this linear map is of cohomological degree~$-1$, consistent with the fact that $\omega$ carries degree~$+1$.
An immediate consequence of the cohomology computation above is the following.
\begin{lem}\label{lem:res}
For any $f \in \sfA$ one has $\Res(\dbar f) = 0$.
\end{lem}

\subsection{The two-cocycle}
Consider the following three-linear map 
\beqn
\phi_3 \colon \sfA \times \sfA \times \sfA \to \C
\eeqn
defined by
\beqn
\phi_3 (f,g,h) = \ep_{ij} \Res(f \del_{z_i} g \del_{z_j} h) .
\eeqn

\begin{lem}
The map $\phi_3$ is totally graded skew symmetric and determines a total degree two cocycle on the abelian dg Lie algebra $\sfA$. 
\end{lem}

As a consequence of this lemma, we see that $\phi_3$ defines a central extension of the abelian dg Lie algebra $\sfA$ 
\beqn
\C \to \sfH \to \sfA .
\eeqn

A particular model for this dg Lie algebra as an $L_\infty$ algebra is the following. 
As a graded vector space $\sfA = \sfH \oplus \C$.
\begin{itemize} 
	\item The one-ary bracket $[-]_1$ is simlpy the differential $\dbar$ which we extend to $\sfH$ by declaring that it annihilate the central element. 
\item There are no two-ary brackets $[-,-]_2 = 0$. 
\item The three-ary bracket is given by 
\beqn
[f,g,h]_3 = \phi_3(f,g,h) = \ep_{ij} \Res(f \del_{z_i} g \del_{z_j} h) .
\eeqn
\end{itemize}

\subsection{Homotopy transfer}

Given any dg Lie algebra $(\lie{g}, \d, [-,-])$ its cohomology $H^\bu(\lie{g},\d)$ has the induced structure of a graded Lie algebra. 
Moreover, via homotopy transer, its cohomology can be given (a possibly inequivalent) structure of an $L_\infty$ algebra with the property that it is equivalent to the original dg Lie algebra. 

The cohomology dg Lie algebra $\sfH$ thus has a canonical $L_\infty$ structure. 

\begin{prop} 
The $L_\infty$ structure on 
\beqn
H^\bu(\sfH) \cong \C[z_i] \oplus \C[\lambda_i]\omega \oplus \C 
\eeqn
obtained by homotopy transfer has trivial $k$-ary operations for all $k \ne 3$.
When $k=3$ the operation is simply $[f,g,h]_3 = \phi_3(f,g,h)$.
\end{prop}

In this proposition, we recall that by Lemma \ref{lem:res} that the residue map descends to a map in $\dbar$-cohomology.

\subsection{Fock module}

Let $\sfA_- = H^1(\sfA)$ and let $\sfA_+ \subset \sfA$ be commutative dg algebra given by the kernel of the projection 
\beqn
\sfA \to \sfA_- . 
\eeqn
Notice that the cohomology of $\sfA_+$ is simply $H^0(\sfA) = \C[z_1,z_2]$. 

\begin{dfn} 
For $\mu,k \in \C$ define the derived Fock module to be 
\beqn
U(\sfH) \otimes_{U(\sfA_+ \oplus \C\cdot K)} \C(\mu,k) .
\eeqn
Here $\C(\mu,k)$ is the $\sfA_+ \oplus \C \cdot K$-module where $1\in \sfA_+$ acts by $\mu$ and $K$ acts by $k$. 
\end{dfn}

\section{The higher `free fermion'}


\subsection{The underlying dg Lie algebra}

Let $\sfA$ be the commutative dg algebra as defined in the previous section. 
Consider the commutative dg algebra $\sfA \oplus \sfA[1]$ which is a square-zero extension of the commutative dg algebra $\sfA$. As a graded vector space it is concentrated in degrees $-1,0,1$. 
We will denote elements of this dg Lie algebra as $(\gamma,\beta)$ according to the direct sum decomposition. 

Define the following bilinear functional $\phi_2$ on $\sfA \oplus \sfA[1]$ by the formula
\beqn
\phi_2(\gamma,\beta) = \phi_2(\beta,\gamma) = \Res(\gamma \beta) .
\eeqn

\begin{lem}
The bilinear functional $\phi_2$ is graded skew symmetric and hence determines a two-cocycle on the abelian dg Lie algebra $\sfA \oplus \sfA[1]$ .
\end{lem} 

From the lemma, we see that $\phi_2$ determines a central extension of dg Lie algebras which we denote by
\beqn
\C \to \oint \sfF \to \sfA \oplus \sfA[1] .
\eeqn

\subsection{Vacuum module} 

Consider the subalgebra
\beqn
\sfA_+ \oplus \sfA_+[1] \oplus \C \cdot K \subset \oint \sfF .
\eeqn
Let $\C_1$ be the module for this subalgebra where $\sfA_+,\sfA_+[1]$ act trivially and the central element $K$ acts by the unit.  

\begin{dfn}
Let $\sfF$ be the induced dg module 
\beqn
U(\oint \sfF) \otimes_{U(\sfA_+ \oplus \sfA_+[1] \oplus \C \cdot K)} \C_1 .
\eeqn
\end{dfn}

Let $L_{0,j}$ be the generator for the action by the holomorphic vector fields $z_j\del_{z_j}$. 
Notice that the original dg Lie algebra $\oint \sfF$ has a symmetry by the group $U(1)$ which acts on $\gamma$ with weight $+1$ and $\beta$ with weight $-1$. 
Let $J_0$ be the infinitesimal generator for this action. 
We can then consider the character 
\beqn
\Tr_{\sfF} \left(p^{L_{0,1}} q^{L_{0,2}} u^{J_0}\right) 
\eeqn
where the trace denotes super trace. 

\begin{prop}
As a series in $p,q,u$ one has
\beqn
\Tr_{\sfF} \left(p^{L_{0,1}} q^{L_{0,2}} u^{J_0}\right) = \Gamma_{ell}(p,q;u)
\eeqn
where $\Gamma_{ell}(p,q;u)$ is the elliptic Gamma function.
\end{prop}

\subsection{'Boson-fermion' correspondence} 

Define an element $J \in \oint \sfF$ by the formula
\beqn
J = \sum_{n,m} \beta[n,m] \gamma[-n-1,-m-1] 
\eeqn

\brian{Some problems with this. First, these elements are in the cohomology of $\oint \sfF$.
Next, I think we need to do some normal ordering.}
\end{document}

