\documentclass[11pt]{amsart}

%\usepackage{../macros-master}
\usepackage{macros-fivebrane}

\begin{document}

\section{Twisted eleven-dimensional supergravity}

\subsection{A model for twisted supergravity}

In \cite{SWspinor} a complete description of the free limit of the holomorphic twist of eleven-dimensional supergravity is given.
Within the BV formalism, the theory is only $\Z/2$ graded.
In \cite{RSW} we have given a proposal for the minimal twist of fully interacting eleven-dimensional supergravity as a $\Z/2$ graded interacting BV theory.
We recall this description here.

The eleven-dimensional theory exists on any manifold of the form $X \times S$ where $X$ is a Calabi--Yau five-fold and $S$ is a real oriented smooth one-dimensional manifold. 
We opt to describe solutions to the equations of motion as Maurer--Cartan elements of a certain sheaf of ($\Z/2$-graded) $L_\infty$ algebras on $X \times S$.
The additional data of a theory in the BV formalism is a graded skew-symmetric pairing on this sheaf.

%\parsec[s:e510]
%
%Let $\Hat{D}^5$ denote the formal five-disk whose algebra of functions is 
%\[
%\cO(\Hat{D}^5) = \C[[z_1,\ldots,z_5]] .
%\]

\parsec[s:Lsugra]

In this paper $\Z/2$ graded $L_\infty$ algebra means a $\Z/2$ graded vector space~$L$ equipped with a collection of operations $\{[-]_k\}_{k = 1,2,\ldots}$ where 
\[
\ell_k \colon L^{\times k} \to \Pi^k L 
\]
are multilinear maps which satisfy the ordinary higher Jacobi relations defining an $L_\infty$ algebra \cite{??}. 
Here, $\Pi^k L = L$ if $k$ is even and $\Pi^k L = \Pi L$ if $k$ is odd. 
Thus, $\ell_k$ is an even operation if $k$ is even and an odd operation if $k$ is odd. 
\footnote{This is not to be confused with a super $L_\infty$ algebra which is an $L_\infty$ algebra internal to the category of super vector spaces.
A $\Z/2$ graded $L_\infty$ algebra is simply what one gets when they apply the forgetful function from $\Z$ graded vector spaces to $\Z/2$ graded vector spaces.}

Let $L_X$ be the following sheaf of $\Z/2$ graded $L_\infty$ algebras on the Calabi--Yau five-fold $X$.
As a sheaf of super vector spaces, $L_X$ is given as the holomorphic sections of a super vector bundle. 
The even part of this super vector bundle is
\[
\T_X \oplus \C_X 
\]
where $\T_X$ is the holomorphic tangent bundle and $\C_X$ is the trivial bundle.
We will denote even sections by $(\mu, \beta)$ according to the decomposition. 
The odd part is 
\[
\C_X \oplus \T^*_X .
\]
We will denote odd elements by $(\nu, \gamma)$ according to the decomposition. 

The $L_\infty$ structure is described as follows. 
First, $\d = [-]_1$, the differential, is simply given by 
\begin{align*}
\d \beta & = \del \beta \in \Omega^1_X \subset L_{X,-} \\
\d \mu & = \div \mu \in \cO_X \subset L_{X, -}
\end{align*}
and $\d \gamma = \d \nu = 0$. 
For $k \geq 2$ the general formula for the $k$-ary brackets is 
\begin{align*}
[\nu_1, \ldots, \nu_{k-2}, \mu_1,\mu_2]_{k} & = \div(\nu_1 \cdots \nu_k \mu_1 \wedge \mu_2) \\
[\nu_1,\ldots, \nu_{k-3}, \mu_1,\mu_2,\gamma]_k & = \nu_1 \cdots \nu_{k-3} (\mu \wedge \mu') \vee \del \gamma .\\
[\nu_1,\ldots,\nu_{k-2}, \mu, \gamma]_k & = \nu_1 \cdots \nu_{k-2} \mu \vee \del \gamma .
\end{align*}
In \cite{RSW} it is shown that this endows $L_{X}$ with the structure of a sheaf of $\Z/2$ graded $L_\infty$ algebras.
There is a sub sheaf given by the sections $\mu$ and $\nu$ which is $L_\infty$-equivalent to the sheaf of holomorphic divergence-free vector fields on $X$. 

As all operations above are given in terms of holomorphic polydifferential operators, the $L_\infty$ algebra structure above induces a $\Z/2$ graded $L_\infty$ algebra structure on the Dolbeault resolution $\Omega^{0,\bu}(X,L_X)$ of the holomorphic vector bundle $L_X$.

Finally, to obtain a local $L_\infty$ algebra on $X \times S$, where $S$ denotes an arbitrary real oriented smooth one-manifold, we simply tensor with the de Rham complex along~$S$.
We denote by $\cL_{sugra}$ the following local $\Z/2$ graded $L_\infty$ algebra on $X \times S$
\beqn
\cL_{sugra} \define \Omega^\bu(S) \hotimes \Omega^{0,\bu}(X,L_X) .
\eeqn

\parsec[s:rsw]
It is shown in \cite{RSW} that $\cL_{sugra}$ is equipped with a non-degenerate odd pairing which is compatible with the $L_\infty$ structure above. 
All together, this data prescribes the structure of a $\Z/2$ graded theory in the BV formalism. 
On $X \times S = \C^5 \times \R$ it is furthermore argued in \cite{RSW} that this BV theory is equivalent to the holomorphic (or minimal) twist of perturbative eleven-dimensional supergravity. 

\parsec[s:sugraobs]

In the BV formalism, the sheaf of fields of the eleven-dimensional theory is $\Pi \cL_{sugra}$. 
As a graded vector space, observables on an open set $U \subset X \times S$ are thus $\cO(\cL_{sugra}(U)) = \Sym (\Pi \cL_{sugra})^\vee$.
The differential is encoded by the $L_\infty$ structure which is exactly the Chevalley--Eilenberg differential. 
In other words, as a $\Z/2$ graded commutative dg algebra, the classical observables of the eleven-dimensional theory supported on an open set $U \subset X \times S$ is 
\[
\clie^\bu\left(\cL_{sugra}(U)\right) .
\]


\subsection{Twisted fivebranes} 

%\subsection{Twisted AdS backgrounds}

Six-dimensional $\cN=(2,0)$ supersymmetry admits two inequivalent classes of twists characterized by the number of directions which are left invariant:
\begin{itemize}
\item 
The holomorphic, or minimal, twist. 
This twist leaves three real directions invariant. 
The holomorphic twist of any $\cN=(2,0)$ theory can be defined on any complex three-fold $Z$ equipped with a square-root of its canonical bundle. 
\item 
The non-minimal twist. 
This twist leaves five real directions invariant.
The non-minimal twist of any $\cN=(2,0)$ theory can be defined on a six-manifold of the form $M^4 \times \Sigma$ where $M^4$ is a smooth four-manifold and $\Sigma$ is a Riemann surface. 
\end{itemize}

By standard arguments, the theory on a stack of $N$ fivebranes in flat space is a theory with $\cN=(2,0)$ supersymmetry. 
In later sections we provide arguments using the technology of twisted holography developed by Costello, Gaiotto, Li, and Paquette \cite{??} to provide a conjectural description of the twist of the theory for general $N > 1$. 
An explicit characterization of the holomorphic and non-minimal twists of a single fivebrane theory has been given in \cite{SWtensor}. 
We briefly recall this description. 

\parsec[s:single]

The holomorphic twist of the fivebrane theory is defined on any complex three-fold $Z$ (which is not necessarily equipped with a Calabi--Yau structure).
In the particular twist we will use we must assume, however, that $Z$ is equipped with a square-root of the canonical bundle $K_Z^{1/2}$. 

The theory is $\Z \times \Z/2$ graded but at first pass we recall its totalization as a $\Z/2$ graded theory. 
There are four fundamental fields $(\alpha, \omega, \phi_1,\phi_2)$ which consist of even Dolbeault forms of type $(2,1)$ and $(3,0)$ on $Z$:
\[
\alpha \in \Omega^{2,1}(Z), \quad \omega \in \Omega^{3,0}(Z),
\]
and a pair of even sections:
\[
(\phi_1,\phi_2) \in \Omega^0(Z , K^{1/2}_Z) \otimes \C^2 .
\]
%where $R \cong \CC^2$ is the fundamental $Sp(1)$ representation. 
The equations of motion read
\beqn
\label{eqn:eom}
\begin{split}
\del \alpha + \dbar \omega & = 0 \\
\dbar \alpha = \dbar \phi_i & = 0 .
\end{split}
\eeqn

There are gauge symmetries for the fields $\mu, \Omega$ determined by a ghost $b$ which is a Dolbeault form of type $(2,0)$ which acts simply by
\beqn
\label{eqn:ghost}
\begin{split}
\mu & \mapsto \mu + \dbar b  \\
\Omega & \mapsto \Omega + \del b .
\end{split}
\eeqn

This theory notoriously does not admit a Lagrangian description. 
However, it can still be put in a degenerate form of the BV formalism where the space of fields above is equipped with an odd Poisson bivector \cite{SWtensor}.
We describe this Poisson bivector explicitly

In the (degenerate) BV formalism, this free theory can be packaged into the data of an abelian $\Z/2$ graded local Lie algebra on $Z$ which is of the form 
\[
\cL_{single} = \Omega^{\leq 1, \bu}(Z) \oplus \Pi \Omega^{0,\bu}(Z, K^{1/2}_Z) \otimes \C^2 .
\]
The full space of BV fields is simply $\Pi \cL_{single}$. 

\parsec[s:singleobs]

The factorization algebra of classical observables of the holomorphic twist of the single fivebrane theory assigns to an open set $U \subset Z$
the cochain complex
\[
\clie^\bu\left(\cL_{single}(U)\right) .
\]

\parsec[s:coupling]

The holographic description of the ${\rm AdS}_7 \times S^4$ background in eleven-dimensional supergravity is obtained by backreacting a number of fivebranes \brian{references}. 
In \cite{RSW} we gave a description of the twisted avatar of this background and described how it arises by backreacting a number of holomorphically twisted fivebranes. 
We briefly recall the relevant setup here. 

As above, let $Z$ be a three-fold equipped with $K^{1/2}_Z$.
Consider the rank two vector bundle $K^{1/2}_Z \otimes \C^2$ on $Z$. 
Via the linear holomorphic symplectic form on $\C^2$, the total space of this vector bundle ${\rm Tot}(K^{1/2}_Z \otimes \C^2)$ is equipped with a Calabi--Yau structure. 
We consider the eleven-dimensional theory on 
\[
\R \times {\rm Tot}(K^{1/2}_Z) 
\]
with a stack of fivebranes supported at $0 \in \R$ and along the zero section in the total space of the vector bundle
\[
0 \times Z \subset \R \times {\rm Tot}(K^{1/2}_Z \otimes \C^2)  .
\]
Let us denote $X = {\rm Tot}(K^{1/2}_Z \otimes \C^2)$. 

It was argued in \cite{RSW} that to leading order the coupling of a stack of twisted fivebranes to the eleven-dimensional theory is given by the nonlocal interaction 
\beqn\label{eqn:br1}
I_{M5} = N\int_{Z} \div^{-1}\mu \vee \Omega +\cdots 
\eeqn
where $\mu \in \Omega^0 (\R) \hotimes \PV^{1,3}(X)$ is a component of a field in the eleven-dimensional theory which satisfies $\div \mu = 0$. 

\subsection{Supergravity at the boundary of AdS}

Suppose we are in the situation in the last section where we place the eleven-dimensional theory on $\R \times X$ where $X = {\rm Tot}(K^{1/2}_Z \otimes \C^2)$. 
We are furthermore placing some number of fivebranes along the locus $Z$ cut out by the zero section. 
If we remove this locus we obtain the following manifold 
\[
\R \times X - \left(0 \times Z\right) \simeq Z \times \left({\rm Tot}(N_Z) - 0\right) 
\]
where on the right hand side ${\rm Tot}(N_Z) - 0$ is the total space of the normal bundle to $Z$ in $\R \times X$ minus the zero section.

\parsec[s:flat]

%In the simple case where $Z = \C^3$ and we identify the total space of $K^{1/2}_Z \otimes \C^2$ with $\C^5$ then the manifold obtained by removing the locus of the brane is homeomorphic to
%\[
%\C^3 \times (\R \times \C^2 - 0) .
%\]
%
%Let $\pi \colon \R \times \C^5 - (0 \times \C^3) \to \C^3 \times \R_{>0}$ be the projection map whose fibers are homeomorphic to the sphere $S^4$ which links the location of the fivebranes.
%We restrict the factorization algebra of the eleven dimensional theory $\Obs_{sugra}$ to the open set obtained by removing the locus of the brane.

In the simple case that $Z = \C^3$ and we identify the total space of $K^{1/2}_Z \otimes \C^2$ with $\C^5$ there is a more direct construction of the factorization algebra $\Obs_{sugra}|_{Z}$. 

Let $\pi \colon \R \times \C^5 \to \C^3$ be the projection map.
Then, via $\pi$ we can pushforward the factorization algebra associated to the eleven-dimensional theory to obtain a factorization algebra
\[
\pi_* \Obs_{sugra} 
\]
on $\C^3$.
This factorization algebra is not the factorization algebra associated to an ordinary sort of field theory on $\C^3$. 
Nevertheless there is a subfactorization algebra which admits a natural grading so that each filtered component can be understood as such.

For any open set $U \subset \C^3$ we can consider the following $\C^\times$ action on the fields of the eleven-dimensional theory supported on $\R \times \C^2 \times U$ defined by
\begin{itemize}
\item On the fields $\mu(t;w,z) \in \Omega^\bu(\R) \otimes \PV^{1,\bu}(\C^2 \times U)$ the action is
\[
\lambda \cdot \mu(t;w,z) = \mu(\lambda t;\lambda w , z).
\]
\item On the fields $\nu(t;w,z) \in \Omega^\bu(\R) \otimes \PV^{0,\bu}(\C^2 \times U)$ the action is
\[
\lambda \cdot \nu(t;w,z) = \nu(\lambda t;\lambda w , z).
\]
\item On the fields $\beta(t;w,z) \in \Omega^\bu(\R) \otimes \Omega^{0,\bu}(\C^2 \times U)$ the action is
\[
\lambda \cdot \beta(t;w,z) = \lambda^{-1} \beta(\lambda t;\lambda w , z).
\]
\item On the fields $\mu(t;w,z) \in \Omega^\bu(\R) \otimes \Omega^{1,\bu}(\C^2 \times U)$ the action is
\[
\lambda \cdot \gamma(t;w,z) = \lambda^{-1} \gamma(\lambda t;\lambda w , z).
\]
\end{itemize}

For each $n \in \ZZ$ and open set $U \subset \C^3$, let 
\[
\pi_* \cL_{sugra}(U)^{(n)} \subset \cL_{sugra}(\R \times \C^2 \times U)
\]
be the weight $n$ eigenspace with respect to this $\C^\times$ action. 
The $\C^\times$ action is compatible with the $\Z/2$ graded $L_\infty$ structure on $\cL_{sugra}$. 
Since the $n$th eigenspace is trivial when $n < -1$, we see that the product
\[
(\Bar{\pi}_* \cL_{sugra})(U) \define \prod_{n \geq -1} (\pi_*\cL_{sugra})(U)^{(n)}
\]
is equipped with the structure of a $\Z/2$ graded $L_\infty$ algebra.
In this way, the assignment 
\[
\Bar{\pi}_* \cL_{sugra} \colon U \mapsto (\Bar{\pi}_* \cL_{sugra})(U) 
\]
defines a sheaf of $\Z/2$ graded $L_\infty$ algebras on $\C^3$. 

\begin{prop}
For each $n$, the sheaf of cochain complexes $U \mapsto \pi_* \cL_{sugra}(U)^{(n)}$ is quasi-isomorphic to one of the form
\[
\Omega^{0,\bu}(U, \cV^{(n)})
\]
for some finite rank super holomorphic vector bundle $\cV^{(n)}$ on $\C^3$. 
In particular, this endows $\Bar{\pi}_* \cL_{sugra}$ with the structure of a pro-vector bundle on $\C^3$. 
\end{prop}

Now, since $\Bar{\pi}_* \cL_{sugra}$ is a pro-vector bundle with a compatible $L_\infty$ structure, the assignment 
\[
U\subset \C^3 \mapsto \clie^\bu\left(\Bar{\pi}_* \cL_{sugra}(U)\right) 
\]
has the structure of a factorization algebra on $\C^3$ which we denote by $\Bar{\pi}_* \Obs_{sugra}$. 


\begin{prop}
Let $Z$ be a Calabi--Yau three-fold and $\pi \colon Z \times \C^2 \times \R \to Z$ be the projection. 
There is a pro local Lie algebra $\cL_{\pi, sugra}$ on $Z$ such that:
\begin{itemize}
\item[(1)] there is a natural inclusion of of factorization algebras on $Z$
\[
\Bar{\pi}_* \Obs_{sugra} \hookrightarrow \pi_* \Obs_{sugra}
\]
which is dense at the level of cohomology. 
\item[(2)] there is a weight grading on $\cL_{\pi,sugra}$ which is concentrated in degrees $\geq -1$ and gives rise to a decomposition of vector bundles
\beqn\label{eqn:decomp3}
\Bar{\pi}_{*} \cL_{sugra} = \prod_{n \geq -1} \cV_{n} 
\eeqn
\item[(3)]
In weight zero, there is an equivalence of local Lie algebras on $Z$ 
\[
\cV_0 \simeq \cE(3|6)|Z .
\]
\end{itemize}
\end{prop}

%We want to argue that $\Obs_{sugra} |_{\C^3} \cong \Bar{\pi}_* \Obs_{sugra}$. 

\parsec

\begin{prop}
On flat space 
\end{prop}

\appendix

\subsection{Enhanced symmetry in a flat background}

\parsec[s:e36]

We recall the definition of the exceptional super Lie algebra $E(3|6)$. 
We follow \cite{Kac_class, KacRudakov}.

The even part of the super Lie algebra $E(3|6)$ is 
\[
\Vect(\Hat{D}^3) \oplus \lie{sl}(2) \otimes \cO(\Hat{D}^3) .
\]
The Lie bracket is determined by the obvious commutators and the action of vector fields on functions. 

The odd part is of the form
\[
\Omega^{-1/2}(\Hat{D}^3) \otimes \C^2 .
\]
where $\Omega^{-1/2}(\Hat{D}^3)$ is the space of formal power series sections of the bundle
\[
\T_{\Hat{D}^3}^{*} \otimes K_{\Hat{D}^3}^{-1/2}
\]
where $K_{\Hat{D}^3}^{-1/2}$ is the negative square-root of the canonical bundle on $\Hat{D}^3$. 
We will write a general element of this module as 
\[
\alpha = \sum_i g^i \d z_i^{-1/2} 
\]
where $g^i \in \cO(\Hat{D}^3)$. 
The action by the even part of the super Lie algebra is by Lie derivative and by the fundamental $\lie{sl}(2)$ action on $\C^2$.
We will write $r \in \C^2$ for a vector in the fundamental representation.
Finally, there is a bracket 
\[
E(3|6)_- \times E(3|6)_- \to E(3|6)_+
\]
defined by
\[
[\alpha_1 \otimes r_1, \alpha_2 \otimes r_2] = - (\alpha_1 \wedge \alpha_2) \otimes (r_1 \wedge r_2) - (\d \alpha_1 \otimes \alpha_2 + \alpha_1 \otimes \d \alpha_2) \otimes (r_1 \odot r_2) .
\]
%The first term in the above equation arises from the composition 
%\[
%(\Hat{\Omega}_3^{-1/2} \otimes \Hat{\Omega}_3^{-1/2}) \otimes (R \otimes R) \xto{\wedge \otimes \wedge} \Hat{\Omega}_3^{-1} \otimes \wedge^2 R \cong \fw_3 .
%\] 
%The second term uses the composition 
%\[
%(\Hat{\Omega}_3^{-1/2} \otimes \Hat{\Omega}_3^{-1/2}) \otimes (R \otimes R) \xto{(\d \otimes \id) \otimes \odot} (\Hat{\Omega}_3^{1/2} \otimes \Hat{\Omega}_3^{-1/2}) \otimes \Sym^2 (R) \cong \lie{sl}(2) \otimes \Hat{\cO}_3 . 
%\] 

\brian{suca enhancement following \cite{SWsuca6d}}

\parsec[s:weight]


\parsec[s:locallieE510]

We have recollected how the super Lie algebra $E(5|10)$ (and its central extension) decomposes as a module for the super Lie algebra $E(3|6)$. 
We now wish to lift this to a decomposition of factorization algebras on $\C^3$, or more generally, factorization algebras on any complex three-fold $Z$. 

We outline an approach to constructing the factorization algebras via enveloping factorization algebras of certain (infinite rank) local Lie algebras on the three-fold $X$. 
For another approach \brian{pointer}. 

We have already discussed the local Lie algebra $\cL_{sugra}$ on $X \times S$ where $X$ is a Calabi--Yau five-fold and $S$ is a smooth one-manifold. 
We take $X = Z \times \C^2$ and $S = \R$. 
and view $\Obs_{sugra} = \clie^\bu(\cL_{sugra})$ as a factorization algebra on $Z \times \C^2 \times \R$. 
Then, we can consider the pushforward of $\Obs_{sugra}$ along the map
\[
\pi \colon Z \times \C^2 \times \R \to Z 
\]
which leads to a factorization algebra $\pi_* \Obs_{sugra}$ on the three-fold $Z$. 

Since $\pi$ is not a proper map (the fibers are not compact) we cannot exhibit $\pi_*\Obs_{sugra}$ as the Chevalley--Eilenberg complex of some ordinary local Lie algebra on $Z$. 
Nevertheless we will find a pro local Lie algebra $\cL_{sugra,\pi}$ on $Z$ which is a particularly nice `small model' for this pushforward factorization algebra.

A local Lie algebra enhancement for $E(3|6)$ has been given in \cite{SWsuca6d} \brian{finish}







\section{Example} 

Consider the holomorphic local Lie algebra on $\C^n$
\[
\Omega^{0,\bu}(\C^n) 
\]
and corresponding factorization algebra $\cF = \clie^\bu(\Omega^{0,\bu}(\C^n))$. 
Now, we can restrict this to a factorization algebra on
\[
\C^n - 0 \subset \C^n
\]
that we denote by $\cF|_{\C^n - 0}$. 

Next, consider the radial projection
\[
r \colon \C^n - 0 \to \R .
\]
Notice that $r_* (\cF|_{\C^n - 0})$ will not be the observables of any ordinary local Lie algebra.
Nevertheless we can come up with a nice algebraic model.  

Let $A_n$ be the usual dg model for the formal punctured disk $\Hat{D}^n - 0$ as defined in \cite{FHK,GWkm}.

\begin{lem}
There is an injective map of factorization algebras on $\RR_{>0}$:
\[
\clie^\bu(\Omega^\bu(\RR_{>0}) \otimes A_n) \to r_* (\cF|_{\C^n - 0}) 
\]
which is dense in cohomology.
\end{lem}

This tells us that the correct `fields' of the dimensionally reduced theory on $\RR_{>0}$ is $\Omega^\bu(\R_{>0}) \otimes A_n$. 


\end{document}






