%\documentclass[11pt]{amsart}%
%
%%\usepackage{../macros-master}
%\usepackage{macros-fivebrane}

%
%\begin{document}
%

\newcommand{\thfd}{\Bar{D}}
\newcommand{\thfdel}{D} 

\section{Twisted eleven-dimensional supergravity}

\subsection{A model for minimally twisted supergravity}

Eleven-dimensional supergravity is the unique supersymmetric theory in eleven dimensions.
It plays an important role as the low energy limit of $M$ theory.

Recently in \cite{CLsugra}, Costello and Li gave a rigorous definition of twisted supergravity.
In any theory of supergravity, the action of supersymmetry is gauged. When one treats the theory in the BRST formalism, the fields then include a collection of bosonic ghosts which homologically enforce invariance under the odd gauge symmetries.

By definition, twisted supergravity is supergravity in the background where a bosonic ghost for supersymmetry takes a nonzero value $Q$, where $Q$ is a twisting supercharge.

Eleven-dimensional supersymmetry admits two inequivalent classes of twists characterized by the number of directions which are left invariant:
\begin{itemize}
\item 
The minimal twist. 
This twist leaves six real directions invariant and is preserved by the subgroup $SU(5)$ of the Lorentz group. 
\item 
The non-minimal twist. 
This twist leaves nine real directions invariant and is preserved by the subgroup $SU(2) \times G_2$ of the Lorentz group. 
\end{itemize}

In this paper we will focus primarily on the minimal twist. 
In \cite{SWspinor} a complete description of the free limit of this twist of eleven-dimensional supergravity is given.
Within the BV formalism, the theory is only $\Z/2$ graded, a point we will elaborate on further soon. 
In \cite{RSW} we have given a proposal for the minimal twist of fully interacting eleven-dimensional supergravity as a $\Z/2$ graded interacting BV theory in a background with global symmetry group $SU(5)$.
Globally, the eleven-dimensional theory exists on any manifold of the form 
\[
S \times X 
\] 
where $X$ is a Calabi--Yau fivefold and $S$ is a real oriented smooth one-dimensional manifold.

In the first few sections we describe the fields and equations of motion of the eleven-dimensional theory. 

%\parsec[s:e510]
%
%Let $\Hat{D}^5$ denote the formal five-disk whose algebra of functions is 
%\[
%\cO(\Hat{D}^5) = \C[[z_1,\ldots,z_5]] .
%\]

\parsec[s:sugrafields]

To begin this section we highlight some of the important features of the eleven-dimensional model considered in \cite{RSW}. 
The theory utilizes the complex structure on the Calabi--Yau fivefold $X$.
We let $\T_X$ stand for the holomorphic tangent bundle of $X$ and $\Bar{\T}_X$ its complex conjugate.
Thus, the complexified tangent bundle of the eleven-manifold $S \times X$ decomposes as
\[
\T_S \oplus \T_X \oplus \Bar{\T}_X 
\]
where $\T_S$ is the complexified tangent bundle of the smooth one-manifold $S$.
Locally, we choose coordinates $(t;z,\zbar)$ where $t$ is a smooth coordinate for $S$ and $z$ is a holomorphic coordinate for $X$. 

Before giving a rigorous description of the fields and equations of motion, we focus on bosonic fields in our model which have an obvious analog in the untwisted theory. 
One of the primary fields of eleven-dimensional supergravity describes deformations of the metric $g$.
Since we work in a background which has $SU(5)$ holonomy we can take the background metric on $S \times X$ to be the product of the flat metric on $S$ with a Calabi--Yau metric $g_{CY}$ on $X$. 

Deformations of the background metric which survive the twist can be identified with Beltrami differentials, that is, sections of 
\[
\Bar{\T}_X^* \otimes \T_X \cong \Bar{\T}^*_X \otimes \Bar{\T}_X^* 
\]
which we will denote by $\mu$. 
In local holomorphic coordinates $\mu$ admits a description like
\[
\mu = \mu^{\Bar{i}}_j (t;z,\zbar) \d \zbar_{\Bar{i}} \otimes \del_{z_i} .
\]
Part of the equations of motion for $\mu$ dictate that it be locally constant along~$S$.
Ordinarily, for $\mu$ to describe a deformation of complex structure in the $X$ direction we would require that it satisfy the Muarer--Cartan equation.
For twisted supergravity, however, we find a modification of this equation which involves the twisted analog of another familiar field.

One of the other bosonic fields of eleven-dimensional supergravity is a three-form that we denote by~$\sf C$. 
On $S \times X$ the space of complex-valued three-forms decomposes into de Rham--Dolbeault forms as in
\[
\oplus_{k+p+q=3} \Omega^k(S) \otimes \Omega^{p,q}(X) .
\]
We denote the homogenous pieces as the space of $(k;p,q)$ forms, these admit local presentations as
\[
\sum_{P,Q} f_{PQ} (t;z,\zbar) \d z_P \d \zbar_Q + \sum_{P',Q'} g_{P'Q'}(t;z,\zbar) \d t \d z_{P'}\d \zbar_{Q'} 
\]
where the first sum is over tuples $P=(p_1,p_2,p_3),Q=(q_1,q_2,q_3)$ such that $\sum_i (p_i + q_i) = 3$, $0 \leq p_i,q_i \leq 3$.
The second sum is over tuples $P',Q'$ such that $\sum_i (p_i'+q_i') = 2$. 

In the minimal twist only forms with $p_i ,p_i'\leq 1$ survive. 
Thus, the components of the three-form which survive live in
\[
\left(\oplus_{k+q = 3} \Omega^k(S) \otimes \Omega^{0,q}(X)\right) \oplus 
\left(\oplus_{k+q = 2} \Omega^k(S) \otimes \Omega^{1,q}(X)\right) .
\]
We denote a general (non-homogenous) element in the first summand by $\beta$ and an element in the second summand by $\gamma$.
When we want to be specific by the form type we use the notation $\beta^{k;q}$ for the $(k;0,q)$ component of the three-form and $\gamma^{k;q}$ for the $(k;1,q)$ component of the three-form.

Along $S$, $\beta$ and $\gamma$ behave like abelian gauge fields.
Thus, locally we can assume that the $\Omega^1(S)$ components are exact and the $\Omega^0(S)$ components are locally constant.


\parsec[s:Lsugra]

We move on to give a complete description of our eleven-dimensional model within the Batalin--Vilkovisky (BV) formalism.
This amounts to introducing antifields and antighosts to the fields and ghosts that we have just described.
One of the key structures is the graded skew-symmetric pairing between fields and antifields (ghosts and antighosts).
Once such a pairing $\omega_{BV}$ is introduced, one can opt to describe solutions to the equations of motion as Maurer--Cartan elements of a certain sheaf of ($\Z/2$-graded) $L_\infty$ algebras on $X \times S$.

An $L_\infty$ algebra is a graded vector space~$L=\oplus_{p\in \Z} L^p$ equipped with a collection of operations $\{[-]_k\}_{k = 1,2,\ldots}$ where 
\[
[-]_k \colon L^{\times k} \to L[2-k] 
\]
are multilinear maps which satisfy the ordinary higher Jacobi relations defining an $L_\infty$ algebra \cite{??}. 
The shift $[2-k]$ means that $[-]_k$ is of cohomological degree $k-2$. 

The BV fields arise as the sections of a sheaf of $L_\infty$ algebras $\cL$ over the spacetime manifold $M$.
There is an overall shift in the relationship between the cohomological grading of $\cL$ and the ghost number in physics terminology. 
In cohomological degree zero sit the ghosts, in cohomological degree zero sit the fields, etc..
In this sense it is more accurate to refer to the BV fields as sections of the shift of the sheaf of $L_\infty$ algebras $\cL[1]$. 
Together with the skew symmetric pairing $\omega_{BV}$ on the fields structure maps $[-]_k$ organize together to define the full BV action which takes the general form
\[
S(\Phi) = \sum_{k \geq 1} \omega_{BV} \left(\Phi , [\Phi,\ldots,\Phi]_{k}\right) .
\]

One feature of our proposal for the minimal twist of eleven-dimensional supergravity is that it only carries an overall $\Z/2$ grading. 
This $\Z/2$ grading totalizes the original ghost grading (which is by the group $\Z$) and the fermion number which is present in the untwisted theory.
We thus make use of $\Z/2$ graded versions of the usual BV formalism.
In this paper $\Z/2$ graded $L_\infty$ algebra means that we just have a $\Z/2$ graded vector space.
There are operations $\{[-]_k\}_{k = 1,2,\ldots}$ which satisfy the same higher Jacobi identities.
The operation $[-]_k$ is even if $k$ is even and odd if $k$ is odd.\footnote{This is not to be confused with a super $L_\infty$ algebra which is an $L_\infty$ algebra internal to the category of super vector spaces.
A $\Z/2$ graded $L_\infty$ algebra is simply what one gets when they apply the forgetful function from $\Z$ graded vector spaces to $\Z/2$ graded vector spaces.}

To build the $L_\infty$ algebra associated to the eleven-dimensional theory, we first introduce a sheaf of $\Z/2$ graded $L_\infty$ algebras on the Calabi--Yau fivefold $X$.
As a sheaf of super vector spaces, it will be given as the holomorphic sections of a super vector bundle that we denote by $L_X$. 
The even part of this super vector bundle is
\[
\T_X \oplus \C_X 
\]
where $\T_X$ is the holomorphic tangent bundle and $\C_X$ is the trivial bundle.
We will denote even sections by $(\mu, \beta)$ according to the decomposition. 
The odd part is 
\[
\C_X \oplus \T^*_X .
\]
We will denote odd elements by $(\nu, \gamma)$ according to the decomposition. 

The $L_\infty$ structure on the sheaf of holomorphic sections of $L_X$ is described as follows. 
First, $\d = [-]_1$, the differential, is simply given by 
\begin{align*}
\d \beta & = \del \beta \in \Omega^1_X \subset L_{X,-} \\
\d \mu & = \div \mu \in \cO_X \subset L_{X, -}
\end{align*}
and $\d \gamma = \d \nu = 0$. 
For $k \geq 2$ the general formula for the $k$-ary brackets is 
\begin{align*}
[\nu_1, \ldots, \nu_{k-2}, \mu_1,\mu_2]_{k} & = \div(\nu_1 \cdots \nu_k \mu_1 \wedge \mu_2) \\
[\nu_1,\ldots, \nu_{k-3}, \mu_1,\mu_2,\gamma]_k & = \nu_1 \cdots \nu_{k-3} (\mu \wedge \mu') \vee \del \gamma .\\
[\nu_1,\ldots,\nu_{k-2}, \mu, \gamma]_k & = \nu_1 \cdots \nu_{k-2} \mu \vee \del \gamma .
\end{align*}
In \cite{RSW} it is shown that this endows $L_{X}$ with the structure of a sheaf of $\Z/2$ graded $L_\infty$ algebras.
There is a sub sheaf given by the sections $\mu$ and $\nu$ which is $L_\infty$-equivalent to the sheaf of holomorphic divergence-free vector fields on $X$. 

As all operations above are given in terms of holomorphic polydifferential operators, the $L_\infty$ algebra structure above induces a $\Z/2$ graded $L_\infty$ algebra structure on the Dolbeault resolution $\Omega^{0,\bu}(X,L_X)$ of the holomorphic vector bundle $L_X$.

Finally, to obtain a local $L_\infty$ algebra on $X \times S$, where $S$ denotes an arbitrary real oriented smooth one-manifold, we simply tensor with the de Rham complex along~$S$.
We denote by $\cL_{sugra}$ the following local $\Z/2$ graded $L_\infty$ algebra on $X \times S$
\beqn
\cL_{sugra} \define \Omega^\bu(S) \hotimes \Omega^{0,\bu}(X,L_X) .
\eeqn

\parsec[s:rsw]
It is shown in \cite{RSW} that $\cL_{sugra}$ is equipped with a non-degenerate odd pairing which is compatible with the $L_\infty$ structure above. 
All together, this data prescribes the structure of a $\Z/2$ graded theory in the BV formalism. 
On $X \times S = \C^5 \times \R$ it is furthermore argued in \cite{RSW} that this BV theory is equivalent to the holomorphic (or minimal) twist of perturbative eleven-dimensional supergravity. 

Notice that the Beltrami differential $\mu^{\Bar{i}}_j \d \zbar_{\Bar{i}} \otimes \del_z$ as well as the components of the three-form which survive the minimal twist sit inside as odd sections of this sheaf of $L_\infty$ algebras. 
\brian{elaborate gauge symmetries, write action, etc.}

\parsec[s:twistedsugra]

In \cite{CostelloM5,CostelloM2}, Costello introduced a program to analyze twists of $M$ theory, starting with an $\Omega$-deformation of the nonminimal twist of eleven-dimensional supergravity. Using a type IIA description, the theory was shown to admit a description in terms of a holomorphic-topological gauge theory in five dimensions with a well-behaved quantization.

The five-dimensional theory is also a $\Z/2$-graded within the BV formalism and exists on any manifold of the form
\[
  S\times Y
\]

where $Y$ is a holomorphic-symplectic surface and $S$ is a real oriented smooth one-dimensional manifold.

The fields of the theory are given by a local dg-lie algebra on $S\times Y$ whose underlying sheaf of cochain complexes is
\beqn
\cL_{5d} \define \Omega^\bu(S) \hotimes \Omega^{0,\bu}(Y) .
\eeq

The sheaf of holomorphic functions on $Y$ has a natural lie bracket given by the holomorphic poisson bracket $\{-,-\}_{Y}$. This can be extended in a graded way to the entire Dolbeault complex; $\cL_{5d}$ is then the tensor product of a commutative algebra and a lie algebra so is canonically a local dg-lie algebra.

Wedging and integrating against the holomorphic volume form on $Y$ defines an odd pairing on $\cL_{5d}$; together with the above, this equips $\cL_{5d}$ with the structure of a $\Z/2$-graded BV theory.

The equations of motion of $\cL_{5d}$ describe locally-constant families of deformations of the fixed holomorphic Poisson structure on $Y$.

\subsection{Twisted fivebranes} 

%\subsection{Twisted AdS backgrounds}

Six-dimensional $\cN=(2,0)$ supersymmetry admits two inequivalent classes of twists characterized by the number of directions which are left invariant:
\begin{itemize}
\item 
The holomorphic, or minimal, twist. 
This twist leaves three real directions invariant. 
The holomorphic twist of any $\cN=(2,0)$ theory can be defined on any complex three-fold $Z$ equipped with a square-root of its canonical bundle. 
\item 
The non-minimal twist. 
This twist leaves five real directions invariant.
The non-minimal twist of any $\cN=(2,0)$ theory can be defined on a six-manifold of the form $M^4 \times \Sigma$ where $M^4$ is a smooth four-manifold and $\Sigma$ is a Riemann surface. 
\end{itemize}

By standard arguments, the theory on a stack of $N$ fivebranes in flat space is a theory with $\cN=(2,0)$ supersymmetry. 
In later sections we provide arguments using the technology of twisted holography developed by Costello, Gaiotto, Li, and Paquette \cite{??} to provide a conjectural description of the twist of the theory for general $N > 1$. 
An explicit characterization of the holomorphic and non-minimal twists of a single fivebrane theory has been given in \cite{SWtensor}. 
We briefly recall this description. 

\parsec[s:single]

The holomorphic twist of the fivebrane theory is defined on any complex three-fold $Z$ (which is not necessarily equipped with a Calabi--Yau structure).
In the particular twist we will use we must assume, however, that $Z$ is equipped with a square-root of the canonical bundle $K_Z^{1/2}$. 

The theory is $\Z \times \Z/2$ graded where $\Z$ is the ghost number (or cohomological degree) and $\Z/2$ is parity. 
There are four fundamental fields of ghost number zero $(\alpha, \omega, \phi_1,\phi_2)$ which consist of even Dolbeault forms of type $(2,1)$ and $(3,0)$ on $Z$:
\[
\alpha \in \Omega^{2,1}(Z), \quad \omega \in \Omega^{3,0}(Z),
\]
and a pair of odd $(0,1)$ forms twisted by the line bundle $K^{1/2}_Z$:
\[
(\phi_1,\phi_2) \in \Pi \Omega^{0,1}(Z , K^{1/2}_Z) \otimes \C^2 .
\]
%where $R \cong \CC^2$ is the fundamental $Sp(1)$ representation. 
The equations of motion read
\beqn
\label{eqn:eom}
\begin{split}
\del \alpha + \dbar \omega & = 0 \\
\dbar \alpha = \dbar \phi_i & = 0 .
\end{split}
\eeqn

There are gauge symmetries for the fields $\mu, \Omega$ determined by a ghost $b$ which is a Dolbeault form of type $(2,0)$ which acts simply by
\beqn
\label{eqn:ghost}
\begin{split}
\mu & \mapsto \mu + \dbar b  \\
\Omega & \mapsto \Omega + \del b .
\end{split}
\eeqn
There are also odd gauge symmetries for the odd fields $\phi_i$ given by
\beqn
\phi_i \mapsto \phi_i + \dbar \chi 
\eeqn
where $\chi \in \Omega^0(Z, K_{Z}^{1/2}) \otimes \C^2$. 

This theory notoriously does not admit a Lagrangian description. 
However, it can still be put in a degenerate form of the BV formalism where the space of fields above is equipped with an odd Poisson bivector \cite{SWtensor}.
We describe this Poisson bivector explicitly

In the (degenerate) BV formalism, this free theory can be packaged into the data of an abelian $\Z \times \Z/2$ graded local Lie algebra on $Z$ which is of the form 
\[
\cL_{single} = \Omega^{\geq 2, \bu}(Z) \oplus \Pi \Omega^{0,\bu}(Z, K^{1/2}_Z) \otimes \C^2 .
\]

\subsection{Twisted membranes}

The theory on a stack of $N$ membranes probing the origin in $\C^{4}/\Z_{k}$ is expected to be described by the ABJM theory - a 3d $\cN=6$ superconformal gauge theory with gauge group $U(N)\times U(N)$ at levels $\pm k$ respectively \cite{} For $k=1,2$ the theory witnesses an enhancement to $\cN=8$ due to a nonperturbative effect. \cite{}

Twists of $\cN=8$ gauge theories in 3d were classified in \cite{}. Moreover, the ABJM theories are examples of 3d $\cN=2$ Chern-Simons matter theories, whose twists were studied in \cite{}. Twists of such occur in two inequivalent classes, similarly labeled by the dimension of the subspace of translations given by the image of the twisting supercharge:

\begin{itemize}
  \item The minimal twist. This twist leaves 2 real directions invariant. The theory can be defined on a 3-manifold of the form $S\times \Sigma$ where $S$ is a smooth oriented one-manifold and $\Sigma$ is a Riemann surface equipped with a fourth-root of its canonical bundle.

  \item The nonminimal twists. There are two such twists, both of which leave 3 real directions invariant, and are permuted under 3d mirror symmetry.
\end{itemize}

\parsec

We briefly recall the description of the minimal twist of the ABJM theory for $N=1$.

\surya{fields, gauge symmetries, etc}

In the BV formalism, this free theory can be packaged into the data of an abelian $\Z$ graded local Lie algebra on $\Sigma$ which takes the form
\[
\Omega^{\bu}(S)\times \left (\Omega^{0,\bu}(\Sigma) \oplus \Pi \Omega^{0,\bu} (\Sigma )\right ) \otimes \C^{4}.
\]

\subsection{States in twisted supergravity}

In eleven-dimensional supergravity, the ${\rm AdS}_7 \times S^4$ and ${\rm AdS}_{4}\times S^{7}$ backgrounds are obtained by backreacting a number of fivebranes and membranes respectively in flat space \brian{references}.

In \cite{RSW} we gave descriptions of twisted versions of these backgrounds. We recall this construction, adapted to a slightly more global situation. We will consider the eleven-dimensional theory on 11-manifolds that arise as total spaces of vector bundles. Placing the theory in the backreacted geometry is a 3-step procedure:

\begin{itemize}
  \item Place the eleven dimensional theory on the complement of the zero section. To do so, we will wish to describe the complement of the zero-section in a way that facilitates natural operations on holomorphic-topological local $L_{\infty}$-algebras.

  \item Deform the theory on the complement of the zero section by a certain Maurer--Cartan element. 
  The Maurer--Cartan element is thought of as the flux sourced by branes wrapping the zero section.

  \item The complement of the zero section is a manifold with boundary---we impose a natural boundary condition.

\end{itemize}

\parsec[s:brkevin]

As a way to highlight the key aspects of the construction, we consider the simplified model of Costello's twisted $M$ theory on $\R \times \C \times C$, where $C$ is a Riemann surface equipped with a nonvanishing holomorphic one-form $\d z$, with some number of twisted `fivebranes' wrapping 
\[
0 \times 0 \times C \subset \R \times \C \times C .
\]
It will be convenient to view $\R \times \C \times C$ as the total space of the real rank two bundle $\R \oplus K_C$ over $C$. 
Denote by $(t,w)$ the fiber coordinates.
Removing the zero section and deforming by the backreaction associated to $C$ corresponds to the twisted version of the $S^2 \times AdS_3$ background. 
We proceed to describe the twisted version of states at the boundary of this version of $AdS$. 
We first proceed before turning on the backreaction.

If we choose a fiberwise partially hermitian metric on the bundle $\R \oplus K_C$ we obtain a projection ${\rm Tot}(\R \oplus K_C) \to \R_+ \times C$ which combines the fiberwise norm with the natural bundle projection.
Denote by $r$ a coordinate for the direction $\R_+$ normal to the brane and let $z \in C$ be a local holomorphic coordinate. 
The fibers of this projection away from $r = 0$ are all homeomorphic to the two-sphere.

Compactification along this projection results in a theory with infinitely many Kaluza--Klein modes roughly parametrized by the Cauchy--Riemann cohomology of the two sphere as a hypersurface in $\R \oplus K_C|_z \simeq \R \times \C$. 
We will elaborate on the appearance of this sort of cohomology momentarily.
The theory admits a natural `vacuum' boundary condition at $r=0$.
In local coordinates, these are fields $\alpha(t,z,w)$ on the complement to the brane which extend to regular functions along the brane.

The `supergravity states' are, by definition, fields which satisfy the linearized equations of motion and satisfy the vacuum boundary condition except at a single point. 
The linearized equations of motion are simply $(\d_{dR} + \dbar) \alpha = 0$. 
Thus, up to equivalence, all solutions to the linearized equations of motion are constant in the real variable $t$, and holomorphic in $z,w$. 

Modifications of the boundary condition at the point~$z = 0$ on the boundary take the form
\[
\alpha = f(w) \delta^{(r)}_{z=0} 
\]
where $f$ is some holomorphic function. 
Here $\delta^{(r)}_{z=0}$ denotes the $r$th derivative of the $\delta$-function at $z=0$. 
It is convenient to parameterize such boundary modifications algebraically by expressions of the form 
\[
\alpha_{k,r} = w^k \delta^{(r)}_{z=0} .
\]
Linear combinations of such states form a dense subspace of all possible modifications at the boundary. 

The reason that the boundary modifications take this form can be seen by understanding in more explicit terms the vacuum boundary condition.
The phase space at the boundary $C$ can be identified with the following cohomology
\[
\Omega^{0,\bu}(C) \otimes \cA^{0;\bu}(\R \times \C - 0) [1]
\]
where $\cA^{0;\bu}$ denotes the mixed de Rham--Dolbeault cohomology of $\R \times \C - 0$ as a manifold equipped with a transversely holomorphic foliation. 
We refer to the section below for a reminder on this geometric structure.

The phase space is equipped with a natural symplectic form given by
\[
\int_C \d z \oint_{S^2} \d w \, \alpha \wedge \alpha' .
\]
There is a natural Lagrangian inside of the phase space which consists of linear combinations of elements $\alpha(z) \otimes f(t,w)$ where $\alpha(z) \in \Omega^{0,\bu}(C)$ and $f(t,w)$ is a smooth function on $\R \times \C - 0$ which extends to zero. 
The linearized equations of motion simply say that $\alpha$ is holomorphic, $f$ is independent of $t$ and depends holomorphically on $w$.

\parsec[s:brfive]

We now consider the situation of backreacting some number of (twisted) fivebranes in our eleven-dimensional model.
Let $Z$ be a three-fold that the fivebranes wrap. 
We also fix a rank 2 holomorphic vector bundle $V\to Z$ such that $\wedge^{2} V \cong K_{Z}$, and let $X = \text{Tot}(V)$. 
The condition on $V$ ensures that $X$ is a Calabi-Yau five-fold---in the main body of the paper we will choose $V$ to be the bundle $K_{Z}^{1/2}\otimes \C^{2}$. 
The eleven-manifold we place our theory on is $\R\times X_V$, which we can view as the total space of the \textit{real} rank five bundle $\R\oplus V$ over $Z$.

We place a stack of $N$ fivebranes wrapping the zero section in $\R\oplus V$.
Denote the complement of the zero section by
\[
M_V = \text{Tot}(\R\oplus V) - 0(Z).
\]
Notice that in \S \ref{s:Lsugra} we have only defined the sheaf of $L_\infty$ algebras $\cL_{sugra}$ on a product of a smooth one-manifold times a Calabi--Yau five-fold.
The eleven-manifold $M_V$ is not of this form, nevertheless there is a generalization of $\cL_{sugra}$ which one can define using the natural geometric structure present in our situation. 

A transversely holomorphic foliation (THF) on a smooth manifold $M$ is an integrable subbundle $F \subset \T_M \otimes \C$ such that $F + \Bar{F} = \T_M \otimes \C$. 
We will say that $F$ equips $M$ with the a THF structure. 
%Suppose $M$ is a manifold equipped with a THF structure and let $\cF$ be the corresponding foliation of even codimension.
The product $M = S \times X$, where $X$ is a complex manifold and $S$ is a smooth manifold has a natural THF structure with $F$ the restriction of the tangent bundle of $N$ along the projection. 
Locally, any THF manifold is split of the form $\R^d \times \C^n$, whose coordinates we will denote by $(x_i ;  z_j)$.
The bundle $F$ is locally spanned by the vector fields $\partial / \partial x_i$'s and $\del/\del \zbar_j$'s. 
(Notice that when $F \cap \Bar{F} = 0$ we are just describing an ordinary complex structure on $M$.)

Any submanifold of a THF manifold is itself a THF manifold. 
We are most interested in the submanifold $M_V \subset {\rm Tot}(\R \oplus V) = \R \times X$ where we equip $\R \times X$ with its standard split THF structure.

We have expressed the fields of the eleven-dimensional theory in terms of a mixed type of de Rham and Dolbeault cohomology. 
Let us focus on the fields $\beta,\gamma$ which on $\R \times X$ combine to form the complex 
\beqn\label{eqn:drdol}
\Omega^{\bu}(\R) \otimes \Omega^{0,\bu}(X) \xto{1 \otimes \del} \Omega^{\bu}(\R) \otimes \Omega^{1,\bu}(X) .
\eeqn
As usual, we leave the $\d_{dR}$ and $\dbar$ operators implicit.
More generally, there is a natural cohomology associated to a THF structure.
Suppose $(M,F)$ is a THF structure and 
denote by $Q$ the (complex) quotient bundle $\T_\C M / F$.
For each $p,q$ denote by $\cA^{p;q}$ smooth sections of the bundle $\wedge^p Q^\vee \otimes \wedge^q F^\vee$. 
The derivative along the leaves of the foliation defined by $V$ defines a map 
\[
\thfd \colon \cA^{p;q} \to \cA^{p;q+1}  .
\]
By integrability one has $\thfd^2 = \thfd \circ \thfd = 0$ and so $\thfd$ equips $\cA^{p;\bu} = \oplus_q \cA^{p;q}[-q]$ with the structure of a cochain complex for each $p$. 
Locally in a split THF structure the operator $D$ is of the form $\d_{dR} + \dbar$ where $\d_{dR}$ is the de Rham differential along $\R^d$ and $\dbar$ is the Dolbeault operator along $\C^n$. 
There is also an analog of the holomorphic $\del$ operator which takes the form $\thfdel \colon \cA^{p;q} \to \cA^{p+1;q}$. 
The obvious exterior product $\cA^{p;q} \times \cA^{r;s} \to \cA^{p+r;q+s}$ further endows 
\[
\left(\cA^{\bu;\bu} (M), \thfd + \thfdel\right) = \left(\oplus_p \cA^{p;\bu}[-p] , \thfd + \thfdel \right) 
\]
with the structure of a bigraded commutative dg algebra.
This complex is simply isomorphic to the de Rham complex of $M$, but this presentation lends itself to more interesting quotient complexes. 
For example, the forms of type $(p,\bu)$ with $p \geq 2$ form an ideal inside of this dg algebra; hence we get a quotient dg algebra 
\beqn\label{thfcoh1}
\left(\cA^{\leq 1;\bu}(M), \thfd + \thfdel\right) = \quad \cA^{0;\bu} \xto{\thfdel} \cA^{1;\bu} .
\eeqn
We leave the $\thfd$ operator implicit in the presentation on the right hand side.
When $M = M_V$, it is this complex that is the THF generalization of the truncated de Rham--Dolbeault complex in \eqref{eqn:drdol}---it is easy to see that it agrees with this complex in the case of a split THF manifold.
There is a similar THF description for the fields $\mu,\nu$ in the eleven-dimensional theory.

%Note that the eleven-manifold $\R \times X$ is equipped with a natural transverseley--holomorphic foliation (THF)---the complexified tangent bundle decomposes as $T_{\R}\oplus T_{Z}\oplus \Bar{T}_{Z}$.
With this THF cohomological description of the eleven-dimensional theory in place we proceed to describe the boundary condition obtained by removing the location of the branes. 
We may choose fiber coordinates of the bundle $t, w_{1}, w_{2}$ of $\R \oplus V$ over $Z$ and a fiberwise partially hermitian metric. 
Explicitly, the corresponding norm defines a map
\begin{align*}
 h \colon  \R\times X & \to \R_{+} \\
  (t, w_{i}, \bar{w_{i}}, p)& \mapsto t^{2} + |w_{1}|^{2}+|w_{2}|^{2}
\end{align*}
Letting $\pi \colon \R\times X \to Z$ be the natural projection, we obtain the $S^{4}$ bundle 
\[
p \define (h,\pi) \colon \R \times X \to \R_{+}\times Z 
\] 
which restricts to an $S^4$ bundle $p|M \colon M \to \R_{>0} \times Z$.
These embeddings and projections fit inside of the following commutative diagram
\[
\begin{tikzcd} 
M \ar[d,"p|M"'] \ar[r,hook] & \R \times X \ar[d,"p"] & \ar[l,hook',"0"'] Z \ar[d,"="] \\
\R_{>0} \times Z \ar[r,hook] & \R_{+} \times Z & \ar[l,hook',"0 \times \id"] Z.
\end{tikzcd}
\]
The inclusions on the left are the natural embeddings. 
The top right inclusion is the zero section of ${\rm Tot}(\R \oplus V) = \R \times X$ and the bottom right inclusion is the embedding at radius $r = 0$.

As we just elaborated, the eleven-dimensional theory is defined on the THF manifold $M$---in the BV formalism this is encoded, in part, by the sheaf of $L_\infty$ algebras $\cL_{sugra}$ on $M$. 
Compactification of this theory along the $S^4$ link corresponds to pushing forward this sheaf along $p|M$.
The resulting sheaf of $L_\infty$ algebras $(p|M)_*\cL_{sugra}$ describes, in the BV formalism, the compactified theory on the seven-manifold $\R_{>0} \times Z$. 

The theory on $M$ extends to a theory on the manifold obtained by filling in the zero section of $\R \times V$; in other words, we know that the theory is defined on the entire space $\R \times X$.
This means that there is a natural way to extend the theory on $\R_{>0} \times Z$ to the seven-manifold with boundary $\R_{+} \times Z$. 
The restriction of this theory to the six-dimensional boundary plays the most important role for us. 

\brian{trying to incorporate below}
%To do this we will make use of the natural foliated geometric structures which we have around. 

Let $V_{M}$ denote the vertical tangent bundle to $(h, \pi)$, which we view as a subbundle of the real tangent bundle $T_{M}$. 
The complexification $V_{M}^{\C}$ inherits some natural subbundles coming from the ambient THF structure. Indeed, if $\iota \colon M \to \R\times X$ is inclusion, then we consider the pair
\[
  V_{M}\cap \iota^{*} T_{\R}, \ \ \ \ \ V\cap \iota^{*} T_{Z}.
\]

The fibers of the composition $V_{M}\to M\to \R_{+}\times Z$ are copies of the tangent bundle of $S^{4}$, and the corresponding fibers of $V_{M}$ are subbundles of $TS^{4}$ that equip the fiber 4-spheres with a generalized Cauchy-Riemann structure. \surya{CITE}

 The underlying sheaf of cochain complexes is given by

\[
\Omega^{\bullet}(\R_{+})\otimes \left ( \begin{tikzcd}
\ul{\rm even} & \ul{\rm odd} \\
\PV^{1,\bu}(Z) \ar[r, "\del_{\Omega}"] & \PV^{0,\bu}(Z)\\
\Omega^{0,\bu}(Z) & \Omega^{1,\bu}(Z) \ar[l, "\del"]
\end{tikzcd}
 \right ) \otimes CR (S^{4}).
\]


Here $CR (S^{4})$ denotes the cohomology of the tangential Cauchy-Riemann complex of $S^{4}$ \surya{CITE}, equipped with the above Cauchy-Riemann structure. Its computation is facilitated by the following lemma:

\begin{lem}
  Let $\R^{d}\times \C^{n}$ be an affine THF manifold, and choose a partial hermitian metric. Let $S^{d+2n-1}$ denote the corresponding unit sphere, equipped with its standard generalized Cauchy-Riemann structure. Then there is a quasi-isomorphism

  \[CR (S^{n+2d-1})\cong \cA^{\bu;\bu}\left ( (\R^{d}\times \C^{n})\setminus 0 \right )\]

  where the right-hand-side denotes the Dolbeault-deRham complex.
\end{lem}

The cohomology of the Dolbeault-deRham complex of $\R\times \C^{2}$ is easy to describe.


It was argued in \cite{RSW} that to leading order the coupling of a stack of twisted fivebranes to the eleven-dimensional theory is given by the nonlocal interaction 
\beqn\label{eqn:br1}
I_{M5} = N\int_{Z} \div^{-1}\mu \vee \Omega +\cdots 
\eeqn
where $\mu \in \Omega^0 (\R) \hotimes \PV^{1,3}(X)$ is a component of a field in the eleven-dimensional theory which satisfies $\div \mu = 0$.

\parsec
Let $C$ be a curve, and let $V\to C$ be a rank 4-holomorphic vector bundle over $C$ such that $\wedge^{4} V = K_{C}$. This condition again ensures that $X = {\rm Tot} V$ is a Calabi-Yau five-fold - in the main body of the paper, we will take $V = K^{1/2}_{C}\otimes \C^{4}$. Abusively letting $V$ also denote its pullback along the canonical projection $\R\times C \to C$, we may view $\R\times X$ as the total space of $V$ on $\R\times C$. As before we will consider wrapping a stack of $N$ membranes along the zero section.

Since $V$ is a complex vector bundle, we may choose a fiberwise hermitian metric, and as before, we may view $\R\times X \setminus \R\times C$ as an $S^{7}$-bundle over $\R_{>0}\times \R\times C$.




%\subsection{Supergravity at the boundary of AdS}
%
%Suppose we are in the situation in the last section where we place the eleven-dimensional theory on $\R \times X$ where $X = {\rm Tot}(K^{1/2}_Z \otimes \C^2)$. 
%We are furthermore placing some number of fivebranes along the locus $Z$ cut out by the zero section. 
%If we remove this locus we obtain the following manifold 
%\[
%\R \times X - \left(0 \times Z\right) \simeq Z \times \left({\rm Tot}(N_Z) - 0\right) 
%\]
%where on the right hand side ${\rm Tot}(N_Z) - 0$ is the total space of the normal bundle to $Z$ in $\R \times X$ minus the zero section.
%
%
%\subsection{Enhanced symmetry in a flat background}
%
%\parsec[s:e36]
%
%We recall the definition of the exceptional super Lie algebra $E(3|6)$. 
%We follow \cite{Kac_class, KacRudakov}.
%
%The even part of the super Lie algebra $E(3|6)$ is 
%\[
%\Vect(\Hat{D}^3) \oplus \lie{sl}(2) \otimes \cO(\Hat{D}^3) .
%\]
%The Lie bracket is determined by the obvious commutators and the action of vector fields on functions. 
%
%The odd part is of the form
%\[
%\Omega^{-1/2}(\Hat{D}^3) \otimes \C^2 .
%\]
%where $\Omega^{-1/2}(\Hat{D}^3)$ is the space of formal power series sections of the bundle
%\[
%\T_{\Hat{D}^3}^{*} \otimes K_{\Hat{D}^3}^{-1/2}
%\]
%where $K_{\Hat{D}^3}^{-1/2}$ is the negative square-root of the canonical bundle on $\Hat{D}^3$. 
%We will write a general element of this module as 
%\[
%\alpha = \sum_i g^i \d z_i^{-1/2} 
%\]
%where $g^i \in \cO(\Hat{D}^3)$. 
%The action by the even part of the super Lie algebra is by Lie derivative and by the fundamental $\lie{sl}(2)$ action on $\C^2$.
%We will write $r \in \C^2$ for a vector in the fundamental representation.
%Finally, there is a bracket 
%\[
%E(3|6)_- \times E(3|6)_- \to E(3|6)_+
%\]
%defined by
%\[
%[\alpha_1 \otimes r_1, \alpha_2 \otimes r_2] = - (\alpha_1 \wedge \alpha_2) \otimes (r_1 \wedge r_2) - (\d \alpha_1 \otimes \alpha_2 + \alpha_1 \otimes \d \alpha_2) \otimes (r_1 \odot r_2) .
%\]
%%The first term in the above equation arises from the composition 
%%\[
%%(\Hat{\Omega}_3^{-1/2} \otimes \Hat{\Omega}_3^{-1/2}) \otimes (R \otimes R) \xto{\wedge \otimes \wedge} \Hat{\Omega}_3^{-1} \otimes \wedge^2 R \cong \fw_3 .
%%\] 
%%The second term uses the composition 
%%\[
%%(\Hat{\Omega}_3^{-1/2} \otimes \Hat{\Omega}_3^{-1/2}) \otimes (R \otimes R) \xto{(\d \otimes \id) \otimes \odot} (\Hat{\Omega}_3^{1/2} \otimes \Hat{\Omega}_3^{-1/2}) \otimes \Sym^2 (R) \cong \lie{sl}(2) \otimes \Hat{\cO}_3 . 
%%\] 
%
%\brian{suca enhancement following \cite{SWsuca6d}}
%
%\parsec[s:weight]
%
%
%\parsec[s:locallieE510]
%
%We have recollected how the super Lie algebra $E(5|10)$ (and its central extension) decomposes as a module for the super Lie algebra $E(3|6)$. 
%We now wish to lift this to a decomposition of factorization algebras on $\C^3$, or more generally, factorization algebras on any complex three-fold $Z$. 
%
%We outline an approach to constructing the factorization algebras via enveloping factorization algebras of certain (infinite rank) local Lie algebras on the three-fold $X$. 
%For another approach \brian{pointer}. 
%
%We have already discussed the local Lie algebra $\cL_{sugra}$ on $X \times S$ where $X$ is a Calabi--Yau five-fold and $S$ is a smooth one-manifold. 
%We take $X = Z \times \C^2$ and $S = \R$. 
%and view $\Obs_{sugra} = \clie^\bu(\cL_{sugra})$ as a factorization algebra on $Z \times \C^2 \times \R$. 
%Then, we can consider the pushforward of $\Obs_{sugra}$ along the map
%\[
%\pi \colon Z \times \C^2 \times \R \to Z 
%\]
%which leads to a factorization algebra $\pi_* \Obs_{sugra}$ on the three-fold $Z$. 
%
%Since $\pi$ is not a proper map (the fibers are not compact) we cannot exhibit $\pi_*\Obs_{sugra}$ as the Chevalley--Eilenberg complex of some ordinary local Lie algebra on $Z$. 
%Nevertheless we will find a pro local Lie algebra $\cL_{sugra,\pi}$ on $Z$ which is a particularly nice `small model' for this pushforward factorization algebra.
%
%A local Lie algebra enhancement for $E(3|6)$ has been given in \cite{SWsuca6d} \brian{finish}
%
%
%
%
%
%
%
%\section{Example} 
%
%Consider the holomorphic local Lie algebra on $\C^n$
%\[
%\Omega^{0,\bu}(\C^n) 
%\]
%and corresponding factorization algebra $\cF = \clie^\bu(\Omega^{0,\bu}(\C^n))$. 
%Now, we can restrict this to a factorization algebra on
%\[
%\C^n - 0 \subset \C^n
%\]
%that we denote by $\cF|_{\C^n - 0}$. 
%
%Next, consider the radial projection
%\[
%r \colon \C^n - 0 \to \R .
%\]
%Notice that $r_* (\cF|_{\C^n - 0})$ will not be the observables of any ordinary local Lie algebra.
%Nevertheless we can come up with a nice algebraic model.  
%
%Let $A_n$ be the usual dg model for the formal punctured disk $\Hat{D}^n - 0$ as defined in \cite{FHK,GWkm}.
%
%\begin{lem}
%There is an injective map of factorization algebras on $\RR_{>0}$:
%\[
%\clie^\bu(\Omega^\bu(\RR_{>0}) \otimes A_n) \to r_* (\cF|_{\C^n - 0}) 
%\]
%which is dense in cohomology.
%\end{lem}
%
%This tells us that the correct `fields' of the dimensionally reduced theory on $\RR_{>0}$ is $\Omega^\bu(\R_{>0}) \otimes A_n$. 


%\end{document}
