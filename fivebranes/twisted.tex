\documentclass[11pt]{amsart}

%\usepackage{../macros-master}
\usepackage{macros-fivebrane}

\begin{document}

\section{Twisted eleven-dimensional supergravity}

\subsection{Observables in a holomorphic-topological theory}

One of the main results of \cite{CG2}.

\begin{dfn} 
Let $\Obs$ be a factorization algebra on a smooth manifold $M$.
The classical \defterm{point observables} at $p \in M$ is the limit $\Obs(p) = \lim_{U \ni p} \Obs(U)$ where the limit runs over open sets $U \subset M$ containing $p$.
\end{dfn}

\parsec

Without loss of generality we will assume our holomorphic-topological local Lie algebra on $M \times X$ is of the form
\[
\cL = \Omega^\bu (M) \hotimes \Omega^{0,\bu}(X, L) 
\]
where $L$ is a graded holomorphic vector bundle on $X$.

\begin{lem}
\label{lem:localops}
Suppose that $\cL$ is a holomorphic-topological local Lie algebra on $\R^m \times \C^n$ and consider the factorization algebra $\Obs = \clie^\bu (\cL)$. 
Then, the Taylor expansion map
\[
\cL(\C^n \times \R^m) \to L_0[[z_0,\ldots,z_n]]
\]
induces a quasi-isomorphism of commutative dg algebras
\[
\Obs(0) \simeq \clie^\bu \left( L_0[[z_1,\ldots,z_n]] \right) .
\]
\end{lem}

 
\subsection{A model for twisted supergravity}

Let $X$ be a Calabi--Yau five-fold. 
In \cite{SWspinor} a complete description of the free limit of the holomorphic twist of eleven-dimensional supergravity is given. 
Within the BV formalism, the theory is only $\Z/2$ graded.

In \cite{RSW} we have given a proposal for the minimal twist of fully interacting eleven-dimensional supergravity as a $\Z/2$ graded interacting BV theory. 

\parsec[s:fields]
The eleven-dimensional theory exists on any manifold of the form $X \times S$ where $X$ is a Calabi--Yau five-fold and $S$ is a real oriented smooth one-dimensional manifold. 

\parsec[s:e510]

Let $\Hat{D}^5$ denote the formal five-disk whose algebra of functions is 
\[
\cO(\Hat{D}^5) = \C[[z_1,\ldots,z_5]] .
\]

\parsec[s:L0]

For us, a $\Z/2$ graded $L_\infty$ algebra is a $\Z/2$ graded vector space $L$ equipped with a collection of operations $\{[-]_k\}_{k = 1,2,\ldots}$ where 
\[
\ell_k \colon L^{\times k} \to \Pi^k L 
\]
are multilinear maps which satisfy the ordinary higher Jacobi relations defining an $L_\infty$ algebra \cite{??}. 
Here, $\Pi^k L = L$ if $k$ is even and $\Pi^k L = \Pi L$ if $k$ is odd. 
Thus, $\ell_k$ is an even operation if $k$ is even and an odd operation if $k$ is odd. 
\footnote{This is not to be confused with a super $L_\infty$ algebra which is an $L_\infty$ algebra internal to the category of super vector spaces.
A $\Z/2$ graded $L_\infty$ algebra is simply what one gets when they apply the forgetful function from $\Z$ graded vector spaces to $\Z/2$ graded vector spaces.}

Let $L_{sugra}$ be the following $\Z/2$ graded $L_\infty$ algebra. 
The even part is 
\[
\Vect (\Hat{D}^5) \oplus \cO (\Hat{D}^5) .
\]
We will denote even elements by $(\mu, \beta)$ according to the decomposition. 
The odd part is 
\[
\cO(\Hat{D}^5) \oplus \Omega^1(\Hat{D}^5) .
\]
We will denote odd elements by $(\nu, \gamma)$ according to the decomposition. 

The $L_\infty$ structure is described as follows. 
First, $\d = [-]_1$, the differential, is simply given by 
\begin{align*}
\d \beta & = \del \beta \in \Omega^1(\Hat{D}^5) \subset L_{sugra,-} \\
\d \mu & = \div \mu \in \cO(\Hat{D}^5) \subset L_{sugra, -}
\end{align*}
and $\d \gamma = \d \nu = 0$. 

For $k \geq 2$ the general formula for the $k$-ary brackets is 
\begin{align*}
[\nu_1, \ldots, \nu_{k-2}, \mu_1,\mu_2]_{k} & = \div(\nu_1 \cdots \nu_k \mu_1 \wedge \mu_2) \\
[\nu_1,\ldots, \nu_{k-3}, \mu_1,\mu_2,\gamma]_k & = \nu_1 \cdots \nu_{k-3} (\mu \wedge \mu') \vee \del \gamma .\\
[\nu_1,\ldots,\nu_{k-2}, \mu, \gamma]_k & = \nu_1 \cdots \nu_{k-2} \mu \vee \del \gamma .
\end{align*}
In \cite{RSW} it is shown that this endows $L_{sugra}$ with the structure of a $\Z/2$ graded $L_\infty$ algebra.

\parsec[s:sugraobs]

By the usual methods of the BV formalism the action functional $S_{sugra}$ described above endows the parity shift of the fields $\cL_{sugra} = \Pi \cE_{sugra}$ with the structure of a holomorphic-topological local $\Z/2$ graded $L_\infty$ algebra. 

On $\C^5 \times \R$ we can describe this super Lie algebra structure explicitly. 
First, by the Dolbeault and de Rham Poincar\'e lemmas it is easy that the even part of the super Lie algebra $\cL(\C^5 \times \R)$ is equivalent to a one-dimensional central summand $\C$ plus the Lie algebra of divergence-free vector fields on $\C^5$:
\[
\Vect_0 (\C^5) = \{X \in \Vect(\C^5) \; | \; \div X = 0\} .
\]
The odd part of the super Lie algebra $\cL(\C^5 \times \R)$ is equivalent to the space of holomorphic one-forms on $\C^5$ modulo exact one-forms
\[
\Omega^{1,hol}(\C^5) / {\rm Im}(\del) 
\]
which is, of course, equivalent to the space of closed holomorphic two-forms $\Omega^{2,hol}_{cl}(\C^5)$. 

\begin{thm}[\cite{RSW}[Theorem 2.1]]
The Taylor expansion map determines a map of $\Z/2$ graded $L_\infty$ algebras
\[
j_\infty \colon \cL_{sugra}(\C^5 \times \R) \to L_{sugra} .
\]
Furthermore, $L_{sugra}$ is equivalent as a $\Z/2$ graded $L_\infty$ algebra to $\Hat{E(5|10)}$. 
\end{thm} 

As an immediate corollary of this result we obtain by Lemma \ref{lem:localops} the following.

\begin{cor}
Let $\Obs_{sugra}$ be the factorization algebra on $\C^5 \times \R$ of classical observables of the minimal twist of eleven-dimensional supergravity.
There is a quasi-isomorphism of commutative dg algebras
\[
\Obs_{sugra} (0) \simeq \clie^\bu \left( \Hat{E(5|10)} \right) .
\]
\end{cor}


\subsection{Twisted fivebranes} 

%\subsection{Twisted AdS backgrounds}

\parsec[s:coupling]

\parsec[s:single]

An explicit characterization of the holomorphic twist of a single fivebrane theory was given in \cite{SWtensor}. 
We briefly recall this desription. 

\parsec[]

The holomorphic twist of the fivebrane theory is defined on any complex three-fold $X$ (which is not necessarily Calabi--Yau).
To define the theory we must assume that $X$ is equipped with a square-root of the canonical bundle $K_X^{1/2}$. 

The theory is $\Z \times \Z/2$ graded but at first pass we recall its totalization as a $\Z/2$ graded theory. 
There are four fundamental fields $(\mu, \Omega, \phi_1,\phi_2)$ which consist of even Dolbeault forms of type $(2,1)$ and $(3,0)$ on $X$:
\[
\mu \in \Omega^{2,1}(X), \quad \Omega \in \Omega^{3,0}(X),
\]
and a pair of even sections:
\[
(\phi_1,\phi_2) \in \Omega^0(X , K^{1/2}_X) \otimes \C^2 .
\]
%where $R \cong \CC^2$ is the fundamental $Sp(1)$ representation. 
The equations of motion read
\beqn
\label{eqn:eom}
\begin{split}
\del \mu + \dbar \Omega & = 0 \\
\dbar \mu = \dbar \phi = \dbar \psi & = 0 .
\end{split}
\eeqn

There are gauge symmetries for the fields $\mu, \Omega$ determined by a ghost $b$ which is a Dolbeault form of type $(2,0)$ which acts simply by
\beqn
\label{eqn:ghost}
\begin{split}
\mu & \mapsto \mu + \dbar b  \\
\Omega & \mapsto \Omega + \del b .
\end{split}
\eeqn

This theory notoriously does not admit a Lagrangian description. 
However, it can still be put in a degenerate form of the BV formalism where the space of fields above is equipped with an odd Poisson bivector \cite{SWtensor}.
We describe this Poisson bivector explicitly  

\parsec[s:zz2]

In the BV formalism the fields of a free theory together with the ghosts and the (partial) antifields combine to form a cochain complex. 

\parsec[s:singleops]

\begin{prop}
Let $\Obs_{fivebrane}$ be the observables of the theory on a single fivebrane on $\C^3$ after performing the holomorphic twist.
There is a quasi-isomorphism of super commutative dg algebras 
\[
\Obs_{fivebrane} (0) \simeq \Sym \left( I(0,0;1;-1)^* \right) .
\]
\end{prop}

\subsection{Enhanced symmetry in a flat background}

\parsec[s:weight]


\parsec[s:e36]

The super Lie algebra $E(3|6)$. 

\begin{lem} 
The map $E(3|6) \hookrightarrow E(5|10)$ lifts to a map of super $L_\infty$ algebras 
\[
E(3|6) \hookrightarrow \Hat{E(5|10)} .
\]
\end{lem}
\begin{proof}
It suffices to show that the central extension of $E(5|10)$ splits when restricted to the subalgebra $E(3|6)$. 
Recall that the even cocycle defining this extension is 
\[
(\mu, \mu', \alpha) \mapsto \<\mu \wedge \mu' , \alpha\>|_{z=0} .
\]
This is clearly zero since the image 
\end{proof}

\end{document}






