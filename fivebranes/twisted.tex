%\documentclass[11pt]{amsart}%
%
%%\usepackage{../macros-master}
%\usepackage{macros-fivebrane}

%
%\begin{document}
%

\section{Twists in dimensions eleven and six}
\label{s:twisted}

In this section we review the descriptions of twists of the key players in $M$-theory including eleven-dimensional supergravity as well as the six-dimensional superconformal field theory associated to the abelian Lie algebra~$\lie{gl}(1)$.

\subsection{A model for minimally twisted supergravity}

%Eleven-dimensional supergravity is the unique supersymmetric theory in eleven dimensions.
%It plays an important role as the low energy limit of $M$-theory.

Recently in \cite{CLsugra}, Costello and Li gave a rigorous definition of twisted supergravity.
The idea is roughly the following.
In any theory of supergravity, the action of supersymmetry is gauged. 
When one treats the theory in the BRST formalism, the fields then include a collection of bosonic ghosts which homologically enforce invariance under the odd gauge symmetries.
By definition, twisted supergravity is supergravity in the background where a bosonic ghost for supersymmetry takes a nonzero value $Q$, where $Q$ is a nilpotent supercharge. 
Such backgrounds have the feature that the worldvolume theories of branes in their presence are twisted in a way that is compatible with the usual notion of twisting supersymmetric gauge theories.

Eleven-dimensional supersymmetry admits two inequivalent classes of twists characterized by the dimension of the subspace of translations spanned by the image of bracketing with the twisting supercharge.
\begin{itemize}
\item 
The minimal twist. 
This twist leaves six real directions invariant and is preserved by the subgroup $SU(5)$ of the Lorentz group. 
\item 
The non-minimal twist. 
This twist leaves nine real directions invariant and is preserved by the subgroup $SU(2) \times G_2$ of the Lorentz group. 
\end{itemize}

In this paper we will focus primarily on the minimal twist. 
In \cite{SWspinor} a complete description of the free limit of this twist of eleven-dimensional supergravity is given.
Within the BV formalism, the theory is only $\Z/2$ graded, a point we will elaborate on further soon. 
In \cite{RSW} we have given a proposal for the minimal twist of fully interacting eleven-dimensional supergravity as a $\Z/2$ graded interacting BV theory in a background with global symmetry group $SU(5)$.

The eleven-dimensional theory described in \cite{RSW} exists on any manifold which is locally of the form 
\[
S \times X 
\] 
where $X$ is a Calabi--Yau fivefold and $S$ is a real oriented smooth one-dimensional manifold.
More generally, the eleven-dimensional theory can be constructed on any eleven-manifold equipped with a transversely holomorphic foliation which is equipped with an appropriate holomorphic volume form on the leaves of the foliation.
We will elaborate on this further in \S \ref{sec:states}.

We recall the fields and equations of motion of this eleven-dimensional theory. 

%\parsec[s:e510]
%
%Let $\Hat{D}^5$ denote the formal five-disk whose algebra of functions is 
%\[
%\cO(\Hat{D}^5) = \C[[z_1,\ldots,z_5]] .
%\]

\parsec[s:sugrafields]

%To begin this section we highlight some of the important features of the eleven-dimensional model considered in \cite{RSW}. 
%Let's first consider the case that the eleven-dimensional manifold is of the form $S \times X$, where $X$ is a Calabi--Yau fivefold.
Let $\T_X$ stand for the holomorphic tangent bundle of the Calabi--Yau fivefold $X$ and $\Bar{\T}_X$ its complex conjugate.
The complexified tangent bundle of the eleven-manifold $S \times X$ decomposes as
\[
\T_S \oplus \T_X \oplus \Bar{\T}_X 
\]
where $\T_S$ is the complexified tangent bundle of the smooth one-manifold $S$.
Locally, we choose coordinates $(t;z,\zbar)$ where $t$ is a smooth coordinate for $S$ and $z$ is a holomorphic coordinate for $X$. 

Momentarily we will give a full presentation of the fields and equations of motion within the Batalin--Vilkovisky (BV) formalism.
But first, we will give a more direct description of the equations of motion and gauge symmetries of the model.
 
One of the primary fields of eleven-dimensional supergravity describes deformations of the metric $g$.
Since we work in a background which has $SU(5)$ holonomy we can take the background metric on $S \times X$ to be the product of the flat metric on $S$ with a Calabi--Yau metric $g_{X}$ on $X$. 

Deformations of the background metric which survive the twist can be identified with Beltrami differentials, that is, sections of 
\[
\Bar{\T}_X^* \otimes \T_X \cong \Bar{\T}^*_X \otimes \Bar{\T}_X^* . 
\]
Here, we have used the isomorphism granted by the K\"ahler metric $g_{X}$.
In local holomorphic coordinates $\mu$ admits a description like
\[
\mu^{\Bar{i}}_j (t;z,\zbar) \d \zbar_{\Bar{i}} \otimes \del_{z_j} .
\]
More generally, the fields of this model include sections of the holomorphic tangent bundle with coefficients in Dolbeault forms on $X$ of arbitrary Dolbeault type and de Rham forms on $S$, which we can write in superfield notation as 
\[
\mu = \mu^{\Bar{I}}_j (t;z,\zbar) \d \zbar_{\Bar{I}} \otimes \del_{z_j} + \mu^{\Bar{I}}_{t,j} (t;z,\zbar) \d t \d \zbar_{\Bar{I}} \otimes \del_{z_j}
\]
where there is a sum over the multi-index $\Bar{I}$ as it ranges over subsets of $\{1,\ldots,5\}$.
In this notation, the component $\mu^{\Bar{I}}_j$ is {\em odd} if $|\Bar{I}|=0,2,4$ is even and {\em even} if $|I| = 1,3,5$ is odd.
Similarly, the component $\mu^{\Bar{I}}_{t,j}$ is {\em even} if $|\Bar{I}|=0,2,4$ is even and {\em odd} if $|I| = 1,3,5$ is odd.
In particular we have odd fields $\mu_j(t;z,\zbar) \del_{z_j}$ which play the role of ghosts for infinitesimal changes of coordinates.
We will see that the equations of motion dictate that we only see changes of holomorphic coordinates.

The field $\mu$ satisfies a condition that it be {\em divergence-free} for the Calabi--Yau structure on the fivefold~$X$.
This means that it is required to satisfy the equation
\beqn
\div \mu = 0
\eeqn
where $\div(-)$ is the divergence with respect to the holomorphic volume form $\Omega$ on $X$.
By including the full space of BV fields, this condition does arise from an action functional principle---it is not an additional constraint like the one present in the six-dimensional superconformal theory.
The field to add in this situation is a Dolbeault form $\nu \in \Omega^{0,\bu}(X)$ which imposes the equation above upon setting $\nu = 0$.
Another piece of the equations of motion for~$\mu$ dictate that it be locally constant along~$S$. 

Ordinarily, for $\mu$ to describe a deformation of complex structure in the $X$ direction we would require that it satisfy the Maurer--Cartan equation.
For twisted supergravity, however, we find a modification of this equation which involves the twisted analog of another familiar field.

The other primary bosonic field of eleven-dimensional supergravity is a three-form.
In the BV formalism we see all types of differential forms corresponding to a tower of ghosts for this higher gauge field together with antifields and antighosts.
 
On $S \times X$ the space of complex-valued differential forms decomposes into de Rham--Dolbeault forms as in
\[
\oplus_{k,p,q} \Omega^k(S) \otimes \Omega^{p,q}(X) .
\]
We refer to the homogenous pieces as the space of $(k;p,q)$ forms, these admit local presentations as
\[
\sum_{P,Q} f_{PQ} (t;z,\zbar) \d z_P \d \zbar_Q + \sum_{P',Q'} g_{P'Q'}(t;z,\zbar) \d t \d z_{P'}\d \zbar_{Q'} 
\]
where the first sum is over tuples~$P=(p_1,p_2,p_3),Q=(q_1,q_2,q_3)$ and
the second sum is over tuples~$P'=(p'_1,p'_2,p'_3),Q=(q'_1,q'_2,q'_3)$

In the minimal twist only forms with $p_i ,p_i'\leq 1$ survive. 
Thus, we have as part of the complex of fields in the twisted theory the space of forms
\[
\left(\oplus_{k,q} \Omega^k(S) \otimes \Omega^{0,q}(X)\right) \oplus 
\left(\oplus_{k',q'} \Omega^{k'}(S) \otimes \Omega^{1,q'}(X)\right) .
\]
We denote a general (non-homogenous) element in the first summand by $\beta$ and an element in the second summand by $\gamma$.
As above we can expand such fields in local coordinates using multi-indices by 
\begin{align*}
\beta & = \beta^{\Bar{I}} \d \zbar_{\Bar{I}} + \beta_t^{\Bar{J}} \d t \d \zbar_{\Bar{J}}\\
\gamma & = \gamma^{i \Bar{I}} \d z_i \d \zbar_{\Bar{I}} + \gamma_t^{j \Bar{J}} \d t \d z_j \d \zbar_{\Bar{J}}.
\end{align*}
%When we want to be specific by the form type we use the notation $\beta^{k;q}(t;z,\zbar)$ for the $(k;0,q)$ component of the three-form and $\gamma^{k;q}(t;z,\zbar)$ for the $(k;1,q)$ component of the three-form.
Fields $\beta^{\Bar{I}}, \gamma_t^{i \Bar{I}}$ with $|\Bar{I}| = $odd (resp. even) are {\em even} (resp. {\em odd}) and fields $\beta_t^{\Bar{I}}, \gamma^{i \Bar{I}}$ with $|\Bar{I}| = $odd (resp. even) are {\em odd} (resp. {\em even}). 

The most important equation of motion of the eleven-dimensional theory involves both the fields $\mu$ and $\gamma$. 
When $\div \mu = 0$ it takes the form
\beqn\label{eqn:eom2}
\dbar \mu + \frac12 [\mu,\mu] = \del \gamma \del \gamma .
\eeqn
Because of the term on the right hand side this equation is not exactly the usual Beltrami equation for deformations of complex structures.
On the left hand side we are implicitly using an identification between the holomorphic tangent bundle $\T_X$ and the bundle $\wedge^4 \T^*_X$ granted by the holomorphic volume form on $X$.
This allows us to view this equation as taking place entirely in the graded space $\Omega^\bu(S) \hotimes \Omega^{4,\bu}(X)$. 
In particular, this equation is inhomogenous in form type.
As an example of the homogenous form type $(0;4,2)$ piece of this equation we have 
\[
\ep^{ijklm} \del_{\zbar_{\bar{j}}} \mu_m^{\bar{i}} + \ep^{ijkmn} \del_{z_l} (\mu_{m} \mu_{n}) =\del_{z_i} \gamma^{j \bar{i}} \del_{z_k} \gamma^{l \bar{j}} 
\]

%Along $S$, $\beta$ and $\gamma$ behave like abelian gauge fields.
%Thus, locally we can assume that the $\Omega^1(S)$ components are exact and the $\Omega^0(S)$ components are locally constant.


\parsec[s:Lsugra]

We move on to give a complete description of our eleven-dimensional model within the Batalin--Vilkovisky (BV) formalism.
In the superfield notation above we have implicitly provided all components of the ghosts, fields, anti-fields, and anti-ghosts.
One of the key structures in the BV formalism is the graded skew-symmetric pairing between fields and antifields (ghosts and antighosts).
Once such a pairing $\omega_{BV}$ is introduced, one can opt to describe solutions to the equations of motion as Maurer--Cartan elements of a certain sheaf of ($\Z/2$-graded) $L_\infty$ algebras on $X \times S$.

An $L_\infty$ algebra is a graded vector space~$L=\oplus_{p\in \Z} L^p$ equipped with a collection of operations $\{[-]_k\}_{k = 1,2,\ldots}$ where 
\[
[-]_k \colon L^{\times k} \to L[2-k] 
\]
are multilinear maps which satisfy the ordinary higher Jacobi relations defining an $L_\infty$ algebra.
The shift $[2-k]$ means that $[-]_k$ is of cohomological degree $k-2$. 

The BV fields arise as the sections of a sheaf of $L_\infty$ algebras $\cL$ over the spacetime manifold $M$.
There is an overall shift in the relationship between the cohomological grading of $\cL$ and the ghost number in physics terminology. 
In cohomological degree zero of $\cL$ sit the ghosts, in cohomological degree one sit the fields, etc..
In this sense it is more accurate to refer to the BV fields as sections of the shift of the sheaf of $L_\infty$ algebras $\cL[1]$. 
We refer to \cite{CG2,ESW} for a further review of these conventions.
Together with the skew symmetric pairing $\omega_{BV}$ of cohomological degree $-1$ on $\cL[1]$, the structure maps $[-]_k$ organize together to define the full BV action which takes the general form
\beqn
\label{eqn:Sbv}
S(\Phi) = \sum_{k \geq 1} \omega_{BV} \left(\Phi , [\Phi,\ldots,\Phi]_{k}\right) ,
\eeqn
which we can think of as a functional on $\cL[1]$ of cohomological degree zero.

One feature of our proposal for the minimal twist of eleven-dimensional supergravity is that it only carries an overall $\Z/2$ grading. 
This $\Z/2$ grading totalizes the original ghost grading (which is by the group $\Z$) and the fermion number which is present in the untwisted theory.
We thus make use of $\Z/2$ graded versions of the usual BV formalism.
In this paper $\Z/2$ graded $L_\infty$ algebra means that we just have a $\Z/2$ graded vector space.
There are operations $\{[-]_k\}_{k = 1,2,\ldots}$ which satisfy the same higher Jacobi identities.
The operation $[-]_k$ is even if $k$ is even and odd if $k$ is odd.\footnote{This is not to be confused with a super $L_\infty$ algebra which is an $L_\infty$ algebra internal to the category of super vector spaces.
A $\Z/2$ graded $L_\infty$ algebra is simply what one gets when they apply the forgetful function from $\Z$ graded vector spaces to $\Z/2$ graded vector spaces.}

To build the $L_\infty$ algebra associated to the eleven-dimensional theory, we first introduce a sheaf of $\Z/2$ graded $L_\infty$ algebras on the Calabi--Yau fivefold $X$.
As a sheaf of super vector spaces, it will be given as the holomorphic sections of a super vector bundle that we denote by $L_X$. 
The even part of this super vector bundle is
\[
\T_X \oplus \C_X 
\]
where $\T_X$ is the holomorphic tangent bundle and $\C_X$ is the trivial bundle.
We will denote even sections by $(\mu, \beta)$ according to the decomposition. 
The odd part is 
\[
\C_X \oplus \T^*_X .
\]
We will denote odd elements by $(\nu, \gamma)$ according to the decomposition. 

The $L_\infty$ structure on the sheaf of holomorphic sections of $L_X$ is described as follows. 
First, $\d = [-]_1$, the differential, is simply given by 
\begin{align*}
\d \beta & = \del \beta \in \Omega^1_X \subset L_{X,-} \\
\d \mu & = \div \mu \in \cO_{X,\nu} \subset L_{X, -}
\end{align*}
and $\d \gamma = \d \nu = 0$. 
For $k \geq 2$ the general formula for the $k$-ary brackets is 
\begin{align*}
[\nu_1, \ldots, \nu_{k-2}, \mu_1,\mu_2]_{k} & = \div(\nu_1 \cdots \nu_k \mu_1 \wedge \mu_2) \in \PV^1_X \\
[\nu_1,\ldots, \nu_{k-3}, \mu_1,\mu_2,\gamma]_k & = \nu_1 \cdots \nu_{k-3} (\mu \wedge \mu') \vee \del \gamma \in \cO_{X,\beta} .\\
[\nu_1,\ldots,\nu_{k-2}, \mu, \gamma]_k & = \nu_1 \cdots \nu_{k-2} \mu \vee \del \gamma \in \Omega^1_X \\
[\gamma_1,\gamma_2] & = \Omega_X \vee (\del \gamma_1 \wedge \del \gamma_2) \in \PV^1_X .
\end{align*}
In \cite{RSW} it is shown that this endows $L_{X}$ with the structure of a sheaf of $\Z/2$ graded $L_\infty$ algebras.
There is a sub sheaf given by the sections $\mu$ and $\nu$ which is $L_\infty$-equivalent to the sheaf of holomorphic divergence-free vector fields on $X$. 

As all operations above are given in terms of holomorphic polydifferential operators, the $L_\infty$ algebra structure above induces a $\Z/2$ graded $L_\infty$ algebra structure on the Dolbeault resolution $\Omega^{0,\bu}(X,L_X)$ of the holomorphic vector bundle $L_X$.

Finally, to obtain a local $L_\infty$ algebra on $S \times X$, where $S$ denotes a real oriented smooth one-manifold, we simply tensor with the de Rham complex along~$S$.
We denote by $\cL_{sugra}$ the following local $\Z/2$ graded $L_\infty$ algebra on $X \times S$
\beqn
\cL_{sugra} \define \Omega^\bu(S) \hotimes \Omega^{0,\bu}(X,L_X) .
\eeqn

It is shown in \cite{RSW} that $\cL_{sugra}$ is equipped with a non-degenerate skew symmetric odd pairing $\omega_{BV,sugra}$ which is compatible with the $L_\infty$ structure above.
All together, this data prescribes the structure of a $\Z/2$ graded theory in the BV formalism where the BV action is as in \eqref{eqn:Sbv}.

The free part of the BV action is easy to describe
\beqn
\label{eqn:free}
\int_{S \times X} \left(\gamma (\d_{dR} + \dbar) \mu\right) \Omega_X + \int_{S \times X} (\beta (\d_{dR} + \dbar) \nu) \Omega_X + \int_{S \times X} (\beta \div \mu) \Omega_X ,
\eeqn
where $\Omega_X$ is the holomorphic volume form on~$X$.
Notice that in the first two terms no holomorphic derivatives appear in the direction of $X$ due to the presence of this holomorphic volume form.

The interacting part of the action is more complicated, partly for the reason that in our presentation above it is given by a formal series in the space of fields, rather than just a polynomial in the fields. 
Explicitly, the interaction can be written as
\beqn
\label{eqn:int}
\frac12 \int_{S \times X} (\mu^2 \del \gamma) \frac{\Omega_X}{1- \nu} + \frac16 \int_{S \times X} \gamma \del \gamma \del \gamma .
\eeqn
Notice that upon varying the field $\gamma$ and assuming that $\d_{dR} \mu = 0$ and $\nu = \div \mu = 0$ we recover the equation of motion \eqref{eqn:eom2} from this action functional. 
%
%Notice that the Beltrami differential $\mu^{\Bar{i}}_j \d \zbar_{\Bar{i}} \otimes \del_z$ as well as the components of the three-form which survive the minimal twist sit inside as odd sections of this sheaf of $L_\infty$ algebras.

\parsec[s:e510]

In the case that $X = \C^5$ and $S = \R$ the eleven-dimensional model we have just described bears a close relationship to the exceptional simple super Lie algebra $E(5|10)$ classified in \cite{KacClass}.
The even part of $E(5|10)$ is the Lie algebra of divergence-free vector fields on the formal five-disk $\Hat{D}^5$. 
The odd part of $E(5|10)$ is the space of closed two-forms $\Omega^{2,cl}(\Hat{D}^5)$ on the formal five-disk.
There is a natural action of divergence-free vector fields on closed two forms. 
Additionally there is a Lie bracket between two closed two-forms $\alpha,\alpha'$ which produces a divergence-free vector field via the standard holomorphic volume form on the five-disk:
\beqn
[\alpha,\alpha'] = \Omega_{\Hat{D}^5}^{-1} \vee ( \alpha \wedge \alpha' ) .
\eeqn

Given a vector bundle $E \to M$, the bundle of $\infty$-jets $J^\infty E \to M$ is a $\infty$-dimensional pro vector bundle whose sections consist of $\infty$-jets of sections of $E$.
If $M = \R^d$ and $E$ is translation invariant, then the bundle of $\infty$-jets can be identified with $E_0 \times \C[[x_i]]$ where $\{x_i\}$ is a chosen coordinate on~$\R^d$. 
The process of taking $\infty$-jets is well-behaved for local Lie algebras:
the $\infty$-jets of a translation invariant local Lie algebra on $\R^d$ at a point carries the natural structure of a Lie algebra.

In \cite{RSW} it is shown that the $\infty$-jets of $\cL_{sugra}$ at $0 \in \R \times \C^5$ is quasi-isomorphic to a central extension $\Hat{E(5|10)}$ of the super Lie algebra $E(5|10)$.
The central extension is defined by the (totally even) three-linear cocycle
\beqn\label{eqn:e510central}
(\mu, \mu' , \alpha) \mapsto \<\mu \wedge \mu' , \alpha\>_{z=w=0} \in \C .
\eeqn
Since this is a three-linear functional the model we use for the central extension is a super $L_\infty$ algebra (with zero one-ary operation) rather than just a super Lie algebra.

%parsec[s:twistedsugra]

%Before proceeding with the appearance of fivebranes in the minimal twist of eleven-dimensional supergravity we discuss a related, but simpler, five-dimensional theory.
%This five-dimensional theory, as argued by Costello \cite{CostelloM5}, arises from placing eleven-dimensional supergravity in the so-called `twisted $\Omega$-background'.
%We will use this theory for the sake of highlighting more accessible examples of the constructions later on in the paper.

%In the works \cite{CostelloM5,CostelloM2} Costello has introduced a program to analyze holography for membranes and fivebranes in the $\Omega$-deformation of nonminimally twisted supergravity defined on multi Taub--NUT spaces.
%The $\Omega$-deformation of the nonminimal twist of eleven dimensional supergravity was constructed directly from the physical theory and was argued to admit a description in terms of a holomorphic-topological gauge theory in five dimensions.
%We do not recall the precise relationship between this five-dimensional theory and eleven-dimensional supergravity, but it will be helpful to recall some features of the five-dimensional theory.

%This five-dimensional theory is formally similar to a non-commutative version of Chern--Simons theory.
%It exists on any manifold of the form
%\[
%  S\times Y
%\]
%where $Y$ is a complex surface equipped with a holomorphic symplectic structure and $S$ is a real oriented smooth one-dimensional manifold.
%We denote the corresponding holomorphic two-form on $Y$ by $\Omega_Y$.

%There are two parameters in this theory $\delta$ and $\ep$, see \textit{loc. cit.}.
%The parameter $\delta$ controls the coupling of the Chern--Simons action, which in this section and going forward we will simply set to be the unit.
%The parameter $\ep$ measures the non-commutativity---
%it appears through the formal deformation quantization of~$Y$ which is the associative algebra $\cO(Y)[[\ep]]$ equipped with the Moyal product $\star_\ep$.

%We consider the following subalgebra of complex-valued forms on $S\times Y$:
%\[
%\cL_{5d} \define \oplus_{k,q}\Omega^{k}(S)\hotimes \Omega^{0,q}(Y) \subset \Omega^\bu(S \times Y) .
%\]
%In local coordinates a general differential form in $\cL_{5d}$ admits a decomposition as
%\[
%\alpha = \alpha^{\Bar I} \d \zbar_{\Bar I} + \alpha^{\Bar I}_t \d t \d \zbar_{\Bar I} .
%\]
%We will use the notation $\alpha^{k;q}$ with $k+q=$ for the corresponding homogenous components.
%We can understand the Moyal product along the holomorphic symplectic surface $Y$ as endowing $\cL_{5d}[[\ep]]$ with the structure of a dg algebra.
%Of course, we can forget down to a dg Lie algebra structure with respect to the Moyal commutator $[-,-]_{\star_{\ep}}$.

%In the BV formalism the space of fields of the five-dimensional theory is $\cL_{5d}[[\ep]] [1]$.
%The fundamental field of the five-dimensional theory is a $\C[[\ep]]$ one-form on~$S \times Y$, which is a degree $0$ element in $\cL_{5d}[[\ep]] [1]$.

%We can take a gauge condition where the $\d t$ component of the one-form $\alpha_t$ vanishes.
%In this case the equations of motion for the one-form field, in components, simply reads
%\beqn
%\del_{\zbar_{\Bar{i}}} \alpha^{\Bar{j}} + \alpha^{\Bar{i}} \star_\ep \alpha^{\Bar{j}} = 0,
%\eeqn
%together with the condition that $\del_t \alpha^{\Bar{i}} = 0$ for all $\Bar{i}, \Bar{j}$.
%Here $\star_{\ep}$ denotes the holomorphic Moyal product on $Y$ and $\del_t$, $$ denote the de Rham and Dolbeault differentials along $S$,$Y$ respectively.

%We may get a better feeling for these equations after assuming that $\alpha^{1;0}$ is holomorphic $\alpha^{{1;0}}$ a VEV such that $\dbar\alpha^{{1;0}}=0$. 
%The second equation then reads
%
%\beqn
%0 = d\alpha^{0;1} + \{\alpha^{1;0}, \alpha^{0;1}\}
%\eeqn

%Gauge symmetries are given by degree $-1$ elements in $\cL_{5d}[[\ep]][1]$ which are simply functions $f \in C^\infty (S\times Y)[[\ep]]$.
%Explicitly, the corresponding infinitesimal gauge transformation for the one-form component of $\alpha$ is given by
%\[
%\alpha \mapsto \alpha + \del_t f \d t +\del_{\zbar_{\Bar{i}}} f \d \zbar_{\Bar{i}} + [f, \alpha]_{\star_\ep} .
%\]
%Here, $[-,-]_{\star_{\ep}}$ is the commutator with respect to the Moyal product $\star_\ep$.

%The full BV action functional describing this model is the non-commutative Chern--Simons action
%\beqn
%S(\alpha) = \int_{S \times Y} \left(\frac12 \alpha \d \alpha + \frac13 \alpha \star_\ep \alpha \star_\ep \alpha \right) \wedge \Omega_Y .
%\eeqn
%Here, $\alpha$ represents an arbitrary element of $\cL_{5d}[[\ep]]$ and $\d$ is the full de Rham differential on $S \times Y$ (notice that that no holomorphic derivatives appear in the kinetic term due to the appearance of $\Omega_Y$).

%In the BV formalism, the full space of BV fields together with equations of motion and gauge symmetries can described by a local dg Lie algebra on $S\times Y$ whose underlying sheaf of cochain complexes is
%\beqn
%\Omega^\bu(S) \hotimes \Omega^{0,\bu}(Y) .
%\eeqn
%The holomorphic poisson bracket $\{-,-\}_{Y}$ can be extended to define a Lie bracket on the entire Dolbeault complex $\Omega^{0,\bu}(Y)$.
%The dg lie algebra structure is then obtained by the tensor product with the commutative dg algebra $\Omega^\bu(S)$. 
%The BV pairing is given by integrating against the holomorphic volume form on $Y$.

\subsection{A more general background}
\label{s:thfmflds}
The eleven-dimensional model for twisted supergravity can be extended to more general geometries than products $S \times X$ where $S$ is a real one-dimensional manifold and $X$ is a Calabi--Yau fivefold.
More generally, we can define the model on an eleven-manifold $M$ equipped with a transversely holomorphic foliation (THF) equipped with the data of a non-vanishing volume form on the leaf space.
For general background on the theory of THFs we refer the reader to \cite{DuchampKalka, KamberTondeur, Rawnsley}.

A THF structure on a smooth manifold $M$ is an integrable subbundle $F \subset \T_M \otimes \C$ such that $F + \Bar{F} = \T_M \otimes \C$.
Equivalently, this allows you to choose an atlas of coordinates $(t_i, z_j)$ on $M$ which transform smoothly in the $t_i$-variables and holomorphically in the $z_j$-variables. 
Accordingly, we may take a local patch in a THF manifold to be of the form $\R^m \times \C^n$.
%Suppose $M$ is a manifold equipped with a THF structure and let $\cF$ be the corresponding foliation of even codimension.
%he product $M = S \times X$, where $X$ is a complex manifold and $S$ is a smooth manifold has a natural THF structure with $F$ the restriction of the tangent bundle of $M$ along the projection.
%Locally, any THF manifold is split of the form $\R^d \times \C^n$, whose coordinates we will denote by $(x_i ;  z_j)$.
The bundle $F$ is locally spanned by the vector fields $\partial / \partial t_i$'s and $\del/\del \zbar_j$'s.
(Notice that when $F \cap \Bar{F} = 0$ we are just describing an ordinary complex structure on $M$.)
We will say that $M$ is of dimension $(m,n)$ (so it is a manifold of real dimension $m + 2n$).

There is a notion of the sheaf of `holomorphic functions' $\cO_F$ on any manifold $M$ equipped with a THF~$F$---in the language of foliations these are the functions on $M$ which are constant along the leaves of $F$.
Locally, on a coordinate patch it is given by functions which only depend on the holomorphic variables $z_i$.
There is a resolution $\cA^\bu_F$ for $\cO_F$ by locally free sheaves on $M$ which is the THF analog of the de Rham and Dolbeault complexes. 
In degree zero $\cA^0_F$ is just the sheaf of smooth functions on $M$.
In degree $k$ one takes $\cA^k_F$ the sheaf of sections of the bundle $\wedge^k F^*$.
The derivative along the leaves of the foliation defined by $F$ defines a map of sheaves
\[
\thfd \colon \cA^{q}_F \to \cA^{q+1}_F  .
\]
By integrability one has $\thfd^2 = \thfd \circ \thfd = 0$ and one can show that the natural map $\cO_F \xto{\simeq} \cA^\bu_F$ provides a resolution.
% and so $\thfd$ equips $\cA^{\bu} = \oplus_q \cA^{p;q}[-q]$ with the structure of a cochain complex for each $p$.
Locally in a split THF structure like $\R^m \times \C^n$ the operator $\thfd$ is of the form $\d_{dR} + \dbar$ where $\d_{dR}$ is the de Rham differential along $\R^d$ and $\dbar$ is the Dolbeault operator along $\C^n$.


The fields of our eleven-dimensional theory may be described as sheaves of flat sections of some natural partially flat bundles on $M$. Associated to a foliation $F$, its normal bundle is the quotient bundle $Q = T^{\C}_{M}/F$. The normal bundle $Q$ and its dual $Q^{*}$ admit natural connections defined along the leaves of $F$, the so-called Bott connection \cite{KamberTondeur}.

These furnish the notion of the sheaf of holomorphic vector fields $\vartheta_F$ and a sheaf of holomorphic one-forms $\Omega^1_F$ on a manifold $M$ equipped with a THF~$F$. These are the sheaves of sections of $Q$ and $Q^{*}$ that are flat with respect to the Bott connection. Locally, on $\R^m \times \C^n$ these are of the form $f^i \del / \del z_i$ and $f^{i}dz_{i}$ where each $f^i \in \cO_F$.

Let $\cA^k_{F}(Q), \cA^{k}_{F}(Q^{*})$ be the sheaves of $C^\infty$-sections of the bundles $\wedge^k F^{*} \otimes Q$ and $\wedge^{k}F^{*}\otimes Q^{*}$ respectively. Then we have fine resolutions

\beqn
\vartheta_F \xto{\simeq} \cA_F^\bu(Q) \ \ \ \ \ \ \ \Omega^1_F \xto{\simeq}  \cA_F^\bu(Q^*)
\eeqn
The sheaf $\vartheta_F$ is equipped with a Lie bracket which extends to endow $\cA^\bu(Q)$ the structure of a dg Lie algebra. Likewise, the sheaf $\Omega^{1}_{F}$ carries the contragradient action of $\vartheta_{F}$ which likewise extends to equip $\cA^{\bu}(Q^{*})$ with the structure of a dg-module over $\cA_{F}^{\bu}(Q)$.

The holomorphic deRham and divergence operators also have analogues in the THF setting, which are now defined on the leaf space. First, for a vector bundle with  $\cA_{F}^{p,q}(Q)$ denote the sheaf of $C^{\infty}$ sections of $\wedge^{q} F^{*}\otimes \wedge^p Q$.
The fields of the eleven-dimensional theory, which we phrased in terms of a mixed type of de Rham and Dolbeault cohomology, can now be elegantly repackaged in terms of the above.

For illustration, let us focus on the fields $\beta,\gamma$ which on $S \times X$ combine to form the complex
\beqn\label{eqn:drdol}
\Omega^{\bu}(S) \otimes \Omega^{0,\bu}(X) \xto{1 \otimes \del} \Omega^{\bu}(S) \otimes \Omega^{1,\bu}(X) .
\eeqn
As usual, we leave the $\d_{dR}$ and $\dbar$ operators implicit.
There is an analog of this operator $1 \otimes \del$ on a manifold $M$ equipped with a THF~$F$.
It is a differential operator 
\beqn
D \colon \cA^{p,\bu}_F \to \cA^{p+1,\bu}_F
\eeqn
which locally is just the holomorphic de Rham operator. 
This operator commutes with $\thfd$.
The generalized $\beta$ and $\gamma$ fields will be sections of the two-term complex of sheaves $\cA^\bu_F \xto{D} \cA^{1,\bu}_F$.

\newcommand\Div{D_\Omega}

To make sense of a THF analogue of the holomorphic divergence operator, we will assume that in addition to having a THF structure that $M$ is equipped with the data of a trivialization $\wedge^{n} Q^* \cong \C_{M}$ which we will denote by~$\Omega_F$. We require that the trivialization is constant along the leaves of $F$, i.e. that $D\Omega_{F} = 0$. If $F \cap \Bar{F} = 0$, then $\Omega_F$ returns the usual notion of a non-vanishing holomorphic top form. For our purposes, this is the appropriate notion of a transverse Calabi--Yau structure

Next we should say where the generalized $\mu$ and $\nu$ fields live. 
The volume form $\Omega_F \colon \C \cong \wedge^n Q^*$ determines an isomorphism $\cA^{p,q}_F \cong \cA^{n-p,q}_F$ for each $p,q$.
Via this isomorphism we can identify $D \colon \cA^{n-1,\bu}_F \to \cA^{n,\bu}_F$ with a differential operator
\beqn
\Div \colon \cA_F^{\bu}(Q) \to \cA^\bu_F .
\eeqn
The $\mu$ and $\nu$ fields will be sections of the two-term complex of sheaves $\cA^\bu_F(\theta_F) \xto{\Div} \cA^\bu_F$. 
This resolves the THF version of the sheaf of holomorphic vector fields on $M$ which preserve the form $\Omega_F$. 

Now we are in a place to describe a BV theory defined on an eleven-manifold $M$ equipped with a THF $F$ and a volume form $\Omega_F$ on the leaf space.
The space of fields of the theory consist of fields $\beta,\gamma, \mu, \nu$ as before but where
\beqn
\beta \in \Pi \cA_F^\bu , \quad \gamma \in \cA^{\bu}_F (Q^*) , \quad \mu \in \Pi \cA^\bu_F(Q), \quad \nu \in \cA_F^\bu  .
\eeqn

Notice that there is a natural odd pairing between the fields $\mu, \gamma$ and $\beta, \nu$.

For convenience, we will let $L$ denote the dg vector bundle
\beqn
\Pi (\C \oplus Q)\to Q^* \oplus \C
\eeqn

with differential given by the block matrix $\operatorname{diag}(D, \Div)$ and succinctly write the space of fields as $\cA^{\bu}_{F}(L)$.

The free part of the action functional (cf. \eqref{eqn:free}) is 
\beqn
\int_M (\gamma \thfd \mu) \Omega_F + \int_M (\beta \thfd \gamma) \Omega_F + \int_M (\beta \Div \mu) \Omega_F .
\eeqn
The interaction (cf. \eqref{eqn:int}) is
\beqn
\label{eqn:int}
\frac12 \int_{M} (\mu^2 D \gamma) \frac{\Omega_F}{1- \nu} + \frac16 \int_{M} \gamma D \gamma D \gamma .
\eeqn
If we assume that $\nu = \Div \mu = 0$ we recover the following equation of motion upon varying $\gamma$
\beqn
\thfd \mu + \frac12 [\mu,\mu] = D \gamma D \gamma .
\eeqn
This is a generalization of equation \eqref{eqn:eom2} for a general THF $F$ on the eleven-manifold~$M$.

\subsection{Twisted fivebranes}

It is expected that the worldvolume theory of fivebranes in $M$-theory on flat space is equipped with $\cN=(2,0)$ superconformal symmetry.
Six-dimensional $\cN=(2,0)$ supersymmetry admits two inequivalent classes of twists characterized by the number of directions which are left invariant:
\begin{itemize}
\item
The holomorphic, or minimal, twist.
This twist leaves three real directions invariant and is stabilized by the double cover of the group $U(3)$.
The holomorphic twist of any $\cN=(2,0)$ theory can be defined on any complex three-fold $Z$ equipped with a square-root of its canonical bundle.
\item
The non-minimal twist.
This twist leaves five real directions invariant and is stabilized by the group $SO(4) \times U(1)$.
The non-minimal twist of any $\cN=(2,0)$ theory can be defined on a six-manifold of the form $M^4 \times \Sigma$ where $M^4$ is a smooth four-manifold and $\Sigma$ is a Riemann surface.
\end{itemize}

An explicit characterization of these twists in the case of the stack of a single fivebrane has been given in \cite{SWtensor}.
We recall the description of the minimal twist.

\parsec[s:single]

The holomorphic twist of the fivebrane theory is defined on any complex three-fold $Z$ (which is not necessarily equipped with a Calabi--Yau structure).
In the particular twist we will use we must assume, however, that $Z$ is equipped with a square-root of the canonical bundle $K_Z^{1/2}$.

The theory is $\Z \times \Z/2$ graded where $\Z$ is the ghost number (or cohomological degree) and $\Z/2$ is parity.
There are four fundamental fields of ghost number zero $(\alpha, \omega, \phi_1,\phi_2)$ which consist of even Dolbeault forms of type $(2,1)$ and $(3,0)$ on $Z$:
\[
\alpha \in \Omega^{2,1}(Z), \quad \omega \in \Omega^{3,0}(Z),
\]
and a pair of odd $(0,1)$ forms twisted by the line bundle $K^{1/2}_Z$:
\[
(\phi_1,\phi_2) \in \Pi \Omega^{0,1}(Z , K^{1/2}_Z) \otimes \C^2 .
\]
%where $R \cong \CC^2$ is the fundamental $Sp(1)$ representation.
The equations of motion read
\beqn
\label{eqn:eom}
\begin{split}
\del \alpha + \dbar \omega & = 0 \\
\dbar \alpha = \dbar \phi_i & = 0 .
\end{split}
\eeqn

There are gauge symmetries for the fields $\mu, \Omega$ determined by a ghost $b$ which is a Dolbeault form of type $(2,0)$ which acts simply by
\beqn
\label{eqn:ghost}
\begin{split}
\mu & \mapsto \mu + \dbar b  \\
\Omega & \mapsto \Omega + \del b .
\end{split}
\eeqn
There are also odd gauge symmetries for the odd fields $\phi_i$ given by
\beqn
\phi_i \mapsto \phi_i + \dbar \chi
\eeqn
where $\chi \in \Omega^0(Z, K_{Z}^{1/2}) \otimes \C^2$.

This theory notoriously does not admit a Lagrangian description.
However, it can still be put in a degenerate form of the BV formalism where the space of fields above is equipped with an odd Poisson bivector \cite{SWtensor}.
In the (degenerate) BV formalism, the space of fields of the theory is
\beqn
\cE_{\lie{gl}(1)} = \Omega^{\geq 2, \bu}(Z)[1] \oplus \Pi \Omega^{0,\bu}(Z, K^{1/2}_Z) \otimes \C^2 [1] ,
\eeqn
and the linear BRST differential is described by the following diagram
\beqn\label{eqn:weight-1a}
\begin{tikzcd}
\ul{-1} & \ul{0}\\
\Pi \Omega^{0,\bu}(Z, K_Z^{1/2} \otimes \C^2) & \\
\Omega^{2,\bu}(Z) \ar[r,"\del"] & \Omega^{3,\bu}(Z) ,
\end{tikzcd}
\eeqn
where the $\dbar$ operator acts implicitly on each summand.

%More generally, there is a natural cohomology associated to a THF structure.
%Suppose $(M,F)$ is a THF structure and
%denote by $Q$ the (complex) quotient bundle $\T_\C M / F$.
%For each $p,q$ denote by $\cA^{p;q}$ smooth sections of the bundle $\wedge^p Q^\vee \otimes \wedge^q F^\vee$.
%The derivative along the leaves of the foliation defined by $V$ defines a map
%\[
%\thfd \colon \cA^{p;q} \to \cA^{p;q+1}  .
%\]
%By integrability one has $\thfd^2 = \thfd \circ \thfd = 0$ and so $\thfd$ equips $\cA^{p;\bu} = \oplus_q \cA^{p;q}[-q]$ with the structure of a cochain complex for each $p$.
%Locally in a split THF structure the operator $D$ is of the form $\d_{dR} + \dbar$ where $\d_{dR}$ is the de Rham differential along $\R^d$ and $\dbar$ is the Dolbeault operator along $\C^n$.
%There is also an analog of the holomorphic $\del$ operator which takes the form $\thfdel \colon \cA^{p;q} \to \cA^{p+1;q}$.
%The obvious exterior product $\cA^{p;q} \times \cA^{r;s} \to \cA^{p+r;q+s}$ further endows
%\[
%\left(\cA^{\bu;\bu} (M), \thfd + \thfdel\right) = \left(\oplus_p \cA^{p;\bu}[-p] , \thfd + \thfdel \right)
%\]
%with the structure of a bigraded commutative dg algebra.
%This complex is simply isomorphic to the de Rham complex of $M$, but this presentation lends itself to more interesting quotient complexes.
%For example, the forms of type $(p,\bu)$ with $p \geq 2$ form an ideal inside of this dg algebra; hence we get a quotient dg algebra
%\beqn\label{thfcoh1}
%\left(\cA^{\leq 1;\bu}(M), \thfd + \thfdel\right) = \quad \cA^{0;\bu} \xto{\thfdel} \cA^{1;\bu} .
%\eeqn
%We leave the $\thfd$ operator implicit in the presentation on the right hand side.
%When $M = M_V$, it is this complex that is the THF generalization of the truncated de Rham--Dolbeault complex in \eqref{eqn:drdol}---it is easy to see that it agrees with this complex in the case of a split THF manifold.
%There is a similar THF description for the fields $\mu,\nu$ in the eleven-dimensional theory.
%
%Any submanifold of a THF manifold is itself a THF manifold.
%We are most interested in the submanifold $M_V \subset {\rm Tot}(\R \oplus V) = \R \times X$ where we equip $\R \times X$ with its standard split THF structure.
%
%\end{document}
