\subsection{Large $N$ Holomorphic Character}

We begin by using the twisted holography proposal to compute an index counting local operators of the holomorphic twist of the theory on a large number of M5 branes. Let us briefly recall the proposal specialized to our situation \surya{this should go elsewhere}. Let $\cL$ denote the local $L_{\infty}$ algebra underlying the minimal twist of 11 dimensional supergravity on $\R\times \C^{3}_{z_{i}}\times \C^{2}_{w_{a}}$. Consider a stack of $N$ M5 branes supported at the origin in $\R\times \C^{2}_{w_{a}}$ and let $\iota : \{0\}\times \C^{3} \times \{0\}\to \R\times \C^{5}$ denote the inclusion of their support. The twisted holography proposal asserts that the algebra of local operators of the minimally twisted M5 brane theory in the large $N$ limit is a deformation of the factorization envelope $\mc U(\mc L) = C_{\bullet}\left (\iota^{*}\cL)_{c} \right )$.

Since quantities like indices are deformation invariants, we need not consider such deformations for now. In fact we may replace $\cL$ with the abelian local dg Lie algebra describing the free limit. Explicitly, $\cL$ is given by the following sheaf of $\Z/2$-graded cochain complexes

\[
\begin{tikzcd}
  &  \ul{\rm odd} & \ul{\rm even} \\
 \mu \in & \Omega^{\bullet}_{\R}\otimes\PV^{1,\bullet}_{C^{5}} \ar[r, "\del_{\Omega}"] & \Omega^{\bullet}_{\R}\otimes\PV^{0,\bullet}_{\C^{5}}\\
 \gamma \in & \Omega^{\bullet}_{\R}\otimes\Omega^{1,\bullet}_{\C^{5}}& \Omega^{\bullet}_{\R}\otimes\Omega^{0,\bullet}_{\C^{5}} \ar[l, "\del"]
\end{tikzcd}
.\]


\parsec
We begin by computing the hypercohomology of the factorization envelope. Strictly speaking, this is not necessary for computing the index, but will be useful in later calculations.

We first introduce the following notation for representatives of classes in compactly supported Dolbeault cohomology. We write $\delta^{(a,b,c)}$ to denote the derivative of the delta function $\del_{z_{1}}^{a}\del_{z_{2}}^{b}\del_{z_{3}}^{c}\delta_{t=w_{a}=0}$.

\begin{lem}
  The hypercohomology of the factorization envelope is given by \[\sym (V_{\gamma}\oplus \Pi V_{\mu})\] where
  \begin{itemize}
    \item $V_{\gamma}$ is spanned by
          \begin{align}
            \delta^{(0,b,c)}w_{1}^{k}w_{2}^{l}dz_{1} \\
            \delta^{(a,b,c)}w_{1}^{k}w_{2}^{l}dz_{2} \\
            \delta^{(a,b,c)}w_{1}^{k}w_{2}^{l}dz_{3} \\
            \delta^{(a,b,c)}w_{1}^{k}w_{2}^{l}dw_{1}\\
            \delta^{(a,b,c)}w_{1}^{k}w_{2}^{l}dw_{2}
          \end{align}

          where $a, b, c, k, l \geq 0$.
    \item $V_{\mu}$ is spanned by
          \begin{align}

          \end{align}
  \end{itemize}
\end{lem}

Here, we have introduced the notation $\delta^{(a,b,c)}$

The index we compute will have four fugacities, which we label as follows:
\begin{itemize}
  \item $t_{1}, t_{2}$ denote generators for the cartan of $\SU(3)$ which acts on $\C^{3}_{z}$.
  \item $r$ denotes a generator for the cartan of a $\SU(2)$ rotating the normal directions $\C^{2}_{w}$
  \item $q$ denotes a generator for the cartan of an additional $U(1)$ acting on $\C^{5}$ preserving the volume form
\end{itemize}

A convenient choice of choice of weights is as in the following table:

\begin{center}
\begin{tabular}{c c c c c c}
  & $z_{1}$ & $z_{2}$ & $z_{3}$ & $w_{1}$ & $w_{2}$ \\
  \hline
  $t_{1}$ & 1 & 0 & -1 & 0 & 0 \\
  $t_{2}$ & 0 & 1 & -1 & 0 & 0 \\
  $r$ & 0 & 0 & 0 & 1 & -1 \\
  $q$ & -1 & -1 & -1 & $\frac{3}{2}$ & $\frac{3}{2}$
\end{tabular}
\end{center}
