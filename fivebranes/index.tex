%\documentclass[11pt]{amsart}

%\usepackage{../macros-master}
%\usepackage{macros-fivebrane}

%\begin{document}

\section{Twisted supergravity states}

operators, indices, etc..
relationship to the $S^5$ partition function.

\subsection{Local operators in a holomorphic-topological theory}

Generally speaking, twists of supersymmetric theories have a mixed holomorphic and topological behavior.


\parsec[]

%Suppose that $V$ is a translation invariant holomorphic vector bundle on $\C^n$ equipped with a $\Z \times \Z/2$ bigrading. 
%Let $\cV$ denote its sheaf of holomorphic sections.
%The {\em space of fields} of a holomorphic field theory on $\C^n$ is the $\Z \times \Z/2$ graded complex of vector bundles
%\[
%\Omega^{0,\bu}(\C^n, V) \cong \Omega^{0,\bu}(\C^n) \otimes V_0 
%\]
%where $V_0$ is the fiber of $V$ at $0 \in \C^n$.
%Our grading conventions are so that $\d \zbar_i$ has bidegree $(1,0)$.
%
%As introduced in \cite{BWhol,LiVertex,CG2}, a {\em holomorphic field theory} is a holomorphic vector bundle $V$ as above equipped additionally with:
%\begin{itemize}
%\item The structure of a local (super) $L_\infty$ algebra on $V[-1]$ with structure maps given by holomorphic polydifferential operators
%\[
%[\cdot]_k \colon \cV[-1]^{\times k} \to \cV[1-k] .
%\]
%\end{itemize}
%A {\em free} holomorphic theory has $[\cdot]_k = 0$ for $k > 1$.
%
%\parsec
In part, the data of a holomorphic-topological theory in the BV formalism is prescribed by a holomorphic-topological local $L_\infty$ algebra $\cL$.
Without loss of generality we will assume our holomorphic-topological local $L_\infty$ algebra on $M \times X$ is of the form
\beqn\label{eqn:cL}
\cL = \Omega^\bu (M) \hotimes \Omega^{0,\bu}(X, L) 
\eeqn
where $L$ is a graded holomorphic vector bundle on $X$.
The differential is required to be of the form
\[
\d_{dR} + \dbar + Q^{hol}
\]
where $\d_{dR}$ is the de Rham differential acting on $M$, $\dbar$ is the Dolbeault operator acting on $X$, and $Q^{hol} \colon L \to L[1]$ is a holomorphic differential operator of cohomological degree $+1$. 
The higher $L_\infty$ structure maps of the local $L_\infty$ algebra are required to be given by holomorphic polydifferential operators. 
For more details we refer to the definitions in \cite{GRWthf}.

In this situation, the factorization algebra of classical observables supported on an open set $U \subset M \times X$ is given by the Chevalley--Eilenberg cochains on the $L_\infty$ algebra $\cL(U)$. 
We will give a concise presentation for the {\em local} operators in a topological-holomorphic theory. 

Suppose that $M \times X = \R^m \times \C^n$ and suppose that the bundle $L \to \C^n$ is translation invariant with fiber $L_0$ over $0 \in \R^m \times \C^n$.
We also assume that all structure maps of the $L_\infty$ algebra are translation invariant.
The jet expansion at $0 \in \R^m \times \C^n$ determines a map of $L_\infty$ algebras
\[
\cL(\C^n \times \R^m) \to L_0 [[x_i, \d x_i,z_j, \zbar_j, \d \zbar_j]] 
\]
The differential on the right hand side is $\d_{dR} + \dbar + Q^{hol} = \d x_i \del_{x_i} + \d \zbar_j \del_{\zbar_j} + Q^{hol}$ where $Q^{hol}$ is some holomorphic differential operator in the $z_j$ variables. 
Since all structure maps are given by holomorphic polydifferential operators, the canonical map 
\[
L_0 [[x_i, \d x_i,z_j, \zbar_j, \d \zbar_j]] \xto{\simeq} L_0 [[z_j]] 
\]
which sends $x_i, \d x_i,\zbar_j \d \zbar_j \mapsto 0$ is a quasi-isomorphism. 
The only remaining differential on the right hand side is~$Q^{hol}$. 
In summary, we see that the jet expansion at $0 \in \R^m \times \C^n$ determines a map of $L_\infty$ algebras $\cL(\R^m \times \C^n) \to L_0[[z_j]]$. 
 
\begin{lem}
\label{lem:localops}
Suppose that $\cL$ is a holomorphic-topological local Lie algebra on $\R^m \times \C^n$ and consider the factorization algebra $\Obs = \clie^\bu (\cL)$. 
Then, the Taylor expansion map
\[
\cL(\C^n \times \R^m) \to L_0[[z_0,\ldots,z_n]]
\]
induces a quasi-isomorphism of commutative dg algebras
\[
\Obs(0) \simeq \clie^\bu \left( L_0[[z_1,\ldots,z_n]] \right) .
\]
Notice that when $\cL$ is abelian with differential $\d_{dR} + \dbar + Q^{hol}$, then there is a quasi-isomorphism
\[
\Obs(0) \simeq \cO \left( L_0[[z_1,\ldots,z_n]][1] \right) 
\]
where the right hand side is equipped with the differential $Q^{hol}$. 
\end{lem}

We recall the reader of the standard dictionary between the space of fields of a BV theory and the local $L_\infty$ algebra---
if the local Lie algebra is $\cL$, then the space of fields is $\cL[1]$. 
The Chevalley--Eilenberg complex of $\cL$ is then functions on the fields $\cO(\cL[1])$ equipped with the non-linear BRST operator.

\parsec[s:pexp]

\brian{characters, plethystic, etc.}

\parsec[s:localops]

We present some simple examples. 

Suppose that $L$ is the trivial bundle on $\C^n$ and consider the abelian holomorphic-topological local Lie algebra 
\[
\Omega^\bu(\R^m) \otimes \Omega^{0,\bu}(\C^n) [-1]
\]
where the differential is just $\d_{dR} + \dbar$. 
Notice that with the shift, functions on $\R^m \times \C^n$ are in cohomological degree $+1$.
Then, the space of local operators is the symmetric algebra on the topological vector space which is linear dual to 
\[
\cO^{hol}(\Hat{D}^n) = \C[[z_1,\ldots,z_n]] .
\]
Via the $n$-dimensional residue one can identify the algebra of local operators with 
\[
\Sym\left(\frac{\d z_1}{z_1} \cdots \frac{\d z_n}{z_n}  \C[z_1^{-1}, \ldots , z_n^{-1}]\right) .
\]

Consider the standard torus action $\C^\times \times \cdots \C^\times$ on $\C^n$. 
The equivariant character of local operators with respect to this symmetry is given by the plethystic exponential of the following single particle index
\[
\frac{1}{(1-q_1)\cdots (1-q_n)} .
\]


%\parsec[s:sugraops]
%
%By the usual methods of the BV formalism the action functional $S_{sugra}$ described above endows the parity shift of the fields $\cL_{sugra} = \Pi \cF_{sugra}$ with the structure of a holomorphic-topological local $\Z/2$ graded $L_\infty$ algebra. 
%
%On $\C^5 \times \R$ we can describe this super Lie algebra structure explicitly. 
%First, by the Dolbeault and de Rham Poincar\'e lemmas it is easy that the even part of the super Lie algebra $\cL(\C^5 \times \R)$ is equivalent to a one-dimensional central summand $\C$ plus the Lie algebra of divergence-free vector fields on $\C^5$:
%\[
%\Vect_0 (\C^5) = \{X \in \Vect(\C^5) \; | \; \div X = 0\} .
%\]
%The odd part of the super Lie algebra $\cL(\C^5 \times \R)$ is equivalent to the space of holomorphic one-forms on $\C^5$ modulo exact one-forms
%\[
%\Omega^{1,hol}(\C^5) / {\rm Im}(\del) 
%\]
%which is, of course, equivalent to the space of closed holomorphic two-forms $\Omega^{2,hol}_{cl}(\C^5)$. 
%
%\begin{thm}[\cite{RSW}[Theorem 2.1]]
%The Taylor expansion map determines a map of $\Z/2$ graded $L_\infty$ algebras
%\[
%j_\infty \colon \cL_{sugra}(\C^5 \times \R) \to L_{sugra} .
%\]
%Furthermore, $L_{sugra}$ is equivalent as a $\Z/2$ graded $L_\infty$ algebra to $\Hat{E(5|10)}$. 
%\end{thm} 
%
%As an immediate corollary of this result we obtain by Lemma \ref{lem:localops} the following.
%
%\begin{cor}
%\label{cor:sugraops}
%Let $\Obs_{sugra}$ be the factorization algebra on $\C^5 \times \R$ of classical observables of the minimal twist of eleven-dimensional supergravity.
%There is a quasi-isomorphism of commutative dg algebras
%\[
%\Obs_{sugra} (0) \simeq \clie^\bu \left( \Hat{E(5|10)} \right) .
%\]
%\end{cor}

\subsection{Global symmetry for twisted fivebranes}

%In \cite{SW6d} a symmetry by the exceptional super Lie algebra $E(3|6)$ on the holomorphic twist of the free tensor multiplet was constructed.
%We recall the action of its even part on local operators of the holomorphic twist of the free tensor multiplet. 
%We recount how it acts on the vector space \eqref{eqn:localfree}.
% 
%Recall that the even part of $E(3|6)$ consists of two summands:
%\[
%\Vect(\Hat{D}^3) \times \lie{sl}(2) \otimes \cO(\Hat{D}^3) .
%\]
%Formal vector fields act naturally by Lie derivative on both summands in \eqref{eqn:localfree}. 
%An $\lie{sl}(2)$-valued function acts on $\Gamma^{hol}(\Hat{D}^3, K^{1/2}) \otimes \C^2$ where we view $\C^2 = {\rm span}\{r_+,r_-\}$ as the fundamental $\lie{sl}(2)$ representation. 

After complexification, the~six-dimensional superconformal algebra is $\lie{osp}(8|4)$ whose even part is $\lie{so}(8) \times \lie{sp}(4)$. 
This algebra contains the $\cN=(2,0)$ supersymmetry algebra whose odd part is four copies of $S_+$, the positive irreducible complex spin representation of $\lie{so}(6)$.
The holomorphic supercharge is a supertranslation 
\[
Q \in \Pi S_+ \otimes \C^4 \subset \lie{osp}(8|4)
\]
which is characterized (up to equivalence) by the properties that $Q^2 = 0$ and that its image
\[
{\rm Im}\left(Q|_{\Pi S_+ \otimes \C^4} \right) \subset \R^6 \otimes_\R \C \cong \C^6
\]
is three-dimensional (spanned by the anti-holomorphic translations). 
The supercharge $Q$ acts on $\lie{osp}(8|4)$ by commutator and the resulting cohomology will automatically act on the holomorphic twist of any six-dimensional superconformal field theory. 
This cohomology can readily be identified with the subalgebra $\lie{osp}(6|2)$, see \cite{SWe36}. 

To compute characters of local operators the fivebrane worldvolume theory we will consider the symmetry of the holomorphic twist of a six-dimensional superconormal theory with respect to the even subalgebra 
\beqn\label{eqn:bosonic1}
\lie{sl}(3) \times \lie{sl}(2) \times \lie{gl}(1) \subset \lie{osp}(6|2) .
\eeqn
We will understand how the twisted superconformal algebra acts on fivebranes via holographic principles.

In the untwisted situation, the symmetry of solutions to eleven-dimensional supergravity in the presence of fivebranes wrapping a six-dimensional affine subspace is exactly the superconformal algebra $\lie{osp}(8|4)$ (after complexification). 
Geometrically, this background is $AdS_7 \times S^4$. 
In \cite{RSW} we have proposed a twisted analog of the AdS background and have shown that solutions to equations of motion of our eleven-dimensional theory in the presence of twisted fivebranes contains the symmetry algebra $\lie{osp}(6|2)$---which is precisely the twisted superconformal algebra.

We recall how the bosonic subalgebra \eqref{eqn:bosonic1} of this twisted superconformal algebra embeds as symmetries of the eleven-dimensional theory in this twisted background. 
This manifests as an embedding of this bosonic subalgebra into the ghosts of the eleven-dimensional theory. 

For convenience we choose coordinates on the eleven manifold as
\[
\C^5 \times \R = \C_z^3 \times \C_w^2 \times \R_t
\]
with $z_i, i=1,2,3$ and $w_a, a=1,2$.
The stack of fivebranes wrap $w_1=w_2=t=0$. 
Important for us is to recall that part of the ghost system for our eleven-dimensional theory consists of divergence-free vector fields on $\C^5$ which are locally constant along $\R$. 

\begin{itemize}
\item
The subalgebra $\lie{sl}(3)$ embeds as vector fields
\[
\sum_{ij} A_{ij} z_i \frac{\del}{\del z_j} \in \PV^{1,0}(\C^5)\otimes \Omega^0(\R) , \quad (A_{ij}) \in \lie{sl}(3) .
\]
\item
The subalgebra $\lie{gl}(1)$ is mapped to the element
\[
\sum_{i=1}^3 z_i \frac{\del}{\del z_i} - \frac32 \sum_{a=1}^2 w_a \frac{\del}{\del w_a} \in \PV^{1,0}(\C^5) \otimes \Omega^0(\R)  .
\] 
Notice that this vector field are divergence-free and restricts to the ordinary dilation (Euler vector field) along $w=0$. 
\item 
The subalgebra $\lie{sl}(2)$ ($R$-symmetry) is mapped to the triple
\[
w_1 \frac{\del}{\del w_2}, w_2 \frac{\del}{\del w_1}, \frac12 \left(w_1 \frac{\del}{\del w_1} - w_2 \frac{\del}{\del w_2}\right) \in \PV^{1,0}(\C^5) \otimes \Omega^0(\R) .
\]
\end{itemize}

\begin{rmk}
In the classification of simple super Lie algebras, Kac makes use of a weight grading $\oplus_{j \geq -2} \fg_j$ of $E(3|6)$ for which the finite-dimensional subalgebra above is the weight zero piece
\cite{KacClass}.
\end{rmk}

The dimension of a Cartan subalgebra is four and accordingly, the equivariant character we study has four fugacities.
We choose these explicitly as follows:
\begin{itemize}
  \item $t_{1}, t_{2}$ denote generators for the Cartan of $\lie{sl}(3)$ which is generated by the vector fields
  \beqn
  z_1 \del_{z_1} - z_2 \del_{z_2} , \quad z_2 \del_{z_2} - z_3 \del_{z_3}
  \eeqn
  on~$\C^{3}_{z}$.
  \item $r$ denotes a generator for the Cartan of a $\lie{sl}(2)$ acting by the fundamental representation on $\C^{2} = {\rm span}\{r_+,r_-\}$. 
  \item $q$ denotes a generator for the Cartan of an additional $\lie{gl}(1)$ which acts by dilations on $\C^3$. 
\end{itemize}
The weights are summarized in table \ref{tbl:weights1}.
Recall that the coordinates $(r_1,r_2)$ are used to label the basis of $\C^2$ in the space of fields $\Gamma^{hol}(\Hat{D}^3, K^{1/2}) \otimes \C^2$. 

\begin{table}
\begin{center}
\begin{tabular}{c c c c c c}
  & $z_{1}$ & $z_{2}$ & $z_{3}$ & $r_{1}$ & $r_{2}$ \\
  \hline
  $t_{1}$ & $1$ & $-1$ & $0$ & $0$ & $0$ \\
  $t_{2}$ & $0$ & $1$ & $-1$ & $0$ & $0$ \\
  $r$ & $0$ & $0$ & $0$ & $1$ & $-1$ \\
  $q$ & $-1$ & $-1$ & $-1$ & $0$ & $0$
\end{tabular}
\caption{Fugacities for a single fivebrane.}
\end{center}
\label{tbl:weights1}
\end{table}



\subsection{Operators on a single fivebrane}

We recalled the description of the holomorphic twist of the theory on a single fivebrane in \ref{s:single}. 
This theory exists on any complex three-fold $X$. 
To describe the algebra of local operators we specialize to $X = \C^3$ and consider the local operators supported at $0 \in \C^3$. 

The theory on a single fivebrane is free and the underlying cochain complex of the abelian holomorphic local Lie algebra on $\C^3$ is given by 
\beqn
\begin{tikzcd}
\ul{-1} & \ul{0} \\
\Omega^{2,\bu}(\C^3) \ar[r,"\del"] & \Omega^{3,\bu}(\C^3) \\
\Pi \Omega^{0,\bu}(\C^3, K_{\C^3}^{1/2} \otimes \C^2) . 
\end{tikzcd} 
\eeqn
Here we recall the $\Z \times \Z/2$ bigrading where the differential has bidegree $(1,0)$. 


\begin{lem}
\label{lem:single}
The $\Z \times \Z/2$ algebra of local operators of the holomorphic single fivebrane theory is quasi-isomorphic to the graded symmetric algebra on the linear dual of the topological vector space
\beqn\label{eqn:localfree}
\Omega^{2}_{cl} (\Hat{D}^3)[1] \oplus \Pi \Omega^0(\Hat{D}^3, K^{1/2}) \otimes \C^2 [1].
\eeqn
\end{lem}

\begin{proof}
The jet expansion at $0 \in \C^3$ determines a map from the section of the abelian holomorphic-topological local Lie algebra on $\C^3$ to the cochain complex
\beqn
\begin{tikzcd}
\ul{-1} & \ul{0} \\
\Omega^{2}(\Hat{D}^3) \ar[r,"\del"] & \Omega^{3}(\Hat{D}^3) \\
\Pi \Omega^0(\Hat{D}^3, K^{1/2} \otimes \C^2) . 
\end{tikzcd} 
\eeqn
On the formal disk all closed two-forms are automatically exact, which implies the lemma.
\end{proof}

\parsec

We now move on to the computation of the character of local operators on a single fivebrane.
According to the weights listed above and using the description of local operators in Lemma \label{lem:single} we have the following contributions to the character.

\begin{itemize}
\item Single particle operators on the odd copy of holomorphic two-forms $\Pi \Omega^{2,hol}$ contribute
\[
- q^2 \frac{t_1  + t_1^{-1} t_2  + t_2^{-1} }{(1-t_1^{-1}q) (1-t_1 t_2^{-1} q) (1-t_2 q)} 
\]
\item Single particle operators on the even copy of holomorphic three-forms $\Omega^{3,hol}$ contribute
\[
q^3 \frac{1}{(1-t_1^{-1}q) (1-t_1 t_2^{-1} q) (1-t_2 q)} 
\]
\item Single particle operators on $K^{1/2} \otimes \C^2$ contribute
\[
q^{3/2}\frac{(r + r^{-1})}{(1-t_1^{-1}q) (1-t_1 t_2^{-1} q) (1-t_2 q)}
\]
\end{itemize}

Putting this all together we obtain the following.

\begin{prop}
The character of local operators of the holomorphic twist of the theory on a single fivebrane is given by the following plethystic exponential
\[
{\rm PExp} \left[f_{tensor} (t_1,t_2,q,r) \right] .
\]
where the single particle index is
\[
f_{tensor} (t_1,t_2,q,r) = \frac{(r + r^{-1})q^{3/2} - (t_1 + t_1^{-1} t_2 + t_2^{-1} )q^2 + q^3}{(1-t_1^{-1}q) (1-t_1 t_2^{-1} q) (1-t_2 q)} .
\]
\end{prop}

\parsec

A degeneration of this character was computed in \cite{KR1}. 
They compute the absolute (non-super) character of the module $I(0,0;1;-1)$ where they additionally specialize $t_1=t_2=r=1$. 
In a similar method to the one used in \cite{KR1}, one can compute the specialized (super) character of $I(0,0;1;-1)$ to find
\[
\frac{2 q^{3/2} - 3 q^2 + q^3}{(1-q)^3}
\]
which agrees with our direct analysis above.

\parsec

There are various specializations of this character that one can consider. 
One which is motivated from supersymmetry produces...

\begin{itemize}
\item The deformation $\d z_1 \otimes r_+$.
\item The deformation $(z_2 \d z_3 - z_3 \d z_2) \otimes r_-$. 
\end{itemize}

Both deformations break the global Cartan subgroup down to $U(1) \times U(1)$ according to the specializations
\beqn\label{eqn:special1}
q = r^2 , \quad t_2 = 1 .
\eeqn
As one can easily check, this specialization yields the following single particle index
\[
f_{tensor}(t_1, 1, q, q^{1/2}) = \frac{q}{1-q} 
\]
which recovers the single particle index of a single chiral boson on the Riemann surface $\Sigma = \C_{z_1}$. 
Notice that although the Cartan subalgebra generated by the vector field $z_1 \del_{z_1} - z_2 \del_{z_2}$ is unbroken by this deformation, the dependence on its fugacity $t_1$ completely drops out of the expression.

%\parsec[s:singleops]
%
%\begin{prop}
%Let $\Obs_{fivebrane}$ be the observables of the theory on a single fivebrane on $\C^3$ after performing the holomorphic twist.
%There is a quasi-isomorphism of super commutative dg algebras 
%\[
%\Obs_{fivebrane} (0) \simeq \Sym \left( I(0,0;1;-1)^* \right) .
%\]
%\end{prop}
%\begin{proof}
%This follows from Lemma \ref{lem:localops}.
%In \cite{SWsuca6d} a dg lift of Kac's module $I(0,0;1;-1)$ was defined with the following bigraded pieces:
%\beqn
%\text{bidegree} (-1,+)\colon \quad \Hat{\Omega}_3^{2} , \qquad \text{bidegree} (0,+)\colon \quad \Hat{\Omega}^{3}_3
%\eeqn
%and
%\beqn
%\text{bidegree} (-1, -)\colon \quad \Hat{\Omega}^{3/2}_{3} \otimes R .
%\eeqn
%The differential is the (algebraic) de Rham differential $\del \colon \Hat{\Omega}^2 \to \Hat{\Omega}^3$. 
%The differential is zero on the odd summand.
%We will call this cochain complex $(E_{fivebrane}(\Hat{D}^3), \del)$.
%
%From the description of the fields of the holomorphic twist of the single fivebrane it is easy to see that the Taylor expansion at $0 \in \C^3$ defines a cochain map
%\[
%(\cE_{fivebrane} (\C^3), Q) \to (E_{fivebrane}(\Hat{D}^3), \del) 
%\]
%That this induces a quasi-isomorphism $J_0^\infty \cE_{fivebrane} \simeq E (\Hat{D}^3)$ follows from the Dolbeault Poincar\'e lemma.
%\end{proof}

\subsection{A stack of two fivebranes}

\parsec[]

Recall that the even part of $E(3|6)$ ... 

\begin{itemize}
\item Single particle operators coming from the copy of holomorphic vector fields $\Vect^{hol}(\C^3)$ contribute
\[
q^3 \frac{t_1^{-1} q + t_1 t_2^{-1} q + t_2 q }{(1-t_1^{-1}q) (1-t_1 t_2^{-1} q) (1-t_2 q)} 
\]
\item Single particle operators coming from $\lie{sl}(2)$-valued holomorphic functions $\lie{sl}(2) \otimes \cO^{hol}(\C^3)$ contribute
\[
q^3\frac{r^2 + r^{-2} + 1}{(1-t_1^{-1}q) (1-t_1 t_2^{-1} q) (1-t_2 q)} 
\]
\item Single particle operators coming from the odd piece of $E(3|6)$ which is $\Omega^{1,hol} \otimes K^{-1/2} \otimes \C^2$ contribute
\[
q^{3}\frac{(q^{1/2} r + q^{1/2} r^{-1})(t_1^{-1} + t_1t_2^{-1} + t_2)}{(1-t_1^{-1}q) (1-t_1 t_2^{-1} q) (1-t_2 q)}
\]
\end{itemize}

\begin{conj}
The character of local operators of the holomorphic twist of the theory on a stack of two fivebranes is given by the following plethystic exponential
\[
{\rm PExp} \left[f_{two} (t_1,t_2,q,r) \right] .
\]
where the single particle index is
\[
f_{two} (t_1,t_2,q,r) = \frac{q^4(t_1^{-1} + t_1 t_2^{-1}  + t_2) + q^3 (r^2 + r^{-2} + 1) - q^{7/2} (r + r^{-1})(t_1^{-1} + t_1t_2^{-1} + t_2)}{(1-t_1^{-1}q) (1-t_1 t_2^{-1} q) (1-t_2 q)} .
\]
\end{conj}

\parsec[]

The specialization of this index $t_1=t_2=r=1$ yields the single particle index
\[
\frac{3q^4 + 3 q^3 - 6 q^{7/2}}{(1-q)^3}. 
\]

\parsec[]

The specialization of this index $q=r^2, t_2=1$ in \eqref{eqn:special1} yields the plethystic exponential of the following single particle index
\[
f_{two}(t_1, 1, q, q^{1/2}) = \frac{q^2}{1-q} 
\]
which is the same as the single particle index of Virasoro vacuum module on the Riemann surface $\Sigma = \C_{z_1}$. 

\subsection{The large $N$ limit}

We now compute the equivariant character counting local operators of the holomorphic twist of the theory on a large number of fivebranes. 
Unlike the analysis above, we will the twisted holography proposal to compute this character. We consider the equivariant character of local operators of the holomorphic twist of a large number of fivebranes with respect to a finite dimensional subalgebra
\[
\lie{sl}(3)\times \lie{sl}(2)\times \lie {gl}(1)\subset E(5|10)
\]
which embeds in the following way:
\begin{itemize}
\item The Lie algebra $\lie{sl}(3)$ is a subalgebra of $\Vect_0(\Hat{D}^3) \subset \Vect_0 (\Hat{D}^5)$ consisting of linear coefficient vector fields $\sum_{i,j} a_{ij} z_i \frac{\del}{\del z_j}.$
where $(a_{ij}) \in \lie{sl}(3)$. 
\item The Lie algebra $\lie{sl}(2)$ rotates the directions $\C_{w_1} \times \C_{w_2}$ transverse to the brane.
\item The one-dimensional Lie algebra $\lie{gl}(1)$ is spanned by the vector field 
\beqn\label{vf1}
\sum_i z_i \frac{\del}{\del z_i} - \frac23 \sum_{j} w_j \frac{\del}{\del w_j} \in \Vect_0(\Hat{D}^5).
\eeqn
\end{itemize}

Correspondingly, the equivariant character thus has four fugacities, which we label as follows:
\begin{itemize}
  \item $t_{1}, t_{2}$ again denote generators for the Cartan of $\lie{sl}(3)$ acting by vector fields on $\C^{3}_{z}$.
  \item $r$ denotes a generators for the Cartan of a $\lie{sl}(2)$ acting by the fundamental representation on the normal directions $\C_{w_1} \times \C_{w_2}$.
  \item $q$ denotes a generator for the Cartan of additional $\lie{gl}(1)$ generated by the divergence-free vector field \eqref{vf1}. 
\end{itemize}

This choice of weights is summarized in Table \ref{tbl:sugra}.

\begin{table}
\begin{center}
\begin{tabular}{c c c c c c}
  & $z_{1}$ & $z_{2}$ & $z_{3}$ & $w_{1}$ & $w_{2}$ \\
  \hline
  $t_{1}$ & $1$ & 0 & $-1$ & 0 & 0 \\
  $t_{2}$ & 0 & 1 & $-1$ & 0 & 0 \\
  $r$ & 0 & 0 & 0 & 1 & $-1$ \\
  $q$ & $-1$ & $-1$ & $-1$ & $\frac{3}{2}$ & $\frac{3}{2}$
\end{tabular}
\caption{Fugacities for the fields of the holomorphic twist of eleven-dimensional supergravity.}
\label{tbl:sugra}
\end{center}
\end{table}

\parsec


\begin{prop}
  The character of local operators of the holomorphic twist of the theory on a large number of fivebranes is given by the following plethystic exponential

  \[ \rm PExp [f_{large N}(t_{1}, t_{2}, q, r)] \]

  where the single particle index is \[f_{large N}(t_{1}, t_{2}, q, r) = \frac{q^4(t_1^{-1}+t_1t_2^{-1}+t_2)-q^2(t_1+t_1^{-1}t_2+t_2^{-1})+(q^{3/2}-q^{9/2})(r+r^{-1})}{(1-t_{1}^{-1}q)(1-t_{2}q)(1-t_{1}t_{2}^{-1}q)(1-rq^{3/2})(1-r^{-1}q^{3/2})}.\]
\end{prop}

\parsec

The specialization $t_1=t_2=r=1$ yields the single particle index
\[
\frac{3 q^4 - 3 q^2 + 2 q^{3/2} - 2 q^{9/2}}{(1-q)^3 (1-q^{3/2})^2} .
\]

\parsec 
The specialization of this index $q=r^2, t_2=1$ in \eqref{eqn:special1} yields the plethystic exponential of the following single particle index
\[
f_{two}(t_1, 1, q, q^{1/2}) = \frac{q}{(1-q)^2}
\]

This plethystic exponential yields the Macmahon function, which is the character of the vacuum module of the $W_{1+\infty}$-algebra.

\parsec The same change of variables in \eqref{eqn:special2} agrees with previously computed indices for single particle states for supergravity on $AdS_{7}\times S^{4}$ \surya{...}

%\end{document}
