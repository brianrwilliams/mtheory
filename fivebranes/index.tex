%\documentclass[11pt]{amsart}
%
%%\usepackage{../macros-master}
%\usepackage{macros-fivebrane}
%
%\begin{document}

\section{Twisted supergravity states}
\label{sec:states}

The first entry of the AdS/CFT dictionary in traditional treatments is a matching between \textit{supergravity states} and local operators in the CFT.
The goal of this section is to provide constructions of spaces of twisted supergravity states in our eleven-dimensional model, via geometric quantization. The state spaces on ${\rm AdS}_{7}\times S^{4}$ and ${\rm AdS}_{4}\times S^{7}$ have a remarkable property---they are naturally modules for certain infinite-dimensional exceptional super Lie algebras. We conclude the section by computing characters for these modules and comparing them with large $N$ indices for fivebranes and membranes in the literature.

Before proceeding with the construction, let us first give some feel for the situation we hope to describe. Suppose we consider a gravitational theory on $AdS_{d+1}\times S^{d^{\prime}}$, which we compactify to view as a theory on $AdS_{d+1}$ with all Kaluza-Klein harmonics included. Let $M^{d}$ denote the conformal boundary of $AdS_{d+1}$. A supergravity state is traditionally defined to be a solution to linearized equations of motion with a given boundary value \cite{WittenHolography}. Typically, this definition is made in situations where the relevant boundary value problem has a unique solution, in which case one may label states by the corresponding boundary values. Moreover, one may think of such boundary values as arising from modifications of a vacuum boundary condition at a point.


\subsection{Twisted Backreactions}
We begin by describing the relevant backgrounds. In eleven-dimensional supergravity, the $AdS_7 \times S^4$ background is obtained by backreacting a number of fivebranes in flat space \cite{Maldacena:1997re,WittenAdS}.
In \cite{RSW} we gave descriptions of twisted versions of this background. We will recall this construction, adapted to a slightly more global situation than considered previously.

We will consider the eleven-dimensional theory on eleven-manifolds that arise as total spaces of vector bundles. Placing the theory in the backreacted geometry is a 2-step procedure:

\begin{itemize}
  \item Place the eleven-dimensional theory on the complement of the zero section. To do so, we will wish to describe the complement of the zero-section in a way that facilitates natural operations on holomorphic-topological local $L_{\infty}$-algebras.

  \item Deform the theory on the complement of the zero section by a certain Maurer--Cartan element.
  The Maurer--Cartan element is thought of as the flux sourced by branes wrapping the zero section.
\end{itemize}

\parsec[s:brkevin]
As a way to highlight the key aspects of the construction, we detail the ingredients in the simplified model of Costello's twisted $M$ theory. The relevant local calculation can be found in the appendix of \cite{}; our goal here is to simply identify the salient global features that allow one to reduce to said local calculation.

We consider the theory on $M = \text{Tot} (\R\oplus K_{C})$, with some number of twisted `fivebranes' wrapping the zero section
\[
0 \times C \subset \R \times \T^* C .
\]
Denote by $t$ the real coordinate and by $w$ the fiber coordinate in $\T^* C$. We wish to describe the complement of the zero section $\mathring M = M - 0 \times C$.

Note that the bundle $\R\oplus K_{C}$ is equipped with a partially flat connection - this data equips the total space $M$ with the data of a transversely holomorphic foliation (THF) \cite{DuchampKalka}.

If we choose a fiberwise partially hermitian metric on the bundle $\R \oplus K_C$ we obtain a projection $p: \R \times \T^*C \to \R_{+} \times C$ which combines the fiberwise norm with the natural bundle projection. The restriction $p| \mathring M$ equips $\mathring M$ with the structure of an $S^{2}$-bundle over $\R_{>0}\times C$. Moreover, the partial flat connection on $\R\oplus K_{C}$ induces a partially flat connection on $\mathring M$. As part of this data, each of the fiber spheres is equipped with a complex foliation of rank 1.

Compactification amounts to pushing forward a local $L_{\infty}$-algebra along $p| \mathring M$. The result is a theory with infinitely many Kaluza--Klein modes along the fiber spheres. In the holomorphic-topological setting, the Kaluza-Klein modes will be modeled by a variant of Cauchy-Riemann cohomology.

Moreover, including the flux sourced by the brane deforms this structure. The lowest lying Kaluza-Klein modes in the deformed theory are equivalent to three-dimensional Chern-Simons theory.

For sake of analogy, we think of the resulting deformation as being a twisted version of $AdS_3 \times S^2$. \footnote{It is an interesting question if this corresponds the actual twist of a five--dimensional supersymmetric background of this form.}
We proceed to describe the twisted version of states at the boundary of this version of $AdS$.
We first proceed before turning on the flux sourced by the brane.

The theory admits a natural `vacuum' boundary condition at $r=0$.
In local coordinates, these are fields $\alpha(t,z,w)$ on the complement to the brane which extend to regular functions along the brane.

The `supergravity states' are, by definition, fields which satisfy the linearized equations of motion and satisfy the vacuum boundary condition except at a single point.
The linearized equations of motion are simply $(\d_{dR} + \dbar) \alpha = 0$.
Thus, up to equivalence, all solutions to the linearized equations of motion are constant in the real variable $t$, and holomorphic in $z,w$.

Modifications of the boundary condition at the point~$z = 0$ on the boundary take the form
\[
\alpha = f(w) \delta^{(r)}_{z=0}
\]
where $f$ is some holomorphic function.
Here $\delta^{(r)}_{z=0}$ denotes the $r$th derivative of the $\delta$-function at $z=0$.
It is convenient to parameterize such boundary modifications algebraically by expressions of the form
\[
\alpha_{k,r} = w^k \delta^{(r)}_{z=0} .
\]
Linear combinations of such states form a dense subspace of all possible modifications at the boundary.

The reason that the boundary modifications take this form can be seen by understanding in more explicit terms the vacuum boundary condition.
The phase space at the boundary $C$ can be identified with the following cohomology
\[
\Omega^{0,\bu}(C) \otimes \cA^{0;\bu}(\R \times \C - 0) [1]
\]
where $\cA^{0;\bu}$ denotes the mixed de Rham--Dolbeault cohomology of $\R \times \C - 0$ as a manifold equipped with a transversely holomorphic foliation \cite{DuchampKalka}.
We refer to the section below for a reminder on this geometric structure.

The phase space is equipped with a natural symplectic form given by
\[
\int_C \d z \oint_{S^2} \d w \, \alpha \wedge \alpha' .
\]
There is a natural Lagrangian inside of the phase space which consists of linear combinations of elements $\alpha(z) \otimes f(t,w)$ where $\alpha(z) \in \Omega^{0,\bu}(C)$ and $f(t,w)$ is a smooth function on $\R \times \C - 0$ which extends to zero.
The linearized equations of motion simply say that $\alpha$ is holomorphic, $f$ is independent of $t$ and depends holomorphically on $w$

\parsec[s:brfive]

We now consider the situation of backreacting some number of (twisted) fivebranes in our eleven-dimensional model.
Let $Z$ be a three-fold that the fivebranes wrap.
We also fix a rank 2 holomorphic vector bundle $W\to Z$ such that $\wedge^{2} W \cong K_{Z}$;
this condition ensures that the total space of $W$ is a Calabi-Yau five-fold. In the main body of the paper we will choose $W$ to be the bundle $K_{Z}^{1/2}\otimes \C^{2}$.

Consider the bundle $\R\oplus W$; this bundle has a canonical partially flat connection. We wish to consider our eleven dimensional model on $M = Tot (\R\oplus W)$ which is the total space of the \textit{real} rank five bundle $\R\oplus W$ over $Z$. The partially flat connection on $\R\oplus W$ equips $M$ with a canonical THF structure $F_{M}\subset T_{M}^{\C}$.

We place a stack of $N$ fivebranes wrapping the zero section in $\R\oplus W$.
Denote the complement of the zero section by
\[
\mathring{M} = \text{Tot}(\R\oplus W) - 0(Z).
\]

We may choose fiber coordinates of the bundle $t, w_{1}, w_{2}$ of $\R \oplus W$ over $Z$ and a fiberwise partially hermitian metric.

Explicitly, the corresponding norm defines a map
\begin{align*}
 h \colon  M & \to \R_{+} \\
  (t, w_{i}, \bar{w_{i}}, p)& \mapsto t^{2} + |w_{1}|^{2}+|w_{2}|^{2}
\end{align*}
Letting $\pi \colon M \to Z$ be the natural projection, we obtain the projection
\[
p \define (h,\pi) \colon M \to \R_{+}\times Z
\]
which restricts to an $S^4$ bundle $p|\mathring {M} \colon \mathring{M} \to \R_{>0} \times Z$.
These embeddings and projections fit inside of the following commutative diagram
\[
\begin{tikzcd}
\mathring{M} \ar[d,"p|\mathring M"'] \ar[r,hook] & M \ar[d,"p"] & \ar[l,hook',"0"'] Z \ar[d,"="] \\
\R_{>0} \times Z \ar[r,hook] & \R_{+} \times Z & \ar[l,hook',"0 \times \id"] Z.
\end{tikzcd}
\]
The inclusions on the left are the natural embeddings.
The top right inclusion is the zero section of $M = {\rm Tot}(\R \oplus V)$ and the bottom right inclusion is the embedding at radius $r = 0$.

As we elaborated in section \ref{s:thfmflds}, the eleven-dimensional theory is defined on the THF manifold $\mathring M$---in the BV formalism this is encoded, in part, by the sheaf of $L_\infty$ algebras $\cL_{sugra}$ on $\mathring M$. Compactification of this theory along the $S^4$ link corresponds to pushing forward this sheaf along $p|\mathring M$. The resulting sheaf of $L_\infty$ algebras $(p|\mathring M)_*\cL_{sugra}$ describes, in the BV formalism, the compactified theory on the seven-manifold $\R_{>0} \times Z$.

We will compute the pushforward using the prescription for pushing forward a $\cA^{\bu}_{F}$-module along a compatible submersion outlined in \cite{KormanThesis}. Recall that we have a THF structure on $M$, given by an involutive subbundle $F_{M}\subset T^{\C}_{M}$, induced from a partially flat conneciton on $\R\oplus W$. The situation is summarized in the diagram below.
\[
  \begin{tikzcd}
    0 \ar[r] & F_{M}\cap T_{M/Z}^{\C}\ar[r]\ar[d] & F_{M}\ar[r]\ar[d]  & \pi^{*}T^{0,1}_{Z}\ar[r]\ar[d] & 0 \\
    0 \ar[r] & T_{M/Z}^{\C}\ar[r] & T_{M}^{\C}\ar[r] & T_{Z}^{\C}\ar[r] & 0.
  \end{tikzcd}
\]

The existence of a partially flat connection amounts to a choice of splitting \[\sigma: \pi^{*}T^{0,1}_{Z}\to F_{M}\] of the top row. Both $F_{M}$ and $\pi^{*}T^{1,0}_{Z}$ are involutive, and the flatness of the connection implies that $\sigma$ preserves the lie brackets on sections.

We have a similar diagram expressing that the sphere bundle $\mathring M \to \R_{>0}\times Z$ is a map of THF manifolds when both the domain and codomain are equipped with the induced THF structures.
\[
  \begin{tikzcd}
    0 \ar[r] & F_{\mathring M}\cap T_{\mathring M/\R_{>0}\times Z}^{\C}\ar[r]\ar[d] & F_{\mathring M}\ar[r]\ar[d]  & h^{*}T_{\R_{>0}}\oplus\pi^{*}T^{0,1}_{Z}\ar[r]\ar[d] & 0 \\
    0 \ar[r] & T_{\mathring M/\R_{>0}\times Z}^{\C}\ar[r] & T_{\mathring M}^{\C}\ar[r] & h^{*}T_{\R_{>o}}\oplus T_{Z}^{\C}\ar[r] & 0.
  \end{tikzcd}
\]

\surya{conditions for top left to be a vector bundle}

We claim that the splitting $\sigma$ induces a splitting $\mathring \sigma : h^{*}T_{\R_{>0}}\oplus \pi^{*}T^{0,1}_{Z}\to F_{\mathring M}$. \surya{is this obvious?} It is clear that $\mathring \sigma$ preserves lie brackets.

For each of the involutive subbundles $F$ appearing in the top row of the above diagram, we wish to describe certain  To make the burden less cumbersome, we will drop mention of the foliation and instead refer to the whether the leaves lie along the base or fiber of the submersion $p$.

Explicitly, we set
\begin{align*}
  \cA^{\bu}_{\mathring M/\R_{>0}\times Z} & = \cA^{\bu}_{F_{\mathring M}\cap T^{\C}_{\mathring M/ \R_{>0}\times Z}} \\
  F_{\R_{>0}\times Z} &  = T_{\R_{>0}}\oplus T^{0,1}_{Z} \text{ so }p^{*}F_{\R_{>0}\times Z} = h^{*}T_{\R_{>0}}\oplus \pi^{*}T^{0,1}_{Z} \\
  \cA^{\bu}_{\R_{>0}\times Z}&  = \cA^{\bu}_{F _{\R_{>0}\times Z}} \text{ so }p^{*}\cA^{\bu}_{\R_{>0}\times Z} = \cA^{\bu}_{p^{*}F _{\R_{>0}\times Z}}
\end{align*}

Our goal is to describe the pushforward $(p|\mathring M)_{*}\cL_{sugra}$ as a $\cA^{\bu}_{\R_{>0}\times Z}$-module.

The splitting affords an isomorphism of $C^{\infty}_{\mathring M}$-modules
\[
  \cA^{\bu}_{\mathring M}(L)\cong \cA^{\bu}_{\R_{>0}\times Z}\otimes \cA^{\bu}_{{\mathring M/ \R_{>0}\times Z}}(L)
\]

Note that since we have a complex of locally free $C^{\infty}_{\mathring M}$-modules, the (a priori derived) functor $(p|\mathring M)_{*}$ has several nice properties. We may commute it past the tensor product above. Moreover, we may apply the functor term-wise. The result is that the pushforward is given by \surya{finish}

We now analyze the THF differential $\thfD$ on $\cA^{\bu}_{\mathring M}(L)$ under the above splitting.
\surya{finish}

In sum, we have proven the following:

\begin{prop}
  The pushforward $(p| \mathring M)_{*}\cL_{sugra}$ is described by the following complex of sheaves
  \beqn
  \cA^\bu_{\R_{>0}\times Z}\otimes \mathbb{R}^\bu p_* \left (\cA^\bu_{\mathring M / \R_{>0}\times Z}(L) \right)
  \eeqn

  The BV pairing is given by
  \beqn
  \eeqn

\end{prop}


The previous result gives a description of the compactification of $\cL_{sugra}$ on the link of fivebranes as a free holomorphic-topological BV theory on $\R_{>0}\times Z$. In particular, the result is local on $R_{>0}\times Z$. For computing the space of states, it will be useful to have a description of its global sections. For the computation of numerical invariants such as graded characters, of the space of states, a more drastic simplification is useful. Indeed, we may anticipate cohomological cancellations in the character and instead compute the character of any cochain complex that a spectral sequence whose first page is the global sections abuts to.

The global sections of the sheaf in proposition \ref{dimred} appears naturally on the $E_{1}$-page of a Leray-Serre-type spectral sequence \cite{KamberTondeur}
\beqn
E_1^{p,q} = \Gamma \left (\R_{>0}\times Z; \cA^p_{\R_{>0}\times Z}\otimes \mathbb{R}^q p _* \left (\cA^\bu_{\mathring M / \R_{>0}\times Z}(L)\right) \right)\implies \mathbb{H}^{p+q} (\cA^\bu_{\mathring M}(L)).
\eeqn
We emphasize that no page of this spectral sequence involves cohomology with respect to the operators $D, D_{\Omega}$; these differentials are kept internal.

In \cite{RSW}, this spectral sequence was implicitly used to give a concrete description of the pushforward $(p|\mathring M)_{*}\cL_{sugra}$ in the case where
\beqn
M = Tot (\R\oplus \C^{2}\to \C^{3}), \ \ \ \ \ \ \mathring M = \R\times \C^2\times \C^3 \setminus (0 \times \C^3).
\eeqn

To recall the result, fix holomorphic coordinates $z_{i}, i=1, 2, 3$ on $\C^{3}$ and holomorphic fiber coordinates $w_{a}, a= 1, 2$ on $\C^{2}$.

\begin{prop}
  Up to completion, the cohomology
  \beqn
  \mathbb H^{\bu} (\cA_{\mathring M}^\bu (L) ) = H^{\bu}\left (\Gamma(\cA_{\mathring M}^{\bu}(L)), \thfd\right)
  \eeqn
  is a direct sum of the cohomology on flat space $H^{\bu}(\Gamma(\cA_{M}^{\bu}(L), \thfd)$ with the complex

  \beeqn
  \begin{tikzcd}[row sep = 1 ex]
    + & - \\ \hline
w_1^{-1} w_2^{-1} \CC[w_1^{-1}, w_2^{-1}][z_1,z_2,z_3] \{\partial_{w_i}\}  \ar[r, dotted, "D_{\Omega}"] & w_1^{-1} w_2^{-1} \CC[w_1^{-1}, w_2^{-1}] [z_1,z_2,z_3] \\
w_1^{-1} w_2^{-1} \CC[w_1^{-1}, w_2^{-1}] [z_1,z_2,z_3] \{\del_{z_i}\} \ar[ur, dotted, "D_{\Omega}"'] \\
w_1^{-1} w_2^{-1} \CC[w_1^{-1}, w_2^{-1}] [z_1,z_2,z_3] \ar[r, dotted, "D"] \ar[dr, dotted, "D"'] & w_1^{-1} w_2^{-1} \CC[w_1^{-1}, w_2^{-1}][z_1,z_2,z_3] \{\d z_i\} \\ & w_1^{-1} w_2^{-1} \CC[w_1^{-1}, w_2^{-1}][z_1,z_2,z_3] \{\d w_i\} .
\end{tikzcd}
\eeqn
\end{prop}

\parsec[s:flux]
The second step in the construction of the backreacted geometry involves deforming the pushforward $(p| \mathring{M})_{*}\cL_{sugra}$ by a certain Maurer-Cartan element. The Maurer-Cartan element is determined by the lowest order term in the coupling between the eleven dimensional theory and the stack of fivebranes. It was argued in \cite{RSW} that the relevant coupling is given by the nonlocal interaction
\beqn\label{eqn:br1}
I_{fivebrane} = N\int_{Z} D_\Omega^{-1}\mu \vee \Omega +\cdots
\eeqn
where $\mu \in \Omega^0 (\R) \hotimes \PV^{1,3}(X)\subset \cA^{\bullet}_{F}(\vartheta_{F})$ is a component of a field in the eleven-dimensional theory which satisfies $\thfd_{\Omega} \mu = 0$.

The relevant Maurer-Cartan element is a solution to the equations of motion for the eleven-dimensional theory deformed by this interaction. After deforming with this interaction, varying with respect to $D_{\Omega}^{-1}\mu$ yields the following equation for $\gamma$
\beqn
\thfd D \gamma + D_\Omega\left (\frac{\mu}{1-\nu}\right )\wedge D \gamma = N \delta_Z.
\eeqn

%\parsec
%Let $C$ be a curve, and let $V\to C$ be a rank 4-holomorphic vector bundle over $C$ such that $\wedge^{4} V = K_{C}$. This condition again ensures that $X = {\rm Tot} V$ is a Calabi-Yau five-fold - in the main body of the paper, we will take $V = K^{1/2}_{C}\otimes \C^{4}$. Abusively letting $V$ also denote its pullback along the canonical projection $\R\times C \to C$, we may view $\R\times X$ as the total space of $V$ on $\R\times C$. As before we will consider wrapping a stack of $N$ membranes along the zero section.

%Since $V$ is a complex vector bundle, we may choose a fiberwise hermitian metric, and as before, we may view $\R\times X \setminus \R\times C$ as an $S^{7}$-bundle over $\R_{>0}\times \R\times C$.



%\parsec[s:sugraops]
%
%By the usual methods of the BV formalism the action functional $S_{sugra}$ described above endows the parity shift of the fields $\cL_{sugra} = \Pi \cF_{sugra}$ with the structure of a holomorphic-topological local $\Z/2$ graded $L_\infty$ algebra. 
%
%On $\C^5 \times \R$ we can describe this super Lie algebra structure explicitly. 
%First, by the Dolbeault and de Rham Poincar\'e lemmas it is easy that the even part of the super Lie algebra $\cL(\C^5 \times \R)$ is equivalent to a one-dimensional central summand $\C$ plus the Lie algebra of divergence-free vector fields on $\C^5$:
%\[
%\Vect_0 (\C^5) = \{X \in \Vect(\C^5) \; | \; \div X = 0\} .
%\]
%The odd part of the super Lie algebra $\cL(\C^5 \times \R)$ is equivalent to the space of holomorphic one-forms on $\C^5$ modulo exact one-forms
%\[
%\Omega^{1,hol}(\C^5) / {\rm Im}(\del) 
%\]
%which is, of course, equivalent to the space of closed holomorphic two-forms $\Omega^{2,hol}_{cl}(\C^5)$. 
%
%\begin{thm}[\cite{RSW}[Theorem 2.1]]
%The Taylor expansion map determines a map of $\Z/2$ graded $L_\infty$ algebras
%\[
%j_\infty \colon \cL_{sugra}(\C^5 \times \R) \to L_{sugra} .
%\]
%Furthermore, $L_{sugra}$ is equivalent as a $\Z/2$ graded $L_\infty$ algebra to $\Hat{E(5|10)}$. 
%\end{thm} 
%
%As an immediate corollary of this result we obtain by Lemma \ref{lem:localops} the following.
%
%\begin{cor}
%\label{cor:sugraops}
%Let $\Obs_{sugra}$ be the factorization algebra on $\C^5 \times \R$ of classical observables of the minimal twist of eleven-dimensional supergravity.
%There is a quasi-isomorphism of commutative dg algebras
%\[
%\Obs_{sugra} (0) \simeq \clie^\bu \left( \Hat{E(5|10)} \right) .
%\]
%\end{cor}
\parsec[s:hilbertspace]
The description of the $S^{4}$ compactification of eleven dimensional supergravity obtained in the previous subsection facilitates a straightforward construciton of the Hilbert space. The compactification is described by the pro-local Lie algebra $(p|\mathring M)_{*} \cL_{sugra}$ on the seven-manifold $\R_{>0}\times Z$. We will construct the Hilbert space by applying geometric quantization ansatz to the phase space at $\infty \in \R_{>0}$.

Our ansatz for geometric quantization will avoid discussion of subtler aspects such as the metaplectic correction. We define the geometric quantization to be given by functions that are constant along the leaves of a lagrangian foliation of the phase space.

The phase space at $\infty\in \R_{>0}$ is computed by restricting to a neighborhood $(a, \infty)\subset \R_{>0}$:
\beqn
(p| \mathring M)_* \cL_{sugra}|_{\infty \times Z}=
\eeqn

There is a symplectic pairing given by
\beqn
\eeqn



\subsection{Global symmetry for twisted $AdS$}
\label{s:global1}

After complexification, the~six-dimensional superconformal algebra is isomorphic to $\lie{osp}(8|4)$.
The even part of this algebra is $\lie{so}(8) \times \lie{sp}(4)$.
This algebra contains the six-dimensional $\cN=(2,0)$ supersymmetry algebra whose odd part is four copies of $S^{6d}_+$, the positive irreducible complex spin representation of $\lie{so}(6)$.
%It also contains the three-dimensional $\cN=8$ supersymmetry algebra whose odd part is eight copies of $S^{3d}$, the irreducible complex spin representation of $\lie{so}(3)$.

The holomorphic supercharge is a supertranslation
\[
Q \in \Pi S^{6d}_+ \otimes \C^4 \subset \lie{osp}(8|4)
\]
which is characterized (up to equivalence) by the properties that $Q^2 = 0$ and that its image
\[
{\rm Im}\left(Q|_{\Pi S_+ \otimes \C^4} \right) \subset \R^6 \otimes_\R \C \cong \C^6
\]
is three-dimensional (spanned by the anti-holomorphic translations). 
The supercharge $Q$ acts on $\lie{osp}(8|4)$ by commutator and the resulting cohomology will automatically act on the holomorphic twist of any six-dimensional superconformal field theory. 
This cohomology can readily be identified with the subalgebra $\lie{osp}(6|2)$, see \cite{SWe36}. 

%Similarly, the minimal twisting supercharge in the three-dimensional $\cN=8$ supersymmetry algebra is an element $Q \in \Pi S^{3d} \otimes \C^8$ which is characterized (up to equivalence) by the property that $Q^2 = 0$ and that the image of $[Q,-]$ is two-dimensional. The cohomology of $\lie{osp}(8|4)$ with respect to this supercharge is also isomorphic to~$\lie{osp}(6|2)$.

In \cite{RSW} we have shown that solutions to equations of motion of our eleven-dimensional theory in the twisted $AdS_{7}\times S^{4}$ background recalled in the previous subsection contains the symmetry algebra $\lie{osp}(6|2)$. This is precisely the twist of the superconformal algebra we just discussed, so this statement may be interpreted as the twisted avatar of the fact that $lie{osp}(8|4)$ acts as isometries on $AdS_{7}\times S^{4}.$ An easy consequence of the result of \cite{RSW} and the definition of $\cH_{sugra}$ in \ref{hilbert} is that the twisted symmetry algebra $\lie{osp}(6|2)$ acts on $\cH_{sugra}$.

We will enumerate states via their weights a Cartan in the bosonic subalgebra of the twisted superconformal algebra.
We recall below how the bosonic piece of the algebra
\beqn
\label{eqn:gut}
\lie{sl}(4) \times \lie{sl}(2) \subset \lie{osp}(6|2) 
\eeqn
embeds as symmetries, or ghosts, of the eleven-dimensional theory in this twisted background.

\brian{enhance to E(3|6)}.
However, we first note that the hilbert space $\cH_{sugra}$ in fact enjoys an action of a much larger symmetry algebra.



\subsection{Mass Spectroscopy on twisted $AdS_7\times S^{4}$}
\label{s:ads7}

For convenience we choose coordinates on the eleven manifold as
\[
\R \times \C^5 = \R_t \times \C^2_w \times \C_z^3 
\]
with $z = (z_i), i=1,2,3$ and $w = (w_a), a=1,2$.
The stack of fivebranes wrap 
\beqn
w_1=w_2=t=0 .
\eeqn
Important for us is to recall that part of the ghost system for our eleven-dimensional theory consists of divergence-free vector fields on $\C^5$ which are locally constant along $\R$. 

The subalgebra \eqref{eqn:gut} of the twisted superconformal algebra $\lie{osp}(6|2)$ embeds as ghosts in our eleven-dimensional model as follows.
\begin{itemize}
\item
The subalgebra $\lie{sl}(3) \subset \lie{sl}(4)$ embeds as vector fields rotating the plane $\C^3_z$
\beqn
\sum_{ij} A_{ij} z_i \frac{\del}{\del z_j} \in \PV^{1,0}(\C^5)\otimes \Omega^0(\R) , \quad (A_{ij}) \in \lie{sl}(3) .
\eeqn
By definition, these vector fields are automatically divergence-free.
Notice that this vector field is divergence-free and restricts to the Euler vector field along $t=w_{a} = 0$.
\item 
The subalgebra $\lie{sl}(2)$ ($R$-symmetry) is mapped to the triple
\beqn
 w_1 \frac{\del}{\del w_2}, \quad w_2 \frac{\del}{\del w_1}, \quad \frac{1}{2}\left (w_1\frac{\del}{\del w_1}-w_2\frac{\del}{\del w_2}\right) \in \PV^{1,0}(\C^5) \otimes \Omega^0(\R) .
\eeqn
\item Scaling on the plane $\C^3$ embeds as the vector field
\beqn\label{eqn:Delta}
        \Delta = \sum_{i=1}^3 z_i\frac{\del}{\del z_i} - \frac 32\sum_{a=1}^2 w_a\frac{\del}{\del w_a}\in \PV^{1,0}(\C^5)\otimes \Omega^0 (\R).
\eeqn
\end{itemize}

%\begin{rmk}
%In the classification of simple super Lie algebras, Kac makes use of a weight grading $\oplus_{j \geq -2} \fg_j$ of the exceptional Lie algebra $E(3|6)$ for which the finite-dimensional subalgebra above is the weight zero piece
%\cite{KacClass}.
%We will make use of this grading in \S \ref{s:kr}.
%\end{rmk}
This describes an embedding of the algebra 
\beqn\label{eqn:gut2}
\lie{sl}(3) \times \lie{sl}(2) \times \lie{gl}(1) \subset \lie{sl}(4) \times \lie{sl}(2) 
\eeqn
into the ghosts of our model.
The dimension of a Cartan subalgebra of $\lie{sl}(3) \times \lie{sl}(2) \times \lie{gl}(1)$ is four and accordingly, the equivariant character we study has four fugacities.
We choose these explicitly as follows:
\begin{itemize}
  \item $t_{1}, t_{2}$ denote generators for the Cartan of $\lie{sl}(3)$ which is spanned by the vector fields
  \beqn
  h_1 = z_1 \frac{\del}{\del {z_1}} - z_2 \frac{\del}{\del{z_2}} , \quad h_2 = z_2 \frac{\del}{\del{z_2}} - z_3 \frac{\del}{\del{z_3}}.
  \eeqn
  \item $r$ denotes a generator for the Cartan of a $\lie{sl}(2)$ which is generated by the element 
  \beqn
  \label{eqn:hCartan}
  h = \frac12 \left(w_1 \frac{\del}{\del w_1} - w_2 \frac{\del}{\del w_2}\right) .
  \eeqn
\item $q$ denotes a generator for the Cartan of the~$\lie{gl}(1)$ which is generated by the element $\Delta$ from equation~$\eqref{eqn:Delta}$. 
\end{itemize}

The twisted supergravity states $\cH_{sugra}^{6d}$ form a representation for $\lie{osp}(6|2)$. 
The weights of twisted supergravity states with respect to the generators of the Cartan subalgebra above are completely determined by the weights of the holomorphic coordinates on $\C^2_w \times \C^3_z$.
These are summarized in table \ref{tbl:sugraM5}.

\begin{table}
\begin{center}
\begin{tabular}{c c c c c c}
  & $z_{1}$ & $z_{2}$ & $z_{3}$ & $w_{1}$ & $w_{2}$ \\
  \hline
  $t_{1}$ & $1$ & 0 & $-1$ & 0 & 0 \\
  $t_{2}$ & 0 & 1 & $-1$ & 0 & 0 \\
  $r$ & 0 & 0 & 0 & 1 & $-1$ \\
  $q$ & $-1$ & $-1$ & $-1$ & $\frac{3}{2}$ & $\frac{3}{2}$
\end{tabular}
\caption{Fugacities for the fields of the holomorphic twist of eleven-dimensional supergravity for the geometry $\R \times \C^5 \setminus \C^3$.}
\label{tbl:sugraM5}
\end{center}
\end{table}

We enumerate single particle supergravity states via computing the super trace of the operator $q^Y t_1^{h_1} t_2^{h_2} r^h$ acting on $\cH^{6d}_{sugra}$:
\beqn
f^{6d}_{sugra}(t_1,t_2,r,q) = \Tr_{\cH_{sugra}^{6d}} (-1)^F t_1^{h_1} t_2^{h_2} r^h q^\Delta .
\eeqn
The super trace means that there is an extra factor of $(-1)^F$, where $F$ is parity (fermion number), when computing the ordinary trace. 
That is, we compute the expression


\begin{prop}
\label{prop:sugraindex1}
The single particle index of the space of twisted supergravity states~$\cH_{sugra}^{6d}$ is given by the following expression
\beqn
\label{eqn:sugra_index}
f_{sugra}^{6d} (t_1,t_2, r, q) = \frac{q^4(t_1^{-1}+t_1t_2^{-1}+t_2)-q^2(t_1+t_1^{-1}t_2+t_2^{-1})+(q^{3/2}-q^{9/2})(r+r^{-1})}{(1-t_{1}^{-1}q)(1-t_{2}q)(1-t_{1}t_{2}^{-1}q)(1-rq^{3/2})(1-r^{-1}q^{3/2})}.
\eeqn
The full (multiparticle) index is defined to be the plethystic exponential 
\beqn
{\rm PExp}\left[f_{sugra}^{6d}(t_1,t_2,r,q)\right] .
\eeqn
\end{prop}

To simplify the form of this index we can introduce a different parametrization of the Cartan of $\lie{sl}(3) \times \lie{sl}(2) \times \lie{gl}(1)$.
First, we can parameterize the Cartan of $\lie{sl}(3)$ by the vector fields
  \beqn\label{eqn:ys}
  -(\log y_1) z_1 \frac{\del}{\del {z_1}} - (\log y_2) z_2 \frac{\del}{\del{z_2}} - (\log y_3) z_3 \frac{\del}{\del{z_3}} .
  \eeqn
where $y_1,y_2,y_3$ are parameters which satisfy the single constraint
\beqn
y_1 y_2 y_3 = 1 .
\eeqn
In terms of the variables $t_1,t_2$ used above we have
\beqn
y_1 = t_1^{-1},\quad y_2 = t_1 t_2^{-1}, \quad y_3 = t_2 .
\eeqn
Second, we can parametrize the Cartan of the remaining subalgebra $\lie{sl}(2) \times \lie{gl}(1)$ by the two vector fields
\beqn
\til{h} = h + \frac12 \Delta \quad \text{and} \quad \Delta
\eeqn
where $\Delta$ is as in equation \eqref{eqn:Delta} and $h$ is as in \eqref{eqn:hCartan}.
We denote by $y$ the generator of the Cartan corresponding to the vector field $\til{h}$ and by $q$ (as above) the generator corresponding to~$\Delta$.
In terms of the variable~$r$ used above we have 
\beqn
y = q^{1/2} r .
\eeqn

Using the paramterization fo the Cartan given by the variables $y_i,y,\Delta$ we obtain the equivalent expression for the index \eqref{eqn:sugra_index} as 
\beqn
\label{eqn:Kim_sugra}
f_{sugra}^{6d} (y_i, y, q) = \frac{q^4(y_1+y_2+y_3)-q^2(y_1^{-1} + y_2^{-1} + y_3^{-1})+(1-q^3)(yq + y^{-1} q^2)}{(1-y_1 q)(1-y_2 q)(1-y_3 q)(1-yq)(1-y^{-1} q^2)},
\eeqn
We note that this matches exactly with the index computed in \cite[Eq. (3.23)]{Kim:2013nva} with the change of variables.
%\beqn
%t_1 = y_1^{-1} , \quad t_2 = y_3, \quad r = q^{-1/2} y 
%\eeqn 

Our formula \eqref{eqn:sugra_index} also matches with \cite[Eq. (3.24)]{Bhattacharya:2008zy} where we use the change of variables
\beqn
q = x^4, \quad t_1 = y_2, \quad t_2 = y_1, \quad r^2 = z .
\eeqn
(Notice the variables $y_1,y_2$ used in \cite{Bhattacharya:2008zy} differ from the variables we introduced in \eqref{eqn:ys}.)
We record a few specializations of this index which we will remark on further in~\S\ref{s:??}.

\parsec 
The specialization of this index $q=r^2, t_2=1$ in \eqref{eqn:special1} yields the plethystic exponential of the following single particle index
\[
f_{sugra}^{6d}(q, t_1, t_2=1, r = q^{1/2}) = \frac{q}{(1-q)^2}
\]

This plethystic exponential yields the Macmahon function, which is the character of the vacuum module of the $W_{1+\infty}$-algebra.

\parsec

The specialization $t_1=t_2=r=1$ yields the single particle index
\[
f_{sugra}^{6d} (q, t_1=t_2=r=1) = \frac{3 q^4 - 3 q^2 + 2 q^{3/2} - 2 q^{9/2}}{(1-q)^3 (1-q^{3/2})^2} .
\]

\parsec The same change of variables in \eqref{eqn:special2} agrees with previously computed indices for single particle states for supergravity on $AdS_{7}\times S^{4}$
