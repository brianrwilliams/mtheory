\subsection{Large $N$ Holomorphic Character}

We begin by using the twisted holography proposal to compute an index counting local operators of the holomorphic twist of the theory on a large number of M5 branes. Let us briefly recall the proposal specialized to our situation \surya{more elaboration elsewhere}. Let $\cL$ denote the local $L_{\infty}$ algebra underlying the minimal twist of 11 dimensional supergravity on $\R_{t}\times \C^{3}_{z_{i}}\times \C^{2}_{w_{a}}$. Consider a stack of $N$ M5 branes supported at the origin in $\R\times \C^{2}_{w_{a}}$ and let $\pi : \R\times \C^{5}\to \C^{3}_{z_{i}}$ denote the canonical projection. The pushforward $\pi_{*} C^{*}(\cL)$ has the structure of a factorization algebra on $\C^{3}$

The twisted holography proposal asserts that the factorization algebra of local observables $\Obs_{M5}$ of the minimally twisted M5 brane theory in the large $N$ limit is a deformation of the Koszul dual $(\pi_{*} C^{\bullet}(\cL))^{!}$.

Since quantities like indices are deformation invariant, we need not consider such deformations for now. In fact in this section we may, and will, replace the local Lie algebra $\cL$ with the abelian local dg Lie algebra which underlies the free limit of the twisted supergravity theory.
Explicitly, $\cL$ is given by the following sheaf of $\Z/2$-graded cochain complexes
\[
\begin{tikzcd}
  &  \ul{\rm even} & \ul{\rm odd} \\
 \mu \in & \Omega^{\bullet}_{\R}\otimes\PV^{1,\bullet}_{\C^{5}} \ar[r, "\del_{\Omega}"] & \Omega^{\bullet}_{\R}\otimes\PV^{0,\bullet}_{\C^{5}}\\
 \gamma \in & \Omega^{\bullet}_{\R}\otimes\Omega^{0,\bullet}_{\C^{5}} \ar[r, "\del"]
& \Omega^{\bullet}_{\R}\otimes\Omega^{1,\bullet}_{\C^{5}}\end{tikzcd}
.\]


\parsec

%Let us describe the restriction $\iota^* \cL$ along the inclusion of the brane.
%The $\mu$-field decomposes into vector field components which point along and transverse to the brane
%\[
%\mu
%\]
%
%is the sheaf of $\Z/2$-graded cochain complexes
%\[
%\begin{tikzcd}
%  &  \ul{\rm even} & \ul{\rm odd} \\
% \mu|_{w=t=0} \in & \PV^{1,\bullet}_{\C^{3}} \ar[r, "\del_{\Omega}"] & \PV^{0,\bullet}_{\C^{3}}\\
% \gamma|_{w=t=0} \in & \Omega^{0,\bullet}_{\C^{3}} \ar[r, "\del"]
%& \Omega^{1,\bullet}_{\C^{3}}\end{tikzcd}
%.\]

We wish to count local operators, which are given by the costalk at $0\in \C^{3}$ of $\Obs_{M5}$. Let us first describe $\Obs_{M5}$ more explicitly. By the Poincare and $\dbar$ Poincare lemmas, we have quasi-isomorphisms for each $U\subset \C^{3}$ open
\[
\Obs_{M5}(U) = \Sym \left( \begin{tikzcd}  \end{tikzcd} \right)
\]

The local observables are given by $\lim _{U\ni 0} \Obs_{M5}(U)$.

We first introduce the following notation for representatives of classes in compactly supported Dolbeault cohomology. We write $\delta^{(a,b,c)}$ to denote the derivative of the delta function $\del_{z_{1}}^{a}\del_{z_{2}}^{b}\del_{z_{3}}^{c}\delta_{t=w_{a}=0}$.

\begin{lem}
  As $\Z/2$-graded vector spaces, we have that \[H^{\bu}\left ( \Obs_{M5}(0)\right )=\Sym (\Pi V_{\gamma}\oplus  V_{\mu})\] where
  \begin{itemize}
    \item $V_{\gamma}$ is spanned by
          \begin{align}
            & \delta^{(0,b,c)}w_{1}^{k}w_{2}^{l}dz_{1} \\
            & \delta^{(a,b,c)}w_{1}^{k}w_{2}^{l}dz_{2}, \ \ \
             \delta^{(a,b,c)}w_{1}^{k}w_{2}^{l}dz_{3} \\
            & \delta^{(a,b,c)}w_{1}^{k}w_{2}^{l}dw_{1}, \ \ \
            \delta^{(a,b,c)}w_{1}^{k}w_{2}^{l}dw_{2}
          \end{align}

          where $a, b, c, k, l \geq 0$.
    \item $V_{\mu}$ is spanned by
          \begin{align}
            & \delta^{(a,b,c)}w_{1}^{k}w_{2}^{l}\del_{w_1}-k\delta^{(a-1,b, c)}w_{1}^{k-1}w_{2}^{l}\del_{z_{1}},\
             \delta^{(a,b,c)}w_{1}^{k}w_{2}^{l}\del_{w_2}-l\delta^{(a-1,b, c)}w_{1}^{k-1}w_{2}^{l-1}\del_{z_{1}}\text{ where }a > 0, b,c,k,l \geq 0 \\
            & \delta^{(0,b,c)}w_{1}^{k}w_{2}^{l}\del_{w_1}-k\delta^{(0,b-1,c)}w_{1}^{k-1}w_{2}^{l}\del_{z_{2}}, \ \delta^{(0,b,c)}w_{1}^{k}w_{2}^{l}\del_{w_2}-l\delta^{(0,b-1,c)}w_{1}^{k}w_{2}^{l-1}\del_{z_{2}}\text{ where }b > 0,c,k,l \geq 0 \\
            & \delta^{(0,0,c)}w_{1}^{k}w_{2}^{l}\del_{w_1}-k\delta^{(0,0,c-1)}w_{1}^{k-1}w_{2}^{l}\del_{z_{3}}, \ \delta^{(0,0,c)}w_{1}^{k}w_{2}^{l}\del_{w_1}-l\delta^{(0,0,c-1)}w_{1}^{k}w_{2}^{l-1}\del_{z_{3}}\text{ where } c > 0,k,l \geq 0 \\
            & \delta^{(0,0,0)}w_{2}^{l}\del_{w_1}, \ \delta^{(0,0,0)}w_{1}^{k}\del_{w_2}\text{ where }k,l \geq 0
          \end{align}
  \end{itemize}
\end{lem}
\begin{proof}
  We compute cohomology using a spectral sequence associated to a two-step filtration by polyvector field and holomorphic form degree. The $E_{1}$ page is the $d+\dbar$ cohomology and is given by the symmetric algebra on

  \[
\begin{tikzcd}
\ul{\rm even} & \ul{\rm odd} \\
H^{\bullet}(\Omega^{0,\bullet}_{c} (\C^{3}, \iota^{*}T_{\C^{5}})) \ar[r, "\del_{\Omega}"] & H^{\bullet}(\Omega^{0,\bullet}_{c} (\C^{3}, \cO) )\\
H^{\bullet}(\Omega^{0,\bullet}_{c} (\C^{3},  \iota^{*}T^{*}_{\C^{5}}))& H^{\bullet}(\Omega^{0,\bullet}_{c} (\C^{3}, \cO)) \ar[l, "\del"]
\end{tikzcd}
.\]

\surya{...}

\end{proof}

The index we compute will have four fugacities, which we label as follows:
\begin{itemize}
  \item $t_{1}, t_{2}$ denote generators for the cartan of $\SU(3)$ which acts on $\C^{3}_{z}$.
  \item $q$ denotes a generator for the cartan of a $\SU(2)$ rotating the normal directions $\C^{2}_{w}$
  \item $r$ denotes a generator for the cartan of an additional $U(1)$ acting on $\C^{5}$ preserving the volume form
\end{itemize}

A convenient choice of choice of weights is as in the following table:

\begin{center}
\begin{tabular}{c c c c c c}
  & $z_{1}$ & $z_{2}$ & $z_{3}$ & $w_{1}$ & $w_{2}$ \\
  \hline
  $t_{1}$ & 1 & 0 & -1 & 0 & 0 \\
  $t_{2}$ & 0 & 1 & -1 & 0 & 0 \\
  $r$ & 0 & 0 & 0 & 1 & -1 \\
  $q$ & -1 & -1 & -1 & $\frac{3}{2}$ & $\frac{3}{2}$
\end{tabular}
\end{center}

\begin{prop}
  The $\SU(3)\times \SU(2)\times U(1)$ character of the large $N$ local operators of the M5 brane theory is given by

  \[ \operatorname{PExp} \left (\frac{r^{3}\left (\begin{aligned} & t_{1}^{-1}r+t_{1}t_{2}^{-1}r+t_{2}r+q^{-1}r^{-3/2}+qr^{-3/2} \\  - & t_{1}r^{-1}-t_{1}^{-1}t_{2}r^{-1}-t_{2}^{-1}r^{-1}-qr^{3/2}-q^{-1}r^{-3/2}
          \end{aligned}\right)}{(1-t_{1}^{-1}r)(1-t_{2}r)(1-t_{1}t_{2}^{-1}q)(1-r^{3/2}q)(1-r^{3/2}q^{-1})}\right ).\]
  \end{prop}
