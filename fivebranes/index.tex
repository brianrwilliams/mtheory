%\documentclass[11pt]{amsart}
%
%%\usepackage{../macros-master}
%\usepackage{macros-fivebrane}
%
%\begin{document}

\section{Twisted supergravity states}
\label{sec:states}

The first entry of the AdS/CFT dictionary in traditional treatments is a matching between \textit{supergravity states} and local operators in the CFT. 
The goal of this section is to provide constructions of spaces of twisted supergravity states in our eleven-dimensional model, via geometric quantization. The state spaces on ${\rm AdS}_{7}\times S^{4}$ and ${\rm AdS}_{4}\times S^{7}$ have a remarkable property---they are naturally modules for certain infinite-dimensional exceptional super Lie algebras. We conclude the section by computing characters for these modules and comparing them with large $N$ indices for fivebranes and membranes in the literature.

Before proceeding with the construction, let us first give some feel for the situation we hope to describe. Suppose we consider a gravitational theory on $AdS_{d+1}\times S^{d^{\prime}}$, which we compactify to view as a theory on $AdS_{d+1}$ with all Kaluza-Klein harmonics included. Let $M^{d}$ denote the conformal boundary of $AdS_{d+1}$. A supergravity state is traditionally defined to be a solution to linearized equations of motion with a given boundary value \cite{}. Typically, this definition is made in situations where the relevant boundary value problem has a unique solution, in which case one may label states by the corresponding boundary values. Moreover, one may think of such boundary values as arising from modifications of a vacuum boundary condition at a point.


\subsection{Twisted Backreactions}
We begin by describing the relevant backgrounds. In eleven-dimensional supergravity, the $AdS_7 \times S^4$ and $AdS_{4}\times S^{7}$ backgrounds are obtained by backreacting a number of fivebranes and membranes respectively in flat space \cite{Maldacena:1997re,WittenAdS}.
In \cite{RSW} we gave descriptions of twisted versions of these backgrounds. We will recall this construction, adapted to a slightly more global situation than is considered in \cite{RSW}.

We will consider the eleven-dimensional theory on eleven-manifolds that arise as total spaces of vector bundles. Placing the theory in the backreacted geometry is a 3-step procedure:

\begin{itemize}
  \item Place the eleven-dimensional theory on the complement of the zero section. To do so, we will wish to describe the complement of the zero-section in a way that facilitates natural operations on holomorphic-topological local $L_{\infty}$-algebras.

  \item Deform the theory on the complement of the zero section by a certain Maurer--Cartan element.
  The Maurer--Cartan element is thought of as the flux sourced by branes wrapping the zero section.
\end{itemize}

\parsec[s:brkevin]
As a way to highlight the key aspects of the construction, we detail the ingredients in the simplified model of Costello's twisted $M$ theory. The relevant local calculation can be found in the appendix of \cite{}; our goal here is to simply identify the salient global features that allow one to reduce to said local calculation.

We consider the theory on $X = \text{Tot} (\R\oplus K_{C})$, with some number of twisted `fivebranes' wrapping the zero section
\[
0 \times C \subset \R \times \T^* C .
\]
Denote by $t$ the real coordinate and by $w$ the fiber coordinate in $\T^* C$. We wish to describe the complement of the zero section $M = X - 0 \times C$.

Note that the bundle $\R\oplus K_{C}$ is equipped with a partially flat connection - this data equips the total space $X$ with the data of a transversely holomorphic foliation (THF) \cite{DuchampKalka}. 

If we choose a fiberwise partially hermitian metric on the bundle $\R \oplus K_C$ we obtain a projection $p: \R \times \T^*C \to \R_{+} \times C$ which combines the fiberwise norm with the natural bundle projection. The restriction $p|M$ equips $M$ with the structure of an $S^{2}$-bundle over $\R_{>0}\times C$. Moreover, the partial flat connection on $\R\oplus K_{C}$ induces a partially flat connection on $M$. As part of this data, each of the fiber spheres is equipped with a complex foliation of rank 1.

Compactification amounts to pushing forward a local $L_{\infty}$-algebra along $p|M$. The result is a theory with infinitely many Kaluza--Klein modes along the fiber spheres. In the holomorphic-topological setting, the Kaluza-Klein modes will be modeled by a variant of Cauchy-Riemann cohomology.

Moreover, including the flux sourced by the brane deforms this structure. The lowest lying Kaluza-Klein modes in the deformed theory are equivalent to 3d Chern-Simons.

For sake of analogy, we think of the resulting deformation as being a twisted version of $AdS_3 \times S^2$. \footnote{It is an interesting question if this corresponds the actual twist of a five--dimensional supersymmetric background of this form.}
We proceed to describe the twisted version of states at the boundary of this version of $AdS$.
We first proceed before turning on the flux sourced by the brane.

The theory admits a natural `vacuum' boundary condition at $r=0$.
In local coordinates, these are fields $\alpha(t,z,w)$ on the complement to the brane which extend to regular functions along the brane.

The `supergravity states' are, by definition, fields which satisfy the linearized equations of motion and satisfy the vacuum boundary condition except at a single point.
The linearized equations of motion are simply $(\d_{dR} + \dbar) \alpha = 0$.
Thus, up to equivalence, all solutions to the linearized equations of motion are constant in the real variable $t$, and holomorphic in $z,w$.

Modifications of the boundary condition at the point~$z = 0$ on the boundary take the form
\[
\alpha = f(w) \delta^{(r)}_{z=0}
\]
where $f$ is some holomorphic function.
Here $\delta^{(r)}_{z=0}$ denotes the $r$th derivative of the $\delta$-function at $z=0$.
It is convenient to parameterize such boundary modifications algebraically by expressions of the form
\[
\alpha_{k,r} = w^k \delta^{(r)}_{z=0} .
\]
Linear combinations of such states form a dense subspace of all possible modifications at the boundary.

The reason that the boundary modifications take this form can be seen by understanding in more explicit terms the vacuum boundary condition.
The phase space at the boundary $C$ can be identified with the following cohomology
\[
\Omega^{0,\bu}(C) \otimes \cA^{0;\bu}(\R \times \C - 0) [1]
\]
where $\cA^{0;\bu}$ denotes the mixed de Rham--Dolbeault cohomology of $\R \times \C - 0$ as a manifold equipped with a transversely holomorphic foliation \cite{DuchampKalka}.
We refer to the section below for a reminder on this geometric structure.

The phase space is equipped with a natural symplectic form given by
\[
\int_C \d z \oint_{S^2} \d w \, \alpha \wedge \alpha' .
\]
There is a natural Lagrangian inside of the phase space which consists of linear combinations of elements $\alpha(z) \otimes f(t,w)$ where $\alpha(z) \in \Omega^{0,\bu}(C)$ and $f(t,w)$ is a smooth function on $\R \times \C - 0$ which extends to zero.
The linearized equations of motion simply say that $\alpha$ is holomorphic, $f$ is independent of $t$ and depends holomorphically on $w$

\parsec[s:brfive]

We now consider the situation of backreacting some number of (twisted) fivebranes in our eleven-dimensional model.
Let $Z$ be a three-fold that the fivebranes wrap.
We also fix a rank 2 holomorphic vector bundle $V\to Z$ such that $\wedge^{2} V \cong K_{Z}$;
this condition ensures that the total space of $V$ is a Calabi-Yau five-fold. In the main body of the paper we will choose $V$ to be the bundle $K_{Z}^{1/2}\otimes \C^{2}$.

Consider the bundle $\R\oplus V$; this bundle has a canonical partially flat connection. We wish to consider our eleven dimensional model on $X = Tot (\R\oplus V)$ which is the total space of the \textit{real} rank five bundle $\R\oplus V$ over $Z$. The partially flat connection on $\R\oplus V$ equips $X$ with a canonical THF structure $F_{X}\subset T_{X}$.

We place a stack of $N$ fivebranes wrapping the zero section in $\R\oplus V$.
Denote the complement of the zero section by
\[
M_V = \text{Tot}(\R\oplus V) - 0(Z).
\]
Notice that in \S \ref{s:Lsugra} we have only defined the sheaf of $L_\infty$ algebras $\cL_{sugra}$ on a product of a smooth one-manifold times a Calabi--Yau five-fold.
The eleven-manifold $M_V$ is not of this form, nevertheless there is a generalization of $\cL_{sugra}$ which one can define using the natural geometric structure present in our situation.

A transversely holomorphic foliation (THF) on a smooth manifold $M$ is an integrable subbundle $F \subset \T_M \otimes \C$ such that $F + \Bar{F} = \T_M \otimes \C$.
We will say that $F$ equips $M$ with the a THF structure.
%Suppose $M$ is a manifold equipped with a THF structure and let $\cF$ be the corresponding foliation of even codimension.
The product $M = S \times X$, where $X$ is a complex manifold and $S$ is a smooth manifold has a natural THF structure with $F$ the restriction of the tangent bundle of $N$ along the projection.
Locally, any THF manifold is split of the form $\R^d \times \C^n$, whose coordinates we will denote by $(x_i ;  z_j)$.
The bundle $F$ is locally spanned by the vector fields $\partial / \partial x_i$'s and $\del/\del \zbar_j$'s.
(Notice that when $F \cap \Bar{F} = 0$ we are just describing an ordinary complex structure on $M$.)

Any submanifold of a THF manifold is itself a THF manifold.
We are most interested in the submanifold $M_V \subset {\rm Tot}(\R \oplus V) = \R \times X$ where we equip $\R \times X$ with its standard split THF structure.

We have expressed the fields of the eleven-dimensional theory in terms of a mixed type of de Rham and Dolbeault cohomology.
Let us focus on the fields $\beta,\gamma$ which on $\R \times X$ combine to form the complex
\beqn\label{eqn:drdol}
\Omega^{\bu}(\R) \otimes \Omega^{0,\bu}(X) \xto{1 \otimes \del} \Omega^{\bu}(\R) \otimes \Omega^{1,\bu}(X) .
\eeqn
As usual, we leave the $\d_{dR}$ and $\dbar$ operators implicit.
More generally, there is a natural cohomology associated to a THF structure.
Suppose $(M,F)$ is a THF structure and
denote by $Q$ the (complex) quotient bundle $\T_\C M / F$.
For each $p,q$ denote by $\cA^{p;q}$ smooth sections of the bundle $\wedge^p Q^\vee \otimes \wedge^q F^\vee$.
The derivative along the leaves of the foliation defined by $V$ defines a map
\[
\thfd \colon \cA^{p;q} \to \cA^{p;q+1}  .
\]
By integrability one has $\thfd^2 = \thfd \circ \thfd = 0$ and so $\thfd$ equips $\cA^{p;\bu} = \oplus_q \cA^{p;q}[-q]$ with the structure of a cochain complex for each $p$.
Locally in a split THF structure the operator $D$ is of the form $\d_{dR} + \dbar$ where $\d_{dR}$ is the de Rham differential along $\R^d$ and $\dbar$ is the Dolbeault operator along $\C^n$.
There is also an analog of the holomorphic $\del$ operator which takes the form $\thfdel \colon \cA^{p;q} \to \cA^{p+1;q}$.
The obvious exterior product $\cA^{p;q} \times \cA^{r;s} \to \cA^{p+r;q+s}$ further endows
\[
\left(\cA^{\bu;\bu} (M), \thfd + \thfdel\right) = \left(\oplus_p \cA^{p;\bu}[-p] , \thfd + \thfdel \right)
\]
with the structure of a bigraded commutative dg algebra.
This complex is simply isomorphic to the de Rham complex of $M$, but this presentation lends itself to more interesting quotient complexes.
For example, the forms of type $(p,\bu)$ with $p \geq 2$ form an ideal inside of this dg algebra; hence we get a quotient dg algebra
\beqn\label{thfcoh1}
\left(\cA^{\leq 1;\bu}(M), \thfd + \thfdel\right) = \quad \cA^{0;\bu} \xto{\thfdel} \cA^{1;\bu} .
\eeqn
We leave the $\thfd$ operator implicit in the presentation on the right hand side.
When $M = M_V$, it is this complex that is the THF generalization of the truncated de Rham--Dolbeault complex in \eqref{eqn:drdol}---it is easy to see that it agrees with this complex in the case of a split THF manifold.
There is a similar THF description for the fields $\mu,\nu$ in the eleven-dimensional theory.

%Note that the eleven-manifold $\R \times X$ is equipped with a natural transverseley--holomorphic foliation (THF)---the complexified tangent bundle decomposes as $T_{\R}\oplus T_{Z}\oplus \Bar{T}_{Z}$.
With this THF cohomological description of the eleven-dimensional theory in place we proceed to describe the boundary condition obtained by removing the location of the branes.
We may choose fiber coordinates of the bundle $t, w_{1}, w_{2}$ of $\R \oplus V$ over $Z$ and a fiberwise partially hermitian metric.
Explicitly, the corresponding norm defines a map
\begin{align*}
 h \colon  X & \to \R_{+} \\
  (t, w_{i}, \bar{w_{i}}, p)& \mapsto t^{2} + |w_{1}|^{2}+|w_{2}|^{2}
\end{align*}
Letting $\pi \colon X \to Z$ be the natural projection, we obtain the $S^{4}$ bundle
\[
p \define (h,\pi) \colon \R \times X \to \R_{+}\times Z
\]
which restricts to an $S^4$ bundle $p|M \colon M \to \R_{>0} \times Z$.
These embeddings and projections fit inside of the following commutative diagram
\[
\begin{tikzcd}
M \ar[d,"p|M"'] \ar[r,hook] & X \ar[d,"p"] & \ar[l,hook',"0"'] Z \ar[d,"="] \\
\R_{>0} \times Z \ar[r,hook] & \R_{+} \times Z & \ar[l,hook',"0 \times \id"] Z.
\end{tikzcd}
\]
The inclusions on the left are the natural embeddings.
The top right inclusion is the zero section of ${\rm Tot}(\R \oplus V) = X$ and the bottom right inclusion is the embedding at radius $r = 0$.

As we just elaborated, the eleven-dimensional theory is defined on the THF manifold $M$---in the BV formalism this is encoded, in part, by the sheaf of $L_\infty$ algebras $\cL_{sugra}$ on $M$.
Compactification of this theory along the $S^4$ link corresponds to pushing forward this sheaf along $p|M$.
The resulting sheaf of $L_\infty$ algebras $(p|M)_*\cL_{sugra}$ describes, in the BV formalism, the compactified theory on the seven-manifold $\R_{>0} \times Z$.

The theory on $M$ extends to a theory on the manifold obtained by filling in the zero section of $\R \times V$; in other words, we know that the theory is defined on the entire space $\R \times X$.
This means that there is a natural way to extend the theory on $\R_{>0} \times Z$ to the seven-manifold with boundary $\R_{+} \times Z$.
The restriction of this theory to the six-dimensional boundary plays the most important role for us.

\brian{trying to incorporate below}
%To do this we will make use of the natural foliated geometric structures which we have around.

Recall that we have a THF structure on $X$ induced from a partially flat conneciton on $\R\oplus V$; this is codified by saying that there is a splitting of the exact sequence

\[
0 \to \ker \to F_{X}\to \pi^{*}T^{1,0}_{Z}\to 0
.\]

Both $F_{X}$ and $\pi^{*}T^{1,0}_{Z}$ are involutive, and the flatness of the connection implies that that the splitting preserves the lie brackets on sections.

Consider the relevant tangent sequence of the map


The fibers of the composition $V_{M}\to M\to \R_{+}\times Z$ are copies of the tangent bundle of $S^{4}$, and the corresponding fibers of $V_{M}$ are subbundles of $TS^{4}$ that equip the fiber 4-spheres with a generalized Cauchy-Riemann structure. \surya{CITE}

 The underlying sheaf of cochain complexes is given by

\[
\Omega^{\bullet}(\R_{+})\otimes \left ( \begin{tikzcd}
\ul{\rm even} & \ul{\rm odd} \\
\PV^{1,\bu}(Z) \ar[r, "\del_{\Omega}"] & \PV^{0,\bu}(Z)\\
\Omega^{0,\bu}(Z) & \Omega^{1,\bu}(Z) \ar[l, "\del"]
\end{tikzcd}
 \right ) \otimes CR (S^{4}).
\]


Here $CR (S^{4})$ denotes the cohomology of the tangential Cauchy-Riemann complex of $S^{4}$ \surya{CITE}, equipped with the above Cauchy-Riemann structure. Its computation is facilitated by the following lemma:

\begin{lem}
  Let $\R^{d}\times \C^{n}$ be an affine THF manifold, and choose a partial hermitian metric. Let $S^{d+2n-1}$ denote the corresponding unit sphere, equipped with its standard generalized Cauchy-Riemann structure. Then there is a quasi-isomorphism

  \[CR (S^{n+2d-1})\cong \cA^{\bu;\bu}\left ( (\R^{d}\times \C^{n})\setminus 0 \right )\]

  where the right-hand-side denotes the Dolbeault-deRham complex.
\end{lem}

The cohomology of the Dolbeault-deRham complex of $\R\times \C^{2}$ is easy to describe.


It was argued in \cite{RSW} that to leading order the coupling of a stack of twisted fivebranes to the eleven-dimensional theory is given by the nonlocal interaction
\beqn\label{eqn:br1}
I_{M5} = N\int_{Z} \div^{-1}\mu \vee \Omega +\cdots
\eeqn
where $\mu \in \Omega^0 (\R) \hotimes \PV^{1,3}(X)$ is a component of a field in the eleven-dimensional theory which satisfies $\div \mu = 0$.

\parsec
Let $C$ be a curve, and let $V\to C$ be a rank 4-holomorphic vector bundle over $C$ such that $\wedge^{4} V = K_{C}$. This condition again ensures that $X = {\rm Tot} V$ is a Calabi-Yau five-fold - in the main body of the paper, we will take $V = K^{1/2}_{C}\otimes \C^{4}$. Abusively letting $V$ also denote its pullback along the canonical projection $\R\times C \to C$, we may view $\R\times X$ as the total space of $V$ on $\R\times C$. As before we will consider wrapping a stack of $N$ membranes along the zero section.

Since $V$ is a complex vector bundle, we may choose a fiberwise hermitian metric, and as before, we may view $\R\times X \setminus \R\times C$ as an $S^{7}$-bundle over $\R_{>0}\times \R\times C$.



%\parsec[s:sugraops]
%
%By the usual methods of the BV formalism the action functional $S_{sugra}$ described above endows the parity shift of the fields $\cL_{sugra} = \Pi \cF_{sugra}$ with the structure of a holomorphic-topological local $\Z/2$ graded $L_\infty$ algebra. 
%
%On $\C^5 \times \R$ we can describe this super Lie algebra structure explicitly. 
%First, by the Dolbeault and de Rham Poincar\'e lemmas it is easy that the even part of the super Lie algebra $\cL(\C^5 \times \R)$ is equivalent to a one-dimensional central summand $\C$ plus the Lie algebra of divergence-free vector fields on $\C^5$:
%\[
%\Vect_0 (\C^5) = \{X \in \Vect(\C^5) \; | \; \div X = 0\} .
%\]
%The odd part of the super Lie algebra $\cL(\C^5 \times \R)$ is equivalent to the space of holomorphic one-forms on $\C^5$ modulo exact one-forms
%\[
%\Omega^{1,hol}(\C^5) / {\rm Im}(\del) 
%\]
%which is, of course, equivalent to the space of closed holomorphic two-forms $\Omega^{2,hol}_{cl}(\C^5)$. 
%
%\begin{thm}[\cite{RSW}[Theorem 2.1]]
%The Taylor expansion map determines a map of $\Z/2$ graded $L_\infty$ algebras
%\[
%j_\infty \colon \cL_{sugra}(\C^5 \times \R) \to L_{sugra} .
%\]
%Furthermore, $L_{sugra}$ is equivalent as a $\Z/2$ graded $L_\infty$ algebra to $\Hat{E(5|10)}$. 
%\end{thm} 
%
%As an immediate corollary of this result we obtain by Lemma \ref{lem:localops} the following.
%
%\begin{cor}
%\label{cor:sugraops}
%Let $\Obs_{sugra}$ be the factorization algebra on $\C^5 \times \R$ of classical observables of the minimal twist of eleven-dimensional supergravity.
%There is a quasi-isomorphism of commutative dg algebras
%\[
%\Obs_{sugra} (0) \simeq \clie^\bu \left( \Hat{E(5|10)} \right) .
%\]
%\end{cor}

\subsection{Global symmetry for twisted $AdS$}
\label{s:global1}

After complexification, the~six-dimensional and three-dimensional superconformal algebras are isomorphic to $\lie{osp}(8|4)$.
The even part of this algebra is $\lie{so}(8) \times \lie{sp}(4)$.
This algebra contains the six-dimensional $\cN=(2,0)$ supersymmetry algebra whose odd part is four copies of $S^{6d}_+$, the positive irreducible complex spin representation of $\lie{so}(6)$.
It also contains the three-dimensional $\cN=8$ supersymmetry algebra whose odd part is eight copies of $S^{3d}$, the irreducible complex spin representation of $\lie{so}(3)$. 

In the six-dimensional case, the holomorphic supercharge is a supertranslation 
\[
Q \in \Pi S^{6d}_+ \otimes \C^4 \subset \lie{osp}(8|4)
\]
which is characterized (up to equivalence) by the properties that $Q^2 = 0$ and that its image
\[
{\rm Im}\left(Q|_{\Pi S_+ \otimes \C^4} \right) \subset \R^6 \otimes_\R \C \cong \C^6
\]
is three-dimensional (spanned by the anti-holomorphic translations). 
The supercharge $Q$ acts on $\lie{osp}(8|4)$ by commutator and the resulting cohomology will automatically act on the holomorphic twist of any six-dimensional superconformal field theory. 
This cohomology can readily be identified with the subalgebra $\lie{osp}(6|2)$, see \cite{SWe36}. 

Similarly, the minimal twisting supercharge in the three-dimensional $\cN=8$ supersymmetry algebra is an element $Q \in \Pi S^{3d} \otimes \C^8$ which is characterized (up to equivalence) by the property that $Q^2 = 0$ and that the image of $[Q,-]$ is two-dimensional. 
The cohomology of $\lie{osp}(8|4)$ with respect to this supercharge is also isomorphic to~$\lie{osp}(6|2)$. 

In the untwisted situation, the symmetry of solutions to eleven-dimensional supergravity in the presence of fivebranes wrapping a six-dimensional affine subspace is exactly the superconformal algebra $\lie{osp}(8|4)$ (after complexification). 
Geometrically, this background is $AdS_7 \times S^4$. 
Similarly, for backreacting membranes the background is $AdS_4 \times S^7$. 
In \cite{RSW} we have proposed a twisted analog of the $AdS$ background and have shown that solutions to equations of motion of our eleven-dimensional theory in the presence of twisted fivebranes and membranes contains the symmetry algebra $\lie{osp}(6|2)$---which is precisely the twists of the superconformal algebras we just discussed.

We will count the single-particle gravitational states in our eleven-dimensional model for the geometry resulting from backreacting twisted fivebranes and membranes.
As recalled above, such states have a symmetry by the twisted superconformal algebra $\lie{osp}(6|2)$.
We will enumerate states via choosing a Cartan in the bosonic subalgebra of the twisted superconformal algebra. Let us recall how the bosonic subalgebra embeds as symmetries of the eleven-dimensional theory in this twisted background.
This manifests as an embedding of this bosonic subalgebra into the ghosts of the eleven-dimensional theory. 
The embedding is distinct for fivebranes and membranes.
We turn first to the fivebrane case. 

\subsection{Supergravity states for twisted $AdS_7$}

For convenience we choose coordinates on the eleven manifold as
\[
\R \times \C^5 = \R_t \times \C^2_w \times \C_z^3 
\]
with $z = (z_i), i=1,2,3$ and $w = (w_a), a=1,2$.
The stack of fivebranes wrap $w_1=w_2=t=0$. 
Important for us is to recall that part of the ghost system for our eleven-dimensional theory consists of divergence-free vector fields on $\C^5$ which are locally constant along $\R$. 

\begin{itemize}
\item
The subalgebra $\lie{sl}(3)$ embeds as vector fields
\beqn
\sum_{ij} A_{ij} z_i \frac{\del}{\del z_j} \in \PV^{1,0}(\C^5)\otimes \Omega^0(\R) , \quad (A_{ij}) \in \lie{sl}(3) .
\eeqn
By definition, these vector fields are automatically divergence-free.

\item
        The generator of the subalgebra $\lie{gl}(1)$ is mapped to the element
        \beqn
        Y = \sum_{i=1}^3 z_i\frac{\del}{\del z_i} - \frac 32\sum_{a=1}^2 w_a\frac{\del}{del w_a}\in \PV^{1,0}(\C^5)\otimes \Omega^0 (\R).
        \eeqn
    Notice that this vector field is divergence-free and restricts to the Euler vector field along $t=w_{a} = 0$.
\item 
The subalgebra $\lie{sl}(2)$ ($R$-symmetry) is mapped to the triple
\beqn
 w_1 \frac{\del}{\del w_2}, w_2 \frac{\del}{\del 1}, \frac{1}{2}\left (w_1\frac{\del}{\del w_1}-w_2\frac{\del}{\del w_2}) \in \PV^{1,0}(\C^5) \otimes \Omega^0(\R) .
\eeqn
\end{itemize}

%\begin{rmk}
%In the classification of simple super Lie algebras, Kac makes use of a weight grading $\oplus_{j \geq -2} \fg_j$ of the exceptional Lie algebra $E(3|6)$ for which the finite-dimensional subalgebra above is the weight zero piece
%\cite{KacClass}.
%We will make use of this grading in \S \ref{s:kr}.
%\end{rmk}

The dimension of a Cartan subalgebra of $\lie{sl}(3) \times \lie{sl}(2) \times \lie{gl}(1)$ is four and accordingly, the equivariant character we study has four fugacities.
We choose these explicitly as follows:
\begin{itemize}
  \item $t_{1}, t_{2}$ denote generators for the Cartan of $\lie{sl}(3)$ which is generated by the vector fields
  \beqn
  h_1 = z_1 \frac{\del}{\del {z_1}} - z_2 \frac{\del}{\del{z_2}} , \quad h_2 = z_2 \frac{\del}{\del{z_2}} - z_3 \frac{\del}{\del{z_3}}.
  \eeqn
   \item $q$ denotes a generator for the Cartan of the~$\lie{gl}(1)$ which is generated by the element $Y$ from equation~$\eqref{eqn:Y}$. 
  \item $r$ denotes a generator for the Cartan of a $\lie{sl}(2)$ which is generated by the element 
  \beqn
  h = \frac12 \left(w_1 \frac{\del}{\del w_1} - w_2 \frac{\del}{\del w_2}\right) .
  \eeqn
\end{itemize}

The twisted supergravity states $\cH_{sugra}^{6d}$ form a representation for $\lie{osp}(6|2)$. 
The weights of twisted supergravity states with respect to the generators of the Cartan subalgebra above are completely determined by the weights of the holomorphic coordinates on $\C^2_w \times \C^3_z$.
These are summarized in table \ref{tbl:sugraM5}.

\begin{table}
\begin{center}
\begin{tabular}{c c c c c c}
  & $z_{1}$ & $z_{2}$ & $z_{3}$ & $w_{1}$ & $w_{2}$ \\
  \hline
  $t_{1}$ & $1$ & 0 & $-1$ & 0 & 0 \\
  $t_{2}$ & 0 & 1 & $-1$ & 0 & 0 \\
  $r$ & 0 & 0 & 0 & 1 & $-1$ \\
  $q$ & $-1$ & $-1$ & $-1$ & $\frac{3}{2}$ & $\frac{3}{2}$
\end{tabular}
\caption{Fugacities for the fields of the holomorphic twist of eleven-dimensional supergravity for the geometry $\R \times \C^5 \setminus \C^3$.}
\label{tbl:sugraM5}
\end{center}
\end{table}

We enumerate single particle supergravity states via computing the super trace of the operator $q^Y t_1^{h_1} t_2^{h_2} r^h$ acting on $\cH^{6d}_{sugra}$:
\beqn
f^{6d}_{sugra}(q,t_1,t_2,r) = \Tr_{\cH_{sugra}^{6d}} (-1)^F q^Y t_1^{h_1} t_2^{h_2} r^h .
\eeqn
The super trace means that there is an extra factor of $(-1)^F$, where $F$ is parity (fermion number), when computing the ordinary trace. 
That is, we compute the expression


\begin{prop}
The single particle index of the space of twisted supergravity states $\cH_{sugra}^{6d}$ is given by the following expression
\beqn
f_{sugra}^{6d} (q, t_{1}, t_{2}, r) = \frac{q^4(t_1^{-1}+t_1t_2^{-1}+t_2)-q^2(t_1+t_1^{-1}t_2+t_2^{-1})+(q^{3/2}-q^{9/2})(r+r^{-1})}{(1-t_{1}^{-1}q)(1-t_{2}q)(1-t_{1}t_{2}^{-1}q)(1-rq^{3/2})(1-r^{-1}q^{3/2})}.
\eeqn
\end{prop}

We record a few specializations of this index which we will remark on further in \S \ref{s:??}.
\parsec 
The specialization of this index $q=r^2, t_2=1$ in \eqref{eqn:special1} yields the plethystic exponential of the following single particle index
\[
f_{sugra}^{6d}(q, t_1, t_2=1, r = q^{1/2}) = \frac{q}{(1-q)^2}
\]

This plethystic exponential yields the Macmahon function, which is the character of the vacuum module of the $W_{1+\infty}$-algebra.

\parsec

The specialization $t_1=t_2=r=1$ yields the single particle index
\[
f_{sugra}^{6d} (q, t_1=t_2=r=1) = \frac{3 q^4 - 3 q^2 + 2 q^{3/2} - 2 q^{9/2}}{(1-q)^3 (1-q^{3/2})^2} .
\]

\parsec The same change of variables in \eqref{eqn:special2} agrees with previously computed indices for single particle states for supergravity on $AdS_{7}\times S^{4}$ \surya{...} \brian{not sure where this was supposed to go?}

\subsection{Supergravity states for twisted $AdS_4$}
For convenience we choose coordinates on the eleven manifold as
\[
\R \times \C^5 = \R_t \times \C^4_w \times \C_z^
\]
with $w = (w_a), a=1,2,3,4$.
The stack of membranes wrap $w_1=w_2=w_{3}=w_{4} = 0$.

\begin{itemize}
\item
The subalgebra $\lie{sl}(4)$ ($R$-symmetry) embeds as vector fields
\beqn
\sum_{ab} A_{ab} w_a \frac{\del}{\del w_b} \in \PV^{1,0}(\C^5)\otimes \Omega^0(\R) , \quad (A_{ab}) \in \lie{sl}(4) .
\eeqn
By definition, these vector fields are automatically divergence-free.

\item
The subalgebra $\lie{sl}(2)$ ($R$-symmetry) is mapped to the triple
\beqn
 \frac{\del}{\del z}, z \frac{\del}{\del z}-\frac{1}{4}\sum_{a=1}^4 w_a\frac{\del}{\del w_a}, \frac12 \left(w_1 \frac{\del}{\del w_1} - w_2 \frac{\del}{\del w_2}\right) \in \PV^{1,0}(\C^5) \otimes \Omega^0(\R) .
\eeqn
\end{itemize}

%\begin{rmk}
%In the classification of simple super Lie algebras, Kac makes use of a weight grading $\oplus_{j \geq -2} \fg_j$ of the exceptional Lie algebra $E(3|6)$ for which the finite-dimensional subalgebra above is the weight zero piece
%\cite{KacClass}.
%We will make use of this grading in \S \ref{s:kr}.
%\end{rmk}
The equivariant character again has four fugacities, which we explicitly choose as follows:
\begin{itemize}
  \item $t_{1}, t_{2}, t_{3}$ denote generators for the Cartan of $\lie{sl}(4)$ which is generated by the vector fields
  \beqn
  h_1 = w_1 \frac{\del}{\del {w_1}} - w_4 \frac{\del}{\del{w_4}} , \quad h_2 = w_2 \frac{\del}{\del{w_2}} - w_4 \frac{\del}{\del{w_4}} , \quad h_3 = w_3\frac{\del}{\del w_3}-w_4\frac{\del}{\del w_4}.
  \eeqn
  \item $q$ denotes a generator for the Cartan of the $\lie{sl}(2)$ which is generated by the vector field
        \beqn
        T =  z \frac{\del}{\del z}-\frac{1}{4}\sum_{a=1}^4 w_a\frac{\del}{\del w_a}
        \eeqn
\end{itemize}


Once again, the twisted spergravity states $\cH_{sugra}^{3d}$ form a representation for $\lie{osp}(6|2)$.
The weights of twisted supergravity states with respect to the generators of the Cartan subalgebra above are completely determined by the weights of the holomorphic coordinates on $\C^4_w \times \C_z$.
These are summarized in table \ref{tbl:sugraM5}.

\begin{table}
\begin{center}
\begin{tabular}{c c c c c c}
  & $z$ & $w_{1}$ & $w_{2}$ & $w_{3}$ & $w_{4}$ \\
  \hline
  $t_{1}$ & $1$ & 0 & $-1$ & 0 & 0 \\
  $t_{2}$ & 0 & 1 & $-1$ & 0 & 0 \\
  $t_{3}$ & 0 & 0 & 0 & 1 & $-1$ \\
  $q$ & $-1$ & $\frac 14$ & $\frac 14$ & $\frac{1}{4}$ & $\frac{1}{4}$
\end{tabular}
\caption{Fugacities for the fields of the holomorphic twist of eleven-dimensional supergravity for the geometry $\R \times \C^5 \setminus \R\times \C$.}
\label{tbl:sugraM5}
\end{center}
\end{table}

We enumerate single particle supergravity states via computing the super trace of the operator $q^T t_1^{h_1} t_2^{h_2} r^h$ acting on $\cH^{3d}_{sugra}$:
\beqn
f^{3d}_{sugra}(q,t_1,t_2,t_3) = \Tr_{\cH_{sugra}^{3d}} (-1)^F q^T t_1^{h_1} t_2^{h_2} t_3^{h_3} .
\eeqn
The super trace means that there is an extra factor of $(-1)^F$, where $F$ is parity (fermion number), when computing the ordinary trace.
That is, we compute the expression


\begin{prop}
The single particle index of the space of twisted supergravity states $\cH_{sugra}^{3d}$ is given by the following expression
\beqn
f_{sugra}^{3d} (q, t_{1}, t_{2}, t_3) = \frac{\left(\begin{aligned}
        & -q^{-7/4}(t_{1}^{-1}+t_{2}^{-1}+t_{3}^{-1}+t_{1}t_{2}t_{3}) +q^{1/4}(t_{1}+t_{2}+t_{3}+t_{1}^{-1}t_{2}^{-1}t_{3}^{-1}) \\
        & +q^{1/2}(1-q)(t_{1}t_{2}+t_{1}t_{3}+t_{2}t_{3}+t_{1}^{-1}t_{2}^{-1}+t_{1}^{-1}t_{3}^{-1}+t_{2}^{-1}t_{3}^{-1})\end{aligned}\right)}{(1-q)(1-t_{1}q^{1/4})(1-t_{2}q^{1/4})(1-t_{3}q^{1/4})(1-t_{1}^{-1}t_{2}^{-1}t_{3}^{-1}q^{1/4})}.
\eeqn
\end{prop}

\parsec Upon performing the change of variables

\beqn
q= x^{2} , \quad t_1 = (y_{2}y_{3})^{1/2}/y_1^{1/2} , \quad t_2 = (y_{1}y_3)^{1/2}/ y_2^{1/2} , \quad t_3 = (y_1 y_2)^{1/2}/y_{3}^{1/2}
\eeqn

the result agrees with previously computed indices for single particle states for supergravity on $AdS_{4}\times S^{7}$ \cite{}
