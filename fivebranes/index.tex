\documentclass[11pt]{amsart}
%
%%\usepackage{../macros-master}
\usepackage{macros-fivebrane}
%
\begin{document}

\section{Twisted supergravity states}

operators, indices, etc..
relationship to the $S^5$ partition function.

%\parsec[s:sugraops]
%
%By the usual methods of the BV formalism the action functional $S_{sugra}$ described above endows the parity shift of the fields $\cL_{sugra} = \Pi \cF_{sugra}$ with the structure of a holomorphic-topological local $\Z/2$ graded $L_\infty$ algebra. 
%
%On $\C^5 \times \R$ we can describe this super Lie algebra structure explicitly. 
%First, by the Dolbeault and de Rham Poincar\'e lemmas it is easy that the even part of the super Lie algebra $\cL(\C^5 \times \R)$ is equivalent to a one-dimensional central summand $\C$ plus the Lie algebra of divergence-free vector fields on $\C^5$:
%\[
%\Vect_0 (\C^5) = \{X \in \Vect(\C^5) \; | \; \div X = 0\} .
%\]
%The odd part of the super Lie algebra $\cL(\C^5 \times \R)$ is equivalent to the space of holomorphic one-forms on $\C^5$ modulo exact one-forms
%\[
%\Omega^{1,hol}(\C^5) / {\rm Im}(\del) 
%\]
%which is, of course, equivalent to the space of closed holomorphic two-forms $\Omega^{2,hol}_{cl}(\C^5)$. 
%
%\begin{thm}[\cite{RSW}[Theorem 2.1]]
%The Taylor expansion map determines a map of $\Z/2$ graded $L_\infty$ algebras
%\[
%j_\infty \colon \cL_{sugra}(\C^5 \times \R) \to L_{sugra} .
%\]
%Furthermore, $L_{sugra}$ is equivalent as a $\Z/2$ graded $L_\infty$ algebra to $\Hat{E(5|10)}$. 
%\end{thm} 
%
%As an immediate corollary of this result we obtain by Lemma \ref{lem:localops} the following.
%
%\begin{cor}
%\label{cor:sugraops}
%Let $\Obs_{sugra}$ be the factorization algebra on $\C^5 \times \R$ of classical observables of the minimal twist of eleven-dimensional supergravity.
%There is a quasi-isomorphism of commutative dg algebras
%\[
%\Obs_{sugra} (0) \simeq \clie^\bu \left( \Hat{E(5|10)} \right) .
%\]
%\end{cor}

\subsection{Global symmetry for twisted fivebranes}

After complexification, the~six-dimensional superconformal algebra is $\lie{osp}(8|4)$ whose even part is $\lie{so}(8) \times \lie{sp}(4)$. 
This algebra contains the six-dimensional $\cN=(2,0)$ supersymmetry algebra whose odd part is four copies of $S_+$, the positive irreducible complex spin representation of $\lie{so}(6)$.
The holomorphic supercharge is a supertranslation 
\[
Q \in \Pi S_+ \otimes \C^4 \subset \lie{osp}(8|4)
\]
which is characterized (up to equivalence) by the properties that $Q^2 = 0$ and that its image
\[
{\rm Im}\left(Q|_{\Pi S_+ \otimes \C^4} \right) \subset \R^6 \otimes_\R \C \cong \C^6
\]
is three-dimensional (spanned by the anti-holomorphic translations). 
The supercharge $Q$ acts on $\lie{osp}(8|4)$ by commutator and the resulting cohomology will automatically act on the holomorphic twist of any six-dimensional superconformal field theory. 
This cohomology can readily be identified with the subalgebra $\lie{osp}(6|2)$, see \cite{SWe36}. 

In the untwisted situation, the symmetry of solutions to eleven-dimensional supergravity in the presence of fivebranes wrapping a six-dimensional affine subspace is exactly the superconformal algebra $\lie{osp}(8|4)$ (after complexification). 
Geometrically, this background is $AdS_7 \times S^4$. 
In \cite{RSW} we have proposed a twisted analog of the AdS background and have shown that solutions to equations of motion of our eleven-dimensional theory in the presence of twisted fivebranes contains the symmetry algebra $\lie{osp}(6|2)$---which is precisely the twisted superconformal algebra.

In this section we will compute characters of the single-particle gravitational states in our eleven-dimensional model for the geometry resulting from backreacting twisted fivebranes.
Such states have a symmetry by the twisted superconformal algebra $\lie{osp}(6|2)$. 
We will enumerate states via considering a Cartan in the even subalgebra
\beqn\label{eqn:bosonic1}
\lie{sl}(3) \times \lie{sl}(2) \times \lie{gl}(1) \subset \lie{osp}(6|2) .
\eeqn
%We will understand how the twisted superconformal algebra acts on fivebranes via holographic principles.



We recall how the bosonic subalgebra \eqref{eqn:bosonic1} of this twisted superconformal algebra embeds as symmetries of the eleven-dimensional theory in this twisted background. 
This manifests as an embedding of this bosonic subalgebra into the ghosts of the eleven-dimensional theory. 

For convenience we choose coordinates on the eleven manifold as
\[
\R \times \C^5 = \R_t \times \C^2_w \times \C_z^3 
\]
with $z = (z_i), i=1,2,3$ and $w = (w_a), a=1,2$.
The stack of fivebranes wrap $w_1=w_2=t=0$. 
Important for us is to recall that part of the ghost system for our eleven-dimensional theory consists of divergence-free vector fields on $\C^5$ which are locally constant along $\R$. 

\begin{itemize}
\item
The subalgebra $\lie{sl}(3)$ embeds as vector fields
\beqn
\sum_{ij} A_{ij} z_i \frac{\del}{\del z_j} \in \PV^{1,0}(\C^5)\otimes \Omega^0(\R) , \quad (A_{ij}) \in \lie{sl}(3) .
\eeqn
By definition, these vector fields are automatically divergence-free.
\item
The generator of the subalgebra $\lie{gl}(1)$ is mapped to the element
\beqn\label{eqn:Y}
Y = \sum_{i=1}^3 z_i \frac{\del}{\del z_i} - \frac32 \sum_{a=1}^2 w_a \frac{\del}{\del w_a} \in \PV^{1,0}(\C^5) \otimes \Omega^0(\R)  .
\eeqn 
Notice that this vector field are divergence-free and restricts to the ordinary dilation (Euler vector field) along $t=w=0$. 
\item 
The subalgebra $\lie{sl}(2)$ ($R$-symmetry) is mapped to the triple
\beqn
w_1 \frac{\del}{\del w_2}, w_2 \frac{\del}{\del w_1}, \frac12 \left(w_1 \frac{\del}{\del w_1} - w_2 \frac{\del}{\del w_2}\right) \in \PV^{1,0}(\C^5) \otimes \Omega^0(\R) .
\eeqn
\end{itemize}

\begin{rmk}
In the classification of simple super Lie algebras, Kac makes use of a weight grading $\oplus_{j \geq -2} \fg_j$ of $E(3|6)$ for which the finite-dimensional subalgebra above is the weight zero piece
\cite{KacClass}.
\end{rmk}

The dimension of a Cartan subalgebra is four and accordingly, the equivariant character we study has four fugacities.
We choose these explicitly as follows:
\begin{itemize}
  \item $t_{1}, t_{2}$ denote generators for the Cartan of $\lie{sl}(3)$ which is generated by the vector fields
  \beqn
  h_1 = z_1 \del_{z_1} - z_2 \del_{z_2} , \quad h_2 = z_2 \del_{z_2} - z_3 \del_{z_3}.
  \eeqn
   \item $q$ denotes a generator for the Cartan of the~$\lie{gl}(1)$ which is generated by the element~$\eqref{eqn:Y}$. 
  \item $r$ denotes a generator for the Cartan of a $\lie{sl}(2)$ which is generated by the element 
  \beqn
  h = \frac12 \left(w_1 \frac{\del}{\del w_1} - w_2 \frac{\del}{\del w_2}\right) .
  \eeqn
\end{itemize}
The weights are summarized in table \ref{tbl:weights1}.
Recall that the coordinates $(r_1,r_2)$ are used to label the basis of $\C^2$ in the space of fields $\Gamma^{hol}(\Hat{D}^3, K^{1/2}) \otimes \C^2$. 

%\begin{table}
%\begin{center}
%\begin{tabular}{c c c c c c}
%  & $z_{1}$ & $z_{2}$ & $z_{3}$ & $r_{1}$ & $r_{2}$ \\
%  \hline
%  $t_{1}$ & $1$ & $-1$ & $0$ & $0$ & $0$ \\
%  $t_{2}$ & $0$ & $1$ & $-1$ & $0$ & $0$ \\
%  $r$ & $0$ & $0$ & $0$ & $1$ & $-1$ \\
%  $q$ & $-1$ & $-1$ & $-1$ & $0$ & $0$
%\end{tabular}
%\caption{Fugacities for a single fivebrane.}
%\end{center}
%\label{tbl:weights1}
%\end{table}

\subsection{Global symmetry for twisted membranes}

\subsection{Counting supergravity states}

We enumerate single particle supergravity states via computing the super trace of the operator $t_1^{h_1} t_2^{h_2} q^Y r^h$. 
The super trace means that there is an extra factor of $(-1)^F$, where $F$ is parity (fermion number), when computing the ordinary trace.

\parsec[s:pexp]

\brian{characters, plethystic, etc.}

%\parsec[s:singleops]
%
%\begin{prop}
%Let $\Obs_{fivebrane}$ be the observables of the theory on a single fivebrane on $\C^3$ after performing the holomorphic twist.
%There is a quasi-isomorphism of super commutative dg algebras 
%\[
%\Obs_{fivebrane} (0) \simeq \Sym \left( I(0,0;1;-1)^* \right) .
%\]
%\end{prop}
%\begin{proof}
%This follows from Lemma \ref{lem:localops}.
%In \cite{SWsuca6d} a dg lift of Kac's module $I(0,0;1;-1)$ was defined with the following bigraded pieces:
%\beqn
%\text{bidegree} (-1,+)\colon \quad \Hat{\Omega}_3^{2} , \qquad \text{bidegree} (0,+)\colon \quad \Hat{\Omega}^{3}_3
%\eeqn
%and
%\beqn
%\text{bidegree} (-1, -)\colon \quad \Hat{\Omega}^{3/2}_{3} \otimes R .
%\eeqn
%The differential is the (algebraic) de Rham differential $\del \colon \Hat{\Omega}^2 \to \Hat{\Omega}^3$. 
%The differential is zero on the odd summand.
%We will call this cochain complex $(E_{fivebrane}(\Hat{D}^3), \del)$.
%
%From the description of the fields of the holomorphic twist of the single fivebrane it is easy to see that the Taylor expansion at $0 \in \C^3$ defines a cochain map
%\[
%(\cE_{fivebrane} (\C^3), Q) \to (E_{fivebrane}(\Hat{D}^3), \del) 
%\]
%That this induces a quasi-isomorphism $J_0^\infty \cE_{fivebrane} \simeq E (\Hat{D}^3)$ follows from the Dolbeault Poincar\'e lemma.
%\end{proof}

\subsection{A stack of two fivebranes}

\parsec[]

Recall that the even part of $E(3|6)$ ... 

\begin{itemize}
\item Single particle operators coming from the copy of holomorphic vector fields $\Vect^{hol}(\C^3)$ contribute
\[
q^3 \frac{t_1^{-1} q + t_1 t_2^{-1} q + t_2 q }{(1-t_1^{-1}q) (1-t_1 t_2^{-1} q) (1-t_2 q)} 
\]
\item Single particle operators coming from $\lie{sl}(2)$-valued holomorphic functions $\lie{sl}(2) \otimes \cO^{hol}(\C^3)$ contribute
\[
q^3\frac{r^2 + r^{-2} + 1}{(1-t_1^{-1}q) (1-t_1 t_2^{-1} q) (1-t_2 q)} 
\]
\item Single particle operators coming from the odd piece of $E(3|6)$ which is $\Omega^{1,hol} \otimes K^{-1/2} \otimes \C^2$ contribute
\[
q^{3}\frac{(q^{1/2} r + q^{1/2} r^{-1})(t_1^{-1} + t_1t_2^{-1} + t_2)}{(1-t_1^{-1}q) (1-t_1 t_2^{-1} q) (1-t_2 q)}
\]
\end{itemize}

\begin{conj}
The character of local operators of the holomorphic twist of the theory on a stack of two fivebranes is given by the following plethystic exponential
\[
{\rm PExp} \left[f_{two} (t_1,t_2,q,r) \right] .
\]
where the single particle index is
\[
f_{two} (t_1,t_2,q,r) = \frac{q^4(t_1^{-1} + t_1 t_2^{-1}  + t_2) + q^3 (r^2 + r^{-2} + 1) - q^{7/2} (r + r^{-1})(t_1^{-1} + t_1t_2^{-1} + t_2)}{(1-t_1^{-1}q) (1-t_1 t_2^{-1} q) (1-t_2 q)} .
\]
\end{conj}

\parsec[]

The specialization of this index $t_1=t_2=r=1$ yields the single particle index
\[
\frac{3q^4 + 3 q^3 - 6 q^{7/2}}{(1-q)^3}. 
\]

\parsec[]

The specialization of this index $q=r^2, t_2=1$ in \eqref{eqn:special1} yields the plethystic exponential of the following single particle index
\[
f_{two}(t_1, 1, q, q^{1/2}) = \frac{q^2}{1-q} 
\]
which is the same as the single particle index of Virasoro vacuum module on the Riemann surface $\Sigma = \C_{z_1}$. 

\subsection{The large $N$ limit}

We now compute the equivariant character counting local operators of the holomorphic twist of the theory on a large number of fivebranes. 
Unlike the analysis above, we will the twisted holography proposal to compute this character. We consider the equivariant character of local operators of the holomorphic twist of a large number of fivebranes with respect to a finite dimensional subalgebra
\[
\lie{sl}(3)\times \lie{sl}(2)\times \lie {gl}(1)\subset E(5|10)
\]
which embeds in the following way:
\begin{itemize}
\item The Lie algebra $\lie{sl}(3)$ is a subalgebra of $\Vect_0(\Hat{D}^3) \subset \Vect_0 (\Hat{D}^5)$ consisting of linear coefficient vector fields $\sum_{i,j} a_{ij} z_i \frac{\del}{\del z_j}.$
where $(a_{ij}) \in \lie{sl}(3)$. 
\item The Lie algebra $\lie{sl}(2)$ rotates the directions $\C_{w_1} \times \C_{w_2}$ transverse to the brane.
\item The one-dimensional Lie algebra $\lie{gl}(1)$ is spanned by the vector field 
\beqn\label{vf1}
\sum_i z_i \frac{\del}{\del z_i} - \frac23 \sum_{j} w_j \frac{\del}{\del w_j} \in \Vect_0(\Hat{D}^5).
\eeqn
\end{itemize}

Correspondingly, the equivariant character thus has four fugacities, which we label as follows:
\begin{itemize}
  \item $t_{1}, t_{2}$ again denote generators for the Cartan of $\lie{sl}(3)$ acting by vector fields on $\C^{3}_{z}$.
  \item $r$ denotes a generators for the Cartan of a $\lie{sl}(2)$ acting by the fundamental representation on the normal directions $\C_{w_1} \times \C_{w_2}$.
  \item $q$ denotes a generator for the Cartan of additional $\lie{gl}(1)$ generated by the divergence-free vector field \eqref{vf1}. 
\end{itemize}

This choice of weights is summarized in Table \ref{tbl:sugra}.

\begin{table}
\begin{center}
\begin{tabular}{c c c c c c}
  & $z_{1}$ & $z_{2}$ & $z_{3}$ & $w_{1}$ & $w_{2}$ \\
  \hline
  $t_{1}$ & $1$ & 0 & $-1$ & 0 & 0 \\
  $t_{2}$ & 0 & 1 & $-1$ & 0 & 0 \\
  $r$ & 0 & 0 & 0 & 1 & $-1$ \\
  $q$ & $-1$ & $-1$ & $-1$ & $\frac{3}{2}$ & $\frac{3}{2}$
\end{tabular}
\caption{Fugacities for the fields of the holomorphic twist of eleven-dimensional supergravity.}
\label{tbl:sugra}
\end{center}
\end{table}

\parsec


\begin{prop}
  The character of local operators of the holomorphic twist of the theory on a large number of fivebranes is given by the following plethystic exponential

  \[ \rm PExp [f_{large N}(t_{1}, t_{2}, q, r)] \]

  where the single particle index is \[f_{large N}(t_{1}, t_{2}, q, r) = \frac{q^4(t_1^{-1}+t_1t_2^{-1}+t_2)-q^2(t_1+t_1^{-1}t_2+t_2^{-1})+(q^{3/2}-q^{9/2})(r+r^{-1})}{(1-t_{1}^{-1}q)(1-t_{2}q)(1-t_{1}t_{2}^{-1}q)(1-rq^{3/2})(1-r^{-1}q^{3/2})}.\]
\end{prop}

\parsec

The specialization $t_1=t_2=r=1$ yields the single particle index
\[
\frac{3 q^4 - 3 q^2 + 2 q^{3/2} - 2 q^{9/2}}{(1-q)^3 (1-q^{3/2})^2} .
\]

\parsec 
The specialization of this index $q=r^2, t_2=1$ in \eqref{eqn:special1} yields the plethystic exponential of the following single particle index
\[
f_{two}(t_1, 1, q, q^{1/2}) = \frac{q}{(1-q)^2}
\]

This plethystic exponential yields the Macmahon function, which is the character of the vacuum module of the $W_{1+\infty}$-algebra.

\parsec The same change of variables in \eqref{eqn:special2} agrees with previously computed indices for single particle states for supergravity on $AdS_{7}\times S^{4}$ \surya{...}

\end{document}
