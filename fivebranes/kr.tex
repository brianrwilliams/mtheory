\documentclass[11pt]{amsart}
%
%%\usepackage{../macros-master}
\usepackage{macros-fivebrane}
%
\begin{document}

\section{A series of finite $N$ conjectures}

A result of \cite{RSW} relates the global sections of the $L_\infty$ algebra $\cL_{sugra}$ describing the minimal twist of eleven-dimensional supergravity on $\R \times \C^5$ with a central ($L_\infty$) extension of the exceptional super Lie algebra $E(5|10)$.
From the point of view of (twisted) extended objects in $M$ theory like membranes and fivebranes there are two very important subalgebras of $E(5|10)$: the exceptional super Lie algebras $E(1|6)$ and $E(3|6)$. 

The exceptional super Lie algebras $E(1|6)$ and $E(3|6)$ play the role of infinite-dimensional `enhanced' superconformal algebras---we expect that they act on the holomorphic twist of any superconformal field theory in dimensions three and six, respectively.
From the point of view of $M$ theory these super Lie algebras act as symmetries of supergravity backgrounds which are twisted analogs of $AdS_4 \times S^7$ and $AdS_7 \times S^4$, respectively---these arise from backreacting, in flat space, some number of twisted membranes and fivebranes, respectively. 

In \cite{KR1}, a $\Z$-graded decomposition of $E(5|10)$ is defined which has the property that the weight zero subalgebra is isomorphic to the exceptional super Lie algebra $E(3|6)$.
In this section we compare our decomposition of the $!$-dual factorization algebra of $\clie^\bu(\cL_{sugra})$ on the complex three-fold $Z = \C^3$, see \S \ref{s:flatdecomp}, to this decomposition considered in \cite{KR1}.

Naturally, every other graded summand appearing in the decomposition of $E(5|10)$ is a module for $E(3|6)$---it turns out that every module that appears in this way is irreducible.
Following the description of these irreducible modules given in \cite{KR1,KR2} we compute (un)refined characters and proceed to give a conjectural procedure for computing (un)refined fivebrane indices.

There is a different decomposition of $E(5|10)$ (and its central extension) which has as its weight zero piece the subalgebra $E(1|6)$; this decomposition is natural from the point of view of membranes. 


%an embedding of $E(3|6)$ into $E(5|10)$ is constructed by considering the following auxiliary weight grading on the super Lie algebra $E(5|10)$. 
%We first split the coordinates on the formal five-disk as
%\[
%(z_i ; w_a) \in \Hat{D}^3 \times \Hat{D}^2 .
%\]
%In the fivebrane picture, the coordinates $z_i$ parameterize the holomorphic directions along the brane and the coordinates $w_a$ are the holomorphic coordinates transverse to the brane.

\subsection{The fivebrane decomposition of $\Hat{E(5|10)}$}

%Following the prescription of twisted holography our ansatz for 

We begin by turning to the algebraic side of things and start by recalling the weight decomposition of $E(5|10)$ constructed in \cite{KR1}.
In \S \ref{s:m5localops} we return to fivebranes and use this algebraic decomposition to characterize local operators of the twisted theory.

\parsec[s:e510weight]

As above, we denote by $\{z_i\}$ the local coordinates along the three-fold $Z=\C^3$ that the fivebrane wraps and $\{w_a\}$ for the transverse holomorphic coordinates to the zero section in ${\rm Tot}(K_{\C^3}^{1/2} \otimes \C^2)$.

The even part of the super Lie algebra $E(5|10)$ is formal vector fields on the formal disk $\Hat{D}^3_z \times \Hat{D}^2_w$, whose elements we denote by $\mu$. 
The odd part is given by the space of closed two-forms on the formal disk, whose elements we denote by $\alpha$.
Equivalently, we can think of the space of closed two-forms as the space of equivalence classes of one-forms $\Omega^1(\Hat{D}^3_z \times \Hat{D}^2) / \text{Im}(\del)$. 

Assign the following weights to the super Lie algebra $E(5|10)$. 
\begin{itemize} 
\item the coordinate $z_i$ has weight zero, $\wt(z_i) = 0$. 
\item the coordinate $w_a$ has weight $+1$, $\wt(w_a) = +1$. 
\item the parity of an element carries an additional weight of $-1$. 
Thus, for example, the odd element $[w_1 \d z_1] \in \Pi \Omega^1(\Hat{D}^5)/{\rm Im}(\del)$ carries weight $+1 - 1 = 0$. 
(If we think about the odd part as the space of closed two-forms then equivalently this grading translates to the one-form symbol $\d(-)$ as carrying weight $-1/2$.)
\end{itemize} 
The weight grading is concentrated in degrees $\geq -1$. 
In particular, there is a decomposition of super vector spaces
\beqn\label{eqn:decomp1}
E(5|10) = \til V_{-1} \times \prod_{n \geq 0} V_n 
\eeqn
with $\til V_{-1}$ being the weight $-1$ subspace and $V_n$ being the weight $n$ subspace for $n \geq 0$.  
It is straightforward to verify that this weight grading is compatible with the super Lie algebra structure on $E(5|10)$. 
In weight zero, $V_0$ is isomorphic to the super Lie algebra $E(3|6)$.
In particular, 
\beqn 
\label{eqn:e36iso}
E(3|6) \cong V_0 \subset E(5|10)
\eeqn
is a sub super Lie algebra and each $\til{V}_{-1}, V_n, n \geq 0$ is a module for $E(3|6)$. 

\parsec[s:e510central]

As we will recall momentarily, our eleven-dimensional model is more closely related to an $L_\infty$ central extension $\Hat{E(5|10)}$ of $E(5|10)$\cite[\S 3]{RSW}. 
The central extension of $E(5|10)$ is defined by the (totally even) three-linear cocycle
\beqn\label{eqn:e510central}
(\mu, \mu' , \alpha) \mapsto \<\mu \wedge \mu' , \alpha\>_{z=w=0} \in \C .
\eeqn

We can extend the weight grading of $E(5|10)$ to a weight grading of the central extension $\Hat{E(5|10)}$ by declaring that the central term have weight $-1$.
In this way, we get a related decomposition of super vector spaces
\beqn\label{eqn:decomp1}
\Hat{E(5|10)} = \prod_{n \geq -1} V_n
\eeqn
We will refer to this as the \textit{fivebrane decomposition} of $\Hat{E(5|10)}$.
Here $V_{-1}$ is a $\C$-extension of the module $\til V_{-1}$ defined in the decomposition above.
Notice that for $n \geq 0$ the $V_n$'s are the same as in the non centrally extended case.
In particular, we have the following. 

\begin{lem} 
The embedding $E(3|6) \hookrightarrow E(5|10)$ lifts to a map of super $L_\infty$ algebras 
\[
E(3|6) \hookrightarrow \Hat{E(5|10)} .
\]
\end{lem}
%\begin{proof}
%It suffices to show that the central extension of $E(5|10)$ splits when restricted to the subalgebra $E(3|6)$. 
%Recall that the even cocycle defining this extension is 
%\[
%(\mu, \mu', \alpha) \mapsto \<\mu \wedge \mu' , \alpha\>|_{z=0} .
%\]
%This is clearly zero since the image 
%\end{proof}

%\parsec[s:localopsdecompose]
%Notice that as a consequence of this decomposition there is a corresponding decomposition of local operators of the eleven-dimensional theory on $\C^5 \times \R$. 
%By corollary \ref{cor:sugraops} there is a decomposition of $\Z/2$-graded cochain complexes 
%\begin{align*}
%\Obs_{sugra} (0) & = \clie^\bu \left( \Hat{V}_{-1} \times \prod_{n \geq 0} V_n \right) \\
%& = \Sym(\Pi \Hat{V}^*_{-1}) \otimes \bigotimes_{n \geq 0} \Sym\left(\Pi V^*_n\right) .
%\end{align*}
%The differential $Q$, which is defined using the Lie bracket on $\Hat{E(5|10)}$ is of the form
%$Q = Q_0 + \Hat{Q}$ where $Q_0$ acts on linear generators via
%\[
%Q_0 \colon V^*_n \to \oplus_{m + \ell = n} V^*_m \otimes V^*_\ell
%\]
%and $\Hat{Q}$ acts on linear generators via
%\[
%\Hat{Q} \colon \CC \to \oplus_{m + \ell + n = -1} V^*_m \otimes V^*_\ell \otimes V^*_n .
%\]

\parsec[s:e510grading]

There is another auxiliary grading that we will make use of which is also defined in \cite{KR,KacClass}. 
The super Lie algebra $E(5|10)$ carries a (non-cohomological) consistent $\Z$-grading by declaring 
\begin{itemize}
\item $\deg z_i = \deg w_a = 2$, so that $\deg \del_{z_i} = \deg \del_{w_a} = -2$. 
\item Parity carries an additional degree of $-3/2$, so that $\deg \d z_i = \deg \d w_a = -1/2$.
\end{itemize}
This consistent $\Z$-grading extends to the central extension of $E(5|10)$ by declaring that the central term has degree $-5$. 
To distinguish this from the `weight' grading above, we will refer to this grading as `degree'.

\subsection{Decomposing fivebrane local operators}

The factorization algebra of classical observables for our model of twisted supergravity is $\Obs_{sugra} = \clie^\bu(\cL_{sugra})$ where $\cL_{sugra}$ is the local $L_\infty$ algebra of \S\ref{s:sugraL}. 
As in \S\ref{s:fact}, let us take the eleven-manifold that this theory is defined on to be $\R \times {\rm Tot}(K^{1/2}_Z \otimes \C^2)$ where $Z$ is a complex three-fold.
In \S\ref{s:??}, we defined the restriction $\Obs_{sugra}|_{Z}$ of the factorization algebra of twisted supergravity observables to the submanifold~$Z$.
%is a dense subfactorization algebra of the pushforward of $\clie^\bu(\cL_{sugra})$ along the obvious projection. 

The general prescription of twisted holography is that the $!$-dual of $\Obs_{sugra}|_Z$ is the factorization algebra of the universal defect supported along $Z$ in the eleven-dimensional theory.
The $!$-dual of the factorization algebra $\Obs_{sugra}|_Z$ is of the form $\clie_\bu (\cG_c)$ where $\cG$ is an infinite-dimensional local $L_\infty$ algebra on $Z$.
In \S \ref{s:weight1} we defined a $\C^\times$ action on $\cG$ which determines a (non-cohomological) weight grading
\beqn\label{eqn:decomp2a}
\cG = \bigoplus_{n \geq -1} \Omega^{0,\bu}(Z, \cV^{(n)}) 
\eeqn
where $\cV^{(n)}$ is some holomorphic vector bundle on $Z$.

For the remainder of this section we focus on the case $Z = \C^3$. 
Our goal is to show that this weight grading is compatible with the one on the central extension of $E(5|10)$ that we recalled in \S \ref{s:e510central}. 
In turn, this will give us a description of fivebrane local operators supported at $0 \in \C^3$ in terms of irreducible representations of the super Lie algebra $E(3|6)$. 

\parsec[]

First, we recall the precise relationship between the central extension of the super Lie algebra $E(5|10)$ and the local Lie algebra~$\cL_{sugra}$ on~$\R \times \C^5$.
The local $L_\infty$ algebra $\cL_{sugra}$ on $\R \times \C^5$ can be written as 
\[
\cL_{sugra} = \Omega^\bu(\R) \otimes \Omega^{0,\bu}(\C^5, L) \simeq \Omega^\bu(\R) \otimes \Omega^{0,\bu}(\C^5) \otimes L_0 ,
\]
where $L$ is a translation invariant holomorphic vector bundle and~$L_0$ its fiber over~$0 \in \C^5$.
As in \S \ref{s:interacting}, the local $L_\infty$ algebra structure on $\cL_{sugra}$ induces an $L_\infty$ structure on the $\infty$-jets at~$0 \in \R \times \C^5$ of $\cL_{sugra}$, which we can identify using the Poincar\'e lemma with~$L_0[[z_1,\ldots,z_5]]$.
Furthermore, the Taylor expansion map at $0 \in \R \times \C^5$ defines an $L_\infty$ map
\beqn
T_0 \colon \cL_{sugra}(\R \times \C^5) \to L_0[[w_1,w_2, z_1,z_2,z_3]] 
\eeqn
where we recall that we have used $\{z_i\}$ for the coordinates along the directions the fivebrane wraps in 
\[
\C^5 \simeq {\rm Tot}(K^{1/2}_Z \otimes \C^2)
\]
and $\{w_a\}$ for the transverse holomorphic coordinates. 

In~\cite{RSW} it is shown that $L_0[[w_1,w_2, z_1,z_2,z_3]]$ is equivalent to the super Lie algebra $\Hat{E(5|10)}$ which is the central extension of $E(5|10)$ by the cocycle in \brian{finish}. 
The restricted local $L_\infty$ algebra $\cG$ sits somewhere in-between $\cL_{sugra}(\R \times \C^5)$ and $\Hat{E(5,10)}$. 

The sheaf of local $L_\infty$ algebras $\cG$ on $\C^3$ is the smooth sections of a holomorphic translation invariant pro vector bundle $G$. 
Let $G_0$ be the fiber over $0 \in \C^3$. 
Similarly as above, since the space $G_0[[z_1,z_2,z_3]]$ is the $\infty$-jets of $\cG$ at $0 \in \C^3$ it is equipped with the structure of an $L_\infty$ algebra and the Taylor expansion map 
\[
T_0^{\C^3} \colon \cG(\C^3) \to G_0[[z_1,z_2,z_3]] 
\]
is a map of $L_\infty$ algebras. 

\begin{lem}
\begin{itemize}
\item[(1)]
There is an equivalence of $L_\infty$ algebras
\[
\Hat{E(5|10)} \simeq G_0 [[z_1,z_2,z_3]] .
\]
\item[(2)]The partial Taylor expansion of sections of $\cL_{sugra}(\R \times \C^5)$ at $w_1=w_2 = 0$ defines a map of $L_\infty$ algebras 
\[
\cL_{sugra}(\R \times \C^5) \to \cG(\C^3) ,
\]
and makes the following diagram commute
\[
\begin{tikzcd}
\cL_{sugra}(\R \times \C^5) \ar[dr,dotted] \ar[rr,"T_0^{\R \times \C^5}"] & &  \Hat{E(5|10)} \\ 
& \cG(\C^3) \ar[ur,"T_0^{\C^3}"'] & .
\end{tikzcd} 
\]
\item[(3)] 
The Taylor expansion map $T_0^{\C^3}$ is $\C^\times$-equivariant. 
\end{itemize}
\end{lem}
%First, we will see how to relate the decomposition at the level of global sections of~$\cL_{sugra}$ over~$Z = \C^3$. 

%In the decomposition \eqref{eqn:decomp2a} the 

In \S \ref{s:}. \brian{finish}

\subsection{Irreducible representations of $E(3|6)$}
We recall the requisite knowledge of irreducible representations of $E(3|6)$ as characterized in \cite{KR1,KR2,KR3}.
Since we have realized the super Lie algebra $E(3|6)$ as the weight zero subalgebra of the central extension of $E(5|10)$ we see that it carries an induced consistent $\Z$-grading by degree. 
Denote by $\fg_j$ the degree $j$ super subspace of $E(3|6)$.
One can check that the algebra is supported in degrees $j \geq -2$ so that as a super vector space
\[
E(3|6) = \prod_{j \geq -2} \fg_j
\]
where for small values of $j$ we have:
\begin{itemize}
\item $\fg_{-2} = {\rm span}\{\del_{z_i}\; | \; {i=1,2,3}\}$ which is a completely even super vector space. 
\item $\fg_{-1} = {\rm span}\{\d z_i \wedge \d w_a \; | \; i=1,2,3, a=1,2\}$ which is a completely odd super vector space.
\item $\fg_0 \cong \lie{sl}(3) \oplus \lie{sl}(2) \oplus \lie{gl}(1)$ which is a completely even super vector space.
Explicitly, the Cartan of $\lie{sl}(3)$ is spanned by 
\[
h_1 = z_1 \del_{z_1} - z_2 \del_{z_2}, \quad h_2 = z_2 \del_{z_2} - z_3 \del_{z_3} .
\]
The Cartan of $\lie{sl}(2)$ is spanned by
\[
h_3 = w_1 \del_{w_1} - w_2 \del_{w_2} .
\]
The Cartan of $\lie{gl}(1)$ is spanned by
\[
Y = \frac23 \sum_{i=1,2,3} z_i \del_{z_i} - \sum_{a=1,2} w_a \del_{w_a} .
\]
\end{itemize}

\parsec[s:vermas]

We are interested in a class of irreducible modules for $E(3|6)$ that can be constructed via resolutions of particular Verma modules \cite{KR2} that we now recall.
Denote by $\fg = E(3|6)$ and let 
\[
\fg_{<0} = \fg_{-2} \oplus \fg_{-1} , \quad \lie{g}_{>0} = \oplus_{j>0} \fg_j 
\]
so that $\fg = \fg_{<0} \oplus \fg_0 \oplus \fg_{>0}$. 
Given a $\fg_0$-module $V$ let 
\beqn\label{eqn:verma1}
M (V) \define U(\fg) \otimes_{U(\fg_{\geq 0})} V 
\eeqn
be the induced $\fg$-module. 
Notice that as super vector spaces~$M(V) = \Sym(\fg_{<0}) \otimes_\C V$.

We will be interested in Verma modules associated to two particular types of $\fg_0$-modules
\begin{align*}
V_A & = \C[z_i, w_a] \\
V_B & = \C[z_i, \del_{w_a}]_{[2]} .
\end{align*}
Here, the subscript $[2]$ means that we shift $\lie{gl}(1)$ to act by the differential operator 
\[
Y_{[2]} = Y + 2 \id_{V_B} .
\]
For $X = A,B$ we let 
\[
V_X^{m,n} = \{f \in V_X \; | \; \sum z_i \del_{z_i} f = m f , \quad \sum w_a \del_{w_a} f = n f.\} \subset V_X
\]
so that $V_X = \oplus_{m,n} V_X^{m,n}$. 
There is a similar decomposition of the resulting Verma modules $M(V_X)$ in terms of the Verma modules $M(V_{X}^{m,n})$. 

An irreducible $\fg_0$-module $F(\lambda_1,\lambda_2,\lambda_3,y)$ is characterized by a highest weight $\lambda = (\lambda_1,\lambda_2,\lambda_3,y)$ whose components we denote by
\begin{align*}
\lambda_i & = \lambda (h_i) \in \Z_{\geq 0}, \\
 y & = \lambda(Y)\in \C . 
\end{align*}
Actually, for the remainder of this section we will be interested in the case $\lambda_2 = 0$; the modules with $\lambda_2 \ne 0$ will play no role.
 
For $\lambda_1,\lambda_3 \geq 0$ there are isomorphisms of $\fg_0$-modules
\begin{align*}
V_A^{\lambda_1,\lambda_3} & \simeq F(\lambda_1,0,\lambda_3,\frac23 \lambda_1 - \lambda_3) \\
V_B^{\lambda_1,-\lambda_3} & \simeq F(\lambda_1,0,\lambda_3,\frac23 \lambda_1 + \lambda_3 +2) .
\end{align*} 
To simplify notation we will get rid of the dependence on $\lambda_2$ and write 
\begin{align*}
F_A(\lambda_1,\lambda_3) & \define F(\lambda_1,0;\lambda_3;\frac23 \lambda_1 - \lambda_3) \\
F_B(\lambda_1,\lambda_3) & \define F(\lambda_1,0;\lambda_3;\frac23 \lambda_1 + \lambda_3 +2)  
\end{align*}
and similarly $M_A(\lambda_1,\lambda_3)$, $M_B(\lambda_1,\lambda_3)$ for the associated Verma modules \eqref{eqn:verma1}.

One of the main results of \cite{KR1} characterizes the irreducible quotients of these Verma modules. 
\begin{thm}[\cite{KR1}]
The Verma modules $M_A(\lambda_1,\lambda_3)$ and $M_A(\lambda_1,\lambda_3)$ have unique irreducible quotients denoted by 
\beqn
I_A(\lambda_1,\lambda_3), \quad \text{and} \quad I_B(\lambda_1,\lambda_3)
\eeqn
respectively. 
Furthermore, for any $\lambda_1,\lambda_3 \in \Z_+$ one has $M_A(\lambda_1,\lambda_3) \ne I_A(\lambda_1,\lambda_3)$ and $M_B(\lambda_1,\lambda_3) \ne I_B(\lambda_1,\lambda_3)$.
\end{thm}

In \textit{loc. cit.} they refer to such irreducible modules $I_A,I_B$ as `degenerate'.

\parsec[]

The last bit of algebraic input we need is a result of \cite{KR2} which uses these irreducible quotients $I_A(\lambda_1,\lambda_3)$ and $I_{B} (\lambda_1,\lambda_3)$ to characterize the $E(3|6)$-representations which appear in the decomposition of $E(5|10)$. 

\begin{thm}[\cite{KR2}]
Consider the decomposition \eqref{eqn:decomp1} of the super Lie algebra
\beqn
E(5|10) = \til V_{-1} \times \prod_{i \geq -1} V_n
\eeqn
where $V_0 \simeq E(3|6)$. 
Then, there are isomorphisms of $E(3|6)$-modules
\begin{align*}
\til V_{-1}^* & \simeq I_A(0,1) \\
E(3|6)^* = V_0^* & \simeq I_A(1,0) \\
V_j^* & \simeq I_B(0,j-1), \quad j \geq 1
\end{align*}
where $(-)^*$ denotes the topological dual.
\end{thm}

For the centrally extension of $E(5|10)$ we need to adjust things slightly. 
The decomposition of the central extension is $\prod_{j \geq -1} V_j$, where only $V_{-1}$ differs from the weight $(-1)$ part $\til V_{-1}$ of $E(5|10)$.
It is easy to see that as $E(3|6)$-modules $V_{-1}^*$ is a direct sum 
\[
V_{-1}^* = \til V_{-1}^* \oplus \C = I_A(0,1) \oplus \C
\]
with $\C$ the trivial module. 
(In \cite[\S 5]{KR2} they refer to the module $V_{-1}^*$ as~$P$.)

\subsection{Modified characters for degenerate modules of type $A$} \label{s:typeA}

The (super) character of an $E(3|6)$-module $M$ is given by the formula
\[
\ch (M) \define {\rm tr}_M q^{Y} t_1^{h_1} t_2^{h_2} r^{h_3} .
\]
Our goal in this section is to compute the explicit formula for a modified version of this character in the case of the degenerate modules $I_A(\lambda_1,\lambda_3)$ of type $A$ which are irreducible quotients of the Verma modules $M_A(\lambda_1,\lambda_3)$. 

As a super vector space, the Verma module $M(V)$ associated to a $\fg_0$-module $V$ is 
\[
M(V) = \Sym(\lie{g}_{<0}) \otimes V .
\]
We define the modified character of a Verma module by the formula
\[
\til{\ch} \, M(V) = \ch \left(\Sym(\lie{g}_{<0}) \right)(q, t_1,t_2,r) \cdot \ch (V) \left(q^{-1}, t_1^{-1}, t_2^{-1}, r^{-1} \right)  .
\]
Define
\[
R(q,t_1,t_2,r) \define \ch \left(\Sym(\lie{g}_{<0}) \right)(q, t_1,t_2,r) ,
\]
so that the character of the Verma module $M(V)$ is a product of $R$ with the character of the $\fg_0$ representation $V$.
%Recall that $\lie{g}_{<0} = \lie{g}_{-2} \oplus \lie{g}_{-1}$ where $\lie{g}_{-2} \cong \C^3$ spanned by $\del_{z_i}$ and $\lie{g}_{-1} \cong \Pi \C^6$ spanned by the odd elements $r_a \d z_i$ for $a=1,2$ and $i=1,2,3$.

Our definition of the modified character relies on a cohomological description of the irreducible modules $I_A(\lambda_1,\lambda_3)$ of type~$A$.
%\[
%I_A(\lambda_1,\lambda_3) = I(\lambda_1,0,\lambda_3;\frac23 \lambda_1 - \lambda_3) .
%\]
In \cite{KR2}, an exact sequence of $E(3|6)$-modules is constructed
\beqn\label{eqn:les1}
0 \leftarrow I_A(\lambda_1,\lambda_3)\leftarrow M_A(\lambda_1,\lambda_3) \leftarrow M_A(\lambda_1+1,\lambda_3+1) \leftarrow \cdots .
\eeqn 
From the long exact sequence \eqref{eqn:les1} we have the following expression for the character
\beqn
\label{eqn:chA1}
\ch I_A(\lambda_1,\lambda_3) = \sum_{j \geq 0} \ch M_A(\lambda_1+j,\lambda_3+j)  .
\eeqn 
Like the Verma modules, we can defined a modified version of this character by the formula
\beqn
\label{eqn:chA2}
\til \ch \, I_A(\lambda_1,\lambda_3) = \sum_{j \geq 0} \til\ch \, M_A(\lambda_1+j,\lambda_3+j)  .
\eeqn 

We can apply formula \eqref{eqn:MA} to get an expression for the $j$th term the sum in \eqref{eqn:chA1}
\beqn\label{eqn:chA3}
\ch M_A(\lambda_1+j,\lambda_3+j) = t^{3 \lambda_3 - 2 \lambda_1+j} R(t,r,t_1,t_2) \ch_{SU(3)}^{\lambda_1+j}(t_1,t_2) \ch_{SU(2)}^{\lambda_3+j} (r)  .
\eeqn
And similarly for the sum in \eqref{eqn:chA2}.

The first ingredient in determining an explicit formula for the modified character of type $A$ Verma modules is a direct calculation.
\begin{lem}
\label{lem:R}
Then
%\beqn
%R = \frac{1}{(1-t_1^{-1} q)(1-t_1t_2^{-1}q)(1-t_2 q)}\prod_{a=+1,-1}(1-t_1^{-1} q^{1/2} r^a)(1-t_1 t_2^{-1} q^{1/2} r^a)(1-t_2 q^{1/2} r^a)
%\eeqn
\beqn
R = \frac{1}{(1-t_1^{-1} q)(1-t_1t_2^{-1}q)(1-t_2 q)}\prod_{a=+1,-1}(1-t_1^{-1} q^{1/2} r^a)(1-t_1 t_2^{-1} q^{1/2} r^a)(1-t_2 q^{1/2} r^a)
\eeqn
\end{lem} 

Recall that the type $A$ degenerate modules are defined by induction from the $\fg_0$-module 
\[
V = V^{\lambda_1,\lambda_3}_A \subset \C[z_i,w_a]
\] 
given by polynomials of homogenous degree $\lambda_1,\lambda_3$ in the variables $z_i,w_a$ respectively.
Thus, we have the following explicit presentation for the modified character of type $A$ modules. 
\begin{prop}
One has
\beqn\label{eqn:MA}
\til \ch \, {M_A(\lambda_1,\lambda_3)} = q^{y} R(q,r,t_1,t_2) \cdot \ch (V_A^{\lambda_1,\lambda_3}) (t_1^{-1},t_2^{-1},r^{-1})
\eeqn
where $y = \lambda_1 - \frac32 \lambda_3$, $R$ is as in Lemma \ref{lem:R} and 
\beqn\label{eqn:su}
\ch(V_A^{\lambda_1,\lambda_3}) = \ch_{SU(3)}^{\lambda_1}(t_1,t_2) \ch_{SU(2)}^{\lambda_3} (r) 
\eeqn
is the $SU(3) \times SU(2)$ character of the representation $V_A^{\lambda_1,\lambda_3}$. 
\end{prop}

In expression \eqref{eqn:su}, $\ch_{SU(3)}^{\lambda_1}$ denotes the character of the irreducible $SU(3)$ representation of highest weight $(\lambda_1,0)$ and $\ch^{\lambda_3}_{SU(2)}$ is the character of the irreducible $SU(2)$ representation of highest weight $\lambda_3$.


%which is
%\[
%0 \leftarrow I(p,0;r;\frac23 p - r) \leftarrow M(V^{p,r}) \leftarrow M(V^{p+1,r+1}) \leftarrow \cdots .
%\]

In the computations of the sections that follow we will make the substitution $q = t^2$ for notational simplicity.

Applying the Weyl character formula we obtain the explicit expressions for the $SU(3)$ character
\begin{align*}
\ch_{SU(3)}^{\lambda_1+j} (t_1,t_2) & = t_1^{\lambda_1+j} t_2^{-\lambda_1 -j} \sum_{k=0}^{\lambda_1 + j} t_1^{-k} t_2^{2k} \frac{1 - t_1^{-k-1} t_2^{-k-1}}{1-t_1^{-1} t_2^{-1}} 
%\\ 
%& = t_1^{-2/3} t_2^{2/3} \sum_{k=0}^{\lambda_1 + j} t_1^{k/2} t_2^{-3k/2} \sum_{l = 0}^k (t_1 t_2)^l .
\end{align*}
and $SU(2)$ character
\begin{align*}
\ch_{SU(2)}^{\lambda_3+j} (r) & = r^{\lambda_3+j}+ r^{\lambda_3+j-2} + \cdots + r^{-\lambda_3-j} \\ & = \frac{r^{\lambda_3+j}-r^{-\lambda_3-j-2}}{1-r^{-2}} .
\end{align*}

We proceed to evaluate the sum in \eqref{eqn:chA1}
\begin{multline}
t^{3\lambda_3-2 \lambda_1} t_1^{-\lambda_1} t_2^{\lambda_1} \sum_{j \geq 0} t^j t_1^{-j} t_2^{j} \left(r^{\lambda_3+j}-r^{-\lambda_3-j-2}\right) R \sum_{k=0}^{\lambda_1 + j}t_1^{2k} t_2^{-k} \frac{1 - t_1^{-k-1} t_2^{-k-1}}{1-t_1^{-1} t_2^{-1}}  \\
= \frac{t^{3\lambda_3-2 \lambda_1} t_1^{-\lambda_1} t_2^{\lambda_1}}{1-t_1^{-1} t_2^{-1}} R \sum_{j\geq 0} t^j t_1^{-j} t_2^{j} \left(r^{\lambda_3+j}-r^{-\lambda_3-j-2}\right) \sum_{k=0}^{\lambda_1 + j} t_1^{2k} t_2^{-k} \left(1 - (t_1t_2)^{-k-1}\right) \\
= \frac{t^{3\lambda_3-2 \lambda_1} t_1^{-\lambda_1} t_2^{\lambda_1}}{1-t_1^{-1} t_2^{-1}} R \sum_{j\geq 0} t^j t_1^{-j} t_2^{j} \left(r^{\lambda_3+j}-r^{-\lambda_3-j-2}\right) \left(\frac{1 - (t_1^2 t_2^{-1})^{\lambda_1 + j + 1}}{1 - t_1^2 t_2^{-1}} - t_1^{-1} t_2^{-1} \frac{1 - (t_1 t_2^{-2})^{\lambda_1 + j + 1}}{1 - t_1 t_2^{-2}} \right) \\ =
\frac{t^{3\lambda_3-2 \lambda_1} t_1^{-\lambda_1} t_2^{\lambda_1}}{(1-t_1^{-1} t_2^{-1})(1 - t_1^2 t_2^{-1})(1-t_1t_2^{-1})} R \sum_{j\geq 0} t^j t_1^{-j} t_2^{j} \left(r^{\lambda_3+j}-r^{-\lambda_3-j-2}\right) \times \\ 
\left[(1-t_1 t_2^{-2})(1-(t_1^2 t_2^{-1})^{\lambda_1+j+1}) - t_1^{-1}t_2^{-1} (1-t_1^2 t_2^{-1}) (1 - (t_1 t_2^{-2})^{\lambda_1+j+1}) \right]
%
%\\ = \frac{t^{3\lambda_3-2 \lambda_1} t_1^{-\lambda_1} t_2^{\lambda_1}}{(1-t_1^{-1} t_2^{-1})(1 - t_1^2 t_2^{-1})(1-t_1t_2^{-1})} \sum_{j\geq 0} t^j t_1^{-j} t_2^{j} \left(r^{\lambda_3+j}-r^{-\lambda_3-j-2}\right) \\ \times \left(1 + t_1^{2\lambda_1 +2j+3}t_2^{-\lambda_1-j-1} - t_1 t_2^{-2} - t_1^{2 \lambda_1 + 2j + 2} t_2^{-\lambda_1 - j -1} \right. \\ \left. - t_1^{-1} t_2^{-1} + t_1^{\lambda_1+j+2} t_2^{-2 \lambda_1 - 2j-4} + t_1 t_2^{-2} + t_1^{\lambda_1 + j} t_2^{-2\lambda_1 - 2 j - 3} \right)
\end{multline}

We split the sum $\sum_{j \geq 0}$ into four terms
\begin{itemize}
\item the first term is 
\beqn
(1-t_1t_2^{-2}) \sum_{j \geq 0} t^j r^{\lambda_3 + j} \left(t_1^{-j} t_2^j - t_1^{2 \lambda_1 + j + 2} t_2^{-\lambda_1 - 1} \right) .
\eeqn
Evaluating the infinite series this becomes
\begin{multline} 
r^{\lambda_3} (1-t_1 t_2^{-2}) \left(\frac1{1-trt_1^{-1}t_2} - \frac{t_1^{2\lambda_1+2}t_2^{-\lambda_1-1}}{1 - t r t_1} \right) \\ = \frac{r^{\lambda_3}(1-t_1 t_2^{-2})}{(1-trt_1^{-1}t_2)(1 - t r t_1)} \left(1 - trt_1-t_1^{2\lambda_1+2} t_2^{-\lambda_1-1}+tr t_1^{2 \lambda_1+1} t_2^{-\lambda_2}\right)
\end{multline}
\item the second term is
\beqn
t_1^{-1} t_2^{-1}(1-t_1^2 t_2^{-1}) \sum_{j \geq 0} t^j r^{\lambda_3 + j} \left(t_1^{-j} t_2^j - t_1^{\lambda_1+1} t_2^{-2\lambda_1 - j - 2}\right) .
\eeqn
\end{itemize}

Combining with the formula for $R(t,r,t_1,t_2)$ we arrive at the formula
\begin{multline} 
\ch M_A(\lambda_1+j,\lambda_3+j) = t^{3 \lambda_3 - 2 \lambda_1+j} 
\end{multline}



\parsec
In this section we deduce an explicit formula for the character of the type $A$ modules $I_A(\lambda_1,\lambda_3)$ after specializing $t_1=t_2=1$. 

\begin{prop}
\label{prop:chA}
Let $\lambda_1,\lambda_3 \geq 0$. 
The $t_1=t_2=1$ specialization of the modified character $\til \ch \, I_A(\lambda_1,\lambda_3)$ is given by the expression
\beqn
\frac{t^{2\lambda_1-3 \lambda_3+6}}{(1-t^2)^3(1-r^{-2})} \left(r^{\lambda_3} p_{\lambda_1}(t^{-1}r) (1-t^{-1}r^{-1})^3 - r^{-\lambda_3-2} p_{\lambda_1}(t^{-1}r^{-1})(1-t^{-1}r)^3 \right) 
\eeqn
where $p_{\lambda_1}(x) = \frac12 \left(\lambda_1^2 x^2 + \lambda_1 x^2 - 2\lambda_1^2x -4 \lambda_1 x + \lambda_1^2 + 3\lambda_1 +2\right)$. 
\end{prop}

\begin{proof}
In this specialization it is easy to see that 
\[
\ch_{SU(3)}^{\lambda_1+j}(t_1=t_2=1) = \frac{(\lambda_1+j+1)(\lambda_1+j+2)}{2} ,
\]
which is just the dimension of polynomials of homogenous degree $\lambda_1+j$ in three variables. 

We thus arrive at the expression for the $j$th term in the sum appearing in equation \eqref{eqn:chA2}:
\beqn
\til\ch M_A(\lambda_1+j,\lambda_3+j)_{t_1=t_2=1} = R(t,r) t^{2 \lambda_1-3\lambda_3} \frac{(\lambda_1+j+1)(\lambda_1+j+2)}{2} t^{-j} \frac{r^{\lambda_3+j}-r^{-\lambda_3-j-2}}{1-r^{-2}} 
\eeqn
where $R(t,r)$ is the specialization at $t_1=t_2=1$ of $R$
\[
R(t,r) = \frac{(1-tr)^3 (1-t r^{-1})^3}{(1-t^2)^3} ,
\]
which we read off from Lemma \ref{lem:R}.
In this expression we have also utilized the $SU(2)$ character formula for the irreducible module of heighest weight $\lambda_3+j$: 
\[
r^{\lambda_3+j}+ r^{\lambda_3+j-2} + \cdots + r^{-\lambda_3-j} = \frac{r^{\lambda_3+j}-r^{-\lambda_3-j-2}}{1-r^{-2}} ,
\]

Now, observe the formal identity 
\[
\sum_{j \geq 0} \frac{(\lambda_1+j+1)(\lambda_1+j+2)}{2} x^j = \frac{p_{\lambda_1}(x)}{(1-x)^3}  
\]
where $p_{\lambda_1}(x)$ is as in the statement of the proposition.
Using this identity, we evaluate
\begin{align*}
\til \ch \, I_A(\lambda_1,\lambda_3)|_{t_1=t_2=1} & = \frac{R(r,t) t^{2 \lambda_1-3 \lambda_3}}{1-r^{-2}} \left(\frac{r^{\lambda_3} p_{\lambda_1}(t^{-1}r)}{(1-t^{-1}r)^3} - \frac{r^{-\lambda_3-2}p_{\lambda_1}(t^{-1}r^{-1})}{(1-t^{-1}r^{-1})^3} \right) \\ 
& = \frac{t^{2\lambda_1-3 \lambda_3+6}}{(1-t^2)^3(1-r^{-2})} \left(r^{\lambda_3} p_{\lambda_1}(t^{-1}r) (1-t^{-1}r^{-1})^3 - r^{-\lambda_3-2} p_{\lambda_1}(t^{-1}r^{-1})(1-t^{-1}r)^3 \right) .
\end{align*}
This completes the proof
\end{proof}

In the case that $\lambda_1=0$ and $\lambda_3=1$ we see that this specialized character simplifies to
\beqn
\til \ch \, I_A(0,1)|_{t_1=t_2=1} = \frac{t^3(r+r^{-1}) + 1 - 3 t^2}{(1-t^2)^3} .
\eeqn

In the case that $\lambda_1 = 1$ and $\lambda_3 = 0$ we see that this unrefined character simplifies to
\beqn
\til \ch \, I_A(1,0)|_{t_1=t_2=1} = \frac{t^6(r^2 + r^{-2} + 1) - 3 t^7 (r + r^{-1}) + 3t^8 }{(1-t^2)^3} .
\eeqn

Finally, for general $\lambda_1,\lambda_3$ we specialize further $t=r$ and substitute $q = t^2$ to obtain
\beqn
\til \ch \, I_A(\lambda_1,\lambda_3) = \frac{q^{\lambda_1-\lambda_3+1}}{1-q} .
\eeqn

\subsection{Characters for degenerate modules of type B} \label{s:typeB}

There is a cohomological description of the irreducible modules of type B similar to the type A case above.
Recall that the degenerate module of type $B$ is $I_B(\lambda_1,\lambda_3) = I(\lambda_1,0,\lambda_3;\frac23 \lambda_1 - \lambda_3+2)$. 
We are only interested in the case $\lambda_1 = 0$ with $\lambda_3\geq 0$.  
In \cite{KR2}, an exact sequence of $E(3|6)$-modules is constructed
\begin{multline}\label{eqn:les1}
0 \leftarrow  I_B(0,\lambda_3) \\ \leftarrow M_B(0,\lambda_3) \leftarrow  M_B(1,\lambda_3-1)  \leftarrow \cdots \leftarrow M(\lambda_3, 0) \\
 \leftarrow M_A(\lambda_3+2, 0) \leftarrow M_A(\lambda_3+3,1) \leftarrow \cdots .
\end{multline}
There are $\lambda_3+1$ terms in the second line above .  
Notice that the bottom line is simply the resolution for the module $I(\lambda_3+2, 0)$ of type $A$. 
Using this long exact sequence we obtain a simple formula for the character of type $B$ modules
\beqn\label{eqn:chB1}
\ch I_B(0,\lambda_3) = \sum_{k=0}^{\lambda_3} \ch M_B(k, \lambda_3-k) + \ch I_A(\lambda_3+2,0) .
\eeqn
And similarly we have the following formula for the modified character
\beqn\label{eqn:chB2}
\til \ch \, I_B(0,\lambda_3) = \sum_{k=0}^{\lambda_3}\til \ch\, M_B(k, \lambda_3-k) + \til \ch \, I_A(\lambda_3+2,0) .
\eeqn

\begin{prop}
\label{prop:chB}
The modified character $\til \ch \, I_B(0,\lambda_3)$ is given by the expression
\begin{multline}
t^{2 \lambda_3 + 6} R(t,r) \frac{1-t^{\lambda_3+1}}{1 - t} \\ + \frac{t^{2\lambda_3 + 10}}{(1-t^2)^3 (1-r^{-2})} \left(p_{\lambda_3+2} (t^{-1} r) (1 - t^{-1} r^{-1})^3 - r^{-2} p_{\lambda_3}(t^{-1} r^{-1})(1-t^{-1}r)^3 \right) 
\end{multline}
where $p_{\lambda_3 + 2}(x)$ is as in Proposition \ref{prop:chA}. 
\end{prop}

\begin{proof}
We have already computed $\til \ch \, I_A(\lambda_1,\lambda_3)$ for any pair $(\lambda_1,\lambda_3)$ above. 
Thus, it suffices to compute the modified character of the type $B$ Verma modules.

As above, the modified character for the Verma module $M_B(k,\lambda_3-k)$, for $0 \leq k \leq \lambda_3$ is given by
\[
\til \ch \, M_B(k,\lambda_3-k)|_{t_1=t_2=1} = t^{3 \lambda_3 - k + 6} R(t,r) .
\]
Thus, \eqref{eqn:chB2} simplifies to 
\[
\til \ch \, I_B(0,\lambda_3) = t^{2 \lambda_3 + 6} R(t,r) \frac{1-t^{\lambda_3+1}}{1 - t} + \til \ch \, I_{A}(\lambda_3+2,0) .
\]
The result now follows from Proposition \ref{prop:chA}.
\end{proof}



\parsec[s:typeBunrefinedKR]


For integer $N$ let 
\[
p_{N}(x) = \frac12 \left(N^2 x^2 + N x^2 - 2N^2x -4 N x + N^2 + 3N +2\right) .
\]
Then the index for $N > 2$ fivebranes after specializing the $SU(3)$ fugacities is
\begin{multline}
t^{2 N} \frac{(1-tr)^3 (1-tr^{-1})^3 (1-t^{\lambda_3+1})}{(1-t^2)^3(1 - t)} \\ + \frac{t^{2N + 4}}{(1-t^2)^3 (1-r^{-2})} \left(p_{N-1} (t^{-1} r) (1 - t^{-1} r^{-1})^3 - r^{-2} p_{N-1}(t^{-1} r^{-1})(1-t^{-1}r)^3 \right) 
\end{multline}

\end{document}
