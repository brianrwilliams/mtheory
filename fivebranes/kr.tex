\documentclass[11pt]{amsart}

%\usepackage{../macros-master}
\usepackage{macros-fivebrane}

\begin{document}

\section{Kac-Rudokov}

In \cite{KacRudakov} an embedding of $E(3|6)$ into $E(5|10)$ is constructed by considering the following auxiliary weight grading on the super Lie algebra $E(5|10)$. 

We first split the coordinates on the formal five-disk as
\[
(z_i ; w_a) \in \Hat{D}^3 \times \Hat{D}^2 .
\]
In the fivebrane picture, the coordinates $z_i$ parameterize the holomorphic directions along the brane and the coordinates $w_a$ are the holomorphic coordinates transverse to the brane.

Assign the following weights to the super Lie algebra $E(5|10)$. 
\begin{itemize} 
\item the coordinate $z_i$ has weight zero, $\wt(z_i) = 0$. 
\item the coordinate $w_a$ has weight $+1$, $\wt(w_a) = +1$. 
\item the parity of an element carries an additional weight of $-1$. 
Thus, for example, the odd element $[w_1 \d z_1] \in \Pi \Omega^1(\Hat{D}^5)/{\rm Im}(\del)$ carries weight $+1 - 1 = 0$. 
\end{itemize} 
The weight grading is concentrated in degrees $\geq -1$. 
In particular, there is a decomposition of super vector spaces
\beqn\label{eqn:decomp1}
E(5|10) = \til V_{-1} \times \prod_{n \geq 0} V_n 
\eeqn
with $\til V_{-1}$ being the weight $-1$ subspace and $V_n$ being the weight $n$ subspace for $n \geq 0$.  
It is straightforward to verify that this weight grading is compatible with the super Lie algebra structure on $E(5|10)$. 
In particular 
\beqn 
\label{eqn:e36iso}
E(3|6) \cong V_0 \subset E(5|10)
\eeqn
is a sub super Lie algebra and each $V_n$ is a module for $E(3|6)$. 

\parsec[s:e510central]

The central extension of $E(5|10)$ is defined by the (totally even) cocycle
\beqn\label{eqn:e510central}
(\mu, \mu' , \alpha) \mapsto \<\mu \wedge \mu' , \alpha\>_{z=w=0} .
\eeqn

We can extend the weight grading of $E(5|10)$ to a weight grading of the central extension $\Hat{E(5|10)}$ by declaring that the central term have weight $-1$.
In this way, we get a related decomposition of super vector spaces
\beqn\label{eqn:decomp1}
E(5|10) = \prod_{n \geq -1} V_n
\eeqn
Here $V_{-1}$ is a $\C$-extension of the module $\til V_{-1}$ defined in the decomposition above.
Notice that for $n \geq 0$ the $V_n$'s are the same as in the non centrally extended case.
In particular, we have the following. 

\begin{lem} 
The embedding $E(3|6) \hookrightarrow E(5|10)$ lifts to a map of super $L_\infty$ algebras 
\[
E(3|6) \hookrightarrow \Hat{E(5|10)} .
\]
\end{lem}
%\begin{proof}
%It suffices to show that the central extension of $E(5|10)$ splits when restricted to the subalgebra $E(3|6)$. 
%Recall that the even cocycle defining this extension is 
%\[
%(\mu, \mu', \alpha) \mapsto \<\mu \wedge \mu' , \alpha\>|_{z=0} .
%\]
%This is clearly zero since the image 
%\end{proof}

\parsec[s:localopsdecompose]
Notice that as a consequence of this decomposition there is a corresponding decomposition of local operators of the eleven-dimensional theory on $\C^5 \times \R$. 
By corollary \ref{cor:sugraops} there is a decomposition of $\Z/2$-graded cochain complexes 
\begin{align*}
\Obs_{sugra} (0) & = \clie^\bu \left( \Hat{V}_{-1} \times \prod_{n \geq 0} V_n \right) \\
& = \Sym(\Pi \Hat{V}^*_{-1}) \otimes \bigotimes_{n \geq 0} \Sym\left(\Pi V^*_n\right) .
\end{align*}
The differential $Q$, which is defined using the Lie bracket on $\Hat{E(5|10)}$ is of the form
$Q = Q_0 + \Hat{Q}$ where $Q_0$ acts on linear generators via
\[
Q_0 \colon V^*_n \to \oplus_{m + \ell = n} V^*_m \otimes V^*_\ell
\]
and $\Hat{Q}$ acts on linear generators via
\[
\Hat{Q} \colon \CC \to \oplus_{m + \ell + n = -1} V^*_m \otimes V^*_\ell \otimes V^*_n .
\]

\parsec[s:e510grading]

The super Lie algebra $E(5|10)$ carries a (non-cohomological) consistent $\Z$-grading by declaring 
\begin{itemize}
\item $\deg z_i = \deg w_a = 2$, so that $\deg \del_{z_i} = \deg \del_{w_a} = -2$. 
\item $\deg \d z_i = \deg \d w_a = -1/2$.
\end{itemize}
This consistent $\Z$-grading extends to the central extension of $E(5|10)$ by declaring that the central term has degree $-5$. 

\subsection{Irreducible representations of $E(3|6)$}
Thinking of the super Lie algebra $E(3|6)$ as the weight zero subalgebra of $\Hat{E(5|10)}$ (or even before centrally extending) then it carries an induced consistent $\Z$-grading. 
Denote by $\fg_j$ the degree $j$ super subspace of $E(3|6)$.
In particular, as a super vector space
\[
E(3|6) = \oplus_{j \geq -2} \fg_j
\]
where
\begin{itemize}
\item $\fg_{-2} = {\rm span}\{\del_{z_i}\; | \; {i=1,2,3}\}$ which is completely even. 
\item $\fg_{-1} = {\rm span}\{\d z_i \wedge \d w_a \; | \; i=1,2,3, a=1,2\}$ which is completely odd. 
\item $\fg_0 \cong \lie{sl}(3) \oplus \lie{sl}(2) \oplus \lie{gl}(1)$ which is completely even. 
Explicitly, the Cartan of $\lie{sl}(3)$ is spanned by 
\[
h_1 = z_1 \del_{z_1} - z_2 \del_{z_2}, \quad h_2 = z_2 \del_{z_2} - z_3 \del_{z_3} .
\]
The Cartan of $\lie{sl}(2)$ is spanned by
\[
h_3 = w_1 \del_{w_1} - w_2 \del_{w_2} .
\]
The Cartan of $\lie{gl}(1)$ is spanned by
\[
Y = \frac23 \sum_{i=1,2,3} z_i \del_{z_i} - \sum_{a=1,2} w_a \del_{w_a} .
\]
\end{itemize}

\parsec[s:vermas]

Denote by $\fg = E(3|6)$ and let 
\[
\fg_{<0} = \fg_{-2} \oplus \fg_{-1} , \quad \lie{g}_{>0} = \oplus_{j>0} \fg_j 
\]
so that $\fg = \fg_{<0} \oplus \fg_0 \oplus \fg_{>0}$. 
Given a $\fg_0$-module $V$ let 
\[
M (V) \define U(\fg) \otimes_{U(\fg_{\geq 0})} V 
\]
be the induced $\fg$-module. 
Notice that as super vector spaces 
\[
M(V) = \Sym(\fg_{<0}) \otimes_\C V .
\]

We will be interested in Verma modules associated to two particular types of $\fg_0$-modules
\begin{align*}
V_A & = \C[z_i, w_a] \\
V_B & = \C[z_i, \del_{w_a}]_{[2]} .
\end{align*}
Here, the subscript $[2]$ means that we shift $\lie{gl}(1)$ to act by the differential operator 
\[
Y_{[2]} = Y + 2 \id_{V_B} .
\]
For $X = A,B$ we let 
\[
V_X^{m,n} = \{f \in V_X \; | \; \sum z_i \del_{z_i} f = m f , \quad \sum w_a \del_{w_a} f = n f.\} 
\]
so that $V_X = \sum_{m,n} V_X^{m,n}$. 
There is a similar decomposition of the resulting Verma modules $M(V_X)$ in terms of the Verma modules $M(V_{X}^{m,n})$. 

An irreducible $\fg_0$-module $F(\lambda_1,\lambda_2;\lambda_3;y)$ is characterized by a highest weight $\lambda$ which we denote by
\begin{align*}
\lambda_i & = \lambda (h_i) \in \Z_{\geq 0}, \\
 y & = \lambda(Y)\in \C . 
\end{align*}
There is an isomorphism of $\fg_0$-modules
\begin{align*}
V_A^{p,r} & \simeq F(\lambda_1,0;\lambda_3;\frac23 \lambda_1 - \lambda_3) \\
V_B^{p,-r} & \simeq F(\lambda_1,0;\lambda_3;\frac23 \lambda_1 + \lambda_3 +2) .
\end{align*} 

\subsection{Character computations}

Define the (super) character of an $E(3|6)$-module $M$ by the formula
\[
\ch \, M  \define {\rm tr}_M q^{Y} t_1^{h_1} t_2^{h_2} r^{h_3} .
\]
As a super vector space, the Verma module $M(V)$ associated to a $\fg_0$-module $V$ is 
\[
M(V) = \Sym(\lie{g}_{<0}) \otimes V .
\]
We have $\lie{g}_{<0} = \lie{g}_{-2} \oplus \lie{g}_{-1}$ where we recall that $\lie{g}_{-2} \cong \C^3$ spanned by $\del_{z_i}$ and $\lie{g}_{-1} \cong \Pi \C^6$ spanned by the odd elements $r_a \d z_i$ for $a=1,2$ and $i=1,2,3$. 


Given this description, there is a simple expression for the character 
\[
\ch\, {M(F(\lambda_1,\lambda_2;\lambda_3;y))} = q^{-y} t_1^{\lambda_1}t_2^{\lambda_2} r^{\lambda_3} R(q,t_1,t_2,r) \cdot {\rm sdim}( V )
\]
where $R = \ch \Sym(\fg_{<0})$. 

\begin{lem}
We have
\beqn
R = \frac{1}{(1-t_1^{-1} q)(1-t_1t_2^{-1}q)(1-t_2 q)}\prod_{a=+1,-1}(1-t_1^{-1} q^{1/2} r^a)(1-t_1 t_2^{-1} q^{1/2} r^a)(1-t_2 q^{1/2} r^a)
\eeqn
\end{lem}

\parsec[s:typeA]

There is a cohomological description of the irreducible modules of type $A$
\[
I(p,0;r;\frac23 p - r) .
\]
In \cite{KR2}, an exact sequence of $E(3|6)$-modules is constructed
\[
0 \leftarrow I(p,0;r;\frac23 p - r) \leftarrow M(p, 0 ; r;\frac23 p - r) \leftarrow M(p+1, 0;r+1;\frac23 p - r - \frac13) \leftarrow \cdots 
\]
which is
\[
0 \leftarrow I(p,0;r;\frac23 p - r) \leftarrow M(V^{p,r}) \leftarrow M(V^{p+1,r+1}) \leftarrow \cdots .
\]

84rPqzleGThz


\end{document}