%\documentclass[11pt]{amsart}
%
%%\usepackage{../macros-master}
%\usepackage{macros-fivebrane}
%
%\begin{document}

\section{Local operators in twisted $M$ theory}

The notion of a factorization algebra captures both the local operators of a theory together with the non-local operators that on can define from the local ones via descent.
From the data of a factorization algebra, one can recover local operators by the following formal construction. 
Let $\Obs$ be the factorization algebra of observables of some theory defined on a smooth manifold $M$.
The space of local operators at point $p \in M$ is, in a precise sense, the limiting behavior of the factorization algebra evaluated on the system of open sets which contain the point~$p$. 

Generally this limit is difficult to compute, but for certain theories it is possible to give a concise expression which captures the essential features of the theory.
For example, in a holomorphic theory, the algebra of local operators is equivalent to the algebra generated by holomorphic derivatives of fields evaluated at a point.

In this section we recall the essentials of the theory of local operators for holomorphic-topological theories. 
We consider a way of counting operators in a topological-holomorphic theory, called the `local character' of a holomorphic-topological theory \cite{SWchar}, and compare it to the superconformal index.
We then present a few simple examples and then go on to set up the theory of local operators associated to factorization algebras~$\clie_\bu(\cG_{N,c})$ we constructed in \S \ref{s:fact}.

\subsection{Local operators in topological-holomorphic theories}

A factorization algebra encodes the many ways to combine observables supported on arbitrary open sets. 
Local operators, on the other hand, exist just at a point in spacetime.
From the factorization algebra perspective one can recover local operators by looking at observables which are supported on \text{every} open set which contains the given point; mathematically this is computed by a limit. 

Precisely, in \cite[Definition 10.1.0.1]{CG2} the space of local operators of a factorization algebra $\cF$ at a point $p \in M$ is defined by the limit $\cF(p) = \lim_{U \ni p} \cF(U)$ which runs over open sets $U \subset M$ containing~$p$.

We will only consider local operators on affine space $\R^d$. 
In this case, we will have the additional property that the factorization algebras are translation invariant.
At the level of local operators this means that the translation map $\tau_{p \to p'}$ induces an isomorphism $\cF(p) \simeq \cF(p')$. 
Without loss of generality, we will consider expressions for local operators at $0 \in \R^d$.

For topological-holomorphic theories the local operators take a very familiar form.
As an algebra they are generated by (holomorphic) derivatives of the fields evaluated at the specified point. 
More precisely, the local operators depend only on the $\infty$-jets of the fields at a point.
In this section we carefully formulate this result and give some examples.

\parsec[s:free]

%Suppose that $V$ is a translation invariant holomorphic vector bundle on $\C^n$ equipped with a $\Z \times \Z/2$ bigrading. 
%Let $\cV$ denote its sheaf of holomorphic sections.
%The {\em space of fields} of a holomorphic field theory on $\C^n$ is the $\Z \times \Z/2$ graded complex of vector bundles
%\[
%\Omega^{0,\bu}(\C^n, V) \cong \Omega^{0,\bu}(\C^n) \otimes V_0 
%\]
%where $V_0$ is the fiber of $V$ at $0 \in \C^n$.
%Our grading conventions are so that $\d \zbar_i$ has bidegree $(1,0)$.
%
%As introduced in \cite{BWhol,LiVertex,CG2}, a {\em holomorphic field theory} is a holomorphic vector bundle $V$ as above equipped additionally with:
%\begin{itemize}
%\item The structure of a local (super) $L_\infty$ algebra on $V[-1]$ with structure maps given by holomorphic polydifferential operators
%\[
%[\cdot]_k \colon \cV[-1]^{\times k} \to \cV[1-k] .
%\]
%\end{itemize}
%A {\em free} holomorphic theory has $[\cdot]_k = 0$ for $k > 1$.
%
%\parsec
A topological-holomorphic theory exists on spacetimes of the form $S \times X$ where $S$ is a smooth manifold and $X$ is a complex manifold (possibly equipped with some auxiliary geometric structures). 
The typical space of fields of a holomorphic-topological theory in the BV formalism is
\beqn\label{eqn:cE}
\cE = \Omega^\bu (S) \hotimes \Omega^{0,\bu}(X, V) 
\eeqn
where $V$ is a graded holomorphic vector bundle on $X$.
The underlying free theory is described by a differential on the space of fields of the form
\[
\d_{dR} + \dbar + Q^{hol} .
\]
Here $\d_{dR}$ is the de Rham differential acting on $S$, $\dbar$ is the Dolbeault operator acting on $X$, and $Q^{hol} \colon V \to V[1]$ is a holomorphic differential operator of cohomological degree~$+1$.
This means that the free, linear equations of motion for a field $\varphi$ take the form
\[
\d_{dR} \varphi + \dbar \varphi + Q^{hol} \varphi = 0 .
\]
Taking into account linear gauge symmetries corresponds to cohomology---solutions to the equations of motion modulo the image of $\d_{dR} + \dbar + Q^{hol}$.

Notice that $\cE$ is a sheaf of cochain complexes---it makes sense to restrict the fields to any open set $U \subset S \times X$. 
The factorization algebra of observables of the free theory whose fields are as above assigns to an open set $U \subset S \times X$ the cochain complex
\[
\Obs \colon U \mapsto \cO(\cE(U)) = \Sym \left(\cE(U)^\vee \right) 
\]
equipped with the induced differential.

Some remarks are in order:
\begin{itemize}
\item If $V$ is a topological vector space then $\cO(V) = \Sym(V^\vee)$ denotes the algebra of polynomials on~$V$.
Here~$V^\vee$ is the topological dual.
\item The topological dual of $\cE(U)$ is $\cE(U)^\vee \simeq \overline{\cE}^!_c(U)$ where the bar denotes distributional sections, the subscript $c$ denotes compact support, and $!$ denotes the Serre dual. 
Explicitly, if $U = U' \times U'' \subset S \times X$ then 
\[
\overline{\cE}^!_c(U' \times U'') \simeq \overline{\Omega}^\bu(U') \otimes \overline{\Omega}^{n,\bu}(U'',V^*)[n+m] 
\]
where $\dim_\C (X) = n$ and $\dim_\R (S) = m$. 
\end{itemize}

Let's restrict to the case that $S \times X = \R^m \times \C^n$ and suppose that the bundle $V \to \C^n$ is translation invariant with fiber $V_0$ over $0 \in \R^m \times \C^n$.
We also assume that the operator $Q^{hol}$ is translation invariant. 

Given a vector bundle $E \to M$, the bundle of $\infty$-jets $J^\infty E \to M$ is a $\infty$-dimensional pro vector bundle whose fiber over a point $p \in M$ is 
If $M = \R^d$ and $E$ is translation invariant, then the bundle of $\infty$-jets can be identified with $E_0 \times \C[[x_i]]$ where  
 
The jet expansion at $0 \in \R^m \times \C^n$ determines a map of cochain complexes
\[
\cE(\C^n \times \R^m) \to V_0 [[x_i, \d x_i,z_j, \zbar_j, \d \zbar_j]] 
\]
The differential on the right hand side is $\d_{dR} + \dbar + Q^{hol} = \d x_i \del_{x_i} + \d \zbar_j \del_{\zbar_j} + Q^{hol}$ where $Q^{hol}$ is some holomorphic differential operator in the $z_j$ variables. 
Since all structure maps are given by holomorphic polydifferential operators, the canonical map 
\[
V_0 [[x_i, \d x_i,z_j, \zbar_j, \d \zbar_j]] \xto{\simeq} V_0 [[z_j]] 
\]
which sends $x_i, \d x_i,\zbar_j \d \zbar_j \mapsto 0$ is a quasi-isomorphism. 
The only remaining differential on the right hand side is~$Q^{hol}$. 
In summary, we see that the jet expansion at $0 \in \R^m \times \C^n$ determines a map of cochain complexes $\cE(\R^m \times \C^n) \to V_0[[z_j]]$. 

\begin{lem}
\label{lem:taylor}
Suppose that $\cE$ is the sheaf of cochain complexes representing the free topological-holomorphic theory on $S \times X = \R^m \times \C^n$ and consider the factorization algebra of observables~$\Obs = \cO (\cE)$. 
Then, the Taylor expansion map
\beqn\label{eqn:taylor}
\cE(\C^n \times \R^m) \to V_0[[z_0,\ldots,z_n]]
\eeqn
induces a quasi-isomorphism of commutative dg algebras
\[
\Obs(0) \simeq \cO \left( V_0[[z_1,\ldots,z_n]] \right) .
\]
%Notice that when $\cL$ is abelian with differential $\d_{dR} + \dbar + Q^{hol}$, then there is a quasi-isomorphism
%\[
%\Obs(0) \simeq \cO \left( V_0[[z_1,\ldots,z_n]][1] \right) 
%\]
%where the right hand side is equipped with the differential $Q^{hol}$. 
\end{lem}
\begin{proof}
Suppose that $D_\R \times D_\C \subset \R^m \times \C^n$ is a product of a real $m$-disk times a complex $n$-disk containing the origin.
The algebra of observables supported on $D_\R \times D_\C$ is quasi-isomorphic to 
\[
\cO\left( \cO^{hol}(D_\C) \otimes V_0 \right) .
\]

Observe that there is a canonical map on fields 
\[
\cO^{hol}(D_\C) \otimes V_0 \to V_0[[z_1,\ldots,z_n]]
\]
given by taking the power series expansion at $0 \in D_\R \times D_\C$. 
If an observables on $D_\R \times D_\C$ depends on only the value of the field and its derivatives at $0 \in D_\R \times D_\C$ then it automatically factors through this map. 
In particular, this means that there is a quasi-isomorphism of local operators with functions on $V_0[[z_1,\ldots,z_n]]$,
\[
\Obs(0) \simeq \cO\left(V_0 [[z_1,\ldots,z_n]]\right).
\] 
\end{proof}

Let's unpack this result explicitly. 
Using the $n$-dimensional residue, we can identify the topological dual of $V_0[[z_1,\ldots,z_n]]$ with the vector space
\beqn
\frac{\d z_1}{z_1} \cdots \frac{\d z_n}{z_n} V_0^* [z_0^{-1}, \ldots,z_n^{-1}] .
\eeqn
This is the space of linear local operators. 
If $\chi \colon V_0 \to \C$ is a dual vector in~$V_0^*$
then we obtain a linear local operator at $0 \in \R^m \times \C^n$ on the space of fields by the assignment
\[
\varphi \mapsto \del_{z_1}^{k_1} \cdots \del_{z_n}^{k_n} \<\chi,\varphi\> (0) 
\]
where $k_i \geq 0$. 
Under the quasi-isomorphism of the lemma above, this corresponds to the linear local operator 
\[
\frac{\d z_1}{z_1^{k_1+1}} \cdots \frac{\d z_n}{z_n^{k_n+1}} \chi .
\]

\parsec[s:interaction]

It is not hard to turn on interactions in the description above. 
An interacting theory in the BV formalism is described by a local $L_\infty$ algebra structure on $\cL = \cE[-1]$, where $\cE$ is the sheaf of fields.
For a topological-holomorphic theory the higher $L_\infty$ structure maps $[\cdot]_k$ of the local $L_\infty$ algebra are required to be given by holomorphic polydifferential operators and $[\cdot]_1 = \d_{dR} + \dbar + Q^{hol}$.  
For more details we refer to the definitions in \cite{GRWthf}.

In this situation, the factorization algebra of classical observables supported on an open set $U \subset S \times X$ is given by the Chevalley--Eilenberg cochains on the $L_\infty$ algebra $\cL(U)$. 
This defines a factorization algebra 
\[
\Obs \colon U \mapsto \clie^\bu(\cL(U)) .
\]
We will now give a concise presentation for the {\em local} operators in a topological-holomorphic theory. 

On $S \times X = \R^m \times \C^n$ we can also ask that all $L_\infty$ structure maps be translation invariant. 
If this is the case, one obtains the induced structure of an $L_\infty$ algebra on the (shift of the) jets of the fields supported at $0 \in \R^m \times \C^n$
\[
V_0 [[z_1,\ldots,z_n]] [-1] .
\]
The $[\cdot]_1$ operation is precisely $Q^{hol}$ as above.
The Taylor expansion map \eqref{eqn:taylor} is a map of $L_\infty$ algebras. 
Combining this with Lemma \ref{lem:taylor}, one gets a quasi-isomorphism of cochain complexes between the local operators of an interacting topological-holomorphic theory in terms of Lie algebra cohomology
\[
\Obs(0) \simeq \clie^\bu\left(V_0[[z_1,\ldots,z_n]][-1]\right) .
\]

%We recall the reader of the standard dictionary between the space of fields of a BV theory and the local $L_\infty$ algebra---
%if the local Lie algebra is $\cL$, then the space of fields is $\cL[1]$. 
%The Chevalley--Eilenberg complex of $\cL$ is then functions on the fields $\cO(\cL[1])$ equipped with the non-linear BRST operator.

\parsec[s:envelope]

There is another way that observables are presented in a degenerate version of the BV formalism.
Suppose that~$\cE$ is the sheaf of sections of some graded vector bundle~$E$ on a manifold~$M$.
We have seen that the observables~$\cO(\cE) = \Sym(\cE^*)$ has the structure of a factorization algebra---we now consider the $!$-dual factorization algebra.
That is, we consider the factorization algebra 
\[
U \subset M \mapsto \Sym \left(\cE_c(U) \right) 
\]
where $U \to \cE_c(U)$ is the cosheaf of compactly supported sections of the bundle~$E$.

%Suppose that $\cL$ is a local Lie algebra on a manifold $M$. 
%Then, one can consider the factorization algebra
%\[
%\cF = \clie_\bu(\cL_c)
%\]
%which assigns to an open set $U$ the cochain complex $\clie_\bu(\cL_c(U))$.
%This is the $!$-dual of the factorization algebra $\clie^\bu(\cL)$.
%For topological-holomorphic local Lie algebras there is still an algorithm for computing $\cF(p)$ for a point~$p \in M$.
%
%We will assume that $\cL$ is a translation invariant topological-holomorphic local Lie algebra whose underlying sheaf of cochain complexes is
%\[
%\cL = \Omega^\bu (\R^m) \hotimes \Omega^{0,\bu}(\C^n, L)
%\]
%Here $L$ is a translation invariant holomorphic vector bundle on $\C^n$ and the differential in the complex is $\d_{dR} + \dbar + Q^{hol}$ as above.

\begin{lem}
\label{lem:envelope}
Suppose that $\cE$ is the sheaf of fields of a free holomorphic theory as in~\eqref{eqn:cE} and consider the factorization algebra~$\cF = \Sym(\cE_c)$. 
Then, the algebra of classical local operators at~$0 \in \C^n$ of the factorization algebra~$\cF$ is quasi-isomorphic to 
\begin{align*}
\cF(0) & \simeq {\rm Sym} \left(\Omega^{n,hol}(\Hat{D}^n,V_0^*)^\vee [-n]\right) \\ & \cong \cO\left(\Omega^{n,hol}(\Hat{D}^n,V_0^*) [n] \right) 
\end{align*}
where the differential on the right hand side is~$Q^{hol}$.
\end{lem}

%\begin{lem}
%\label{lem:envelope}
%Suppose that $\cL$ is a topological-holomorphic local Lie algebra on $S \times X = \R^m \times \C^n$ and let $\cF$ be the factorization algebra $\clie_\bu(\cL_c)$.
%Moreover, assume that $Q^{hol}$ is an elliptic holomorphic differential operator. 
%Then, there is a spectral sequence converging to $H^\bu(\cF(0))$ whose first page is the $Q^{hol}$ cohomology of
%\[
%\cO \left(\d^n z L_0^*[[z_1,\ldots,z_n]] [n+m-1] \right) .
%\]
%\end{lem}
\begin{proof}
First, notice that as graded topological vector spaces one has an isomorphism for any open set $U \subset M$ 
\beqn\label{eqn:dist}
\left(\overline{\cE}^!(U)\right)^\vee \simeq \cE_c(U) 
\eeqn
%
%We use the spectral sequence induced by the filtration by the homogenous degree of a local operator.
%The first page is the cohomology of 
%\[
%\lim_{U \ni 0} \Sym(\cL_c(U)[1]) 
%\]
%with respect to the linear differential $\d_{dR} + \dbar + Q^{hol}$ which acts on $\cL_c(U)$ and extends to the symmetric algebra by the rule that it is a derivation.
This implies there is an isomorphism
\beqn\label{eqn:dist2}
\Sym(\cE_c(U)) \simeq \cO\left(\overline{\cE}^!(U)\right) 
\eeqn
for any open set $U$.
By assumption, the linear differential $[\cdot]_1$ is elliptic, in particular the embedding of smooth sections into distributional sections
\beqn\label{eqn:dist3}
\cE^!(U) \hookrightarrow \overline{\cE}^! (U)
\eeqn
is a quasi-isomorphism for any open set~$U$. 

We can assume that $U \subset \C^n$ is a Stein open set containing~$0 \in \C^n$.
%of the form $U' \times U'' \subset \R^m \times \C^n$ with $U' \subset \R^m$ contractible and $U''\subset \C^n$ Stein.
Then we have a sequence of quasi-isomorphisms
\begin{align*}
\overline{\cE}^! (U) & \simeq \cE^!(U) \\ & \simeq \Omega^{n,\bu}(U, V^*)[n].
\end{align*}
The result now follows from Lemma~\ref{lem:taylor}.

%Thus, the first page of this spectral sequence is isomorphic to the cohomology of the local operators $\Obs(0)$ of the free theory whose underlying cochain complex of fields is 
%\[
%\cE = \Omega^\bu(\R^m) \hotimes \Omega^{0,\bu}(\C^n , K_{\C^n} \otimes L^*[n+m-1]).
%\]
%In the notation of Equation \eqref{eqn:cE}, the holomorphic vector bundle $V$ is 
%\[
%K_{\C^n} \otimes L^*[n+m-1] .
%\]
%The result now follows from Lemma~\ref{lem:taylor} where we have used $\d^n z$ for basis for the line $K_{\C^n}|_0$. 
\end{proof}

\subsection{Local characters for topological-holomorphic theories}\label{s:localchar}

Suppose that $\cF$ is the factorization algebra of observables of a topological-holomorphic theory on $\R^m \times \C^n$. 
We will restrict our attention to cases where $\cF$, as a graded vector space, is of the form $\Sym(\cE^*)$ or $\Sym(\cE_c)$ where $\cE$ is of the form \eqref{eqn:cE}.

The local character $\chi_\cF ({\bf q})$ is, by definition, the graded character of algebra of local operators $\cF(0)$ with respect to some group of symmetries $H$, see \cite{SWchar}.
The particular group of symmetries depends on the theory, and we will present some examples momentarily. 

By assumption, as a graded algebra, the algebra of local operators $\cF(0)$ of a topological-holomorphic theory is of the form
\beqn
\cF(0) = \Sym (\lie{s})
\eeqn
where $\lie{s}$ is a graded topological vector space which we interpret as the linear local operators.

We will also assume that the group of symmetries $H$ acting on $\cF(0)$ arises from an action of $H$ on the linear local operators $\lie{s}$. 
Denote by $f_{\cF}({\bf q})$ the character of $\lie{s}$ with respect to this group action---this is the so-called `single particle' character. 
The full character of $\cF(0)$ is then given as the plethystic exponential of this single particle character
\beqn
\chi_{\cF}({\bf q}) = {\rm PExp}\left[f_{\cF}({\bf q}) \right] .
\eeqn

\subsection{Examples}

We present some simple examples. 

\begin{eg}
Suppose that $V$ is the trivial bundle on $\C^n$ and consider the theory whose fields are
\[
\cE = \Omega^\bu(\R^m) \otimes \Omega^{0,\bu}(\C^n) 
\]
where the differential is just $\d_{dR} + \dbar$. 
Then, the space of local operators is the symmetric algebra on the topological vector space which is linear dual to 
\[
\cO^{hol}(\Hat{D}^n) = \C[[z_1,\ldots,z_n]] .
\]
Via the $n$-dimensional residue one can identify the algebra of local operators with 
\[
\Sym\left(\frac{\d z_1}{z_1} \cdots \frac{\d z_n}{z_n}  \C[z_1^{-1}, \ldots , z_n^{-1}]\right) ,
\]
where $\lie{s} \simeq \frac{\d z_1}{z_1} \cdots \frac{\d z_n}{z_n}  \C[z_1^{-1}, \ldots , z_n^{-1}]$ is (equivalent to) the space of linear local operators. 

Consider the standard torus action $\C^\times \times \cdots \times \C^\times$ on $\C^n$. 
We would like to observe that the character of local operators with respect to this symmetry would be given by the plethystic exponential of the single particle index (the character of the space of linear local operators) which is immediate to compute:
\[
\frac{1}{(1-q_1)\cdots (1-q_n)} .
\]
However, the plethystic exponential cannot be applied to such an expression since as a power series in $q_1,\ldots,q_n$ there is a nonzero constant term.
This is related to the fact that there is an infinite number of operators for which the fugacities satisfy $q_1=\ldots=q_n=1$, so counting local operators in this way is ill-defined. 
One can remedy this by introducing a single extra variable fugacity $y$ and modify the single particle index to 
\[
\frac{y}{(1-q_1)\cdots (1-q_n)} .
\]
The plethystic exponential of such an expression returns the local character
\[
\chi(q_1,\ldots,q_n,y) = \prod_{k_1,\ldots,k_n \geq 0} \frac{1}{1-y q_1^{k_1}\cdots q_n^{k_n}}
\]
which now makes sense as a power series in the variables $y,q_1,\ldots,q_n$.
\end{eg}

Its instructive to see how local operators differ between $!$-dual factorization algebras.
Let us first point out a simple example. 
\begin{eg}
Consider the sheaf of cochain complexes
\[
\cE = \Omega^{0,\bu}\left(\C, K_{\C}^{\otimes r}\right),
\]
where $r \in \Z$ and the differential is~$\dbar$. 
Then, we can consider both the factorization algebra $\Obs = \cO(\cE)$ and its $!$-dual $\Obs^! = \Sym(\cE_c)$. 

The $\infty$-jets at $0 \in \C$ of $\cE$ is quasi-isomorphic to $\Gamma(\Hat{D}^n, K^{\otimes r}) = \d z^{\otimes r} \C[[z]]$. 
Thus the algebra of local operators $\Obs(0)$ is quasi-isomorphic to 
\[
\Obs(0) \simeq \cO \left(\Gamma(\Hat{D}, K^{\otimes r})\right) .
\]
In particular, the character of local operators $\Obs(0)$ is the plethystic exponential of
\[
\frac{q^{r}}{1-q} 
\]
where $q$ represents the fugacity for the standard~$\C^\times$ action on~$\C$.
Notice that when $r = 0$ we run into a similar problem as in the previous example. 
It is therefore convenient to introduce an extra fugacity $y$ which enters the single particle character as
\[
\frac{y q^{r}}{1-q}  .
\]

%Then, we have the factorization algebra which assigns to $U \subset \C$ the complex
%\begin{align*}
%\cF(U) & = \clie_\bu(\cL_c(U)) \\ & = \Sym\left(\Omega^{0,\bu}_c\left(U, K_U^{\otimes r}\right) [1] \right)
%\end{align*}
%where the differential is $\dbar$. 

%Serre duality induces an isomorphism
%\begin{align*}
%\Omega^{0,\bu}_c(\C, K_\C^{\otimes r}) \cong \left( \Gamma^{hol} (\C , K_\C \otimes K_\C^{-r})\right)^* \\
%= \left( \Gamma^{hol} (\C , K_\C^{1-r})\right)^* .
%\end{align*}

On the other hand, by Lemma \ref{lem:envelope} we see that the local operators associated to the $!$-dual $\Obs^!(0)$ is identified with the vector space
\[
\cO\left(\Gamma(\Hat{D}, K^{1-r})[1]\right) .
\]
In particular, the character of local operators $\Obs^!(0)$ is the plethystic exponential of
\[
-\frac{q^{1-r}}{1-q} 
\]
where $q$ represents the fugacity for the standard $\C^\times$ action on $\C$.
This time, when $r=1$ there is a problem with defining the plethystic exponential. 
To get an expression that makes sense for all $r$ we can again introduce a variable $y$ which enters the single particle character as
\[
- \frac{y q^{1-r}}{1-q} .
\]
\end{eg}

\subsection{Local characters for twisted superconformal theories}\label{s:localchar}

Suppose that $\cF$ is the factorization algebra of observables of a topological-holomorphic theory on $\R^m \times \C^n$. 
The local character $\chi_\cF ({\bf q})$ is, by definition, the graded character of algebra of local operators $\cF(0)$ with respect to some group of symmetries \cite{SWchar}.
The particular group of symmetries depends on the theory.
In this section we focus on local characters of factorization algebras that arise as twists of six-dimensional $\cN=(2,0)$ supersymmetric theories.

The (complexified) superconformal algebra in dimension six is $\lie{osp}(8|4)$. 
The holomorphic twist of this superconformal algebra is $\lie{osp}(6|2)$. 
We will consider the symmetry by the bosonic subalgebra
\beqn\label{eqn:cartan3}
\lie{sl}(3) \times \lie{sl}(2) \times \lie{gl}(1) \subset \lie{osp}(6|2)  .
\eeqn
The corresponding generators of the Cartan, as in \S \ref{sec:states}, were denoted $h_1,h_2,h,$ and $\Delta$ and the respective fugacities $t_1,t_2,r,q$.

We have described how this subalgebra embeds as fields in the twist of eleven-dimensional supergravity in \S \ref{s:ads7}. 
In particular, the holomorphic twist of any six-dimensional superconformal theory will have as a symmetry the subalgebra \eqref{eqn:cartan3}.
If the corresponding factorization algebra is $\cF$, and the local operators $\cF(0)$, the local character is then defined by the formal expression
\beqn
\chi_{\cF}(t_1,t_2,r,q) = {\rm Tr}_{\cF(0)} \left((-1)^F t_1^{h_1} t_2^{h_2} r^h q^\Delta\right) .
\eeqn
In the next section we will compute these characters in the case that the factorization algebra $\cF$ is $\clie_\bu(\cG_{N,c})$ where $N = 1,2,\ldots$.

We pointed out in \S \ref{sec:states} an alternative parametrization of the fugacities in terms of the parameters $y_1,y_2,y_3,y,q$ which satisfy the constraint $y_1 y_2 y_3 = 1$.
These parameters are related by $y_1=t_1^{-1}, y_2 = t_1 t_2^{-1}, y_3 = t_2$ and $y = q^{1/2} r$. 
We will also consider formulas for the local character in terms of these variables.
  
%Note that for local operators which are the symmetric algebra of some graded vector space we can compute, as usual, the character as the plethystic exponential of a the single particle local character. 
%That is, the character of linear local operators.

\subsection{A relationship to the superconformal index}
\label{sec:sucaindex}
There is a general relationship between the local character for the holomorphic twist of six-dimensional $\cN=(2,0)$ supersymmetric theories and the more well-studied superconformal index.

The superconformal index of a superconformal field theory is the Witten index of the theory in the radial quantization.
In our situation we look at the Hilbert space $\cH$ of the theory on $S^{5}$ and consider the index heuristically of the form
\beqn
\Tr_{\cH} (-1)^F  x_1^{G_1} \cdots x_n^{G_n} q^{\Delta}
\eeqn
where $\{G_i\}$ are a collection of charges that commute with the holomorphic supercharge $Q$ and its superconformal adjoint $S = Q^\dagger$. 
Here, $\Delta = [Q,S]$.
We choose three elements $G_1,G_2,G_3$ in such a way that they become the elements $h_1,h_2,h$ upon taking $Q$-cohomology (and so automatically commute with $Q$ and $S$). 
Thus we consider the following index
\beqn
\cI (t_1,t_2,r,q) \define \Tr_{\cH} (-1)^F t_1^{h_1} t_2^{h_2} r^h q^\Delta .
\eeqn
After tracing over $\cH$ one can identify the superconformal index with the partition function of the model on a space which is topologically equivalent to~$S^{5} \times S^1$.

The Witten index is protected under twisting---in our setup the index $\cI(t_1,t_2,r,q)$ can be computed in the minimal twist of the superconformal theory we start with.
The minimal twist of the Hilbert space $\cH^Q$ is exactly the space of holomorphic local operators at $0$ in $\C^3$, for details see~\cite{SWchar}. 
Thus, the index $\cI(t_1,t_2,r,q)$ agrees with the holomorphic character $\chi(t_1,t_2,r,q)$ defined above.

%\[
%(\C^3 - 0) / \sim  \; \simeq \; S^5 \times S^1 .
%\]
%The perspective of the holomorphic twist allows us to holomorphic theory agrees with the partition function on the product of spheres is basically goes by the process of `radial quantization'. 
%Consider the restriction of the theory to $\C^3 - 0 \subset \C^3$ and its dimensional reduction to quantum mechanics along
%\[
%|-| \colon \C^3 - 0 \to \RR_{>0} .
%\]
%The fiber of this map over a point is $S^5$. 
%By the nature of holomorphic QFT, we can extract from the OPE in the radial direction a canonical associative ($A_\infty$) algebra $\cA_{\lie{u}(1)} = \int_{\C^3 - 0} \Obs$ which is roughly the value of the theory on $S^5$. 
%There is a canonical boundary condition of the quantum mechanics theory at radius $r = 0$ given by the local operators $\Obs(0)$ at $0 \in \C^3$ which, in turn, has the structure of a $\cA$-module. 
%By standard arguments placing this quantum mechanics theory on circle $S^1$ results in the trace of the $\cA$-module $\Obs(0)$ 
%\[
%Z(S^{5} \times S^1) = {\rm Tr}_{\cA} (\Obs(0)) .
%\]
%From the trace on the right-hand side we can recover the character as defined above. 
%Indeed, the $E(3|6)$-module structure on local operators factors through a map 
%$E(3|6) \to \cA$ since we wrote down the explicit Hamiltonians above in \eqref{eq:ham1}, for instance. 

\subsection{Comparison to `states'}


%\subsection{Categorifying the index for free theories}
%
%In the case of both membranes and fivebranes we constructed a particular restriction of the local $L_\infty$ algebra $\cL_{sugra}$ to the respective worldvolume theories which we denoted by $\Bar{\pi}_* \cL_{sugra}$. 
%There are two important sub local $L_\infty$ algebras 
%\[
%\begin{tikzcd}
%& \Bar{\pi}_*\cL_{sugra} & \\
%\Bar{\pi}_*\cL_{sugra}^{(-1)} \ar[ur] & & \Bar{\pi}_*\cL_{sugra}^{(0)} \ar[ul] .
%\end{tikzcd}
%\]
%This diagram induces a diagram of factorization algebras
%\[
%\begin{tikzcd}
%& \left(\Obs_{sugra}|_Z\right)^! & \\
%\clie_\bu(\Bar{\pi}_*\cL_{sugra,c}^{(-1)}) \ar[ur] & & \clie_\bu(\Bar{\pi}_*\cL_{sugra,c}^{(0)}) \ar[ul].
%\end{tikzcd}
%\]

%\parsec[s:sugraops]
%
%By the usual methods of the BV formalism the action functional $S_{sugra}$ described above endows the parity shift of the fields $\cL_{sugra} = \Pi \cF_{sugra}$ with the structure of a holomorphic-topological local $\Z/2$ graded $L_\infty$ algebra. 
%
%On $\C^5 \times \R$ we can describe this super Lie algebra structure explicitly. 
%First, by the Dolbeault and de Rham Poincar\'e lemmas it is easy that the even part of the super Lie algebra $\cL(\C^5 \times \R)$ is equivalent to a one-dimensional central summand $\C$ plus the Lie algebra of divergence-free vector fields on $\C^5$:
%\[
%\Vect_0 (\C^5) = \{X \in \Vect(\C^5) \; | \; \div X = 0\} .
%\]
%The odd part of the super Lie algebra $\cL(\C^5 \times \R)$ is equivalent to the space of holomorphic one-forms on $\C^5$ modulo exact one-forms
%\[
%\Omega^{1,hol}(\C^5) / {\rm Im}(\del) 
%\]
%which is, of course, equivalent to the space of closed holomorphic two-forms $\Omega^{2,hol}_{cl}(\C^5)$. 
%
%\begin{thm}[\cite{RSW}[Theorem 2.1]]
%The Taylor expansion map determines a map of $\Z/2$ graded $L_\infty$ algebras
%\[
%j_\infty \colon \cL_{sugra}(\C^5 \times \R) \to L_{sugra} .
%\]
%Furthermore, $L_{sugra}$ is equivalent as a $\Z/2$ graded $L_\infty$ algebra to $\Hat{E(5|10)}$. 
%\end{thm} 
%
%As an immediate corollary of this result we obtain by Lemma \ref{lem:localops} the following.
%
%\begin{cor}
%\label{cor:sugraops}
%Let $\Obs_{sugra}$ be the factorization algebra on $\C^5 \times \R$ of classical observables of the minimal twist of eleven-dimensional supergravity.
%There is a quasi-isomorphism of commutative dg algebras
%\[
%\Obs_{sugra} (0) \simeq \clie^\bu \left( \Hat{E(5|10)} \right) .
%\]
%\end{cor}

%\end{document}
