\documentclass[11pt]{amsart}
%
%%\usepackage{../macros-master}
\usepackage{macros-fivebrane}
%
\begin{document}

\section{Local operators in twisted $M$ theory}

The notion of a factorization algebra captures both the local operators of a theory together with the non-local operators that on can define from the local ones via descent.
From the data of a factorization algebra, one can recover local operators by the following formal construction. 
Let $\Obs$ be the factorization algebra of observables of some theory defined on a smooth manifold $M$.
The space of local operators at point $p \in M$ is, in a precise sense, the limiting behavior of the factorization algebra evaluated on the system of open sets which contain the point~$p$. 

Generally this limit is difficult to compute, but for certain theories it is possible to give a concise expression which captures the essential features of the theory.
For example, in a holomorphic theory, the algebra of local operators is equivalent to the algebra generated by holomorphic derivatives of fields evaluated at a point.

In this section we recall the essentials of the theory of local operators for holomorphic-topological theories. 

\subsection{Local operators in topological-holomorphic theories}

A factorization algebra encodes the many ways to combine observables supported on arbitrary open sets. 
Local operators, on the other hand, exist just at a point in spacetime.
From the factorization algebra perspective one can recover local operators by looking at observables which are supported on \text{every} open set which contains the given point; mathematically this is computed by a limit. 

Precisely, in \cite[Definition 10.1.0.1]{CG2} the space of local operators of a factorization algebra $\cF$ at a point $p \in M$ is defined by the limit $\cF(p) = \lim_{U \ni p} \cF(U)$ which runs over open sets $U \subset M$ containing~$p$.

We will only consider local operators on affine space $\R^d$. 
In this case, we will have the additional property that the factorization algebras are translation invariant.
At the level of local operators this means that the translation map $\tau_{p \to p'}$ induces an isomorphism $\cF(p) \simeq \cF(p')$. 
Without loss of generality, we will consider expressions for local operators at $0 \in \R^d$.

For topological-holomorphic theories the local operators take a very familiar form.
As an algebra they are generated by (holomorphic) derivatives of the fields evaluated at the specified point. 
More precisely, the local operators depend only on the $\infty$-jets of the fields at a point.
In this section we carefully formulate this result and give some examples.

\parsec[s:free]

%Suppose that $V$ is a translation invariant holomorphic vector bundle on $\C^n$ equipped with a $\Z \times \Z/2$ bigrading. 
%Let $\cV$ denote its sheaf of holomorphic sections.
%The {\em space of fields} of a holomorphic field theory on $\C^n$ is the $\Z \times \Z/2$ graded complex of vector bundles
%\[
%\Omega^{0,\bu}(\C^n, V) \cong \Omega^{0,\bu}(\C^n) \otimes V_0 
%\]
%where $V_0$ is the fiber of $V$ at $0 \in \C^n$.
%Our grading conventions are so that $\d \zbar_i$ has bidegree $(1,0)$.
%
%As introduced in \cite{BWhol,LiVertex,CG2}, a {\em holomorphic field theory} is a holomorphic vector bundle $V$ as above equipped additionally with:
%\begin{itemize}
%\item The structure of a local (super) $L_\infty$ algebra on $V[-1]$ with structure maps given by holomorphic polydifferential operators
%\[
%[\cdot]_k \colon \cV[-1]^{\times k} \to \cV[1-k] .
%\]
%\end{itemize}
%A {\em free} holomorphic theory has $[\cdot]_k = 0$ for $k > 1$.
%
%\parsec
A topological-holomorphic theory exists on spacetimes of the form $S \times X$ where $S$ is a smooth manifold and $X$ is a complex manifold (possibly equipped with some auxiliary geometric structures). 
The typical space of fields of a holomorphic-topological theory in the BV formalism is
\beqn\label{eqn:cE}
\cE = \Omega^\bu (S) \hotimes \Omega^{0,\bu}(X, V) 
\eeqn
where $V$ is a graded holomorphic vector bundle on $X$.
The underlying free theory is described by a differential on the space of fields of the form
\[
\d_{dR} + \dbar + Q^{hol} .
\]
Here $\d_{dR}$ is the de Rham differential acting on $S$, $\dbar$ is the Dolbeault operator acting on $X$, and $Q^{hol} \colon V \to V[1]$ is a holomorphic differential operator of cohomological degree~$+1$.
This means that the free, linear equations of motion for a field $\varphi$ take the form
\[
\d_{dR} \varphi + \dbar \varphi + Q^{hol} \varphi = 0 .
\]
Taking into account linear gauge symmetries corresponds to cohomology---solutions to the equations of motion modulo the image of $\d_{dR} + \dbar + Q^{hol}$.

Notice that $\cE$ is a sheaf of cochain complexes---it makes sense to restrict the fields to any open set $U \subset S \times X$. 
The factorization algebra of observables of the free theory whose fields are as above assigns to an open set $U \subset S \times X$ the cochain complex
\[
\Obs \colon U \mapsto \cO(\cE(U)) = \Sym \left(\cE(U)^\vee \right) 
\]
equipped with the induced differential.

Some remarks are in order:
\begin{itemize}
\item If $V$ is a topological vector space then $\cO(V) = \Sym(V^\vee)$ denotes the algebra of polynomials on~$V$.
Here~$V^\vee$ is the topological dual.
\item The topological dual of $\cE(U)$ is $\cE(U)^\vee \simeq \overline{\cE}^!_c(U)$ where the bar denotes distributional sections, the subscript $c$ denotes compact support, and $!$ denotes the Serre dual. 
Explicitly, if $U = U' \times U'' \subset S \times X$ then 
\[
\overline{\cE}^!_c(U' \times U'') \simeq \overline{\Omega}^\bu(U') \otimes \overline{\Omega}^{n,\bu}(U'',V^*)[n+m] 
\]
where $\dim_\C (X) = n$ and $\dim_\R (S) = m$. 
\end{itemize}

Let's restrict to the case that $S \times X = \R^m \times \C^n$ and suppose that the bundle $V \to \C^n$ is translation invariant with fiber $V_0$ over $0 \in \R^m \times \C^n$.
We also assume that the operator $Q^{hol}$ is translation invariant. 

Given a vector bundle $E \to M$, the bundle of $\infty$-jets $J^\infty E \to M$ is a $\infty$-dimensional pro vector bundle whose fiber over a point $p \in M$ is 
If $M = \R^d$ and $E$ is translation invariant, then the bundle of $\infty$-jets can be identified with $E_0 \times \C[[x_i]]$ where  
 
The jet expansion at $0 \in \R^m \times \C^n$ determines a map of cochain complexes
\[
\cE(\C^n \times \R^m) \to V_0 [[x_i, \d x_i,z_j, \zbar_j, \d \zbar_j]] 
\]
The differential on the right hand side is $\d_{dR} + \dbar + Q^{hol} = \d x_i \del_{x_i} + \d \zbar_j \del_{\zbar_j} + Q^{hol}$ where $Q^{hol}$ is some holomorphic differential operator in the $z_j$ variables. 
Since all structure maps are given by holomorphic polydifferential operators, the canonical map 
\[
V_0 [[x_i, \d x_i,z_j, \zbar_j, \d \zbar_j]] \xto{\simeq} V_0 [[z_j]] 
\]
which sends $x_i, \d x_i,\zbar_j \d \zbar_j \mapsto 0$ is a quasi-isomorphism. 
The only remaining differential on the right hand side is~$Q^{hol}$. 
In summary, we see that the jet expansion at $0 \in \R^m \times \C^n$ determines a map of cochain complexes $\cE(\R^m \times \C^n) \to V_0[[z_j]]$. 

\begin{lem}
\label{lem:taylor}
Suppose that $\cE$ is the sheaf of cochain complexes representing the free topological-holomorphic theory on $S \times X = \R^m \times \C^n$ and consider the factorization algebra of observables~$\Obs = \cO (\cE)$. 
Then, the Taylor expansion map
\beqn\label{eqn:taylor}
\cE(\C^n \times \R^m) \to V_0[[z_0,\ldots,z_n]]
\eeqn
induces a quasi-isomorphism of commutative dg algebras
\[
\Obs(0) \simeq \cO \left( V_0[[z_1,\ldots,z_n]] \right) .
\]
%Notice that when $\cL$ is abelian with differential $\d_{dR} + \dbar + Q^{hol}$, then there is a quasi-isomorphism
%\[
%\Obs(0) \simeq \cO \left( V_0[[z_1,\ldots,z_n]][1] \right) 
%\]
%where the right hand side is equipped with the differential $Q^{hol}$. 
\end{lem}
\begin{proof}
Suppose that $D_\R \times D_\C \subset \R^m \times \C^n$ is a product of a real $m$-disk times a complex $n$-disk containing the origin.
The algebra of observables supported on $D_\R \times D_\C$ is quasi-isomorphic to 
\[
\cO\left( \cO^{hol}(D_\C) \otimes V_0 \right) .
\]

Observe that there is a canonical map on fields 
\[
\cO^{hol}(D_\C) \otimes V_0 \to V_0[[z_1,\ldots,z_n]]
\]
given by taking the power series expansion at $0 \in D_\R \times D_\C$. 
If an observables on $D_\R \times D_\C$ depends on only the value of the field and its derivatives at $0 \in D_\R \times D_\C$ then it automatically factors through this map. 
In particular, this means that there is a quasi-isomorphism of local operators with functions on $V_0[[z_1,\ldots,z_n]]$,
\[
\Obs(0) \simeq \cO\left(V_0 [[z_1,\ldots,z_n]]\right).
\] 
\end{proof}

Let's unpack this result explicitly. 
Using the $n$-dimensional residue, we can identify the topological dual of $V_0[[z_1,\ldots,z_n]]$ with the vector space
\beqn
\frac{\d z_1}{z_1} \cdots \frac{\d z_n}{z_n} V_0^* [z_0^{-1}, \ldots,z_n^{-1}] .
\eeqn
This is the space of linear local operators. 
If $\chi \colon V_0 \to \C$ is a dual vector in~$V_0^*$
then we obtain a linear local operator at $0 \in \R^m \times \C^n$ on the space of fields by the assignment
\[
\varphi \mapsto \del_{z_1}^{k_1} \cdots \del_{z_n}^{k_n} \<\chi,\varphi\> (0) 
\]
where $k_i \geq 0$. 
Under the quasi-isomorphism of the lemma above, this corresponds to the linear local operator 
\[
\frac{\d z_1}{z_1^{k_1+1}} \cdots \frac{\d z_n}{z_n^{k_n+1}} \chi .
\]

\parsec[s:interaction]

It is not hard to turn on interactions in the description above. 
An interacting theory in the BV formalism is described by a local $L_\infty$ algebra structure on $\cL = \cE[-1]$, where $\cE$ is the sheaf of fields.
For a topological-holomorphic theory the higher $L_\infty$ structure maps $[\cdot]_k$ of the local $L_\infty$ algebra are required to be given by holomorphic polydifferential operators and $[\cdot]_1 = \d_{dR} + \dbar + Q^{hol}$.  
For more details we refer to the definitions in \cite{GRWthf}.

In this situation, the factorization algebra of classical observables supported on an open set $U \subset S \times X$ is given by the Chevalley--Eilenberg cochains on the $L_\infty$ algebra $\cL(U)$. 
This defines a factorization algebra 
\[
\Obs \colon U \mapsto \clie^\bu(\cL(U)) .
\]
We will now give a concise presentation for the {\em local} operators in a topological-holomorphic theory. 

On $S \times X = \R^m \times \C^n$ we can also ask that all $L_\infty$ structure maps be translation invariant. 
If this is the case, one obtains the induced structure of an $L_\infty$ algebra on the (shift of the) jets of the fields supported at $0 \in \R^m \times \C^n$
\[
V_0 [[z_1,\ldots,z_n]] [-1] .
\]
The $[\cdot]_1$ operation is precisely $Q^{hol}$ as above.
The Taylor expansion map \eqref{eqn:taylor} is a map of $L_\infty$ algebras. 
Combining this with Lemma \ref{lem:taylor}, one gets a quasi-isomorphism of cochain complexes between the local operators of an interacting topological-holomorphic theory in terms of Lie algebra cohomology
\[
\Obs(0) \simeq \clie^\bu\left(V_0[[z_1,\ldots,z_n]][-1]\right) .
\]

%We recall the reader of the standard dictionary between the space of fields of a BV theory and the local $L_\infty$ algebra---
%if the local Lie algebra is $\cL$, then the space of fields is $\cL[1]$. 
%The Chevalley--Eilenberg complex of $\cL$ is then functions on the fields $\cO(\cL[1])$ equipped with the non-linear BRST operator.

\parsec[s:envelope]

There is another way that observables are presented in a degenerate version of the BV formalism.
Suppose that~$\cE$ is the sheaf of sections of some graded vector bundle~$E$ on a manifold~$M$.
We have seen that the observables~$\cO(\cE) = \Sym(\cE^\vee)$ has the structure of a factorization algebra---we now consider the $!$-dual factorization algebra.
That is, we consider the factorization algebra 
\[
U \subset M \mapsto \Sym \left(\cE_c(U) \right) 
\]
where $U \to \cE_c(U)$ is the cosheaf of compactly supported sections of the bundle~$E$.

%Suppose that $\cL$ is a local Lie algebra on a manifold $M$. 
%Then, one can consider the factorization algebra
%\[
%\cF = \clie_\bu(\cL_c)
%\]
%which assigns to an open set $U$ the cochain complex $\clie_\bu(\cL_c(U))$.
%This is the $!$-dual of the factorization algebra $\clie^\bu(\cL)$.
%For topological-holomorphic local Lie algebras there is still an algorithm for computing $\cF(p)$ for a point~$p \in M$.
%
%We will assume that $\cL$ is a translation invariant topological-holomorphic local Lie algebra whose underlying sheaf of cochain complexes is
%\[
%\cL = \Omega^\bu (\R^m) \hotimes \Omega^{0,\bu}(\C^n, L)
%\]
%Here $L$ is a translation invariant holomorphic vector bundle on $\C^n$ and the differential in the complex is $\d_{dR} + \dbar + Q^{hol}$ as above.

\begin{lem}
\label{lem:envelope}
Suppose that $\cE$ is the sheaf of fields of a free holomorphic theory as in~\eqref{eqn:cE} and consider the factorization algebra~$\cF = \Sym(\cE_c)$. 
Then, the algebra of classical local operators at~$0 \in \C^n$ of the factorization algebra~$\cF$ is quasi-isomorphic to 
\begin{align*}
\cF(0) & \simeq {\rm Sym} \left(\Omega^{n,hol}(\Hat{D}^n,V_0^*)^\vee [-n]\right) \\ & \cong \cO\left(\Omega^{n,hol}(\Hat{D}^n,V_0^*) [n] \right) 
\end{align*}
where the differential on the right hand side is~$Q^{hol}$.
\end{lem}

%\begin{lem}
%\label{lem:envelope}
%Suppose that $\cL$ is a topological-holomorphic local Lie algebra on $S \times X = \R^m \times \C^n$ and let $\cF$ be the factorization algebra $\clie_\bu(\cL_c)$.
%Moreover, assume that $Q^{hol}$ is an elliptic holomorphic differential operator. 
%Then, there is a spectral sequence converging to $H^\bu(\cF(0))$ whose first page is the $Q^{hol}$ cohomology of
%\[
%\cO \left(\d^n z L_0^*[[z_1,\ldots,z_n]] [n+m-1] \right) .
%\]
%\end{lem}
\begin{proof}
First, notice that as graded topological vector spaces one has an isomorphism for any open set $U \subset M$ 
\beqn\label{eqn:dist}
\left(\overline{\cE}^!(U)\right)^\vee \simeq \cE_c(U) 
\eeqn
%
%We use the spectral sequence induced by the filtration by the homogenous degree of a local operator.
%The first page is the cohomology of 
%\[
%\lim_{U \ni 0} \Sym(\cL_c(U)[1]) 
%\]
%with respect to the linear differential $\d_{dR} + \dbar + Q^{hol}$ which acts on $\cL_c(U)$ and extends to the symmetric algebra by the rule that it is a derivation.
This implies there is an isomorphism
\beqn\label{eqn:dist2}
\Sym(\cE_c(U)) \simeq \cO\left(\overline{\cE}^!(U)\right) 
\eeqn
for any open set $U$.
By assumption, the linear differential $[\cdot]_1$ is elliptic, in particular the embedding of smooth sections into distributional sections
\beqn\label{eqn:dist3}
\cE^!(U) \hookrightarrow \overline{\cE}^! (U)
\eeqn
is a quasi-isomorphism for any open set~$U$. 

We can assume that $U \subset \C^n$ is a Stein open set containing~$0 \in \C^n$.
%of the form $U' \times U'' \subset \R^m \times \C^n$ with $U' \subset \R^m$ contractible and $U''\subset \C^n$ Stein.
Then we have a sequence of quasi-isomorphisms
\begin{align*}
\overline{\cE}^! (U) & \simeq \cE^!(U) \\ & \simeq \Omega^{n,\bu}(U, V^*)[n].
\end{align*}
The result now follows from Lemma~\ref{lem:taylor}.

%Thus, the first page of this spectral sequence is isomorphic to the cohomology of the local operators $\Obs(0)$ of the free theory whose underlying cochain complex of fields is 
%\[
%\cE = \Omega^\bu(\R^m) \hotimes \Omega^{0,\bu}(\C^n , K_{\C^n} \otimes L^*[n+m-1]).
%\]
%In the notation of Equation \eqref{eqn:cE}, the holomorphic vector bundle $V$ is 
%\[
%K_{\C^n} \otimes L^*[n+m-1] .
%\]
%The result now follows from Lemma~\ref{lem:taylor} where we have used $\d^n z$ for basis for the line $K_{\C^n}|_0$. 
\end{proof}

\parsec[s:examples]

We present some simple examples. 

\begin{eg}
Suppose that $V$ is the trivial bundle on $\C^n$ and consider the theory whose fields are
\[
\cE = \Omega^\bu(\R^m) \otimes \Omega^{0,\bu}(\C^n) 
\]
where the differential is just $\d_{dR} + \dbar$. 
Then, the space of local operators is the symmetric algebra on the topological vector space which is linear dual to 
\[
\cO^{hol}(\Hat{D}^n) = \C[[z_1,\ldots,z_n]] .
\]
Via the $n$-dimensional residue one can identify the algebra of local operators with 
\[
\Sym\left(\frac{\d z_1}{z_1} \cdots \frac{\d z_n}{z_n}  \C[z_1^{-1}, \ldots , z_n^{-1}]\right) .
\]

Consider the standard torus action $\C^\times \times \cdots \C^\times$ on $\C^n$. 
We would like to observe that the equivariant character of local operators with respect to this symmetry is given by the plethystic exponential of the single particle index which is immediate to compute:
\[
\frac{1}{(1-q_1)\cdots (1-q_n)} .
\]
However, the plethystic exponential cannot be applied to such an expression since as a power series in $q_1,\ldots,q_n$ there is a nonzero constant term.
This is related to the fact that there is an infinite number of operators for which the fugacities satisfy $q_1=\ldots=q_n=1$. 
One can remedy this by introducing a single extra variable fugacity $y$ and modify the single particle index to 
\[
\frac{y}{(1-q_1)\cdots (1-q_n)} .
\]
The plethystic exponential of such an expression is
\[
\prod_{k_1,\ldots,k_n \geq 0} \frac{1}{1-y q_1^{k_1}\cdots q_n^{k_n}}
\]
which now makes sense as a power series in the variables $y,q_1,\ldots,q_n$.
\end{eg}

Its instructive to see how local operators differ between $!$-dual factorization algebras.
Let us first point out a simple example. 
\begin{eg}
Consider the sheaf of cochain complexes
\[
\cE = \Omega^{0,\bu}\left(\C, K_{\C}^{\otimes r}\right),
\]
where $r \in \Z$ and the differential is~$\dbar$. 
Then, we can consider both the factorization algebra $\Obs = \cO(\cE)$ and its $!$-dual $\Obs^! = \Sym(\cE_c)$. 

The $\infty$-jets at $0 \in \C$ of $\cE$ is quasi-isomorphic to $\Gamma(\Hat{D}^n, K^{\otimes r}) = \d z^{\otimes r} \C[[z]]$. 
Thus the algebra of local operators $\Obs(0)$ is quasi-isomorphic to 
\[
\Obs(0) \simeq \cO \left(\Gamma(\Hat{D}, K^{\otimes r})\right) .
\]
In particular, the character of local operators $\Obs(0)$ is the plethystic exponential of
\[
\frac{q^{r}}{1-q} 
\]
where $q$ represents the fugacity for the standard~$\C^\times$ action on~$\C$.
Notice that when $r = 0$ we run into a similar problem as in the previous example. 
It is therefore convenient to introduce an extra fugacity $y$ which enters the single particle character as
\[
\frac{y q^{r}}{1-q}  .
\]

%Then, we have the factorization algebra which assigns to $U \subset \C$ the complex
%\begin{align*}
%\cF(U) & = \clie_\bu(\cL_c(U)) \\ & = \Sym\left(\Omega^{0,\bu}_c\left(U, K_U^{\otimes r}\right) [1] \right)
%\end{align*}
%where the differential is $\dbar$. 

%Serre duality induces an isomorphism
%\begin{align*}
%\Omega^{0,\bu}_c(\C, K_\C^{\otimes r}) \cong \left( \Gamma^{hol} (\C , K_\C \otimes K_\C^{-r})\right)^* \\
%= \left( \Gamma^{hol} (\C , K_\C^{1-r})\right)^* .
%\end{align*}

On the other hand, by Lemma \ref{lem:envelope} we see that the local operators associated to the $!$-dual $\Obs^!(0)$ is identified with the vector space
\[
\cO\left(\Gamma(\Hat{D}, K^{1-r})[1]\right) .
\]
In particular, the character of local operators $\Obs^!(0)$ is the plethystic exponential of
\[
-\frac{q^{1-r}}{1-q} 
\]
where $q$ represents the fugacity for the standard $\C^\times$ action on $\C$.
This time, when $r=1$ there is a problem with defining the plethystic exponential. 
To get an expression that makes sense for all $r$ we can again introduce a variable $y$ which enters the single particle character as
\[
- \frac{y q^{1-r}}{1-q} .
\]
\end{eg}

\subsection{A relationship to the supersymmetric index}

\subsection{Comparison to `states'}

\subsection{Operators on a single fivebrane}

We recalled the description of the holomorphic twist of the theory on a single fivebrane in~\S\ref{s:single}. 
This theory exists on any complex three-fold~$Z$ and hence we get a factorization algebra of classical observables~$\Obs^{6d}_{\lie{u}(1)}$ on $Z$.
We will specialize to~$Z=\C^3$ to characterize local operators~$\Obs^{6d}_{\lie{u}(1)}(0)$ supported at~$0 \in \C^3$. 

The theory on a single fivebrane is free and the underlying cochain complex of fields $\cE_{\lie{u}(1)}$ with linear BRST differential is
\beqn
\begin{tikzcd}
\ul{-1} & \ul{0} \\
\Omega^{2,\bu}(Z) \ar[r,"\del"] & \Omega^{3,\bu}(Z) \\
\Pi \Omega^{0,\bu}(Z, K_{Z}^{1/2} \otimes \C^2) . 
\end{tikzcd} 
\eeqn
Here we recall the $\Z \times \Z/2$ bigrading where the differential has bidegree $(1,0)$. 

Before moving on to characterizing local operators we take a brief moment to comment on the holographic appearance of the theory on a single fivebrane. 

\parsec[s:fiverelatesingle]

In \eqref{eqn:lim2} we presented the universal factorization algebra of classical observables on the fivebrane 
\[
(\Bar{\pi}_* \Obs_{sugra})^! = \clie_\bu(\cG_{Z,c}) 
\]
as a limit.
The first nonzero term in this limit diagram is 
\[
\clie_\bu(\cG_{1,c})
\]
where $\cG_1$ is the abelian local Lie algebra $\cG_{1} = \cG / \cG^{(\geq 0)} \cong \cG^{(-1)}$---this is just the weight $(-1)$ piece of the decomposition of $\cG$. 

\begin{prop}
There is a quasi-isomorphism of factorization algebras valued in $\Z/2$ graded commutative dg algebras on the three-fold~$Z$
\[
\clie_\bu(\cG_{1,c}) \xto{\simeq} \Obs_{\lie{u}(1)} .
\]
\end{prop}

\begin{proof}
In weight $(-1)$ the abelian local Lie algebra takes the form
\[
\cG_Z^{(-1)} \simeq \Omega^{0,\bu}(Z,\cV^{(-1)}) 
\]
where $\cV^{(-1)}$ was defined in \ref{s:weight-1}.
As a sheaf of $\Z/2$ gradedcochain complexes the factorization algebra $\clie_\bu(\cG_Z^{(-1)})$ assigns to an open set $U\subset Z$ the graded symmetric algebra on the complex
\beqn\label{eqn:weight-1a}
\begin{tikzcd}
\ul{odd} & \ul{even}\\
\Omega_c^{0,\bu}(U, K_Z^{1/2} \otimes \C^2) & \\
\Omega_c^{0,\bu}(U) \ar[r,"\del"] & \Omega_c^{1,\bu}(U) .
\end{tikzcd}
\eeqn
On the other hand, as a $\Z/2$ graded cochain complex, the factorization algebra of observables of the theory on a single fivebrane is of the form 
\[
\cO(\cE_{\lie{u}(1)}(U)) = \Sym(\Pi \overline{\cE}_{\lie{u}(1),c}^!(U)) .
\]
It is immediate to see that as a $\Z/2$ graded cochain complex $\cE_c^!(U)$ is exactly \eqref{eqn:weight-1a}.
The result then follows by applying ellipticity.
\end{proof}

We remark that the Chevalley--Eilenberg complex of an $L_\infty$ algebra $\clie_\bu(\lie{g})$ does not have the structure of a commutative dg algebra.
However, when $\lie{g}$ is abelian (so, a cochain complex) we can identify this complex with the symmetric algebra on the cochain complex $\fg^*[-1]$.

\parsec[]

We move on to characterizing the character of the holomorphic twist of the theory on a single fivebrane.
By the previous result we can compute this character either from a first principles description of the theory, or holographically by focusing on the weight $(-1)$ part of the decomposition of $\Bar{\pi}_* \Obs_{sugra}$.

\begin{lem}
\label{lem:single}
The $\Z \times \Z/2$ graded algebra of local operators $\Obs^{6d}_{\lie{u}(1)}(0)$ of the holomorphic twist of the worldvolume theory of a single fivebrane is quasi-isomorphic to the graded symmetric algebra on the linear dual of the topological vector space
\beqn\label{eqn:localfree}
V_0[[z_1,z_2,z_3]] \simeq \Omega^{2}_{cl} (\Hat{D}^3)[1] \oplus \Pi \Omega^0(\Hat{D}^3, K^{1/2}) \otimes \C^2 [1].
\eeqn
\end{lem}

\begin{proof}
The jet expansion at $0 \in \C^3$ determines a map from the sections of the abelian holomorphic-topological local Lie algebra on $\C^3$ to the cochain complex
\beqn
\begin{tikzcd}
\ul{-1} & \ul{0} \\
\Omega^{2}(\Hat{D}^3) \ar[r,"\del"] & \Omega^{3}(\Hat{D}^3) \\
\Pi \Omega^0(\Hat{D}^3, K^{1/2} \otimes \C^2) . 
\end{tikzcd} 
\eeqn
On the formal disk all closed two-forms are automatically exact, which implies the lemma.
\end{proof}

\parsec

We move on to the straightforward computation of the character of local operators on a single fivebrane which we will present as the plethystic exponential of the character of linear local operators
\beqn
\chi^{6d}_{\lie{u}(1)} (t_1,t_2,q,r) = {\rm PExp} \big[f^{6d}_{\lie{u}(1)}(t_1,t_2,q,r) \big] .
\eeqn
According to the weights listed above and using the description of local operators in Lemma \ref{lem:single} we have the following contributions to the single particle character~$f_{single}$.

\begin{itemize}
\item Single particle operators on the odd copy of holomorphic two-forms $\Pi \Omega^{2,hol}$ contribute
\[
- q^2 \frac{t_1  + t_1^{-1} t_2  + t_2^{-1} }{(1-t_1^{-1}q) (1-t_1 t_2^{-1} q) (1-t_2 q)} 
\]
\item Single particle operators on the even copy of holomorphic three-forms $\Omega^{3,hol}$ contribute
\[
q^3 \frac{1}{(1-t_1^{-1}q) (1-t_1 t_2^{-1} q) (1-t_2 q)} 
\]
\item Single particle operators on $K^{1/2} \otimes \C^2$ contribute
\[
q^{3/2}\frac{(r + r^{-1})}{(1-t_1^{-1}q) (1-t_1 t_2^{-1} q) (1-t_2 q)}
\]
\end{itemize}

Putting this all together we obtain the following.

\begin{prop}
The character $\chi_{u(1)}(q,t_1,t_2,r)$ of local operators of the holomorphic twist of the theory on a single fivebrane is given by the plethystic exponential of the single particle index
\[
f^{6d}_{u(1)} (q,t_1,t_2,r) = \frac{(r + r^{-1})q^{3/2} - (t_1 + t_1^{-1} t_2 + t_2^{-1} )q^2 + q^3}{(1-t_1^{-1}q) (1-t_1 t_2^{-1} q) (1-t_2 q)} .
\]
\end{prop}

\parsec

There are various degenerations, or specializations, of this character which are interesting to consider.
These specializations involve restricting the character above to a subalgebra of the full Cartan that we considered above.

One degeneration of this character involves specializing $t_1=t_2=1$ which results in the $U(1) \times SU(2)$ character:
\beqn
f_{u(1)}(t,r) = \frac{(r+r^{-1})q^{3/2} - 3 q^{2} + q^3}{(1-q)^3} .
\eeqn
%They compute the absolute (non-super) character of the module $I(0,0;1;-1)$ where they additionally specialize $t_1=t_2=r=1$. 
%In a similar method to the one used in \cite{KR1}, one can compute the specialized (super) character of $I(0,0;1;-1)$ to find
%\[
%\chi_{u(1)} (q,t_1=t_2=r=1) = \frac{2 q^{3/2} - 3 q^2 + q^3}{(1-q)^3} .
%\]
In \S \ref{s:kr} we will compare this to the character of a certain irreducible module for the exceptional super Lie algebra $E(3|6)$ considered in \cite{KR2}.

\parsec

Another specialization is related to a particular deformations by elements in the (twisted) superconformal algebra.
Recall that after performing the holomorphic twist of six-dimensional $\cN=(2,0)$ supersymmetry, the residual superconformal algebra is~$\lie{osp}(6|2)$. 
We have recalled in \ref{s:global1} how the bosonic part of this algebra is represented by fields of the eleven-dimensional theory. 

There are two types of odd elements of~$\lie{osp}(6|2)$ that also have a natural interpretation in the eleven-dimensional theory.
The odd part of~$\lie{osp}(6|2)$ can be identified with 
\[
\C^3 \otimes \C^2 \oplus \Sym^2(\C^3) \otimes \C^2 
\]
where $\C^3, \C^2$ are the fundamental $SU(3)$ and $SU(2)$ representations. 

\begin{itemize}
\item The space $\C^3 \otimes \C^2$ is spanned by twisted supergravity fields of the form $\gamma = w_a \d z_i$. 
\item The deformation $(z_2 \d z_3 - z_3 \d z_2) \otimes r_-$. 
\end{itemize}

Both deformations break the global Cartan subgroup down to $U(1) \times U(1)$ according to the specializations
\beqn\label{eqn:special1}
q = r^2 , \quad t_2 = 1 .
\eeqn
As one can easily check, this specialization yields the following single particle index
\[
f_{tensor}(t_1, 1, q, q^{1/2}) = \frac{q}{1-q} 
\]
which recovers the single particle index of a single chiral boson on the Riemann surface $\Sigma = \C_{z_1}$. 
Notice that although the Cartan subalgebra generated by the vector field $z_1 \del_{z_1} - z_2 \del_{z_2}$ is unbroken by this deformation, the dependence on its fugacity $t_1$ completely drops out of the expression.

%\subsection{Categorifying the index for free theories}
%
%In the case of both membranes and fivebranes we constructed a particular restriction of the local $L_\infty$ algebra $\cL_{sugra}$ to the respective worldvolume theories which we denoted by $\Bar{\pi}_* \cL_{sugra}$. 
%There are two important sub local $L_\infty$ algebras 
%\[
%\begin{tikzcd}
%& \Bar{\pi}_*\cL_{sugra} & \\
%\Bar{\pi}_*\cL_{sugra}^{(-1)} \ar[ur] & & \Bar{\pi}_*\cL_{sugra}^{(0)} \ar[ul] .
%\end{tikzcd}
%\]
%This diagram induces a diagram of factorization algebras
%\[
%\begin{tikzcd}
%& \left(\Obs_{sugra}|_Z\right)^! & \\
%\clie_\bu(\Bar{\pi}_*\cL_{sugra,c}^{(-1)}) \ar[ur] & & \clie_\bu(\Bar{\pi}_*\cL_{sugra,c}^{(0)}) \ar[ul].
%\end{tikzcd}
%\]

%\parsec[s:sugraops]
%
%By the usual methods of the BV formalism the action functional $S_{sugra}$ described above endows the parity shift of the fields $\cL_{sugra} = \Pi \cF_{sugra}$ with the structure of a holomorphic-topological local $\Z/2$ graded $L_\infty$ algebra. 
%
%On $\C^5 \times \R$ we can describe this super Lie algebra structure explicitly. 
%First, by the Dolbeault and de Rham Poincar\'e lemmas it is easy that the even part of the super Lie algebra $\cL(\C^5 \times \R)$ is equivalent to a one-dimensional central summand $\C$ plus the Lie algebra of divergence-free vector fields on $\C^5$:
%\[
%\Vect_0 (\C^5) = \{X \in \Vect(\C^5) \; | \; \div X = 0\} .
%\]
%The odd part of the super Lie algebra $\cL(\C^5 \times \R)$ is equivalent to the space of holomorphic one-forms on $\C^5$ modulo exact one-forms
%\[
%\Omega^{1,hol}(\C^5) / {\rm Im}(\del) 
%\]
%which is, of course, equivalent to the space of closed holomorphic two-forms $\Omega^{2,hol}_{cl}(\C^5)$. 
%
%\begin{thm}[\cite{RSW}[Theorem 2.1]]
%The Taylor expansion map determines a map of $\Z/2$ graded $L_\infty$ algebras
%\[
%j_\infty \colon \cL_{sugra}(\C^5 \times \R) \to L_{sugra} .
%\]
%Furthermore, $L_{sugra}$ is equivalent as a $\Z/2$ graded $L_\infty$ algebra to $\Hat{E(5|10)}$. 
%\end{thm} 
%
%As an immediate corollary of this result we obtain by Lemma \ref{lem:localops} the following.
%
%\begin{cor}
%\label{cor:sugraops}
%Let $\Obs_{sugra}$ be the factorization algebra on $\C^5 \times \R$ of classical observables of the minimal twist of eleven-dimensional supergravity.
%There is a quasi-isomorphism of commutative dg algebras
%\[
%\Obs_{sugra} (0) \simeq \clie^\bu \left( \Hat{E(5|10)} \right) .
%\]
%\end{cor}

\subsection{A conjectural description of a stack of two fivebranes}

In \S\ref{fact:summary} we saw that the decomposition of the local $L_\infty$ algebra $\cG = \cG_Z$ on $Z$ induces a tower of factorization algebras 
\[
\clie_{\bu}(\cG_{1,c}) \leftarrow \clie_{\bu}(\til \cG_{2,c}) \leftarrow \cdots .
\]
We now turn to the factorization algebra $\clie_{\bu}(\til \cG_{2,c})$.

Recall that $\til \cG_{2}$ is the local $L_\infty$ algebra on $Z$ defined as $\til \cG_{2} = \cG / \cG^{\geq 1}$. 
Since $\cG$ is concentrated in weights $\geq -1$ we see that $\til \cG_{2}$ is of the form
\[
\til \cG_2 = \cG_2 \ltimes \cG_1 
\]
where $\cG_1 = \cG^{(-1)}$ is the weight $(-1)$ piece and $\cG_2 = \cG^{\geq 0} / \cG^{\geq 1} = \cG^{(0)}$.  
We focus mostly on the factorization algebra $\clie_\bu(\cG_{2,c})$.

We have already characterized the local dg Lie algebra $\cG_{2}$ as the weight zero part of $\cG$ on on any threefold $Z$ in \S\ref{s:weight0}. 
We deduce that $\cG_2$ can be given a $\Z \times \Z/2$ grading where, as usual, we refer to $\Z$ as the cohomological grading and $\Z/2$ as the parity. 
With this lift to a $\Z \times \Z/2$ grading, the even part of $\cG_2$ is
\[
\Omega^{0,\bu}(Z, \T_Z) \oplus \Omega^{0,\bu}(Z) \otimes \lie{sl}(2) 
\]
with its natural cohomological grading by Dolbeault form type. 
The odd part of $\cG_2$ is
\[
\Omega^{1,\bu}(Z, K_Z^{-1/2}) \otimes \C^2 .
\]
The differential is $\dbar$ and the Lie bracket has been described in \S\ref{s:weight0}.

On $Z = \C^3$ this local dg Lie algebra is related to the exceptional simple super Lie algebra $E(3|6)$ classified by Kac \cite{s:KacClass}. 
Indeed, one can show (see the forthcoming work \cite{SW6d}) that the fiber of the $\infty$-jet bundle of $\cG_2$ at $0 \in \C^3$ is quasi-isomorphic to $E(3|6)$. 

\parsec

We continue by computing the character of local operators associated to the factorization algebra $\clie_\bu(\cG_{2,c})$. 

\begin{itemize}
\item Single particle operators coming from the copy of holomorphic vector fields $\Vect^{hol}(\C^3)$ contribute
\[
q^3 \frac{t_1^{-1} q + t_1 t_2^{-1} q + t_2 q }{(1-t_1^{-1}q) (1-t_1 t_2^{-1} q) (1-t_2 q)} 
\]
\item Single particle operators coming from $\lie{sl}(2)$-valued holomorphic functions $\lie{sl}(2) \otimes \cO^{hol}(\C^3)$ contribute
\[
q^3\frac{r^2 + r^{-2} + 1}{(1-t_1^{-1}q) (1-t_1 t_2^{-1} q) (1-t_2 q)} 
\]
\item Single particle operators coming from the odd piece of $E(3|6)$ which is $\Omega^{1,hol} \otimes K^{-1/2} \otimes \C^2$ contribute
\[
q^{3}\frac{(q^{1/2} r + q^{1/2} r^{-1})(t_1^{-1} + t_1t_2^{-1} + t_2)}{(1-t_1^{-1}q) (1-t_1 t_2^{-1} q) (1-t_2 q)}
\]
\end{itemize}

\begin{conj}
The character of local operators of the holomorphic twist of the theory on a stack of two fivebranes is given by the following plethystic exponential
\[
\chi_{\lie{sl}(2)}^{6d} (q,t_1,t_2,r) = {\rm PExp} \left[f_{\lie{sl}(2)}^{6d} (t_1,t_2,q,r) \right] .
\]
where the single particle index is
\[
f_{\lie{sl}(2)}^{6d} (q, t_1,t_2,r) = \frac{q^4(t_1^{-1} + t_1 t_2^{-1}  + t_2) + q^3 (r^2 + r^{-2} + 1) - q^{7/2} (r + r^{-1})(t_1^{-1} + t_1t_2^{-1} + t_2)}{(1-t_1^{-1}q) (1-t_1 t_2^{-1} q) (1-t_2 q)} .
\]
\end{conj}

Following the general relationship between characters of holomorphic twists of supersymmetric theories and supersymmetric indices, this our conjecture for the supersymmetric index of the six-dimensional superconformal field theory associated to the Lie algebra $\lie{sl}(2)$. 
Similarly, the index associated to the $\lie{gl}(2)$ theory is conjectured to be simply the product 
\[
\chi_{\lie{gl}(2)}^{6d} (q,t_1,t_2,r) = \chi_{\lie{sl}(2)}^{6d} (q,t_1,t_2,r) \cdot \chi_{\lie{u}(1)}^{6d}(q,t_1,t_2,r)
\]
where the index for the $\lie{u}(1)$ theory is given in the previous section. 
As a factorization algebra \brian{finish}

\parsec[]

The specialization of this index $t_1=t_2=r=1$ yields the single particle index
\[
\frac{3q^4 + 3 q^3 - 6 q^{7/2}}{(1-q)^3}. 
\]

\parsec[]

The specialization of this index $q=r^2, t_2=1$ in \eqref{eqn:special1} yields the plethystic exponential of the following single particle index
\[
f_{two}(t_1, 1, q, q^{1/2}) = \frac{q^2}{1-q} 
\]
which is the same as the single particle index of Virasoro vacuum module on the Riemann surface $\Sigma = \C_{z_1}$. 

\subsection{Operators on a single membrane}


\end{document}