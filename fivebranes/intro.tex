%\documentclass[11pt]{amsart}
%
%%\usepackage{../macros-master}
%\usepackage{macros-fivebrane}
%
%\begin{document}

\section{Introduction}
Superconformal field theories admit a plethora of exactly computable, protected quantities, giving them a distinguished role in supersymmetric physics. 
Since it is believed that all supersymmetric field theories flow to superconformal fixed points, such quantities provide robust invariants of supersymmetric field theories. 
Examples of such a quantities are superconformal indices, which are generating functions for the $R$-charges of BPS local operators. 
Crucial steps towards a more microscopic understanding of superconformal indices were taken in \cite{}, articulated through the construction of twisting.

Introduced by Witten \cite{WittenTwist} and further developed by Costello \cite{CostelloHol}, twisting refers to a localization, or fixed-point, construction for theories equipped with an action of a supersymmetry algebra. 
Operationally, one modifies the BRST differential of the theory by a nilpotent supercharge---the result is a theory for which the infinitesimal translations in the image of the nilpotent supercharge act homotopically trivially. 
When a twist exists (which is almost always) a supersymmetric field theory admits a so-called minimal twist; the resulting theory is a holomorphic-topological field theory that is holomorphic in the maximal number of spacetime directions.

One of the insights of \cite{} was that superconformal indices count exactly local operators in the minimal twist---accordingly, we may think of the algebra of local operators in the minimal twist as categorifying the index. 
Moreover, the algebra of local operators is part of the richer structure of a \textit{factorization algebra}. 
Whilst the former governs the behavior of observables supported at points, the latter organizes observables that are supported on any open set (for example, non-local operators obtained from local ones via a descent procedure). 

A principal goal of the present paper is to initiate a study of factorization algebras associated to the minimal twists of the six-dimensional~$\cN=(2,0)$ superconformal field theories.
%and three-dimensional $\cN=8$ superconformal field theories. 
As is well-known, the six-dimensional~$\cN=(2,0)$ supersymmetric theory is quite elusive---it admits no known Lagrangian formulation, and outside of the abelian case, is not known to admit a field theoretic realization. 
In light of this, we propose to access the minimal twist and its local operators via the proposal of \textit{twisted holography}.

\subsection{Twisted Holography}
Introduced by Costello and Li in \cite{CLsugra}, the twisted holography proposal posits an avatar of the AdS/CFT correspondence that holds at the level of supersymmetric twists. 
Fundamentally, it conjectures a duality between the algebra of observables of the (twisted) gravitational theory and the algebra of observables of the gauge theory.
There is an exciting body of work being developed around this program including tests of this proposal from both the gravitational and gauge theory side.

As we mentioned above, the precise mathematical structure modeling observables of a quantum field theory is that of a factorization algebra.
The predicted type of duality between the factorization algebras associated to a gravitational theory and to the worldvolume theory of a number of branes is a general version of \textit{Koszul duality}.
Ordinary Koszul duality for associative algebras (so quantum mechanical systems) associates to a (augmented) algebra $A$ a dual $A^!$ whose

A more concrete, albeit informal, statement of twisted holography is as follows. 
Let $X$ be a smooth manifold, and let $\Obs_{grav}$ denote a factorization algebra on $X$ that we view as the observables of the bulk gravitational theory. 
Suppose we have, in addition, a stack of $N$ branes, wrapping a submanifold $\iota \colon Y\to X$, whose worldvolume theory has a factorization algebra of observables $\Obs_{brane}(N)$.

\brian{to do}
\begin{itemize}
\item 6d theory how ubiquitous it is. 
\item more elaboration on twisted holography.
\end{itemize}


\begin{expect}[Twisted Holographic Principle]
  There is a map of factorization algebras
  \[
        (\Obs_{grav}|_{Y})^{!}\to \Obs^{N}_{brane}
  \]

      that becomes an equivalence in the large $N$ limit.
\end{expect}



Nonetheless, the above expectation has been tested in many examples \cite{}.


\subsection{Infinite dimensional symmetry enhancement by exceptional simple lie superalgebras}

\brian{to do}

\subsection{Outline}

The paper is organized as follows. The first section is the present introduction.

In section 2 we begin by recalling descriptions of minimal twists of eleven dimensional supergravity \cite{} and the abelian six-dimensional $\cN=(2,0)$ \cite{} and three-dimensional $\cN=8$ \cite{} superconformal field theories.

%\end{document}
