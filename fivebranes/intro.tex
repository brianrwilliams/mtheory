\documentclass[11pt]{amsart}

\linespread{1.15} %for editing

%\usepackage{../macros-master}
\usepackage{macros-fivebrane}

\begin{document}

\section{Introduction}
Superconformal field theories admit a plethora of exactly computable, protected quantities, giving them a distinguished role in supersymmetric physics. 
Since it is believed that all supersymmetric field theories flow to superconformal fixed points, such quantities provide robust invariants of supersymmetric field theories. 
Examples of such a quantity is the superconformal index, which is a generating function for some collection of $R$-charges of certain BPS local operators. 
Crucial steps towards a more microscopic understanding of superconformal indices were taken in \brian{surya fill in this empty citation} \cite{}, articulated through the construction of twisting.

Introduced by Witten \cite{WittenTwist} and further developed by Costello \cite{CostelloHol}, twisting refers to a localization, or fixed-point, construction for theories equipped with an action of a supersymmetry algebra. 
Operationally, one modifies the BRST differential of the theory by a nilpotent supercharge---the result is a theory for which the infinitesimal translations in the image of the nilpotent supercharge act homotopically trivially. 
When a twist exists (which is almost always) a supersymmetric field theory admits a so-called minimal twist; the resulting theory is a holomorphic-topological field theory that is holomorphic in the maximal number of spacetime directions.

One of the insights of \brian{surya fill in this empty citation} \cite{} was that superconformal indices count exactly local operators in the minimal twist---accordingly, we may think of the algebra of local operators in the minimal twist as categorifying the index. 
Moreover, the algebra of local operators is part of the richer structure of a \textit{factorization algebra}. 
Whilst the former governs the behavior of observables supported at points, the latter organizes observables that are supported on any open set (for example, non-local operators obtained from local ones via a descent procedure). 

A principal goal of the present paper is to initiate the study of the factorization algebras of observables associated to the minimal twists of the six-dimensional~$\cN=(2,0)$ superconformal field theories.
As a factorization algebra encodes, in particular, the local operators of a theory we can extract from them familiar quantities like the superconformal index.
%and three-dimensional $\cN=8$ superconformal field theories. 
As is well-known, the six-dimensional~$\cN=(2,0)$ supersymmetric theory is quite elusive---it admits no known Lagrangian formulation, and outside of the abelian case, is not known to admit a field theoretic realization. 
In light of this, we propose to access the minimal twist and its local operators via the proposal of \textit{twisted holography}.
Before getting into the details of this proposal, we point out a collection of exceptional super Lie algebras which will play a pivotal role in our approach; and appear quite naturally from the point of view of twisted supergravity and twisted superconformal field theory.

\subsection{Infinite dimensional symmetry enhancement by exceptional simple super Lie algebras}

In \S\ref{s:twisted} we will recall a description of the minimal twist of eleven-dimensional supergravity, the low energy limit of $M$-theory, following \cite{RSW}.
In this twist, the theory is holomorphic in a maximal number of directions, which is five complex directions, and topological in the remaining real direction. 
Surprisingly, there is a novel relationship between this theory and a certain exceptional simple super Lie algebra called $E(5|10)$. 
On flat space, $\R \times \C^5$ we have shown that our description yields a presentation of the algebra of local operators which takes the following form 
\beqn
\clie^\bu\left(\Hat{E(5|10)}\right) .
\eeqn
This is the Chevalley--Eilenberg complex (computing Lie algebra cohomology) of a particular central extension of the super Lie algebra $E(5|10)$. 
We emphasize that this statement about local operators is only interesting when one works in the Batalin--Vilkovisky formalism taking into account all ghosts, fields, anti-fields, etc.. 

There is another exceptional super Lie algebra around when one starts to consider fivebranes inside of eleven-dimensional supergravity. 
In this twist, fivebranes are completely holomorphic objects which, in flat space, wrap three complex directions in the eleven-dimensional bulk theory
\beqn
\C^3 \subset \R \times \C^5 .
\eeqn
An important facet of the wordvolume theory associated to a stack of fivebranes in $M$-theory is that it is a superconformal theory with $\cN=(2,0)$ supersymmetry. 

In six dimensions (after complexifying) the superconformal algebra is the super Lie algebra $\lie{osp}(8|4)$, whose even part is $\lie{so}(8) \times \lie{sp}(4)$. 
Twisting involves the choice of a holomorphic supercharge $Q$ which leaves three directions invariant and breaks this super Lie algebra down to the smaller super Lie algebra $\lie{osp}(6|2)$.
The $Q$-twist of any six-dimensional superconformal field theory has a symmetry by this super Lie algebra. 

As we just pointed out, the super Lie algebra $E(5|10)$ (or rather its central extension) describes the ghost system for symmetries of eleven-dimensional supergravity after the minimal twist. 
It is quite easy to write down an explicit realization of the residual superconformal algebra $\lie{osp}(6|2)$ on $E(5|10)$. 
In fact, $\lie{osp}(6|2)$ sits inside of a small (but still infinite-dimensional) super Lie algebra called $E(3|6)$. 

\begin{conj}[with Ingmar Saberi]
After the holomorphic twist, the six-dimensional superconformal symmetry algebra gets enhanced to the exceptional simple super Lie algebra $E(3|6)$.
As a consequence, after twisting, the space of local operators of any six-dimensional superconformal theory is a representation for $E(3|6)$.
\end{conj}

A consequence of this conjecture is that we can interpret, for example, the superconformal index as a character for $E(3|6)$. 
For reference to future formulas, we point out that we choose a Cartan for $E(3|6)$ whose basis is labeled by the standard fugacities which are typically used to count the superconformal index.
As we turn to holography, we would like to keep in mind this proposed large symmetry algebra which exists after twisting six-dimensional $\cN=(2,0)$ supersymmetry. 
Fitting with this conjecture, we will identify our proposed spaces of local operators associated to the worldvolume theory on stacks of fivebranes with known representations for $E(3|6)$. 

\subsection{Our approach via twisted holography}
Introduced by Costello and Li in \cite{CLsugra}, the twisted holography proposal posits an avatar of the AdS/CFT correspondence that holds at the level of supersymmetric twists. 
Fundamentally, it conjectures a duality between the algebra of observables of the (twisted) gravitational theory and the algebra of observables of the gauge theory.
There is an exciting body of work being developed around this program including tests of this proposal from both the gravitational and gauge theory side.
The type of twisted holography which is at play in this paper is a twisted version of the $AdS_7/CFT_6$ duality involving fivebranes inside of eleven-dimensional supergravity.  

We mentioned that the precise mathematical structure modeling observables of a quantum field theory is that of a factorization algebra.
The predicted type of duality between the factorization algebras associated to a gravitational theory and to the worldvolume theory of a number of branes is a general version of \textit{Koszul duality}.
Ordinary Koszul duality for associative algebras (so quantum mechanical systems) associates to a (augmented) algebra $A$ a dual algebra $A^!$ whose appropriate derived category of modules is the same as that of $A$. 
Following the work of \cite{CLsugra,CP1} (see also the review in \cite{PWkoszul}) there is a simple physical interpretation of Koszul duality.
If $A$ is the algebra of operators of some bulk quantum field theory (perturbatively we can even consider a theory of gravity) then $A^!$ is the algebra of operators on the universal topological line defect.
Universal here means that algebra of operators on any other line defect which couples to the bulk system admits a unique map of algebras from~$A^!$. 

The general theory of Koszul duality for factorization algebras has not been developed, and we do not do so in this paper.
This sort of duality would allow one to make sense of universality statements as above for higher dimensional, possibly non-topological, defects in an arbitrary bulk quantum field theory.
Nevertheless, we can make the following ansatz for the Koszul dual $\cF^!$, which we refer to as the $!$-dual in this paper to avoid confusion, of a factorization algebra~$\cF$ of observables of some bulk quantum field theory.
It is the universal factorization algebra, along a specified defect, which couples to the bulk quantum field theory.
While this heuristic definition sounds natural, it does not lend itself to an explicit description.
However, for factorization algebras of bulk quantum field theories which can be expressed in a particular way we can refer to a local version of Noether's theorem in order to get our hands on it, see~\S\ref{s:noether}.

Let us now make a more concrete, yet slightly informal, statement of twisted holography which fits into the approach of this paper.
Let $X$ be a smooth manifold, and let $\Obs_{grav}$ denote a factorization algebra on $X$ that we view as the observables of the bulk gravitational theory. 
Suppose we have, in addition, a stack of $N$ branes, wrapping a closed submanifold $Y\hookrightarrow X$ whose worldvolume theory has a factorization algebra of observables $\Obs_{brane}(N)$.
In the context of branes, it is natural to posit that the universal theory along the brane is given by the large $N$ limit of the factorization algebra $\Obs_{brane}(N)$.

Note that $\Obs_{grav}$ is a factorization algebra on $X$, while $\Obs_{brane}(N)$ is a factorization algebra on the closed submanifold $Y$ so we cannot yet compare them.
We can, however, restrict $\Obs_{grav}$ to a factorization algebra just on $Y$, which we denote by $\Obs_{grav}|_Y$.\footnote{In general this is given by some limit construction like for sheaves, but we will use a particularly nice model in the context of holomorphic-topological factorization algebras in this paper.}

\begin{expect}[Twisted holographic principle following \cite{CLsugra}]
There is a map of factorization algebras
\[
  (\Obs_{grav}|_{Y})^{!}\to \Obs_{brane}(N)
\]
that becomes an equivalence in the large $N$ limit.
\end{expect}

From the point of view of the bulk gravitational theory, one can understand $(\Obs_{grav}|_{Y})^{!}$ as a factorization algebra enhancement of the space of (twisted) multi-particle states of the gravitational theory on the manifold $X - Y$ obtained by subtracting the brane. 
In most examples the above principle is too naive as we are forgetting a key ingredient, the so-called backreaction of the branes. 
This deforms the geometric structure used to define the gravitational theory on $X - Y$---concretely, it will correspond to a certain deformation of the factorization algebra $(\Obs_{grav}|_{Y})^{!}$. 

The above expectation can be tested in instances where both sides of the duality admit explicit descriptions. 
This has been carried out in many examples including:
\begin{itemize}
\item $D1$ branes in the topological string on the deformed conifold \cite{costello2021twisted}.
Here, the duality is formulated in terms of vertex algebras which are avatars of holomorphic factorization algebras on a Riemann surface.
This is also understood as a deformed version of the physical duality between 4d $\cN=4$ supersymmetric Yang--Mills and type IIB superstring theory on $AdS_5$. 
\item A system of twisted $D1-D5$ branes in type IIB string theory on twisted $AdS_3 \times S^3 \times T^4$ \cite{CP1}. 
Upon compactifying along $T^4$ this can be understood as a variant of the above example where the deformed conifold is replaced by a supersymmetric version.
\item Membranes and fivebranes in twisted $\Omega$-deformed $M$-theory on Taub-NUT space \cite{CostelloM5,CostelloM2}.
In the particular $\Omega$-background membranes are localized to a topological quantum mechanical system where the duality can be phrased in terms of associative algebras (and hence ordinary Koszul duality). 
The $\Omega$-background localizes fivebranes to a complex plane and the duality can be formulated in terms of vertex algebras (one-dimensional holomorphic factorization algebras). 
\end{itemize}

A feature of the above examples is that via some method one can understand, independently, the algebra of observables on both sides of the duality. 
For fivebranes in $M$-theory, before turning on the $\Omega$-background, there is no known description which is robust enough to fit inside of the factorization algebra formulation of a quantum field theory.
So, rather than prove, or test, a version of the twisted holographic principle, the objective of this paper is to \textit{use} the twisted holographic principal to obtain an explicit description of the factorization algebra associated to the worldvolume theory on a stack of fivebranes (after performing the minimal, holomorphic, twist). 

Rather surprisingly, we will argue that in the classical limit the factorization algebra associated to the worldvolume theory on a finite number of twisted fivebranes is a particular piece of the $!$-dual of the factorization algebra associated to the bulk gravitational theory. 
After taking a certain classical limit, this argument relies on a presentation of the $!$-dual as
\beqn
(\Obs_{grav}|_{Y})^{!} \simeq \cF_{-1} \otimes \cF_0 \otimes \cF_1 \otimes \cdots 
\eeqn
for some factorization algebras $\cF_{-1},\cF_0, \cF_{1},\ldots$ which are defined using a certain weight decomposition of the bulk gravitational theory.
In fact, this decomposition is intimately related to a certain decomposition of the exceptional simple super Lie algebra $E(5|10)$ that we have already mentioned plays an important role in the minimal twist of eleven-dimensional supergravity.
The strange indexing conventions will be explained in \S\ref{eqn:fact}. 

Using this presentation, our conjecture for the factorization algebra associated to the worldvolume theory on a stack of $N$ twisted fivebranes, after taking the same classical limit, has the form
\beqn\label{eqn:finiteTensor}
\Obs_{brane} (N) = \cF_{-1} \otimes \cF_2 \otimes \cdots \otimes \cF_{N-2} .
\eeqn
In other words, we simply truncate the tensor decomposition at order $N$.
The surprising part is that the factorization algebras $\cF_k$ are directly identifiable from the point of view of the gravitational theory!

A related theory is the minimal twist of the six-dimensional $\cN=(2,0)$ superconformal theory associated to a Lie algebra of type $A_{N-1}$.
We denote the corresponding factorization algebra by $\Obs_{A_{N-1}}$ for now. 
This is obtained from the worldvolume theory simply by throwing away the modes propagating transverse to the brane, which in our presentation above corresponds to stripping off the first factor $\cF_{-1}$. 
Thus, at the level of factorization algebras we have a similar proposed decomposition
\beqn
\Obs_{A_{N-1}} = \cF_0 \otimes \cF_1 \otimes \cdots \otimes \cF_{N-2} .
\eeqn

We want to emphasize that these expressions only hold after taking this classical limit and taking some truncation of the differential present which is present on the left-hand side.\footnote{An instructive avatar of this decomposition to keep in mind is the description of the $W_N$ vertex algebra as being generated by a collection of operators of spins $2,3,\ldots, N$.
Only in the classical limit does the $W_N$ vertex algebra decompose as a pure tensor product.}
This decomposition as a tensor product of factorization algebras will break down at the quantum level.
Nevertheless, the description provides a very effective way to compute certain protected quantities like the superconformal index.

As the formalism of Koszul duality and factorization algebras is rather abstract, it is useful to first indicate what our approach looks like at the level of the superconformal index. 
On one hand, we have the index of multi-particle states of the supergravity theory on the manifold obtained by removing the location of the brane; we denote this by $\chi_{grav}({\bf x})$.
It is a function of the fugacities ${\bf x}$ which in this setting are given by coordinate functions on the four-dimensional Cartan subalgebra of~$\lie{osp}(6|2)$.
Typically, this multi-particle index can be obtained from the single-particle index $f_{grav}({\bf x})$ of gravitational states through the plethystic exponential. 
Since the index is only really sensitive to the minimal twist, it is reasonable that we can compute the index using our formulation of twisted eleven-dimensional supergravity.
We will obtain known formulas for the index of eleven-dimensional supergravity states on $AdS_7$ in \S\ref{s:states} using our description of twisted supergravity. 

While we don't need to know the precise relationship at the moment, we can point out that the data of a factorization algebra encodes, in particular, the algebra of local operators of a system.
This is convenient from the point of view of the superconformal index since as we have already pointed out we can compute it as the index of local operators of the twisted theory.
We will see how the descriptions above lead to a presentation for the superconformal index $\chi_N({\bf x})$ of the worldvolume theory on a finite stack of $N$ fivebranes. 

If we are working simply on the worldvolume $\C^3 \cong \R^6$, our description in \eqref{eqn:finiteTensor} posits that after taking the classical limit the local operators on a stack of $N$ twisted fivebranes is
\beqn
\cF_{-1}(0) \otimes \cF_0(0) \otimes \cdots \otimes \cF_{N-2}(0) ,
\eeqn
where $\cF_k(0)$ stands for the algebra of local operators associated to the factorization algebra $\cF_k$.
Forgetting about the differentials for the time being, the local operators $\cF_k(0)$ are given as the free algebra on some vector space $V_k$ of linear local opeartors.
Thus, to compute the index of local operators of $\Obs_{brane}(N)$ it suffices to compute the index of all of the $V_k$---this is the single particle index---and then apply the plethystic exponential.
For now, denote by $g_k({\bf x})$ the index of the vector space $V_k$.
This quantity is directly computable from the gravitational side.

Putting all of this together, our approach is based on the observation that we can express the single particle supergravity index as
\beqn
f_{grav}({\bf x}) = \sum_{k=-1}^\infty g_k({\bf x}) .
\eeqn
Then, the avatar of our conjectural description in \eqref{eqn:finiteTensor} at the level of the superconformal index is
\beqn
f_N ({\bf x}) = \sum_{k = -1}^{N-2} g_k({\bf x}) .
\eeqn
To obtain the superconformal index we simply apply the plethystic exponential
\beqn
\chi_N({\bf x}) = \prod_{k=-1}^{N-2} {\rm PExp}\left[g_k({\bf x})\right] .
\eeqn
This is a formula for the index on a finite stack of $N$ fivebranes. 
A related quantity is the index of the superconformal field theory associated to the Lie algebra $A_{N-1}$. 
To obtain this we simply throw away contributions coming from a single fivebrane, which results in the expression
\beqn
\chi_{A_{N-1}}({\bf x}) = \prod_{k=0}^{N-2} {\rm PExp}\left[g_k({\bf x})\right] .
\eeqn

From these formulas it is manifest that the holographic relation $\chi_{grav} = \lim_{N \to \infty} \chi_N$ holds. 
The next thing we turn to is evidence for our claim that $\chi_N({\bf x})$ really is the superconformal index associated to fivebranes. 
In ~\S\ref{s:finite} we will show that for small values of $N$ our closed of formulas, when appropriately expanded, agree with approximations which have been found in the literature.
We will also check that a number of specializations of the fugacities agree with certain unrefined indices which have been computed via other means.
Even stronger evidence would involve the factorization algebras $\Obs_{brane}(N)$, and we will briefly sketch how this should go in later sections but leave a full treatment to future work.

To state some explicit results and to make concrete conjectures for these indices we will take a look at small values of~$N$. 

\subsection*{The abelian $\cN=(2,0)$ theory}

The abelian six-dimensional $\cN=(2,0)$ theory is the worldvolume theory on a single fivebrane. 
It admits a field theoretic description as the so-called $\cN=(2,0)$ tensor multiplet. 
In \cite{SWtensor} an explicit description of the minimal, holomorphic, twist of this theory is given---it defines a complex three-dimensional factorization algebra that we will denote by $\Obs_1$.
In this paper we only consider the classical limit of this factorization algebra, but there is a simple quantization of its natural shifted Poisson structure which we will use in this section.\footnote{From the holographic point of view, this quantum parameter corresponds to including effects from the backreaction.}

The superconformal index can be computed from the local character of the factorization aglebra $\Obs_1$. 
We recover the standard formulas in the literature in terms of the five fugacities $y_1,y_2,y_3,y,q$ which satisfy the constraint $y_1 y_2 y_3 = 1$, see for instance \cite{Kim:2013nva}.

Let us place this factorization algebra on the threefold $\C^\times \times \C^2$ and consider the reduction along the circle $S^1 \subset \C^\times$. 
In \cite{SWtensor} we have shown that the naive dimensional reduction (where we throw out higher Kaluza--Klein modes along $S^1 \subset \C^\times$) of this factorization algebra agrees with the factorization algebra associated to the minimal twist of perturbative supersymmetric Yang--Mills theory for the group $U(1)$. 
If we keep track of Kaluza--Klein modes, our ansatz is that the full dimensional reduction 
\beqn
\oint_{S^1} \Obs_1
\eeqn
is the factorization algebra associated to the minimal twist of supersymmetric Yang--Mills theory on $\R_{\geq 0} \times \C^2$ including contributions from monopole operators.
In particular, we can recover the Hilbert space of the theory by looking at the value of the factorization algebra along $0 \times \C^2 \subset \R_{\geq 0} \times \C^2$.

There is a nice relationship to the AGT correspondence which we can remark on in this abelian case. 
In the notation we are using for fugacities, the so-called Schur limit of the superconformal index is simply $y_1=y=1$.
For the abelian theory this immediately gives the function $\frac{q}{1-q}$ which is the well-known vacuum character of the Heisenberg vertex algebra. 
Geometrically, the Schur limit corresponds to working with an equivariant version of the Hilbert space.
In this setting, this is the equivariant cohomology of the moduli space of instantons of rank one.
This specialization then agrees with the famous result of Nakajima which characterizes the equivariant cohomology of the moduli space of rank one instantons (also known as the Hilbert scheme of points) with the Heisenberg vacuum module.

If we return to six dimensions, one can use the radial direction in $\C^\times \times \C^2$ to endow the factorization algebra $\oint_{S^1} \Obs_1$ with the structure of an (dg) associative algebra. 
This Schur limit corresponds to a particular deformation of this algebra by an explicit Maurer--Cartan element in the residual superconformal algebra $\lie{osp}(6|2)$.
We point out that this deformation is of the same spirit considered in \cite{BeemEtAl}.
One can identify (though we do not do that in this paper) this deformation of $\oint_{S^1} \Obs_1$ with the Heisenberg algebra. 
Thus, we expect that the dg algebra $\oint_{S^1} \Obs_1$, {\em before} deforming, should act on the non-equivariant cohomology of the moduli space of instantons of rank one.

\subsection*{The $\cN=(2,0)$ theory of type $A_1$} 

\begin{conj}\label{conj:6dtwo}
The superconformal index of the six-dimensional $\cN=(2,0)$ theory of type $A_1$ is 
\[
\chi_{A_1} (y_i,y,q) = {\rm PExp} \left[f_{A_1}(y_i,y,q) \right] .
\]
where the single particle index $f_{A_1}(y_i,y,q)$ is 
\beqn\label{eqn:A1}
\frac{q^4(y_1+y_2+y_3) + q^2 (y^2 + q + q^2 y^{-2}) - q^{3} (y + q y^{-1})(y_1^{-1} + y_2^{-1} + y_3^{-1})}{(1-y_1q) (1-y_2 q) (1-y_3 q)}.
\eeqn
\end{conj}

\subsection{Outline}

The paper is organized as follows. The first section is the present introduction.

In section 2 we begin by recalling descriptions of minimal twists of eleven dimensional supergravity \cite{} and the abelian six-dimensional $\cN=(2,0)$ \cite{} and three-dimensional $\cN=8$ \cite{} superconformal field theories.

\end{document}
