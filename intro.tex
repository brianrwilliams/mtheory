%\documentclass[11pt]{amsart}
%
%\usepackage{macros-mtheory,amsaddr}
%\usepackage{array}
%
%\def\vep{\varepsilon}
%
%\addbibresource{cfs.bib}
%
%%\linespread{1.2} %for editing
%%\usepackage{mathpazo}
%
%\begin{document}

Twists of supersymmetric field theories have been a subject of active and fruitful study for many years. 
The notion of twisting was introduced by Witten and dates back to the first example~\cite{WittenTwist}, in which a topological field theory related to Donaldson invariants of four-manifolds was constructed as a twist of four-dimensional $\N=2$ supersymmetric gauge theory.
Shortly afterwards (thirty years ago), the field had expanded to include a diverse set of examples related to topological field theory~\cite{BlauThompson}. 

For many years, twisting constructions were motivated by the goal of producing topological field theories, and correspondingly emphasized the role of particular modifications of  Lorentz symmetry via ``twisting homomorphisms'' designed to produce scalar supercharges.
From a more modern perspective, though, twisting a supersymmetric field theory is just a systematic way of taking invariants with respect to a fixed supercharge. This requires that the supercharge square to zero, so that the supercharge defines an odd abelian subalgebra of the supersymmetry algebra.\footnote{More generally, one could consider ``omega backgrounds'' by taking the derived invariants of a more general subalgebra in which the odd part is not necessarily abelian.}
The variety of appropriate square-zero elements was studied systematically in~\cite{NV}, and has deep relations not only to the  classification of  twists, but to the construction  of supersymmetric field theories via the pure spinor formalism~\cite{Cederwall,EHSW}. The name ``pure spinor'' originates because the  variety of  square-zero elements always contains a minimal stratum that is related to the space of Cartan pure spinors; in turn, these correspond to choices of complex structure on the  spacetime, at least when the dimension is even. 
As such, any $d$-dimensional  supersymmetric theory that admits a twist admits a \emph{minimal} or~\emph{holomorphic} twist, in which $n=\lfloor d/2\rfloor$ of the translations act nontrivially. Other twists can be understood as further twists of this holomorphic theory.
The importance and interest of holomorphic twists was emphasized in the work of Costello~\cite{CostelloHol}, who developed the notion mathematically, building on earlier work on holomorphic field theories in the physics literature~\cite{NekThesis}. (Constructions related to holomorphic twists had appeared in the context of supersymmetric indices; see~\cite{Romelsberger}, for example.) 

In a supergravity theory, local supersymmetry is already gauged, so that the standard notion of twisting is {\em a priori} not a sensible operation. Twisted supergravity thus needs to be  understood differently.
In their seminal paper~\cite{CLsugra}, Costello and Li gave a definition of twisted supergravity: it is supergravity in a background where the bosonic ghost field representing a twisting supercharge takes a nonzero value.
Twisting a supersymmetric field theory, in the classical sense, can then be  understood as coupling the theory to supergravity and placing it in such a background.

Using ideas related to string field theory and the topological B-model, Costello and Li provided conjectural descriptions of many examples of ten-dimensional theories of twisted supergravity. They used these descriptions to rigorously construct quantizations of open-closed string field theories~\cite{CLbcov1,CLtypeI}. 
A very active and promising direction of current research uses these twisted ten-dimensional models to formulate twisted versions of holography; see~\cite{CostelloM2, Costello_2021, costello2021twisted, Ishtiaque_2020, budzik2021giant, Gaiotto:2020vqj}, for example.
In \cite{CostelloM5} supergravity in the $\Omega$-background has been introduced, and has been further studied in \cite{Gaiotto:2019wcc, Oh:2020hph, Oh:2021bwi}. 

In this paper, we provide perturbative descriptions of all twists of eleven-dimensional supergravity (before turning on any $\Omega$-background).
%By `twists' we mean ones which arise from setting a bosonic ghost to take a nonzero value of a particular supertranslation in the 11-dimensional supersymmetry algebra. 
Thanks to a straightforward classification, which we recall below, there are essentially two types of twists which deform the flat background $\RR^{11}$: 
\begin{itemize}
\item The minimal (or holomorphic) twist. 
The resulting twisted theory is defined on $\CC^5 \times \RR$ and is holomorphic along $\CC^5$ and topological along the `time' direction~$\RR$. 
The twist is $SU(5)$ invariant and involves a fixed choice of a Calabi--Yau form on $\CC^5$. 
\item The non-minimal twist. 
The resulting twisted theory is defined on $\CC^2 \times \RR^7$ and is holomorphic along $\CC^2$ and topological along $\RR^7$.
The twist is $SU(2)$ invariant and involves a fixed choice of a hyper K\"ahler structure on $\CC^2$.  
\end{itemize}

In~\cite{SWspinor}, the last two authors used the pure spinor formalism to describe the {\em free field limit} of the minimal twist of 11-dimensional supergravity on flat space, following work of Cederwall~\cite{Ced-towards,Ced-11d}. In fact, the result of \cite{SWspinor} shows that the pure spinor formalism is compatible with twisting, so that the twist of any multiplet can be recovered from the algebraic geometry of a neighborhood of the corresponding point in the scheme of square-zero elements.
(In separate work~\cite{EagerHahner}, the free-field limit of the non-minimal twist was derived from the component-field multiplet as used in the physics literature.)
%\brian{also reference fabian and richard}
The resulting theory is $\ZZ/2$ graded; the grading is inherited from the totalization of ghost number and intrinsic parity in the original, untwisted, eleven-dimensional theory.
In \S\ref{sec:dfn} we extend this free limit of the twist of 11-dimensional supergravity to a fully interacting theory in the Batalin--Vilkovisky (BV) formalism.\footnote{One subtlety is that unlike the usual BV formalism which involves a $\ZZ$-grading by ghost number, the 11-dimensional theory is only $\ZZ/2$ graded.
All odd fields can therefore be considered as ``ghosts'' in the twisted theory.}
Our main conjecture is that this interacting BV theory is equivalent to the minimal twist of 11-dimensional supergravity on flat space.

\begin{conj}
The $11$-dimensional partially holomorphic-topological theory on $\CC^5 \times \RR$ that we will define in \S \ref{s:dfn} is equivalent to twisted supergravity on $\RR^{11}$ where the bosonic ghost takes value the holomorphic supercharge. 
\end{conj}

%\brian{Ingmar, maybe you can comment on the status of this conjecture.}
In this paper we will provide evidence for this conjecture on a variety of fronts. 
In \S\ref{sec:susy}, we will show that this theory has all of the residual supersymmetries present after performing the holomorphic twist. 

In \S\ref{sec:dimred}, we compute dimensional reductions and show that our model is compatible with descriptions of twists of lower dimensional twists of supergravity. 
For instance, reducing along a circle in a complex plane agrees with the conjectural description of the $SU(4)$ twist of type IIA supergravity of \cite{CLsugra}.

In \S\ref{sec:ads} we propose a description of supergravity in twisted versions of both the ${\rm AdS}_7$ and ${\rm AdS}_4$ backgrounds. 
We show that the residual symmetries of supergravity in these backgrounds are present in our twisted background.

Finally, we describe the non-minimal twist of 11-dimensional supergravity as a background of our theory on $\CC^5 \times \RR$. 
This non-minimal twist is invariant for the group $G_2 \times SU(2)$. 
We find a match with a conjectural description of this $G_2 \times SU(2)$ invariant twist formulated by Costello in \cite{CostelloM5} and further developed in \cite{RY}. 
%Surprisingly, at the twisted level these symmetries 

\subsection*{A geometric description of the model} 

Our 11-dimensional theory is defined more generally on the product of manifolds 
\[
X \times S
\]
where $X$ is a Calabi--Yau five-fold and $S$ is a smooth oriented real one-dimensional manifold. 
The theory is of a holomorphic gravitational flavor as it describes ``partial'' deformations of complex structures along $X$. 
We will explain what we mean by this momentarily. 

In our theory there is a field which encodes this partial deformation of complex structure on the Calabi--Yau manifold $X$.
It is an even field 
\[
\mu \in \Omega^{0,1} (X , \T_X) \otimes C^\infty(S) 
\]
where $\T_X$ denotes the holomorphic tangent bundle on $X$.
Locally, $\mu$ can be decomposed as Beltrami-like differential
\[
\mu = \mu^i_j (z,\zbar, t) \d \zbar_i \frac{\partial}{\partial z_j} .
\]
Physically speaking, $\mu$ is a component of the metric which survives in the twisted theory, see \S \ref{s:components}. 

Next, there are two fields 
\[
\gamma^{1,0} \in \Omega^{1,0} (X) \otimes C^\infty(S), \quad \gamma^{1,2} \in \Omega^{1,2}(X) \otimes C^\infty(S) .
\]
The field $\gamma^{1,2}$ is a component of the supergravity $3$-form $C$-field that survives in the twisted theory. 
The field $\gamma^{1,0}$ can be interpreted as a component of the one-form which is a ghost-for-a-ghost of the $C$-field.\footnote{The $C$-field has a gauge symmetry of the form $\delta C = \d B$ where $B$ is a two-form.
This ghost $B$ field has an additional gauge symmetry $\delta B = \d A$ for $A$ a one-form.
The field $\gamma^{1,0}$ is a component of $A$.}

There is a background where the equation of motion involving the $\mu, \gamma^{1,0}, \gamma^{1,2}$ fields reads
\begin{align*}
\dbar \mu + \frac12 [\mu, \mu] + \Omega^{-1} \vee \left(\del \gamma^{1,0} \wedge \del \gamma^{1,2} \right) & = 0 \\
%(\dbar + \mu) \gamma^{1,0} = 
%(\dbar + \mu) \gamma^{1,2} & = 0 
%\\
%\div \mu & = 0 \\
%\d_S \mu = \d_S \gamma^{1,0} = \d_S \gamma^{1,2} = & = 0 .
\end{align*}
Additionally, there are the conditions that $\mu$ preserve the holomorphic volume form on $X$ and that all fields are locally constant along the topological direction $S$.  
%In the last line $\d_S$ is the de Rham differential on $L$ and implies that $\mu, \gamma^{1,0}$, and $\gamma^{1,2}$ are locally constant along $L$. 
Notice that were it not for the presence of the last term in the first equation then this would be the integrability equation for $\mu$ to determines a complex structure on $X$.
%The second two equations resemble the conditions that $\gamma^{1,0}, \gamma^{1,2}$ be holomorphic with the complex structure determined by $\mu$. 
%Finally, the fourth equation constrains $\mu$ to preserve the holomorphic volume form on $X$. 

If we work in a background where one of $\gamma^{1,0}$ or $\gamma^{1,2}$ is zero then we see that $\mu$ is exactly a deformation of complex structure along $X$. 
In terms of the 11-dimensional geometry, this is a background which describes deformations of the natural transverse holomorphic foliation (THF structure) on $X \times S$ which further preserves the holomorphic volume form along the leaves.
We further unpack the equations of motion in more general backgrounds in \S \ref{s:components}, but leave a complete study for future work.
%The equations of motion state that $\gamma$ is constant along $\RR$ and holomorphic along $\CC^5$. 
%Further, $\gamma$ is only defined up to the additional of a total holomorphic derivative $\gamma \sim \gamma + \del \beta$ where $\beta$ is a holomorphic function on $\CC^5$ which is constant along $\RR$. 


\subsection*{Appearance of exceptional Lie superalgebras}

The gauge symmetries of a field configuration in any theory form a Lie algebra. 
In the Batalin--Vilkovisky formalism, one combines fields of all ghost number into a single object which, together with the linear BRST operator, has the structure of a dg Lie algebra.\footnote{In some models one actually obtains an $L_\infty$ algebra.} 
From the point of view of deformation theory, this dg Lie algebra describes the formal moduli space of deformations of the particular field configuration. 

The simplest field configuration in the twisted theory is the flat background which corresponds to considering to our 11-dimensional theory on $\CC^5 \times \RR$ where we equip $\CC^5$ with the flat holomorphic volume form. 
In this case we find a striking relationship to a certain infinite-dimensional simple super Lie algebra studied by Kac and collaborators \cite{KacBible,KacE510}.

\begin{thm}
The global symmetry algebra of the 11-dimensional theory on $\CC^5 \times \RR$ is equivalent to a central extension of the exceptional super Lie algebra $E(5,10)$. 
\end{thm}

In particular, correlation functions involving observables of the 11-dimensional theory will be constrained by the infinite-dimensional symmetry algebra $E(5,10)$. 
Given our main conjecture that the interacting 11-dimensional BV theory on $\CC^5 \times \RR$ is the twist of supergravity, we obtain the following.

\begin{conj} 
A central extension of the super Lie algebra $E(5,10)$ is a symmetry of supergravity on $\RR^{11}$ which preserves the background where the bosonic ghost takes value equal to a holomorphic supercharge $Q$.
\end{conj}

\subsection*{Relationship to other twists of supergravity} 

Motivated by the topological string, Costello and Li formulated conjectural descriptions of certain twists of 10-dimensional theories of supergravity. 
All of the descriptions center around the theory of Kodaira--Spencer gravity, otherwise known as BCOV theory. 
This theory was introduced in \cite{BCOV} as the closed-string field theory of the topological $B$-model on Calabi--Yau three-folds.
It was further extended to all Calabi--Yau manifolds in \cite{CLbcov1}. 
It is a sort of holomorphic version of gravity which at the genus zero  describes fluctuations of the complex structure of the Calabi--Yau manifold. 
We recall relevant aspects of Kodaira--Spencer gravity in \ref{s:BCOV}. 

Through the dimensional reduction of our 11-dimensional theory we will find a match with Costello and Li's descriptions of twists of 10-dimensional supergravity in terms of BCOV theory.
In Table \ref{table:IIsugra}, we provide a summary of the conjectural twists of type IIA and type IIB supergravity. 
The notation $\CC^n_B \times \RR^{2m}_A$ uses terminology from the topological string. 
It means that the twist of supergravity we are considering arises from a certain subsector of a hybrid topological string which is the $B$-model into the Calabi--Yau manifold $\CC^n$ and the $A$-model into the symplectic manifold $\RR^{2m}$. 

\begin{table}[]
\begin{tabular}{c|c|c|c|c|c|c|c|}
\cline{2-7}
& SU(5) twist & SU(4) twist & SU(3) twist & SU(2) twist & SU(1) twist & top twist  \\ \hline
\multicolumn{1}{|l|}{IIA} & \S \ref{s:su5IIA} & $\CC^4_B \times \RR^2_A$ & & $\CC^2_B \times \RR^6_A$ & $\begin{matrix} \text{perturbatively} \\ \text{trivial} \end{matrix}$ & $\RR^{10}_A$  \\ \hline
\multicolumn{1}{|l|}{IIB} & $\CC^5_B$ & & $\CC^3_B \times \RR^4_A$ & & $\CC_B \times \RR^8_A$ & \\ \hline
\end{tabular} \smallskip
\caption{A summary of twists of 10-dimensional supergravity}
\label{table:IIsugra}
\end{table}

In \S \ref{s:su4red} we will show that the reduction along $\{0\} \times \RR \times \{0\} \subset \CC^4 \times \CC \times \RR$ is equivalent to the $SU(4)$ twist of type IIA supergravity. 
The topological string approach does not lead to a description of the minimal, or holomorphic, twist of type IIA supergravity. 
In \S \ref{s:su5IIA}, we describe the reduction along the line $\{0\} \times \RR \subset \CC^5 \times \RR$ to obtain a conjectural description of the holomorphic $SU(5)$ twist of type IIA supergravity. 


\subsection*{Future work} 

\subsection*{Acknowledgements}


%\end{document}
