\documentclass{amsart}
\usepackage[utf8]{inputenc}
\usepackage{hyperref}
\let\ref\hyperref
\let\hat\widehat
\usepackage{amsfonts} %necessary for Blackboard Bold
\usepackage{amsmath} %does A LOT
\usepackage{amssymb} %necessary for things like \partial
\usepackage{amsthm} %for thms and such
\usepackage{macros}
\usepackage{tikz-cd}
\usepackage{tikz-feynman}
\usepackage{xparse}
\usepackage[centerboxes,boxsize=1ex,nobaseline,aligntableaux=center]{ytableau}
\setcounter{secnumdepth}{4}


\title{Twisted graviton spectra of $AdS_4\times S^7$ and $AdS_7\times S^4$} 
\author{Surya Raghavendran and Brian R. Williams}
\begin{document}

\maketitle

\begin{abstract} %required
\end{abstract}

\tableofcontents

Having described our eleven-dimensional model on flat spacetimes, we now pursue descriptions on maximally symmetric spacetimes. We begin by describing twisted versions of the $AdS_4\times S^7$ and $AdS_7\times S^4$ backgrounds. In eleven-dimensional supergravity, these backgrounds arise as near-horizon limits of the backgrounds sourced by some number of M2 and M5 branes in flat space respectively. In the first section of this chapter \ref{sec:ads}, we describe an analogous procedure natively in our twisted context. 

To do so, we motivate ansatzae for the leading order couplings of our eleven-dimensional model to M2 and M5 branes. These couplings determine certain curved deformations of the $L_\infty$-algebra underlying our model - we conjecture that deforming the theory in the complement of the brane by a solution to the resulting curved Maurer-Cartan equation is perturbatively equivalent to the twist of the theory on $AdS_4\times S^7$ and $AdS_7\times S^4$. 

The next two sections provide evidence for this conjecture. We begin in section \ref{sec:states} with numerical checks. We give definitions of supergravity states in our twisted $AdS_4\times S^7$ and $AdS_7\times S^4$ backgrounds, which can be thought of as particular field configurations that are localized at points on the conformal boundary of $AdS$. We compute characters of the proposed state spaces and find exact matches with counts of gravitons on $AdS_4\times S^7$ and $AdS_7\times S^4$ respectively. 

The next strand of evidence we pursue is by matching symmetries. In the physical theory, the $AdS_4\times S^7$ and $AdS_7\times S^4$ backgrounds carry actions of the 3d $\mc N=8$ and 6d $\mc N=(2,0)$ superconformal algebras. We show that our conjectural descriptions of twists of these backgrounds carry actions of the minimal twists of the corresponding superconformal algebras. 

With these pieces of evidence in hand, in sections \ref{sec:e16}, \ref{sec:e36},  we then turn to study some representation theoretic aspects of the state spaces constructed in section \ref{sec:states}. We identify certain $\C^\times$ actions on our eleven-dimensional model that combine rescalings in directions normal to branes with a certain rescaling of the space of fields - this induces a decomposition of the space of fields that we dub the \textit{graviton decomposition}. The weight $0$ parts of these decompositions are certain local $L_\infty$-algebras whose costalks recover the linearly compact super-Lie algebras $E(1|6)$ and $E(3|6)$. We thus see that these linearly compact super-Lie algebras act on the spaces of supergravity states constructed in section \ref{sec:states}. We explicate their action on nonzero weight spaces of the graviton decomposition.

In the final section of the chapter, we motivate some current work in progress that leverages the uncovered appearance of exceptional linearly compact super-Lie algebras for holographic means. Eleven-dimensional supergravity on $AdS_4\times S^7$ and $AdS_7\times S^4$ is expected to be equivalent to the large $N$ limit of the worldvolume theories of $N$ M2 branes and $N$ M5 branes respectively. 
%Typically, this definition is made in situations where the relevant boundary value problem has a unique solution, in which case one may label states by the corresponding boundary values. Moreover, one may think of such boundary values as arising from modifications of a vacuum boundary condition at a point.


\documentclass[../main.tex]{subfiles}

\newcommand\til{\widetilde}
\begin{document} 
\section{Twisted AdS space}
\label{sec:ads}

In eleven-dimensional supergravity the $AdS_7 \times S^4$ and $AdS_4\times S^7$ backgrounds are obtained by backreacting a number of M5 branes and M2 branes in flat space \cite{Maldacena:1997re,WittenAdS}. In this section, we will give an account of this procedure at the level of our twisted theory in eleven-dimensions. 
The main outcome is a geometric definition of twisted analogs of $AdS$ spaces. 
Before describing the specific examples of interest, we begin with some generalities.

Suppose we have a theory of gravity on the total space of a vector bundle. In this thesis, we are interested in holomorphic-topological field theories, and in this context, the bundle projection is a map of THF manifolds, and the gravitational theory is a local moduli problem that describes, in part, deformations of the THF structure on the total space. Operationally, producing the theory in the backreacted geometry is the output of the following two-step procedure. 

\begin{itemize}
  \item Restrict the theory on the complement of the zero section. 
  If the theory is defined on flat space $\R^d$ to begin with, and the brane lives a long a coordinate plane $\R^{n} \subset \R^d$, then this amounts to restricting the theory on $\R^d \setminus \R^n$.
  \item Deform the theory on the complement of the zero section by a certain Maurer--Cartan element, thought of as the flux sourced by branes wrapping the zero section. More rigorously, the zero section determines a certain curved Maurer-Cartan equation, and the desired Maurer-Cartan element is a solution to this equation. 
\end{itemize}

This procedure is implemented at the level of the $\Omega$-deformed nonminimal twist on flat space in the appendix of \cite{CostelloM5}, and in \cite{raghavendran2022holographic} the procedure is carried about for M5 branes in our eleven-dimensional model in some global generality. 
As another example, in \cite{CGhol} the authors show that carrying out this procedure for branes in the topological $B$-model on $\C^3$ results in the deformed conifold.
For the purposes of this thesis, we will mostly content ourselves with examples on flat space.

\subsection{Twisted supergravity}

In \cite{CLsugra}, Costello and Li gave a rigorous definition of what it means to {\em twist} supergravity, which is nicely compatible with the mathematically more familiar notion of twisting supersymmetric gauge theories \cite{WittenTwist}.
In physical terms, twisted supergravity is supergravity in a background where a bosonic ghost for supersymmetry takes a nonzero value $Q$, where $Q$ is a nilpotent supercharge.
In principle, one can take the supercharge to be any square-zero covariantly constant spinor (so it satisfies the equations of motion of supergravity).
A class of such supercharges lie in the supersymmetry algebra.

The eleven-dimensional supersymmetry algebra admits two types of twists characterized by the number of invariant directions that the supercharge determines:
\begin{itemize}
\item 
The $SU(5)$ twist. 
This twist leaves six real directions invariant and is stabilized by the subgroup $SU(5)$ of the eleven-dimensional Lorentz group.\footnote{We always work in Euclidean signature.}
\item 
The $G_2$ twist.
This twist leaves nine real directions invariant and is preserved by the subgroup $SU(2) \times G_2$ of the Lorentz group. 
\end{itemize}

In \cite{RSW} we have proposed a mathematical model for the $SU(5)$-twist of eleven-dimensional supergravity.
Many tests of this proposal were performed, including a consistency check that the dimensional reduction of the model agrees with proposals for twists of type IIA supergravity given in terms of topological strings~\cite{CLsugra}.

The eleven-dimensional theory exists on any manifold which is locally of the form 
\beqn\label{eqn:local}
\R \times Z
\eeqn
where $Z$ is a Calabi--Yau fivefold.
The model is topological along $\R$ in the sense that translations in this direction act in a homotopically trivial way on the theory.
Likewise, locally on the Calabi--Yau fivefold, all anti-holomorphic translations act in a homotopically trivial way---in this sense, the theory is {\em holomorphic} along $Z$ \cite{BWhol}.
More generally, the eleven-dimensional theory can be constructed on any eleven-manifold equipped with a transversely holomorphic foliation which is equipped with an appropriate holomorphic volume form on the leaves of the foliation.

As examples, we point out two classes of THF eleven-manifolds which we will consider in this paper, depending on whether we consider stack of M2 or M5 branes respectively:
\begin{itemize}
\item Let $M$ be a three-manifold equipped with a THF of real codimension two. 
Locally such a manifold is of the form $\R \times \C$.
Let $K_M$ be the canonical bundle of $M$ and suppose we have fixed a fourth-root $K^{1/4}$.\footnote{If $\cF$ is a THF of codimension two then $\T_M / \T_\cF \otimes_\R \C = Q \oplus \bar{Q}$ for the bundle $Q$ which is locally spanned by $\del_z$.
We let $K = Q^\vee$.}
Then we consider the THF eleven-manifold 
\beqn\label{eqn:thfm2}
\text{Tot} \left(K_M^{1/4} \otimes \C^4 \to M\right) .
\eeqn
The M2 branes will wrap the zero section of this total bundle.
Notice that the power of $1/4$ is to guarantee the Calabi--Yau condition.
\item Let $X$ be a complex threefold equipped with $K_X^{1/2}$.
We can consider the THF eleven-manifold
\beqn\label{eqn:thfm5}
\R \times \text{Tot} \left(K_X^{1/2} \otimes \C^2 \to X \right) .
\eeqn
The M5 branes will wrap the zero section and lie at a point in $\R$.
Notice that in this case the manifold is globally of the form $\R \times Z$ where $Z$ is a Calabi--Yau fivefold.
\end{itemize}


In \cite{RSW,RWindex} we describe how physical fields in supergravity are manifest in the twisted model.
We review a small part of this.
If $\til{Z}$ is an eleven-manifold equipped with a THF structure as above, the fields will live in a space of the form
\beqn
\cA^\bu(\til{Z}, \til{V})
\eeqn
where
\begin{itemize}
\item The graded bundle $\til{V}$ is holomorphic with respect to the THF structure.
\item $\cA^\bu (\til{Z}, \til{V})$ denotes the THF cohomology of $\til{Z}$ with coefficients in $\til{V}$.
\end{itemize}
To simplify discussion in the remainder of this section we will only work in the local model~\eqref{eqn:local} where $\til{Z} = \R \times Z$ for $Z$ a Calabi--Yau fivefold.
Then, the space of fields is of the form
\beqn
\Omega^\bu(\R) \otimes \Omega^{0,\bu}(Z, V)
\eeqn
where $V$ is some holomorphic vector bundle on $Z$.\footnote{The tensor product $\otimes$ is the completed tensor product with respect to natural topologies present in the space of sections of a smooth vector bundle.}
Since the theory is of topological/holomorphic nature, there is a linear part of the BRST differential of the form $\d + \dbar$ where $\d$ is the de Rham differential on $\R$ and $\dbar$ corresponds to the holomorphic structure on the bundle $V$.

Deformations of the background metric which survive the twist can be identified with Beltrami differentials, that is, sections of 
\[
\Bar{\T}_Z^* \otimes \T_Z \cong \Bar{\T}^*_Z \otimes \Bar{\T}_Z^* . 
\]
More generally, the fields of this model include sections of the holomorphic tangent bundle with coefficients in Dolbeault forms on $Z$ of arbitrary Dolbeault type and de Rham forms on $\R$, which we can write in superfield notation as 
\[
\mu = \mu^{\Bar{I}}_j (t;z,\zbar) \d \zbar_{\Bar{I}} \otimes \del_{z_j} + \mu^{\Bar{I}}_{t,j} (t;z,\zbar) \d t \d \zbar_{\Bar{I}} \otimes \del_{z_j}
\]
where the sum over the multi-index $\Bar{I}$ ranges over subsets of $\{1,\ldots,5\}$.
We view this as a section
\beqn
\mu \in \Omega^\bu(\R) \otimes \Omega^{0,\bu}(Z, \Pi \T_Z) .
\eeqn
In this notation, the component $\mu^{\Bar{I}}_j$ is {\em odd} if $|\Bar{I}|=0,2,4$ is even and {\em even} if $|I| = 1,3,5$ is odd.
The parity shift of the holomorphic tangent bundle is to ensure that deformations of the background metric which survive the twist are viewed as even fields.
In particular we have odd fields $\mu_j(t;z,\zbar) \del_{z_j}$ which play the role of ghosts for infinitesimal changes of coordinates.

Part of the linear equations of motion in our model imply that the field $\mu$ satisfies a condition that it be {\em divergence-free} for the Calabi--Yau structure on the fivefold~$Z$.
This condition is imposed, cohomologically, by adding another field which is simply a differential form
\beqn
\nu \in \Omega^\bu(\R) \otimes \Omega^{0,\bu}(Z) 
\eeqn
and a linear part of the differential $\delta \nu = \div \mu$ encoding the divergence-free condition.
Another piece of the linear equations of motion imply that it $\mu$ be constant along~$\R$. 

Ordinarily, for $\mu$ to describe a deformation of complex structure in the $Z$ direction we would require that it satisfy the Maurer--Cartan equation.
For twisted supergravity, however, we find a modification of this equation which involves the twisted analog of another familiar field---the higher form gauge field in supergravity.

In the BV formalism one will see all types of differential forms corresponding to a tower of ghosts for this higher gauge field together with antifields and antighosts.
In local coordinates, the only components of this tower of differential forms which survives the twist are
\begin{align*}
\beta & = \beta^{\Bar{I}} \d \zbar_{\Bar{I}} + \beta_t^{\Bar{J}} \d t \d \zbar_{\Bar{J}} \in \Omega^{\bu}(\R) \otimes \Omega^{0,\bu}(Z) \\
\gamma & = \gamma^{i \Bar{I}} \d z_i \d \zbar_{\Bar{I}} + \gamma_t^{j \Bar{J}} \d t \d z_j \d \zbar_{\Bar{J}} \in \Omega^{\bu} (\R) \otimes \Omega^{1,\bu}(Z).
\end{align*}
%When we want to be specific by the form type we use the notation $\beta^{k;q}(t;z,\zbar)$ for the $(k;0,q)$ component of the three-form and $\gamma^{k;q}(t;z,\zbar)$ for the $(k;1,q)$ component of the three-form.
Fields $\beta^{\Bar{I}}, \gamma_t^{i \Bar{I}}$ with $|\Bar{I}| = $odd (resp. even) are {\em even} (resp. {\em odd}) and fields $\beta_t^{\Bar{I}}, \gamma^{i \Bar{I}}$ with $|\Bar{I}| = $odd (resp. even) are {\em odd} (resp. {\em even}). 
The three-forms $\beta^{\Bar{i}\Bar{j}\Bar{k}} \d \zbar_{\Bar{i}} \d \zbar_{\Bar{j}} \d \zbar_{\Bar{k}}$, $\beta_t^{\Bar{i}\Bar{j}} \d t \ \zbar_{\Bar{i}} \d \zbar_{\Bar{j}}$, $\gamma^{i \Bar{j} \Bar{k}} \d z_i \d \zbar_{\Bar{j}} \d \zbar_{\Bar{k}}$, and $\gamma_t^{i \Bar{j}} \d t \d z_i \d \zbar_{\Bar{j}}$ comprise components of the supergravity three-form $C$ which survive the $SU(5)$ twist. 

The most important equation of motion of the eleven-dimensional theory involves both the fields $\mu$ and $\gamma$. 
When $\div \mu = 0$ it takes the form
\beqn\label{eqn:eom2}
\dbar \mu + \frac12 [\mu,\mu] = \del \gamma \del \gamma .
\eeqn
Because of the term on the right hand side this equation is not exactly the usual Beltrami equation for deformations of complex structures.
On the left hand side we are implicitly using an identification between the holomorphic tangent bundle $\T_Z$ and the bundle $\wedge^4 \T^*_Z$ granted by the holomorphic volume form on $Z$.

In \cite{RSW} we find the above equation of motion together with its gauge symmetries by formulating the theory within the BV formalism.
We briefly recount what this entails.
In the BV formalism, the space of BV fields $\cE$ is given as the sections of some graded vector bundle on spacetime. 
The grading is typically by the abelian group $\Z$, which is usually referred to as the cohomological grading.
In cohomological degree zero of $\cE$ sit the usual (sometimes called {\em physical}) fields.
The ghosts comprise the cohomological degree $-1$ part of $\cE$. 
In cohomological degree one sit the antifields, etc..
Another key piece of structure is a skew symmetric pairing $\omega_{BV}$ of cohomological degree $-1$ on $\cE$ which is concretely a pairing between the fields and antifields (and ghosts and antighosts).
The full BV action functional $S_{BV}$ is a functional on the space of fields which encodes not only the original action functional on fields, but through its dependence on ghosts and antifields it also encodes the gauge symmetries (and gauge symmetries for gauge symmetries, etc.) of the theory. 

For our eleven-dimensional model the BV action functional takes the following form
\begin{multline}
S_{BV} = \int_{\R \times Z} \bigg[\beta \wedge (\dbar + \d) \nu + \gamma \wedge (\dbar + \d) \mu +  \beta \wedge \div \mu  \bigg] \\
+ \frac{1}{2} \int_{\R \times Z}  \mu^2 \vee \del \gamma + \int_{\R \times Z} \gamma \del \gamma \del \gamma + \cdots .
\end{multline}
Here:
\begin{itemize}
\item the first line is the free part of the action.
The operator $\dbar$ is the $\dbar$-operator on $Z$, the operator $\d$ is the de Rham operator on $\R$, and the operator $\div$ is the holomorphic divergence operator on $Z$ corresponding to the holomorphic volume form.
\item We implicitly use the holomorphic volume form in the above integrals.
\item The term 
\beqn\label{eqn:J}
J(\gamma) \define \int_{\R \times Z} \gamma \del \gamma \del \gamma .
\eeqn
is the holomorphic avatar of the famous Chern--Simons term of the higher form gauge field present in supergravity.
\item The $\cdots$ indicate additional terms needed so that this functional satisfies the classical master equation (in this presentation there are, in fact, infinitely many terms).
They take a form which is reminiscent of the genus zero BCOV action for the fivefold $Z$ (we make a further comment on this below).
\end{itemize}

In what follows we will define $S_{BF}$ as
\beqn
S_{BV} = S_{BF} + J ,
\eeqn
with $J$ the Chern-Simons term \eqref{eqn:J}.
We will not need an explicit formula for $S_{BF}$ but we remark that reason for the terminology $S_{BF}$ is that this term in the action is homotopy equivalent to a standard BF type action, see \cite{RSW}.

In \cite{RSW} we have performed a number of consistency checks that this model indeed describes the $SU(5)$ twist of supergravity.
Famously, $M$-theory on a circle is expected to be equivalent to type IIA string theory where the length of the circle is proportional to the string coupling constant.
At the level of supergravity, the $S^1$-reduction of eleven-dimensional supergravity is ten-dimensional supergravity of type IIA.
Here we utilize a description of the twist of type IIA strings using a version of Kodaira--Spencer theory developed in \cite{CLbcov1,CLsugra}.

%From the point of view of the $SU(5)$ twist there are essentially two types of circles one can consider.
%The first is a `holomorphic circle'.
%This is present when we assume our Calabi--Yau fivefold is of the form $Z = \C^\times \times Y$ with $Y$ a Calabi--Yau fourfold.
%We assume that $\C^\times$ is equipped with the standard volume form $\d z$.
%The $S^1$ reduction along a circle in $\C^\times$ results in a twist of type IIA supergravity with global symmetry $SU(4)$ on the manifold
%\beqn
%\R \times \R \times Y .
%\eeqn
%More generally, this twist can be defined when $\R^2$ is replaced by a smooth oriented two-dimensional manifold $M^2$.
%In \cite{CLsugra}, Costello and Li have given a conjectural description of this particular twist of type IIA (and many other twists) using methods of topological string theory.
%In this case, the conjecture of Costello and Li is that the twist is equivalent to a mixed topological string theory which is given by the A-model along $M^2$ and the B-model along $Y$.
%In perturbation theory, this admits a description in terms of Kodaira--Spencer theory along the fourfold $Y$.
%Via an explicit dimensional reduction of our model we find an exact match with this expectation \cite{RSW}.
%The dimensional reduction of our eleven-dimensional model on
%\beqn
%\R \times \C \times Y
%\eeqn
%along $S^1 \subset \C$ is equivalent to the mixed topological A-model along $\R^2$ and B-model along $Y$ as considered in \cite{CLsugra}.
%
%The other type of reduction one can consider is that along a `topological' circle.
%This corresponds to placing our eleven-dimensional model on
%\beqn
%S^1 \times Z
%\eeqn
%and performing reduction along $S^1 \times Z \to Z$.
%What results is a theory with $SU(5)$ global symmetry.
%We conjecture that this is the $SU(5)$ twist of type IIA supergravity.
%There is no expected description of the $SU(5)$ twist in terms of topological strings like in the $SU(4)$ or $SU(2)$ cases.

%
%There are two bosonic fields of particular importance.
%First is a section
%\beqn
%\mu^{0;0,1} \in \Omega^0(\R) \otimes \Omega^{0,1}(X, \T_X) .
%\eeqn
%Here $\T_X$ is the holomorphic tangent bundle on $X$.
%The linear part of the BRST differential requires this field be constant in the real direction.
%Thus, this field has the same form as a Beltrami differential on $X$.
%There are particular field configurations where the non-linear equations of motion for $\mu$ imply that it is a complex structure deformation of $X$, but the full equations of motion are more complicated.
%Another bosonic field of importance is a section
%\beqn
%\gamma^{0;0,1} \in \Omega^0(\R) \otimes \Omega^{1,1}(X) .
%\eeqn
%
%The field $\mu$ satisfies a condition that it be {\em divergence-free} for the Calabi--Yau structure on the fivefold~$X$.
%This means that it is required to satisfy the equation
%\beqn
%\div \mu = 0
%\eeqn
%where $\div(-)$ is the divergence with respect to the holomorphic volume form $\Omega$ on $X$.

%\subsection{The $G_2$ twist}
%
%We will give a description to of the $G_2$ twist of eleven-dimensional supergravity.
%Independent derivations of this twist have appeared in \cite{CostelloM5, EagerHahner}.
%The theory is defined on any manifold of the form
%\beqn
%M \times Y
%\eeqn
%where $M$ is a manifold with $G_2$ holonomy and $Y$ is a manifold with $SU(2)$ holonomy.
%Unfortunately, the perturbative description we are about to give depends in no way on the $G_2$ structure and can be defined when $M$ is any smooth seven-dimensional manifold.
%To see the full $G_2$ structure appearing one must take into account non-perturbative effects (much like in the target space description of the topological A-model).
%
%The fundamental field is simply a differential form
%\beqn
%\alpha \in \Omega^\bu(M) \otimes \Omega^{0,\bu}(Y) [1] .
%\eeqn
%The linear BRST differential is simply $\d + \dbar$ where $\d$ is the de Rham differential on $M$ and $\dbar$ is the Dolbeault operator for $Y$.
%
%Being a manifold with $SU(2)$ holonomy, the algebra of holomorphic fnuctions on $Z$ is equipped with a holomorphic Poisson bracket $\{-,-\}$.
%Since this bracket involves only holomorphic differntial operators, it can be extended to a Poisson bracket on the dg algebra $\Omega^{0,\bu}(Y)$.
%The theory utilizes this bracket in an essential way.
%Indeed, the full BV action is
%\beqn
%\frac12 \int_{M \times Y} \Omega \, \alpha \d \alpha + \frac16 \int_{M \times Y} \Omega \, \alpha \{\alpha, \alpha\} .
%\eeqn
%where $\Omega \in \Omega^{2,0}(Y)$ is the holomorphic volume form.
%This description appeared in \cite{RYsduality} and was independently derived from a further twist of the $SU(5)$ twist in \cite{RSW}. 
%
%Like the $SU(5)$ twist, this theory is only $\Z/2$ graded.
%Nevertheless, one convenient way to think about this theory is as a higher dimensional Chern--Simons theory for a particular (dg) Lie algebra.
%Given any odd-dimensional smooth manifold $M$ and a Lie algebra $\lie{g}$ equipped with an invariant non-degenerate symmetric inner product one can consider a $\Z/2$ graded version of Chern--Simons theory whose BV action is given by the usual formula $\int_M CS(A)$.
%It is only in the case that $\dim M = 3$ that this theory admits a $\Z$-grading.
%The theory we are considering here can be interpreted as $\Z/2$ graded Chern--Simons theory on the seven-manifold $M$ for the Lie algebra of holomorphic functions on $Y$ equipped with the holomorphic Poisson bracket.
%Here, the invariant pairing is given by integrating against the holomorphic volume form $\Omega$.

\subsection{The ${\mr AdS}_4 \times S^7$ background}

In this section we introduce the analog of the ${\mr AdS}_4 \times S^7$ background in our conjectural description of the minimal twist of eleven-dimensional supergravity. 

\parsec[]

We begin by viewing the eleven-dimensional manifold
\[
\operatorname{Tot}(K^{1/4}_M \otimes \C^4\to M)
\]
as in \eqref{eqn:thfm2}.
Even when we restrict ourselves to flat space $M = \R \times \C$ this way of presenting the eleven-manifold $\R \times \C^5$ is a convenient mechanism to record weights under natural scaling actions. 
We will use $w_a$ to denote holomorphic fiber coordinates on $K^{1/4}_M \otimes \C^4$.

We carry out the above procedure. 
Consider a stack of $N$ twisted M2 branes wrapping the zero section $M$.
Our ansatz for this coupling is heuristically of the form
\[
I_{M2}(\gamma) = N\int_{M} \gamma + \cdots
\] 
which is nonzero only on the component of $\gamma$ which is a top form along $M$.
We have only indicated the lowest order coupling, the $\cdots$ indicate higher-order couplings which will be higher order in the fields of the eleven-dimensional theory and explicitly involve the fields in the worldvolume theory. 

This coupling is justified by comparison with the physical theory and by dimensional reduction. 
Indeed, as discussed in the previous section, the component of $\gamma$ which participates in the above coupling is a component of the three-form $C$-field of eleven dimensional supergravity. 
Thus, the proposal mirrors electric couplings of M2 branes in the physical theory, which simply involves integrating the $C$-field over the worldvolume of the brane. 

Moreover, reducing on a circle transverse to the M2 brane yields the $SU(4)$ twist of type IIA supergravity with $N$ $D2$ branes wrapping $M$. 
As is shown in \cite{CLsugra}, an electric coupling of D2 branes to the $SU(4)$ twist of type IIA supergravity is given by 
\[
I_{D2}(\gamma) = N \int_{M} \gamma + \cdots
\] 
where $\gamma$ now denotes the 1-form field of the $SU(4)$ twist of type IIA supergravity. It is immediate that the pullback of $I_{M2}$ along the map in the proof of proposition \ref{prop:dimred} recovers $I_{D2}$. 

\parsec[sec:m2backreact]

The backreacted geometry will be given by a solution to the equations of motion upon deforming the eleven-dimensional action by the interaction $I_{M2}(\gamma)$.
For the twisted version of $AdS$ space we should start with the three-manifold $M$ being just $\R \times \C$, so that the resulting theory is defined originally just on flat space.

Varying the deformed action with respect to $\gamma$,
we obtain the equation of motion
\begin{equation}\label{eqn:ads4eom1}
\dbar \mu + \frac12 [\mu, \mu] + \partial\gamma\partial\gamma = N \Omega^{-1} \delta_{w=0} .
\end{equation}
Here $[-,-]$ is the Schouten bracket. 
Varying $\beta$, we obtain the equation of motion
\begin{equation}\label{eqn:adseom2}
\div \mu = 0 .
\end{equation}

\begin{lem}\label{lem:m2flux}
Let
\beqn\label{eqn:FM2}
 F_{M2} = \Omega^{-1} \frac{6}{(2\pi i)^4} \frac{\sum_{a=1}^4 \bar{w}_a \d \bar{w}_1 \cdots \widehat{\d \bar{w}_a} \cdots \d \bar{w}_4}{\|w\|^{8}} \d w_1 \d w_2 \d w_3 \d w_4 .
\eeqn
Then the background where $\mu = N F_{M2}$ and $\gamma = 0$
satisfies the above equations of motion in the presence of a stack of $N$ M2 branes:
\begin{align*}
\dbar (N F_{M2}) + \frac12 [N F_{M2}, N F_{M2}] & = N \Omega^{-1} \delta_{w=0} \\
\div (N F_{M2}) & = 0  .
\end{align*}
Here we set all components of the field $\gamma$ equal to zero (as well as the fields $\nu,\beta$). 
\end{lem}

\begin{proof}
Upon specializing $\gamma = 0$, the last term in the first equation above vanishes. The equation $\dbar F_{M2} = \Omega^{-1} \delta_{w=0}$ characterizes the Bochner--Martinelli kernel representing the residue class on $\C^4 \, \setminus \, 0$. 
It is clear that $\div F_{M2} = 0$ and 
\[
[F_{M2}, F_{M2}] = 0
\] 
by simple type reasons. 
\end{proof}

We summarize the output of our computation with a definition.

\begin{defn}\label{defn:ads4}
Let $\mc E^N_{AdS_4}$ denote the classical BV theory on 
\beqn\label{eqn:totalm2}
\operatorname{Tot} (K_\C^{1/4}\otimes \C^4 \to \R\times \C_z)\setminus 0(\R\times \C)
\eeqn
given by the sheaf of cochain complexes $\mc E |_{(\R\times \C)\times (\C^4\setminus \{0\} )}$, with BV pairing induced from $\mc E$, deformed by the interaction \[S_{BF,\infty}(\mu + NF_{M2}, \nu, \beta, \gamma) + J(\gamma).\]
\end{defn}


\begin{rmk}
By definition, the theory $\cE_{AdS_4}^N$ is a deformation of our eleven-dimensional model on \eqref{eqn:totalm2} by the $N$-dependent action functional $S_{BF, \infty}(\mu = N F_{M2})$.
Upon expanding this action around $NF_{M2}$, the cubic term in $S_{BF,\infty}$ will contribute a differential which acts on $\gamma$ and $\mu$ by bracketing with $NF_{M2}$. We accordingly denote this differential $[NF_{M2}, - ] $, and we see that that linearized BRST complex underlying $\mc E_{AdS_4}$ is \[\left ( \mc E |_{(\R\times \C)\times (\C^4\setminus \{0\} ) }, \delta^{(1)} + [NF_{M2},-] \right )\] where $\delta^{(1)}$ denotes the original linearized BRST differential of the eleven-dimensional theory defined on flat space.
\end{rmk}

\begin{conj}\label{conj:ads4}
The minimal twist of eleven-dimensional supergravity on the $AdS_4\times S^7$ background with $N$ units of M2 brane flux supported on $S^7$ is perturbatively equivalent to $\mc E^N_{AdS_4}$.
\end{conj}

To verify this conjecture, we should directly twist eleven-dimensional supergravity on the $AdS_4\times S^7$ spacetime. Doing so seems difficult - while it is likely not hard to identify the covariantly constant nilpotent spinors which define the twist, it seems more difficult to establish a perturbative equivalence with our description above. A modification of the pure spinor superfield formalism to symmetric spaces such as cosets for the superconformal group might make such checks more feasible. In lieu of such, we will instead pursue other consistency checks in the following two sections{}. 

\subsection{The ${\mr AdS}_7 \times S^4$ background}

We similarly introduce an analog of the ${\mr AdS}_7 \times S^4$ background in our description of the minimal twist of eleven-dimensional supergravity.
We begin with our twisted theory defined on
\[
\R \times \operatorname{Tot}(K_{\C^3}^{1/2}\otimes \C^2 \to X)
\]
with $X$ a complex threefold as in \eqref{eqn:thfm5}.
As above, we will momentarily be only concerned with flat space which means taking $X = \C^3$.
We once again will use $w_a$ to denote holomorphic fiber coordinates on $K^{1/2}_{X}\otimes \C^2$, and we use $t$ to denote a fiber coordinate on $\R$.

\parsec[sec:m5coupling]

To repeat the procedure in the previous subsection, we begin by defining our eleven-dimensional theory couples to M5 branes (to first order).
Consider a stack of $N$ M5 branes wrapping the zero section $X$. 
We consider the heuristic nonlocal coupling 
\[
I_{M5} = N\int_{X} \div^{-1}\mu \vee \Omega +\cdots 
\]
Note that this expression is only nonzero on the component of $\mu$ in $\PV^{1,3}(X)$. 
We argue that this coupling is consistent with expectations from the physical theory and from dimensional reduction. 

The twisted field $\mu^{1,3}$ is a component of the Hodge star of the metric flux in the physical theory.
In the physical theory, M5 branes magnetically couple to the $C$-field; the coupling involves choosing a primitive for the Hodge star of the $G$-flux and integrating it over the M5 worldvolume. Our twist contains no fields corresponding to components of such a primitive; hence such a magnetic coupling is reflected in the appearance of $\div^{-1}$. 

We may once again justify this coupling by dimensional reduction to IIA supergravity. 
When $X = \C^2 \times \C$ we reduce on the circle along the last copy of $\C$ that the M5 branes wrap to get the $SU(4)$ invariant twist of type IIA supergravity on $\C^4 \times \R^2$ with $N$ $D4$ branes wrapping $\C^2 \times \R$.

In \cite{CLsugra}, it is shown that the magnetic coupling of $D4$ branes to the $SU(4)$ twist of IIA is of the form
\[
N \int _{\C^2 \times \R} \div^{-1} \mu \vee \Omega_{\C^4} + \cdots .
\]
Again, we have only explicitly indicated the first-order piece of the coupling. 

\parsec[s:m5backreact]

The backreacted geometry will be given by a solution to the equations of motion upon deforming the eleven-dimensional action by the interaction $I_{M5}(\mu)$.

Varying the potential $\div^{-1} \mu$, we obtain the following equation of motion involving the field $\gamma$:
\begin{equation}\label{eqn:m5eom1}
\dbar \del \gamma + \div \left(\frac{1}{1-\nu} \mu\right) \wedge \del \gamma = N \delta_{w_1=w_2=t=0} .
\end{equation}
Notice that there is an extra derivative compared to the equation of motion arising from varying the field $\mu$. 
This equation only depends on $\gamma$ through its field strength $\del \gamma$.

Varying $\gamma$ we obtain the equation of motion 
\begin{equation}\label{eqn:m5eom2}
(\dbar + \d_\R) \mu + \del \gamma \del \gamma = 0 .
\end{equation} 
Again, this only depends on $\gamma$ through its field strength $\del \gamma$.
It is at this point that we restrict ourselves to the case $X = \C^3$.

\begin{lem}
\label{lem:ads7flux}
Let
\beqn\label{eqn:FM2}
F_{M5} = \frac{1}{(2\pi i)^3} \frac{\bar{w}_1 \d \bar{w}_2 \wedge \d t - \bar{w}_2 \d \bar{w}_1 \wedge \d t + t \d \bar{w}_1 \wedge \d \bar{w}_2}{(\|w\|^2 + t^2)^{5/2}} \wedge \d w_1 \wedge \d w_2
\eeqn
Then, $\del\gamma = N F_{M5}$, $\mu = 0$, and $\nu = 0$ satisfies the equations of motion in the presence of a stack of $N$ M5 branes sourced by the term $N \delta_{w_1=w_2=t=0}$:
\begin{align*}
\dbar (NF_{M5}) + \d_{\R} (NF_{M5}) & = N \delta_{w_1=w_2=t=0}  \\ 
(NF_{M5}) \wedge (NF_{M5}) & = 0 .
\end{align*}
Here, we set all components of the field $\mu$ equal to zero (as well as the fields $\nu,\beta$). 
\end{lem}

\begin{proof}
The first equation,
\[
\dbar F + \d_{\R} F = N \delta_{w_1=w_2=t=0},
\]
characterizes the kernel representing $N$ times the residue class for a four-sphere in 
\[
(\C^2 \times \R) \setminus 0 \simeq S^4 \times \R .
\] 
That is
\[
\oint_{S^4} N F = N 
\]
for any four-sphere centered at $0 \in \C^2 \times \R$.

The second equation $F \wedge F = 0$ follows by simple type reasons. 
\end{proof}

Once again, we summarize the lemma above with a definition.

\begin{defn}\label{defn:ads7}
Let $\mc E^N_{AdS_7}$ denote the classical BV theory on 
\beqn
\operatorname{Tot}(\R \oplus K_{\C^3}^{1/2}\otimes \C^2 \to \C^3_z)\setminus 0(\C^3_z)
\eeqn
given by the sheaf of cochain complexes $\mc E |_{\C^3\times (\R\times \C^2\setminus \{0\} )}$, with BV pairing induced from that on $\mc E$, deformed by the interaction \[S_{BF,\infty}(\mu, \nu, \beta, \gamma +N\del^{-1} F_{M5}) + J(\gamma + N \del^{-1} F_{M5}).\]
\end{defn}

\begin{rmk}
Note that both terms in the action only depend on $\gamma$ through its holomorphic derivatives so the above expression for the action is indeed well-defined. 

As before, upon expanding the interactions around $NF_{M5}$, the cubic terms in both $S_{BF,\infty}$ and $J$ will contribute differentials. From $S_{BF,\infty}$, we get a differential which takes a $\mu$ type field to the Schouten bracket $N[F_{M5}, \mu]$ and from $J$, we get a differential which acts as $\gamma\mapsto N F_{M5} \wedge \del \gamma$. We accordingly denote this differential $[NF_{M5}, - ] $, and linear BRST complex is
\[
\left (\mc E |_{\C^3\times (\R\times \C^2\setminus \{0\} )}, \delta^{(1)} + [NF_{M5},-] \right )
\] where $\delta^{(1)}$ denotes the original linearized BRST differential.
\end{rmk}

\begin{conj}\label{conj:ads7}
The minimal twist of eleven-dimensional supergravity on the $AdS_7\times S^4$ background with $N$ units of M5 brane flux supported on $S^4$ is perturbatively equivalent to $\mc E_{AdS_7}$. 
\end{conj}

%Part of our goal in the remainder of the paper is to justify conjectures \ref{conj:ads4} and \ref{conj:ads7}.
\end{document}


%\documentclass[11pt]{amsart}
%
%%\usepackage{../macros-master}
%\usepackage{macros-fivebrane}
%
%\begin{document}

\section{Twisted supergravity states}
\label{sec:states}

The first entry of the AdS/CFT dictionary in traditional treatments is a matching between \textit{supergravity states} and local operators in the CFT. 
The goal of this section is to provide constructions of spaces of twisted supergravity states in our eleven-dimensional model, via geometric quantization. The state spaces on ${\rm AdS}_{7}\times S^{4}$ and ${\rm AdS}_{4}\times S^{7}$ have a remarkable property---they are naturally modules for certain infinite-dimensional exceptional super Lie algebras. We conclude the section by computing characters for these modules and comparing them with large $N$ indices for fivebranes and membranes in the literature.

Before proceeding with the construction, let us first give some feel for the situation we hope to describe. Suppose we consider a gravitational theory on $AdS_{d+1}\times S^{d^{\prime}}$, which we compactify to view as a theory on $AdS_{d+1}$ with all Kaluza-Klein harmonics included. Let $M^{d}$ denote the conformal boundary of $AdS_{d+1}$. A supergravity state is traditionally defined to be a solution to linearized equations of motion with a given boundary value \cite{}. Typically, this definition is made in situations where the relevant boundary value problem has a unique solution, in which case one may label states by the corresponding boundary values. Moreover, one may think of such boundary values as arising from modifications of a vacuum boundary condition at a point.


\subsection{Twisted Backreactions}
We begin by describing the relevant backgrounds. In eleven-dimensional supergravity, the $AdS_7 \times S^4$ and $AdS_{4}\times S^{7}$ backgrounds are obtained by backreacting a number of fivebranes and membranes respectively in flat space \cite{Maldacena:1997re,WittenAdS}.
In \cite{RSW} we gave descriptions of twisted versions of these backgrounds. We will recall this construction, adapted to a slightly more global situation than is considered in \cite{RSW}.

We will consider the eleven-dimensional theory on eleven-manifolds that arise as total spaces of vector bundles. Placing the theory in the backreacted geometry is a 3-step procedure:

\begin{itemize}
  \item Place the eleven-dimensional theory on the complement of the zero section. To do so, we will wish to describe the complement of the zero-section in a way that facilitates natural operations on holomorphic-topological local $L_{\infty}$-algebras.

  \item Deform the theory on the complement of the zero section by a certain Maurer--Cartan element.
  The Maurer--Cartan element is thought of as the flux sourced by branes wrapping the zero section.
\end{itemize}

\parsec[s:brkevin]
As a way to highlight the key aspects of the construction, we detail the ingredients in the simplified model of Costello's twisted $M$ theory. The relevant local calculation can be found in the appendix of \cite{}; our goal here is to simply identify the salient global features that allow one to reduce to said local calculation.

We consider the theory on $X = \text{Tot} (\R\oplus K_{C})$, with some number of twisted `fivebranes' wrapping the zero section
\[
0 \times C \subset \R \times \T^* C .
\]
Denote by $t$ the real coordinate and by $w$ the fiber coordinate in $\T^* C$. We wish to describe the complement of the zero section $M = X - 0 \times C$.

Note that the bundle $\R\oplus K_{C}$ is equipped with a partially flat connection - this data equips the total space $X$ with the data of a transversely holomorphic foliation (THF) \cite{DuchampKalka}. 

If we choose a fiberwise partially hermitian metric on the bundle $\R \oplus K_C$ we obtain a projection $p: \R \times \T^*C \to \R_{+} \times C$ which combines the fiberwise norm with the natural bundle projection. The restriction $p|M$ equips $M$ with the structure of an $S^{2}$-bundle over $\R_{>0}\times C$. Moreover, the partial flat connection on $\R\oplus K_{C}$ induces a partially flat connection on $M$. As part of this data, each of the fiber spheres is equipped with a complex foliation of rank 1.

Compactification amounts to pushing forward a local $L_{\infty}$-algebra along $p|M$. The result is a theory with infinitely many Kaluza--Klein modes along the fiber spheres. In the holomorphic-topological setting, the Kaluza-Klein modes will be modeled by a variant of Cauchy-Riemann cohomology.

Moreover, including the flux sourced by the brane deforms this structure. The lowest lying Kaluza-Klein modes in the deformed theory are equivalent to 3d Chern-Simons.

For sake of analogy, we think of the resulting deformation as being a twisted version of $AdS_3 \times S^2$. \footnote{It is an interesting question if this corresponds the actual twist of a five--dimensional supersymmetric background of this form.}
We proceed to describe the twisted version of states at the boundary of this version of $AdS$.
We first proceed before turning on the flux sourced by the brane.

The theory admits a natural `vacuum' boundary condition at $r=0$.
In local coordinates, these are fields $\alpha(t,z,w)$ on the complement to the brane which extend to regular functions along the brane.

The `supergravity states' are, by definition, fields which satisfy the linearized equations of motion and satisfy the vacuum boundary condition except at a single point.
The linearized equations of motion are simply $(\d_{dR} + \dbar) \alpha = 0$.
Thus, up to equivalence, all solutions to the linearized equations of motion are constant in the real variable $t$, and holomorphic in $z,w$.

Modifications of the boundary condition at the point~$z = 0$ on the boundary take the form
\[
\alpha = f(w) \delta^{(r)}_{z=0}
\]
where $f$ is some holomorphic function.
Here $\delta^{(r)}_{z=0}$ denotes the $r$th derivative of the $\delta$-function at $z=0$.
It is convenient to parameterize such boundary modifications algebraically by expressions of the form
\[
\alpha_{k,r} = w^k \delta^{(r)}_{z=0} .
\]
Linear combinations of such states form a dense subspace of all possible modifications at the boundary.

The reason that the boundary modifications take this form can be seen by understanding in more explicit terms the vacuum boundary condition.
The phase space at the boundary $C$ can be identified with the following cohomology
\[
\Omega^{0,\bu}(C) \otimes \cA^{0;\bu}(\R \times \C - 0) [1]
\]
where $\cA^{0;\bu}$ denotes the mixed de Rham--Dolbeault cohomology of $\R \times \C - 0$ as a manifold equipped with a transversely holomorphic foliation \cite{DuchampKalka}.
We refer to the section below for a reminder on this geometric structure.

The phase space is equipped with a natural symplectic form given by
\[
\int_C \d z \oint_{S^2} \d w \, \alpha \wedge \alpha' .
\]
There is a natural Lagrangian inside of the phase space which consists of linear combinations of elements $\alpha(z) \otimes f(t,w)$ where $\alpha(z) \in \Omega^{0,\bu}(C)$ and $f(t,w)$ is a smooth function on $\R \times \C - 0$ which extends to zero.
The linearized equations of motion simply say that $\alpha$ is holomorphic, $f$ is independent of $t$ and depends holomorphically on $w$

\parsec[s:brfive]

We now consider the situation of backreacting some number of (twisted) fivebranes in our eleven-dimensional model.
Let $Z$ be a three-fold that the fivebranes wrap.
We also fix a rank 2 holomorphic vector bundle $V\to Z$ such that $\wedge^{2} V \cong K_{Z}$;
this condition ensures that the total space of $V$ is a Calabi-Yau five-fold. In the main body of the paper we will choose $V$ to be the bundle $K_{Z}^{1/2}\otimes \C^{2}$.

Consider the bundle $\R\oplus V$; this bundle has a canonical partially flat connection. We wish to consider our eleven dimensional model on $X = Tot (\R\oplus V)$ which is the total space of the \textit{real} rank five bundle $\R\oplus V$ over $Z$. The partially flat connection on $\R\oplus V$ equips $X$ with a canonical THF structure $F_{X}\subset T_{X}$.

We place a stack of $N$ fivebranes wrapping the zero section in $\R\oplus V$.
Denote the complement of the zero section by
\[
M_V = \text{Tot}(\R\oplus V) - 0(Z).
\]
Notice that in \S \ref{s:Lsugra} we have only defined the sheaf of $L_\infty$ algebras $\cL_{sugra}$ on a product of a smooth one-manifold times a Calabi--Yau five-fold.
The eleven-manifold $M_V$ is not of this form, nevertheless there is a generalization of $\cL_{sugra}$ which one can define using the natural geometric structure present in our situation.

A transversely holomorphic foliation (THF) on a smooth manifold $M$ is an integrable subbundle $F \subset \T_M \otimes \C$ such that $F + \Bar{F} = \T_M \otimes \C$.
We will say that $F$ equips $M$ with the a THF structure.
%Suppose $M$ is a manifold equipped with a THF structure and let $\cF$ be the corresponding foliation of even codimension.
The product $M = S \times X$, where $X$ is a complex manifold and $S$ is a smooth manifold has a natural THF structure with $F$ the restriction of the tangent bundle of $N$ along the projection.
Locally, any THF manifold is split of the form $\R^d \times \C^n$, whose coordinates we will denote by $(x_i ;  z_j)$.
The bundle $F$ is locally spanned by the vector fields $\partial / \partial x_i$'s and $\del/\del \zbar_j$'s.
(Notice that when $F \cap \Bar{F} = 0$ we are just describing an ordinary complex structure on $M$.)

Any submanifold of a THF manifold is itself a THF manifold.
We are most interested in the submanifold $M_V \subset {\rm Tot}(\R \oplus V) = \R \times X$ where we equip $\R \times X$ with its standard split THF structure.

We have expressed the fields of the eleven-dimensional theory in terms of a mixed type of de Rham and Dolbeault cohomology.
Let us focus on the fields $\beta,\gamma$ which on $\R \times X$ combine to form the complex
\beqn\label{eqn:drdol}
\Omega^{\bu}(\R) \otimes \Omega^{0,\bu}(X) \xto{1 \otimes \del} \Omega^{\bu}(\R) \otimes \Omega^{1,\bu}(X) .
\eeqn
As usual, we leave the $\d_{dR}$ and $\dbar$ operators implicit.
More generally, there is a natural cohomology associated to a THF structure.
Suppose $(M,F)$ is a THF structure and
denote by $Q$ the (complex) quotient bundle $\T_\C M / F$.
For each $p,q$ denote by $\cA^{p;q}$ smooth sections of the bundle $\wedge^p Q^\vee \otimes \wedge^q F^\vee$.
The derivative along the leaves of the foliation defined by $V$ defines a map
\[
\thfd \colon \cA^{p;q} \to \cA^{p;q+1}  .
\]
By integrability one has $\thfd^2 = \thfd \circ \thfd = 0$ and so $\thfd$ equips $\cA^{p;\bu} = \oplus_q \cA^{p;q}[-q]$ with the structure of a cochain complex for each $p$.
Locally in a split THF structure the operator $D$ is of the form $\d_{dR} + \dbar$ where $\d_{dR}$ is the de Rham differential along $\R^d$ and $\dbar$ is the Dolbeault operator along $\C^n$.
There is also an analog of the holomorphic $\del$ operator which takes the form $\thfdel \colon \cA^{p;q} \to \cA^{p+1;q}$.
The obvious exterior product $\cA^{p;q} \times \cA^{r;s} \to \cA^{p+r;q+s}$ further endows
\[
\left(\cA^{\bu;\bu} (M), \thfd + \thfdel\right) = \left(\oplus_p \cA^{p;\bu}[-p] , \thfd + \thfdel \right)
\]
with the structure of a bigraded commutative dg algebra.
This complex is simply isomorphic to the de Rham complex of $M$, but this presentation lends itself to more interesting quotient complexes.
For example, the forms of type $(p,\bu)$ with $p \geq 2$ form an ideal inside of this dg algebra; hence we get a quotient dg algebra
\beqn\label{thfcoh1}
\left(\cA^{\leq 1;\bu}(M), \thfd + \thfdel\right) = \quad \cA^{0;\bu} \xto{\thfdel} \cA^{1;\bu} .
\eeqn
We leave the $\thfd$ operator implicit in the presentation on the right hand side.
When $M = M_V$, it is this complex that is the THF generalization of the truncated de Rham--Dolbeault complex in \eqref{eqn:drdol}---it is easy to see that it agrees with this complex in the case of a split THF manifold.
There is a similar THF description for the fields $\mu,\nu$ in the eleven-dimensional theory.

%Note that the eleven-manifold $\R \times X$ is equipped with a natural transverseley--holomorphic foliation (THF)---the complexified tangent bundle decomposes as $T_{\R}\oplus T_{Z}\oplus \Bar{T}_{Z}$.
With this THF cohomological description of the eleven-dimensional theory in place we proceed to describe the boundary condition obtained by removing the location of the branes.
We may choose fiber coordinates of the bundle $t, w_{1}, w_{2}$ of $\R \oplus V$ over $Z$ and a fiberwise partially hermitian metric.
Explicitly, the corresponding norm defines a map
\begin{align*}
 h \colon  X & \to \R_{+} \\
  (t, w_{i}, \bar{w_{i}}, p)& \mapsto t^{2} + |w_{1}|^{2}+|w_{2}|^{2}
\end{align*}
Letting $\pi \colon X \to Z$ be the natural projection, we obtain the $S^{4}$ bundle
\[
p \define (h,\pi) \colon \R \times X \to \R_{+}\times Z
\]
which restricts to an $S^4$ bundle $p|M \colon M \to \R_{>0} \times Z$.
These embeddings and projections fit inside of the following commutative diagram
\[
\begin{tikzcd}
M \ar[d,"p|M"'] \ar[r,hook] & X \ar[d,"p"] & \ar[l,hook',"0"'] Z \ar[d,"="] \\
\R_{>0} \times Z \ar[r,hook] & \R_{+} \times Z & \ar[l,hook',"0 \times \id"] Z.
\end{tikzcd}
\]
The inclusions on the left are the natural embeddings.
The top right inclusion is the zero section of ${\rm Tot}(\R \oplus V) = X$ and the bottom right inclusion is the embedding at radius $r = 0$.

As we just elaborated, the eleven-dimensional theory is defined on the THF manifold $M$---in the BV formalism this is encoded, in part, by the sheaf of $L_\infty$ algebras $\cL_{sugra}$ on $M$.
Compactification of this theory along the $S^4$ link corresponds to pushing forward this sheaf along $p|M$.
The resulting sheaf of $L_\infty$ algebras $(p|M)_*\cL_{sugra}$ describes, in the BV formalism, the compactified theory on the seven-manifold $\R_{>0} \times Z$.

The theory on $M$ extends to a theory on the manifold obtained by filling in the zero section of $\R \times V$; in other words, we know that the theory is defined on the entire space $\R \times X$.
This means that there is a natural way to extend the theory on $\R_{>0} \times Z$ to the seven-manifold with boundary $\R_{+} \times Z$.
The restriction of this theory to the six-dimensional boundary plays the most important role for us.

\brian{trying to incorporate below}
%To do this we will make use of the natural foliated geometric structures which we have around.

Recall that we have a THF structure on $X$ induced from a partially flat conneciton on $\R\oplus V$; this is codified by saying that there is a splitting of the exact sequence

\[
0 \to \ker \to F_{X}\to \pi^{*}T^{1,0}_{Z}\to 0
.\]

Both $F_{X}$ and $\pi^{*}T^{1,0}_{Z}$ are involutive, and the flatness of the connection implies that that the splitting preserves the lie brackets on sections.

Consider the relevant tangent sequence of the map


The fibers of the composition $V_{M}\to M\to \R_{+}\times Z$ are copies of the tangent bundle of $S^{4}$, and the corresponding fibers of $V_{M}$ are subbundles of $TS^{4}$ that equip the fiber 4-spheres with a generalized Cauchy-Riemann structure. \surya{CITE}

 The underlying sheaf of cochain complexes is given by

\[
\Omega^{\bullet}(\R_{+})\otimes \left ( \begin{tikzcd}
\ul{\rm even} & \ul{\rm odd} \\
\PV^{1,\bu}(Z) \ar[r, "\del_{\Omega}"] & \PV^{0,\bu}(Z)\\
\Omega^{0,\bu}(Z) & \Omega^{1,\bu}(Z) \ar[l, "\del"]
\end{tikzcd}
 \right ) \otimes CR (S^{4}).
\]


Here $CR (S^{4})$ denotes the cohomology of the tangential Cauchy-Riemann complex of $S^{4}$ \surya{CITE}, equipped with the above Cauchy-Riemann structure. Its computation is facilitated by the following lemma:

\begin{lem}
  Let $\R^{d}\times \C^{n}$ be an affine THF manifold, and choose a partial hermitian metric. Let $S^{d+2n-1}$ denote the corresponding unit sphere, equipped with its standard generalized Cauchy-Riemann structure. Then there is a quasi-isomorphism

  \[CR (S^{n+2d-1})\cong \cA^{\bu;\bu}\left ( (\R^{d}\times \C^{n})\setminus 0 \right )\]

  where the right-hand-side denotes the Dolbeault-deRham complex.
\end{lem}

The cohomology of the Dolbeault-deRham complex of $\R\times \C^{2}$ is easy to describe.


It was argued in \cite{RSW} that to leading order the coupling of a stack of twisted fivebranes to the eleven-dimensional theory is given by the nonlocal interaction
\beqn\label{eqn:br1}
I_{M5} = N\int_{Z} \div^{-1}\mu \vee \Omega +\cdots
\eeqn
where $\mu \in \Omega^0 (\R) \hotimes \PV^{1,3}(X)$ is a component of a field in the eleven-dimensional theory which satisfies $\div \mu = 0$.

\parsec
Let $C$ be a curve, and let $V\to C$ be a rank 4-holomorphic vector bundle over $C$ such that $\wedge^{4} V = K_{C}$. This condition again ensures that $X = {\rm Tot} V$ is a Calabi-Yau five-fold - in the main body of the paper, we will take $V = K^{1/2}_{C}\otimes \C^{4}$. Abusively letting $V$ also denote its pullback along the canonical projection $\R\times C \to C$, we may view $\R\times X$ as the total space of $V$ on $\R\times C$. As before we will consider wrapping a stack of $N$ membranes along the zero section.

Since $V$ is a complex vector bundle, we may choose a fiberwise hermitian metric, and as before, we may view $\R\times X \setminus \R\times C$ as an $S^{7}$-bundle over $\R_{>0}\times \R\times C$.



%\parsec[s:sugraops]
%
%By the usual methods of the BV formalism the action functional $S_{sugra}$ described above endows the parity shift of the fields $\cL_{sugra} = \Pi \cF_{sugra}$ with the structure of a holomorphic-topological local $\Z/2$ graded $L_\infty$ algebra. 
%
%On $\C^5 \times \R$ we can describe this super Lie algebra structure explicitly. 
%First, by the Dolbeault and de Rham Poincar\'e lemmas it is easy that the even part of the super Lie algebra $\cL(\C^5 \times \R)$ is equivalent to a one-dimensional central summand $\C$ plus the Lie algebra of divergence-free vector fields on $\C^5$:
%\[
%\Vect_0 (\C^5) = \{X \in \Vect(\C^5) \; | \; \div X = 0\} .
%\]
%The odd part of the super Lie algebra $\cL(\C^5 \times \R)$ is equivalent to the space of holomorphic one-forms on $\C^5$ modulo exact one-forms
%\[
%\Omega^{1,hol}(\C^5) / {\rm Im}(\del) 
%\]
%which is, of course, equivalent to the space of closed holomorphic two-forms $\Omega^{2,hol}_{cl}(\C^5)$. 
%
%\begin{thm}[\cite{RSW}[Theorem 2.1]]
%The Taylor expansion map determines a map of $\Z/2$ graded $L_\infty$ algebras
%\[
%j_\infty \colon \cL_{sugra}(\C^5 \times \R) \to L_{sugra} .
%\]
%Furthermore, $L_{sugra}$ is equivalent as a $\Z/2$ graded $L_\infty$ algebra to $\Hat{E(5|10)}$. 
%\end{thm} 
%
%As an immediate corollary of this result we obtain by Lemma \ref{lem:localops} the following.
%
%\begin{cor}
%\label{cor:sugraops}
%Let $\Obs_{sugra}$ be the factorization algebra on $\C^5 \times \R$ of classical observables of the minimal twist of eleven-dimensional supergravity.
%There is a quasi-isomorphism of commutative dg algebras
%\[
%\Obs_{sugra} (0) \simeq \clie^\bu \left( \Hat{E(5|10)} \right) .
%\]
%\end{cor}

\subsection{Global symmetry for twisted $AdS$}
\label{s:global1}

After complexification, the~six-dimensional and three-dimensional superconformal algebras are isomorphic to $\lie{osp}(8|4)$.
The even part of this algebra is $\lie{so}(8) \times \lie{sp}(4)$.
This algebra contains the six-dimensional $\cN=(2,0)$ supersymmetry algebra whose odd part is four copies of $S^{6d}_+$, the positive irreducible complex spin representation of $\lie{so}(6)$.
It also contains the three-dimensional $\cN=8$ supersymmetry algebra whose odd part is eight copies of $S^{3d}$, the irreducible complex spin representation of $\lie{so}(3)$. 

In the six-dimensional case, the holomorphic supercharge is a supertranslation 
\[
Q \in \Pi S^{6d}_+ \otimes \C^4 \subset \lie{osp}(8|4)
\]
which is characterized (up to equivalence) by the properties that $Q^2 = 0$ and that its image
\[
{\rm Im}\left(Q|_{\Pi S_+ \otimes \C^4} \right) \subset \R^6 \otimes_\R \C \cong \C^6
\]
is three-dimensional (spanned by the anti-holomorphic translations). 
The supercharge $Q$ acts on $\lie{osp}(8|4)$ by commutator and the resulting cohomology will automatically act on the holomorphic twist of any six-dimensional superconformal field theory. 
This cohomology can readily be identified with the subalgebra $\lie{osp}(6|2)$, see \cite{SWe36}. 

Similarly, the minimal twisting supercharge in the three-dimensional $\cN=8$ supersymmetry algebra is an element $Q \in \Pi S^{3d} \otimes \C^8$ which is characterized (up to equivalence) by the property that $Q^2 = 0$ and that the image of $[Q,-]$ is two-dimensional. 
The cohomology of $\lie{osp}(8|4)$ with respect to this supercharge is also isomorphic to~$\lie{osp}(6|2)$. 

In the untwisted situation, the symmetry of solutions to eleven-dimensional supergravity in the presence of fivebranes wrapping a six-dimensional affine subspace is exactly the superconformal algebra $\lie{osp}(8|4)$ (after complexification). 
Geometrically, this background is $AdS_7 \times S^4$. 
Similarly, for backreacting membranes the background is $AdS_4 \times S^7$. 
In \cite{RSW} we have proposed a twisted analog of the $AdS$ background and have shown that solutions to equations of motion of our eleven-dimensional theory in the presence of twisted fivebranes and membranes contains the symmetry algebra $\lie{osp}(6|2)$---which is precisely the twists of the superconformal algebras we just discussed.

We will count the single-particle gravitational states in our eleven-dimensional model for the geometry resulting from backreacting twisted fivebranes and membranes.
As recalled above, such states have a symmetry by the twisted superconformal algebra $\lie{osp}(6|2)$.
We will enumerate states via choosing a Cartan in the bosonic subalgebra of the twisted superconformal algebra. Let us recall how the bosonic subalgebra embeds as symmetries of the eleven-dimensional theory in this twisted background.
This manifests as an embedding of this bosonic subalgebra into the ghosts of the eleven-dimensional theory. 
The embedding is distinct for fivebranes and membranes.
We turn first to the fivebrane case. 

\subsection{Supergravity states for twisted $AdS_7$}

For convenience we choose coordinates on the eleven manifold as
\[
\R \times \C^5 = \R_t \times \C^2_w \times \C_z^3 
\]
with $z = (z_i), i=1,2,3$ and $w = (w_a), a=1,2$.
The stack of fivebranes wrap $w_1=w_2=t=0$. 
Important for us is to recall that part of the ghost system for our eleven-dimensional theory consists of divergence-free vector fields on $\C^5$ which are locally constant along $\R$. 

\begin{itemize}
\item
The subalgebra $\lie{sl}(3)$ embeds as vector fields
\beqn
\sum_{ij} A_{ij} z_i \frac{\del}{\del z_j} \in \PV^{1,0}(\C^5)\otimes \Omega^0(\R) , \quad (A_{ij}) \in \lie{sl}(3) .
\eeqn
By definition, these vector fields are automatically divergence-free.

\item
        The generator of the subalgebra $\lie{gl}(1)$ is mapped to the element
        \beqn
        Y = \sum_{i=1}^3 z_i\frac{\del}{\del z_i} - \frac 32\sum_{a=1}^2 w_a\frac{\del}{del w_a}\in \PV^{1,0}(\C^5)\otimes \Omega^0 (\R).
        \eeqn
    Notice that this vector field is divergence-free and restricts to the Euler vector field along $t=w_{a} = 0$.
\item 
The subalgebra $\lie{sl}(2)$ ($R$-symmetry) is mapped to the triple
\beqn
 w_1 \frac{\del}{\del w_2}, w_2 \frac{\del}{\del 1}, \frac{1}{2}\left (w_1\frac{\del}{\del w_1}-w_2\frac{\del}{\del w_2}) \in \PV^{1,0}(\C^5) \otimes \Omega^0(\R) .
\eeqn
\end{itemize}

%\begin{rmk}
%In the classification of simple super Lie algebras, Kac makes use of a weight grading $\oplus_{j \geq -2} \fg_j$ of the exceptional Lie algebra $E(3|6)$ for which the finite-dimensional subalgebra above is the weight zero piece
%\cite{KacClass}.
%We will make use of this grading in \S \ref{s:kr}.
%\end{rmk}

The dimension of a Cartan subalgebra of $\lie{sl}(3) \times \lie{sl}(2) \times \lie{gl}(1)$ is four and accordingly, the equivariant character we study has four fugacities.
We choose these explicitly as follows:
\begin{itemize}
  \item $t_{1}, t_{2}$ denote generators for the Cartan of $\lie{sl}(3)$ which is generated by the vector fields
  \beqn
  h_1 = z_1 \frac{\del}{\del {z_1}} - z_2 \frac{\del}{\del{z_2}} , \quad h_2 = z_2 \frac{\del}{\del{z_2}} - z_3 \frac{\del}{\del{z_3}}.
  \eeqn
   \item $q$ denotes a generator for the Cartan of the~$\lie{gl}(1)$ which is generated by the element $Y$ from equation~$\eqref{eqn:Y}$. 
  \item $r$ denotes a generator for the Cartan of a $\lie{sl}(2)$ which is generated by the element 
  \beqn
  h = \frac12 \left(w_1 \frac{\del}{\del w_1} - w_2 \frac{\del}{\del w_2}\right) .
  \eeqn
\end{itemize}

The twisted supergravity states $\cH_{sugra}^{6d}$ form a representation for $\lie{osp}(6|2)$. 
The weights of twisted supergravity states with respect to the generators of the Cartan subalgebra above are completely determined by the weights of the holomorphic coordinates on $\C^2_w \times \C^3_z$.
These are summarized in table \ref{tbl:sugraM5}.

\begin{table}
\begin{center}
\begin{tabular}{c c c c c c}
  & $z_{1}$ & $z_{2}$ & $z_{3}$ & $w_{1}$ & $w_{2}$ \\
  \hline
  $t_{1}$ & $1$ & 0 & $-1$ & 0 & 0 \\
  $t_{2}$ & 0 & 1 & $-1$ & 0 & 0 \\
  $r$ & 0 & 0 & 0 & 1 & $-1$ \\
  $q$ & $-1$ & $-1$ & $-1$ & $\frac{3}{2}$ & $\frac{3}{2}$
\end{tabular}
\caption{Fugacities for the fields of the holomorphic twist of eleven-dimensional supergravity for the geometry $\R \times \C^5 \setminus \C^3$.}
\label{tbl:sugraM5}
\end{center}
\end{table}

We enumerate single particle supergravity states via computing the super trace of the operator $q^Y t_1^{h_1} t_2^{h_2} r^h$ acting on $\cH^{6d}_{sugra}$:
\beqn
f^{6d}_{sugra}(q,t_1,t_2,r) = \Tr_{\cH_{sugra}^{6d}} (-1)^F q^Y t_1^{h_1} t_2^{h_2} r^h .
\eeqn
The super trace means that there is an extra factor of $(-1)^F$, where $F$ is parity (fermion number), when computing the ordinary trace. 
That is, we compute the expression


\begin{prop}
The single particle index of the space of twisted supergravity states $\cH_{sugra}^{6d}$ is given by the following expression
\beqn
f_{sugra}^{6d} (q, t_{1}, t_{2}, r) = \frac{q^4(t_1^{-1}+t_1t_2^{-1}+t_2)-q^2(t_1+t_1^{-1}t_2+t_2^{-1})+(q^{3/2}-q^{9/2})(r+r^{-1})}{(1-t_{1}^{-1}q)(1-t_{2}q)(1-t_{1}t_{2}^{-1}q)(1-rq^{3/2})(1-r^{-1}q^{3/2})}.
\eeqn
\end{prop}

We record a few specializations of this index which we will remark on further in \S \ref{s:??}.
\parsec 
The specialization of this index $q=r^2, t_2=1$ in \eqref{eqn:special1} yields the plethystic exponential of the following single particle index
\[
f_{sugra}^{6d}(q, t_1, t_2=1, r = q^{1/2}) = \frac{q}{(1-q)^2}
\]

This plethystic exponential yields the Macmahon function, which is the character of the vacuum module of the $W_{1+\infty}$-algebra.

\parsec

The specialization $t_1=t_2=r=1$ yields the single particle index
\[
f_{sugra}^{6d} (q, t_1=t_2=r=1) = \frac{3 q^4 - 3 q^2 + 2 q^{3/2} - 2 q^{9/2}}{(1-q)^3 (1-q^{3/2})^2} .
\]

\parsec The same change of variables in \eqref{eqn:special2} agrees with previously computed indices for single particle states for supergravity on $AdS_{7}\times S^{4}$ \surya{...} \brian{not sure where this was supposed to go?}

\subsection{Supergravity states for twisted $AdS_4$}
For convenience we choose coordinates on the eleven manifold as
\[
\R \times \C^5 = \R_t \times \C^4_w \times \C_z^
\]
with $w = (w_a), a=1,2,3,4$.
The stack of membranes wrap $w_1=w_2=w_{3}=w_{4} = 0$.

\begin{itemize}
\item
The subalgebra $\lie{sl}(4)$ ($R$-symmetry) embeds as vector fields
\beqn
\sum_{ab} A_{ab} w_a \frac{\del}{\del w_b} \in \PV^{1,0}(\C^5)\otimes \Omega^0(\R) , \quad (A_{ab}) \in \lie{sl}(4) .
\eeqn
By definition, these vector fields are automatically divergence-free.

\item
The subalgebra $\lie{sl}(2)$ ($R$-symmetry) is mapped to the triple
\beqn
 \frac{\del}{\del z}, z \frac{\del}{\del z}-\frac{1}{4}\sum_{a=1}^4 w_a\frac{\del}{\del w_a}, \frac12 \left(w_1 \frac{\del}{\del w_1} - w_2 \frac{\del}{\del w_2}\right) \in \PV^{1,0}(\C^5) \otimes \Omega^0(\R) .
\eeqn
\end{itemize}

%\begin{rmk}
%In the classification of simple super Lie algebras, Kac makes use of a weight grading $\oplus_{j \geq -2} \fg_j$ of the exceptional Lie algebra $E(3|6)$ for which the finite-dimensional subalgebra above is the weight zero piece
%\cite{KacClass}.
%We will make use of this grading in \S \ref{s:kr}.
%\end{rmk}
The equivariant character again has four fugacities, which we explicitly choose as follows:
\begin{itemize}
  \item $t_{1}, t_{2}, t_{3}$ denote generators for the Cartan of $\lie{sl}(4)$ which is generated by the vector fields
  \beqn
  h_1 = w_1 \frac{\del}{\del {w_1}} - w_4 \frac{\del}{\del{w_4}} , \quad h_2 = w_2 \frac{\del}{\del{w_2}} - w_4 \frac{\del}{\del{w_4}} , \quad h_3 = w_3\frac{\del}{\del w_3}-w_4\frac{\del}{\del w_4}.
  \eeqn
  \item $q$ denotes a generator for the Cartan of the $\lie{sl}(2)$ which is generated by the vector field
        \beqn
        T =  z \frac{\del}{\del z}-\frac{1}{4}\sum_{a=1}^4 w_a\frac{\del}{\del w_a}
        \eeqn
\end{itemize}


Once again, the twisted spergravity states $\cH_{sugra}^{3d}$ form a representation for $\lie{osp}(6|2)$.
The weights of twisted supergravity states with respect to the generators of the Cartan subalgebra above are completely determined by the weights of the holomorphic coordinates on $\C^4_w \times \C_z$.
These are summarized in table \ref{tbl:sugraM5}.

\begin{table}
\begin{center}
\begin{tabular}{c c c c c c}
  & $z$ & $w_{1}$ & $w_{2}$ & $w_{3}$ & $w_{4}$ \\
  \hline
  $t_{1}$ & 0 & 1 & 0 & 0 & $-1$ \\
  $t_{2}$ & 0 & 0 & 1 & 0 & $-1$ \\
  $t_{3}$ & 0 & 0 & 0 & 1 & $-1$ \\
  $q$ & $-1$ & $\frac 14$ & $\frac 14$ & $\frac{1}{4}$ & $\frac{1}{4}$
\end{tabular}
\caption{Fugacities for the fields of the holomorphic twist of eleven-dimensional supergravity for the geometry $\R \times \C^5 \setminus \R\times \C$.}
\label{tbl:sugraM5}
\end{center}
\end{table}

We enumerate single particle supergravity states via computing the super trace of the operator $q^T t_1^{h_1} t_2^{h_2} r^h$ acting on $\cH^{3d}_{sugra}$:
\beqn
f^{3d}_{sugra}(q,t_1,t_2,t_3) = \Tr_{\cH_{sugra}^{3d}} (-1)^F q^T t_1^{h_1} t_2^{h_2} t_3^{h_3} .
\eeqn
The super trace means that there is an extra factor of $(-1)^F$, where $F$ is parity (fermion number), when computing the ordinary trace.
That is, we compute the expression


\begin{prop}
The single particle index of the space of twisted supergravity states $\cH_{sugra}^{3d}$ is given by the following expression
\beqn
f_{sugra}^{3d} (q, t_{1}, t_{2}, t_3) = \frac{\left(\begin{aligned}
        & -q^{-7/4}(t_{1}^{-1}+t_{2}^{-1}+t_{3}^{-1}+t_{1}t_{2}t_{3}) +q^{1/4}(t_{1}+t_{2}+t_{3}+t_{1}^{-1}t_{2}^{-1}t_{3}^{-1}) \\
        & +q^{1/2}(1-q)(t_{1}t_{2}+t_{1}t_{3}+t_{2}t_{3}+t_{1}^{-1}t_{2}^{-1}+t_{1}^{-1}t_{3}^{-1}+t_{2}^{-1}t_{3}^{-1})\end{aligned}\right)}{(1-q)(1-t_{1}q^{1/4})(1-t_{2}q^{1/4})(1-t_{3}q^{1/4})(1-t_{1}^{-1}t_{2}^{-1}t_{3}^{-1}q^{1/4})}.
\eeqn
\end{prop}

\parsec Upon performing the change of variables

\beqn
q= x^{2} , \quad t_1 = (y_{2}y_{3})^{1/2}/y_1^{1/2} , \quad t_2 = (y_{1}y_3)^{1/2}/ y_2^{1/2} , \quad t_3 = (y_1 y_2)^{1/2}/y_{3}^{1/2}
\eeqn

the result agrees with previously computed indices for single particle states for supergravity on $AdS_{4}\times S^{7}$ \cite{}


\section{Twisted supergavity on AdS space}
\label{sec:ads}

So far, we have mostly given evidence for the 11-dimensional theory as a twist of supergravity in a flat background. 
We now turn to twisted versions of AdS backgrounds of 11-dimensional supergravity. 

In $M$-theory, AdS backgrounds arise from backreacting some number of branes. 
For $M2$ branes, the backreacted geometry is ${\rm AdS}_4 \times S^7$.
For the $M5$ branes, the backreacted geometry is ${\rm AdS}_7 \times S^4$. 

According to the AdS/CFT correspondence, supergravity on such backgrounds should be dual to the relevant worldvolume theory. 
In this section, we do not directly refer to the worldvolume theories on the holomorphic twists of the $M2$ and $M5$ branes.
Rather, we identify the fields sourcing the branes at the level of the twisted 11-dimensional theory.
In turn, we give a proposal for the twisted AdS background. 
We will show that the twist of the superconformal algebra is a global symmetry of this twisted background. 

\subsection{Superconformal algebras}

The complex form of the algebra of isometries for supergravity in both the ${\rm AdS}_4$ and ${\rm AdS}_7$ backgrounds is $\lie{osp}(8|2)$ (though, their real forms differ). 
This agrees with the complex form of the 6d $\cN=(2,0)$ superconformal algebra and the 3d $\cN=8$ superconformal algebra. 
The bosonic part of this algebra is isomorphic to $\lie{so}(8) \oplus \lie{sp}(2) \cong \lie{so}(8) \oplus \lie{so}(5)$. 

The minimal supercharge $Q$ acting on 11-dimensional supersymmetry algebra is an element of this superconformal algebra. 
In \cite{SWsuco2}, the second two authors show that the $Q$-cohomology is isomorphic to $\lie{osp}(6|1)$. 
This super Lie algebra will play the role of the isometries in the twisted AdS background. 

\subsection{The ${\rm AdS}_4 \times S^7$ background}

In this section we introduce the analog of the ${\rm AdS}_4 \times S^7$ background in our conjectural description of the minimal twist of 11-dimensional supergravity. 
%In the physical AdS background, the only bosonic fields which are non-zero are the metric and the \brian{finish}

\parsec[]

Decompose the 11-dimensional manifold $\CC^5 \times \RR$ as
\[
 \CC^4_w\times \CC_z \times \RR .
\]

Analogous to before, the ${\rm AdS}_4 \times S^7$ background arises from backreacting M2 branes. Consider a stack of $N$ $M2$ branes wrapping $\R\times \C_z$. A natural interaction to consider is 
\[
I_{M2}(\gamma) = N\int_{\C_z} \gamma + \cdots
\] 
which is nonzero only on the component of $\gamma$ in $\Omega^1(\R)\otimes \Omega^{1,1}(\C^5)$. Unlike the case of $M5$ branes, the coupling does not involve choosing a primitive for a field strength - it is an electric coupling.
We have only indicated the lowest order coupling, the $\cdots$ indicate higher order couplings which will be higher order in the fields of the 11d theory and explicitly involve the fields in the worldvolume theory. 

This coupling is justified by comparison with the physical theory and by dimensional reduction. 
Indeed, as discussed in section \ref{s:components}, the component of $\gamma$ which participates in the above coupling is a component of the $C$-field of eleven dimensional supergravity. Thus, the proposal mirrors electric couplings of $M2$ branes in the physical theory, which simply involves integrating the $C$-field over the worldvolume of the brane. 

Moreover, reducing on a circle transverse to the $M2$ brane yields the $SU(4)$ twist of type IIA supergravity on $\R^2\times \C_z\times \C^3$ with $N$ $D2$ branes wrapping $\R\times \C_z$. As is shown in \cite{CLsugra}, an electric coupling of D2 branes to the $SU(4)$ twist of type IIA supergravity is given by 
\[
I_{D2}(\gamma) = N \int_{\R\times\C_z} \gamma + \cdots
\] 
where $\gamma$ now denotes the 1-form field of the $SU(4)$ twist of type IIA supergravity. It is immediate that the pullback of $I_{M2}$ along the map in the proof of proposition \ref{prop:dimred} recovers $I_{D2}$. 


% Based on the discussion above, it is natural to expect that there is a field of twisted 11-dimensional supergravity which sources the twist of a stack of $N$ $M2$ branes living on the submanifold $\CC_z \times \RR \cong \{w=0\} \subset \CC^5$. 

% The differential form which sources the brane is an element
% \[
% \til{F} \in \Omega^{4,3} (\CC^4_w \, \setminus \, 0) \otimes \Omega^{0,0} (\CC_z) \otimes \Omega^{0} (\RR) \subset \Omega^{\bu} \left(\CC^5 \times \RR \, \setminus \, \{w=0\}\right) .
% \]
% Equivalently, we can think about this as a distributional valued form $\til{F} \in \Bar{\Omega}^{4,3}(\CC^5) \otimes \Omega^0 (\RR)$ which satisfies the distributional equation
% \[
% \dbar \til{F} = N \delta_{w=0} 
% \]
% where $\delta_{w=0}$ is the Dirac $\delta$-distributional for the submanifold $\{w=0\} = \CC_z \times \RR$. 

% Using the Calabi--Yau form we can identify such a differential form with a field of twisted supergravity. 
% Indeed $F = \til{F} \wedge \Omega^{-1}$ is a distributional field of type
% \[
% F \in \Bar{\PV}^{1,3}(\CC^5) \otimes \Omega^0 (\RR) .
% \]
% For $F$ to make sense as a background of twisted supergravity it must satisfies the appropriate (possibly nonlinear) equation of motion, which we now verify. 

\parsec[sec:m2backreact]

The backreacted geoemtry will be given by a solution to the equations of motion upon deforming the 11d action by the interaction $I_{M2}(\gamma)$. 
Varying the deformed action with respect to $\gamma$ 
we obtain the equation of motion
\beqn\label{eqn:ads4eom1}
\dbar \mu + \frac12 [\mu, \mu] + \partial\gamma\partial\gamma = N \Omega^{-1} \delta_{w=0} \\
\eeqn
Here $[-,-]$ is the Schouten bracket. 
Varying $\beta$ we obtain the equation of motion
\beqn\label{eqn:adseom2}
\div \mu = 0 .
\eeqn

\begin{lem}
Let
\[
 F_{M2} = \frac{6}{(2\pi i)^4} \frac{\sum_{a=1}^4 \wbar_a \d \wbar_1 \cdots \Hat{\d \wbar_a} \cdots \d \wbar_4}{\|w\|^{8}} \partial_z .
\]
Then, the background where $\mu = N F_{M2}$ and $\gamma = 0$
satisfies the above equations of motion in the presence of a stack of $N$ $M2$ branes:
\begin{align*}
\dbar (N F_{M2}) + \frac12 [N F_{M2}, N F_{M2}] & = N \Omega^{-1} \delta_{w=0} \\
\div (N F_{M2}) & = 0  .
\end{align*}
Here, we set all components of the field $\gamma$ equal to zero (as well as the fields $\nu,\beta$). 
\end{lem}

\begin{proof}
Upon specializing $\gamma = 0$, the last term in the first equation above vanishes. The equation $\dbar F_{M2} = \Omega^{-1} \delta_{w=0}$ characterizes the Bochner--Martinelli kernel representing the residue class on $\CC^4 \, \setminus \, 0$. 
It is clear that $\div F_{M2} = 0$ and 
\[
[F_{M2}, F_{M2}] = 0
\] 
by simple type reasons. 
\end{proof}

\parsec[]

To provide evidence for the claim that this is the twisted analog of the AdS geometry we will match the symmetries present in the physical theory and those in the twisted theory. 

We have recalled that the $Q$-cohomology of $\lie{osp}(8|2)$ is isomorphic to the super Lie algebra $\lie{osp}(6|1)$. 
We will define an embedding of $\lie{osp}(6|1)$ into the 11-dimensional theory on $\CC^5 \times \RR \setminus \{w=0\}$ which corresponds to the twist of the 3d superconformal algebra.
We first focus on the case where the flux $N=0$, in this case the embedding can be extended to all of $\CC^5 \times \RR$. 

\parsec[] 

The bosonic part of $\lie{osp}(6|1)$ is the direct sum Lie algebra $\lie{sl}(4) \oplus \lie{sl}(2)$. 
The Lie algebra $\lie{sl}(2)$ represents special conformal transformations in $\CC_z$; the vector fields representing these transformations are not divergence-free so must be slightly adjusted. 
The Lie algebra $\lie{sl}(4)$ represents rotations along the plane $\CC^4_w$.   

\begin{itemize}
\item The bosonic summand $\lie{sl}(2)$ is mapped to the vector fields:
\[
\frac{\del}{\del z} ,\quad z \frac{\del}{\del z} - \frac14 \sum_{a=1}^4 w_a \frac{\del}{\del w_a} , \quad z \left(z \frac{\del}{\del z} - \frac12 \sum_{a=1}^4 w_a \frac{\del}{\del w_a} \right) \in \PV^{1,0}(\CC^5) \otimes \Omega^0(\RR) .
\]
Notice that these vector fields are divergence-free and along $w=0$ reduce to the usual special conformal transformations.
\item The bosonic summand $\lie{sl}(4)$ is mapped to the $4$-dimensional rotations: 
\[
\sum_{a,b=1}^4 B_{ab} w_a \frac{\del}{\del w_b} \in \PV^{1,0}(\CC^5) \otimes \Omega^0(\RR) , \quad (B_{ab}) \in \lie{sl}(4) .
\]
\end{itemize}

The odd part of the algebra $\lie{osp}(6|1)$ is $\wedge^4 W \otimes R$ where $W$ is the fundamental $\lie{sl}(4)$ representation and $R$ is the fundamental $\lie{sl}(2)$ representation. 
It is natural to split $R = \CC_{+1} \oplus \CC_{-1}$ so that the odd part decomposes as
\[
(\wedge^2 \CC^4)_{+1} \oplus (\wedge^2 \CC^4)_{-1} .
\]

\begin{itemize}
\item 
The fermionic summand $(\wedge^2 \CC^4)_{+1}$ consists of the supertranslations. 
It is mapped to the fields: 
\[
\frac{1}{2} (w_a \d w_b - w_b \d w_a) \in \Omega^{1,0}(\CC^5) \otimes \Omega^0(\RR) , \quad a,b=1,2,3,4 .
\] 
\item The fermionic summand $(\wedge^2 \CC^4)_{-1}$ consists of the remaining superconformal transformations. 
It is mapped to the fields: 
\[
\frac{1}{2} z (w_a \d w_b - w_b \d w_a) \in \Omega^{1,0}(\CC^5) \otimes \Omega^0(\RR) , \quad a,b=1,2,3,4. 
\] 
\end{itemize}


\begin{lem}\label{lem:m2emb}
These assignments define an embedding of $\lie{osp}(6|1)$ into the linearized BRST cohomology of the fields of the 11-dimensional theory on $\CC^5 \times \RR$. 
Equivalently, it defines an embedding
\[
i_{M2} \colon \lie{osp}(6|1) \hookrightarrow E(5,10) .
\]
\end{lem} 
\begin{proof}
The second assertion follows from Theorem \ref{thm:global} that as a super Lie algebra the linearized BRST cohomology of the global symmetry algebra of the 11-dimensional theory on $\CC^5 \times \RR$ is the trivial central extension of $E(5,10)$. 
Recall that the odd part of $E(5,10)$ is precisely the module of closed two-forms on $\CC^5$. 
To explicitly describe the embedding into $E(5,10)$ we simply apply the de Rham differential to the last two formulas above.
Recall, we are using the holomorphic coordinates $(z,w_1,\ldots,w_4)$ on $\CC^5$ where $z$ is the holomorphic coordinate along the $M2$ brane. 
\begin{itemize}
\item 
The fermionic summand $(\wedge^2 \CC^4)_{+1}$ embeds into closed two-forms as
\[
\d w_a \wedge \d w_b, \quad a,b=1,2,3,4. 
\] 
\item The fermionic summand $(\wedge^2 \CC^4)_{-1}$ embeds into closed two-forms as
\[
z \d w_a  \wedge \d w_b + \frac12 \d z \wedge (w_a \d w_b - w_b \d w_a) , \quad a,b=1,2,3,4. 
\] 
\end{itemize}
\end{proof}
\parsec[]

Next, we turn on a nontrivial unit of flux $N \ne 0$. 
Since not all of the fields we wrote down above commute with the flux $N F_{M2}$, they are not compatible with the total differential $\delta^{(1)} + [N F_{M2}, -]$ acting on the fields supported on $\CC^5 \times \RR \setminus \{w=0\}$. 
Nevertheless, we have the following. 

\begin{prop}
\label{prop:brads4}
There exist $N$-dependent corrections to the fields defining the embedding of $\lie{osp}(6|1)$ summarized above which are closed for the modified BRST differential $\delta^{(1)} + [N F_{M2},-]$. 
Furthermore, these order $N$ corrections define an embedding of $\lie{osp}(6|1)$ inside the cohomology of the fields of 11-dimensional theory on $\CC^5 \times \RR \setminus \CC \times \RR$ with respect to the differential $\delta^{(1)} + [N F_{M2},-]$.
\end{prop}

\begin{proof}
Let $\cL(\CC^5 \times \RR \setminus \{w=0\})$ denote the super $L_\infty$ algebra obtained by parity shifting the fields of the 11-dimensional theory. 
We make the identification 
\[
(\CC^5 \times \RR) \setminus \{w=0\} \cong (\CC_w^4 \setminus 0) \times \CC_z \times \RR .
\]

Set $F = F_{M2}$ for notational convenience. Recall that we are viewing $F$ as an element of $\PV^{1,3}(\CC_w^4 \setminus 0) \otimes \Omega^{0,0}(\CC_z) \otimes \Omega^0(\RR)$. 
The operator $[F,-]$ acts on the fields according to two types of maps:
\begin{align*}
[F ,-] & \colon \PV^{i,\bu}(\CC^4_w \setminus 0) \otimes \PV^{j,\bu} (\CC_z) \otimes \Omega^\bu (\RR) \to \PV^{i,\bu+3}(\CC^4_w \setminus 0) \otimes \PV^{j,\bu} (\CC_z) \otimes \Omega^\bu (\RR) \\
[F,-] & \colon \Omega^{i,\bu}(\CC^4_w \setminus 0) \otimes \Omega^{j,\bu} (\CC_z) \otimes \Omega^\bu (\RR) \to \Omega^{i,\bu+3}(\CC^4_w \setminus 0) \otimes \Omega^{j,\bu} (\CC_z) \otimes \Omega^\bu (\RR).
\end{align*}

%\brian{get the filtration straight}


The first page of the spectral sequence is the cohomology with respect to the original linearized BRST differential $\delta^{(1)}$. 
Recall that the linearized BRST differential decomposes as
\[
\delta^{(1)} = \dbar + \d_{\RR} + \div |_{\mu \to \nu} + \del |_{\beta \to \gamma}  .
\]
To compute this page, we use an auxiliary spectral sequence which simply filters by the holomorphic form and polyvector field type. 
This first page of this auxiliary spectral sequence is simply given by the cohomology with respect to $\dbar + \d_{\RR}$. 
This cohomology is given by
\begin{equation}
  \label{eqn:ads4ss} 
  \begin{tikzcd}[row sep = 1 ex]
    + & - \\ \hline
H^\bu(\CC^4\setminus 0, \T) \otimes H^\bu(\CC, \cO) & H^\bu(\CC^4 \setminus 0, \cO) \otimes H^\bu(\CC, \cO) \\
H^\bu(\CC^4\setminus 0, \cO) \otimes H^\bu(\CC, \T) \\
H^\bu(\CC^4\setminus 0, \cO) \otimes H^\bu(\CC, \cO) & H^\bu(\CC^4\setminus 0, \cO) \otimes H^\bu(\CC, \Omega^1) \\ & H^\bu(\CC^4\setminus 0, \Omega^1) \otimes H^\bu(\CC, \cO)  
\end{tikzcd}
\end{equation}
where $\T$ denotes the holomorphic tangent sheaf, $\Omega^1$ denotes the sheaf of holomorphic one-forms, and $\cO$ is the sheaf of holomorphic functions.

The cohomology of $\CC$ is concentrated in degree zero and there is a dense embedding
\[
\CC[z] \hookrightarrow H^\bu(\CC, \cF) 
\]
for $\cF = \cO, \T$, or $\Omega^1$. 

For $\cF = \cO, \T$, or $\Omega^1$, the cohomology $H^\bu(\CC^4 \setminus 0, \cF)$ is concentrated in degrees $0$ and $3$. 
There are the following dense embeddings 
\begin{align*}
\CC[w_1,\ldots, w_4] & \hookrightarrow H^0(\CC^4 \setminus 0, \cO) \\ 
\CC[w_1,\ldots, w_4] \{\partial_{w_i}\} & \hookrightarrow H^0(\CC^4 \setminus 0, \T) \\
\CC[w_1,\ldots, w_4] \{\d w_i\} & \hookrightarrow H^0(\CC^4 \setminus 0, \Omega^1) 
\end{align*}
and
\begin{align*}
(w_1\cdots w_4)^{-1} \CC[w_1^{-1},\ldots, w_4^{-1}] & \hookrightarrow H^3(\CC^4 \setminus 0, \cO) \\ 
(w_1\cdots w_4)^{-1} \CC[w_1^{-1},\ldots, w_4^{-1}] \{\partial_{w_i}\} & \hookrightarrow H^3(\CC^4 \setminus 0, \T) \\
(w_1\cdots w_4)^{-1} \CC[w_1^{-1},\ldots, w_4^{-1}] \{\d w_i\} & \hookrightarrow H^3(\CC^4 \setminus 0, \Omega^1) .
\end{align*}

It follows that (up to completion) the cohomology 
\[
H^\bu(\cL(\CC^5 \times \RR \setminus \{w=0\}) , \dbar)
\]
is the direct sum of $H^\bu(\cL(\CC^5 \times \RR), \dbar)$ with 
\begin{equation}
  \label{eqn:ads4ss2} 
  \begin{tikzcd}[row sep = 1 ex]
    - & + \\ \hline
H^3(\CC^4\setminus 0, \cO)[z] \{\partial_{w_i}\}  \ar[r, dotted, "\div"] & H^3(\CC^4 \setminus 0, \cO) [z] \\
H^3(\CC^4\setminus 0, \cO) [z] \partial_z \ar[ur, dotted, "\div"'] \\
H^3(\CC^4\setminus 0, \cO) [z] \ar[r, dotted, "\del"] \ar[dr, dotted, "\del"'] & H^3(\CC^4\setminus 0, \cO)[z] \d z \\ & H^3(\CC^4\setminus 0, \Omega^1)[z] \{\d w_i\} .
\end{tikzcd}
\end{equation}
The remaining piece of the original BRST operator is drawn in dotted lines. 
The first page of the spectral sequence converging to the cohomology with respect to $\delta^{(1)} + [N F, -]$ is given by the cohomology of the global symmetry algebra on $\CC^5 \times \RR$, which we computed in \S \ref{sec:global}, plus the cohomology of the above complex with respect to dotted line operators. 
In this description the image of the flux $F$ at this page in the spectral sequence corresponds to the following class 
\[
[F] = (w_1 \cdots w_4)^{-1} \partial_z \in H^3(\CC^4\setminus 0, \cO) [z] \partial_z .
\]

The next page of the spectral sequence is given by computing the cohomology with respect to the operator $[N F,-]$. 
As observed above, this operator maps Dolbeault degree zero elements to Dolbeault degree three elements. 
For degree reasons, there are no further differentials and the spectral sequence collapses after the second page. 

The embedding of $\lie{osp}(6|1)$ we wrote down in lemma \ref{lem:m2emb} lands in the kernel of the original BRST operator $\delta^{(1)}$. 
To see that it this embedding can be lifted to the full cohomology we need to check that any element in the image of the original embedding is annihilated by $\big[ N [F] , - \big]$. 
This is a direct calculation. 
For instance, recall that an element in the image of the odd summand $(\wedge^2 \CC^2)_{-1}$ (which corresponds to a superconformal transformation) is of the form $z w_a \wedge \d w_b = z(w_a \d w_b - w_b \d w_a)$. 
We have
\[
\big[[F] , z(w_a \d w_b - w_b \d w_a) \big] = (w_1\cdots w_4)^{-1} (w_a \d w_b - w_b \d w_a) = 0
\]
since the class $(w_1\cdots w_4)^{-1}$ is in the kernel of the operator given by multiplication by $w_a$ for any $a = 1,\ldots 4$. 
\end{proof}

\subsection{The ${\rm AdS}_7 \times S^4$ background}

In this section we introduce the analog of the ${\rm AdS}_7 \times S^4$ background in our description of the minimal twist of 11-dimensional supergravity. Decompose the eleven dimensional spacetime as $\C^3_z\times \C^2_w\times \R$.

%In the physical AdS background, the only bosonic fields which are non-zero are the metric and the \brian{finish}

\parsec[sec:m5coupling]

Analogous to the physical theory, the ${\rm AdS}_7 \times S^4$ background in the holomorphic twist will arise by backreacting $M5$ branes. To this effect, we begin by discussing how the 11d theory couples to M5 branes. 
Consider a stack of $N M5$ branes wrapping 
\[
\{w_1=w_2=t=0\} \subset \C^3_z\times \C^2_w\times \R 
\] 

It is natural to consider the nonlocal interaction 
\[
I_{M5} = N\int_{\C^3_z} \div^{-1}\mu \vee \Omega +\cdots 
\]
Note that this expression is only nonzero on the component of $\mu$ in $\PV^{1,3}$. 
We argue that this coupling is consistent with expectations from the physical theory and from dimensional reduction. 

The twisted field $\mu^{1,3}$ is a component of the Hodge star of the $G$-flux in the physical theory \ref{s:components}. 
In the physical theory, M5 branes magnetically couple to the $C$-field; the coupling involves choosing a primitive for the hodge star of the $G$-flux and integrating it over the $M5$ worldvolume. Our twist contains no fields corresponding to components of such a primitive; hence such a magnetic coupling is reflected in the appearance of $\div^{-1}$. 

\parsec[]

We obtain a deeper justification for this coupling through dimensional reduction to type IIA supergravity. 
Reducing on the circle along the directions the $M5$ branes wrap yields the $SU(4)$ invariant twist of type IIA supergravity on $\CC^4 \times \RR^2$ with $N$ $D4$ branes wrapping $\CC^2 \times \RR$. 

In \cite{CLsugra}, it is shown that the magnetic coupling of $D4$ branes to the $SU(4)$ twist of IIA is of the form
\[
N \int _{\C^2 \times \RR} \div^{-1} \mu \vee \Omega_{\C^4} + \cdots .
\]
Again, we have only explicitly indicated the first-order piece of the coupling. 

\parsec[s:m5backreact]

The backreacted geometry will be given by a solution to the equations of motion upon deforming the 11-dimensional action by the interaction $I_{M5}(\mu)$. 

Varying the potential $\div^{-1} \mu$, we obtain the following equation of motion involving the field $\gamma$:
\beqn\label{eqn:m5eom1}
\dbar \del \gamma + \div \left(\frac{1}{1-\nu} \mu\right) \wedge \del \gamma = N \delta_{w_1=w_2=t=0} .
\eeqn
Notice that there is an extra derivative compared to the equation of motion arising from varying the field $\mu$. 
This equation only depends on $\gamma$ through its field strength $\del \gamma$. 

Varying $\gamma$ we obtain the equation of motion 
\beqn\label{eqn:m5eom2}
(\dbar + \d_\RR) \mu + \del \gamma \del \gamma = 0 .
\eeqn 
Again, this only depends on $\gamma$ through its field strength $\del \gamma$.


\begin{lem}
\label{lem:ads7flux}
Let
\[
F_{M5} = \frac{1}{(2\pi i)^3} \frac{\wbar_1 \d \wbar_2 \wedge \d t - \wbar_2 \d \wbar_1 \wedge \d t + t \d \wbar_1 \wedge \d \wbar_2}{(\|w\|^2 + t^2)^{5/2}} \wedge \d w_1 \wedge \d w_2
\]
%and suppose 
%\[
%\til F \in \Bar{\Omega}^{1,2} (\CC^5) \otimes \Bar{\Omega}^1(\RR)
%\]
%satisfies $\del \til F = F$. 
Then, $\del\gamma = N F_{M5}$, $\mu = 0$, and $\nu = 0$ satisfies the equations of motion in the presence of a stack of $N$ $M5$ branes sourced by the term $N \delta_{w_1=w_2=t=0}$:
\begin{align*}
\dbar (NF_{M5}) + \d_{\RR} (NF_{M5}) & = N \delta_{w_1=w_2=t=0}  \\ 
(NF_{M5}) \wedge (NF_{M5}) & = 0 .
\end{align*}
Here, we set all components of the field $\mu$ equal to zero (as well as the fields $\nu,\beta$). 
\end{lem}

\begin{proof}
The first equation equation 
\[
\dbar F + \d_{\RR} F = N \delta_{w_1=w_2=t=0}
\]
characterizes the kernel representing $N$ times the residue class for a $4$-sphere in 
\[
(\CC^2 \times \RR) \setminus 0 \simeq S^4 \times \RR .
\] 
That is
\[
\oint_{S^4} N F = N 
\]
for any $4$-sphere centered at $0 \in \CC^2 \times \RR$.

The second equation $F \wedge F = 0$ follows by simple type reasons. 
\end{proof}

\parsec[s:m5embedding]

To provide evidence for the claim that this is the twisted analog of the AdS geometry we will match the symmetries present in the physical theory on ${\rm AdS}_7 \times S^4$ and those in the twisted theory. 

We have recalled that the $Q$-cohomology of $\lie{osp}(8|2)$ is isomorphic to the super Lie algebra $\lie{osp}(6|1)$. 
We will define an embedding of $\lie{osp}(6|1)$ into the 11-dimensional theory on $\CC^5 \times \RR \setminus \{w_1=w_2=t=0\}$ which corresponds to the twist of the 6d superconformal algebra.

We first focus on the case where the flux $N=0$.
In this case the embedding can be extended to all of $\CC^5 \times \RR$. 



Recall that we have chosen coordinates of the form
\[
\CC^5 \times \RR = \CC_z^3 \times \CC_w^2 \times \RR_t
\]
with $z_i, i=1,2,3$ and $w_a, a=1,2$.
The stack of $M5$ branes wrap $w_1=w_2=t=0$. 

The embedding of the bosonic piece of $\lie{osp}(6|1)$ can be described as follows. Recall that the bosonic part of $\lie{osp}(6|1)$ is the direct sum Lie algebra
\[
\lie{sl}(4) \oplus \lie{sl}(2) .
\]
which we write as $\lie{sl}(W) \oplus \lie{sl}(R)$ with $W,R$ complex four, two dimensional complex vector spaces. The roles of the $\lie{sl}(4)$ and $\lie{sl}(2)$ summands are interchanged compared to the case of the $M2$ brane. 
The Lie algebra $\lie{sl}(4)$ represents conformal transformations along the plane $\CC^3_z$.
Since not all such infinitesimal transformations are divergence-free, there precise formulas must be adjusted.   
The Lie algebra $\lie{sl}(2)$ represents rotations in $\CC^2_w$; the vector fields representing these transformations are automatically divergence free.
In more detail, the embedding of the bosonic piece can be given by the following explicit formulas. 

\begin{itemize}

\item
The bosonic abelian subalgebra $\CC^3 \subset \lie{sl}(4)$ is mapped to the translations 
\[
\frac{\del}{\del z_i} \in \PV^{1,0}(\CC^5) \otimes \Omega^0(\RR) , \quad i=1,2,3.
\]

\item
The bosonic subalgebra $\lie{sl}(3) \subset \lie{sl}(4)$ is mapped to the 
rotations
\[
A_{ij} z_i \frac{\del}{\del z_j} \in \PV^{1,0}(\CC^5)\otimes \Omega^0(\RR) , \quad (A_{ij}) \in \lie{sl}(3) .
\]

\item
The bosonic subalgebra $\CC \subset \lie{sl}(4)$ is mapped to the element
\[
\sum_{i=1}^3 z_i \frac{\del}{\del z_i} - \frac32 \sum_{a=1}^2 w_a \frac{\del}{\del w_a} \in \PV^{1,0}(\CC^5) \otimes \Omega^0(\RR)  .
\] 
Notice that these vector fields are divergence-free and restrict to the ordinary dilation along $w=0$. 
\item 
The bosonic subalgebra of $\lie{sl}(4)$ describing special conformal transformations on $\CC^3$ is mapped to the elements 
\[
z_j \left(\sum_{i=1}^3 z_i \frac{\del}{\del z_i} - 2 \sum_{a=1}^2 w_a \frac{\del}{\del w_a} \right) \in \PV^{1,0}(\CC^5) \otimes \Omega^0(\RR) .
\] 
Notice that these vector fields are divergence-free and restrict to the ordinary special conformal transformations along $w=0$. 
\item 
The bosonic summand $\lie{sl}(2)$ is mapped to the triple
\[
w_1 \frac{\del}{\del w_2}, w_2 \frac{\del}{\del w_1}, \frac12 \left(w_1 \frac{\del}{\del w_1} - w_2 \frac{\del}{\del w_2}\right) \in \PV^{1,0}(\CC^5) \otimes \Omega^0(\RR) .
\]
\end{itemize}

The odd part of the algebra $\lie{osp}(6|1)$ is $\wedge^4 W \otimes R$ where $W$ is the fundamental $\lie{sl}(4)$ representation and $R$ is the fundamental $\lie{sl}(2)$ representation. 
It is natural to split $W = L \oplus \CC$ with $L = \CC^3$ the fundamental $\lie{sl}(3) \subset \lie{sl}(4)$ representation. 
Then the odd part decomposes as
\[
L \otimes R \oplus \wedge^2 L \otimes R \cong \CC^3 \otimes \CC^2 \oplus \wedge^2 \CC^3 \otimes \CC .
\]

\begin{itemize} 
\item The summand $L \otimes R$ consists of the remaining 6d superstranlsations. 
It is mapped to the fields 
\[
z_i \d w_a \in \Omega^{1,0}(\CC^5) \otimes \Omega^0(\RR) ,\quad a=1,2, \quad i =1,2,3.
\] 
\item The summand $\wedge^2 L \otimes R$ consists of the remaining 6d superconformal transformations. 
It is mapped to the fields
\[
\frac12 w_a (z_i \d z_j - z_j \d z_i) \in \Omega^{1,0}(\CC^5)\otimes \Omega^0(\RR) , \quad a = 1,2, \quad k = 1,2,3. 
\]
\end{itemize}

\begin{lem}
These assignments define an embedding of $\lie{osp}(6|1)$ into the linearized BRST cohomology of the fields of the 11-dimensional theory on $\CC^5 \times \RR$. 
Equivalently, it defines an embedding
\[
i_{M5} \colon \lie{osp}(6|1) \hookrightarrow E(5,10) .
\]
\end{lem} 

\begin{proof}
To explicitly describe the embedding into $E(5,10)$ we simply apply the de Rham differential to the last two formulas above.
Recall, we are using the holomorphic coordinates $(z_1,z_2,z_3, w_1,w_2)$ on $\CC^5$ where $z_i$ are the holomorphic coordinates along the $M5$ brane. 
\begin{itemize}
\item 
The fermionic summand $L \otimes R$ embeds into closed two-forms as
\[
\d z_i \wedge \d w_a, \quad i=1,2,3, \quad a=1,2. 
\] 
\item 
The fermionic summand $\wedge^2 L \otimes R$ embeds into closed two-forms as
\[
w_a \d z_i  \wedge \d z_j + \frac12 \d w_a \wedge (z_i \d z_j - z_j \d z_i) , \quad i,j=1,2,3, \quad a=1,2. 
\] 
\end{itemize}
\end{proof}
\parsec[]

Next, we turn on a nontrivial unit of flux $N \ne 0$. 
Since not all of the fields we wrote down above commute with the flux $N F$, they are not compatible with the total differential $\delta^{(1)} + [N F, -]$ acting on the fields supported on $\CC^5 \times \RR \setminus \{w_1=w_2=t=0\}$. 
Nevertheless, we have the following.

\begin{prop}
\label{prop:brads7}
There exists $N$-dependent corrections to the embedding $i_{M5}$ which is compatible with the modified BRST differential $\delta^{(1)} + [N F_{M5},-]$. 
Furthermore, these order $N$ corrections define an embedding of $\lie{osp}(6|1)$ inside the cohomology of the fields of 11-dimensional theory on $\CC^5 \times \RR \setminus \CC \times \RR$ with respect to the differential $\delta^{(1)} + [N  F_{M5},-]$.
\end{prop}

\parsec[s:thfcohomology]

The proof of the above proposition follows from another indirect cohomological argument. 
Before getting to the proof, we introduce the relevant cohomology. 

The 11-dimensional theory is built from fields which live in the following tensor product of complexes 
\[
\Omega^{0,\bu}(\CC^5) \otimes \Omega^\bu(\RR).
\]
Precisely, this is where the $\beta,\nu$ fields live. 
The $\mu,\gamma$ fields live in versions of this complex where we take Dolbeault forms with coefficients in the holomorphic tangent and cotangent bundles, respectively. 

Another way to think about this complex is to first consider the full de Rham complex $\Omega^\bu(\CC^5 \times \RR)$, which includes both holomorphic and anti-holomorphic forms in the $\CC^5$ direction. 
The dg algebra of all de Rham forms on $\CC^5 \times \RR$ has an ideal generated by the holomorphic one forms $\{\d z_i\}_{i=1,\ldots,5}$.
There is an isomorphism of dg algebras
\[
\Omega^{0,\bu}(\CC^5) \otimes \Omega^\bu(\RR) \cong \Omega^\bu(\CC^5 \times \RR) \, / \, (\d z_1,\ldots, \d z_5) .
\]
The advantage of this presentation is that we can define a complex associated to more general manifolds that are not products of a complex manifold with a smooth manifold.\footnote{More generally, we are describing the cohomology of a manifold equipped with a topological holomorphic foliation.}

For the M5 brane it was convenient to relabel the holomorphic coordinates on $\CC^5$ by $z_1,z_2,z_3,w_1,w_2$. 
At the twisted level, the geometry arising from backreacting $M5$ branes is based on the manifold 
\[
\CC^5 \times \RR \setminus \CC^3 \cong \CC_z^3 \times (\CC^2_w \times \RR \setminus 0) .
\]
The $\beta,\nu$ fields of 11-dimensional theory on this submanifold of $\CC^5 \times \RR$ live in the complex 
\[
\Omega^\bu\bigg(\CC^5 \times \RR \setminus \CC^3\bigg) \, / \, (\d z_1,\d z_2,\d z_3, \d w_1, \d w_2)  .
\]
The $\mu,\gamma$ fields live in similar complexes where we introduce a (trivial) vector bundle on $\CC^5 \times \RR \setminus \CC^3$. 

Since the $\CC^3$ wraps $w_1=w_2=t=0$ we can apply a version of the K\"unneth formula to identify this complex with 
\[
\Omega^{0,\bu}(\CC^3_z) \otimes \bigg( \Omega^\bu\left(\CC^2_w \times \RR \setminus 0 \right) \, / \, (\d w_1, \d w_2) \bigg).
\]

The cohomology of the Dolbeault complex of $\CC^3_z$ is easy to compute. 
The cohomology of the bit in parentheses is concentrated in degrees zero and two. 
In degree zero, there is a dense embedding
\[
\CC[w_1,w_2] \hookrightarrow H^0 \bigg( \Omega^\bu\left(\CC^2_w \times \RR \setminus 0 \right) \, / \, (\d w_1, \d w_2) \bigg)
\]
In degree two, there is a dense embedding
\[
w_{1}^{-1} w_2^{-1} \CC[w_1,w_2] \hookrightarrow H^2 \bigg( \Omega^\bu\left(\CC^2_w \times \RR \setminus 0 \right) \, / \, (\d w_1, \d w_2) \bigg).
\]

It will be useful to explain this last embedding in more detail. 
Consider the homogenous element $w_1^{-1} w_2^{-1}$. 
This represents the class of the Dolbeault-de Rham $2$-form
\[
\frac{\wbar_1 \d \wbar_2 \wedge \d t - \wbar_2 \d \wbar_1 \wedge \d t + t \d \wbar_1 \wedge \d \wbar_2}{(\|w\|^2 + t^2)^{5/2}} .
\]
Notice that if we wedge with the volume form $\d w_1 \d w_2$ this is the unit $N=1$ flux introduced in Lemma \ref{lem:ads7flux}. 
The homogenous element $w_1^{-n-1} w_2^{-m-1}$ represents the class of the holomorphic derivatives $\partial_{w_1}^n \partial_{w_2}^{m}$ applied to this $2$-form. 

Observe that when restricted to $\CC^5 \times \RR \setminus \CC^3$ the holomorphic tangent bundle along $\CC^5$ is still trivializable. 

\parsec[]

Let's turn to the proof of Proposition \ref{prop:brads7}.
We proceed completely analogously to the case of backreacted $M2$ branes as in the proof of Proposition \ref{prop:brads4}. 

\begin{proof}[Proof of Proposition \ref{prop:brads7}]
Let $\cL (\CC^5 \times \RR \setminus \{w_1=w_2=t=0\})$ denote the super $L_\infty$ algebra obtained by parity shifting the fields of the 11-dimensional theory on $\CC^5 \times \RR \setminus \{w_1=w_2=t=0\}$. 

There is a spectral sequence which converges to the cohomology of the fields with respect to the deformed linear BRST differential $\delta^{(1)} + [N F_{M5},-]$ whose first page
is the cohomology with respect to the original linearized BRST differential $\delta^{(1)}$. 
Recall that the linearized BRST differential decomposes as
\[
\delta^{(1)} = \dbar + \d_{\RR} + \div |_{\mu \to \nu} + \del |_{\beta \to \gamma}  .
\]
To compute this page, we use an auxiliary spectral sequence which simply filters by the holomorphic form and polyvector field type. 
This first page of this auxiliary spectral sequence is simply given by the cohomology of the fields supported on 
\[
\CC^5 \times \RR \setminus \{w_1=w_2=t=0\} \cong \CC_z^3 \times (\CC^2_w \times \RR \setminus 0)
\]
with respect to $\dbar + \d_{\RR}$. 

To compute this cohomology we follow the discussion in \S \ref{s:thfcohomology}.
Just as in the case of the $M2$ brane, we see that the $\dbar + \d_{\RR}$ cohomology is (up to completions) is the direct sum of the cohomology on flat space $H^\bu(\cL(\CC^5 \times \RR), \dbar)$ with
\begin{equation}
  \label{eqn:ads7ss2} 
  \begin{tikzcd}[row sep = 1 ex]
    + & - \\ \hline
w_1^{-1} w_2^{-1} \CC[w_1^{-1}, w_2^{-1}][z_1,z_2,z_3] \{\partial_{w_i}\}  \ar[r, dotted, "\div"] & w_1^{-1} w_2^{-1} \CC[w_1^{-1}, w_2^{-1}] [z_1,z_2,z_3] \\
w_1^{-1} w_2^{-1} \CC[w_1^{-1}, w_2^{-1}] [z_1,z_2,z_3] \{\del_{z_i}\} \ar[ur, dotted, "\div"'] \\
w_1^{-1} w_2^{-1} \CC[w_1^{-1}, w_2^{-1}] [z_1,z_2,z_3] \ar[r, dotted, "\del"] \ar[dr, dotted, "\del"'] & w_1^{-1} w_2^{-1} \CC[w_1^{-1}, w_2^{-1}][z_1,z_2,z_3] \{\d z_i\} \\ & w_1^{-1} w_2^{-1} \CC[w_1^{-1}, w_2^{-1}][z_1,z_2,z_3] \{\d w_i\} .
\end{tikzcd}
\end{equation}

Recall that the flux $F$ was defined as the image under $\del$ of some $\gamma$-type field. 
Therefore, the class $[F]$ does not live inside this page of the spectral sequence, but the operator $[[F], -]$ does act on this page nevertheless. 
For instance, if $f^i(z,w) \d z_i$ is a one-form living in $H^0(\CC^5, \Omega^1) \otimes H^0(\RR)$, then
\[
[ [F] , f^i (z,w) \d z_i ] = \ep_{ijk} w_1^{-1} w_2^{-1} \partial_{z_j} f^i(z,w) \del_{z_k} 
\]
which is an element in 
\[
\CC[w_1^{-1}, w_2^{-1}][z_1,z_2,z_3] \{\del_{z_i}\} \subset H^0(\CC^3, \T) \otimes H^2 \big(\Omega^\bu(\CC^2 \times \RR \setminus 0) / (\d w_1 , \d w_2) \big) .
\]

The first page of the spectral sequence converging to the cohomology with respect to $\delta^{(1)} + [N F, -]$ is given by the cohomology of the global symmetry algebra on $\CC^5 \times \RR$, which we computed in \S \ref{sec:global}, plus the cohomology with respect to dotted line operators in \eqref{eqn:ads7ss2}. 

The next page of the spectral sequence is given by computing the cohomology with respect to the operator $[N F,-]$. 
This operator maps Dolbeault-de Rham degree zero elements to Dolbeault-de Rham degree two elements. 
For degree reasons, there are no further differentials and the spectral sequence collapses after the second page. 

The embedding of $\lie{osp}(6|1)$ for $N=0$ lives in the kernel of the original BRST operator $\delta^{(1)}$. 
To see that it this embedding can be lifted to the full cohomology we need to check that any element in the image of the original embedding is annihilated by $\big[ N [F] , - \big]$. 
This is a direct calculation. 
For instance, recall that an element in the image of the odd summand $\wedge^2 L \otimes R = \wedge^2 \CC^3 \otimes \CC^2$ (which corresponds to a superconformal transformation) is of the form $w_a (z_i \d z_j - z_j \d z_i)$, $a=1,2, i,j=1,2,3$. 
We have
\[
\big[[F] , w_a (z_i \d z_j - z_j \d z_i)\big] = 2 \ep_{ijk} (w_1^{-1} w_2^{-1}) \cdot w_a \del_{z_k} = 0
\]
since the class $w_1^{-1} w_2^{-1}$ is in the kernel of the operator given by multiplication by $w_a$ for $a=1,2$.
Verifying that the remaining elements in the image of $i_{M5}$ are in the kernel of $\big[ [F], -\big]$ is similar.
This completes the proof.
\end{proof}



\documentclass[../main.tex]{subfiles}

\begin{document} 

\section{$E(1|6)$ modules from gravitons on $AdS_4\times S^7$}\label{sec:e16}

Having justified that the spaces of supergravity states constructed in the previous subsection are in fact counting gravitons on $AdS_4\times S^7$ and $AdS_7\times S^4$ respectively, we turn to studying representation theoretic properties of these state spaces. In this section, we focus on the case of gravitons on $AdS_4\times S^7$, using the description of the state space afforded by proposition \ref{prop:altstates} which describes it as the costalk of a factorization envelopes of the boundary conditions $\Omega^\bullet _\R\otimes \Omega^{0,\bullet}_\C (\mc L^{{r=0}}_{AdS_4} )$.

We construct a certain $\C^\times$-action on the boundary fields $\Omega^\bullet _\R\otimes \Omega^{0,\bullet}_\C (\mc L^{{r=0}}_{AdS_4} )$ equipped with the $L_\infty$ structure from remark \label{rmk:nottransferred} with the feature that the zeroth weight spaces are a local version of another exceptional linearly compact super-Lie algebras, $E(1|6)$. This in particular readily gives a decomposition of the state space $\mc H_{AdS_4}$ into $E(1|6)$-modules. We explicitly characterize the summands of this decomposition with their module structures and give closed form expressions for their characters.

\subsection{The graviton decomposition of twisted $AdS_4\times S^7$}
\parsec{}
We define a decomposition of the space of states $\mc H_{AdS_4}$.
It is induced by a decomposition of the boundary fields $\cA^\bu_{M} (\mc L^{{r=0}}_{AdS_4} )$ introduced in section \ref{sec:transversebc}  arising from a $\C^\times$ action which mixes fiberwise rescalings on spacetime with a fiberwise rescaling of the space of fields. 
We first consider a general three-manifold $M$ equipped with a THF; we specialize to $M = \R \times \C$ shortly.

Define the $\C^*$-action on holomorphic sections of $\cL_{AdS_4}^{r=0}$:
\begin{itemize}
\item On the sections 
\[
\mu(w_a,z) \in \C[w_1, \cdots, w_4]\{\del_{w_a}\} \otimes \cO_M \oplus \C[w_1, \cdots, w_4]\otimes T_M \] 
the action is
\[
\lambda \cdot \mu(w_a,z) = \mu(\lambda w_a , z).
\]
\item On the fields $\nu(w_a,z) \in \C[w_1, \cdots, w_4] \otimes \cO_M$ the action is
\[
\lambda \cdot \nu(w_a,z) = \nu(\lambda w , z).
\]
\item On the fields $\beta(w_a,z) \in  \C[w_1, \cdots, w_4] \otimes \cO_M$ the action is
\[
\lambda \cdot \beta(w_a,z) = \lambda^{-2} \beta(\lambda w_a , z).
\]
\item On the fields 
\[
\gamma(w_a,z) \in  \C[w_1, \cdots, w_4] \{\d w_a\} \otimes \cO_M \oplus  \C[w_1, \cdots, w_4] \otimes \Omega^1_M
\] 
the action is
\[
\lambda \cdot \gamma(w_a,z) = \lambda^{-2} \gamma(\lambda w _a, z).
\]
\end{itemize}

This $\C^*$-action on holomorphic sections extends to an action on the de Rham--Dolbeault complex $\cA^\bu_M(\cL_{AdS_7}^{r=0})$ in such a way that preserves the de Rham--Dolbeault operator.
In fact, this $\C^*$-action preserves the full local $L_\infty$ structure present on the parity shift of this complex of vector bundles.

\begin{prop}\label{prop:ads4decomp}
The $L_\infty$ structure on $\Pi \cA_M^\bu(\mc L^{r=0}_{AdS_4} )$ identified in section \ref{sec:transversebc} is equivariant for this $\C^\times$ action.
\end{prop}

This result induces a product decomposition 
\[
 \cA_M^\bu(\mc L^{r=0}_{AdS_4} ) = \prod _{n\geq -2} \mc F^{(n)}_{M}
\]
\brian{gradings again}
where $\cF^{(n)}_M \subset \cA_M^\bu(\mc L^{r=0}_{AdS_4} )$ is the weight $n$ eigenspace with respect to the above $\C^\times$ action thought of as a complex of super vector bundles on $M$. 
In particular, we see that $\mc F^{(0)}_{M}$ is itself a local dg Lie algebra, for which  every $\mc F^{(n)}_{M}$ is a module.

\subsection{The lowest pieces and 3d $\mc N=8$ BLG theory}\label{sec:BLG}
\parsec[] The first nontrivial case is the weight ${(-2)}$ piece. We have the following

\begin{lem}
There is an isomorphism of elliptic complexes \[\mc F^{(-2)}_{M} \simeq \Omega^\bullet_{M}.\]
\end{lem}
\begin{proof}
The only sections which contribute are those of type $\beta$ or $\gamma$ with no form components along the fiber directions. Therefore, we see directly that \[\mc F^{(-2)}_{M} \simeq \cA^\bu_M \xrightarrow{\del} \cA^\bu_M(\Omega^1) = \Omega^\bu_M.\] 
\end{proof}

\parsec[] The next nontrivial case is the weight ${(-1)}$ piece. 

\begin{lem}
There is an isomorphism of elliptic complexes 
\[ \mc F^{(-1)}_{M} \cong \cA^\bu_M \left (K^{1/4}\otimes (\C^4)^* \oplus \Pi K^{3/4}\otimes \C^4 \right ) \]
\end{lem}
\begin{proof}
The sections of this specified weight are:
\begin{itemize}
\item sections of type $\mu$ of the form $\mu_a(z) \del_{w_a}$. 
As the $w_a$ are fiber coordinates on $K^{1/4}$, these fields transform as sections of $K^{1/4}_\C$.
\item sections of type $\gamma$ of the form $\gamma_a(z) \d w_a$. 
These fields transform as sections of $K^{3/4}$. 
\end{itemize}
\end{proof}

\parsec[]
We wish to flag an appearance of $\mc F^{(-1)}_{\R\times \C}$ in supersymmetric physics in three-dimensions. There is a highly supersymmetric Chern-Simons-matter theory discovered independently by Bagger-Lambert \cite{Bagger_2007}, \cite{Bagger:2007jr} and Gustavsson \cite{Gustavsson:2007vu}. The aptly named BLG theory has $\mc N=8$ superconformal symmetry, and admits a holomorphic-topological twist that was characterized by Garner in \cite{Garner2022vds}.

The sheaf of complexes $\mc F^{(-1)}_{\R\times \C}$ matches the field contents of the holmorphic-topological twist of the BLG theory, and as such, it can be equipped with an $L_\infty$ structure under which it is perturbatively equivalent to the twisted BLG theory. In work-in-progress with Garner and Williams, we show that the action of $\mc F^{(0)}_{\R\times \C}$ on $\mc F^{(-1)}_{\R\times \C}$ in fact preserves this $L_\infty$-structure. 

\brian{it's an abelian L-infinity structure right? So we can just say complexes.}

\surya{i thought the BLG theory has a quartic superpotential.}

\brian{The thing we don't mention so much is the BV structure.}

\brian{I think we can at least comment that the BV structure comes from the BR.}

\subsection{The zero-th piece: A local version of $E(1|6)$}
\parsec[] The next nontrivial case is the weight zero piece $\cF^{(0)}_M$. 
This factor is special because it carries the induced structure of a local $L_\infty$ algebra on $M$. 
We will prove that it is equivalent to a local Lie algebra version of the exceptional super Lie algebra $E(1|6)$. 

We first recall the definition of this super Lie algebra \cite{KacBible}. 
The even part $E(1|6)_0$ is the semidirect product Lie algebra 
\beqn
\Vect(\Hat{D}) \ltimes \left( \cO(\Hat{D}) \otimes \mf{sl}(4) \right )
\eeqn
where $\Vect(\Hat{D})$ and $\cO(\Hat{D})$ denote vector fields and functions on the formal disk.

The odd part is the (unique) nontrivial extension of $E(1|6)_0$-modules 
\[0\to \Sym^2 (\C^4)\otimes \Gamma(\Hat{D}, K^{1/2}_{\Hat{D}}) \to E(1|6)_1 \to \wedge^2 (\C^4) \otimes \Gamma(\Hat{D}, K^{-1/2}_{\Hat{D}}) \to 0 
\]
where $K^{\pm 1/2}_{\Hat{D}}$ are positive and negative square roots of the canonical bundle on $\Hat{D}$.

To characterize this super Lie algebra we must define the following brackets.
\begin{itemize}
\item Given elements $A \otimes f \d z^{1/2}\in \Sym^2 (\C^4)\otimes K^{1/2}_\C$ and $B\otimes g \del_z^{1/2}\in \wedge^2 (\C^4)\otimes K^{-1/2}_\C$, we have that
\[
[A \otimes f \d z^{1/2}, B\otimes g \del_z^{1/2}] = A \star B \otimes fg \in \mf{sl}(4)\otimes \mc O.
\]
Here, $\star$ refers to the hodge star of $B$ and we are viewing $A$ and $*B$ as symmetric and skew-symmetric $4\times 4$ matrices respectively; their product is traceless. 

\item The final nontrivial bracket is more complicated. 
Given elements $B \otimes f \del_z^{1/2}, B'\otimes g \del_z^{1/2} \in \Gamma( \hat D, \wedge^2 (\C^4) \otimes K^{1/2}_\C)$, we have
\begin{align*}
[B\otimes f \d z^{-1/2} , B' \otimes g \d z^{-1/2} ] & = \tr (B \star B') \otimes fg \del_z \\ & + \frac12 (B \star B')_0 \otimes \left (\del (f \d z^{-1/2} ) g \d z^{-1/2} + f \d z^{-1/2} \del (g \d z^{-1/2} ) \right ) \\
& \in \Gamma (\hat D, T) \ltimes \left (\mf {sl}(4) \otimes \Gamma (\hat D, \mc O) \right ) .
\end{align*}
where again $\star$ denotes the Hodge star and the subscript of zero denotes projection to the traceless part. 
\brian{simplify last line so it is obviously a vector field}

\brian{Choose $\d z^{-1/2}$ or $\del_z^{1/2}$}
\end{itemize}

The relationship between this super Lie algebra and our decomposition is established through the following result.

\begin{prop}
Consider the case $M = \R \times \C$.
There is a quasi-isomorphism of super Lie algebras
\[
J^\infty_0\mc F^{(0)}_{\R\times \C} \simeq E(1|6).
\]
where the left hand side denotes the $\infty$-jets at $0\in \R\times \C$ of the local super-Lie algebra $\mc F^{(0)}_{\R\times \C}$. 
\end{prop}
\begin{proof}
We will characterize the local $L_\infty$-algebra $\mc F^{(0)}_{M}$ for general $M$. 
We will show that it is quasi-isomorphic to a local version of $E(1|6)$ where sections over the formal disk can be upgraded to sections over any open set in the THF manifold $M$.

The underlying complex of vector bundles $\mc F^{(0)}_{M}$ is the de Rham--Dolbeault complex of the complex of bundles 
\begin{equation}
\begin{tikzcd}
\ul{even} & \ul{odd} \\
\C\{w_a\del_{w_b}\}\otimes \mc O\ar[r, "\del^W_\Omega"]  & \mc O \\ 
\T \ar[ur] & \\
\Sym^2 (\C^4)\ar[r, "\del_W"]\ar[dr] & \C\{w_a\d w_b\} \otimes K^{-1/2}\ \\
& \Sym^2 (\C^4)\otimes \Omega^1\otimes K^{-1/2}
\end{tikzcd}
\end{equation}
Of course, the differentials are just appropriate components of the divergence operator and holomorphic deRham operator. We can compute cohomology by way of a spectral sequence whose first page is the cohomology with respect to $\del^W_\Omega + \del_W$. We see that the differential $\del^W$ maps surjectively onto functions and its kernel is isomorphic to $\mf{sl}(4)\otimes \mc O$. Likewise, the differential $\del_W$ is the canonical inclusion of $\mf{sl}(4)$ representations $\Sym^2 (\C^4)\to \C^4\otimes \C^4$. 
Its cokernel is a copy of $\wedge^2 \C^4$. 

Thus, we see that this page of the spectral sequence is given by
\begin{equation}
\mc E_M(1|6) \define \cA^\bu_M \left (
\begin{tikzcd}
\ul{even} & \ul{odd} \\
\T & \wedge^2 (\C^4) \otimes K^{-1/2} \\
\mf {sl}(4)\otimes \mc O & \Sym^2 (\C^4)\otimes K^{1/2}
\end{tikzcd} \right)
\end{equation}
and there are no non-zero differentials so the spectral sequence degenerates.

To see that the Lie structure induced from the $L_\infty$-structure on $\cA_M^\bu (\mc L^{r=0}_{AdS_4} )$ is in fact given by the same formulae as the brackets on $E(1|6)$ in equation \ref{defn:e(1|6)}, we provide an explicit quasi-isomorphism $\Psi^{(0)} \colon \mc E_M(1|6) \to \mc F^{(0)}_{M}$.
This quasi-isomorphism arises from a map on holomorphic sections.

\begin{itemize}
\item Given a section $\mu$ of $\T_M$ define
\begin{align*}
\Psi^{(0)} (\mu) &= \mu - \frac 14 (\del_\Omega \mu) w_a\del_{w_a} 
\end{align*}
where we view the right hand side as a section of $\T \oplus \C\{w_a \del_{w_a}\} \otimes \cO$.
\item Given a section $C = (C_{ab})$ of $\mf{sl}(4)\otimes \mc O$ define
\[
\Psi^{(0)} (C) = C_{ab}w_a\del_{w_b}
\]
which we view as a section of $\C\{w_a \del_{w_a}\} \otimes \cO$.
\item Given a section $B = (B_{ab})$ of $\wedge^2 (\C^4) \otimes K^{-1/2}$ define
\[
\Psi^{(0)} (B) = B_{ab} w_a \d w_b .
\]
which we view as a section of $\C\{w_{[a} \d w_{b]}\} \otimes K^{-1/2} = \wedge^2 \C^4 \otimes K^{-1/2}$.
\item Given a section 
$A=(A_{ab})$ of $\Sym^2 (\C^4) \otimes K^{1/2}$ define
\begin{align*}
\Psi^{(0)} (A_{ab}) = A_{ab} w_a w_b 
\end{align*}
which we view as a section of $\Sym^2 (\C^4) \otimes \Omega^1 \otimes K^{-1/2}$.
\end{itemize}
It is easy to see that $\Psi^{(0)}$ is a quasi-isomorphism and a straightforward if lengthy check confirms that it preserves brackets. 

The result about $\infty$-jets at $0$ in affine space $M = \R \times \C$ follows immediately.
\end{proof}

\begin{rmk}
We note that the map $i_{M2}$ from lemma \ref{lem:m2emb} in fact defines a Lie map from $\mf{osp}(6|2)$ to the sections of the boundary condition $\cA_{\R \times \C}^\bu (\mc L^{r=0}_{AdS_4} )$ over every open set containing the origin. 
The image of the map lands exactly in the step $\mc F^{(0)}_{\R\times \C}$ of the decomposition from proposition \ref{prop:ads4decomp}. 
Therefore we see that $E(1|6)$ contains $\mf{osp}(6|2)$ as a finite dimensional subalgebra.  
\end{rmk}

\subsection{General summands and $E(1|6)$-modules}
We now move on to giving an explicit description of the general summand $\mc F^{(j)}$ for $j \geq 1$. 

We first fix some notation for irreducible highest weight representations of $\mf{sl}(4)$. Let $\mf {h}\subset \mf{sl}$ be the Cartan given by diagonal matrices and let $L_i\in \mf {h}^*$ be the linear functional that picks out the $i$-th diagonal entry. We may accordingly write $\mf h^* = \C \{L_1, L_2, L_3, L_4\}/(L_1+\cdots + L_4)$. We will write $\Gamma_{a_1,a_2, a_3}$ for the irreducible representation of $\mf {sl}(4)$ of highest weight $(a_1+a_2+a_3)L_1+(a_2+a_3)L_2 + a_3L_3$. 

\begin{prop}
Let $j\geq 1$. The complex of vector bundles $\mc F^{(j)}_{\R\times \C}$ is quasi-isomorphic to 
\begin{equation}
\Omega^\bullet_\R\otimes \Omega^{0,\bullet}_\C \left (
\begin{tikzcd}
\ul{even} & \ul{odd} \\
\Gamma_{j,1,0} \otimes K^{-j/4} & \Sym^{j+2}(\C^4) \otimes \Omega^1 \otimes K^{-(j+2)/4} \\
\Sym^j (\C^4) \otimes \T \otimes K^{-j/4} & \Gamma_{j+1, 0, 1} \otimes K^{-(j+2)/4}
\end{tikzcd} \right )
\end{equation}


\end{prop}
\begin{proof}
We begin by noting that we can explicitly describe $\mc F_{\R\times \C}^{(j)}$ as $\Omega^\bullet _\R \otimes \Omega^{0,\bullet}_\C (F^{(j)})$ where $F^{(j)}$ denotes the following dg-vector bundle:
\begin{equation}
\begin{tikzcd}
\ul{even} & \ul{odd} \\
\Sym^{j+1} (\C^4) \otimes (\C^4)^* \otimes K^{-j/4} \ar[dr, "\del^W_\Omega"] \\ & \Sym^j (\C^4) \otimes K^{-j/4} \\ 
\Sym^j (\C^4) \otimes \T \otimes K^{-j/4} \ar[ur, "\del^V_\Omega"'] & \\
& \Sym^{j+2}(\C^4) \otimes \T^*\otimes K^{-(j+2)/4}\\ 
\Sym^{j+2} (\C^4) \otimes K^{-(j+2)/4}\ar[ur, "\del_V"] \ar[dr, "\del_W"'] \\
& K^{-(j+2)/4}\otimes \Sym^{j+1}(\C^4) \otimes \C^4 . 
\end{tikzcd}
\end{equation}

Note that the differentials here are all $\mf{sl}(4)$ equivariant maps, tensored with a differential operator acting on sections of a bundle on $\C$. In particular

\begin{itemize}
\item The differential $\del^W_\Omega$ involves the canonical projection 
\[\Sym^{j+1} (\C^4)\otimes (\C^4)^* \to \Sym^j (\C^4).\] Its kernel is precisely the irreducible highest weight representation $\Gamma_{j+1, 0 ,1}$.

\item The differential $\del_W$ is the canonical inclusion 
\[\Sym^{j+2} (\C^4) \to \Sym^{j+1}(\C^4) \otimes \C^4.\] Its cokernel is the irreducible highest weight representation $\Gamma_{j, 1, 0}$.
\end{itemize}

We can compute the cohomology using a spectral sequence whose first page is given by the cohomology with respect to $\del^W_\Omega + \del_W$. There are no further differentials on this page so the result follows. 
\end{proof}

\subsection{Characters of $E(1|6)$-modules}

Note that the decomposition of the state space $\Sym (\mc H_{AdS_4} ) = \prod_{j\geq -2} \mc U (\mc F^{(j)} _{\R\times \C} )(0)$ gives a product formula for the characters computed in proposition \ref{prop:ads4index}

\brian{Might be useful to point out that a systematic characterization of irreps of $E(1|6)$ is incomplete.}

\brian{But, we expect the modules appearing above to be irreducible.}

\[\chi \left (\Sym (\mc H_{AdS_4} ) \right )= \prod_{j\geq -2} \chi \left ( \mc U (\mc F^{(j)} _{\R\times \C} )(0)\right ).\]

We end this section by computing each of the characters $\chi \left ( \mc U (\mc F^{(j)} _{\R\times \C} )(0)\right )$. We will express our characters in terms of characters of highest weight respresentations of $\mf{sl(4)}$ which we denote $\chi^{\mf{sl}(4)}(\Gamma_{a_1,a_2, a_3} )$.

\parsec[]
From the characterization in \ref{}, the lowest step of the decomposition $\mc F^{(-2)}$ is just given by the deRham complex on $\R\times \C$, and accordingly the character of $\mc F^{(-2)}_{c}(0)$ is the constant function 1.

\parsec[]
We proceed to the next step of the decomposition, using the characterization in \ref{}.

\begin{prop}
The character $\chi \left ( \mc U(\mc F_{\R\times \C}^{(-1)})(0)\right )$ is given by the plethystic exponential of the following expression:
\begin{equation}
f_{-1}(t_1, t_2, t_3, q) = \frac{q\left (q^{-3/4}(t_1+ t_2+t_3 + t_1^{-1} t_2^{-1} t_3^{-1} )-q^{-1/4}(t_1^{-1} + t_2^{-1}+t_3^{-1} + t_1t_2t_3)\right )}{(1-q)}
\end{equation}
\end{prop}
\begin{proof}
The proof proceeds by the same trick as in the proof of proposition \ref{prop:altstates}. To describe the costalk, we wish to compute a limit of sections of $\mc F^{(-1)}_{\R\times \C,c}$ on open sets of the form $I\times D$ containing the origin in $\R\times \C$. Using ellipticity, we can describe such sections as a module over the ring generated by holomorphic derivatives of the delta function. 

Accordingly, we have contributions from the following summands:
\begin{itemize}
\item An even copy of $\C^4\otimes \C\{\d z^{3/4}\} \otimes \C[\del_z]\delta_{z=0}$. The character of this summand is
\[
\frac{q\left (q^{-3/4}\chi^{\mf{sl}(4)}(\Gamma_{1,0,0}) \right )}{(1-q)} =\frac{q\left (q^{-3/4}(t_1+ t_2+t_3 + t_1^{-1} t_2^{-1} t_3^{-1} )\right )}{(1-q)}
\]
\item An odd copy of $\C^4\otimes \C\{\d z^{1/4}\} \otimes \C[\del_z]\delta_{z=0}$. The character of this summand is
\[
\frac{-q\left (q^{-1/4}\chi^{\mf{sl}(4)}(\Gamma_{0,0,1}) \right )}{(1-q)} =\frac{-q\left (q^{-3/4}(t_1^{-1} + t_2^{-1} +t_3^{-1}  + t_1 t_2 t_3)\right )}{(1-q)}
\]
\end{itemize}
\end{proof}

Note that under the change of fugacities in \ref{}, this matches exactly with the single particle index for the theory on a single M2 brane \cite[Eq. (2.32)]{Bhattacharya:2008zy}.

\parsec[]
We continue to the next step of the decomposition given by $\mc{F}^{(0)}_{\R\times \C}$. 

Arguing similarly as in the proof of the previous proposition, we have the following.
\begin{prop}
The character $\chi \left ( \mc U (\mc{F}^{(0)}_{\R\times \C})(0)\right )$ is given by the plethystic exponential of the following expression:
\begin{equation}
f_0(t_1, t_2, t_3, q) = \frac{q}{(1-q)}\left ( q^{1/2}\chi^{\mf{sl}(4)}(\Gamma_{0,1,0})  + q^{-1/2}\chi^{\mf{sl}(4)}(\Gamma_{2,0,0})  - q - \chi^{\mf{sl}(4)}(\Gamma_{1,0,1}) \right)
\end{equation}
\end{prop}

\parsec[]
Finally, we continue to the general step of the decomposition.

\begin{prop}
Let $j\geq 1$. The character $\chi \left ( \mc U (\mc F^{(j)}_{\R\times \C} ) (0)\right )$ is the plethystic exponential of the following expression:
\begin{equation}
f_j(t_1, t_2, t_3, q) = \frac{q}{(1-q)}\left (\begin{aligned} q^{(j-2)/4}\chi^{\mf{sl}(4)}(\Gamma_{j+2,0,0})  &+ q^{(j+2)/4}\chi^{\mf{sl}(4)}(\Gamma_{j+1,0,1}) \\ - q^{j/4}\chi^{\mf{sl}(4)}(\Gamma_{j,1,0}) & - q^{(j+1)/4}\chi^{\mf{sl}(4)}(\Gamma_{j,0,0}) \end{aligned}\right)
\end{equation}
\end{prop}

\parsec[]
As a consequence, of the above we have that $f_{AdS_4} (t_i, q) = \sum_{j\geq -2} f_j (t_i, q)$, or explicitly:
\begin{align*}
& \frac{q\left (\begin{aligned} q^{1/4}(t_1+ t_2 + t_3+t_1^{-1}t_2^{-1}t_3^{-1}) &+ q^{-1} \\- q^{-1/4}(t_1^{-1}+ t_2^{-1} + t_3^{-1}+t_1t_2t_3) &- q   \end{aligned}\right)}{(1-q)(1-q^{1/4}t_1)(1-q^{1/4}t_2)(1-q^{1/4}t_3)(1-q^{1/4}t_1^{-1}t_2^{-1}t_3^{-1})}  \\ 
& =  1 + \frac{q}{1-q} \left (\begin{aligned} & q^{-3/4}\chi^{\mf{sl}(4)}(\Gamma_{1,0,0}) - q^{-1/4}\chi^{\mf{sl}(4)}(\Gamma_{0,0,1}) \\+ & q^{1/2}\chi^{\mf{sl}(4)}(\Gamma_{0,1,0})  + q^{-1/2}\chi^{\mf{sl}(4)}(\Gamma_{2,0,0})  - q - \chi^{\mf{sl}(4)}(\Gamma_{1,0,1}) \end{aligned}\right ) \\
& + \frac{q}{1-q} \sum_{j\geq 1} \left (\begin{aligned} q^{(j-2)/4}\chi^{\mf{sl}(4)}(\Gamma_{j+2,0,0})  &+ q^{(j+2)/4}\chi^{\mf{sl}(4)}(\Gamma_{j+1,0,1}) \\ - q^{j/4}\chi^{\mf{sl}(4)}(\Gamma_{j,1,0}) & - q^{(j+1)/4}\chi^{\mf{sl}(4)}(\Gamma_{j,0,0}) \end{aligned}\right ) 
\end{align*}

In \cite[Eq. (2.15, 2.16)]{Bhattacharya:2008zy}, the index counting gravitons on $f_{AdS_4}$ is expressed as a sum of characters of irreducible representations of the 3d $\mc N = 8$ superconformal algebra that the authors call \textit{graviton representations}. Comparison with the above expansion suggests the following conjecture

\begin{conj}\label{conj:e16gravitonrep}
For $j\geq -1$, the minimal twist of the $j+2$nd graviton representation in \cite[Eq. (2.15, 2.16)]{Bhattacharya:2008zy} is exactly $\mc F^{(j)}_{\R\times \C, c}(0)$. 
\end{conj}

\begin{rmk}\label{rmk:e16enhance}
This conjecture implies that the minimal twist of these graviton representations, which is a priori a module for the minimally twisted 3d $\mc N=8$ superconformal algebra $\mf{osp}(6|2)$, is in fact a module for the larger infinite dimensional super-Lie algebra $E(1|6)$. This can be thought of as analogous to the enhancement of conformal symmetries to the action of the Witt algebra of vector fields in 2d chiral conformal field theory. Such symmetry enhancements in 3 dimensions is the topic of joint work in progress with Garner and Williams.
\end{rmk}

\end{document}

\documentclass[../main.tex]{subfiles}

\begin{document} 

\section{$E(3|6)$-modules from gravitons on $AdS_7\times S^4$}\label{sec:e36}

We now repeat the analysis of the previous subsection for gravitons on $AdS_7\times S^4$ respectively, making use of the description of supergravity states on $AdS_7\times S^4$ as the costalk of the factorization envelopes of the boundary condition and $\Omega^{0,\bullet}_{\C^3} (\mc L^{{r=0}}_{AdS_7})$

As before, we construct certain $\C^\times$ actions on the boundary fields $\Omega^{0,\bullet}_{\C^3} (\mc L^{{r=0}}_{AdS_7})$; we find that the zeroth weight space is a local version of another exceptional linearly compact super-Lie algebra $E(3|6)$. The decomposition of $\mc H_{AdS_7}$ as a direct sum of $E(3|6)$ modules incidentally turns out to be very closely related to a decomposition of $E(5|10)$ into $E(3|6)$ modules studied in \cite{KacRudakov}.

\subsection{The graviton decomposition of twisted $AdS_7\times S^4$}
%We wish to consider a particular decomposition of the space of states $\mc H_{AdS_7}$. It is induced by a decomposition of the boundary fields $\Omega^{0,\bullet}_{\C^3} (\mc L^{{r=0}}_{AdS_7} )$ introduced in section \ref{sec:transversebc}. 

%For simplicity let us consider the eleven-dimensional theory on flat space $\R \times \C^5$ with some number of fivebranes supported on
%\[
%0 \times 0 \times \C^3 \subset \C^5 .
%\]
%is that (after taking into account the backreaction) the observables on a large number of fivebranes is Koszul dual to the factorization algebra 
%
%\parsec[s:flatdecomp]
%
%%In the simple case where $Z = \C^3$ and we identify the total space of $K^{1/2}_Z \otimes \C^2$ with $\C^5$ then the manifold obtained by removing the locus of the brane is homeomorphic to
%%\[
%%\C^3 \times (\R \times \C^2 - 0) .
%%\]
%%
%%Let $\pi : \R \times \C^5 - (0 \times \C^3) \to \C^3 \times \R_{>0}$ be the projection map whose fibers are homeomorphic to the sphere $S^4$ which links the location of the fivebranes.
%%We restrict the factorization algebra of the eleven dimensional theory $\Obs_{sugra}$ to the open set obtained by removing the locus of the brane.
%
%In the simple case that $Z = \C^3$ and we identify the total space of $K^{1/2}_Z \otimes \C^2$ with $\C^5$ there is a more direct construction of the factorization algebra $\Obs_{sugra}|_{Z}$. 
%
%Let $\pi : \R \times \C^5 \to \C^3$ be the projection map.
%Then, via $\pi$ we can pushforward the factorization algebra associated to the eleven-dimensional theory to obtain a factorization algebra
%\[
%\pi_* \Obs_{sugra} 
%\]
%on $\C^3$.
%This factorization algebra is not the factorization algebra associated to an ordinary sort of field theory on $\C^3$. 
%Nevertheless there is a subfactorization algebra which admits a natural grading so that each filtered component can be understood as such.

Recall the boundary condition at zero in twisted $AdS$ space $\Omega^{0,\bu}(\cL_{AdS_7}^{r=0})$ that we introduced in section \ref{sec:transversebc}.
Recall that this boundary condition makes sense for any complex threefold $X$ equipped with $K^{1/2}_X$.
The parity shift of these boundary fields $\Omega^{0,\bu}(\Pi \cL_{AdS_7}^{r=0})$ is equipped with an $L_\infty$ structure inherited from the BV action of the eleven-dimensional theory.

Define the $\C^\times$ action on $\mc L^{{r=0}}_{AdS_7}$ as follows.
\begin{itemize}
\item On the fields $\mu(w_a,z_i) \in \C[w_1, w_2]\{\del_{w_a}\} \otimes \cO_X \oplus \C[w_1, w_2] \otimes \T_X$ the action is
\[
\lambda \cdot \mu(w_a,z_i) = \mu(\lambda w_a , z_i).
\]
\item On the fields $\nu(w_a,z_i) \in \C[w_1, w_2]\otimes \cO_X$ the action is
\[
\lambda \cdot \nu(w_a,z_i) = \nu(\lambda w_a , z).
\]
\item On the fields $\beta(w_a,z_i) \in \C[w_1, w_2]\otimes \cO_X$ the action is
\[
\lambda \cdot \beta(w_a,z_i) = \lambda^{-1} \beta(\lambda w_a , z_i).
\]
\item On the fields $\gamma(w_a,z_i) \in \C[w_1, w_2] \{\d w_a\}\otimes \cO_X \oplus \C[w_1, w_2]\otimes \Omega^1_X$ the action is
\[
\lambda \cdot \gamma(w_a,z_i) = \lambda^{-1} \gamma(\lambda w _a, z_i).
\]
\end{itemize}

The following proposition is a straightforward if lengthy computation - we state it without proof.
\begin{prop}\label{prop:ads7decomp}
The $L_\infty$ structure on $\Omega^{0,\bullet}_{X} (\mc L^{{r=0}}_{AdS_7} )$ is equivariant for this $\C^\times$ action. 
\end{prop}

%\begin{proof}
%The only nontrivial bracket to check compatibility with the $\C^\times$ action is the one which takes two $\gamma$-type sections to a $\mu$-type section of the form
%\[
%[\gamma_1, \gamma_2] = \Omega^{-1} \vee (\del \gamma_1 \wedge \del \gamma_2) 
%\]
%where $\Omega \vee (-)$ is the operation which takes a four-form 
%
%It suffices to check equivariance in local coordinates; there are three cases.
%Suppose $\gamma_1(t;w,z) = f^i(t;w,z) \d z_i$ and $\gamma_2(t;w,z) = g^j(t;w,z) \d z_j$where $f^i,g^j$ are de Rham--Dolbeault forms for $i,j=1,2,3$. 
%Then $\lambda \cdot \gamma_1 = \lambda^{-1} f^i(\lambda t;\lambda w, z) \d z_i$ and $\lambda \cdot \gamma_2 = \lambda^{-1} g^j(\lambda t;\lambda w, z) \d z_j$. 
%Thus, if the $\mu$-type field is expanded as $[\gamma_1,\gamma_2] = F^k(t;w,z) \del_{z_k} + G^a (t;w,z) \del_{w_a}$ for some de Rham--Dolbeault forms $F^k,G^a$, $k=1,2,3,a=1,2$, it suffices to show that 
%\begin{equation}
%\begin{array}{llll}
%\label{eqn:lambda1} \lambda \cdot \left(F^k(t;w,z) \del_{z_k} \right) & = \lambda^{-2} F^k(\lambda t;\lambda w,z) \del_{z_k} \\
%\lambda \cdot \left(G^a(t;w,z) \del_{w_a} \right) & = \lambda^{-2} G^a(\lambda t;\lambda w,z) \del_{w_a} .
%\end{array}
%\end{equation}
%We expand $\del \gamma_1 \wedge \del \gamma_2$ to three terms
%\begin{multline}
%\del_{w_a} f^i(t;w,z) \del_{z_l} g^j(t;w,z) \d w_a \d z_i \d z_l \d z_j + \del_{z_k} f^i(t;w,z) \del_{w_b} g^j(t;w,z) \d z_k \d z_i \d w_b \d z_j \\  + \del_{w_a} f^i(t;w,z) \del_{w_b} g^j(t;w,z) \d w_a \d z_i \d w_b \d z_j .
%\end{multline}
%Thus in the notation above we have 
%\begin{align*}
%F^k (t;w,z) & = \epsilon_{ab} \epsilon_{ijk} \del_{w_a} f^i(t;w,z) \del_{w_b} g^j(t;w,z) \\
%G^a(t;w,z) & = \epsilon_{ilj} \epsilon_{ab} \del_{w_a} f^i(t;w,z) \del_{z_l} g^j(t;w,z)
%+ \epsilon_{kij} \epsilon_{ab} \del_{z_k} f^i(t;w,z) \del_{w_a} g^j(t;w,z)  .
%\end{align*}
%We compute the action of $\lambda \in \C^\times$ on $\mu = F^k \del_{z_k}$
%\[
%\lambda \cdot \left(F^k \del_{z_k} \right) = \epsilon_{ab} \epsilon_{ijk} \lambda^{-1} \del_{w_a} f^i(\lambda t;\lambda w,z) \lambda^{-1} \del_{w_b} g^j(\lambda t;\lambda w,z) \del_{z_k} .
%\]
%Thus the first part of \eqref{eqn:lambda1} is satisfied. 
%Similarly we observe that $\lambda \cdot \left(G^a(t;w,z) \del_{w_a} \right)  = \lambda^{-2} G^a(\lambda t;\lambda w,z) \del_{w_a}$. 
%
%The next case is when $\gamma_1(t;w,z) = f^i(t;w,z) \d z_i$ and $\gamma_2(t;w,z) = g^a(t;w,z) \d w_a$where $f^i,g^a$ are de Rham--Dolbeault forms for $i=1,2,3$ and $a=1,2$. 
%Then $\lambda \cdot \gamma_1 = \lambda^{-1} f^i(\lambda t;\lambda w, z) \d z_i$ and $\lambda \cdot \gamma_2 = g^a(\lambda t;\lambda w, z) \d w_a$. 
%Thus, if the $\mu$-type field is expanded as $[\gamma_1,\gamma_2] = F^k(t;w,z) \del_{z_k} + G^a (t;w,z) \del_{w_a}$ for some de Rham--Dolbeault forms $F^k,G^a$, $k=1,2,3,a=1,2$, it suffices to show that 
%\begin{equation}
%\begin{array}{llll}
%\label{eqn:lambda2} \lambda \cdot \left(F^k(t;w,z) \del_{z_k} \right) & = \lambda^{-1} F^k(\lambda t;\lambda w,z) \del_{z_k} \\
%\lambda \cdot \left(G^a(t;w,z) \del_{w_a} \right) & = \lambda^{-1} G^a(\lambda t;\lambda w,z) \del_{w_a} .
%\end{array}
%\end{equation}
%We expand $\del \gamma_1 \wedge \del \gamma_2$ to three terms
%\begin{multline}
%\del_{w_b} f^i(t;w,z) \del_{z_j} g^a(t;w,z) \d w_b \d z_i \d z_j \d w_a + \del_{z_k} f^i(t;w,z) \del_{w_b} g^a(t;w,z) \d z_k \d z_i \d w_b \d w_a \\  + \del_{z_j} f^i(t;w,z) \del_{z_k} g^a(t;w,z) \d z_j \d z_i \d z_k \d w_a .
%\end{multline}
%Thus in the notation above we have 
%\begin{align*}
%F^k (t;w,z) & = \epsilon_{ab} \epsilon_{ijk} \del_{w_a} f^i(t;w,z) \del_{z_l} g^b(t;w,z) + \epsilon_{ab} \epsilon_{kij} \del_{z_k} f^i(t;w,z) \del_{w_a} g^b(t;w,z) \\
%G^a(t;w,z) & = \epsilon_{ab} \epsilon_{jik} \del_{z_j} f^i(t;w,z) \del_{z_k} g^b(t;w,z) .
%\end{align*}
%We compute the action of $\lambda \in \C^\times$ on $\mu = G^a \del_{z_a}$
%\[
%\lambda \cdot \left(G^a \del_{w_a} \right) = \epsilon_{ab} \epsilon_{jik} \del_{z_j} f^i(t;w,z) \del_{z_k} g^b(t;w,z) \lambda^{-1} \del_{w_a} .
%\]
%Thus the second part of \eqref{eqn:lambda2} is satisfied. 
%Similarly we observe that $\lambda \cdot \left(F^k(t;w,z) \del_{z_k} \right)  = \lambda^{-2} F^k(\lambda t;\lambda w,z) \del_{z_k}$. 
%
%The last case is when $\gamma_1(t;w,z) = f^a(t;w,z) \d w_a$ and $\gamma_2(t;w,z) = g^b(t;w,z) \d w_b$where $f^a,g^b$ are de Rham--Dolbeault forms for and $a,b=1,2$.
%The argument is nearly identical so we omit the details. 
%\end{proof}

 
This result induces a product decomposition 
\begin{equation}
\label{eqn:Gdecomp}
\Omega^{0,\bullet}_{X} (\mc L^{{r=0}}_{AdS_7}) = \prod_{n \geq -1} \mc{G}_{X}^{(n)}
\end{equation}
where
\[
\mc{G}_X^{(n)}\subset \Omega^{0,\bullet}_{X} (\mc L^{{r=0}}_{AdS_7} )
\]
is the weight $n$ eigenspace with respect to the above $\C^\times$ action.  
In particular, we see that $\mc{G}^{(0)}_{X}$ is itself a local Lie algebra (that we will soon describe). 
Moreover, every $\mc{G}_{X}^{(n)}$, $n \geq -1$ is a (local) module for this local dg-Lie algebra.

%The $\C^\times$ action is compatible with the $\Z/2$ graded $L_\infty$ structure on $\mc{L}_{sugra}$. 
%Since the $n$th eigenspace is trivial when $n < -1$, we see that the product
%\[
%\mc{G}_Z (U)  \prod_{n \geq -1} (\pi_*\mc{L}_{sugra})(U)^{(n)}
%\]
%is equipped with the structure of a $\Z/2$ graded $L_\infty$ algebra.
%In this way, the assignment 
%\[
%\Bar{\pi}_* \mc{L}_{sugra} : U \mapsto (\Bar{\pi}_* \mc{L}_{sugra})(U) 
%\]
%defines a sheaf of $\Z/2$ graded $L_\infty$ algebras on $Z$. 

%\begin{lem}
%\label{lem:technical}
%For each $n$, $\mc{G}
%\[
%U \mapsto \Omega^{0,\bullet}(U, \mc{V}_{fivebrane}^{(n)})
%\]
%for some finite rank super holomorphic vector bundle $\mc{V}_{fivebrane}^{(n)}$ on $Z$.
%For each $n$, the differential is of the form $\dbar + Q^{hol}$.
%In particular, this endows $\Bar{\pi}_* \mc{L}_{sugra}$ with the structure of a pro-vector bundle on $Z$. 
%\end{lem}
%
%\begin{proof}
%In \S \ref{s:Lsugra} we introduced a holomorphic vector bundle $L_X$ defined on any Calabi--Yau five-fold $X$ whose sheaf of holomorphic sections was equipped with a $\Z/2$ graded $L_\infty$ structure. 
%Here, we consider the Calabi--Yau fivefold $X = X_Z$ as defined above.
%For any open subset $U \subset Z$ of the threefold the value of $\mc{L}_{sugra}(\pi^{-1}U)$ as a sheaf of super vector spaces is 
%\[
%\Omega^\bullet(\R) \otimes \Omega^{0,\bullet}(p^{-1} U, L_{X_Z}),
%\]
%where $p : X_Z \to Z$ is the rank two bundle over $Z$. 
%The one-ary structure map, or differential, is of the form $\d_{dR} + \dbar + Q^{hol}$ where $Q^{hol}$ is some odd holomorphic differential operator acting on sections of $L_{X_Z}$. 
%
%Since the de Rham complex of $\R$ is contractible we have a quasi-isomorphism of $L_\infty$ algebras
%\[
%\mc{L}_{sugra}(\pi^{-1}U) \simeq \Omega^{0,\bullet}(p^{-1} U, L_{X_Z})
%\]
%for every $U \subset Z$. 
%Without loss of generality, we can assume that $U$ is a coordinate chart for the vector bundle $X_Z$ so that $X_Z|_U \cong U \times \C^2$ and where the bundle $L_{X_Z}$ splits as 
%\[
%L_{X_Z}|_{\pi^{-1} U} \cong L' \boxtimes L'' 
%\]
%where $L'$ is a holomorphic super vector bundle on $U$ and $L''$ is.a holomorphic vector on the fiber $\C^2$.
%Then, we can further identify this with a Dolbeault complex of the form
%\[
%\Omega^{0,\bullet} \left(U, L' \otimes \Omega^{0,\bullet}(\C^2, L'') \right) . 
%\]
%By the Dolbeault Poincar\'e lemma applied to the holomorphic vector bundle $L''$ over $\C^2$,
%\[
%\Omega^{0,\bullet} \left(U, L' \otimes M\right) 
%\]
%where we now interpret $L' \otimes M$ as an infinite rank bundle over $U \subset Z$---
%the remaining differential is $\dbar_U + Q^{hol}$ where $Q^{hol}$ is a holomorphic differential operator.
%
%With these simplifications, it suffices to show that the weight $n$ eigenspace of $L' \otimes M$ is finite rank over $U$ \brian{almost finished}
%%we can fix a quasi-isomorphism $M \simeq \Omega^{0,\bullet}(\C^2, L'')$ where $M$ is a 
%\end{proof}


\subsection{The lowest piece}

The first non trivial case is the weight $(-1)$ piece.

\begin{lem}
There is an equivalence of abelian local Lie algebras 
\[
\mc{G}_{X} ^{(-1)}\cong \Omega^{0,\bullet}_{X} \left ( 
\begin{tikzcd}
\ul{+} & \ul{-} \\
\C^2\otimes K^{1/2}_{X} & \\ 
\mc{O}_{X} \ar[r, "\del"] & \Omega^1_{X} 
\end{tikzcd}
\right )
\] 
\end{lem}
\begin{proof}
We readily see that the fields of weight $-1$ include
\begin{itemize}
\item 
fields of type $\mu$ of the form $\mu_a (z_i)\del_{w_a}$. As $w_a$ are fiber coordinates on $K^{1/2}_{X}$, these fields transform as sections of $K^{1/2}_{X}$. 
\item 
fields of type $\beta$ with no $w_a$-dependence. These fields constitute a copy of $\mc {O}_{X}$.
\item 
fields of type $\gamma$ of the form $\gamma_i(z_i)\d z_i$. These fields constitute a copy of $\Pi \Omega^1_{X}$. 
\end{itemize}
Since $\del$ is weight zero for this $\C^\times$ action, the fields of the last two type combine to give the complex of sheaves
\[
\mc{O}_{X}  \xto{\del} \Pi \Omega^{1}_{X}  .
\]

\end{proof}

\parsec[]
In \cite{SWtensor} Saberi and Williams, the authors studied the minimal twist of the 6d $\mc N=(2,0)$ abelian tensor multiplet. The twist is a free theory and can be defined on any complex three-fold admitting a square root of its canonical bundle. On $X$, the $\Z\times \Z/2$ graded sheaf of complexes $\mc E_{tens}$ encoding its field content is given by 

\begin{equation}
\begin{tikzcd}
\ul{-1} & \ul{0} \\
\Pi \C^2\otimes \Omega^{0,\bullet}_{X} \otimes  K_{X}^{1/2}  & \\
\Omega^{2,\bullet}_{X} \ar[r,"\del"] & \Omega^{3,\bullet}_{X} 
\end{tikzcd} 
\end{equation}
Here we recall in the $\Z \times \Z/2$ bigrading the differential has bidegree $(1,0)$. 

We observe the following:
\begin{prop}
\label{prop:factabelian}
There is a quasi-isomorphism of factorization algebras valued in $\Z/2$ graded commutative dg algebras on $X$
\[
\mc U (\mc G^{(-1)}_{X} )\xto{\simeq} \clie^\bullet(\Pi \mc E_{tens})
\]
\end{prop}

\begin{proof}
Recall that the factorization algebra $\mc U(\mc{G}_{X}^{(-1)})$ assigns to an open set $U\subset X$ the graded symmetric algebra on the complex
\begin{equation}\label{eqn:weight-1}
\begin{tikzcd}
\ul{-} & \ul{+}\\
\Omega_{X, c}^{0,\bullet}(U) \ar[r,"\del"] & \Omega_{X, c}^{1,\bullet}(U) \\
\Omega_{X, c}^{0,\bullet}(U, \C^2\otimes K^{1/2}) & 
\end{tikzcd}
\end{equation}
On the other hand, if we totalize the $\Z\times\Z/2$-grading on $\mc E_{tens}$ to a $\Z/2$-grading, the factorization algebra $\clie^\bullet (\Pi \mc E_{tens})$ assigns to an open set $U\subset X$ the symmetric algebra on the complex 
\begin{equation}
\begin{tikzcd}
\ul{-} & \ul{+}\\
\Omega_{X}^{2,\bullet}(U)^\vee \ar[r,"\del"] & \Omega_{X}^{3,\bullet}(U)^\vee \\
& \Omega_{X}^{0,\bullet}(U, \C^2\otimes K^{1/2})^\vee \\
\end{tikzcd}
\end{equation}

Here the superscript refers to the topological dual, which is described in terms of compactly supported distributional sections of the Serre dual vector bundle. Thus, we see that the above complex is the same as 

\begin{equation}
\begin{tikzcd}
\ul{-} & \ul{+}\\
\overline \Omega_{X, c}^{0,\bullet}(U) \ar[r,"\del"] & \overline \Omega_{X, c}^{1,\bullet}(U) \\
\overline \Omega_{X, c}^{0,\bullet}(U, \C^2\otimes K^{1/2}) & 
\end{tikzcd}
\end{equation}
where the degree shift is coming from Serre duality. The result then follows from the fact that by ellipticity, the natural inclusion $\Omega^{0,\bullet}_{X,c}\to\overline \Omega^{0,\bullet}_{X,c}$ is a quasi-isomorphism.
\end{proof}

\subsection{The zero-th piece: a local version of $E(3|6)$}

As before, the weight zero summand $\mc{G}_{X} ^{(0)}$ is special because it carries the induced structure of a local $L_\infty$-algebra on ${X} $ inherited from the $L_\infty$ structure on $\Pi\Omega^{0,\bullet} (\mc L^{r = 0}_{AdS_7} )$ identified in section \ref{sec:transversebc}. We will prove that it is equivalent to a local Lie algebra version of the exceptional super-Lie algebra $E(3|6)$ \cite{KacBible}.

We first recall the definition of this super-Lie algebra. 

\begin{defn}\label{defn:e(3|6)}
Let $E(3|6)$ be the following super-Lie algebra.
\begin{itemize}
\item The even part, $E(3|6)_0$ is given by the semidirect product Lie algebra $\Gamma(\widehat D, T) \ltimes \left ( \mf{sl}(2) \otimes \Gamma (\widehat D, \mc O) \right )$. 
\item The odd part, $E(3|6)_1$ is given by $\C^2\otimes \Gamma (\widehat D, \Omega^1(K^{-1/2} ))$. 
\end{itemize}
The remaining brackets to be specified, are as follows:
\begin{itemize}
\item The action of $E(3|6)_0$ on $E(3|6)_1$ is given by the Lie derivative, along with the fundamental action of $\mf{sl}(2)$. 
\item The bracket between two odd elements is given by 
\begin{align*}
[v_1\otimes f_i \d z_i \otimes (\del_{z_1}\del_{z_2}\del_{z_3})^{1/2} &, v_2\otimes g_j \d z_j \otimes (\del_{z_1}\del_{z_2}\del_{z_3})^{1/2} ]  \\ 
 = & \  \omega (v_1, v_2) \eps^{ijk} f_i g_j \del_{z_k}  \\
 + & \  (v_1 \odot v_2) \left ( \del (f_i \d z_i ) g_j \d z_j  - f_i \d z_i \del (g_j \d z_j)  \right )\vee (\del_{z_1}\del_{z_2}\del_{z_3} )
\end{align*}
where $\omega$ denotes a symplectic form on $\C^2$ and $\odot: \C^2\otimes \C^2 \to \mf{sl}(2)$ is the canonical $\mf {sl}(2)$-equivariant projection. 
\end{itemize}
\end{defn}

The relationship between this super-Lie algebra and our decomposition is established through the following result.

\begin{prop}\label{prop:g0e36}
There is an equivalence of super-Lie algebras
\[
J^\infty_0\mc G^{(0)}_{\C^3} \cong E(3|6)
\]
\end{prop}
where the left-hand side denotes the $\infty$-jets at zero of the local $L_\infty$ algebra $\mc G^{(0)}_{\C^3}$. 
\begin{proof}
We begin by characterizing the local $L_\infty$-algebra $\mc G^{(0)}_{\C^3}$. We claim that it is quasi-isomorphic to a local version of $E(3|6)$. 

Indeed, we readily see that the weight zero sections consist of the following cochain complex

\begin{equation}
\Omega^{0,\bullet}_{\C^3} \left (
\begin{tikzcd}
\ul{even} & \ul{odd} \\
\C\{w_a\del_{w_b}\}\otimes \mc O\ar[r, "\del^W_\Omega" description]  & \mc O \\ 
\T \ar[ur, "\del^Z_\Omega" description] & \\
\C^2\otimes \mc O \ar[r, "\del_W" description]\ar[dr, "\del_Z" description] & \C\{\d w_a\} \otimes K^{-1/2}\ \\
& \C^2 \otimes \Omega^1\otimes K^{-1/2}
\end{tikzcd} \right)
\end{equation}

The differentials are again components of the divergence operator and holomorphic deRham operator. We can compute cohomology by way of a spectral sequence whose first page is the cohomology with respect to $\del^W_\Omega + \del_W$. We see that the differential $\del^W_\Omega$ maps surjectively onto functions and its kernel is isomorphic to $\mf {sl}(2) \otimes \mc O$. Likewise, the differential $\del_W$ is just the identity map between $\C^2$ and $\C\{\d w_a\}$. 

Thus we see that this page of the spectral sequence is given by 
\begin{equation}
\mc E(3|6) \define \Omega^{0,\bullet}_{\C^3} \left (
\begin{tikzcd}
\ul{even} & \ul{odd} \\
\T & \C^2\otimes \Omega^1_{\C^3} (K^{-1/2}) \\
\mf {sl}(2)\otimes \mc O
\end{tikzcd} \right)
\end{equation}
and there are no non-zero differentials so the spectral sequence degenerates.

To see that the Lie structure induced from the $L_\infty$-structure on $\Omega^{0,\bullet}_{\C^3} (\mc L^{r=0}_{AdS_7} )$ is in fact given by the same formulae as the brackets on $E(3|6)$ given in \ref{defn:e(3|6)}, it will be useful to provide an explicit quasi-isomorphism $\Phi^{(0)} : \mc E(3|6) \to \mc G^{(0)}_{\C^3}$. On an open set $U\subset \C^3$, this is defined as follows.

\begin{itemize}
\item Given a section $g_i(z)\del_{z_i}\in \Omega^{0,\bullet}_{\C^3} (U, T)$ where $g_i(z)$ is a Dolbeault form on $U$, we define
\begin{align*}
\Phi^{(0)} (g_i (z) \del_{z_i} ) &= g_i (z) \del_{z_i} - \frac 12(\del_{z_i} g_i (z)) w_a \del_{w_a} \\
&\in \Omega^{0,\bullet}_{\C^3} \left ( U, T \oplus \C \{w_a \del_{w_b} \} \otimes \mc O \right ).  
\end{align*}

\item Given a section $A\otimes g(z)\in \Omega^{0,\bullet}_{\C^3} (U,\mf {sl}(2) \otimes \mc O)$ where $g(z)$ is a Dolbeault form on $U$, and $A_{ab}\in \mf {sl}(2)$ we define
\begin{align*}
\Phi^{(0)} (A_{ab}\otimes g(z) ) &=  g(z) A_{ab}w_a\del_{w_b}\\
&\in \Omega^{0,\bullet}_{\C^3} \left ( \C \{w_a \del_{w_b} \} \otimes \mc O \right ).  
\end{align*}

\item Given a section $v\otimes g_i(z)\d z_i (\del_{z_1}\del_{z_2}\del_{z_3} ) \in \Omega^{0,\bullet}_{\C^3} (U, \C^2\otimes \Omega^1\otimes K^{-1/2})$ where $g_i(z)$ is a Dolbeault form on $U$ and $v\in \C^2$, we define
\begin{align*}
\Phi^{(0)} (v\otimes g_i(z)\d z_i (\del_{z_1}\del_{z_2}\del_{z_3} ) ) &= (w_1(v)w_1+w_2(v)w_2) \otimes g_i(z)  \\
&\in \Omega^{0,\bullet}_{\C^3} \left ( \C^2\otimes \Omega^1\otimes K^{-1/2}  \right ).  
\end{align*}
\end{itemize}

The result then follows from computing the limit of $\mc E(3|6) (D^3)$ over open sets containing the origin.
\end{proof}
%
%\begin{rmk}
%We note that the map $i_{M5}$ from lemma \ref{lem:m5emb} in fact defines a Lie map from $\mf{osp}(6|2)$ to the sections of the boundary condition $\Pi \Omega^{0,\bullet}_{\C^3} (\mc L^{r=0}_{AdS_7} )$ over every open set containing the origin. The image of the map lands exactly in the step $\mc G^{(0)}_{\C^3}$ of the decomposition from proposition \ref{prop:ads7decomp}. Therefore we see that $E(3|6)$ contains $\mf{osp}(6|2)$ as a finite dimensional subalgebra.  
%\end{rmk}
%\begin{rmk}
%In \cite{KacClass} an embedding of super Lie algebras from $E(3|6)$ into $E(5|10)$ is constructed.
%Recall that in the case $Z = \C^3$, the $\infty$-jets at $0 \in \C^3$ of the local Lie algebra $\mc{E}(3|6)$ is precisely the exceptional super Lie algebra $E(3|6)$.
%Similarly, the $\infty$-jets at $0 \in \C^3$ of the local $L_\infty$ algebra $\mc{G}_{\C^3}$ is $\widehat{E(5|10)}$.
%The embedding of local $L_\infty$ algebras on $Z$ from $\mc{E}(3|6)$ into $\mc{G}_Z$ that we just described agrees with the embedding of \cite{KacClass} upon taking $\infty$-jets.
%(Note that the central term in $\widehat{E(5|10)}$ plays no role herem since it sits in $\C^\times$-weight $-1$.)
%\end{rmk}

%\begin{itemize}
%\item For $\Vect^{hol}(Z)$ there is the standard commutator of holomorphic vector fields. 
%This acts on the sections of $\mf{sl}(2) \otimes \mc{O}_Z$ and $\Pi\Omega^{1,hol}(Z, K^{-1/2}_Z \otimes \C^2)$ by Lie derivative. 
%\item On sections of $\mf{sl}(2) \otimes \mc{O}_Z$ there is the matrix commutator. 
%This acts on the odd part $\Pi\Omega^{1,hol}(Z, K^{-1/2}_Z \otimes \C^2)$ where we view $\C^2$ as the fundamental $\mf{sl}(2)$ representation. 
%\item Finally, and most interestingly, there is a bracket of the form
%\[
%\Pi\Omega^{1,hol}(Z, K^{-1/2}_Z \otimes \C^2) \times \Pi\Omega^{1,hol}(Z, K^{-1/2}_Z \otimes \C^2) \to \Vect^{hol}(Z) \oplus \mf{sl}(2) \otimes \mc{O}_Z 
%\]
%given in coordinates by
%\begin{multline}
%[f^{ai}(z) w_a \d z_i, g^{bj} (z) w_b \d z_j] = \epsilon_{ijk} \epsilon_{ab} f^{ai} (z) g^{bj}(z) \del_{z_k} \\ + \epsilon_{ijk} \epsilon_{bc}  \del_{z_k} f^{ai}(z) g^{bj}(z) w_a \del_{w_c} + \epsilon_{ijk} \epsilon_{ac} f^{ai}(z) \del_{z_k} g^{bj}(z) w_b \del_{w_c} .
%\end{multline}
%The top line is a holomorphic vector field on~$Z$ and the bottom line is a $\mf{sl}(2)$-valued holomorphic function on~$Z$.
%\end{itemize}


%\parsec[s:mainfact]
%
%Since $\mc{G}_Z$ is a pro-vector bundle with a compatible $L_\infty$ structure, the assignment 
%\[
%U\subset Z \mapsto \clie^\bullet\left(\Bar{\pi}_* \mc{L}_{sugra}(U)\right) 
%\]
%has the structure of a factorization algebra on the three-fold $Z$. 
%We denote this factorization algebra on $Z$ by $\Obs_{sugra}|_Z$. 

%\begin{prop}
%Let $Z$ be a Calabi--Yau three-fold and $\pi : Z \times \C^2 \times \R \to Z$ be the projection. 
%There is a pro local Lie algebra $\Bar{\pi}_*\mc{L}_{sugra}$ and a factorization algebra $\Bar{\pi}_* \Obs_{sugra} = \clie^\bullet(\Bar{\pi}_* \mc{L}_{sugra})$ on $Z$ such that:
%\begin{itemize}
%\item[(1)] there is a natural inclusion of of factorization algebras on $Z$
%\[
%\Bar{\pi}_* \Obs_{sugra} \hookrightarrow \pi_* \Obs_{sugra}
%\]
%which is dense at the level of cohomology. 
%\item[(2)] there is a weight grading on $\mc{L}_{\pi,sugra}$ which is concentrated in degrees $\geq -1$ and gives rise to a decomposition of vector bundles
%\begin{equation}\label{eqn:decomp3}
%\Bar{\pi}_{*} \mc{L}_{sugra} = \prod_{n \geq -1} \mc{V}_{n} 
%\end{equation}
%\item[(3)]
%In weight zero, there is an equivalence of local Lie algebras on $Z$ 
%\[
%\mc{V}_0 \simeq \mc{E}(3|6)|Z 
%\]
%where $\mc{E}(3|6)|Z$ is a local Lie algebra enhancement of the exceptional simple super Lie algebra $E(3|6)$. 
%\end{itemize}
%\end{prop}

%We want to argue that $\Obs_{sugra} |_{\C^3} \cong \Bar{\pi}_* \Obs_{sugra}$. 

\subsection{General summands and $E(3|6)$-modules}

We move on to give the following general description of the weight $j$ component $\mc{G}_{\C^3} ^{(j)}$.
Since we have already described $j = -1,0$ we focus on $j \geq 1$.

\begin{prop}
\label{prop:Vj}
Let $j \geq 1$. 
The complex of vector bundles $\mc{G}^{(j)}$ is quasi-isomorphic to
\begin{equation}
\label{eqn:Gj}
\Omega^{0,\bullet}_{X} \left (\begin{tikzcd}
\ul{even} & \ul{odd} \\
\Sym^{j}(\C^2) \otimes \T \otimes K^{-j/2} & \Sym^{j-1}(\C^2) \otimes K^{-(j+1)/2}\\
\Sym^{j+2}(\C^2) \otimes K^{-j/2} & \Sym^{j+1}(\C^2) \otimes \T^* \otimes K^{-(j+1)/2} .
\end{tikzcd}\right )
\end{equation}
\end{prop}

\begin{proof}
We begin by noting that we can explicitly describe the weight $j$ component $\mc G^{(j)}_{X}$ as $\Omega^{0,\bullet}_{X} (G^{(j)})$ where $G^{(j)}$ is the following dg-vector bundle
\begin{equation}
\begin{tikzcd}
\ul{even} & \ul{odd} \\
\Sym^{j+1}(\C^2) \otimes \C^2 \ar[dr, "\del^W_\Omega"'] \otimes K^{-j/2}\\ 
& \Sym^j (\C^2) \otimes K^{-j/2} \\ 
\Sym^j (\C^2) \otimes  \T \otimes K^{-j/2} \ar[ur, "\del^Z_\Omega"]  & \\
& \Sym^{j+1}(\C^2)  \otimes  \Omega^1 \otimes K^{-(j+1)/2} \\ 
\Sym^{j+1}(\C^2) \otimes K^{-(j+1)/2} \ar[ur, "\del_Z"] \ar[dr, "\del_W"'] \\
& \Sym^j(\C^2) \otimes \C^2\otimes K^{-(j+1)/2} 
\end{tikzcd}
\end{equation}

Note that the differentials here are $\mf{sl}(2)$-equivariant maps, tensored with a differential operator acting on sections of a bundle on $X$. In particular

\begin{itemize}
\item The differential $\del^W_\Omega$ is the canonical projection \[\Sym^{j+1}(\C^2) \otimes \C^2 \cong \Sym^{j+2} (\C^2) \oplus \Sym ^j (\C^2) \twoheadrightarrow \Sym^j (\C^2)\] tensored with the identity acting on $K^{-j/2}$. 

\item The differential $\del_W$ is the canonical inclusion \[\Sym^{j+1}(\C^2) \hookrightarrow \Sym^{j-1}(\C^2) \oplus \Sym^{j+1}(\C^2) \cong \Sym^j(\C^2) \otimes \C^2 .\] tensored with the identity acting on $K^{-(j+1)/2}$.
\end{itemize}

There is a spectral sequence whose first term is computed by the $\del^W_\Omega +\del_W$-cohomology. The result is the complex of sheaves in equation \ref{eqn:Gj}. There are no further differentials so the spectral sequence collapses at this page and the result follows.
\end{proof}

\iffalse
\parsec[]
We wish to explicate the $E(3|6)$-module structure on each of the costalks $\Omega^{0,\bullet}_{\C^3, c} (G^{(j)} )(0)$. To do so, it will be useful to explicate the quasi-isomorphism whose existence was determined in the proof of the preceding proposition. To this end, we describe an explicit quasi-isomorphism 
\[\Phi: \Omega^{0,\bullet}_{\C^3} (G^{(j)} ) \to \mc G^{(j)}_{\C^3}\] 
as follows. 

Let $U\subset \C^3$ be open. 
\begin{itemize}
\item Given a section
\[
f(w) \otimes g_i(z) \del_{z_i}\in \Omega^{0,\bullet}_{\C^3} \left (U, \Sym^j(\C^2) \otimes \T \otimes K^{-j/2} \right )
\] where $f(w) \in \Sym^j (\C^2)$ is a homogenous degree $j$ polynomial in $w_1,w_2$ and the $g_i(z)$'s are Dolbeault forms on $\C^3$, we define
\begin{align*}
\Phi(f(w) \otimes g_i(z) \del_{z_i}) &= f(w) g(z) \del_{z_i} - \frac{1}{j+2} \left(\del_{z_i} g_i(z)\right) f(w) w_a\del_{w_a} \\
&\in \Omega^{0,\bullet}_{\C^3} \left ( U, (\Sym^j (\C^2) \otimes T\otimes K^{-j/2} )\oplus ( \Sym^{j+1}(\C^2) \otimes \C^2 \otimes K^{-j/2} ) \right )
\end{align*}

\item Given a section
\[
f(w) \otimes g(z)\in \Omega^{0,\bullet}_{\C^3} \left (U, \Sym^{j+2} \otimes K^{-j/2} \right)
\] where $f(w) \in \Sym^{j+2} (\C^2)$ is a homogenous degree $j+2$ polynomial in the variables $w_1,w_2$ and $g(z)$ is a Dolbeault form on $\C^3$, we define
\begin{align*}
\Phi ( f(w) \otimes g(z) ) &= g(z) (\del _{w_1} f(w) \del_{w_2} - \del _{w_2} f(w) \del_{w_1} ) \\
&\in \Omega^{0,\bullet}_{\C^3} \left ( U, \Sym^{j+1} (\C^2) \otimes T \otimes K^{-j/2}\right )
\end{align*}

\item Given a section
\[
f(w) \otimes g(z)  \in \Omega^{0,\bullet}_{\C^3} \left (U, \Sym^{j-1}(\C^2) \otimes K^{-(j+1)/2} \right)
\]
where $f(w) \in \Sym^{j-1} (\C^2)$ is a homogenous degree $j-1$ polynomial in the variables $w_1,w_2$ and $g(z)$ is a Dolbeault form on $Z$ we define
\begin{align*}
\Phi( f(w) \otimes g(z) ) &= \frac12 g(z)f(w)(w_1 \d w_2 - w_2 \d w_1)  \\
& \in \Omega^{0,\bullet}_{\C^3}\left  (U, \Sym^j (\C^2) \otimes \C^2 \otimes K^{-(j+1)/2} \right )
\end{align*}

\item
Given a section
\[
f(w) \otimes g^i(z)  \d z_i\in \Omega^{0,\bullet}_{\C^3} \left (U, \Sym^{j+1}(\C^2) \otimes T^* \otimes K^{-(j+1)/2} \right)
\] 
where $f(w) \in \Sym^{j+1} (\C^2)$ and the $g^i(z)$'s are Dolbeault forms on $\C^3$, we define 
\begin{align*}
\Phi(f(w) \otimes g^i(z) \d z_i) & = f(w) g^i (z) \d z_i \\
& \in \Omega^{0,\bullet}_{\C^3} \left (U, \Sym^{j+1} (\C^2) \otimes \Omega^1 \otimes K^{-(j+1)/2} \right )
\end{align*}
\end{itemize}

The following is immediate:

\begin{lem}\label{lem:qisG}
The map $\Phi: \Omega^{0,\bullet}_{\C^3} (G^{(j)} ) \to \mc G^{(j)}_{\C^3}$ is a quasi-isomorphism.
\end{lem}


\parsec[]
Recall that as an immediate corollary of proposition \ref{prop:ads7decomp}, the local super-Lie algebra $\mc E(3|6)$ acts on the abelian local Lie algebras $\mc G^{(j)} _{\C^3}$. By way of the explicit quasi-isomorphisms  $\mc{G}^{(0)} \simeq \mc{E}(3|6)$ and $\mc{G}^{(j)} \simeq \Omega^{0,\bullet}_{\C^3}( G^{(j)})$ we can explicate this module structure.

Recall that $\mc{E}(3|6)$ was given by  $\Omega^{0,\bullet}_{\C^3} (G^{(0)})$ where 
\begin{equation}
G^{(0)} = 
\begin{tikzcd}
\ul{even} & \ul{odd} \\
\T & \C^2\otimes \Omega^1_{\C^3} (K^{-1/2}) \\
\mf {sl}(2)\otimes \mc O
\end{tikzcd}
\end{equation}
The local Lie algebra structure on $\mc{E}(3|6)$ arises from a Lie algebra structure on the sheaf of holomorphic sections of $G^{(0)}$. Likewise, the $\mc{E}(3|6)$-module structure on $\mc{G}^{(j)} \simeq \Omega^{0,\bullet}_{\C^3} (G^{(j)})$ arises from an action of the sheaf of holomorphic sections of $G^{(0)}$ on the sheaf of holomorphic sections of $G^{(j)}$, where the action is expressed through holomorphic differential operators.

Let us explicate this action. It is easy to see how holomorphic sections of the even part of the super-vector bundle $G^{(0)}$ will act. Sections of $T$ are holomorphic vector fields and act by Lie derivative. Likewise, sections of $\mf{sl}(2) \otimes \mc O$ act via the corresponding $\mf{sl}(2)$-representation.

We need only explain how holomorphic sections of the odd component $\C^2 \otimes \T^* \otimes K^{-1/2}$ of $G^{(0)}$ act. We first give a global description of this action, and then we will write down the explicit formula in local coordinates. In local coordinates, we may express such a section as  $w_a g^i(z) \d z_i$ where $g^i(z)$ is a holomorphic function. The action is as follows:
 
\begin{itemize}
\item The odd part of $\mc{V}^{(0)}$ acts on the component $S^{j}(\C^2) \otimes \T_Z \otimes K^{-j/2}_Z$ through the composition
\begin{equation}
\begin{tikzcd}
\left(\T^*_Z \otimes K^{-1/2}_Z \otimes \C^2\right) \otimes \left(S^{j}(\C^2) \otimes \T_Z \otimes K^{-j/2}_Z\right) \ar[r,"\cong"] & \left(S^{j-1}(\C^2) \oplus S^{j+1}(\C^2)\right) \otimes \left(\T_Z \otimes \T^*_Z \otimes K^{-(j+1)/2}_Z\right) \ar[dl] \ar[d] \\
S^{j-1}(\C^2) \otimes K^{-(j+1)/2}_Z & S^{j+1}(\C^2) \otimes \T^*_Z \otimes K^{-(j+1)/2}_Z 
\end{tikzcd}
\end{equation}
Here, the leftmost downward arrow is the evident $\mf{sl}(2)$ projection together with the canonical pairing between sections of $\T_Z$ and $\T^*_Z$.
The rightmost downward arrow is the other $\mf{sl}(2)$ projection together with the Lie derivative of holomorphic one-forms.  
Given a local section $f(w) \otimes h_i(z) \del_{z_i}$ of $S^j(\C^2) \otimes \T_Z \otimes K_Z^{-j/2}$ an explicit formula for this action is
\begin{align*}
(w_a g^i(z) \d z_i) \cdot (f(w) \otimes h_k(z) \del_{z_k}) & = \epsilon_{ab} (\del_{w_b} f(w)) (g^i h_i)(z)  \\ & + w_a f(w) L_{h_k \del_{z_k}} (g^i \d z_i) .
\end{align*}
\item The odd part of $\mc{V}^{(0)}$ acts on the component $S^{j+2}(\C^2) \otimes K^{-j/2}_Z$ through the composition
\begin{equation}
\begin{tikzcd}
\left(\T^*_Z \otimes K^{-1/2}_Z \otimes \C^2\right) \otimes \left(S^{j+2}(\C^2) \otimes K^{-j/2}_Z\right) \ar[r,"\cong"] & \left(S^{j+1}(\C^2) \oplus S^{j+3}(\C^2)\right) \otimes \left(\T^*_Z \otimes K^{-(j+1)/2}_Z\right) \ar[d] \\
& S^{j+1}(\C^2) \otimes \T^*_Z \otimes K^{-(j+1)/2}_Z
\end{tikzcd}
\end{equation}
where the downward arrow is induced by the evident $\mf{sl}(2)$ projection.
\item The odd part of $\mc{V}^{(0)}$ acts on the component $S^{j-1}(\C^2) \otimes K^{-(j+1)/2}_Z$ through the composition
\begin{equation}
\begin{tikzcd}
\left(\T^*_Z \otimes K^{-1/2}_Z \otimes \C^2\right) \otimes \left(S^{j-1}(\C^2) \otimes K^{-j/2}_Z\right) \ar[r,"\cong"] & \left(S^{j-2}(\C^2) \oplus S^{j}(\C^2)\right) \otimes \left(\T^*_Z \otimes K^{-1}_Z K^{-j/2}_Z\right) \ar[d] \\
& S^{j}(\C^2) \otimes \T_Z \otimes K^{-j/2}_Z
\end{tikzcd}
\end{equation}
where the downward arrow is induced by the evident $\mf{sl}(2)$ projection together with the holomorphic de Rham operator taking holomorphic one-forms to holomorphic two-forms. 
\item Finally, the odd part of $\mc{V}^{(0)}$ acts on the component $S^{j+1}(\C^2) \otimes \T^*_Z \otimes K^{-(j+1)/2}_Z$ through the composition
\begin{equation}
\begin{tikzcd}
\left(\T^*_Z \otimes K^{-1/2}_Z \otimes \C^2\right) \otimes \left(S^{j+1}(\C^2) \otimes \T^*_Z \otimes K^{-j/2}_Z\right) \ar[r,"\cong"] & \left(S^{j}(\C^2) \oplus S^{j+2}(\C^2)\right) \otimes \left(\T^*_Z \otimes \T^*_Z \otimes K^{-1}_Z K^{-j/2}_Z\right) \ar[dl] \ar[d] \\
S^j (\C^2) \otimes \T_Z \otimes K_Z^{-j/2} & S^{j+2}(\C^2)  \otimes K^{-j/2}_Z
\end{tikzcd}
\end{equation}
where the leftmost downward arrow is induced by the evident $\mf{sl}(2)$ projection.
The rightmost downward arrow is induced by the remaining $\mf{sl}(2)$ projection together with the holomorphic de Rham operator taking holomorphic two-forms to holomorphic three-forms.
\end{itemize}
\fi

\parsec[s:kacrelation]

In the case $X = \C^3$, the decomposition of $\Omega^{0,\bullet}_{\C^3} (\mc L^{r=0}_{AdS_7} )$ in equation \eqref{eqn:Gdecomp} is closely related to a decomposition of the exceptional simple super Lie algebra $E(5|10)$ studied in \cite{KR2}. In \ref{thm:global}, we showed that the global sections of the parity shifted fields of our eleven-dimensional theory on flat space is quasi-isomorphic to a Lie 2-extension of $E(5|10)$, which we denoted $\widehat{E(5|10)}$. More precisely, we found a Lie 2-extension of a version of $E(5|10)$ built out of polynomials rather than Taylor series. Given that the $L_\infty$ structure on $\Pi \Omega^{0,\bullet}_{\C^3} (\mc L^{r=0}_{AdS_7} )$ is given by the same formulas as that on $\Pi \mc E$, it is easy to see that the space of $\infty$-jets of $\Pi \Omega^{0,\bullet}_{\C^3} (\mc L^{r=0}_{AdS_7} )$ at the origin, which following lemma \ref{} has underlying vector space $\mc H_{AdS_7}$, is quasi-isomorphic to $\widehat{E(5|10)}$. 

In \cite{KR2} the following weight decomposition of $E(5|10)$ is constructed. Splitting $\C^5 = \C^2_{w_a}\times \C^3_{z_i}$ as we have been doing, we stipulate that
\begin{itemize} 
\item the coordinate $z_i$ has weight zero. 
\item the coordinate $w_a$ has weight $+1$.
\item the parity of an element carries an additional weight of $-1$. 
Thus, for example, the odd element $[\d w_1 \d z_1] \in \Omega^{2,cl}(\widehat{D}^5)$ carries weight $+1 - 1 = 0$. Viewing the odd part as the space of closed two-forms, then equivalently this grading translates to the one-form symbol $\d(-)$ as carrying weight $-1/2$.
\end{itemize} 

It is straightforward to verify that this weight grading is compatible with the super Lie algebra structure on $E(5|10)$. Moreover, we see that similarly to the decomposition of $\mc H_{AdS_7}$ induced by our $\C^\times$ action in \ref{prop:ads7decomp}, the weight grading is concentrated in degrees $\geq -1$.  In particular, there is a decomposition of super vector spaces
\begin{equation}\label{eqn:decomp1}
E(5|10) = \tilde U_{-1} \times \prod_{j \geq 0} U_j 
\end{equation}
Further, this decomposition also has the property that the 0-th piece $U_0$ is isomorphic to $E(3|6)$. As such, each $U_j$ is an $E(3|6)$-module; Kac characterizes these modules explicitly and identifies them as certain irreducible $E(3|6)$-modules. In the notation of \cite{KR2} we have that $\tilde U_{-1} = I(0,0;1;-1)^*$ and for $j\geq 1$, $U_j = I(0,0;j-1;j+1)^*$. 

The decomposition of the local $L_\infty$-algebra $\Omega^{0,\bullet}_{\C^3}(\mc L^{r=0}_{AdS_7} )$ afforded by proposition \ref{prop:ads7decomp} induces a weight grading of $\widehat{E(5|10)}$ which extends the one on $E(5|10)$ that we have just described by declaring that the central term have weight $-1$.
In this way, we get a related decomposition of super $L_\infty$ algebras
\begin{equation}\label{eqn:decomp2}
\widehat{E(5|10)} = \prod_{j \geq -1} U_j            
\end{equation}                      
were $U_{-1}$ is a $\C$-extension of $\tilde U_{-1}$ defined in the decomposition \eqref{eqn:decomp1} and for $j \geq 0$ the $U_j$'s are the same as in the non centrally extended case. As a corollary, we see that each of the $E(3|6)$ modules which we have identified as the costalk at 0 $\mc G^{(j)}_{\C^3, c} (0)$ is in fact irreducible \surya{is this last sentence actually true? these aren't the same modules}

\subsection{Characters of $E(3|6)$-modules}\label{e36char}
The decomposition of the state space 
\[
\Sym \left (\mc H_{AdS_7}\right ) = \prod_{j\geq -1} \mc U (\mc G^{(j)})(0) 
\] 
gives a product formula for the characters computed in proposition \ref{prop:sugraindex1}

\[ \chi \left ( \Sym \mc H_{AdS_7} \right ) = \prod_{j\geq -1} \chi \left (\mc U (\mc G^{(j)})(0) \right ) \]

We end this section by computing each of the characters $\chi \left (\mc U (\mc G^{(j)})(0) \right )$. We will express our characters in terms of characters of highest weight representations of $\mf{sl}(2)$ and $\mf {sl}(3)$, which we denote by $\chi_k^{\mf{sl}(2)}$ and $\chi^{\mf{sl}(3)}_{[k,l]}$.

\parsec[]
We begin with the lowest step of the decomposition, using the characterization given in \ref{subsec:g-1}. 

\begin{prop}
\label{prop:6done}
The character $\chi \left ( \mc U (\mc G^{(-1)}_{\C^3} ) (0)\right )$ is given by the plethystic exponential of the following expression:
\begin{equation}\label{eqn:6done}
g_{-1} (t_1,t_2,r,q) = \frac{q^{3/2}(r + r^{-1}) - q^2(t_1 + t_1^{-1} t_2 + t_2^{-1} ) + q^3}{(1-t_1^{-1}q) (1-t_1 t_2^{-1} q) (1-t_2 q)} .
\end{equation}
\end{prop}
\begin{proof}
Note that in light of proposition \ref{prop:factabelian} we can equivalently compute the character of the costalk at the origin of $\clie^\bullet (\Pi \mc E_{tens})$. We first give a more explicit description of the costalk as a cochain complex. Proceeding exactly analogously to the proof of lemma \ref{lem:ads4states}, we see that the costalk is given by the symmetric algebra on the following cochain complex
\[
\begin{tikzcd}
\ul{-} & \ul{+}\\
\C\{\d z_i\d z_j\} \otimes \C [\del_{z_1}, \del_{z_2}, \del_{z_3}]\delta_{z_i = 0} \ar[r,"\del"] & \C\{\d z_1\d z_2\d z_3\} \otimes \C [\del_{z_1}, \del_{z_2}, \del_{z_3}]\delta_{z_i = 0} \\
& \C^2\otimes \C\{\d z_i^{1/2}\} \otimes \C [\del_{z_1}, \del_{z_2}, \del_{z_3}]\delta_{z_i = 0} 
\end{tikzcd}
\]

Computing summand-by-summand, we see:
\begin{itemize}
\item the odd summand $\C\{\d z_i\d z_j\} \otimes \C [\del_{z_1}, \del_{z_2}, \del_{z_3}]\delta_{z_i = 0}$ contributes
\[
- q^2 \frac{\chi^{\mf{sl}(3)}_{[0,1]}(t_1,t_2)}{(1-t_1^{-1}q) (1-t_1 t_2^{-1} q) (1-t_2 q)} = - q^2 \frac{t_1  + t_1^{-1} t_2  + t_2^{-1} }{(1-t_1^{-1}q) (1-t_1 t_2^{-1} q) (1-t_2 q)}.
\]
where $\chi^{\mf{sl}(3)}_{[1,0]}(t_1,t_2)$ is the $\mf{sl}(3)$ character of highest weight $[1,0]$.
\item the even summand $\C [\del_{z_1}, \del_{z_2}, \del_{z_3}]\delta_{z_i = 0}$ contributes
\[
q^3 \frac{1}{(1-t_1^{-1}q) (1-t_1 t_2^{-1} q) (1-t_2 q)}.
\]
\item the even summand $\C^2\otimes \C\{\d z_i^{1/2}\} \otimes \C [\del_{z_1}, \del_{z_2}, \del_{z_3}]\delta_{z_i = 0}$ contributes
\[
q^{3/2} \frac{\chi_{1}^{\mf{sl}(2)} (r)}{(1-t_1^{-1}q) (1-t_1 t_2^{-1} q) (1-t_2 q)} = q^{3/2}\frac{(r + r^{-1})}{(1-t_1^{-1}q) (1-t_1 t_2^{-1} q) (1-t_2 q)}.
\]
where $\chi_{1}^{\mf{sl}(2)} (r)$ is the $\mf{sl}(2)$ character of highest weight one.
\end{itemize}
\end{proof}

In terms of the parameters $y_1,y_2,y_3,y,q$ introduced in \ref{} this single particle character reads
\begin{equation}
\label{eqn:6done1}
g_{-1} (y_i,y,q) = \frac{qy + q^2y^{-1} - q^2(y^{-1}_1+y^{-1}_2+y^{-1}_3) + q^3}{(1-y_1q) (1-y_2 q) (1-y_3 q)} .
\end{equation}

The expression matches exactly with the index of the abelian six-dimensional superconformal theory. For example, compare with \cite[Eq. (3.1)]{Kim:2013nva} or \cite[Eq. (3.35)]{Bhattacharya:2008zy}.

From now on, we will give all formulas for the index in terms of the parameters $y_1,y_2,y_3,y,q$.

\parsec[]
We continue to the next step of the decomposition, which is given by  $\mc G^{(0)}_{\C^3}$.

\begin{prop}
The character $\chi \left( \mc U(\mc G^{(0)} _{\C^3}) (0)\right )$  is the plethystic exponential of following expression:
\begin{equation}\label{eqn:6dtwo}
g_{0} (y_i,y,q) = \frac{q^4(y_1+y_2+y_3) + q^2 (y^2 + q + q^2 y^{-2}) - q^{3} (y + q y^{-1})(y_1^{-1} + y_2^{-1} + y_3^{-1})}{(1-y_1q) (1-y_2 q) (1-y_3 q)}.
\end{equation}
\end{prop}
\begin{proof}
As usual, we wish to describe the costalk $\mc G^{(0)}_{\C^3, c} (0)$  more explicitly. By the same argument as in the proofs of propositions \ref{}, \ref{}, we may use elliptic regularity to describe the compactly supported smooth sections on a disc in terms of derivatives of the delta function at the origin in $\C^3$. 

Accordingly, we have contributions from the following summands.
\begin{itemize}
\item An even copy of $\C\{\del_{z_i} \} \otimes \C[\del_{z_1}, \del_{z_2}, \del_{z_3}] \delta_{z_i=0}$. The character of this summand is
\[
q^4 \frac{\chi^{\mf{sl}(3)}_{[1,0]}(y_i)}{(1-y_1q) (1-y_2 q) (1-y_3 q)}  = q^4 \frac{y_1 + y_2 + y_3}{(1-y_1q) (1-y_2 q) (1-y_3 q)}.
\]
\item An even copy of $\mf{sl}(2) \otimes \C[\del_{z_1}, \del_{z_2}, \del_{z_3}] \delta_{z_i=0}$. The character of this summand is
\[
q^3 \frac{\chi_2^{\mf{sl}(2)} (q^{-1/2}y)}{(1-y_1q) (1-y_2 q) (1-y_3 q)}  = \frac{q^2 y^2 + q^3 + q^4 y^{-2}}{(1-y_1q) (1-y_2 q) (1-y_3 q)}.
\]
\item An odd copy of $\C^2\otimes \C\{\d z_i \} \otimes \C[\del_{z_1}, \del_{z_2}, \del_{z_3}] \delta_{z_i=0}$. The character of this summand is 
\[
q^{7/2}\frac{\chi^{\mf{sl}(2)}_{1}(q^{-1/2} y) \, \chi_{[0,1]}^{\mf{sl}(3)} (y_i)}{(1-y_1q) (1-y_2 q) (1-y_3 q)} = q^{3}\frac{(y + q y^{-1})(y_1^{-1} + y_2^{-1} + y_3^{-1})}{(1-y_1q) (1-y_2 q) (1-y_3 q)} .
\]
\end{itemize}
\end{proof}

\parsec[]
Finally, we continue to the general step of the decomposition.

\begin{prop}
Let $j\geq 1$. The character $\chi \left( \mc U(\mc G^{(j)} _{\C^3}) (0)\right )$  is the plethystic exponential of following expression:
\begin{equation}\label{eqn:6dtwo}
g_{j} (y_i,y,q) =  \frac{ q^{3} \left( \begin{aligned} q^{1 + 3 j/2} \chi^{\mf{sl}(2)}_{j}(q^{-1/2} y)\chi^{\mf{sl}(3)}_{[1,0]}(y_i) + \  & q^{3j/2} \chi^{\mf{sl}(2)}_{j+2}(q^{-1/2} y)  \\
 - q^{3(j+1)/2} \chi^{\mf{sl}(2)}_{j-1}(q^{-1/2}y) - \ & q^{-1 + 3(j+1)/2} \chi^{\mf{sl}(2)}_{j+1} (q^{-1/2} y) \chi^{\mf{sl}(3)}_{[0,1]}(y_i) \end{aligned} \right)}{(1-y_1 q)(1-y_2q)(1-y_3q)}.
\end{equation}
\end{prop}
\begin{proof}
We proceed exactly analogously to all the previous cases. Using elliptic regularity on sections of $\mc G^{(j)}_{\C^3}$ over a disc containing the origin, we see that $\mc U (\mc G^{(j)}_{\C^3} )(0)$ is a symmetric algebra on a cochain complex with the following summands

\begin{itemize}
\item An even copy of $\Sym^j (\C^2) \otimes \C\{\del_{z_i}\}\otimes \C[\del_{z_1}, \del_{z_2}, \del_{z_3}]\delta_{z_i=0}\otimes (\del_{z_1}\del_{z_2}\del_{z_3})^{-j/2}$ which contributes 
\begin{equation}
\frac{ q^{3} \left(  q^{1 + 3 j/2} \chi^{\mf{sl}(2)}_{j}(q^{-1/2} y)\chi^{\mf{sl}(3)}_{[1,0]}(y_i) \right)}{(1-y_1 q)(1-y_2q)(1-y_3q)}
\end{equation} 

\item An even copy of $\Sym^{j+2} (\C^2) \otimes \C[\del_{z_1}, \del_{z_2}, \del_{z_3}]\delta_{z_i=0}\otimes (\del_{z_1}\del_{z_2}\del_{z_3})^{-j/2}$ which contributes 
\begin{equation}
\frac{ q^{3} \left( q^{3j/2} \chi^{\mf{sl}(2)}_{j+2}(q^{-1/2} y)  \right)}{(1-y_1 q)(1-y_2q)(1-y_3q)}
\end{equation} 

\item An odd copy of $\Sym^{j-1} (\C^2) \otimes \C[\del_{z_1}, \del_{z_2}, \del_{z_3}]\delta_{z_i=0}\otimes (\del_{z_1}\del_{z_2}\del_{z_3})^{-(j+1)/2}$ which contributes 
\begin{equation}
\frac{ - q^{3} \left( q^{3(j+1)/2} \chi^{\mf{sl}(2)}_{j-1}(q^{-1/2}y) \right)}{(1-y_1 q)(1-y_2q)(1-y_3q)}
\end{equation} 

\item An odd copy of $\Sym^{j+1} (\C^2) \otimes \C\{\d z_i\}\otimes \C[\del_{z_1}, \del_{z_2}, \del_{z_3}]\delta_{z_i=0}\otimes (\del_{z_1}\del_{z_2}\del_{z_3})^{-(j+1)/2}$ which contributes 
\begin{equation}
\frac{ - q^{3} \left( q^{-1 + 3(j+1)/2} \chi^{\mf{sl}(2)}_{j+1} (q^{-1/2} y) \chi^{\mf{sl}(3)}_{[0,1]}(y_i) \right)}{(1-y_1 q)(1-y_2q)(1-y_3q)}
\end{equation} 

\end{itemize}
\end{proof}

\parsec[]
As a consequence, we have that $f_{AdS_7} (y_i, y, q) = \sum_{j\geq -1} g_j (y_i, y, q)$, or explicitly:
\begin{align*}
& \frac{q^4(y_1+y_2+y_3)-q^2(y_1^{-1} + y_2^{-1} + y_3^{-1})+ (1-q^3)(yq + y^{-1} q^2)}{(1-y_1 q)(1-y_2 q)(1-y_3 q)(1-yq)(1-y^{-1} q^2)}  \\ 
& =  \frac{qy + q^2y^{-1} - q^2(y^{-1}_1+y^{-1}_2+y^{-1}_3) + q^3}{(1-y_1q) (1-y_2 q) (1-y_3 q)} \\
& +  \frac{q^4(y_1+y_2+y_3) + q^2 (y^2 + q + q^2 y^{-2}) - q^{3} (y + q y^{-1})(y_1^{-1} + y_2^{-1} + y_3^{-1})}{(1-y_1q) (1-y_2 q) (1-y_3 q)} \\
& + \sum_{j\geq 1} \frac{ q^{3} \left( \begin{aligned} q^{1 + 3 j/2} \chi^{\mf{sl}(2)}_{j}(q^{-1/2} y)\chi^{\mf{sl}(3)}_{[1,0]}(y_i) + \  & q^{3j/2} \chi^{\mf{sl}(2)}_{j+2}(q^{-1/2} y)  \\
 - q^{3(j+1)/2} \chi^{\mf{sl}(2)}_{j-1}(q^{-1/2}y) - \ & q^{-1 + 3(j+1)/2} \chi^{\mf{sl}(2)}_{j+1} (q^{-1/2} y) \chi^{\mf{sl}(3)}_{[0,1]}(y_i) \end{aligned} \right)}{(1-y_1 q)(1-y_2q)(1-y_3q)}
\end{align*}

In \cite[Eq. (3.22, 3.23)]{Bhattacharya:2008zy}, the index counting gravitons on $f_{AdS_7}$ is expressed as a sum of characters of irreducible representations of the 6d $\mc N = (2,0)$ superconformal algebra. In \cite[Table 24]{cordova2016multiplets} these representations are labeled as $\mc D_1[0,0,0]^{(0,m)}_{2m}$ where $m \geq 1$. The characters of these modules have been computed (see for example \cite[Eq. (166)]{Arai_2020} and match exactly with $g_{m-2}(y_i, y, q)$ after a suitable change of variables. Thus, we conjecture the following

\begin{conj}
For $j\geq -1$, the minimal twist of $\mc D_1[0,0,0]^{(0,j+2)}_{2(j+2)}$ is exactly $\mc G^{(j)}_{\C^3, c}(0)$.
\end{conj}

\begin{rmk}\label{rmk:e36enhance}
As we remarked in \ref{rmk:e16enhance}, this conjecture implies that the minimal twist of $\mc D_1[0,0,0]^{(0,j+2)}_{2(j+2)}$ which is a priori a module for the minimally twisted 6d $\mc N=(2,0)$ superconformal algebra $\mf{osp}(6|2)$, is in fact a module for the larger infinite dimensional super-Lie algebra $E(3|6)$. This can be thought of as analogous to the enhancement of conformal symmetries to the action of the Witt algebra of vector fields in 2d chiral conformal field theory.
\end{rmk}

\end{document}

%The subalgebra $\mc{G}^{(\geq k)} = \oplus_{j \geq k} \mc{G}^{(j)}$ is an ideal for every $k$. 
%This sequence of ideals induces a limit diagram of vector bundles
%\begin{equation}\label{eqn:lim}
%0 = \mc{G} / \mc{G}^{(\geq -1)} \leftarrow \mc{G} / \mc{G}^{(\geq 0)} \leftarrow \mc{G} / \mc{G}^{(\geq 1)} \leftarrow \cdots ,
%\end{equation}
%where $0$ is the zero vector bundle.
%For each $k$, we point out that $\mc{G} / \mc{G}^{(\geq k)}$ is genuinely a finite rank vector bundle on the worldvolume.
%The resulting filtration of the $!$-dual
%\begin{equation}
%\left(\Bar{\pi}_*\Obs_{sugra}\right)^! = \clie_\bullet(\mc{G}_{c})
%\end{equation} 
%of the factorization algebra \eqref{eqn:factgrad} is of the form
%\begin{equation}\label{eqn:fil1}
%\clie_\bullet (\mc{G}_{1,c}) \subset \clie_\bullet (\mc{G}_{2,c}) \subset \cdots \subset \clie_\bullet(\mc{G}_c) .
%\end{equation}
%Similarly, there is a filtration on the factorization algebra $\clie_\bullet (\tilde \mc{G}_c)$ of the form
%\begin{equation}
%\label{eqn:fil2}
%\clie_\bullet(\tilde \mc{G}_{2,c}) \subset \clie_\bullet(\tilde \mc{G}_{3,c}) \subset \cdots \subset \clie_\bullet(\tilde \mc{G}_c).
%\end{equation}



%\end{document}


%%\documentclass[11pt]{amsart}
%
%%\usepackage{../macros-master}
%\usepackage{macros-fivebrane}
%
%\begin{document}

\section{Local operators in twisted $M$ theory}

The notion of a factorization algebra captures both the local operators of a theory together with the non-local operators that on can define from the local ones via descent.
From the data of a factorization algebra, one can recover local operators by the following formal construction. 
Let $\Obs$ be the factorization algebra of observables of some theory defined on a smooth manifold $M$.
The space of local operators at point $p \in M$ is, in a precise sense, the limiting behavior of the factorization algebra evaluated on the system of open sets which contain the point~$p$. 

Generally this limit is difficult to compute, but for certain theories it is possible to give a concise expression which captures the essential features of the theory.
For example, in a holomorphic theory, the algebra of local operators is equivalent to the algebra generated by holomorphic derivatives of fields evaluated at a point.

In this section we recall the essentials of the theory of local operators for holomorphic-topological theories. 
We consider a way of counting operators in a topological-holomorphic theory, called the `local character' of a holomorphic-topological theory \cite{SWchar}, and compare it to the superconformal index.
We will then present a few examples including expressions for the local character on the twists of a theory on a single membrane and fivebrane and will find the expected match with the physics literature.

\subsection{Local operators in topological-holomorphic theories}

A factorization algebra encodes the many ways to combine observables supported on arbitrary open sets. 
Local operators, on the other hand, exist just at a point in spacetime.
From the factorization algebra perspective one can recover local operators by looking at observables which are supported on \text{every} open set which contains the given point; mathematically this is computed by a limit. 

Precisely, in \cite[Definition 10.1.0.1]{CG2} the space of local operators of a factorization algebra $\cF$ at a point $p \in M$ is defined by the limit $\cF(p) = \lim_{U \ni p} \cF(U)$ which runs over open sets $U \subset M$ containing~$p$.

We will only consider local operators on affine space $\R^d$. 
In this case, we will have the additional property that the factorization algebras are translation invariant.
At the level of local operators this means that the translation map $\tau_{p \to p'}$ induces an isomorphism $\cF(p) \simeq \cF(p')$. 
Without loss of generality, we will consider expressions for local operators at $0 \in \R^d$.

For topological-holomorphic theories the local operators take a very familiar form.
As an algebra they are generated by (holomorphic) derivatives of the fields evaluated at the specified point. 
More precisely, the local operators depend only on the $\infty$-jets of the fields at a point.
In this section we carefully formulate this result and give some examples.

\parsec[s:free]

%Suppose that $V$ is a translation invariant holomorphic vector bundle on $\C^n$ equipped with a $\Z \times \Z/2$ bigrading. 
%Let $\cV$ denote its sheaf of holomorphic sections.
%The {\em space of fields} of a holomorphic field theory on $\C^n$ is the $\Z \times \Z/2$ graded complex of vector bundles
%\[
%\Omega^{0,\bu}(\C^n, V) \cong \Omega^{0,\bu}(\C^n) \otimes V_0 
%\]
%where $V_0$ is the fiber of $V$ at $0 \in \C^n$.
%Our grading conventions are so that $\d \zbar_i$ has bidegree $(1,0)$.
%
%As introduced in \cite{BWhol,LiVertex,CG2}, a {\em holomorphic field theory} is a holomorphic vector bundle $V$ as above equipped additionally with:
%\begin{itemize}
%\item The structure of a local (super) $L_\infty$ algebra on $V[-1]$ with structure maps given by holomorphic polydifferential operators
%\[
%[\cdot]_k \colon \cV[-1]^{\times k} \to \cV[1-k] .
%\]
%\end{itemize}
%A {\em free} holomorphic theory has $[\cdot]_k = 0$ for $k > 1$.
%
%\parsec
A topological-holomorphic theory exists on spacetimes of the form $S \times X$ where $S$ is a smooth manifold and $X$ is a complex manifold (possibly equipped with some auxiliary geometric structures). 
The typical space of fields of a holomorphic-topological theory in the BV formalism is
\beqn\label{eqn:cE}
\cE = \Omega^\bu (S) \hotimes \Omega^{0,\bu}(X, V) 
\eeqn
where $V$ is a graded holomorphic vector bundle on $X$.
The underlying free theory is described by a differential on the space of fields of the form
\[
\d_{dR} + \dbar + Q^{hol} .
\]
Here $\d_{dR}$ is the de Rham differential acting on $S$, $\dbar$ is the Dolbeault operator acting on $X$, and $Q^{hol} \colon V \to V[1]$ is a holomorphic differential operator of cohomological degree~$+1$.
This means that the free, linear equations of motion for a field $\varphi$ take the form
\[
\d_{dR} \varphi + \dbar \varphi + Q^{hol} \varphi = 0 .
\]
Taking into account linear gauge symmetries corresponds to cohomology---solutions to the equations of motion modulo the image of $\d_{dR} + \dbar + Q^{hol}$.

Notice that $\cE$ is a sheaf of cochain complexes---it makes sense to restrict the fields to any open set $U \subset S \times X$. 
The factorization algebra of observables of the free theory whose fields are as above assigns to an open set $U \subset S \times X$ the cochain complex
\[
\Obs \colon U \mapsto \cO(\cE(U)) = \Sym \left(\cE(U)^\vee \right) 
\]
equipped with the induced differential.

Some remarks are in order:
\begin{itemize}
\item If $V$ is a topological vector space then $\cO(V) = \Sym(V^\vee)$ denotes the algebra of polynomials on~$V$.
Here~$V^\vee$ is the topological dual.
\item The topological dual of $\cE(U)$ is $\cE(U)^\vee \simeq \overline{\cE}^!_c(U)$ where the bar denotes distributional sections, the subscript $c$ denotes compact support, and $!$ denotes the Serre dual. 
Explicitly, if $U = U' \times U'' \subset S \times X$ then 
\[
\overline{\cE}^!_c(U' \times U'') \simeq \overline{\Omega}^\bu(U') \otimes \overline{\Omega}^{n,\bu}(U'',V^*)[n+m] 
\]
where $\dim_\C (X) = n$ and $\dim_\R (S) = m$. 
\end{itemize}

Let's restrict to the case that $S \times X = \R^m \times \C^n$ and suppose that the bundle $V \to \C^n$ is translation invariant with fiber $V_0$ over $0 \in \R^m \times \C^n$.
We also assume that the operator $Q^{hol}$ is translation invariant. 

Given a vector bundle $E \to M$, the bundle of $\infty$-jets $J^\infty E \to M$ is a $\infty$-dimensional pro vector bundle whose fiber over a point $p \in M$ is 
If $M = \R^d$ and $E$ is translation invariant, then the bundle of $\infty$-jets can be identified with $E_0 \times \C[[x_i]]$ where  
 
The jet expansion at $0 \in \R^m \times \C^n$ determines a map of cochain complexes
\[
\cE(\C^n \times \R^m) \to V_0 [[x_i, \d x_i,z_j, \zbar_j, \d \zbar_j]] 
\]
The differential on the right hand side is $\d_{dR} + \dbar + Q^{hol} = \d x_i \del_{x_i} + \d \zbar_j \del_{\zbar_j} + Q^{hol}$ where $Q^{hol}$ is some holomorphic differential operator in the $z_j$ variables. 
Since all structure maps are given by holomorphic polydifferential operators, the canonical map 
\[
V_0 [[x_i, \d x_i,z_j, \zbar_j, \d \zbar_j]] \xto{\simeq} V_0 [[z_j]] 
\]
which sends $x_i, \d x_i,\zbar_j \d \zbar_j \mapsto 0$ is a quasi-isomorphism. 
The only remaining differential on the right hand side is~$Q^{hol}$. 
In summary, we see that the jet expansion at $0 \in \R^m \times \C^n$ determines a map of cochain complexes $\cE(\R^m \times \C^n) \to V_0[[z_j]]$. 

\begin{lem}
\label{lem:taylor}
Suppose that $\cE$ is the sheaf of cochain complexes representing the free topological-holomorphic theory on $S \times X = \R^m \times \C^n$ and consider the factorization algebra of observables~$\Obs = \cO (\cE)$. 
Then, the Taylor expansion map
\beqn\label{eqn:taylor}
\cE(\C^n \times \R^m) \to V_0[[z_0,\ldots,z_n]]
\eeqn
induces a quasi-isomorphism of commutative dg algebras
\[
\Obs(0) \simeq \cO \left( V_0[[z_1,\ldots,z_n]] \right) .
\]
%Notice that when $\cL$ is abelian with differential $\d_{dR} + \dbar + Q^{hol}$, then there is a quasi-isomorphism
%\[
%\Obs(0) \simeq \cO \left( V_0[[z_1,\ldots,z_n]][1] \right) 
%\]
%where the right hand side is equipped with the differential $Q^{hol}$. 
\end{lem}
\begin{proof}
Suppose that $D_\R \times D_\C \subset \R^m \times \C^n$ is a product of a real $m$-disk times a complex $n$-disk containing the origin.
The algebra of observables supported on $D_\R \times D_\C$ is quasi-isomorphic to 
\[
\cO\left( \cO^{hol}(D_\C) \otimes V_0 \right) .
\]

Observe that there is a canonical map on fields 
\[
\cO^{hol}(D_\C) \otimes V_0 \to V_0[[z_1,\ldots,z_n]]
\]
given by taking the power series expansion at $0 \in D_\R \times D_\C$. 
If an observables on $D_\R \times D_\C$ depends on only the value of the field and its derivatives at $0 \in D_\R \times D_\C$ then it automatically factors through this map. 
In particular, this means that there is a quasi-isomorphism of local operators with functions on $V_0[[z_1,\ldots,z_n]]$,
\[
\Obs(0) \simeq \cO\left(V_0 [[z_1,\ldots,z_n]]\right).
\] 
\end{proof}

Let's unpack this result explicitly. 
Using the $n$-dimensional residue, we can identify the topological dual of $V_0[[z_1,\ldots,z_n]]$ with the vector space
\beqn
\frac{\d z_1}{z_1} \cdots \frac{\d z_n}{z_n} V_0^* [z_0^{-1}, \ldots,z_n^{-1}] .
\eeqn
This is the space of linear local operators. 
If $\chi \colon V_0 \to \C$ is a dual vector in~$V_0^*$
then we obtain a linear local operator at $0 \in \R^m \times \C^n$ on the space of fields by the assignment
\[
\varphi \mapsto \del_{z_1}^{k_1} \cdots \del_{z_n}^{k_n} \<\chi,\varphi\> (0) 
\]
where $k_i \geq 0$. 
Under the quasi-isomorphism of the lemma above, this corresponds to the linear local operator 
\[
\frac{\d z_1}{z_1^{k_1+1}} \cdots \frac{\d z_n}{z_n^{k_n+1}} \chi .
\]

\parsec[s:interaction]

It is not hard to turn on interactions in the description above. 
An interacting theory in the BV formalism is described by a local $L_\infty$ algebra structure on $\cL = \cE[-1]$, where $\cE$ is the sheaf of fields.
For a topological-holomorphic theory the higher $L_\infty$ structure maps $[\cdot]_k$ of the local $L_\infty$ algebra are required to be given by holomorphic polydifferential operators and $[\cdot]_1 = \d_{dR} + \dbar + Q^{hol}$.  
For more details we refer to the definitions in \cite{GRWthf}.

In this situation, the factorization algebra of classical observables supported on an open set $U \subset S \times X$ is given by the Chevalley--Eilenberg cochains on the $L_\infty$ algebra $\cL(U)$. 
This defines a factorization algebra 
\[
\Obs \colon U \mapsto \clie^\bu(\cL(U)) .
\]
We will now give a concise presentation for the {\em local} operators in a topological-holomorphic theory. 

On $S \times X = \R^m \times \C^n$ we can also ask that all $L_\infty$ structure maps be translation invariant. 
If this is the case, one obtains the induced structure of an $L_\infty$ algebra on the (shift of the) jets of the fields supported at $0 \in \R^m \times \C^n$
\[
V_0 [[z_1,\ldots,z_n]] [-1] .
\]
The $[\cdot]_1$ operation is precisely $Q^{hol}$ as above.
The Taylor expansion map \eqref{eqn:taylor} is a map of $L_\infty$ algebras. 
Combining this with Lemma \ref{lem:taylor}, one gets a quasi-isomorphism of cochain complexes between the local operators of an interacting topological-holomorphic theory in terms of Lie algebra cohomology
\[
\Obs(0) \simeq \clie^\bu\left(V_0[[z_1,\ldots,z_n]][-1]\right) .
\]

%We recall the reader of the standard dictionary between the space of fields of a BV theory and the local $L_\infty$ algebra---
%if the local Lie algebra is $\cL$, then the space of fields is $\cL[1]$. 
%The Chevalley--Eilenberg complex of $\cL$ is then functions on the fields $\cO(\cL[1])$ equipped with the non-linear BRST operator.

\parsec[s:envelope]

There is another way that observables are presented in a degenerate version of the BV formalism.
Suppose that~$\cE$ is the sheaf of sections of some graded vector bundle~$E$ on a manifold~$M$.
We have seen that the observables~$\cO(\cE) = \Sym(\cE^*)$ has the structure of a factorization algebra---we now consider the $!$-dual factorization algebra.
That is, we consider the factorization algebra 
\[
U \subset M \mapsto \Sym \left(\cE_c(U) \right) 
\]
where $U \to \cE_c(U)$ is the cosheaf of compactly supported sections of the bundle~$E$.

%Suppose that $\cL$ is a local Lie algebra on a manifold $M$. 
%Then, one can consider the factorization algebra
%\[
%\cF = \clie_\bu(\cL_c)
%\]
%which assigns to an open set $U$ the cochain complex $\clie_\bu(\cL_c(U))$.
%This is the $!$-dual of the factorization algebra $\clie^\bu(\cL)$.
%For topological-holomorphic local Lie algebras there is still an algorithm for computing $\cF(p)$ for a point~$p \in M$.
%
%We will assume that $\cL$ is a translation invariant topological-holomorphic local Lie algebra whose underlying sheaf of cochain complexes is
%\[
%\cL = \Omega^\bu (\R^m) \hotimes \Omega^{0,\bu}(\C^n, L)
%\]
%Here $L$ is a translation invariant holomorphic vector bundle on $\C^n$ and the differential in the complex is $\d_{dR} + \dbar + Q^{hol}$ as above.

\begin{lem}
\label{lem:envelope}
Suppose that $\cE$ is the sheaf of fields of a free holomorphic theory as in~\eqref{eqn:cE} and consider the factorization algebra~$\cF = \Sym(\cE_c)$. 
Then, the algebra of classical local operators at~$0 \in \C^n$ of the factorization algebra~$\cF$ is quasi-isomorphic to 
\begin{align*}
\cF(0) & \simeq {\rm Sym} \left(\Omega^{n,hol}(\Hat{D}^n,V_0^*)^\vee [-n]\right) \\ & \cong \cO\left(\Omega^{n,hol}(\Hat{D}^n,V_0^*) [n] \right) 
\end{align*}
where the differential on the right hand side is~$Q^{hol}$.
\end{lem}

%\begin{lem}
%\label{lem:envelope}
%Suppose that $\cL$ is a topological-holomorphic local Lie algebra on $S \times X = \R^m \times \C^n$ and let $\cF$ be the factorization algebra $\clie_\bu(\cL_c)$.
%Moreover, assume that $Q^{hol}$ is an elliptic holomorphic differential operator. 
%Then, there is a spectral sequence converging to $H^\bu(\cF(0))$ whose first page is the $Q^{hol}$ cohomology of
%\[
%\cO \left(\d^n z L_0^*[[z_1,\ldots,z_n]] [n+m-1] \right) .
%\]
%\end{lem}
\begin{proof}
First, notice that as graded topological vector spaces one has an isomorphism for any open set $U \subset M$ 
\beqn\label{eqn:dist}
\left(\overline{\cE}^!(U)\right)^\vee \simeq \cE_c(U) 
\eeqn
%
%We use the spectral sequence induced by the filtration by the homogenous degree of a local operator.
%The first page is the cohomology of 
%\[
%\lim_{U \ni 0} \Sym(\cL_c(U)[1]) 
%\]
%with respect to the linear differential $\d_{dR} + \dbar + Q^{hol}$ which acts on $\cL_c(U)$ and extends to the symmetric algebra by the rule that it is a derivation.
This implies there is an isomorphism
\beqn\label{eqn:dist2}
\Sym(\cE_c(U)) \simeq \cO\left(\overline{\cE}^!(U)\right) 
\eeqn
for any open set $U$.
By assumption, the linear differential $[\cdot]_1$ is elliptic, in particular the embedding of smooth sections into distributional sections
\beqn\label{eqn:dist3}
\cE^!(U) \hookrightarrow \overline{\cE}^! (U)
\eeqn
is a quasi-isomorphism for any open set~$U$. 

We can assume that $U \subset \C^n$ is a Stein open set containing~$0 \in \C^n$.
%of the form $U' \times U'' \subset \R^m \times \C^n$ with $U' \subset \R^m$ contractible and $U''\subset \C^n$ Stein.
Then we have a sequence of quasi-isomorphisms
\begin{align*}
\overline{\cE}^! (U) & \simeq \cE^!(U) \\ & \simeq \Omega^{n,\bu}(U, V^*)[n].
\end{align*}
The result now follows from Lemma~\ref{lem:taylor}.

%Thus, the first page of this spectral sequence is isomorphic to the cohomology of the local operators $\Obs(0)$ of the free theory whose underlying cochain complex of fields is 
%\[
%\cE = \Omega^\bu(\R^m) \hotimes \Omega^{0,\bu}(\C^n , K_{\C^n} \otimes L^*[n+m-1]).
%\]
%In the notation of Equation \eqref{eqn:cE}, the holomorphic vector bundle $V$ is 
%\[
%K_{\C^n} \otimes L^*[n+m-1] .
%\]
%The result now follows from Lemma~\ref{lem:taylor} where we have used $\d^n z$ for basis for the line $K_{\C^n}|_0$. 
\end{proof}

\subsection{Local characters for topological-holomorphic theories}\label{s:localchar}

Suppose that $\cF$ is the factorization algebra of observables of a topological-holomorphic theory on $\R^m \times \C^n$. 
We will restrict our attention to cases where $\cF$, as a graded vector space, is of the form $\Sym(\cE^*)$ or $\Sym(\cE_c)$ where $\cE$ is of the form \eqref{eqn:cE}.

The local character $\chi_\cF ({\bf q})$ is, by definition, the graded character of algebra of local operators $\cF(0)$ with respect to some group of symmetries $H$, see \cite{SWchar}.
The particular group of symmetries depends on the theory, and we will present some examples momentarily. 

By assumption, as a graded algebra, the algebra of local operators $\cF(0)$ of a topological-holomorphic theory is of the form
\beqn
\cF(0) = \Sym (\lie{s})
\eeqn
where $\lie{s}$ is a graded topological vector space which we interpret as the linear local operators.

We will also assume that the group of symmetries $H$ acting on $\cF(0)$ arises from an action of $H$ on the linear local operators $\lie{s}$. 
Denote by $f_{\cF}({\bf q})$ the character of $\lie{s}$ with respect to this group action---this is the so-called `single particle' character. 
The full character of $\cF(0)$ is then given as the plethystic exponential of this single particle character
\beqn
\chi_{\cF}({\bf q}) = {\rm PExp}\left[f_{\cF}({\bf q}) \right] .
\eeqn

\subsection{Examples}

We present some simple examples. 

\begin{eg}
Suppose that $V$ is the trivial bundle on $\C^n$ and consider the theory whose fields are
\[
\cE = \Omega^\bu(\R^m) \otimes \Omega^{0,\bu}(\C^n) 
\]
where the differential is just $\d_{dR} + \dbar$. 
Then, the space of local operators is the symmetric algebra on the topological vector space which is linear dual to 
\[
\cO^{hol}(\Hat{D}^n) = \C[[z_1,\ldots,z_n]] .
\]
Via the $n$-dimensional residue one can identify the algebra of local operators with 
\[
\Sym\left(\frac{\d z_1}{z_1} \cdots \frac{\d z_n}{z_n}  \C[z_1^{-1}, \ldots , z_n^{-1}]\right) ,
\]
where $\lie{s} \simeq \frac{\d z_1}{z_1} \cdots \frac{\d z_n}{z_n}  \C[z_1^{-1}, \ldots , z_n^{-1}]$ is (equivalent to) the space of linear local operators. 

Consider the standard torus action $\C^\times \times \cdots \times \C^\times$ on $\C^n$. 
We would like to observe that the character of local operators with respect to this symmetry would be given by the plethystic exponential of the single particle index (the character of the space of linear local operators) which is immediate to compute:
\[
\frac{1}{(1-q_1)\cdots (1-q_n)} .
\]
However, the plethystic exponential cannot be applied to such an expression since as a power series in $q_1,\ldots,q_n$ there is a nonzero constant term.
This is related to the fact that there is an infinite number of operators for which the fugacities satisfy $q_1=\ldots=q_n=1$, so counting local operators in this way is ill-defined. 
One can remedy this by introducing a single extra variable fugacity $y$ and modify the single particle index to 
\[
\frac{y}{(1-q_1)\cdots (1-q_n)} .
\]
The plethystic exponential of such an expression returns the local character
\[
\chi(q_1,\ldots,q_n,y) = \prod_{k_1,\ldots,k_n \geq 0} \frac{1}{1-y q_1^{k_1}\cdots q_n^{k_n}}
\]
which now makes sense as a power series in the variables $y,q_1,\ldots,q_n$.
\end{eg}

Its instructive to see how local operators differ between $!$-dual factorization algebras.
Let us first point out a simple example. 
\begin{eg}
Consider the sheaf of cochain complexes
\[
\cE = \Omega^{0,\bu}\left(\C, K_{\C}^{\otimes r}\right),
\]
where $r \in \Z$ and the differential is~$\dbar$. 
Then, we can consider both the factorization algebra $\Obs = \cO(\cE)$ and its $!$-dual $\Obs^! = \Sym(\cE_c)$. 

The $\infty$-jets at $0 \in \C$ of $\cE$ is quasi-isomorphic to $\Gamma(\Hat{D}^n, K^{\otimes r}) = \d z^{\otimes r} \C[[z]]$. 
Thus the algebra of local operators $\Obs(0)$ is quasi-isomorphic to 
\[
\Obs(0) \simeq \cO \left(\Gamma(\Hat{D}, K^{\otimes r})\right) .
\]
In particular, the character of local operators $\Obs(0)$ is the plethystic exponential of
\[
\frac{q^{r}}{1-q} 
\]
where $q$ represents the fugacity for the standard~$\C^\times$ action on~$\C$.
Notice that when $r = 0$ we run into a similar problem as in the previous example. 
It is therefore convenient to introduce an extra fugacity $y$ which enters the single particle character as
\[
\frac{y q^{r}}{1-q}  .
\]

%Then, we have the factorization algebra which assigns to $U \subset \C$ the complex
%\begin{align*}
%\cF(U) & = \clie_\bu(\cL_c(U)) \\ & = \Sym\left(\Omega^{0,\bu}_c\left(U, K_U^{\otimes r}\right) [1] \right)
%\end{align*}
%where the differential is $\dbar$. 

%Serre duality induces an isomorphism
%\begin{align*}
%\Omega^{0,\bu}_c(\C, K_\C^{\otimes r}) \cong \left( \Gamma^{hol} (\C , K_\C \otimes K_\C^{-r})\right)^* \\
%= \left( \Gamma^{hol} (\C , K_\C^{1-r})\right)^* .
%\end{align*}

On the other hand, by Lemma \ref{lem:envelope} we see that the local operators associated to the $!$-dual $\Obs^!(0)$ is identified with the vector space
\[
\cO\left(\Gamma(\Hat{D}, K^{1-r})[1]\right) .
\]
In particular, the character of local operators $\Obs^!(0)$ is the plethystic exponential of
\[
-\frac{q^{1-r}}{1-q} 
\]
where $q$ represents the fugacity for the standard $\C^\times$ action on $\C$.
This time, when $r=1$ there is a problem with defining the plethystic exponential. 
To get an expression that makes sense for all $r$ we can again introduce a variable $y$ which enters the single particle character as
\[
- \frac{y q^{1-r}}{1-q} .
\]
\end{eg}

\subsection{A relationship to the superconformal index}

We will turn to computing the local character for twists of the single membrane and fivebrane theories momentarily and comment on the relationship to known formulas in the physics literature. 
First, we point out a general relationship between the local character as we've defined it and the superconformal index. 

The superconformal index of a superconformal field theory is the Witten index of the theory in the radial quantization.
Explicitly, if the theory lives in a $d$-dimensional space time and $\cH$ denotes the Hilbert space on $S^{d-1}$, then the index is
\beqn
\cI(q,x_1,\ldots,x_n) = \Tr_{\cH} (-1)^F q^{\delta} x_1^{G_1} \cdots x_n^{G_n} 
\eeqn
where $\{G_i\}$ are a collection of charges that commute with a particular subset of elements of the superconformal algebra. 
We can choose that each $G_i$ commute with a minimal supercharge $Q$ and its adjoint $Q^{\dagger}$ in the superconformal algebra.
Necessarily the $G_i$ also commute with $\delta = [Q,Q^{\dagger}]$.
After tracing over $\cH$ one can identify the superconformal index with the partition function of the model on a space which is topologically equivalent to~$S^{d-1} \times S^1$.

The Witten index is protected under twisting---in our setup the index $\cI(q,x_1,\ldots,x_n)$ can be computed in the minimal twist of the superconformal theory we start with.
Let us specialize to the case of the six-dimensional superconformal index. 
In this case, we choose our fugacities to be $x_1 = t_1, x_2 = t_2$ and $x_3 = r$ as generators for the Cartan in the twisted superconformal algebra, see equation \eqref{eqn:cartan2}. 

The minimal twist of the Hilbert space $\cH^Q$ is exactly the space of holomorphic local operators at $0$ in $\C^3$, for details see~\cite{SWchar}. 
Thus, the index $\cI(q,t_1,t_2,r)$ agrees with the holomorphic character $\chi(q,t_1,t_2,r)$.

%\[
%(\C^3 - 0) / \sim  \; \simeq \; S^5 \times S^1 .
%\]
%The perspective of the holomorphic twist allows us to holomorphic theory agrees with the partition function on the product of spheres is basically goes by the process of `radial quantization'. 
%Consider the restriction of the theory to $\C^3 - 0 \subset \C^3$ and its dimensional reduction to quantum mechanics along
%\[
%|-| \colon \C^3 - 0 \to \RR_{>0} .
%\]
%The fiber of this map over a point is $S^5$. 
%By the nature of holomorphic QFT, we can extract from the OPE in the radial direction a canonical associative ($A_\infty$) algebra $\cA_{\lie{u}(1)} = \int_{\C^3 - 0} \Obs$ which is roughly the value of the theory on $S^5$. 
%There is a canonical boundary condition of the quantum mechanics theory at radius $r = 0$ given by the local operators $\Obs(0)$ at $0 \in \C^3$ which, in turn, has the structure of a $\cA$-module. 
%By standard arguments placing this quantum mechanics theory on circle $S^1$ results in the trace of the $\cA$-module $\Obs(0)$ 
%\[
%Z(S^{5} \times S^1) = {\rm Tr}_{\cA} (\Obs(0)) .
%\]
%From the trace on the right-hand side we can recover the character as defined above. 
%Indeed, the $E(3|6)$-module structure on local operators factors through a map 
%$E(3|6) \to \cA$ since we wrote down the explicit Hamiltonians above in \eqref{eq:ham1}, for instance. 

\subsection{Comparison to `states'}


%\subsection{Categorifying the index for free theories}
%
%In the case of both membranes and fivebranes we constructed a particular restriction of the local $L_\infty$ algebra $\cL_{sugra}$ to the respective worldvolume theories which we denoted by $\Bar{\pi}_* \cL_{sugra}$. 
%There are two important sub local $L_\infty$ algebras 
%\[
%\begin{tikzcd}
%& \Bar{\pi}_*\cL_{sugra} & \\
%\Bar{\pi}_*\cL_{sugra}^{(-1)} \ar[ur] & & \Bar{\pi}_*\cL_{sugra}^{(0)} \ar[ul] .
%\end{tikzcd}
%\]
%This diagram induces a diagram of factorization algebras
%\[
%\begin{tikzcd}
%& \left(\Obs_{sugra}|_Z\right)^! & \\
%\clie_\bu(\Bar{\pi}_*\cL_{sugra,c}^{(-1)}) \ar[ur] & & \clie_\bu(\Bar{\pi}_*\cL_{sugra,c}^{(0)}) \ar[ul].
%\end{tikzcd}
%\]

%\parsec[s:sugraops]
%
%By the usual methods of the BV formalism the action functional $S_{sugra}$ described above endows the parity shift of the fields $\cL_{sugra} = \Pi \cF_{sugra}$ with the structure of a holomorphic-topological local $\Z/2$ graded $L_\infty$ algebra. 
%
%On $\C^5 \times \R$ we can describe this super Lie algebra structure explicitly. 
%First, by the Dolbeault and de Rham Poincar\'e lemmas it is easy that the even part of the super Lie algebra $\cL(\C^5 \times \R)$ is equivalent to a one-dimensional central summand $\C$ plus the Lie algebra of divergence-free vector fields on $\C^5$:
%\[
%\Vect_0 (\C^5) = \{X \in \Vect(\C^5) \; | \; \div X = 0\} .
%\]
%The odd part of the super Lie algebra $\cL(\C^5 \times \R)$ is equivalent to the space of holomorphic one-forms on $\C^5$ modulo exact one-forms
%\[
%\Omega^{1,hol}(\C^5) / {\rm Im}(\del) 
%\]
%which is, of course, equivalent to the space of closed holomorphic two-forms $\Omega^{2,hol}_{cl}(\C^5)$. 
%
%\begin{thm}[\cite{RSW}[Theorem 2.1]]
%The Taylor expansion map determines a map of $\Z/2$ graded $L_\infty$ algebras
%\[
%j_\infty \colon \cL_{sugra}(\C^5 \times \R) \to L_{sugra} .
%\]
%Furthermore, $L_{sugra}$ is equivalent as a $\Z/2$ graded $L_\infty$ algebra to $\Hat{E(5|10)}$. 
%\end{thm} 
%
%As an immediate corollary of this result we obtain by Lemma \ref{lem:localops} the following.
%
%\begin{cor}
%\label{cor:sugraops}
%Let $\Obs_{sugra}$ be the factorization algebra on $\C^5 \times \R$ of classical observables of the minimal twist of eleven-dimensional supergravity.
%There is a quasi-isomorphism of commutative dg algebras
%\[
%\Obs_{sugra} (0) \simeq \clie^\bu \left( \Hat{E(5|10)} \right) .
%\]
%\end{cor}

%\end{document}


%\section{Conjectures for operators on fivebranes}

In conjecture \ref{conj:fact} we have formulated a conjectural description of the factorization algebra of classical observables $\Obs_{N}$ associated to the worldvolume theory on a stack of $N$ fivebranes in the holomorphic twist.
In this section we begin to provide some evidence for this description at the level of local operators, which is just a small piece of the factorization algebra.
As we reviewed just in the previous section, the space of local operators is what categorifies the specific superconformal index that we study in this paper.
%The structure of a three-dimensional holomorphic factorization algebra induces algebraic operations on the algebra of local operators

For each $N$ we have constructed a local Lie algebra $\cG_N$ on the worldvolume three-fold $Z$. 
Our conjecture is that the factorization algebra associated to a stack of $N$ holomorphically twisted fivebranes is $\Obs_N \simeq \clie_\bu(\cG_{N,c})$, where $\cG_{N,c}$ denotes the cosheaf of compactly supported sections. 
The main goal of this section is to extract the algebra of local operators from this factorization algebra and to then deduce explicit formulas for the local character, and hence the superconformal index.

For a stack of $N=1$ fivebranes, which corresponds to the abelian six-dimensional superconformal field theory, we find that our local character matches exactly with the expressions in the literature. 
This is not a surprise as we have shown that even at the level of factorization algebras $\clie_\bu(\cG_{1,c})$ is quasi-isomorphic to~$\Obs_1$, see Proposition~\ref{prop:factabelian}.

The main computation of this section is a closed formula for the local character of the factorization algebra $\clie_\bu(\cG_{N,c})$ for $N > 1$, see Theorem \ref{thm:finite}. 
Following conjecture \ref{conj:fact} and the general discussion of \S \ref{sec:sucaindex} we are led to hypothesize a closed formula for the superconformal index for the theory on a finite number of fivebranes (in flat space).
As far as the authors are aware of there is no closed formula for the refined superconformal index (with four independent fugacities) for the theory on a stack of $N > 1$ fivebranes.
For small values of $N$ we expand our closed formulas to low orders in the fugacity $q$ (which roughly counts instanton charge) to match exactly with expressions in the literature. 

\subsection{Operators on a single fivebrane}

We deduce the character of the holomorphic twist of the theory on a single fivebrane and will find an exact match with the index of the six-dimensional superconformal theory associated to the abelian Lie algebra $\lie{gl}(1)$.
By the proposition \ref{prop:factabelian} we can compute this character either from a first principles description of the theory, or holographically by focusing on the weight $(-1)$ part of the decomposition of $\Bar{\pi}_* \Obs_{sugra}$.

\begin{lem}
\label{lem:single}
The $\Z \times \Z/2$ graded algebra of local operators $\Obs_{1}(0)$ of the holomorphic twist of the worldvolume theory of a single fivebrane is quasi-isomorphic to the graded symmetric algebra on the linear dual of the topological vector space
\beqn\label{eqn:localfree}
V_0[[z_1,z_2,z_3]] \simeq \Omega^{2}_{cl} (\Hat{D}^3)[1] \oplus \Pi \Omega^0(\Hat{D}^3, K^{1/2}) \otimes \C^2 [1].
\eeqn
\end{lem}

\begin{proof}
The jet expansion at $0 \in \C^3$ determines a map from the sections of the abelian holomorphic-topological local Lie algebra on $\C^3$ to the cochain complex
\beqn
\begin{tikzcd}
\ul{-1} & \ul{0} \\
\Omega^{2}(\Hat{D}^3) \ar[r,"\del"] & \Omega^{3}(\Hat{D}^3) \\
\Pi \Omega^0(\Hat{D}^3, K^{1/2}_{\Hat{D}^3}\otimes \C^2) . 
\end{tikzcd} 
\eeqn
On the formal disk all closed two-forms are automatically exact, which implies the lemma.
\end{proof}

We present the character of $\Obs_1(0)$ as the plethystic exponential of the character $f_1(t_1,t_2,r,q)$ of the space of linear local operators
\beqn
\chi_{1} (t_1,t_2,r,q) = {\rm PExp} \big[f_1(t_1,t_2,r,q) \big] .
\eeqn
According to the weights listed above and using the description of local operators in Lemma \ref{lem:single} we have the following contributions to the single particle character~$f_{1}(t_1,t_2,r,q)$.

\begin{itemize}
\item Single particle operators on the odd copy of holomorphic two-forms $\Pi \Omega^{2,hol}$ contribute
\[
- q^2 \frac{\chi^{\lie{sl}(3)}_{[0,1]}(t_1,t_2)}{(1-t_1^{-1}q) (1-t_1 t_2^{-1} q) (1-t_2 q)} = - q^2 \frac{t_1  + t_1^{-1} t_2  + t_2^{-1} }{(1-t_1^{-1}q) (1-t_1 t_2^{-1} q) (1-t_2 q)}
\]
where $\chi^{\lie{sl}(3)}_{[1,0]}(t_1,t_2)$ is the $\lie{sl}(3)$ character of highest weight $[1,0]$.
\item Single particle operators on the even copy of holomorphic three-forms $\Omega^{3,hol}$ contribute
\[
q^3 \frac{1}{(1-t_1^{-1}q) (1-t_1 t_2^{-1} q) (1-t_2 q)} 
\]
\item Single particle operators on $K^{1/2} \otimes \C^2$ contribute
\[
q^{3/2} \frac{\chi_{1}^{\lie{sl}(2)} (r)}{(1-t_1^{-1}q) (1-t_1 t_2^{-1} q) (1-t_2 q)} = q^{3/2}\frac{(r + r^{-1})}{(1-t_1^{-1}q) (1-t_1 t_2^{-1} q) (1-t_2 q)}
\]
where $\chi_{1}^{\lie{sl}(2)} (r)$ is the $\lie{sl}(2)$ character of highest weight one.
\end{itemize}

Putting this all together we obtain the following.

\begin{prop}
\label{prop:6done}
The local character $\chi_{1}(t_1,t_2,r,q)$ of the holomorphic twist of the theory on a single fivebrane is given by the plethystic exponential of the single particle character
\beqn\label{eqn:6done}
f_{1} (t_1,t_2,r,q) = \frac{q^{3/2}(r + r^{-1}) - q^2(t_1 + t_1^{-1} t_2 + t_2^{-1} ) + q^3}{(1-t_1^{-1}q) (1-t_1 t_2^{-1} q) (1-t_2 q)} .
\eeqn
\end{prop}

In terms of the parameters $y_1,y_2,y_3,y,q$ this single particle character reads
\beqn
\label{eqn:6done1}
f_{1} (y_i,y,q) = \frac{qy + q^2y^{-1} - q^2(y^{-1}_1+y^{-1}_2+y^{-1}_3) + q^3}{(1-y_1q) (1-y_2 q) (1-y_3 q)} .
\eeqn
The expression matches exactly with the index of the abelian six-dimensional superconformal theory.
For example, compare with \cite[Eq. (3.1)]{Kim:2013nva} or \cite[Eq. (3.35)]{Bhattacharya:2008zy}.
From now on, we will give all formulas for the index in terms of the parameters $y_1,y_2,y_3,y,q$.

Generally speaking, after twisting there are enhancements of symmetries which are present in the original theory. 
In \cite{SW6d} we have shown that at the level of the holomorphic twist the twisted superconformal algebra $\lie{osp}(6|2)$ gets enhanced to the infinite-dimensional exceptional super Lie algebra $E(3|6)$ \cite{KacClass}. 
For the case of the single fivebrane theory, this implies that the local operators $\Obs_{1}(0)$ form a representation for $E(3|6)$. 

This fact also follows from our holographic analysis. 
In \S \ref{s:fact} we have expressed the restriction of the factorization algebra of observables of twisted eleven-dimensional supergravity to the three-fold $Z$ as the Chevalley--Eilenberg cochains of a local $L_\infty$ algebra $\cG$. 
Recall that we have a decomposition of local Lie algebras $\cG = \oplus_{j \geq -1} \cG^{(j)}$ on the three-fold $Z$. 
In Proposition \ref{lem:single} we have shown that $\clie_\bu (\cG_{c}^{(-1)})$ is equivalent to the factorization algebra $\Obs_{1}$.
On $\C^3$, the global sections of the local Lie algebra $\cG^{(0)}$ is closely related to $E(3|6)$---the $\infty$-jets of $\cG^{(0)}$ at $0 \in \C^3$ is quasi-isomorphic to $E(3|6)$.
Combining these facts we see that $\Obs_{1}(0)$ is a module for $E(3|6)$. 

In appendix \S \ref{s:kr} we will describe this module from a purely representation theoretic point of view and compare our expression for the character to the character of a certain irreducible module for the exceptional super Lie algebra $E(3|6)$ considered in~\cite{KR2}.

%\parsec
%
%There are various degenerations, or specializations, of this character which are interesting to consider.
%These specializations involve restricting the character above to a subalgebra of the full Cartan that we considered above.
%
%One degeneration of this character involves specializing $t_1=t_2=1$ which results in the $U(1) \times SU(2)$ character:
%\beqn
%f_{1}(r,q) = \frac{(r+r^{-1})q^{3/2} - 3 q^{2} + q^3}{(1-q)^3} .
%\eeqn
%They compute the absolute (non-super) character of the module $I(0,0;1;-1)$ where they additionally specialize $t_1=t_2=r=1$. 
%In a similar method to the one used in \cite{KR1}, one can compute the specialized (super) character of $I(0,0;1;-1)$ to find
%\[
%\chi_{u(1)} (q,t_1=t_2=r=1) = \frac{2 q^{3/2} - 3 q^2 + q^3}{(1-q)^3} .
%\]

\parsec
There are various degenerations, or specializations, of this character which are interesting to consider.
A particularly meaningful one is related to two different deformations of the theory by elements in the (twisted) superconformal algebra and is known as the Schur limit of the index.

Recall that after performing the holomorphic twist the residual superconformal algebra is~$\lie{osp}(6|2)$.
We have recalled in \S\ref{s:global1} how the bosonic part of this algebra is represented by fields of the eleven-dimensional theory. 
There are two types of odd elements of~$\lie{osp}(6|2)$ that also have a natural interpretation in the eleven-dimensional theory.
The odd part of~$\lie{osp}(6|2)$ can be identified with the twelve-dimensional space
\[
\C^3 \otimes \C^2 \oplus \wedge^2(\C^3) \otimes \C^2 
\]
where $\C^3, \C^2$ are the fundamental $\lie{sl}(3)$ and $\lie{sl}(2)$ representations, respectively. 
The $\lie{gl}(1)$ factor in the bosonic part of $\lie{osp}(6|2)$ acts with weight $1/2$ on both summands. 

\begin{itemize}
\item The summand $\C^3 \otimes \C^2$ embeds into the ghosts of twisted supergravity via the $\gamma$-type fields which satisfy
\[
\del \gamma = \d w_a \d z_i .
\]
where $i=1,2,3$ and $a = 1,2$.
Note that $\gamma$ appears to be ambiguous up to a closed holomorphic one-form, but since there is a linear gauge symmetry which sends $\beta \mapsto \del \beta$, it implies that $\gamma$ is unique up to a BRST exact term. 
Since in our model all closed one-forms are rendered trivial in cohomology
\item The summand $\wedge^2(\C^3) \otimes \C^2$ embeds as another $\gamma$-type field which satisfies 
\[
\del \gamma = w_a \d z_i \d z_j .
\]
\end{itemize}

Both deformations break the global Cartan subalgebra down to $\lie{gl}(1) \times \lie{gl}(1)$ according to the specializations
\beqn\label{eqn:special1}
y=1 , \quad y_3 = 1 .
\eeqn
Notice that due to the constraint $y_1y_2y_3=1$ this forces $y_1 = y_2^{-1}$.
As one can easily check, this specialization yields the following single particle index
\[
f_{1}(y_1, y_1^{-1},y_3=1, y=1, q) = \frac{q}{1-q} 
\]
which recovers the single particle index of a single chiral boson on the Riemann surface $\Sigma = \C_{z_1}$. 
Notice that the dependence on the parameter $y_1$ has completely dropped out even though we have not specialized it to any value.
%Notice that although the Cartan subalgebra generated by the vector field $z_1 \del_{z_1} - z_2 \del_{z_2}$ is unbroken by this deformation, the dependence on its fugacity $t_1$ completely drops out of the expression.

\subsection{A conjectural description of operators on a stack of two fivebranes}

In \S\ref{sec:factsummary} we saw that the decomposition of the local $L_\infty$ algebra $\cG = \cG_Z$ on $Z$ induces a filtration of the factorization algebra $\clie_\bu(\cG_c)$. 
\[
\clie_{\bu}(\cG_{1,c}) \subset \clie_{\bu}(\cG_{2,c}) \subset \cdots .
\]
We now turn to the factorization algebra $\clie_{\bu}(\cG_{2,c})$.

Recall that $\cG_{2}$ is the local $L_\infty$ algebra on $Z$ defined as $\cG_{2} = \cG / \cG^{\geq 1}$. 
Since $\cG$ is concentrated in weights $\geq -1$ we see that $\til \cG_{2}$ is of the form
\[
\cG_2 = \til \cG_2 \ltimes \cG_1 
\]
where $\cG_1 = \cG^{(-1)}$ is the weight $(-1)$ piece and $\til \cG_2 = \cG^{\geq 0} / \cG^{\geq 1} = \cG^{(0)}$.  
We focus mostly on the factorization algebra $\clie_\bu(\til \cG_{2,c})$.

We have already characterized the local dg Lie algebra $\til \cG_{2} = \cG^{(0)}$ as the weight zero part of $\cG$ on on any threefold $Z$ in \S\ref{s:weight0}. 
We have also shown that $\cG^{(0)}$ is equivalent to the local Lie algebra $\cE(3|6)$. 
The even part of $\cE(3|6)$ is
\[
\Omega^{0,\bu}(Z, \T_Z) \oplus \Omega^{0,\bu}(Z) \otimes \lie{sl}(2) 
\]
with its natural cohomological grading by Dolbeault form type. 
The odd part of $\cE(3|6)$ is
\[
\Omega^{1,\bu}(Z, K_Z^{-1/2}) \otimes \C^2 .
\]
The differential is $\dbar$ and the Lie bracket has been described in \S\ref{s:weight0}.

%On $Z = \C^3$ this local dg Lie algebra is related to the exceptional simple super Lie algebra $E(3|6)$ classified by Kac \cite{KacClass}. 
%Indeed, one can show (see the forthcoming work \cite{SW6d}) that the fiber of the $\infty$-jet bundle of $\cG_2$ at $0 \in \C^3$ is quasi-isomorphic to $E(3|6)$. 

\parsec

We continue by computing the character of local operators associated to the factorization algebra $\clie_\bu(\cG_{2,c})$ using Lemma~\ref{lem:envelope}.
For simplicity we will use the fugacities $y_i, y, q$.

\begin{itemize}
\item Single particle operators coming from the copy of holomorphic vector fields $\Vect^{hol}(\C^3)$ contribute
\[
q^4 \frac{\chi^{\lie{sl}(3)}_{[1,0]}(y_i)}{(1-y_1q) (1-y_2 q) (1-y_3 q)}  = q^4 \frac{y_1 + y_2 + y_3}{(1-y_1q) (1-y_2 q) (1-y_3 q)} 
\]
\item Single particle operators coming from $\lie{sl}(2)$-valued holomorphic functions $\lie{sl}(2) \otimes \cO^{hol}(\C^3)$ contribute
\[
q^3 \frac{\chi_2^{\lie{sl}(2)} (q^{-1/2}y)}{(1-y_1q) (1-y_2 q) (1-y_3 q)}  = \frac{q^2 y^2 + q^3 + q^4 y^{-2}}{(1-y_1q) (1-y_2 q) (1-y_3 q)} 
\]
%\[
%q^3\frac{r^2 + r^{-2} + 1}{(1-t_1^{-1}q) (1-t_1 t_2^{-1} q) (1-t_2 q)} 
%\]
\item Single particle operators coming from the odd piece of $E(3|6)$ which is $\Omega^{1,hol} \otimes K^{-1/2} \otimes \C^2$ contribute
\[
q^{7/2}\frac{\chi^{\lie{sl}(2)}_{1}(q^{-1/2} y) \, \chi_{[0,1]}^{\lie{sl}(3)} (y_i)}{(1-y_1q) (1-y_2 q) (1-y_3 q)} = q^{3}\frac{(y + q y^{-1})(y_1^{-1} + y_2^{-1} + y_3^{-1})}{(1-y_1q) (1-y_2 q) (1-y_3 q)} 
\]
\end{itemize}

Combining these expressions we obtain the following.

\begin{prop} \label{prop:6dtwo}
The character of local operators of the factorization algebra $\clie_\bu(\til \cG_{2,c})$ on $\C^3$ is given by the plethystic exponential of the following expression
\beqn\label{eqn:6dtwo}
\til f_{2} (y_i,y,q) = \frac{q^4(y_1+y_2+y_3) + q^2 (y^2 + q + q^2 y^{-2}) - q^{3} (y + q y^{-1})(y_1^{-1} + y_2^{-1} + y_3^{-1})}{(1-y_1q) (1-y_2 q) (1-y_3 q)}.
\eeqn
%\beqn\label{eqn:6dtwo}
%f_{2} (t_1,t_2,r,q) = \frac{q^4(t_1^{-1} + t_1 t_2^{-1}  + t_2) + q^3 (r^2 + r^{-2} + 1) - q^{7/2} (r + r^{-1})(t_1 + t_1^{-1} t_2 + t_2^{-1})}{(1-t_1^{-1}q) (1-t_1 t_2^{-1} q) (1-t_2 q)} .
%\eeqn
\end{prop}

Recall that our conjecture for the factorization algebra associated to the holomorphic twist of the six-dimensional worldvolume theory associated to a stack of two fivebranes is $\Obs_2 \simeq \clie_\bu(\cG_{2,c}) \cong \clie_\bu(\cG_{1,c}) \otimes \clie_\bu(\til \cG_{2,c})$. 
And after removing the center of mass degrees of freedom, our conjecture is $\til \Obs_2 \simeq \clie_\bu(\til \cG_{2,c})$.
We can now state a decategorified version of conjecture \ref{conj:fact} at the level of superconformal indices, or local characters.

\begin{conj}\label{conj:6dtwo}
The superconformal index of the six-dimensional superconformal theory associated to the Lie algebra $\lie{sl}(2)$ is given by
\[
\til \chi_{2} (y_i,y,q) = {\rm PExp} \left[\til f_2(y_i,y,q) \right] .
\]
where $\til f_2(y_i,y,q)$ is as in \eqref{eqn:6dtwo}.
\end{conj}

Similarly, the index associated to the $\lie{gl}(2)$ theory, which is the local character of $\clie_\bu(\cG_{2,c})$, is conjectured to be simply the product 
\[
\chi_{2} (y_i,y,q) = \chi_{2} (y_i,y,q) \cdot \chi_{1}(y_i,y,q)
\]
where the character $\chi_{1}$ for the $\lie{gl}(1)$ theory is given in proposition~\ref{prop:6done}
Equivalently, $\chi_2$ is the plethystic exponential of $f_2 = f_1 + \til f_2$. 

%\parsec[]
%
%The specialization of this index $t_1=t_2=r=1$ yields the single particle index
%\[
%\frac{3q^4 + 3 q^3 - 6 q^{7/2}}{(1-q)^3}. 
%\]

\parsec[]

The Schur limit $y=1, y_3=1$ of $\til f_2$ in \eqref{eqn:special1} yields 
\[
\til f_{2}(y_1, y_2, y_3=1, y=1, q) = \frac{q^2}{1-q} 
\]
which is the single particle index of Virasoro vacuum module on the Riemann surface $\Sigma = \C_{z_1}$. 

\subsection{A closed formula for the finite $N$ index}

Before exhibiting the general formula for the local character of the factorization algebra $\clie_\bu(\cG_{N,c})$ on $\C^3$ we set up some notation. 
As above, we let $\chi_k^{\lie{sl}(2)}$ and $\chi^{\lie{sl}(3)}_{[k,l]}$ denote the highest weight $\lie{sl}(2)$ and $\lie{sl}(3)$ characters. 
We also define the following expression which appears in the denominator in all of our characters
\beqn
d(y_i,y,q) = (1-y_1 q)(1-y_2q)(1-y_3q) .
\eeqn 
To simplify formulas, we will temporarily denote the single particle character for the $N=1$ theory $\Obs_1$ by 
\beqn
g_{-1} (y_i,y,q) = f_1(y_i,y,q)
\eeqn
where $f_1(y_i,y,q)$ is as in equation \eqref{eqn:6done1} and also denote by 
\beqn
g_0 (y_i,y,q) = \til f_2(y_i,y,q)
\eeqn
where $\til f_2(y_i,y,q)$ is as in equation \eqref{eqn:6dtwo}. 
Thus $g_2$ is the single particle local character of $\clie_\bu(\til \cG_{2,c}) = \clie_{\bu}(\cG_c^{(0)})$.
Finally, for $k \geq 1$ let
%\begin{align*}
%f_k (y_1,y_2,y_3,y,q) & \define q^{3k/2} \left(q \chi^{\lie{sl}(2)}_{k-2}(q^{-1/2} y)(y_1 + y_2 + y_3) + \chi^{\lie{sl}(2)}_k(q^{-1/2} y) \right. \\
%& \left.  - q \chi^{\lie{sl}(2)}_{k-3}(q^{-1/2}y) - \chi^{\lie{sl}(2)}_{k-1} (q^{-1/2} y) (y_1^{-1} + y_2^{-1} + y_3^{-1} ) \right) .
%\end{align*}
%\begin{align*}
%g_k (y_i,y,q) & \define q^{3} \left(q^{1 + 3 (k-2)/2} \chi^{\lie{sl}(2)}_{k-2}(q^{-1/2} y)(y_1 + y_2 + y_3) + q^{3(k-2)/2} \chi^{\lie{sl}(2)}_k(q^{-1/2} y) \right. \\
%& \frac{\left.  - q^{3(k-1)/2} \chi^{\lie{sl}(2)}_{k-3}(q^{-1/2}y) - q^{-1 + 3(k-1)/2} \chi^{\lie{sl}(2)}_{k-1} (q^{-1/2} y) (y_1^{-1} + y_2^{-1} + y_3^{-1} ) \right)}{d(y_i,y,q)} .
%\end{align*}
\beqn
\label{eqn:gk}
\begin{array}{lllll}
g_k (y_i,y,q) \define & q^{3} \left(q^{1 + 3 k/2} \chi^{\lie{sl}(2)}_{k}(q^{-1/2} y)\chi_{[1,0]}(y_i) + q^{3k/2} \chi^{\lie{sl}(2)}_{k+2}(q^{-1/2} y) \right. \\
&\displaystyle \frac{\left.  - q^{3(k+1)/2} \chi^{\lie{sl}(2)}_{k-1}(q^{-1/2}y) - q^{-1 + 3(k+1)/2} \chi^{\lie{sl}(2)}_{k+1} (q^{-1/2} y) \chi_{[0,1]}(y_i) \right)}{d(y_i,y,q)} .
\end{array}
\eeqn
%and hence the conjectural single particle index for the superconformal theory associated to the Lie algebra $\lie{sl}(2)$. 

\begin{thm}
\label{thm:finite}
Let $N \geq 3$. 
The local character of the factorization algebra $\clie_{\bu}(\cG_{N,c})$ is
\beqn
\chi_{N}(y_1,y_2,y_3,y,q) = \text{PExp}\left[\sum_{k=-1}^{N-2} g_k(y_1,y_2,y_3,y,q)\right].
\eeqn
Similarly, the local character of the factorization algebra $\clie_\bu(\til{\cG}_{N,c})$ is 
\beqn
\til{\chi}_{N}(y_1,y_2,y_3,y,q) = \text{PExp}\left[\sum_{k=0}^{N-2} g_k(y_1,y_2,y_3,y,q)\right].
\eeqn
\end{thm}
\begin{proof}
By Lemma~\ref{lem:envelope} the character of $\clie_\bu (\cG_{N,c})$ is given by 
\beqn
\chi_N = \text{PExp} \left[f_N\right]
\eeqn
where $f_N$ is the single particle local character.
Thus, it suffices to show that $f_N = \sum_{k = -1}^{N-2} g_k$.
Recall that from the description \eqref{eqn:gN} we have, as local Lie algebras:
\beqn
\cG_N = \cG / \cG^{(\geq N-2)} ,
\eeqn 
for $N \geq 1$. 
In particular, as a super vector bundle on the threefold $Z = \C^3$ we have
\[
\cG_N = \cG^{(-1)} \oplus \cG^{(0)} \oplus \cdots \oplus \cG^{(N-2)} .
\]
%\beqn
%\clie_{\bu}(\cG_{2,c}) \leftarrow \clie_{\bu}(\cG_{3,c}) \leftarrow \cdots \leftarrow \clie_\bu (\cG_{N,c}) .
%\eeqn
So, it suffices to observe that $g_k$ is the single particle index of the factorization algebra $\clie_\bu(\cG^{(k)}_c)$, which is a direct observation using the description of $\cG^{(k)}$ we have given in Proposition \ref{prop:Vj}.
\end{proof}

We thus arrive at the following conjecture for the index of the worldvolume theory on a stack of a finite number of fivebranes which we phrase in terms of the six-dimensional superconformal theory associated to the Lie algebra $\lie{sl}(N)$.

\begin{conj} 
The superconformal index of the six-dimensional superconformal theory associated to the Lie algebra $\lie{sl}(N)$ is $\til \chi_{N}(y_1,y_2,y_3,y,q)$. 
\end{conj}

We proceed to give some concrete evidence for this conjecture.
First, we show that when we take the limit as $N \to \infty$ that we recover the index computed from the gravitational side.

\parsec

It follows from the limit description in \eqref{eqn:lim} that the large $N$ limit of $\til{\chi}_N$ is precisely the multiparticle supergravity index we computed in proposition~\ref{prop:sugraindex1}. 
Alternatively, we have the following direct proof of this fact. 

\begin{prop}
One has
\beqn
\chi_{sugra}(y_i, y, q) = \lim_{N \to \infty} \chi_N(y_i,y,q)
\eeqn
\end{prop}

\begin{proof}
It suffices to show that at the level of single particle indices one has
\beqn
f_{sugra}(y_i, y, q) = \lim_{N \to \infty} \til{f}_N(y_i,y,q) ,
\eeqn
where $\til{f}_N = \sum_{k = -1}^{N-2} g_k$. 

We will use the following identity 
\beqn
\sum_{k=0}^\infty q^{3k/2} \chi_{k}^{\lie{sl}(2)}(q^{-1/2}y) = \frac{1}{(1-q y)(1-q^2 y^{-1})} .
\eeqn
We will denote this expression by $S(y,q)$.

Using this identity one can directly see that the result reduces to observing that
\begin{multline}
\left(q^4 (y_1+y_2+y_3) + 1 - q^6 - q^2 (y_1^{-1} + y_2^{-1} + y_3^{-1})\right)S(y,q) - 1 + q^3= \\
\left(q^4(y_1+y_2+y_3)-q^2(y_1^{-1} + y_2^{-1} + y_3^{-1})+(1-q^3)(yq + y^{-1} q^2) \right) S(y,q) .
\end{multline}


%\begin{itemize}
%\item $q^4 \sum_{k=0}^\infty q^{3k/2} \chi_{k}^{\lie{sl}(2)}(q^{-1/2} y)(y_1+y_2+y_3) = \frac{q^4(y_1+y_2+y_3)}{(1-q y)(1-q^2 y^{-1})}$. 
%\end{itemize}

\end{proof}

As an immediate corollary we have the following result.
\begin{cor}
For any $N \geq 1$ one has
\beqn
\chi_{sugra}(y_i,y,q) = \til{\chi}_N(y_i,y,q) \mod q^{N+1} .
\eeqn
\end{cor}
\begin{proof}
This follows from observing that at the level of single particle states $f_N$ is of order $q^{N}$.
\end{proof}

\parsec

We can also apply the Schur limit  $y,y_3\to 1$ to $\chi_N$.
\begin{prop}
Upon specializing $y=1,y_3=1$ (so that $y_1 y_2 = 1$) one has the following single particle index
\beqn
f_N (y_1,y_2, y_3=1,y=1,q) = \frac{q^2 + q^3 + \cdots + q^{N}}{1-q} 
\eeqn
The plethystic exponential of the right hand side is the vacuum character of the $W_{N}$ vertex algebra.
\end{prop}
\begin{proof}
By induction it suffices to show that the specialization of the single particle local character $g_k$ of the factorization algebra $\clie_\bu(\cG^{(k)})$ is $q^{k+2} / (1-q)$. 
We have already seen this in the case $k=-1,0$, so it suffices to show this when $k \geq 1$.

First observe that the denominator becomes
\beqn
d(y_1,y_2,y_3=1,y=1,q) = (1-y_1 q)(1-y_2q) (1-q) .
\eeqn

Next, we observe that the numerator of $g_k (y_1,y_2,y_3=1,y=1,q)$ can be factored as
\begin{align*}
q^{3 + 3k/2} \left(q^{-(k+2)/2} + q^{-(k-2)/2} - q^{-k/2} (y_1+y_2) \right) 
& = q^{k+2} (1 + q^2 - q (y_1 + y_2)) \\
& = q^{k+2} (1 - y_1 q) (1-y_2 q) 
\end{align*}
where in the last line we have used $y_1 y_2 = 1$.
The result follows.
\end{proof}

\subsection{Comparisons to expansions of superconformal indices}

In the final section we would like to exhibit a series of direct consistency checks with our conjectural exact formula for the index of the non-abelian six-dimensional superconformal theory with a few expansions that have appeared in recent literature. 

\parsec
Let us first focus on the superconformal theory associated to the Lie algebra $\lie{sl}(2)$ (so type $A_1$).
Our conjecture for the superconformal index in this case is the plethystic exponential of $\til f_2 (y_i,y,q)$ from equation \eqref{eqn:6dtwo}.
We expand the formal single particle index $\til f_2 (y_i, y, q)$ as a series in the variable~$q$, yielding
\begin{align*}
\til f_2 (y_i,y,q) & = y^2 q^2 + \left(1 - \chi_{[0,1]}(y_i) y + \chi_{[1,0]} y^2 \right) q^3 \\
& + \left(y^{-2} - \chi_{[0,1]}(y_i) y^{-1} + 2 \chi_{[1,0]}(y_i) - \chi_{[0,1]} (y_i) y + \chi_{[2,0]}(y_i) y^2 \right) q^4 + O(q^5) .
\end{align*}
From this expression, we obtain the $q$-expansion of the index $\til \chi_2(y_i,y,q) = \text{PExp}[\til f_2]$ as 
\begin{align*}
\til \chi_2(y_i,y,q) & = 1 + y^2 q^2 + \left(1-\chi_{[0,1]}(y_i) y + \chi_{[1,0]}(y_i)y^2\right)q^3 \\ 
& + \left(y^{-2} - \chi_{[0,1]}(y_i) y^{-1} + 2 \chi_{[1,0]}(y_i) - \chi_{[0,1]} (y_i) y + \chi_{[2,0]}(y_i) y^2 + y^4\right)q^4 + O(q^5)
\end{align*}

Similarly, for the $\lie{gl}(2)$ theory $\chi_2 = \text{PExp}[f_1 + \til f_2] = \chi_1 \cdot \til \chi_2$ we find the expansion
\begin{align*}
\chi_2 (y_i,y,q) & = y q + \left(y^{-1} - \chi_{[0,1]}(y_i) + \chi_{[1,0]}(y_i) y + 2y^2 \right) q^2 \\ 
& + \left( \chi_{[1,0]}(y_i) y^{-1} - (\chi_{[1,1]}(y_i)-2) + (\chi_{[2,0]}(y_i) - 2 \chi_{[0,1]}(y_i)) y + 2 \chi_{[1,0]}(y_i) y^2 + 2y^3\right) q^3 \\ & + O(q^4) .
\end{align*}

We observe that these $q$-expansions agree precisely with the expansions in \cite{Kim:2013nva} for the $\lie{gl}(2)$ theory  
(see equations (3.51) and (3.65) of \textit{loc. cit.}).

\parsec

We proceed to compare expansions of our exact expression for the $\lie{gl}(3)$ theory to those in \cite{Kim:2013nva}. 
Recall that the conjectural $\lie{gl}(3)$ index is given by the local character of the holomorphic factorization algebra $\Obs_3$:
\beqn
\chi_3 (y_i,y,q) = \text{PExp}[f_3(y_i,y,q)] = \chi_2(y_i,y,q) \cdot \text{PExp}[g_3(y_i,y,q)]  .
\eeqn
Here, $f_3(y_i,y,q)$ is the single particle local character for the holomorphic factorization algebra $\Obs_3$ and $g_3(y_i,y,q)$ is given in equation \eqref{eqn:gk}. 

Since $g_3(y_i,y,q) = y^3 q^3 + O(q^4)$ we see that $\chi_3$ and $\chi_2$ agree up to order $q^2$ and the difference at order $q^3$ is simply
\beqn
\chi_3(y_i,y,q) - \chi_2(y_i,y,q) = y^3q^3 + O(q^4) .
\eeqn
This is again in exact agreement with the index for the $\lie{gl}(3)$ theory computed \cite{Kim:2013nva} up to order $q^3$ (see equation (3.79) of \textit{loc. cit.}). 

\parsec

Next, we compare to expansions for the $\lie{sl}(N)$ theory computed in \cite{Imamura}.
It will be convenient to change the variables $(y_i, y, q) \to (y_i,x,q)$ where 
\beqn
x = qy .
\eeqn 
We will again expand in powers of $q$.\footnote{To match precisely with the equations in \cite{Imamura} we note that it is necessary to relable the variables $y_i \leftrightarrow u_i$, $x \leftrightarrow \check{x}$, and $q \leftrightarrow y$ where the variable $y$ is distinct from the one we use in this paper!}

Starting with the $\lie{sl}(2)$ theory we find that up to order $q^4$ the single particle index is
\begin{align*}
\til f_2 (y_i,x,q) & = x^2 + \chi_{[1,0]}(y_i) x^2 q \\
& + \left(-\chi_{[0,1]}(y_i) x + \chi_{[2,0]}(y_i) x^2 \right) q^2 +  \left(1-x-\chi_{[1,1]}(y_i) x + \chi_{[3,0]}(y_i) x^2 \right) q^3 \\
& + \left(2 \chi_{[1,0]}(y_i) - \chi_{[2,1]}(y_i) x + \chi_{[4,0]}(y_i)x^2 \right)q^4 + O(q^5) .
\end{align*}

It follows that the plethystic exponential $\til \chi_2(y_i,y,q)$ of this expression has $q$-expansion
\begin{align*}
\til \chi_2(y_i,y,q) & = \frac{1}{1-x^2} +  \frac{x^2}{1-x^2} \chi_{[1,0]}(y_i) q \\
& + \left(- x \chi_{[0,1]}(y_i) +  x^2 (1+x^2)\chi_{[2,0]}(y_i)\right) \frac{1}{1-x^2} q^2 \\ 
& + \left( 1 - x - x^3 + x^6 + (-x - x^3 + x^4) \chi_{[1,1]}(y_i)  + (x^2 + x^4 + x^6)\chi_{[3,0]}(y_i)  \right) \frac{1}{1-x^2} q^3 \\
& + O(q^4) 
\end{align*}
This agrees with the expansion in \cite{Imamura} (see equation (68)) except for the $\lie{sl}(3)$-scalar term at order $q^3$. 
We find $(1-x-x^3+x^6) / (1-x^2) = (1-x^3-x^4-x^5) / (1+x)$ whereas Imamura's result is $1 / (1+x)$. 
It would be interesting to explain the physical or representation theoretic source of this discrepancy.

Similarly, we can obtain the $q$-expansions for the local character $\chi_3(y_i,x,q)$ of the factorization algebra $\Obs_3$ and compare it to the $q$-expansion for the superconformal index of the $\lie{sl}(3)$ theory in \cite{Imamura}. 
Up to order $q^3$ we have
\begin{align*}
\chi_3(y_i,x,q) & = \frac{1}{(1-x^2)(1-x^3)} + \frac{x^2}{(1-x)(1-x^3)} \chi_{[1,0]}(y_i) q \\
& + \left((-x -x^2 + x^5)\chi_{[0,1]}(y_i) + (x^2 + x^3 + x^4 + x^5+ x^6) \chi_{[2,0]}(y_i) \right) \frac{1}{(1-x^2)(1-x^3)} q^2 \\
& + 
\end{align*}


\section{Holographic Speculations}\label{sec:holspec}
We began this thesis with some remarks on how dualities between physical theories can often be used to uncover novel equivalences between the mathematical objects that describe them. In this final section of the thesis, we offer some speculations to this effect.
We caution the reader that a large portion of this section involves recalling constructions from physics without any attention to rigor for motivational purposes.

In sections \ref{sec:BLG}, \ref{subsec:g-1}, we commented on how minimal twists of the 3d $\mc N=8$ BLG theory and 6d $\mc N=(2,0)$ tensor multiplets are are visible as pieces of the graviton decompositions of twisted $AdS_4\times S^7$ and $AdS_7\times S^4$ respectively. Famous instances of the AdS/CFT correspondence posit equivalences between the higher rank 3d $\mc N=8$ theories studied by ABJM and the higher rank 6d $\mc N=(2,0)$ theories of type $A_N$ with M-theory on $AdS_4\times S^7$ and $AdS_7\times S^4$ respectively. It is natural to wonder whether the twisted holograpahy proposal mentioned in the introduction can be applied to our descriptions of the twisted $AdS_4\times S^7$ and $AdS_7\times S^4$ backgrounds to study the minimal twists of the higher rank 3d $\mc N=8$ and 6d $\mc N=(2,0)$ superconformal field theories respectively.

Our goal in this section is to posit some expectations regarding the minimal twist of the 6d $\mc N=(2,0)$ theory of type $A_N$. This theory is notorious for being both ubiquitous and nebulous. On the one hand, almost every superconformal field theory that has had interesting applications to geometry, topology, or representation theory occurs as one of its dimensional reductions, so it has long been expected to contain very rich mathematics. On the other hand, it does not admit a Lagrangian description. Its only free parameter is the rank of an ADE Lie algebra, and outside of the abelian case, a field realization is not even known.

We begin by recalling some features of the AdS/CFT correspondence. We will begin with a more physical language, and work towards some concrete mathematical expectations. 

\subsection{The AdS/CFT correspondence}

Traditional formulations of the AdS/CFT correspondence relate two theories, schematically denoted $T_{CFT}$ and $T_{grav}$ on manifolds $M_{1}, M_{2}$ respectively, together with a conformal diffeomorphism $\del M_{2}\cong M_{1}$. The theories have the feature that boundary values of fields of $T_{grav}$ denoted $\phi |_{\del}$, may be identified with sources for $T_{CFT}$ denoted $J$. The two theories are considered to be holographically dual when their partition functions are equivalent $Z_{CFT}[J] = Z_{grav}[\phi |_{\del}]$. 

In examples of stringy origin, $T_{CFT}$ describes the low energy dynamics of a stack of $N$ branes in supergravity, in the large $N$ limit, and $T_{grav}$ describes gravitational dynamics in the background the branes source. 

\parsec[]
Let's identify some salient features of the primordial example of such a duality so as to inform our desiderata in the sequel.

\begin{conj}[Maldacena \cite{Maldacena:1997re}, \cite{WittenAdS}]
The following are equivalent:
\begin{itemize}
\item $\mc N=4$ super Yang-Mills theory with gauge group $SU(N)$. In addition to the rank of the gauge group, the theory has a parameter the Yang-Mills coupling constant $g_{YM}$.
\item type IIB superstring theory on $AdS_5\times S^5$ with $N$ units of five-form flux on $S^5$. The theory has two free parameters, the string coupling $g_s$ and a parameter $L/\ell_s$ which describes the scale of AdS relative to the length of the string. \end{itemize}

Under this equivalence, the parameters of the two sides are identified as follows $g_{YM}^2 = 2\pi g_s$ and $2g_{YM}^2N = (L/\ell_s)^4$. 
\end{conj}

It is convenient to introduce a parameter $\lambda = g^2_{YM} N$, the so-called \textit{'t Hooft coupling}; in the perturbative regime where the number of colors is also large (a limit that we will introduce momentarily), the $\beta$-function keeps $\lambda$ of the same order.

It is very difficult to perform explicit calculations of most observables associated to either theory at generic values of the parameters on either side. However, there are certain limits which afford more tractability. 

\begin{itemize}
\item The first limit we can take involves sending the string coupling $g_s$ to zero and keeping the parameter $L/\ell_s$ fixed. In this limit, contributions from higher genus worldsheets in string perturbation theory are suppressed. Under the above identification of parameters, we see that this limit should involve taking $g_{YM}\to 0$ while keeping the 't Hooft coupling finite; that is, we must take the large $N$ limit of the gauge theory. This limit is traditionally referred to as the \textit{'t Hooft limit}. Corrections in $\frac{1}{N}$ then correspond to turning on quantum effects in string theory.

\item After taking the 't Hooft limit, we may further consider the limit where $L/\ell_s$ is large. In this limit, strings are small and particle like compared to the scale of AdS and the theory looks like classical type IIB supergravity on $AdS_5\times S^5$. On the gauge theory side, this corresponds to the limit where the 't Hooft coupling is large. As such, we see that even this simplified form of the AdS/CFT correspondence is extremely powerful as it relates strongly coupled gauge theory to classical perturbative supergravity!
\end{itemize}

\subsection{BPS observables in AdS/CFT}\label{bpsadscft}

Many checks of the AdS/CFT correspondence involve computing quantities on either side that are independent of the coupling and comparing them. Such quantities are typically BPS, and as such can be studied at the level of twists. We introduce two such quantities which we will further expand on in our relevant example below. 

\parsec[]
Suppose that $T_{CFT}$ is superconformal, such as in the above example. In such examples, one expects that the superconformal algebra in fact acts on $M_2$ as isometries, at least asymptotically.

Superconformal field theories admit a plethora of protected quantities that can be computed exactly at weak coupling. One such quantity is the superconformal index, which in a Hamiltonian formulation of the theory can be thought of as a Witten-index in radial quantization. Schematically, such a quantity takes the form \[\operatorname{Tr} _{\mc H} \left ( (-1)^F \exp (-\beta \{ Q, \overline Q\} ) x_1^{J_1}\cdots x_n^{J_n}y_1^{H_1}\cdots y_n^{H_n} \right )\] where $(-1)^F$ is the fermion number operator, $\beta$ is an inverse temperature, $Q$ is a supercharge and the $x_i$ are fugacities keeping track of charges under angular momenta, and $y_i$ are fugacities keeping track of charges under R-symmetries. The superconformal index gives a generating function for the difference between bosonic and fermionic states annihilated by a particular supercharge. Under an operator-state correspondence, the superconformal index can also be thought of as a signed count of local operators preserved under a single supercharge. 

In terms of partition functions, the superconformal index is gotten by a partition function on a twisted product $M_{1} = S^{1}\times_\omega S^{d-1}$ where the twisting $\omega$ is determined by a background connection for the global symmetries of the problem. The expectation that the AdS/CFT correspondence can be expressed as an equality of partition functions therefore suggests a recapitulation of the superconformal index in gravitational terms. An exciting body of work aims to make this gravitiational incarnation precise, see for example \cite{murthy2020growth} and references therein.  

Note that by definition, the superconformal index provides a lower bound on the number of fractionally BPS states of $T_{CFT}$. It is often the case, however, that $T_{grav}$ includes in its spectrum, black holes, which are expected to have a thermodynamic entropy proportional to the event-horizon-area at leading order, as given by the Beckenstein-Hawking formula. As such, the growth of states in $T_{CFT}$, and hopefully the superconformal index, should reflect this. 

\parsec[]
Another such quantity is the algebra of BPS local operators in $T_{CFT}$. This vector space underlying this algebra is precisely a costalk of the factorization algebra of observables of a twist of $T_{CFT}$. In light of the aforementioned operator-state correspondence, this can be thought of as categorifying the superconformal index. Under the AdS/CFT dictionary, local operators of $T_{CFT}$ are supposed to match with certain kinds of states in $T_{grav}$. 

Moreover, both kinds of objects transform in representations of a superconformal algebra and the map between them preserves the actions. Local operators in the CFT are equipped with an interesting algebraic structure given by operator-product-expansion, and the AdS/CFT correspondence intertwines this algebraic structure with scattering of supergravity states. Indeed, the equality of partition functions along with the matching of sources for CFT local operators with boundary values of gravitational fields gives a prescription for computing correlation functions between CFT local operators by varying the gravitational action evaluated on field configurations subject to certain boundary values with respect to the boundary value. This recipe can be recast as a tree-level computation in the gravitational theory, involving computation of so-called Witten diagrams \cite{Witten:AdS}.

\subsection{Twisted holography}
Introduced by Costello and Li in \cite{CLsugra}, the twisted holography proposal posits an avatar of the AdS/CFT correspondence that holds at the level of factorization algebras associated to supersymmetric twists of $T_{CFT}$ and $T_{grav}$. There is an exciting body of work being developed around this program including tests of this proposal from both the gravitational and gauge theory sides.

\parsec{}
Concretely, the twisted holography proposal suggests that the type of duality between the factorization algebras associated to a gravitational theory and to the worldvolume theory of a number of branes is a general version of \textit{Koszul duality}.

Ordinary Koszul duality for associative algebras (so quantum mechanical systems) associates to an (augmented) algebra $A$ a dual algebra $A^!$ whose appropriate derived category of modules is the same as that of $A$.
Following the work of \cite{CLsugra, CP1} (see also the review in \cite{PWkoszul}) there is a simple physical interpretation of Koszul duality.
If $A$ is the algebra of operators of some bulk quantum field theory (perturbatively we can even consider a theory of gravity) then $A^!$ is the algebra of operators on the universal topological line defect.
Universal here means that algebra of operators on any other line defect which couples to the bulk system admits a unique map of algebras from~$A^!$.

The general theory of Koszul duality for factorization algebras has not been developed, and we do not do so here, but see \cite{LurieHA} for the case of $\mb E_n$-algebras and  \cite{gui2022quadratic}, \cite{tamarkin2003deformations} for the case of particular kinds of vertex algebras. This sort of duality would allow one to make sense of universality statements as above for higher dimensional, possibly non-topological, defects in an arbitrary bulk quantum field theory. Roughly, one expects the Koszul dual of a factorization algebra to be the factorization algebra corepresenting the functor of looking at solutions to a Maurer-Cartan equation in a tensor product. 


\parsec{}
Let us now make a more concrete, yet slightly informal, statement of twisted holography which fits into the approach of this thesis. Let $X$ be a smooth manifold, and let $\Obs_{grav}$ denote a factorization algebra on $X$ that we view as the observables of a bulk gravitational theory. Suppose we have, in addition, a stack of $N$ branes, wrapping a closed submanifold $Y\hookrightarrow X$ whose worldvolume theory has a factorization algebra of observables $\Obs_{CFT}^N$. 

Note that $\Obs_{grav}$ is a factorization algebra on $X$, while $\Obs_{CFT}^N$ is a factorization algebra on the closed submanifold $Y$ so we cannot yet compare them.
We can, however, attempt to restrict $\Obs_{grav}$ to a factorization algebra just on $Y$, which we denote by $\Obs_{grav}|_Y$.

\begin{expect}[Twisted holographic principle following \cite{CLsugra}]\label{twistedholog}
There is a map of factorization algebras
\[
  (\Obs_{grav}|_{Y})^{!}\to \Obs_{CFT}^N
\]
that becomes an equivalence in the large $N$ limit.
\end{expect}

As we recalled in the previous subsection, traditional formulations of the AdS/CFT correspondence relate local operators of the gauge theory to states of the gravitational theory on AdS. Therefore, a natural desideratum in relating the above to more traditional statements is a precise relation between the source of the above map and gravitational states in $AdS$. Moreover, there is an operational definition of the operator-product-expansion on the costalk of a Koszul dual factorization algebra which realizes the expectation that Koszul duality corepresents the functor taking Maurer-Cartan elements in the tensor product. Another desideratum is to relate the output of this procedure with the scattering product on gravitational states computed by Witten diagrams. 

\begin{rmk}
In this context, the definition of Koszul duality involves another ingredient, namely the backreaction of branes wrapping $Y$. This is meant to capture the fact that $(\Obs_{grav}|_Y)$ may not be canonically augmented, but we may try to deform it in a way that kills off the obstruction to being augmented. More precisely, one expects that the correct version of Koszul duality for application in holographic contexts is a version of \textit{curved} Koszul duality for factorization algebras. 
\end{rmk}

\begin{rmk}
For finite $N$, this map will in general be neither injective nor surjective. The kernel and cokernel of this map for finite $N$ correspond to interesting nonperturbative effects in the gravitational theory. For instance, in gauge theories:

\begin{itemize}
\item This map has a kernel given by trace relations. Syzygies between trace relations are conjecturally related to the worldvolume theories of certain other branes in the gravitational theory, so-called \textit{giant gravitons} \cite{Gaiotto:2021xce}, \cite{choi2023quantum}, \cite{Imamura} \footnote{We thank Ji-Hoon Lee for conversations related to this topic}

\item This map also has a cokernel. By fiat, these are classes that exist in the finite $N$ cohomology of the observables of a gauge theory that are not in the image of the natural map from the large $N$ theory. Recent developments in cohomological approaches to counting quantum microstates of $\frac{1}{16}$-BPS black holes in $AdS_5\times S^5$ \cite{choi2023quantum} \cite{Chang_2023} \cite{Chang_2013} can be cast as trying to characterize the cokernel of a specific example of this map. 
\end{itemize}
\end{rmk}

\parsec[]
The above expectation can be tested in instances where both sides of the duality admit explicit descriptions. This has been carried out in many examples including:
\begin{itemize}
  \item A stack of $D3$ branes in twisted $\Omega$-deformed type IIB supergravity on flat space. The theory on the stack of $D3$ branes is dual to the closed string B-model on the deformed conifold \cite{costello2021twisted}. This can be understood as a twisted $\Omega$-deformed version of the physical AdS/CFT duality between 4d $\mc{N}=4$ super Yang-Mills and type IIB string theory on $AdS_{5}\times S^{5}$. Here the duality can be formulated in terms of vertex algebras. 
  
  \item M2 branes and M5 branes in twisted $\Omega$-deformed $M$-theory on Taub-NUT space \cite{CostelloM5,CostelloM2}. In the particular $\Omega$-background, M2 branes are localized to a topological quantum mechanical system where the duality can be phrased in terms of associative algebras and ordinary Koszul duality. The koszul dual algebra bears close relations to the spherical Cherednik algebra. The $\Omega$-background localizes M5 to a complex plane and the observables of the localized theory are an affine $W_{N}$ vertex algebra. Many celebrated features of the representation theory of these algebras and their relations with geometry have found natural explanations from the perspective of this twist of M-theory \cite{gaiotto2020miura}, \cite{Oh:2021wes}.
\end{itemize}

The example we consider is closely related to the second of these. Indeed, there is an odd nilpotent element in $\mf{osp}(6|2)$, which we refer to as $S$ in the sequel. Using the inner action of $\mf{osp}(6|2)$ on our eleven-dimensional model on twisted $AdS_7\times S^4$ as identified in proposition \ref{prop:brads7}, $S$ affords a deformation of our model. This is the deformation considered in \cite{BeemEtAl}, and it induces a specialization of characters called the Schur limit.
\parsec[]

\subsection{M5 branes, holomorphy, and holography}
The results in the second half of this thesis can be viewed as baby steps in investigating twisted holography for the minimal twist of the 6d $\mc N=(2,0)$ theory. Let us begin by spelling out the objects in expectation \ref{twistedholog} adapted to our setting. 

\begin{itemize}
\item The 11d spacetime manifold $X$ is $\R\times \C^5$ and $Y$ is a copy of $\C^3$. 
\item The factorization algebra $\Obs_{grav}|_{\C^3}$ has the feature that its semiclassical free limit is the factorization algebra denoted $\clie ^\bullet \left ( \Pi \Omega^{0,\bullet}_{\C^3} (\mc L^N_{AdS_7\times S^4})\right )$ in definition \ref{defn:ads7states}.
\item The factorization algebra $\Obs^N_{CFT}$ describes local observables in the minimal twist of the theory on a stack of $N$ M5 branes wrapping $\C^3$. 
\end{itemize}

Our goal is to try and use this map, and expectations about its kernel and cokernel, to give a concrete description of the target. There have been various approaches to try and characterize the spectrum of $\frac{1}{16}$-BPS states in the 6d $\mc N=(2,0)$ theories of type $A_{N-1}$, which furnish consistency checks to test our proposal against. Some of these involve applications of instanton counting techniques in 5d $\mc N=2$ gauge theory \cite{Kim2013nva} and some of them involve holographic techniques \cite{Imamura}.

As we remarked in subsection \ref{bpsadscft}, the first ingredient is a map of representations of the superconformal algebra between local operators of the CFT and supergravity states. In order to codify such a matching in terms of the kinds of data in the statement of expectation \ref{twistedholog}, we require a matching between supergravity states and the costalk at the origin of the factorization algebra $(\Obs_{grav}|_{\C^3})^!$. This is precisely the content of proposition \ref{prop:altstates}.

\parsec[]

We have the following conjectural large $N$ statements

\begin{conj}[R-Saberi-Williams]\label{conj:classical}
There is an equivalence of holomorphic $\mb{P}_0$-factorization algebras \[ \left( \clie^\bullet (\Pi\Omega^{0,\bullet}_{\C^3} (\mc L^N_{AdS_7\times S^4}))\right)^!\cong \mc U_{\omega} \left ( \Pi\Omega^{0,\bullet}_{\C^3} (\mc L^{r=0}_{AdS_7\times S^4} )\right ).\] where the right hand side denotes a twisted factorization envelope of the local $L_\infty$-algebra $\Pi\Omega^{0,\bullet}_{\C^3} (\mc L^{r=0}_{AdS_7\times S^4} )$. Moreover, upon deforming by the Maurer-Cartan element $S\in \mf {osp}(6|2)$, the factorization algebra $\mc U_{\omega} \left ( \Pi\Omega^{0,\bullet}_{\C^3} (\mc L^{r=0}_{AdS_7\times S^4} )\right )$ has no sections away from a copy of $\C\subset \C^3$, and its restriction to this copy of $\C$ is equivalent to a twisted factorization envelope of the local Lie algebra $\operatorname {Diff}_{\C}$ of holomorphic differential operators on $\C$. 
\end{conj}

Here, the twisting cocycle $\omega$ comes from the shifted Poisson tensor that was induced by the flux in section \ref{sec:ads}. The content in verifying this conjecture is to explicitly compute the twisting coming from the flux sourced by the brane, and check that upon deforming by the element $S$, it induces the correct cocycle on $\operatorname{Diff}(\C^\times)$

The comment in \ref{eqn:winfty} constitutes a very meager consistency check for the second part of this conjecture, where we observe that the Schur limit of the character of the costalk of $\mc U_{\omega} \left ( \Pi\Omega^{0,\bullet}_{\C^3} (\mc L^{r=0}_{AdS_7\times S^4} ) \right)$ recovers the vacuum character of the $W_{1+\infty}$ vertex algebra.

There is a deformation of the twisted factorization envelope of $\operatorname{Diff}_\C$ which yields the $\mc W_{1+\infty}$ vertex algebra, also referred to as the affine Yangian of $\mf {gl}(1)$. In \cite{CostelloM5}, Costello finds this deformation from a loop level computation in his 5d noncommutative gauge theory. We also expect to be able to lift this to the minimal twist. We summarize this expectation in a conjecture. 

\begin{conj}[R-Saberi-Williams]
There is an equivalence of holomorphic factorization algebras \[ \left( \clie^\bullet_\hbar (\Pi\Omega^{0,\bullet}_{\C^3} (\mc L^N_{AdS_7\times S^4}))\right)^!\cong \mc U_{\hbar, \omega} \left ( \Pi\Omega^{0,\bullet}_{\C^3} (\mc L^{r=0}_{AdS_7\times S^4} )\right ).\] where the right hand side denotes a deformation of the factorization algebra in the previous conjecture induced by loop-level effects in our eleven-dimensional model. Moreover, upon deforming by the Maurer-Cartan element $S\in \mf {osp}(6|2)$, the factorization algebra $\mc U_{\hbar, \omega} \left ( \Pi\Omega^{0,\bullet}_{\C^3} (\mc L^{r=0}_{AdS_7\times S^4} )\right )$ has no sections away from a copy of $\C\subset \C^3$, and its restriction to this copy of $\C$ is equivalent to the $\mc {W}_{1+\infty}$ vertex algebra.
\end{conj}

\parsec[]
We now move on to finite $N$ statements. For the lowest steps of the filtration, we can make some very concrete statements.

\begin{conj}[R-Saberi-Williams]
Upon deforming by $S\in \mf {osp}(6|2)$, the factorization algebra $\mc U_\omega (\mc G^{(-1)}_{\C^3} )$ has no sections away from a copy of $\C\subset \C^3$ and its restriction to this copy of $\C$ is equivalent to the Heisenberg vertex algebra.
\end{conj}
To check this conjecture, it remains to compute the pullback of the twisting cocycle $\omega$ under the inclusion of $\mc G^{(0)}$ and see that it deforms to the Heisenberg cocycle. 

\parsec[]
There is a distinguished Lie sub-algebra of the algebra of differential operators on $\C^\times$ which is given by the Witt-algebra of vector fields. The central extension of $\operatorname{Diff} (\C^\times)$ induced by $\omega$ above restricts to the Virasoro central extension. Similarly, in proposition \ref{prop:g0e36} we have identified a distinguished local super-Lie algebra inside $\Pi\Omega^{0,\bullet} _{\C^3}(\mc L^{r=0} _{AdS_7\times S^4} )$ given by $\mc E(3|6)$.

Accordingly, we conjecture the following
\begin{conj}[R-Saberi-Williams]
Upon deforming by $S\in \mf{osp}(6|2)$, the factorization algebra $\mc U_\omega( \mc E(3|6) )$ has no sections away from a copy of $\C\subset \C^3$ and its restriction to this copy of $\C$ is equivalent to the Virasoro vertex algebra.
\end{conj}

Again, to check this conjecture it remains to compute the pullback of the twisting cocycle $\omega$ along the inclusion $\mc E(3|6)\to \Pi\Omega^{0,\bullet}_{\C^3} (\mc L^{r=0}_{AdS_7\times S^4} )$ and compare its deformation with the cocycle giving the Virasoro central extension.

We can once again perform a consistency check at the level of characters of costalks. Indeed, we see that the plethystic exponential of the specialized character $g_0(y=1, y_3=1, q) = \frac{q^2}{1-q}$ is exactly the vacuum character of the Virasoro algebra. 

Moreover, note that combining with conjecture \ref{conj:classical}, we expect maps 
\[\mc U_\omega( \mc E(3|6) )\to \left (\clie^\bullet (\Pi\Omega^{0,\bullet}_{\C^3} (\mc L^N_{AdS_7\times S^4}))\right)^! \to \Obs^N_{CFT}\] for every $N$. This map can be thought of as a Noether-type map associated to an  $\mc E(3|6)$-symmetry of the minimal twist of any finite rank 6d $\mc N=(2,0)$ theory \cite{CG2}.

\parsec[]
More generally, the $\mc W_{1+\infty}$ algebra has as quotients, the $\mc W_N$ algebras when the central charge is set equal to $N$. Accordingly, we dream of the following:

\begin{spec}
Under an integrality condition on the central charge, the map \[ (\Obs_{grav} |_{\C^3} )^! \to \Obs^N_{CFT}\] factors as
\[
\begin{tikzcd}
(\Obs_{grav} |_{\C^3} )^! \ar[r]\ar[d]  & \Obs^N_{CFT} \\
\mc U_{\hbar, \omega} (\Omega^{0,\bullet}_{\C^3} (\mc L^N_{AdS_7\times S^4}))/\mc U_{\hbar, \omega} (\prod _{j\geq {N-1}} \mc G^{(j)}_{\C^3} ) \ar[ur]
\end{tikzcd}
\]
\end{spec}

We can perform a consistency check of the above speculation at the level of characters of costalks. It is expected that the superconformal deformation deforms $\Obs^N_{CFT}$ to the $\mc{W}_N$ vertex algebra. On the other hand, we have that 


\begin{prop}
Upon specializing $y=1,y_3=1$ (so that $y_1 y_2 = 1$), one has 

\begin{align*}
\chi \left ( \Omega^{0,\bullet}_{\C^3,c} (\mc L^N_{AdS_7\times S^4})(0)/ \left ( \mc G^{(-1)}_{\C^3,c}\times \prod _{j\geq {N-1}} \mc G^{(j)}_{\C^3,c}\right )(0)\right ) & = \sum_{j \geq 0}^{N-2} g_j (y_1,y_2, y_3=1,y=1,q) \\
& = \frac{q^2 + q^3 + \cdots + q^{N}}{1-q} 
\end{align*}

The plethystic exponential of the right hand side agrees with the vacuum character of the $W_{N}$ vertex algebra.
\end{prop}
\begin{proof}
By induction it suffices to show that the specialization of the single particle local character $g_j$ of the factorization algebra $\mc U(\mc{G}^{(j)})$ is $q^{j+2} / (1-q)$. 
We have already seen this in the case $j=-1,0$, so it suffices to show this when $k \geq 1$.

First observe that the denominator becomes
\begin{equation}
(1-y_1 q)(1-y_2q) (1-q) .
\end{equation}

Next, we observe that the numerator of $g_j (y_1,y_2,y_3=1,y=1,q)$ can be factored as
\begin{align*}
q^{3 + 3j/2} \left(q^{-(j+2)/2} + q^{-(j-2)/2} - q^{-j/2} (y_1+y_2) \right) 
& = q^{j+2} (1 + q^2 - q (y_1 + y_2)) \\
& = q^{j+2} (1 - y_1 q) (1-y_2 q) 
\end{align*}
where in the last line we have used $y_1 y_2 = 1$.
The result follows.
\end{proof}

\parsec{}
In \cite{raghavendran2022holographic} we try to explicitly characterize the discrepancy between the characters of $\mc U (\mc G^{(j)}_{\C^3})$ and expectations about the superconformal index of the finite rank 6d $\mc N=(2,0)$ theories computed via instanton counting techniques \cite{Kim:2013nva} and the "giant graviton expansion" \cite{Arai_2020,}, \cite{Imamura}. It would be interesting to try and categorify the discrepancies and identify them in terms of modules for $E(3|6)$.


\iffalse
\parsec{}

We would also like to point out compatibility of our expression with a certain ``minimally reduced'' index considered in \cite{Gaiotto:2021xce}. 
This minimal reduction is the result of sending certain parameters to zero while keeping some expression in the fugacities fixed.
To consider it it is useful to make the following change of variables: 
\begin{equation}
z_i = y_i q, \quad w_1 = yq, \quad w_2 = y^{-1} q^2 .
\end{equation}
These variables satisfy the constraint $z_1 z_2 z_3 = w_1 w_2$. 

This minimally reduced index corresponds to taking the following limit in the new fugacities
\begin{equation}\label{eqn:limitgaitto}
z_3 , w_2 \to 0 .
\end{equation}
This yields an index which only accounts for operators which transform trivially with respect to the symmetries that the fugacities $z_3,w_2$ correspond to. 
This will result in an index which has three remaining fugacities.

\begin{prop}
\label{prop:gaiotto}
The limit $z_3 , w_2 \to 0$ of the expression $\chi_{N}(z_i,w_a)$ is 
\begin{equation}
\prod_{a=1}^N \prod_{b,c \geq 0} \frac{1-w_1^{a-1}z_1^{b+1} z_2^{c+1}}{1-w_1^a z_1^b z_2^c} .
\end{equation}
\end{prop}
\begin{proof}
It is easy to see that the $z_3,w_2 \to 0$ limit of $g_{-1}$ is 
\begin{equation}
g_{-1}(z_1,z_2,w_1) = \frac{w_1 - z_1 z_2}{(1-z_1)(1-z_2)} 
\end{equation}
and the $z_3,w_2 \to 0$ limit of $g_0$ is 
\begin{equation}
g_0(z_1,z_2,w_3) = w_1 g_{-1}(z_1,z_2,w_1) .
\end{equation}

In the coordinates $z_i,w_a$ the expression $g_k$, for $k \geq 1$, in \eqref{eqn:gk} can be written as
\begin{equation}
\label{eqn:gk}
\begin{array}{lllll}
g_k (z_i,w_a) \define & \left( z_1z_2z_3 p_k(w_1,w_2) (z_1+z_2+z_3) + p_{k+2}(w_1,w_2)  \right. \\
&\displaystyle \frac{\left.  -z_1z_2z_3 p_{k-1}(w_1,w_2) - p_{k+1}(w_1,w_2) (z_1z_2+z_2z_3+z_1z_3) \right)}{(1-z_1)(1-z_2)(1-z_3)} .
\end{array}
\end{equation}
Here $p_k(w_1,w_2) = \sum_{i+j=k} w_1^i w_2^j$. 

From this expression it is easy to see that $\lim_{z_3,w_2 \to 0} g_k$ is 
\begin{equation}
g_k(z_1,z_2,w_1) = w_1^{k+1} g_{-1}(z_1,z_2,w_1) .
\end{equation}
The result follows from applying the plethystic exponential.
\end{proof}

The $z_3,w_2 \to 0$ limit of our index is quite similar, though not exactly, the index of a four-dimensional $\mc{N}=1$ theory on $\C^2$ where the fugacities $z_1,z_2$ count holomorphic derivatives in each of the complex directions.
Also, note that this minimally reduced index further reduces to the Schur limit (so the character of the $W_N$ vertex algebra) by specializing $z_1 = w_1$.

\subsection{Comparisons to expansions of superconformal indices}

In the final section we would like to exhibit a series of direct consistency checks with our conjectural exact formula for the index of the non-abelian six-dimensional superconformal theory with a number of expansions that have appeared in recent literature. 

\parsec
Let us first focus on the superconformal theory associated to the Lie algebra $\mf{sl}(2)$ (so type $A_1$).
Our conjecture for the superconformal index in this case is the plethystic exponential of $\tilde f_2 (y_i,y,q)$ from equation \eqref{eqn:6dtwo}.
We expand the formal single particle index $\tilde f_2 (y_i, y, q)$ as a series in the variable~$q$, yielding
\begin{align*}
\tilde f_2 (y_i,y,q) & = y^2 q^2 + \left(1 - \chi_{[0,1]}(y_i) y + \chi_{[1,0]} y^2 \right) q^3 \\
& + \left(y^{-2} - \chi_{[0,1]}(y_i) y^{-1} + 2 \chi_{[1,0]}(y_i) - \chi_{[0,1]} (y_i) y + \chi_{[2,0]}(y_i) y^2 \right) q^4 + O(q^5) .
\end{align*}
From this expression, we obtain the $q$-expansion of the index $\tilde \chi_2(y_i,y,q) = \text{PExp}[\tilde f_2]$ as 
\begin{align*}
\tilde \chi_2(y_i,y,q) & = 1 + y^2 q^2 + \left(1-\chi_{[0,1]}(y_i) y + \chi_{[1,0]}(y_i)y^2\right)q^3 \\ 
& + \left(y^{-2} - \chi_{[0,1]}(y_i) y^{-1} + 2 \chi_{[1,0]}(y_i) - \chi_{[0,1]} (y_i) y + \chi_{[2,0]}(y_i) y^2 + y^4\right)q^4 + O(q^5)
\end{align*}

Similarly, for the $\mf{gl}(2)$ theory $\chi_2 = \text{PExp}[f_1 + \tilde f_2] = \chi_1 \cdot \tilde \chi_2$ we find the expansion
\begin{align*}
\chi_2 (y_i,y,q) & = y q + \left(y^{-1} - \chi_{[0,1]}(y_i) + \chi_{[1,0]}(y_i) y + 2y^2 \right) q^2 \\ 
& + \left( \chi_{[1,0]}(y_i) y^{-1} - (\chi_{[1,1]}(y_i)-2) + (\chi_{[2,0]}(y_i) - 2 \chi_{[0,1]}(y_i)) y + 2 \chi_{[1,0]}(y_i) y^2 + 2y^3\right) q^3 \\ & + O(q^4) .
\end{align*}

We observe that these $q$-expansions agree precisely with the expansions in \cite{Kim:2013nva} for the $\mf{gl}(2)$ theory  
(see equations (3.51) and (3.65) of \textit{loc. cit.}).

\parsec

We proceed to compare expansions of our exact expression for the $\mf{gl}(3)$ theory to those in \cite{Kim:2013nva}. 
Recall that our conjectural $\mf{gl}(3)$ index is given by the local character of the holomorphic factorization algebra $\Obs_3$:
\begin{equation}
\chi_3 (y_i,y,q) = \text{PExp}[f_3(y_i,y,q)] = \chi_2(y_i,y,q) \cdot \text{PExp}[g_3(y_i,y,q)]  .
\end{equation}
Here, $f_3(y_i,y,q)$ is the single particle local character for the holomorphic factorization algebra $\Obs_3$ and $g_3(y_i,y,q)$ is given in equation \eqref{eqn:gk}. 

Since $g_3(y_i,y,q) = y^3 q^3 + O(q^4)$ we see that $\chi_3$ and $\chi_2$ agree up to order $q^2$ and the difference at order $q^3$ is simply
\begin{equation}
\chi_3(y_i,y,q) - \chi_2(y_i,y,q) = y^3q^3 + O(q^4) .
\end{equation}
This is again in exact agreement with the index for the $\mf{gl}(3)$ theory computed \cite{Kim:2013nva} up to order $q^3$ (see equation (3.79) of \textit{loc. cit.}). 

\parsec

Next, we compare to expansions for the $\mf{sl}(N)$ theory computed in \cite{Imamura}, where the method of the `giant graviton' expansion is used.
It will be convenient to change the variables $(y_i, y, q) \to (y_i,x,q)$ where 
\begin{equation}
x = qy .
\end{equation} 
We will again expand in powers of $q$.\footnote{To match precisely with the equations in \cite{Imamura} we note that it is necessary to relable the variables $y_i \leftrightarrow u_i$, $x \leftrightarrow \check{x}$, and $q \leftrightarrow y$ where the variable $y$ is distinct from the one we use in this paper!}

Starting with the $\mf{sl}(2)$ theory we find that up to order $q^4$ the single particle index is
\begin{align*}
\tilde f_2 (y_i,x,q) & = x^2 + \chi_{[1,0]}(y_i) x^2 q \\
& + \left(-\chi_{[0,1]}(y_i) x + \chi_{[2,0]}(y_i) x^2 \right) q^2 +  \left(1-x-\chi_{[1,1]}(y_i) x + \chi_{[3,0]}(y_i) x^2 \right) q^3 \\
& + \left(2 \chi_{[1,0]}(y_i) - \chi_{[2,1]}(y_i) x + \chi_{[4,0]}(y_i)x^2 \right)q^4 + O(q^5) .
\end{align*}

It follows that the plethystic exponential $\tilde \chi_2(y_i,y,q)$ of this expression has $q$-expansion
\begin{align*}
\tilde \chi_2(y_i,y,q) & = \frac{1}{1-x^2} +  \frac{x^2}{1-x^2} \chi_{[1,0]}(y_i) q \\
& + \left(- x \chi_{[0,1]}(y_i) +  x^2 (1+x^2)\chi_{[2,0]}(y_i)\right) \frac{1}{1-x^2} q^2 \\ 
& + \left( 1 - x - x^3 + x^6 + (-x - x^3 + x^4) \chi_{[1,1]}(y_i)  + (x^2 + x^4 + x^6)\chi_{[3,0]}(y_i)  \right) \frac{1}{1-x^2} q^3 \\
& + O(q^4) 
\end{align*}
This agrees with the expansion in \cite{Imamura} (see equation (68)) except for the $\mf{sl}(3)$-scalar term at order $q^3$. 
We find $(1-x-x^3+x^6) / (1-x^2) = (1-x^3-x^4-x^5) / (1+x)$ whereas Imamura's result is $1 / (1+x)$. 

Similarly, we can obtain the $q$-expansions for the local character $\chi_3(y_i,x,q)$ of the factorization algebra $\Obs_3$ and compare it to the $q$-expansion for the superconformal index of the $\mf{sl}(3)$ theory in \cite{Imamura}. 
Up to order $q^2$ we have
\begin{align*}
\chi_3(y_i,x,q) & = \frac{1}{(1-x^2)(1-x^3)} + \frac{x^2}{(1-x)(1-x^3)} \chi_{[1,0]}(y_i) q \\
& + \left((-x -x^2 + x^5)\chi_{[0,1]}(y_i) + (x^2 + x^3 + x^4 + x^5+ x^6) \chi_{[2,0]}(y_i) \right) \frac{1}{(1-x^2)(1-x^3)} q^2 \\
& + O(q^3) .
\end{align*}
Again, we find a discrepancy of our expansion compared to \cite{Imamura} at order~$q^3$.
It would be interesting to explain the physical or representation theoretic sources of these discrepancies in each of these cases.
\fi





\bibliographystyle{amsalpha}
\bibliography{bibliography.bib}

\end{document}
