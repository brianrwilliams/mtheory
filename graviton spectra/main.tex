\documentclass{amsart}
\usepackage[dvipsnames]{xcolor}

\usepackage{macros}
\setcounter{secnumdepth}{4}

\usepackage{subfiles}

\newcommand\brian[1]{\colorbox{BurntOrange}{#1}}
\newcommand\surya{\todo[color={blue!100!green!33},author=Surya]}


\title{Twisted graviton spectra of $AdS_4\times S^7$ and $AdS_7\times S^4$} 
\author{Surya Raghavendran and Brian R. Williams}
\begin{document}

\maketitle

\begin{abstract} %required
\end{abstract}

\tableofcontents

In \cite{CLsugra}, Costello and Li gave a rigorous definition of what it means to {\em twist} a theory of supergravity, which is nicely compatible with the mathematically more familiar notion of twisting supersymmetric gauge theories \cite{WittenTwist}.
In physical terms, twisted supergravity is supergravity in a background where a bosonic ghost for supersymmetry takes a nonzero value $Q$, where $Q$ is a nilpotent supercharge.
In principle, one can take the supercharge to be any square-zero covariantly constant spinor (so it satisfies the equations of motion of supergravity).
A class of such supercharges lie in the supersymmetry algebra, which is of the sort that we will focus on in this paper.

Following this procedure, we have given a proposal for the twists of eleven-dimensional supergravity on flat space in \cite{RSW,SWspinor}.
See also \cite{EHsugra} for details on the further topological, non-minimal, twist.
The model is akin to models of ten-dimensional twisted supergravity in terms of a certain part of topological string theory proposed in \cite{CLsugra} and further pursued in \cite{CLtypeI,CLbcov,CPkoszul} which generalizes the use of Kodaira--Spencer theory of Calabi--Yau manifolds in the ordinary $B$-model on Calabi--Yau threefolds \cite{bcov}.
The goal of this paper is to give a descriptions of the minimally twisted model of eleven-dimensional supergravity placed on maximally symmetric spacetimes; thus providing further evidence for our model as a twist of supergravity.
Insomuch, we produce definitions of twisted versions of the $AdS_4\times S^7$ and $AdS_7\times S^4$ spacetimes.

In supergravity, these backgrounds arise as near-horizon limits of the backgrounds sourced by some number of M2 and M5 branes in flat space respectively. 
To follow an analogous procedure natively in our twisted context we introduce leading order couplings of our eleven-dimensional model to twisted M2 and M5 branes. 
These couplings determine explicit deformations of formal moduli problem described by our twisted model on complement of the brane in flat spacetime.
Perturbatively, the undeformed formal moduli space can be captured by the data of an $L_\infty$ algebra; we conjecture that deforming the theory in the complement of the brane by a solution to the resulting curved Maurer-Cartan equation is perturbatively equivalent to the twist of the theory on $AdS_4\times S^7$ and $AdS_7\times S^4$.

We give definitions of supergravity states in our twisted $AdS_4\times S^7$ and $AdS_7\times S^4$ backgrounds, which can be thought of as particular field configurations that are localized at points on the conformal boundary of $AdS$. 
We compute graded characters of the proposed state spaces and find exact matches with counts of gravitons on $AdS_4\times S^7$ and $AdS_7\times S^4$ respectively.

One of the insights of \cite{SWchar} was that supersymmetric indices count exactly local operators in the minimal twist---accordingly, we may think of the space of local operators in the minimal twist as categorifying the index.
Of course, combining observables also endows local operators with an algebraic structure.
In the approach to quantum field theory developed by Costello and Gwilliam \cite{fact2} that we follow here, the local operators quantum field theory are endowed with the structure of a factorization algebra.
Therefore, an appealing aspect of our approach is that the space of supergravity states is obtained as the value of a factorization algebra on a small open disk.
A complete characterization of this factorization algebra would follow, in principle, from the twisted holography proposal of by Costello, Gaiotto, Li, and Paquette in \cite{CLsugra,CGhol,CPkoszul} which is a highly structured avatar of the AdS/CFT correspondence that is expected to hold at the level of supersymmetric twists.
We do not pursue that proposal in our context here.

The next strand of evidence we pursue is by matching symmetries. In the physical theory, the $AdS_4\times S^7$ and $AdS_7\times S^4$ backgrounds carry actions of the 3d $\mc N=8$ and 6d $\mc N=(2,0)$ superconformal algebras. We show that our conjectural descriptions of twists of these backgrounds carry actions of the minimal twists of the corresponding superconformal algebras. 

With these pieces of evidence in hand, in sections \ref{sec:e16}, \ref{sec:e36},  we then turn to study some representation theoretic aspects of the state spaces constructed in section~\ref{sec:states}. We identify certain $\C^\times$ actions on our eleven-dimensional model that combine rescalings in directions normal to branes withoh a certain rescaling of the space of fields---this induces a decomposition of the space of fields that we dub the \textit{graviton decomposition}. The weight $0$ parts of these decompositions are certain local $L_\infty$-algebras whose costalks recover the linearly compact super-Lie algebras $E(1|6)$ and $E(3|6)$. We thus see that these linearly compact super-Lie algebras act on the spaces of supergravity states constructed in section \ref{sec:states}. We explicate their action on nonzero weight spaces of the graviton decomposition.

In the final section of the chapter, we motivate some current work in progress that leverages the uncovered appearance of exceptional linearly compact super-Lie algebras for holographic means. Eleven-dimensional supergravity on $AdS_4\times S^7$ and $AdS_7\times S^4$ is expected to be equivalent to the large $N$ limit of the worldvolume theories of $N$ M2 branes and $N$ M5 branes respectively. 
%Typically, this definition is made in situations where the relevant boundary value problem has a unique solution, in which case one may label states by the corresponding boundary values. Moreover, one may think of such boundary values as arising from modifications of a vacuum boundary condition at


\subfile{content/br}

\subfile{content/index}

%\subfile{content/suco}

\subfile{content/e16}

\subfile{content/e36}

%%\documentclass[11pt]{amsart}
%
%%\usepackage{../macros-master}
%\usepackage{macros-fivebrane}
%
%\begin{document}

\section{Local operators in twisted $M$ theory}

The notion of a factorization algebra captures both the local operators of a theory together with the non-local operators that on can define from the local ones via descent.
From the data of a factorization algebra, one can recover the space of local operators by the following formal construction. 
Let $\Obs$ be the factorization algebra of observables of some theory defined on a smooth manifold $M$.
The space of local operators at point $p \in M$ is, in a precise sense, the limiting behavior of the factorization algebra evaluated on the system of open sets which contain the point~$p$. 

Generally this limit is difficult to compute, but for certain theories it is possible to give a concise expression which captures the essential features of the theory.
For example, in a holomorphic theory, the algebra of local operators is equivalent to the algebra generated by holomorphic derivatives of fields evaluated at a point.

In this section we recall the essentials of the theory of local operators for holomorphic-topological theories. 
We consider a way of counting operators in a topological-holomorphic theory, called the `local character' of a holomorphic-topological theory \cite{SWchar}, and compare it to the superconformal index.
We then present a few simple examples and then go on to set up the theory of local operators associated to factorization algebras~$\clie_\bu(\cG_{N,c})$ we constructed in \S \ref{s:fact}.

\subsection{Local operators in topological-holomorphic theories}

A factorization algebra encodes the many ways to combine observables supported on arbitrary open sets. 
Local operators, on the other hand, exist just at a point in spacetime.
From the factorization algebra perspective one can recover local operators by looking at observables which are supported on \text{every} open set which contains the given point; mathematically this is computed by a limit. 

Precisely, in \cite[Definition 10.1.0.1]{CG2} the space of local operators of a factorization algebra $\cF$ at a point $p \in M$ is defined by the limit $\cF(p) = \lim_{U \ni p} \cF(U)$ which runs over open sets $U \subset M$ containing~$p$.

We will only consider local operators on affine space $\R^d$. 
In this case, we will have the additional property that the factorization algebras are translation invariant.
At the level of local operators this means that the translation map $\tau_{p \to p'}$ induces an isomorphism $\cF(p) \simeq \cF(p')$. 
Without loss of generality, we will consider expressions for local operators at $0 \in \R^d$.

For topological-holomorphic theories the local operators take a very familiar form.
As an algebra they are generated by (holomorphic) derivatives of the fields evaluated at the specified point. 
More precisely, the local operators depend only on the $\infty$-jets of the fields at a point.
In this section we carefully formulate this result and give some examples.

\parsec[s:free]

%Suppose that $V$ is a translation invariant holomorphic vector bundle on $\C^n$ equipped with a $\Z \times \Z/2$ bigrading. 
%Let $\cV$ denote its sheaf of holomorphic sections.
%The {\em space of fields} of a holomorphic field theory on $\C^n$ is the $\Z \times \Z/2$ graded complex of vector bundles
%\[
%\Omega^{0,\bu}(\C^n, V) \cong \Omega^{0,\bu}(\C^n) \otimes V_0 
%\]
%where $V_0$ is the fiber of $V$ at $0 \in \C^n$.
%Our grading conventions are so that $\d \zbar_i$ has bidegree $(1,0)$.
%
%As introduced in \cite{BWhol,LiVertex,CG2}, a {\em holomorphic field theory} is a holomorphic vector bundle $V$ as above equipped additionally with:
%\begin{itemize}
%\item The structure of a local (super) $L_\infty$ algebra on $V[-1]$ with structure maps given by holomorphic polydifferential operators
%\[
%[\cdot]_k \colon \cV[-1]^{\times k} \to \cV[1-k] .
%\]
%\end{itemize}
%A {\em free} holomorphic theory has $[\cdot]_k = 0$ for $k > 1$.
%
%\parsec
A topological-holomorphic theory exists on spacetimes of the form $S \times X$ where $S$ is a smooth manifold and $X$ is a complex manifold (possibly equipped with some auxiliary geometric structures). 
The typical space of fields of a holomorphic-topological theory in the BV formalism is
\beqn\label{eqn:cE}
\cE = \Omega^\bu (S) \hotimes \Omega^{0,\bu}(X, V) 
\eeqn
where $V$ is a graded holomorphic vector bundle on $X$.
The underlying free theory is described by a differential on the space of fields of the form
\[
\d_{dR} + \dbar + Q^{hol} .
\]
Here $\d_{dR}$ is the de Rham differential acting on $S$, $\dbar$ is the Dolbeault operator acting on $X$, and $Q^{hol} \colon V \to V[1]$ is a holomorphic differential operator of cohomological degree~$+1$.
This means that the free, linear equations of motion for a field $\varphi$ take the form
\[
\d_{dR} \varphi + \dbar \varphi + Q^{hol} \varphi = 0 .
\]
Taking into account linear gauge symmetries corresponds to cohomology---solutions to the equations of motion modulo the image of $\d_{dR} + \dbar + Q^{hol}$.

Notice that $\cE$ is a sheaf of cochain complexes---it makes sense to restrict the fields to any open set $U \subset S \times X$. 
The factorization algebra of observables of the free theory whose fields are as above assigns to an open set $U \subset S \times X$ the cochain complex
\[
\Obs \colon U \mapsto \cO(\cE(U)) = \Sym \left(\cE(U)^\vee \right) 
\]
equipped with the induced differential.

Some remarks are in order:
\begin{itemize}
\item If $V$ is a topological vector space then $\cO(V) = \Sym(V^\vee)$ denotes the algebra of polynomials on~$V$.
Here~$V^\vee$ is the topological dual.
\item The topological dual of $\cE(U)$ is $\cE(U)^\vee \simeq \overline{\cE}^!_c(U)$ where the bar denotes distributional sections, the subscript $c$ denotes compact support, and $!$ denotes the Serre dual. 
Explicitly, if $U = U' \times U'' \subset S \times X$ then 
\[
\overline{\cE}^!_c(U' \times U'') \simeq \overline{\Omega}^\bu(U') \otimes \overline{\Omega}^{n,\bu}(U'',V^*)[n+m] 
\]
where $\dim_\C (X) = n$ and $\dim_\R (S) = m$. 
\end{itemize}

Let's restrict to the case that $S \times X = \R^m \times \C^n$ and suppose that the bundle $V \to \C^n$ is translation invariant with fiber $V_0$ over $0 \in \R^m \times \C^n$.
We also assume that the operator $Q^{hol}$ is translation invariant.   
 
The jet expansion at $0 \in \R^m \times \C^n$ determines a map of cochain complexes
\[
\cE(\C^n \times \R^m) \to V_0 [[x_i, \d x_i,z_j, \zbar_j, \d \zbar_j]] 
\]
The differential on the right hand side is $\d_{dR} + \dbar + Q^{hol} = \d x_i \del_{x_i} + \d \zbar_j \del_{\zbar_j} + Q^{hol}$ where $Q^{hol}$ is some holomorphic differential operator in the $z_j$ variables. 
Since all structure maps are given by holomorphic polydifferential operators, the canonical map 
\[
V_0 [[x_i, \d x_i,z_j, \zbar_j, \d \zbar_j]] \xto{\simeq} V_0 [[z_j]] 
\]
which sends $x_i, \d x_i,\zbar_j \d \zbar_j \mapsto 0$ is a quasi-isomorphism. 
The only remaining differential on the right hand side is~$Q^{hol}$. 
In summary, we see that the jet expansion at $0 \in \R^m \times \C^n$ determines a map of cochain complexes $\cE(\R^m \times \C^n) \to V_0[[z_j]]$. 

\begin{lem}
\label{lem:taylor}
Suppose that $\cE$ is the sheaf of cochain complexes representing the free topological-holomorphic theory on $S \times X = \R^m \times \C^n$ and consider the factorization algebra of observables~$\Obs = \cO (\cE)$. 
Then, the Taylor expansion map
\beqn\label{eqn:taylor}
\cE(\C^n \times \R^m) \to V_0[[z_0,\ldots,z_n]]
\eeqn
induces a quasi-isomorphism of commutative dg algebras
\[
\Obs(0) \simeq \cO \left( V_0[[z_1,\ldots,z_n]] \right) .
\]
%Notice that when $\cL$ is abelian with differential $\d_{dR} + \dbar + Q^{hol}$, then there is a quasi-isomorphism
%\[
%\Obs(0) \simeq \cO \left( V_0[[z_1,\ldots,z_n]][1] \right) 
%\]
%where the right hand side is equipped with the differential $Q^{hol}$. 
\end{lem}
\begin{proof}
Suppose that $D_\R \times D_\C \subset \R^m \times \C^n$ is a product of a real $m$-disk times a complex $n$-disk containing the origin.
The algebra of observables supported on $D_\R \times D_\C$ is quasi-isomorphic to 
\[
\cO\left( \cO^{hol}(D_\C) \otimes V_0 \right) .
\]

Observe that there is a canonical map on fields 
\[
\cO^{hol}(D_\C) \otimes V_0 \to V_0[[z_1,\ldots,z_n]]
\]
given by taking the power series expansion at $0 \in D_\R \times D_\C$. 
If an observables on $D_\R \times D_\C$ depends on only the value of the field and its derivatives at $0 \in D_\R \times D_\C$ then it automatically factors through this map. 
In particular, this means that there is a quasi-isomorphism of local operators with functions on $V_0[[z_1,\ldots,z_n]]$,
\[
\Obs(0) \simeq \cO\left(V_0 [[z_1,\ldots,z_n]]\right).
\] 
\end{proof}

Let's unpack this result explicitly. 
Using the $n$-dimensional residue, we can identify the topological dual of $V_0[[z_1,\ldots,z_n]]$ with the vector space
\beqn
\frac{\d z_1}{z_1} \cdots \frac{\d z_n}{z_n} V_0^* [z_0^{-1}, \ldots,z_n^{-1}] .
\eeqn
This is the space of linear local operators. 
If $\chi \colon V_0 \to \C$ is a dual vector in~$V_0^*$
then we obtain a linear local operator at $0 \in \R^m \times \C^n$ on the space of fields by the assignment
\[
\varphi \mapsto \del_{z_1}^{k_1} \cdots \del_{z_n}^{k_n} \<\chi,\varphi\> (0) 
\]
where $k_i \geq 0$. 
Under the quasi-isomorphism of the lemma above, this corresponds to the linear local operator 
\[
\frac{\d z_1}{z_1^{k_1+1}} \cdots \frac{\d z_n}{z_n^{k_n+1}} \chi .
\]

\parsec[s:interaction]

It is not hard to turn on interactions in the description above. 
An interacting theory in the BV formalism is described by a local $L_\infty$ algebra structure on $\cL = \cE[-1]$, where $\cE$ is the sheaf of fields.
For a topological-holomorphic theory the higher $L_\infty$ structure maps $[\cdot]_k$ of the local $L_\infty$ algebra are required to be given by holomorphic polydifferential operators and $[\cdot]_1 = \d_{dR} + \dbar + Q^{hol}$.  
For more details we refer to the definitions in \cite{GRWthf}.

In this situation, the factorization algebra of classical observables supported on an open set $U \subset S \times X$ is given by the Chevalley--Eilenberg cochains on the $L_\infty$ algebra $\cL(U)$. 
This defines a factorization algebra 
\[
\Obs \colon U \mapsto \clie^\bu(\cL(U)) .
\]
We will now give a concise presentation for the {\em local} operators in a topological-holomorphic theory. 

On $S \times X = \R^m \times \C^n$ we can also ask that all $L_\infty$ structure maps be translation invariant. 
If this is the case, one obtains the induced structure of an $L_\infty$ algebra on the (shift of the) jets of the fields supported at $0 \in \R^m \times \C^n$
\[
V_0 [[z_1,\ldots,z_n]] [-1] .
\]
The $[\cdot]_1$ operation is precisely $Q^{hol}$ as above.
The Taylor expansion map \eqref{eqn:taylor} is a map of $L_\infty$ algebras. 
Combining this with Lemma \ref{lem:taylor}, one gets a quasi-isomorphism of cochain complexes between the local operators of an interacting topological-holomorphic theory in terms of Lie algebra cohomology
\[
\Obs(0) \simeq \clie^\bu\left(V_0[[z_1,\ldots,z_n]][-1]\right) .
\]

%We recall the reader of the standard dictionary between the space of fields of a BV theory and the local $L_\infty$ algebra---
%if the local Lie algebra is $\cL$, then the space of fields is $\cL[1]$. 
%The Chevalley--Eilenberg complex of $\cL$ is then functions on the fields $\cO(\cL[1])$ equipped with the non-linear BRST operator.

\parsec[s:envelope]

There is another way that observables are presented in a degenerate version of the BV formalism.
Suppose that~$\cE$ is the sheaf of sections of some graded vector bundle~$E$ on a manifold~$M$.
We have seen that the observables~$\cO(\cE) = \Sym(\cE^*)$ has the structure of a factorization algebra---we now consider the $!$-dual factorization algebra.
That is, we consider the factorization algebra 
\[
U \subset M \mapsto \Sym \left(\cE_c(U) \right) 
\]
where $U \to \cE_c(U)$ is the cosheaf of compactly supported sections of the bundle~$E$.

%Suppose that $\cL$ is a local Lie algebra on a manifold $M$. 
%Then, one can consider the factorization algebra
%\[
%\cF = \clie_\bu(\cL_c)
%\]
%which assigns to an open set $U$ the cochain complex $\clie_\bu(\cL_c(U))$.
%This is the $!$-dual of the factorization algebra $\clie^\bu(\cL)$.
%For topological-holomorphic local Lie algebras there is still an algorithm for computing $\cF(p)$ for a point~$p \in M$.
%
%We will assume that $\cL$ is a translation invariant topological-holomorphic local Lie algebra whose underlying sheaf of cochain complexes is
%\[
%\cL = \Omega^\bu (\R^m) \hotimes \Omega^{0,\bu}(\C^n, L)
%\]
%Here $L$ is a translation invariant holomorphic vector bundle on $\C^n$ and the differential in the complex is $\d_{dR} + \dbar + Q^{hol}$ as above.

\begin{lem}
\label{lem:envelope}
Suppose that $\cE$ is the sheaf of fields of a free holomorphic theory as in~\eqref{eqn:cE} and consider the factorization algebra~$\cF = \Sym(\cE_c)$. 
Then, the algebra of classical local operators at~$0 \in \C^n$ of the factorization algebra~$\cF$ is quasi-isomorphic to 
\begin{align*}
\cF(0) & \simeq {\rm Sym} \left(\Omega^{n,hol}(\Hat{D}^n,V_0^*)^\vee [-n]\right) \\ & \cong \cO\left(\Omega^{n,hol}(\Hat{D}^n,V_0^*) [n] \right) 
\end{align*}
where the differential on the right hand side is~$Q^{hol}$.
\end{lem}

%\begin{lem}
%\label{lem:envelope}
%Suppose that $\cL$ is a topological-holomorphic local Lie algebra on $S \times X = \R^m \times \C^n$ and let $\cF$ be the factorization algebra $\clie_\bu(\cL_c)$.
%Moreover, assume that $Q^{hol}$ is an elliptic holomorphic differential operator. 
%Then, there is a spectral sequence converging to $H^\bu(\cF(0))$ whose first page is the $Q^{hol}$ cohomology of
%\[
%\cO \left(\d^n z L_0^*[[z_1,\ldots,z_n]] [n+m-1] \right) .
%\]
%\end{lem}
\begin{proof}
First, notice that as graded topological vector spaces one has an isomorphism for any open set $U \subset M$ 
\beqn\label{eqn:dist}
\left(\overline{\cE}^!(U)\right)^\vee \simeq \cE_c(U) 
\eeqn
%
%We use the spectral sequence induced by the filtration by the homogenous degree of a local operator.
%The first page is the cohomology of 
%\[
%\lim_{U \ni 0} \Sym(\cL_c(U)[1]) 
%\]
%with respect to the linear differential $\d_{dR} + \dbar + Q^{hol}$ which acts on $\cL_c(U)$ and extends to the symmetric algebra by the rule that it is a derivation.
This implies there is an isomorphism
\beqn\label{eqn:dist2}
\Sym(\cE_c(U)) \simeq \cO\left(\overline{\cE}^!(U)\right) 
\eeqn
for any open set $U$.
By assumption, the linear differential $[\cdot]_1$ is elliptic, in particular the embedding of smooth sections into distributional sections
\beqn\label{eqn:dist3}
\cE^!(U) \hookrightarrow \overline{\cE}^! (U)
\eeqn
is a quasi-isomorphism for any open set~$U$. 

We can assume that $U \subset \C^n$ is a Stein open set containing~$0 \in \C^n$.
%of the form $U' \times U'' \subset \R^m \times \C^n$ with $U' \subset \R^m$ contractible and $U''\subset \C^n$ Stein.
Then we have a sequence of quasi-isomorphisms
\begin{align*}
\overline{\cE}^! (U) & \simeq \cE^!(U) \\ & \simeq \Omega^{n,\bu}(U, V^*)[n].
\end{align*}
The result now follows from Lemma~\ref{lem:taylor}.

%Thus, the first page of this spectral sequence is isomorphic to the cohomology of the local operators $\Obs(0)$ of the free theory whose underlying cochain complex of fields is 
%\[
%\cE = \Omega^\bu(\R^m) \hotimes \Omega^{0,\bu}(\C^n , K_{\C^n} \otimes L^*[n+m-1]).
%\]
%In the notation of Equation \eqref{eqn:cE}, the holomorphic vector bundle $V$ is 
%\[
%K_{\C^n} \otimes L^*[n+m-1] .
%\]
%The result now follows from Lemma~\ref{lem:taylor} where we have used $\d^n z$ for basis for the line $K_{\C^n}|_0$. 
\end{proof}

\subsection{Local characters for topological-holomorphic theories}\label{s:localchar}

Suppose that $\cF$ is the factorization algebra of observables of a topological-holomorphic theory on $\R^m \times \C^n$. 
We will restrict our attention to cases where $\cF$, as a graded vector space, is of the form $\Sym(\cE^*)$ or $\Sym(\cE_c)$ where $\cE$ is of the form \eqref{eqn:cE}.

The local character $\chi_\cF ({\bf q})$ is, by definition, the graded character of algebra of local operators $\cF(0)$ with respect to some group of symmetries $H$, see \cite{SWchar}.
The particular group of symmetries depends on the theory, and we will present some examples momentarily. 

By assumption, as a graded algebra, the space of local operators $\cF(0)$ of a topological-holomorphic theory is of the form
\beqn
\cF(0) = \Sym (\lie{s})
\eeqn
where $\lie{s}$ is a graded topological vector space which we interpret as the linear local operators.

We will also assume that the group of symmetries $H$ acting on $\cF(0)$ arises from an action of $H$ on the linear local operators $\lie{s}$. 
Denote by $f_{\cF}({\bf q})$ the character of $\lie{s}$ with respect to this group action---this is the so-called `single particle' character. 
The full character of $\cF(0)$ is then given as the plethystic exponential of this single particle character
\beqn
\chi_{\cF}({\bf q}) = {\rm PExp}\left[f_{\cF}({\bf q}) \right] .
\eeqn

\subsection{Examples}

We present some simple examples. 

\begin{eg}
Suppose that $V$ is the trivial bundle on $\C^n$ and consider the theory whose fields are
\[
\cE = \Omega^\bu(\R^m) \otimes \Omega^{0,\bu}(\C^n) 
\]
where the differential is just $\d_{dR} + \dbar$. 
Then, the space of local operators is the symmetric algebra on the topological vector space which is linear dual to 
\[
\cO^{hol}(\Hat{D}^n) = \C[[z_1,\ldots,z_n]] .
\]
Via the $n$-dimensional residue one can identify the algebra of local operators with 
\[
\Sym\left(\frac{\d z_1}{z_1} \cdots \frac{\d z_n}{z_n}  \C[z_1^{-1}, \ldots , z_n^{-1}]\right) ,
\]
where $\lie{s} \simeq \frac{\d z_1}{z_1} \cdots \frac{\d z_n}{z_n}  \C[z_1^{-1}, \ldots , z_n^{-1}]$ is (equivalent to) the space of linear local operators. 

Consider the standard torus action $\C^\times \times \cdots \times \C^\times$ on $\C^n$. 
We would like to observe that the character of local operators with respect to this symmetry would be given by the plethystic exponential of the single particle index (the character of the space of linear local operators) which is immediate to compute:
\[
\frac{1}{(1-q_1)\cdots (1-q_n)} .
\]
However, the plethystic exponential cannot be applied to such an expression since as a power series in $q_1,\ldots,q_n$ there is a nonzero constant term.
This is related to the fact that there is an infinite number of operators for which the fugacities satisfy $q_1=\ldots=q_n=1$, so counting local operators in this way is ill-defined. 
One can remedy this by introducing a single extra variable fugacity $y$ and modify the single particle index to 
\[
\frac{y}{(1-q_1)\cdots (1-q_n)} .
\]
The plethystic exponential of such an expression returns the local character
\[
\chi(q_1,\ldots,q_n,y) = \prod_{k_1,\ldots,k_n \geq 0} \frac{1}{1-y q_1^{k_1}\cdots q_n^{k_n}}
\]
which now makes sense as a power series in the variables $y,q_1,\ldots,q_n$.
\end{eg}

Its instructive to see how local operators differ between $!$-dual factorization algebras.
Let us first point out a simple example. 
\begin{eg}
Consider the sheaf of cochain complexes
\[
\cE = \Omega^{0,\bu}\left(\C, K_{\C}^{\otimes r}\right),
\]
where $r \in \Z$ and the differential is~$\dbar$. 
Then, we can consider both the factorization algebra $\Obs = \cO(\cE)$ and its $!$-dual $\Obs^! = \Sym(\cE_c)$. 

The $\infty$-jets at $0 \in \C$ of $\cE$ is quasi-isomorphic to $\Gamma(\Hat{D}^n, K^{\otimes r}) = \d z^{\otimes r} \C[[z]]$. 
Thus the algebra of local operators $\Obs(0)$ is quasi-isomorphic to 
\[
\Obs(0) \simeq \cO \left(\Gamma(\Hat{D}, K^{\otimes r})\right) .
\]
In particular, the character of local operators $\Obs(0)$ is the plethystic exponential of
\[
\frac{q^{r}}{1-q} 
\]
where $q$ represents the fugacity for the standard~$\C^\times$ action on~$\C$.
Notice that when $r = 0$ we run into a similar problem as in the previous example. 
It is therefore convenient to introduce an extra fugacity $y$ which enters the single particle character as
\[
\frac{y q^{r}}{1-q}  .
\]

%Then, we have the factorization algebra which assigns to $U \subset \C$ the complex
%\begin{align*}
%\cF(U) & = \clie_\bu(\cL_c(U)) \\ & = \Sym\left(\Omega^{0,\bu}_c\left(U, K_U^{\otimes r}\right) [1] \right)
%\end{align*}
%where the differential is $\dbar$. 

%Serre duality induces an isomorphism
%\begin{align*}
%\Omega^{0,\bu}_c(\C, K_\C^{\otimes r}) \cong \left( \Gamma^{hol} (\C , K_\C \otimes K_\C^{-r})\right)^* \\
%= \left( \Gamma^{hol} (\C , K_\C^{1-r})\right)^* .
%\end{align*}

On the other hand, by Lemma \ref{lem:envelope} we see that the local operators associated to the $!$-dual $\Obs^!(0)$ is identified with the vector space
\[
\cO\left(\Gamma(\Hat{D}, K^{1-r})[1]\right) .
\]
In particular, the character of local operators $\Obs^!(0)$ is the plethystic exponential of
\[
-\frac{q^{1-r}}{1-q} 
\]
where $q$ represents the fugacity for the standard $\C^\times$ action on $\C$.
This time, when $r=1$ there is a problem with defining the plethystic exponential. 
To get an expression that makes sense for all $r$ we can again introduce a variable $y$ which enters the single particle character as
\[
- \frac{y q^{1-r}}{1-q} .
\]
\end{eg}

\subsection{Local characters for twisted superconformal theories}\label{s:localchar}

Suppose that $\cF$ is the factorization algebra of observables of a topological-holomorphic theory on $\R^m \times \C^n$. 
The local character $\chi_\cF ({\bf q})$ is, by definition, the graded character of algebra of local operators $\cF(0)$ with respect to some group of symmetries \cite{SWchar}.
The particular group of symmetries depends on the theory.
In this section we focus on local characters of factorization algebras that arise as twists of six-dimensional $\cN=(2,0)$ supersymmetric theories.

The (complexified) superconformal algebra in dimension six is $\lie{osp}(8|4)$. 
The holomorphic twist of this superconformal algebra is $\lie{osp}(6|2)$. 
We will consider the symmetry by the bosonic subalgebra
\beqn\label{eqn:cartan3}
\lie{sl}(3) \times \lie{sl}(2) \times \lie{gl}(1) \subset \lie{osp}(6|2)  .
\eeqn
The corresponding generators of the Cartan, as in \S \ref{sec:states}, were denoted $h_1,h_2,h,$ and $\Delta$ and the respective fugacities $t_1,t_2,r,q$.

We have described how this subalgebra embeds as fields in the twist of eleven-dimensional supergravity in \S \ref{s:ads7}. 
In particular, the holomorphic twist of any six-dimensional superconformal theory will have as a symmetry the subalgebra \eqref{eqn:cartan3}.
If the corresponding factorization algebra is $\cF$, and the local operators $\cF(0)$, the local character is then defined by the formal expression
\beqn
\chi_{\cF}(t_1,t_2,r,q) = {\rm Tr}_{\cF(0)} \left((-1)^F t_1^{h_1} t_2^{h_2} r^h q^\Delta\right) .
\eeqn
In the next section we will compute these characters in the case that the factorization algebra $\cF$ is $\clie_\bu(\cG_{N,c})$ where $N = 1,2,\ldots$.

We pointed out in \S \ref{sec:states} an alternative parametrization of the fugacities in terms of the parameters $y_1,y_2,y_3,y,q$ which satisfy the constraint $y_1 y_2 y_3 = 1$.
These parameters are related by $y_1=t_1^{-1}, y_2 = t_1 t_2^{-1}, y_3 = t_2$ and $y = q^{1/2} r$. 
We will also consider formulas for the local character $\chi(y_i,y,q)$ in terms of these variables.
  
%Note that for local operators which are the symmetric algebra of some graded vector space we can compute, as usual, the character as the plethystic exponential of a the single particle local character. 
%That is, the character of linear local operators.

\subsection{A relationship to the superconformal index}
\label{sec:sucaindex}
The local character for the holomorphic twist of a six-dimensional $\cN=(2,0)$ supersymmetric theory agrees with the well-known superconformal index.
Generally, in any dimension, the superconformal index counts states $\cH^Q$ which are annihilated by a particular supercharge~$Q$.
The index is defined as a function on the Cartan of a commuting subalgebra with respect to $Q$.
For six-dimensional superconformal theories, a natural choice of a supercharge is the holomorphic twisting supercharge. 
Then the index is sensitive to the so-called $\tfrac{1}{16}$-BPS states.

Recall that the odd part of the $\cN=(2,0)$ supersymmetry algebra is $S_+ \otimes R$ where $S_+$ is the positive irreducible spin representation of $\lie{so}(6)$ and $R \cong \C^4$.
Square-zero supercharges $Q \in S_+ \otimes R$ are stratified by the rank of the corresponding map $R \to (S_+)^* \cong S_-$.
A holomorphic supercharge $Q$ has rank one (such elements automatically square to zero). 
Thus, the superconformal index counts precisely the states in the holomorphic twist.
In the terminology above these states comprise the algebra of local operators $\cH^Q = \Obs(0)$ in the holomorphic twist of the six-dimensional $\cN=(2,0)$ theory.

The six-dimensional superconformal algebra (before twisting) is~$\lie{osp}(8|4)$.
The Cartan of the Lie super algebra is six-dimensional generated by elements
\[
H, J_1,J_2,J_3,R_1,R_2 .
\]
The holomorphic twisting supercharge $Q \in \lie{osp}(8|4)$ and the (super) commuting subalgebra is $\lie{osp}(6|2)$ together with the element 
\[
\Delta \define [Q,Q^\dagger] = H - (J_1 + J_2 + J_3) - 2 (R_1 + R_2) 
\]
where $Q^\dagger$ denotes the superconformal partner to the supercharge $Q$. 
The superconformal index counts states which saturate the BPS bound $\Delta \geq 0$ as a representation for the subalgebra $\lie{osp}(6|2)$. 
To fit with the notation used in this paper, the superconformal index can be written as
\beqn
\cI(y_i,y,q) = \Tr_{\cH^Q} (-1)^F q^{H + \frac13 (J_1+J_2+J_3)} y_1^{J_1} y_2^{J_2} y_3^{J_3} y^{R_1 - R_2} .
\eeqn
This agrees precisely with the local character $\chi(y_i,y,q)$ with the evident change of coordinates for the Cartan of $\lie{osp}(6|2)$. 

\subsection{Exceptional symmetry and a finite $N$ conjecture}

Generally speaking, after twisting there are enhancements of symmetries which are present in the original theory. 
We expect that the same occurs for any six-dimensional superconformal theory. 
In \cite{SW6d} we have shown that at the level of the holomorphic twist the twisted superconformal algebra $\lie{osp}(6|2)$ gets enhanced to the infinite-dimensional exceptional super Lie algebra $E(3|6)$ \cite{KacClass}. 
For the case of the theory on a stack of $N$ fivebranes, whose factorization algebra we denote by $\Obs_N$, this implies that the local operators $\Obs_{N}(0)$ form a representation for $E(3|6)$.

Our goal is to gain knowledge of the structure of $\Obs_N(0)$ as an $E(3|6)$-representation from our holographic analysis of the previous section.
Indeed, in \S \ref{s:fact} we have expressed the restriction of the factorization algebra of observables of twisted eleven-dimensional supergravity to the three-fold $Z$ as the Chevalley--Eilenberg cochains of a local $L_\infty$ algebra $\cG$. 
Recall that we have a decomposition of local Lie algebras $\cG = \oplus_{j \geq -1} \cG^{(j)}$ on the three-fold $Z$. 
From this decomposition we have defined a family of local Lie algebras $\cG_{N}$ on $Z = \C^3$ for $N=1,2,\ldots$.

In \S \ref{sec:factsummary} we explained the expectation that to an open set $U \subset \C^3$, the Lie algebra cohomology of $\cG_{N,c}(U)$ is equivalent to the observables $\Obs_N(U)$ of the six-dimensional theory supported on $U$. 
Each $\cG_N$ is acted on by the local Lie algebra $\cG^{(0)} = \cE(3|6)$.
The $\infty$-jets of $\cE(3|6)$ at $0 \in \C^3$ is exactly the exceptional super Lie algebra $E(3|6)$.  
Thus, for every $N$, the space of local operators of the factorization algebra $\clie_\bu(\cG_{N,c})$ is naturally and $E(3|6)$-representation. 
At the level of local operators we can make the following conjecture, which we will further elucidate at the level of characters for $E(3|6)$ in the next section.

\begin{conj}
\label{conj:ops}
Let $\Obs_N(0)$ be the local operators of the theory on a stack of $N$ fivebranes wrapping $\C^3$ in $\R \times \C^5$. 
There is an equivalence of $E(3|6)$-representations
\beqn
\Obs_N(0) \simeq \clie_\bu (\cG_{N,c}) (0) .
\eeqn
Similarly, let $\til \Obs_{N}(0)$ be the local operators of the theory on a stack of $N \geq 2$ fivebranes with the center of mass degrees of freedom removed. 
There is an equivalence of $E(3|6)$-representations
\beqn
\til \Obs_N(0) \simeq \clie_\bu (\til\cG_{N,c}) (0) .
\eeqn
\end{conj}


%The Witten index is protected under twisting---in our setup the index $\cI(t_1,t_2,r,q)$ can be computed in the minimal twist of the superconformal theory we start with.
%The minimal twist of the Hilbert space $\cH^Q$ is exactly the space of holomorphic local operators at $0$ in $\C^3$, for details see~\cite{SWchar}. 
%Thus, the index $\cI(t_1,t_2,r,q)$ agrees with the holomorphic character $\chi(t_1,t_2,r,q)$ defined above.

%The superconformal index of a superconformal field theory is the Witten index of the theory in the radial quantization.
%In our situation we look at the Hilbert space $\cH$ of the theory on $S^{5}$ and consider the index heuristically of the form
%\beqn
%\Tr_{\cH} (-1)^F  x_1^{G_1} \cdots x_n^{G_n} q^{\Delta}
%\eeqn
%where $\{G_i\}$ are a collection of charges that commute with the holomorphic supercharge $Q$ and its superconformal adjoint $S = Q^\dagger$. 
%Here, $\Delta = [Q,S]$.
%We choose three elements $G_1,G_2,G_3$ in such a way that they become the elements $h_1,h_2,h$ upon taking $Q$-cohomology (and so automatically commute with $Q$ and $S$). 
%Thus we consider the following index
%\beqn
%\cI (t_1,t_2,r,q) \define \Tr_{\cH} (-1)^F t_1^{h_1} t_2^{h_2} r^h q^\Delta .
%\eeqn
%After tracing over $\cH$ one can identify the superconformal index with the partition function of the model on a space which is topologically equivalent to~$S^{5} \times S^1$.



%\[
%(\C^3 - 0) / \sim  \; \simeq \; S^5 \times S^1 .
%\]
%The perspective of the holomorphic twist allows us to holomorphic theory agrees with the partition function on the product of spheres is basically goes by the process of `radial quantization'. 
%Consider the restriction of the theory to $\C^3 - 0 \subset \C^3$ and its dimensional reduction to quantum mechanics along
%\[
%|-| \colon \C^3 - 0 \to \RR_{>0} .
%\]
%The fiber of this map over a point is $S^5$. 
%By the nature of holomorphic QFT, we can extract from the OPE in the radial direction a canonical associative ($A_\infty$) algebra $\cA_{\lie{u}(1)} = \int_{\C^3 - 0} \Obs$ which is roughly the value of the theory on $S^5$. 
%There is a canonical boundary condition of the quantum mechanics theory at radius $r = 0$ given by the local operators $\Obs(0)$ at $0 \in \C^3$ which, in turn, has the structure of a $\cA$-module. 
%By standard arguments placing this quantum mechanics theory on circle $S^1$ results in the trace of the $\cA$-module $\Obs(0)$ 
%\[
%Z(S^{5} \times S^1) = {\rm Tr}_{\cA} (\Obs(0)) .
%\]
%From the trace on the right-hand side we can recover the character as defined above. 
%Indeed, the $E(3|6)$-module structure on local operators factors through a map 
%$E(3|6) \to \cA$ since we wrote down the explicit Hamiltonians above in \eqref{eq:ham1}, for instance. 

%\subsection{Comparison to `states'}


%\subsection{Categorifying the index for free theories}
%
%In the case of both membranes and fivebranes we constructed a particular restriction of the local $L_\infty$ algebra $\cL_{sugra}$ to the respective worldvolume theories which we denoted by $\Bar{\pi}_* \cL_{sugra}$. 
%There are two important sub local $L_\infty$ algebras 
%\[
%\begin{tikzcd}
%& \Bar{\pi}_*\cL_{sugra} & \\
%\Bar{\pi}_*\cL_{sugra}^{(-1)} \ar[ur] & & \Bar{\pi}_*\cL_{sugra}^{(0)} \ar[ul] .
%\end{tikzcd}
%\]
%This diagram induces a diagram of factorization algebras
%\[
%\begin{tikzcd}
%& \left(\Obs_{sugra}|_Z\right)^! & \\
%\clie_\bu(\Bar{\pi}_*\cL_{sugra,c}^{(-1)}) \ar[ur] & & \clie_\bu(\Bar{\pi}_*\cL_{sugra,c}^{(0)}) \ar[ul].
%\end{tikzcd}
%\]

%\parsec[s:sugraops]
%
%By the usual methods of the BV formalism the action functional $S_{sugra}$ described above endows the parity shift of the fields $\cL_{sugra} = \Pi \cF_{sugra}$ with the structure of a holomorphic-topological local $\Z/2$ graded $L_\infty$ algebra. 
%
%On $\C^5 \times \R$ we can describe this super Lie algebra structure explicitly. 
%First, by the Dolbeault and de Rham Poincar\'e lemmas it is easy that the even part of the super Lie algebra $\cL(\C^5 \times \R)$ is equivalent to a one-dimensional central summand $\C$ plus the Lie algebra of divergence-free vector fields on $\C^5$:
%\[
%\Vect_0 (\C^5) = \{X \in \Vect(\C^5) \; | \; \div X = 0\} .
%\]
%The odd part of the super Lie algebra $\cL(\C^5 \times \R)$ is equivalent to the space of holomorphic one-forms on $\C^5$ modulo exact one-forms
%\[
%\Omega^{1,hol}(\C^5) / {\rm Im}(\del) 
%\]
%which is, of course, equivalent to the space of closed holomorphic two-forms $\Omega^{2,hol}_{cl}(\C^5)$. 
%
%\begin{thm}[\cite{RSW}[Theorem 2.1]]
%The Taylor expansion map determines a map of $\Z/2$ graded $L_\infty$ algebras
%\[
%j_\infty \colon \cL_{sugra}(\C^5 \times \R) \to L_{sugra} .
%\]
%Furthermore, $L_{sugra}$ is equivalent as a $\Z/2$ graded $L_\infty$ algebra to $\Hat{E(5|10)}$. 
%\end{thm} 
%
%As an immediate corollary of this result we obtain by Lemma \ref{lem:localops} the following.
%
%\begin{cor}
%\label{cor:sugraops}
%Let $\Obs_{sugra}$ be the factorization algebra on $\C^5 \times \R$ of classical observables of the minimal twist of eleven-dimensional supergravity.
%There is a quasi-isomorphism of commutative dg algebras
%\[
%\Obs_{sugra} (0) \simeq \clie^\bu \left( \Hat{E(5|10)} \right) .
%\]
%\end{cor}

%\end{document}


%\section{Conjectures for indices of operators on fivebranes}
\label{s:finite}

In conjecture \ref{conj:ops} we have formulated a conjectural description of the space of local operators $\Obs_{N}(0)$ associated to the worldvolume theory on a stack of $N$ fivebranes in the holomorphic twist.
As we reviewed just in the previous section, the space of local operators is what categorifies the superconformal index that we study in this paper.
In this section we begin to provide some evidence for this description at the level of characters.
%Of course, this is just a small piece of the full algebra structure present in the local operators.
%The structure of a three-dimensional holomorphic factorization algebra induces algebraic operations on the algebra of local operators


For a stack of $N=1$ fivebranes, which corresponds to the abelian six-dimensional superconformal field theory, we find that our local character matches exactly with the expressions in the literature. 
This is not a surprise as we have shown that even at the level of factorization algebras $\clie_\bullet(\mc{G}_{1,c})$ is quasi-isomorphic to the classical limit of~$\Obs_1$, see Proposition~\ref{prop:factabelian}.

The main computation of this section is a closed formula for the local character of the factorization algebra $\clie_\bullet(\mc{G}_{N,c})$ for $N > 1$, see Theorem \ref{thm:finite}. 
Following conjecture \ref{conj:ops} and the general discussion of \S \ref{sec:sucaindex} we are led to hypothesize a closed formula for the superconformal index for the theory on a finite number of fivebranes (in flat space).
As far as the authors are aware of there is no closed formula for the refined superconformal index (with four independent fugacities) for the theory on a stack of $N > 1$ fivebranes.
For small values of $N$ we expand our closed formulas to low orders in the fugacity $q$ (which roughly counts instanton charge) to match exactly with expressions in the literature. 

\subsection{Operators on a single fivebrane}


In Proposition \ref{lem:single} we have shown that $\clie_\bullet (\mc{G}_{c}^{(-1)})$ is equivalent to the factorization algebra $\Obs^{cl}_{1}$ encoding the classical observables  of the holomorphic twist on a single fivebrane.
On $\C^3$, the global sections of the local Lie algebra $\mc{G}^{(0)}$ is closely related to $E(3|6)$---the $\infty$-jets of $\mc{G}^{(0)}$ at $0 \in \C^3$ is quasi-isomorphic to $E(3|6)$.
Combining these facts we see that $\Obs_{1}(0)$ is a module for $E(3|6)$. 
This module turns out to be a one-dimensional extension of an irreducible $E(3|6)$-module which was classified in~\cite{KR2}. 

%and compare our expression for the character to the character of a certain irreducible module for the exceptional super Lie algebra $E(3|6)$ considered in~\cite{KR2}.

%\parsec
%
%There are various degenerations, or specializations, of this character which are interesting to consider.
%These specializations involve restricting the character above to a subalgebra of the full Cartan that we considered above.
%
%One degeneration of this character involves specializing $t_1=t_2=1$ which results in the $U(1) \times SU(2)$ character:
%\begin{equation}
%f_{1}(r,q) = \frac{(r+r^{-1})q^{3/2} - 3 q^{2} + q^3}{(1-q)^3} .
%\end{equation}
%They compute the absolute (non-super) character of the module $I(0,0;1;-1)$ where they additionally specialize $t_1=t_2=r=1$. 
%In a similar method to the one used in \cite{KR1}, one can compute the specialized (super) character of $I(0,0;1;-1)$ to find
%\[
%\chi_{u(1)} (q,t_1=t_2=r=1) = \frac{2 q^{3/2} - 3 q^2 + q^3}{(1-q)^3} .
%\]

\parsec
There are various degenerations, or specializations, of this character which are interesting to consider.
A particularly meaningful one is related to two different deformations of the theory by elements in the (twisted) superconformal algebra and is known as the Schur limit of the index.

Recall that after performing the holomorphic twist the residual superconformal algebra is~$\mf{osp}(6|2)$.
We have recalled in \S\ref{s:global1} how the bosonic part of this algebra is represented by fields of the eleven-dimensional theory. 
There are two types of odd elements of~$\mf{osp}(6|2)$ that also have a natural interpretation in the eleven-dimensional theory.
The odd part of~$\mf{osp}(6|2)$ can be identified with the twelve-dimensional space
\[
\C^3 \otimes \C^2 \oplus \wedge^2(\C^3) \otimes \C^2 
\]
where $\C^3, \C^2$ are the fundamental $\mf{sl}(3)$ and $\mf{sl}(2)$ representations, respectively. 
The $\mf{gl}(1)$ factor in the bosonic part of $\mf{osp}(6|2)$ acts with weight $1/2$ on both summands. 

\begin{itemize}
\item The summand $\C^3 \otimes \C^2$ embeds into the ghosts of twisted supergravity via the $\gamma$-type fields which satisfy
\[
\del \gamma = \d w_a \d z_i .
\]
where $i=1,2,3$ and $a = 1,2$.
Note that $\gamma$ appears to be ambiguous up to a closed holomorphic one-form, but since there is a linear gauge symmetry which sends $\beta \mapsto \del \beta$, it implies that $\gamma$ is unique up to a BRST exact term. 
Since in our model all closed one-forms are rendered trivial in cohomology
\item The summand $\wedge^2(\C^3) \otimes \C^2$ embeds as another $\gamma$-type field which satisfies 
\[
\del \gamma = w_a \d z_i \d z_j .
\]
\end{itemize}

Both deformations break the global Cartan subalgebra down to $\mf{gl}(1) \times \mf{gl}(1)$ according to the specializations
\begin{equation}\label{eqn:special1}
y=1 , \quad y_3 = 1 .
\end{equation}
Notice that due to the constraint $y_1y_2y_3=1$ this forces $y_1 = y_2^{-1}$.
As one can easily check, this specialization yields the following single particle index
\[
f_{1}(y_1, y_1^{-1},y_3=1, y=1, q) = \frac{q}{1-q} 
\]
which recovers the single particle index of a single chiral boson on the Riemann surface $\Sigma = \C_{z_1}$. 
Notice that the dependence on the parameter $y_1$ has completely dropped out even though we have not specialized it to any value.
%Notice that although the Cartan subalgebra generated by the vector field $z_1 \del_{z_1} - z_2 \del_{z_2}$ is unbroken by this deformation, the dependence on its fugacity $t_1$ completely drops out of the expression.

\subsection{A conjectural description of operators on a stack of two fivebranes}

In \S\ref{sec:factsummary} we saw that the decomposition of the local $L_\infty$ algebra $\mc{G} = \mc{G}_Z$ on $Z$ induces a filtration of the factorization algebra $\clie_\bullet(\mc{G}_c)$. 
\[
\clie_{\bullet}(\mc{G}_{1,c}) \subset \clie_{\bullet}(\mc{G}_{2,c}) \subset \cdots .
\]
We now turn to the factorization algebra $\clie_{\bullet}(\mc{G}_{2,c})$.

Recall that $\mc{G}_{2}$ is the local $L_\infty$ algebra on $Z$ defined as $\mc{G}_{2} = \mc{G} / \mc{G}^{\geq 1}$. 
Since $\mc{G}$ is concentrated in weights $\geq -1$ we see that $\tilde {\mc{G}}_{2}$ is of the form
\[
\mc{G}_2 = \tilde {\mc{G}}_2 \ltimes \mc{G}_1 
\]
where $\mc{G}_1 = \mc{G}^{(-1)}$ is the weight $(-1)$ piece and $\tilde {\mc{G}}_2 = \mc{G}^{\geq 0} / \mc{G}^{\geq 1} = \mc{G}^{(0)}$.  
We focus mostly on the factorization algebra $\clie_\bullet(\tilde {\mc{G}}_{2,c})$.

We have already characterized the local dg Lie algebra $\tilde {\mc{G}}_{2} = \mc{G}^{(0)}$ as the weight zero part of $\mc{G}$ on on any threefold $Z$ in \S\ref{s:weight0}. 
We have also shown that $\mc{G}^{(0)}$ is equivalent to the local Lie algebra $\mc{E}(3|6)$. 
The even part of $\mc{E}(3|6)$ is
\[
\Omega^{0,\bullet}(Z, \T_Z) \oplus \Omega^{0,\bullet}(Z) \otimes \mf{sl}(2) 
\]
with its natural cohomological grading by Dolbeault form type. 
The odd part of $\mc{E}(3|6)$ is
\[
\Omega^{1,\bullet}(Z, K_Z^{-1/2}) \otimes \C^2 .
\]
The differential is $\dbar$ and the Lie bracket has been described in \S\ref{s:weight0}.

%On $Z = \C^3$ this local dg Lie algebra is related to the exceptional simple super Lie algebra $E(3|6)$ classified by Kac \cite{KacClass}. 
%Indeed, one can show (see the forthcoming work \cite{SW6d}) that the fiber of the $\infty$-jet bundle of $\mc{G}_2$ at $0 \in \C^3$ is quasi-isomorphic to $E(3|6)$. 

\parsec

We continue by computing the character of local operators associated to the factorization algebra $\clie_\bullet(\mc{G}_{2,c})$ using Lemma~\ref{lem:envelope}.
For simplicity we will use the fugacities $y_i, y, q$.



Combining these expressions we obtain the following.

\begin{prop} \label{prop:6dtwo}
The character of local operators of the factorization algebra $\clie_\bullet(\tilde {\mc{G}}_{2,c})$ on $\C^3$ is given by the plethystic exponential of the following expression
\begin{equation}\label{eqn:6dtwo}
\tilde f_{2} (y_i,y,q) = \frac{q^4(y_1+y_2+y_3) + q^2 (y^2 + q + q^2 y^{-2}) - q^{3} (y + q y^{-1})(y_1^{-1} + y_2^{-1} + y_3^{-1})}{(1-y_1q) (1-y_2 q) (1-y_3 q)}.
\end{equation}
%\begin{equation}\label{eqn:6dtwo}
%f_{2} (t_1,t_2,r,q) = \frac{q^4(t_1^{-1} + t_1 t_2^{-1}  + t_2) + q^3 (r^2 + r^{-2} + 1) - q^{7/2} (r + r^{-1})(t_1 + t_1^{-1} t_2 + t_2^{-1})}{(1-t_1^{-1}q) (1-t_1 t_2^{-1} q) (1-t_2 q)} .
%\end{equation}
\end{prop}

Recall that our conjecture for the space of local operators associated to the holomorphic twist of the six-dimensional worldvolume theory on a stack of two fivebranes is $\Obs_2 (0) \simeq \clie_\bullet(\mc{G}_{2,c})(0) \mc{O}ng \clie_\bullet(\mc{G}_{1,c})(0) \otimes \clie_\bullet(\tilde {\mc{G}}_{2,c})(0)$. 
And after removing the center of mass degrees of freedom, our conjecture is $\tilde \Obs_2 \simeq \clie_\bullet(\tilde {\mc{G}}_{2,c})$.

Just as in the abelian case, the local operators $\tilde \Obs_2(0)$ form a module over $E(3|6)$.
It turns out that this module is irreducible~\cite{KR2}.

We can now state a decategorified version of conjecture \ref{conj:ops} at the level of superconformal indices, or local characters.

\begin{conj}\label{conj:6dtwo}
The superconformal index of the six-dimensional superconformal theory of type $A_1$ is given by
\[
\tilde \chi_{2} (y_i,y,q) = {\rm PExp} \left[\tilde f_2(y_i,y,q) \right] .
\]
where $\tilde f_2(y_i,y,q)$ is as in \eqref{eqn:6dtwo}.
\end{conj}

Similarly, the index associated to the $\mf{gl}(2)$ theory, which is the local character of $\clie_\bullet(\mc{G}_{2,c})$, is conjectured to be simply the product 
\[
\chi_{2} (y_i,y,q) = \chi_{2} (y_i,y,q) \cdot \chi_{1}(y_i,y,q)
\]
where the character $\chi_{1}$ for the $\mf{gl}(1)$ theory is given in proposition~\ref{prop:6done}
Equivalently, $\chi_2$ is the plethystic exponential of $f_2 = f_1 + \tilde f_2$. 

%\parsec[]
%
%The specialization of this index $t_1=t_2=r=1$ yields the single particle index
%\[
%\frac{3q^4 + 3 q^3 - 6 q^{7/2}}{(1-q)^3}. 
%\]

\parsec[]

The Schur limit $y=1, y_3=1$ of $\tilde f_2$ in \eqref{eqn:special1} yields 
\[
\tilde f_{2}(y_1, y_2, y_3=1, y=1, q) = \frac{q^2}{1-q} 
\]
which is the single particle index of Virasoro vacuum module on the Riemann surface $\Sigma = \C_{z_1}$. 

\subsection{A closed formula for the finite $N$ index}

Before exhibiting the general formula for the local character of the factorization algebra $\clie_\bullet(\mc{G}_{N,c})$ on $\C^3$ we set up some notation. 
As above, we let $\chi_k^{\mf{sl}(2)}$ and $\chi^{\mf{sl}(3)}_{[k,l]}$ denote the highest weight $\mf{sl}(2)$ and $\mf{sl}(3)$ characters. 
We also define the following expression which appears in the denominator in all of our characters
\begin{equation}
d(y_i,y,q) = (1-y_1 q)(1-y_2q)(1-y_3q) .
\end{equation} 
To simplify formulas, we will temporarily denote the single particle character for the $N=1$ theory $\Obs_1$ by 
\begin{equation}
g_{-1} (y_i,y,q) = f_1(y_i,y,q)
\end{equation}
where $f_1(y_i,y,q)$ is as in equation \eqref{eqn:6done1} and also denote by 
\begin{equation}
g_0 (y_i,y,q) = \tilde f_2(y_i,y,q)
\end{equation}
where $\tilde f_2(y_i,y,q)$ is as in equation \eqref{eqn:6dtwo}. 
Thus $g_2$ is the single particle local character of $\clie_\bullet(\tilde {\mc{G}}_{2,c}) = \clie_{\bullet}(\mc{G}_c^{(0)})$.
Finally, for $k \geq 1$ let
%\begin{align*}
%f_k (y_1,y_2,y_3,y,q) & \define q^{3k/2} \left(q \chi^{\mf{sl}(2)}_{k-2}(q^{-1/2} y)(y_1 + y_2 + y_3) + \chi^{\mf{sl}(2)}_k(q^{-1/2} y) \right. \\
%& \left.  - q \chi^{\mf{sl}(2)}_{k-3}(q^{-1/2}y) - \chi^{\mf{sl}(2)}_{k-1} (q^{-1/2} y) (y_1^{-1} + y_2^{-1} + y_3^{-1} ) \right) .
%\end{align*}
%\begin{align*}
%g_k (y_i,y,q) & \define q^{3} \left(q^{1 + 3 (k-2)/2} \chi^{\mf{sl}(2)}_{k-2}(q^{-1/2} y)(y_1 + y_2 + y_3) + q^{3(k-2)/2} \chi^{\mf{sl}(2)}_k(q^{-1/2} y) \right. \\
%& \frac{\left.  - q^{3(k-1)/2} \chi^{\mf{sl}(2)}_{k-3}(q^{-1/2}y) - q^{-1 + 3(k-1)/2} \chi^{\mf{sl}(2)}_{k-1} (q^{-1/2} y) (y_1^{-1} + y_2^{-1} + y_3^{-1} ) \right)}{d(y_i,y,q)} .
%\end{align*}
\begin{equation}
\label{eqn:gk}
\begin{array}{lllll}
g_k (y_i,y,q) \define & q^{3} \left(q^{1 + 3 k/2} \chi^{\mf{sl}(2)}_{k}(q^{-1/2} y)\chi^{\mf{sl}(3)}_{[1,0]}(y_i) + q^{3k/2} \chi^{\mf{sl}(2)}_{k+2}(q^{-1/2} y) \right. \\
&\displaystyle \frac{\left.  - q^{3(k+1)/2} \chi^{\mf{sl}(2)}_{k-1}(q^{-1/2}y) - q^{-1 + 3(k+1)/2} \chi^{\mf{sl}(2)}_{k+1} (q^{-1/2} y) \chi^{\mf{sl}(3)}_{[0,1]}(y_i) \right)}{d(y_i,y,q)} .
\end{array}
\end{equation}
%and hence the conjectural single particle index for the superconformal theory associated to the Lie algebra $\mf{sl}(2)$. 

\begin{thm}
\label{thm:finite}
Let $N \geq 3$. 
The local character of the factorization algebra $\clie_{\bullet}(\mc{G}_{N,c})$ is
\begin{equation}
\chi_{N}(y_1,y_2,y_3,y,q) = \text{PExp}\left[\sum_{k=-1}^{N-2} g_k(y_1,y_2,y_3,y,q)\right].
\end{equation}
Similarly, the local character of the factorization algebra $\clie_\bullet(\tilde{\mc{G}}_{N,c})$ is 
\begin{equation}
\tilde{\chi}_{N}(y_1,y_2,y_3,y,q) = \text{PExp}\left[\sum_{k=0}^{N-2} g_k(y_1,y_2,y_3,y,q)\right].
\end{equation}
\end{thm}
\begin{proof}
By Lemma~\ref{lem:envelope} the character of $\clie_\bullet (\mc{G}_{N,c})$ is given by 
\begin{equation}
\chi_N = \text{PExp} \left[f_N\right]
\end{equation}
where $f_N$ is the single particle local character.
Thus, it suffices to show that $f_N = \sum_{k = -1}^{N-2} g_k$.
Recall that from the description \eqref{eqn:gN} we have, as local Lie algebras:
\begin{equation}
\mc{G}_N = \mc{G} / \mc{G}^{(\geq N-2)} ,
\end{equation} 
for $N \geq 1$. 
In particular, as a super vector bundle on the threefold $Z = \C^3$ we have
\[
\mc{G}_N = \mc{G}^{(-1)} \oplus \mc{G}^{(0)} \oplus \cdots \oplus \mc{G}^{(N-2)} .
\]
%\begin{equation}
%\clie_{\bullet}(\mc{G}_{2,c}) \leftarrow \clie_{\bullet}(\mc{G}_{3,c}) \leftarrow \cdots \leftarrow \clie_\bullet (\mc{G}_{N,c}) .
%\end{equation}
So, it suffices to observe that $g_k$ is the single particle index of the factorization algebra $\clie_\bullet(\mc{G}^{(k)}_c)$, which is a direct observation using the description of $\mc{G}^{(k)}$ we have given in Proposition \ref{prop:Vj}.
\end{proof}

We thus arrive at the following conjecture for the index of the worldvolume theory on a stack of a finite number of fivebranes which we phrase in terms of the six-dimensional superconformal theory associated to the Lie algebra $\mf{sl}(N)$.

\begin{conj} 
The superconformal index of the six-dimensional superconformal theory associated to the Lie algebra of type $A_{N-1}$ is $\tilde \chi_{N}(y_1,y_2,y_3,y,q)$. 
\end{conj}

We proceed to give some concrete evidence for this conjecture.
First, we show that when we take the limit as $N \to \infty$ that we recover the index computed from the gravitational side.

\parsec

It follows from the limit description in \eqref{eqn:lim} that the large $N$ limit of $\chi_N$ is precisely the multiparticle supergravity index we computed in proposition~\ref{prop:sugraindex1}. 
Alternatively, we have the following direct proof of this fact. 

\begin{prop}
One has
\begin{equation}
\chi_{sugra}(y_i, y, q) = \lim_{N \to \infty} \chi_N(y_i,y,q)
\end{equation}
\end{prop}

\begin{proof}
It suffices to show that at the level of single particle indices one has
\begin{equation}
f_{sugra}(y_i, y, q) = \lim_{N \to \infty} f_N(y_i,y,q) ,
\end{equation}
where $f_N = \sum_{k = -1}^{N-2} g_k$. 

We will use the following identity 
\begin{equation}
\sum_{k=0}^\infty q^{3k/2} \chi_{k}^{\mf{sl}(2)}(q^{-1/2}y) = \frac{1}{(1-q y)(1-q^2 y^{-1})} .
\end{equation}
We will denote this expression by $S(y,q)$.

Using this identity one can directly see that the result reduces to observing that
\begin{multline}
\left(q^4 (y_1+y_2+y_3) + 1 - q^6 - q^2 (y_1^{-1} + y_2^{-1} + y_3^{-1})\right)S(y,q) - 1 + q^3= \\
\left(q^4(y_1+y_2+y_3)-q^2(y_1^{-1} + y_2^{-1} + y_3^{-1})+(1-q^3)(yq + y^{-1} q^2) \right) S(y,q) .
\end{multline}


%\begin{itemize}
%\item $q^4 \sum_{k=0}^\infty q^{3k/2} \chi_{k}^{\mf{sl}(2)}(q^{-1/2} y)(y_1+y_2+y_3) = \frac{q^4(y_1+y_2+y_3)}{(1-q y)(1-q^2 y^{-1})}$. 
%\end{itemize}

\end{proof}

As an immediate corollary we have the following result.
\begin{cor}
For any $N \geq 1$ one has
\begin{equation}
\chi_{sugra}(y_i,y,q) = \tilde{\chi}_N(y_i,y,q) \mod q^{N+1} .
\end{equation}
\end{cor}
\begin{proof}
This follows from observing that at the level of single particle states $f_N$ is of order $q^{N}$.
\end{proof}

\section{Holographic Speculations}\label{sec:holspec}
We began this thesis with some remarks on how dualities between physical theories can often be used to uncover novel equivalences between the mathematical objects that describe them. In this final section of the thesis, we offer some speculations to this effect.
We caution the reader that a large portion of this section involves recalling constructions from physics without any attention to rigor for motivational purposes.

In sections \ref{sec:BLG}, \ref{subsec:g-1}, we commented on how minimal twists of the 3d $\mc N=8$ BLG theory and 6d $\mc N=(2,0)$ tensor multiplets are are visible as pieces of the graviton decompositions of twisted $AdS_4\times S^7$ and $AdS_7\times S^4$ respectively. Famous instances of the AdS/CFT correspondence posit equivalences between the higher rank 3d $\mc N=8$ theories studied by ABJM and the higher rank 6d $\mc N=(2,0)$ theories of type $A_N$ with M-theory on $AdS_4\times S^7$ and $AdS_7\times S^4$ respectively. It is natural to wonder whether the twisted holograpahy proposal mentioned in the introduction can be applied to our descriptions of the twisted $AdS_4\times S^7$ and $AdS_7\times S^4$ backgrounds to study the minimal twists of the higher rank 3d $\mc N=8$ and 6d $\mc N=(2,0)$ superconformal field theories respectively.

Our goal in this section is to posit some expectations regarding the minimal twist of the 6d $\mc N=(2,0)$ theory of type $A_N$. This theory is notorious for being both ubiquitous and nebulous. On the one hand, almost every superconformal field theory that has had interesting applications to geometry, topology, or representation theory occurs as one of its dimensional reductions, so it has long been expected to contain very rich mathematics. On the other hand, it does not admit a Lagrangian description. Its only free parameter is the rank of an ADE Lie algebra, and outside of the abelian case, a field realization is not even known.

We begin by recalling some features of the AdS/CFT correspondence. We will begin with a more physical language, and work towards some concrete mathematical expectations. 

\subsection{The AdS/CFT correspondence}

Traditional formulations of the AdS/CFT correspondence relate two theories, schematically denoted $T_{CFT}$ and $T_{grav}$ on manifolds $M_{1}, M_{2}$ respectively, together with a conformal diffeomorphism $\del M_{2}\cong M_{1}$. The theories have the feature that boundary values of fields of $T_{grav}$ denoted $\phi |_{\del}$, may be identified with sources for $T_{CFT}$ denoted $J$. The two theories are considered to be holographically dual when their partition functions are equivalent $Z_{CFT}[J] = Z_{grav}[\phi |_{\del}]$. 

In examples of stringy origin, $T_{CFT}$ describes the low energy dynamics of a stack of $N$ branes in supergravity, in the large $N$ limit, and $T_{grav}$ describes gravitational dynamics in the background the branes source. 

\parsec[]
Let's identify some salient features of the primordial example of such a duality so as to inform our desiderata in the sequel.

\begin{conj}[Maldacena \cite{Maldacena:1997re}, \cite{WittenAdS}]
The following are equivalent:
\begin{itemize}
\item $\mc N=4$ super Yang-Mills theory with gauge group $SU(N)$. In addition to the rank of the gauge group, the theory has a parameter the Yang-Mills coupling constant $g_{YM}$.
\item type IIB superstring theory on $AdS_5\times S^5$ with $N$ units of five-form flux on $S^5$. The theory has two free parameters, the string coupling $g_s$ and a parameter $L/\ell_s$ which describes the scale of AdS relative to the length of the string. \end{itemize}

Under this equivalence, the parameters of the two sides are identified as follows $g_{YM}^2 = 2\pi g_s$ and $2g_{YM}^2N = (L/\ell_s)^4$. 
\end{conj}

It is convenient to introduce a parameter $\lambda = g^2_{YM} N$, the so-called \textit{'t Hooft coupling}; in the perturbative regime where the number of colors is also large (a limit that we will introduce momentarily), the $\beta$-function keeps $\lambda$ of the same order.

It is very difficult to perform explicit calculations of most observables associated to either theory at generic values of the parameters on either side. However, there are certain limits which afford more tractability. 

\begin{itemize}
\item The first limit we can take involves sending the string coupling $g_s$ to zero and keeping the parameter $L/\ell_s$ fixed. In this limit, contributions from higher genus worldsheets in string perturbation theory are suppressed. Under the above identification of parameters, we see that this limit should involve taking $g_{YM}\to 0$ while keeping the 't Hooft coupling finite; that is, we must take the large $N$ limit of the gauge theory. This limit is traditionally referred to as the \textit{'t Hooft limit}. Corrections in $\frac{1}{N}$ then correspond to turning on quantum effects in string theory.

\item After taking the 't Hooft limit, we may further consider the limit where $L/\ell_s$ is large. In this limit, strings are small and particle like compared to the scale of AdS and the theory looks like classical type IIB supergravity on $AdS_5\times S^5$. On the gauge theory side, this corresponds to the limit where the 't Hooft coupling is large. As such, we see that even this simplified form of the AdS/CFT correspondence is extremely powerful as it relates strongly coupled gauge theory to classical perturbative supergravity!
\end{itemize}

\subsection{BPS observables in AdS/CFT}\label{bpsadscft}

Many checks of the AdS/CFT correspondence involve computing quantities on either side that are independent of the coupling and comparing them. Such quantities are typically BPS, and as such can be studied at the level of twists. We introduce two such quantities which we will further expand on in our relevant example below. 

\parsec[]
Suppose that $T_{CFT}$ is superconformal, such as in the above example. In such examples, one expects that the superconformal algebra in fact acts on $M_2$ as isometries, at least asymptotically.

Superconformal field theories admit a plethora of protected quantities that can be computed exactly at weak coupling. One such quantity is the superconformal index, which in a Hamiltonian formulation of the theory can be thought of as a Witten-index in radial quantization. Schematically, such a quantity takes the form \[\operatorname{Tr} _{\mc H} \left ( (-1)^F \exp (-\beta \{ Q, \overline Q\} ) x_1^{J_1}\cdots x_n^{J_n}y_1^{H_1}\cdots y_n^{H_n} \right )\] where $(-1)^F$ is the fermion number operator, $\beta$ is an inverse temperature, $Q$ is a supercharge and the $x_i$ are fugacities keeping track of charges under angular momenta, and $y_i$ are fugacities keeping track of charges under R-symmetries. The superconformal index gives a generating function for the difference between bosonic and fermionic states annihilated by a particular supercharge. Under an operator-state correspondence, the superconformal index can also be thought of as a signed count of local operators preserved under a single supercharge. 

In terms of partition functions, the superconformal index is gotten by a partition function on a twisted product $M_{1} = S^{1}\times_\omega S^{d-1}$ where the twisting $\omega$ is determined by a background connection for the global symmetries of the problem. The expectation that the AdS/CFT correspondence can be expressed as an equality of partition functions therefore suggests a recapitulation of the superconformal index in gravitational terms. An exciting body of work aims to make this gravitiational incarnation precise, see for example \cite{murthy2020growth} and references therein.  

Note that by definition, the superconformal index provides a lower bound on the number of fractionally BPS states of $T_{CFT}$. It is often the case, however, that $T_{grav}$ includes in its spectrum, black holes, which are expected to have a thermodynamic entropy proportional to the event-horizon-area at leading order, as given by the Beckenstein-Hawking formula. As such, the growth of states in $T_{CFT}$, and hopefully the superconformal index, should reflect this. 

\parsec[]
Another such quantity is the algebra of BPS local operators in $T_{CFT}$. This vector space underlying this algebra is precisely a costalk of the factorization algebra of observables of a twist of $T_{CFT}$. In light of the aforementioned operator-state correspondence, this can be thought of as categorifying the superconformal index. Under the AdS/CFT dictionary, local operators of $T_{CFT}$ are supposed to match with certain kinds of states in $T_{grav}$. 

Moreover, both kinds of objects transform in representations of a superconformal algebra and the map between them preserves the actions. Local operators in the CFT are equipped with an interesting algebraic structure given by operator-product-expansion, and the AdS/CFT correspondence intertwines this algebraic structure with scattering of supergravity states. Indeed, the equality of partition functions along with the matching of sources for CFT local operators with boundary values of gravitational fields gives a prescription for computing correlation functions between CFT local operators by varying the gravitational action evaluated on field configurations subject to certain boundary values with respect to the boundary value. This recipe can be recast as a tree-level computation in the gravitational theory, involving computation of so-called Witten diagrams \cite{Witten:AdS}.

\subsection{Twisted holography}
Introduced by Costello and Li in \cite{CLsugra}, the twisted holography proposal posits an avatar of the AdS/CFT correspondence that holds at the level of factorization algebras associated to supersymmetric twists of $T_{CFT}$ and $T_{grav}$. There is an exciting body of work being developed around this program including tests of this proposal from both the gravitational and gauge theory sides.

\parsec{}
Concretely, the twisted holography proposal suggests that the type of duality between the factorization algebras associated to a gravitational theory and to the worldvolume theory of a number of branes is a general version of \textit{Koszul duality}.

Ordinary Koszul duality for associative algebras (so quantum mechanical systems) associates to an (augmented) algebra $A$ a dual algebra $A^!$ whose appropriate derived category of modules is the same as that of $A$.
Following the work of \cite{CLsugra, CP1} (see also the review in \cite{PWkoszul}) there is a simple physical interpretation of Koszul duality.
If $A$ is the algebra of operators of some bulk quantum field theory (perturbatively we can even consider a theory of gravity) then $A^!$ is the algebra of operators on the universal topological line defect.
Universal here means that algebra of operators on any other line defect which couples to the bulk system admits a unique map of algebras from~$A^!$.

The general theory of Koszul duality for factorization algebras has not been developed, and we do not do so here, but see \cite{LurieHA} for the case of $\mb E_n$-algebras and  \cite{gui2022quadratic}, \cite{tamarkin2003deformations} for the case of particular kinds of vertex algebras. This sort of duality would allow one to make sense of universality statements as above for higher dimensional, possibly non-topological, defects in an arbitrary bulk quantum field theory. Roughly, one expects the Koszul dual of a factorization algebra to be the factorization algebra corepresenting the functor of looking at solutions to a Maurer-Cartan equation in a tensor product. 


\parsec{}
Let us now make a more concrete, yet slightly informal, statement of twisted holography which fits into the approach of this thesis. Let $X$ be a smooth manifold, and let $\Obs_{grav}$ denote a factorization algebra on $X$ that we view as the observables of a bulk gravitational theory. Suppose we have, in addition, a stack of $N$ branes, wrapping a closed submanifold $Y\hookrightarrow X$ whose worldvolume theory has a factorization algebra of observables $\Obs_{CFT}^N$. 

Note that $\Obs_{grav}$ is a factorization algebra on $X$, while $\Obs_{CFT}^N$ is a factorization algebra on the closed submanifold $Y$ so we cannot yet compare them.
We can, however, attempt to restrict $\Obs_{grav}$ to a factorization algebra just on $Y$, which we denote by $\Obs_{grav}|_Y$.

\begin{expect}[Twisted holographic principle following \cite{CLsugra}]\label{twistedholog}
There is a map of factorization algebras
\[
  (\Obs_{grav}|_{Y})^{!}\to \Obs_{CFT}^N
\]
that becomes an equivalence in the large $N$ limit.
\end{expect}

As we recalled in the previous subsection, traditional formulations of the AdS/CFT correspondence relate local operators of the gauge theory to states of the gravitational theory on AdS. Therefore, a natural desideratum in relating the above to more traditional statements is a precise relation between the source of the above map and gravitational states in $AdS$. Moreover, there is an operational definition of the operator-product-expansion on the costalk of a Koszul dual factorization algebra which realizes the expectation that Koszul duality corepresents the functor taking Maurer-Cartan elements in the tensor product. Another desideratum is to relate the output of this procedure with the scattering product on gravitational states computed by Witten diagrams. 

\begin{rmk}
In this context, the definition of Koszul duality involves another ingredient, namely the backreaction of branes wrapping $Y$. This is meant to capture the fact that $(\Obs_{grav}|_Y)$ may not be canonically augmented, but we may try to deform it in a way that kills off the obstruction to being augmented. More precisely, one expects that the correct version of Koszul duality for application in holographic contexts is a version of \textit{curved} Koszul duality for factorization algebras. 
\end{rmk}

\begin{rmk}
For finite $N$, this map will in general be neither injective nor surjective. The kernel and cokernel of this map for finite $N$ correspond to interesting nonperturbative effects in the gravitational theory. For instance, in gauge theories:

\begin{itemize}
\item This map has a kernel given by trace relations. Syzygies between trace relations are conjecturally related to the worldvolume theories of certain other branes in the gravitational theory, so-called \textit{giant gravitons} \cite{Gaiotto:2021xce}, \cite{choi2023quantum}, \cite{Imamura} \footnote{We thank Ji-Hoon Lee for conversations related to this topic}

\item This map also has a cokernel. By fiat, these are classes that exist in the finite $N$ cohomology of the observables of a gauge theory that are not in the image of the natural map from the large $N$ theory. Recent developments in cohomological approaches to counting quantum microstates of $\frac{1}{16}$-BPS black holes in $AdS_5\times S^5$ \cite{choi2023quantum} \cite{Chang_2023} \cite{Chang_2013} can be cast as trying to characterize the cokernel of a specific example of this map. 
\end{itemize}
\end{rmk}

\parsec[]
The above expectation can be tested in instances where both sides of the duality admit explicit descriptions. This has been carried out in many examples including:
\begin{itemize}
  \item A stack of $D3$ branes in twisted $\Omega$-deformed type IIB supergravity on flat space. The theory on the stack of $D3$ branes is dual to the closed string B-model on the deformed conifold \cite{costello2021twisted}. This can be understood as a twisted $\Omega$-deformed version of the physical AdS/CFT duality between 4d $\mc{N}=4$ super Yang-Mills and type IIB string theory on $AdS_{5}\times S^{5}$. Here the duality can be formulated in terms of vertex algebras. 
  
  \item M2 branes and M5 branes in twisted $\Omega$-deformed $M$-theory on Taub-NUT space \cite{CostelloM5,CostelloM2}. In the particular $\Omega$-background, M2 branes are localized to a topological quantum mechanical system where the duality can be phrased in terms of associative algebras and ordinary Koszul duality. The koszul dual algebra bears close relations to the spherical Cherednik algebra. The $\Omega$-background localizes M5 to a complex plane and the observables of the localized theory are an affine $W_{N}$ vertex algebra. Many celebrated features of the representation theory of these algebras and their relations with geometry have found natural explanations from the perspective of this twist of M-theory \cite{gaiotto2020miura}, \cite{Oh:2021wes}.
\end{itemize}

The example we consider is closely related to the second of these. Indeed, there is an odd nilpotent element in $\mf{osp}(6|2)$, which we refer to as $S$ in the sequel. Using the inner action of $\mf{osp}(6|2)$ on our eleven-dimensional model on twisted $AdS_7\times S^4$ as identified in proposition \ref{prop:brads7}, $S$ affords a deformation of our model. This is the deformation considered in \cite{BeemEtAl}, and it induces a specialization of characters called the Schur limit.
\parsec[]

\subsection{M5 branes, holomorphy, and holography}
The results in the second half of this thesis can be viewed as baby steps in investigating twisted holography for the minimal twist of the 6d $\mc N=(2,0)$ theory. Let us begin by spelling out the objects in expectation \ref{twistedholog} adapted to our setting. 

\begin{itemize}
\item The 11d spacetime manifold $X$ is $\R\times \C^5$ and $Y$ is a copy of $\C^3$. 
\item The factorization algebra $\Obs_{grav}|_{\C^3}$ has the feature that its semiclassical free limit is the factorization algebra denoted $\clie ^\bullet \left ( \Pi \Omega^{0,\bullet}_{\C^3} (\mc L^N_{AdS_7\times S^4})\right )$ in definition \ref{defn:ads7states}.
\item The factorization algebra $\Obs^N_{CFT}$ describes local observables in the minimal twist of the theory on a stack of $N$ M5 branes wrapping $\C^3$. 
\end{itemize}

Our goal is to try and use this map, and expectations about its kernel and cokernel, to give a concrete description of the target. There have been various approaches to try and characterize the spectrum of $\frac{1}{16}$-BPS states in the 6d $\mc N=(2,0)$ theories of type $A_{N-1}$, which furnish consistency checks to test our proposal against. Some of these involve applications of instanton counting techniques in 5d $\mc N=2$ gauge theory \cite{Kim2013nva} and some of them involve holographic techniques \cite{Imamura}.

As we remarked in subsection \ref{bpsadscft}, the first ingredient is a map of representations of the superconformal algebra between local operators of the CFT and supergravity states. In order to codify such a matching in terms of the kinds of data in the statement of expectation \ref{twistedholog}, we require a matching between supergravity states and the costalk at the origin of the factorization algebra $(\Obs_{grav}|_{\C^3})^!$. This is precisely the content of proposition \ref{prop:altstates}.

\parsec[]

We have the following conjectural large $N$ statements

\begin{conj}[R-Saberi-Williams]\label{conj:classical}
There is an equivalence of holomorphic $\mb{P}_0$-factorization algebras \[ \left( \clie^\bullet (\Pi\Omega^{0,\bullet}_{\C^3} (\mc L^N_{AdS_7\times S^4}))\right)^!\cong \mc U_{\omega} \left ( \Pi\Omega^{0,\bullet}_{\C^3} (\mc L^{r=0}_{AdS_7\times S^4} )\right ).\] where the right hand side denotes a twisted factorization envelope of the local $L_\infty$-algebra $\Pi\Omega^{0,\bullet}_{\C^3} (\mc L^{r=0}_{AdS_7\times S^4} )$. Moreover, upon deforming by the Maurer-Cartan element $S\in \mf {osp}(6|2)$, the factorization algebra $\mc U_{\omega} \left ( \Pi\Omega^{0,\bullet}_{\C^3} (\mc L^{r=0}_{AdS_7\times S^4} )\right )$ has no sections away from a copy of $\C\subset \C^3$, and its restriction to this copy of $\C$ is equivalent to a twisted factorization envelope of the local Lie algebra $\operatorname {Diff}_{\C}$ of holomorphic differential operators on $\C$. 
\end{conj}

Here, the twisting cocycle $\omega$ comes from the shifted Poisson tensor that was induced by the flux in section \ref{sec:ads}. The content in verifying this conjecture is to explicitly compute the twisting coming from the flux sourced by the brane, and check that upon deforming by the element $S$, it induces the correct cocycle on $\operatorname{Diff}(\C^\times)$

The comment in \ref{eqn:winfty} constitutes a very meager consistency check for the second part of this conjecture, where we observe that the Schur limit of the character of the costalk of $\mc U_{\omega} \left ( \Pi\Omega^{0,\bullet}_{\C^3} (\mc L^{r=0}_{AdS_7\times S^4} ) \right)$ recovers the vacuum character of the $W_{1+\infty}$ vertex algebra.

There is a deformation of the twisted factorization envelope of $\operatorname{Diff}_\C$ which yields the $\mc W_{1+\infty}$ vertex algebra, also referred to as the affine Yangian of $\mf {gl}(1)$. In \cite{CostelloM5}, Costello finds this deformation from a loop level computation in his 5d noncommutative gauge theory. We also expect to be able to lift this to the minimal twist. We summarize this expectation in a conjecture. 

\begin{conj}[R-Saberi-Williams]
There is an equivalence of holomorphic factorization algebras \[ \left( \clie^\bullet_\hbar (\Pi\Omega^{0,\bullet}_{\C^3} (\mc L^N_{AdS_7\times S^4}))\right)^!\cong \mc U_{\hbar, \omega} \left ( \Pi\Omega^{0,\bullet}_{\C^3} (\mc L^{r=0}_{AdS_7\times S^4} )\right ).\] where the right hand side denotes a deformation of the factorization algebra in the previous conjecture induced by loop-level effects in our eleven-dimensional model. Moreover, upon deforming by the Maurer-Cartan element $S\in \mf {osp}(6|2)$, the factorization algebra $\mc U_{\hbar, \omega} \left ( \Pi\Omega^{0,\bullet}_{\C^3} (\mc L^{r=0}_{AdS_7\times S^4} )\right )$ has no sections away from a copy of $\C\subset \C^3$, and its restriction to this copy of $\C$ is equivalent to the $\mc {W}_{1+\infty}$ vertex algebra.
\end{conj}

\parsec[]
We now move on to finite $N$ statements. For the lowest steps of the filtration, we can make some very concrete statements.

\begin{conj}[R-Saberi-Williams]
Upon deforming by $S\in \mf {osp}(6|2)$, the factorization algebra $\mc U_\omega (\mc G^{(-1)}_{\C^3} )$ has no sections away from a copy of $\C\subset \C^3$ and its restriction to this copy of $\C$ is equivalent to the Heisenberg vertex algebra.
\end{conj}
To check this conjecture, it remains to compute the pullback of the twisting cocycle $\omega$ under the inclusion of $\mc G^{(0)}$ and see that it deforms to the Heisenberg cocycle. 

\parsec[]
There is a distinguished Lie sub-algebra of the algebra of differential operators on $\C^\times$ which is given by the Witt-algebra of vector fields. The central extension of $\operatorname{Diff} (\C^\times)$ induced by $\omega$ above restricts to the Virasoro central extension. Similarly, in proposition \ref{prop:g0e36} we have identified a distinguished local super-Lie algebra inside $\Pi\Omega^{0,\bullet} _{\C^3}(\mc L^{r=0} _{AdS_7\times S^4} )$ given by $\mc E(3|6)$.

Accordingly, we conjecture the following
\begin{conj}[R-Saberi-Williams]
Upon deforming by $S\in \mf{osp}(6|2)$, the factorization algebra $\mc U_\omega( \mc E(3|6) )$ has no sections away from a copy of $\C\subset \C^3$ and its restriction to this copy of $\C$ is equivalent to the Virasoro vertex algebra.
\end{conj}

Again, to check this conjecture it remains to compute the pullback of the twisting cocycle $\omega$ along the inclusion $\mc E(3|6)\to \Pi\Omega^{0,\bullet}_{\C^3} (\mc L^{r=0}_{AdS_7\times S^4} )$ and compare its deformation with the cocycle giving the Virasoro central extension.

We can once again perform a consistency check at the level of characters of costalks. Indeed, we see that the plethystic exponential of the specialized character $g_0(y=1, y_3=1, q) = \frac{q^2}{1-q}$ is exactly the vacuum character of the Virasoro algebra. 

Moreover, note that combining with conjecture \ref{conj:classical}, we expect maps 
\[\mc U_\omega( \mc E(3|6) )\to \left (\clie^\bullet (\Pi\Omega^{0,\bullet}_{\C^3} (\mc L^N_{AdS_7\times S^4}))\right)^! \to \Obs^N_{CFT}\] for every $N$. This map can be thought of as a Noether-type map associated to an  $\mc E(3|6)$-symmetry of the minimal twist of any finite rank 6d $\mc N=(2,0)$ theory \cite{CG2}.

\parsec[]
More generally, the $\mc W_{1+\infty}$ algebra has as quotients, the $\mc W_N$ algebras when the central charge is set equal to $N$. Accordingly, we dream of the following:

\begin{spec}
Under an integrality condition on the central charge, the map \[ (\Obs_{grav} |_{\C^3} )^! \to \Obs^N_{CFT}\] factors as
\[
\begin{tikzcd}
(\Obs_{grav} |_{\C^3} )^! \ar[r]\ar[d]  & \Obs^N_{CFT} \\
\mc U_{\hbar, \omega} (\Omega^{0,\bullet}_{\C^3} (\mc L^N_{AdS_7\times S^4}))/\mc U_{\hbar, \omega} (\prod _{j\geq {N-1}} \mc G^{(j)}_{\C^3} ) \ar[ur]
\end{tikzcd}
\]
\end{spec}

We can perform a consistency check of the above speculation at the level of characters of costalks. It is expected that the superconformal deformation deforms $\Obs^N_{CFT}$ to the $\mc{W}_N$ vertex algebra. On the other hand, we have that 


\begin{prop}
Upon specializing $y=1,y_3=1$ (so that $y_1 y_2 = 1$), one has 

\begin{align*}
\chi \left ( \Omega^{0,\bullet}_{\C^3,c} (\mc L^N_{AdS_7\times S^4})(0)/ \left ( \mc G^{(-1)}_{\C^3,c}\times \prod _{j\geq {N-1}} \mc G^{(j)}_{\C^3,c}\right )(0)\right ) & = \sum_{j \geq 0}^{N-2} g_j (y_1,y_2, y_3=1,y=1,q) \\
& = \frac{q^2 + q^3 + \cdots + q^{N}}{1-q} 
\end{align*}

The plethystic exponential of the right hand side agrees with the vacuum character of the $W_{N}$ vertex algebra.
\end{prop}
\begin{proof}
By induction it suffices to show that the specialization of the single particle local character $g_j$ of the factorization algebra $\mc U(\mc{G}^{(j)})$ is $q^{j+2} / (1-q)$. 
We have already seen this in the case $j=-1,0$, so it suffices to show this when $k \geq 1$.

First observe that the denominator becomes
\begin{equation}
(1-y_1 q)(1-y_2q) (1-q) .
\end{equation}

Next, we observe that the numerator of $g_j (y_1,y_2,y_3=1,y=1,q)$ can be factored as
\begin{align*}
q^{3 + 3j/2} \left(q^{-(j+2)/2} + q^{-(j-2)/2} - q^{-j/2} (y_1+y_2) \right) 
& = q^{j+2} (1 + q^2 - q (y_1 + y_2)) \\
& = q^{j+2} (1 - y_1 q) (1-y_2 q) 
\end{align*}
where in the last line we have used $y_1 y_2 = 1$.
The result follows.
\end{proof}

\parsec{}
In \cite{raghavendran2022holographic} we try to explicitly characterize the discrepancy between the characters of $\mc U (\mc G^{(j)}_{\C^3})$ and expectations about the superconformal index of the finite rank 6d $\mc N=(2,0)$ theories computed via instanton counting techniques \cite{Kim:2013nva} and the "giant graviton expansion" \cite{Arai_2020,}, \cite{Imamura}. It would be interesting to try and categorify the discrepancies and identify them in terms of modules for $E(3|6)$.


\iffalse
\parsec{}

We would also like to point out compatibility of our expression with a certain ``minimally reduced'' index considered in \cite{Gaiotto:2021xce}. 
This minimal reduction is the result of sending certain parameters to zero while keeping some expression in the fugacities fixed.
To consider it it is useful to make the following change of variables: 
\begin{equation}
z_i = y_i q, \quad w_1 = yq, \quad w_2 = y^{-1} q^2 .
\end{equation}
These variables satisfy the constraint $z_1 z_2 z_3 = w_1 w_2$. 

This minimally reduced index corresponds to taking the following limit in the new fugacities
\begin{equation}\label{eqn:limitgaitto}
z_3 , w_2 \to 0 .
\end{equation}
This yields an index which only accounts for operators which transform trivially with respect to the symmetries that the fugacities $z_3,w_2$ correspond to. 
This will result in an index which has three remaining fugacities.

\begin{prop}
\label{prop:gaiotto}
The limit $z_3 , w_2 \to 0$ of the expression $\chi_{N}(z_i,w_a)$ is 
\begin{equation}
\prod_{a=1}^N \prod_{b,c \geq 0} \frac{1-w_1^{a-1}z_1^{b+1} z_2^{c+1}}{1-w_1^a z_1^b z_2^c} .
\end{equation}
\end{prop}
\begin{proof}
It is easy to see that the $z_3,w_2 \to 0$ limit of $g_{-1}$ is 
\begin{equation}
g_{-1}(z_1,z_2,w_1) = \frac{w_1 - z_1 z_2}{(1-z_1)(1-z_2)} 
\end{equation}
and the $z_3,w_2 \to 0$ limit of $g_0$ is 
\begin{equation}
g_0(z_1,z_2,w_3) = w_1 g_{-1}(z_1,z_2,w_1) .
\end{equation}

In the coordinates $z_i,w_a$ the expression $g_k$, for $k \geq 1$, in \eqref{eqn:gk} can be written as
\begin{equation}
\label{eqn:gk}
\begin{array}{lllll}
g_k (z_i,w_a) \define & \left( z_1z_2z_3 p_k(w_1,w_2) (z_1+z_2+z_3) + p_{k+2}(w_1,w_2)  \right. \\
&\displaystyle \frac{\left.  -z_1z_2z_3 p_{k-1}(w_1,w_2) - p_{k+1}(w_1,w_2) (z_1z_2+z_2z_3+z_1z_3) \right)}{(1-z_1)(1-z_2)(1-z_3)} .
\end{array}
\end{equation}
Here $p_k(w_1,w_2) = \sum_{i+j=k} w_1^i w_2^j$. 

From this expression it is easy to see that $\lim_{z_3,w_2 \to 0} g_k$ is 
\begin{equation}
g_k(z_1,z_2,w_1) = w_1^{k+1} g_{-1}(z_1,z_2,w_1) .
\end{equation}
The result follows from applying the plethystic exponential.
\end{proof}

The $z_3,w_2 \to 0$ limit of our index is quite similar, though not exactly, the index of a four-dimensional $\mc{N}=1$ theory on $\C^2$ where the fugacities $z_1,z_2$ count holomorphic derivatives in each of the complex directions.
Also, note that this minimally reduced index further reduces to the Schur limit (so the character of the $W_N$ vertex algebra) by specializing $z_1 = w_1$.

\subsection{Comparisons to expansions of superconformal indices}

In the final section we would like to exhibit a series of direct consistency checks with our conjectural exact formula for the index of the non-abelian six-dimensional superconformal theory with a number of expansions that have appeared in recent literature. 

\parsec
Let us first focus on the superconformal theory associated to the Lie algebra $\mf{sl}(2)$ (so type $A_1$).
Our conjecture for the superconformal index in this case is the plethystic exponential of $\tilde f_2 (y_i,y,q)$ from equation \eqref{eqn:6dtwo}.
We expand the formal single particle index $\tilde f_2 (y_i, y, q)$ as a series in the variable~$q$, yielding
\begin{align*}
\tilde f_2 (y_i,y,q) & = y^2 q^2 + \left(1 - \chi_{[0,1]}(y_i) y + \chi_{[1,0]} y^2 \right) q^3 \\
& + \left(y^{-2} - \chi_{[0,1]}(y_i) y^{-1} + 2 \chi_{[1,0]}(y_i) - \chi_{[0,1]} (y_i) y + \chi_{[2,0]}(y_i) y^2 \right) q^4 + O(q^5) .
\end{align*}
From this expression, we obtain the $q$-expansion of the index $\tilde \chi_2(y_i,y,q) = \text{PExp}[\tilde f_2]$ as 
\begin{align*}
\tilde \chi_2(y_i,y,q) & = 1 + y^2 q^2 + \left(1-\chi_{[0,1]}(y_i) y + \chi_{[1,0]}(y_i)y^2\right)q^3 \\ 
& + \left(y^{-2} - \chi_{[0,1]}(y_i) y^{-1} + 2 \chi_{[1,0]}(y_i) - \chi_{[0,1]} (y_i) y + \chi_{[2,0]}(y_i) y^2 + y^4\right)q^4 + O(q^5)
\end{align*}

Similarly, for the $\mf{gl}(2)$ theory $\chi_2 = \text{PExp}[f_1 + \tilde f_2] = \chi_1 \cdot \tilde \chi_2$ we find the expansion
\begin{align*}
\chi_2 (y_i,y,q) & = y q + \left(y^{-1} - \chi_{[0,1]}(y_i) + \chi_{[1,0]}(y_i) y + 2y^2 \right) q^2 \\ 
& + \left( \chi_{[1,0]}(y_i) y^{-1} - (\chi_{[1,1]}(y_i)-2) + (\chi_{[2,0]}(y_i) - 2 \chi_{[0,1]}(y_i)) y + 2 \chi_{[1,0]}(y_i) y^2 + 2y^3\right) q^3 \\ & + O(q^4) .
\end{align*}

We observe that these $q$-expansions agree precisely with the expansions in \cite{Kim:2013nva} for the $\mf{gl}(2)$ theory  
(see equations (3.51) and (3.65) of \textit{loc. cit.}).

\parsec

We proceed to compare expansions of our exact expression for the $\mf{gl}(3)$ theory to those in \cite{Kim:2013nva}. 
Recall that our conjectural $\mf{gl}(3)$ index is given by the local character of the holomorphic factorization algebra $\Obs_3$:
\begin{equation}
\chi_3 (y_i,y,q) = \text{PExp}[f_3(y_i,y,q)] = \chi_2(y_i,y,q) \cdot \text{PExp}[g_3(y_i,y,q)]  .
\end{equation}
Here, $f_3(y_i,y,q)$ is the single particle local character for the holomorphic factorization algebra $\Obs_3$ and $g_3(y_i,y,q)$ is given in equation \eqref{eqn:gk}. 

Since $g_3(y_i,y,q) = y^3 q^3 + O(q^4)$ we see that $\chi_3$ and $\chi_2$ agree up to order $q^2$ and the difference at order $q^3$ is simply
\begin{equation}
\chi_3(y_i,y,q) - \chi_2(y_i,y,q) = y^3q^3 + O(q^4) .
\end{equation}
This is again in exact agreement with the index for the $\mf{gl}(3)$ theory computed \cite{Kim:2013nva} up to order $q^3$ (see equation (3.79) of \textit{loc. cit.}). 

\parsec

Next, we compare to expansions for the $\mf{sl}(N)$ theory computed in \cite{Imamura}, where the method of the `giant graviton' expansion is used.
It will be convenient to change the variables $(y_i, y, q) \to (y_i,x,q)$ where 
\begin{equation}
x = qy .
\end{equation} 
We will again expand in powers of $q$.\footnote{To match precisely with the equations in \cite{Imamura} we note that it is necessary to relable the variables $y_i \leftrightarrow u_i$, $x \leftrightarrow \check{x}$, and $q \leftrightarrow y$ where the variable $y$ is distinct from the one we use in this paper!}

Starting with the $\mf{sl}(2)$ theory we find that up to order $q^4$ the single particle index is
\begin{align*}
\tilde f_2 (y_i,x,q) & = x^2 + \chi_{[1,0]}(y_i) x^2 q \\
& + \left(-\chi_{[0,1]}(y_i) x + \chi_{[2,0]}(y_i) x^2 \right) q^2 +  \left(1-x-\chi_{[1,1]}(y_i) x + \chi_{[3,0]}(y_i) x^2 \right) q^3 \\
& + \left(2 \chi_{[1,0]}(y_i) - \chi_{[2,1]}(y_i) x + \chi_{[4,0]}(y_i)x^2 \right)q^4 + O(q^5) .
\end{align*}

It follows that the plethystic exponential $\tilde \chi_2(y_i,y,q)$ of this expression has $q$-expansion
\begin{align*}
\tilde \chi_2(y_i,y,q) & = \frac{1}{1-x^2} +  \frac{x^2}{1-x^2} \chi_{[1,0]}(y_i) q \\
& + \left(- x \chi_{[0,1]}(y_i) +  x^2 (1+x^2)\chi_{[2,0]}(y_i)\right) \frac{1}{1-x^2} q^2 \\ 
& + \left( 1 - x - x^3 + x^6 + (-x - x^3 + x^4) \chi_{[1,1]}(y_i)  + (x^2 + x^4 + x^6)\chi_{[3,0]}(y_i)  \right) \frac{1}{1-x^2} q^3 \\
& + O(q^4) 
\end{align*}
This agrees with the expansion in \cite{Imamura} (see equation (68)) except for the $\mf{sl}(3)$-scalar term at order $q^3$. 
We find $(1-x-x^3+x^6) / (1-x^2) = (1-x^3-x^4-x^5) / (1+x)$ whereas Imamura's result is $1 / (1+x)$. 

Similarly, we can obtain the $q$-expansions for the local character $\chi_3(y_i,x,q)$ of the factorization algebra $\Obs_3$ and compare it to the $q$-expansion for the superconformal index of the $\mf{sl}(3)$ theory in \cite{Imamura}. 
Up to order $q^2$ we have
\begin{align*}
\chi_3(y_i,x,q) & = \frac{1}{(1-x^2)(1-x^3)} + \frac{x^2}{(1-x)(1-x^3)} \chi_{[1,0]}(y_i) q \\
& + \left((-x -x^2 + x^5)\chi_{[0,1]}(y_i) + (x^2 + x^3 + x^4 + x^5+ x^6) \chi_{[2,0]}(y_i) \right) \frac{1}{(1-x^2)(1-x^3)} q^2 \\
& + O(q^3) .
\end{align*}
Again, we find a discrepancy of our expansion compared to \cite{Imamura} at order~$q^3$.
It would be interesting to explain the physical or representation theoretic sources of these discrepancies in each of these cases.
\fi





\bibliographystyle{amsalpha}
\bibliography{bibliography.bib}

\end{document}
