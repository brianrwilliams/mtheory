\documentclass[../main.tex]{subfiles}

\begin{document} 


\subsection{Local characters for twisted superconformal theories}\label{s:localchar}

Suppose that $\mc{F}$ is the factorization algebra of observables of a topological-holomorphic theory on $\R^m \times \C^n$. 
The local character $\chi_\mc{F} ({\bf q})$ is, by definition, the graded character of algebra of local operators $\mc{F}(0)$ with respect to some group of symmetries \cite{SWchar}.
The particular group of symmetries depends on the theory.
In this section we focus on local characters of factorization algebras that arise as twists of six-dimensional $\mc{N}=(2,0)$ supersymmetric theories.

The (complexified) superconformal algebra in dimension six is $\mf{osp}(8|4)$. 
The holomorphic twist of this superconformal algebra is $\mf{osp}(6|2)$. 
We will consider the symmetry by the bosonic subalgebra
\begin{equation}\label{eqn:cartan3}
\mf{sl}(3) \times \mf{sl}(2) \times \mf{gl}(1) \subset \mf{osp}(6|2)  .
\end{equation}
The corresponding generators of the Cartan, as in \S \ref{sec:states}, were denoted $h_1,h_2,h,$ and $\Delta$ and the respective fugacities $t_1,t_2,r,q$.

We have described how this subalgebra embeds as fields in the twist of eleven-dimensional supergravity in \S \ref{s:ads7}. 
In particular, the holomorphic twist of any six-dimensional superconformal theory will have as a symmetry the subalgebra \eqref{eqn:cartan3}.
If the corresponding factorization algebra is $\mc{F}$, and the local operators $\mc{F}(0)$, the local character is then defined by the formal expression
\begin{equation}
\chi_{\mc{F}}(t_1,t_2,r,q) = {\rm Tr}_{\mc{F}(0)} \left((-1)^F t_1^{h_1} t_2^{h_2} r^h q^\Delta\right) .
\end{equation}
In the next section we will compute these characters in the case that the factorization algebra $\mc{F}$ is $\clie_\bullet(\mc{G}_{N,c})$ where $N = 1,2,\ldots$.

We pointed out in \S \ref{sec:states} an alternative parametrization of the fugacities in terms of the parameters $y_1,y_2,y_3,y,q$ which satisfy the constraint $y_1 y_2 y_3 = 1$.
These parameters are related by $y_1=t_1^{-1}, y_2 = t_1 t_2^{-1}, y_3 = t_2$ and $y = q^{1/2} r$. 
We will also consider formulas for the local character $\chi(y_i,y,q)$ in terms of these variables.
  
%Note that for local operators which are the symmetric algebra of some graded vector space we can compute, as usual, the character as the plethystic exponential of a the single particle local character. 
%That is, the character of linear local operators.

\subsection{A relationship to the superconformal index}
\label{sec:sucaindex}
The local character for the holomorphic twist of a six-dimensional $\mc{N}=(2,0)$ supersymmetric theory agrees with the well-known superconformal index.
Generally, in any dimension, the superconformal index counts states $\mc{H}^Q$ which are annihilated by a particular supercharge~$Q$.
The index is defined as a function on the Cartan of a commuting subalgebra with respect to $Q$.
For six-dimensional superconformal theories, a natural choice of a supercharge is the holomorphic twisting supercharge. 
Then the index is sensitive to the so-called $\tfrac{1}{16}$-BPS states.

Recall that the odd part of the $\mc{N}=(2,0)$ supersymmetry algebra is $S_+ \otimes R$ where $S_+$ is the positive irreducible spin representation of $\mf{so}(6)$ and $R \cong \C^4$.
Square-zero supercharges $Q \in S_+ \otimes R$ are stratified by the rank of the corresponding map $R \to (S_+)^* \cong S_-$.
A holomorphic supercharge $Q$ has rank one (such elements automatically square to zero). 
Thus, the superconformal index counts precisely the states in the holomorphic twist.
In the terminology above these states comprise the algebra of local operators $\mc{H}^Q = \Obs(0)$ in the holomorphic twist of the six-dimensional $\mc{N}=(2,0)$ theory.

The six-dimensional superconformal algebra (before twisting) is~$\mf{osp}(8|4)$.
The Cartan of the Lie super algebra is six-dimensional generated by elements
\[
H, J_1,J_2,J_3,R_1,R_2 .
\]
The holomorphic twisting supercharge $Q \in \mf{osp}(8|4)$ and the (super) commuting subalgebra is $\mf{osp}(6|2)$ together with the element 
\[
\Delta \define [Q,Q^\dagger] = H - (J_1 + J_2 + J_3) - 2 (R_1 + R_2) 
\]
where $Q^\dagger$ denotes the superconformal partner to the supercharge $Q$. 
The superconformal index counts states which saturate the BPS bound $\Delta \geq 0$ as a representation for the subalgebra $\mf{osp}(6|2)$. 
To fit with the notation used in this paper, the superconformal index can be written as
\begin{equation}
\mc{I}(y_i,y,q) = \tr_{\mc{H}^Q} (-1)^F q^{H + \frac13 (J_1+J_2+J_3)} y_1^{J_1} y_2^{J_2} y_3^{J_3} y^{R_1 - R_2} .
\end{equation}
This agrees precisely with the local character $\chi(y_i,y,q)$ with the evident change of coordinates for the Cartan of $\mf{osp}(6|2)$. 

\subsection{Exceptional symmetry and a finite $N$ conjecture}

Generally speaking, after twisting there are enhancements of symmetries which are present in the original theory. 
We expect that the same occurs for any six-dimensional superconformal theory. 
In \cite{SW6d} we have shown that at the level of the holomorphic twist the twisted superconformal algebra $\mf{osp}(6|2)$ gets enhanced to the infinite-dimensional exceptional super Lie algebra $E(3|6)$ \cite{KacClass}. 
For the case of the theory on a stack of $N$ fivebranes, whose factorization algebra we denote by $\Obs_N$, this implies that the local operators $\Obs_{N}(0)$ form a representation for $E(3|6)$.

Our goal is to gain knowledge of the structure of $\Obs_N(0)$ as an $E(3|6)$-representation from our holographic analysis of the previous section.
Indeed, in \S \ref{s:fact} we have expressed the restriction of the factorization algebra of observables of twisted eleven-dimensional supergravity to the three-fold $Z$ as the Chevalley--Eilenberg cochains of a local $L_\infty$ algebra $\mc{G}$. 
Recall that we have a decomposition of local Lie algebras $\mc{G} = \oplus_{j \geq -1} \mc{G}^{(j)}$ on the three-fold $Z$. 
From this decomposition we have defined a family of local Lie algebras $\mc{G}_{N}$ on $Z = \C^3$ for $N=1,2,\ldots$.

In \S \ref{sec:factsummary} we explained the expectation that to an open set $U \subset \C^3$, the Lie algebra cohomology of $\mc{G}_{N,c}(U)$ is equivalent to the observables $\Obs_N(U)$ of the six-dimensional theory supported on $U$. 
Each $\mc{G}_N$ is acted on by the local Lie algebra $\mc{G}^{(0)} = \mc{E}(3|6)$.
The $\infty$-jets of $\mc{E}(3|6)$ at $0 \in \C^3$ is exactly the exceptional super Lie algebra $E(3|6)$.  
Thus, for every $N$, the space of local operators of the factorization algebra $\clie_\bullet(\mc{G}_{N,c})$ is naturally and $E(3|6)$-representation. 
At the level of local operators we can make the following conjecture, which we will further elucidate at the level of characters for $E(3|6)$ in the next section.

\begin{conj}
\label{conj:ops}
Let $\Obs_N(0)$ be the local operators of the theory on a stack of $N$ fivebranes wrapping $\C^3$ in $\R \times \C^5$. 
There is an equivalence of $E(3|6)$-representations
\begin{equation}
\Obs_N(0) \simeq \clie_\bullet (\mc{G}_{N,c}) (0) .
\end{equation}
Similarly, let $\tilde \Obs_{N}(0)$ be the local operators of the theory on a stack of $N \geq 2$ fivebranes with the center of mass degrees of freedom removed. 
There is an equivalence of $E(3|6)$-representations
\begin{equation}
\tilde \Obs_N(0) \simeq \clie_\bullet (\tilde{\mc{G}}_{N,c}) (0) .
\end{equation}
\end{conj}

\end{document}

%The Witten index is protected under twisting---in our setup the index $\mc{I}(t_1,t_2,r,q)$ can be computed in the minimal twist of the superconformal theory we start with.
%The minimal twist of the Hilbert space $\mc{H}^Q$ is exactly the space of holomorphic local operators at $0$ in $\C^3$, for details see~\cite{SWchar}. 
%Thus, the index $\mc{I}(t_1,t_2,r,q)$ agrees with the holomorphic character $\chi(t_1,t_2,r,q)$ defined above.

%The superconformal index of a superconformal field theory is the Witten index of the theory in the radial quantization.
%In our situation we look at the Hilbert space $\mc{H}$ of the theory on $S^{5}$ and consider the index heuristically of the form
%\begin{equation}
%\tr_{\mc{H}} (-1)^F  x_1^{G_1} \cdots x_n^{G_n} q^{\Delta}
%\end{equation}
%where $\{G_i\}$ are a collection of charges that commute with the holomorphic supercharge $Q$ and its superconformal adjoint $S = Q^\dagger$. 
%Here, $\Delta = [Q,S]$.
%We choose three elements $G_1,G_2,G_3$ in such a way that they become the elements $h_1,h_2,h$ upon taking $Q$-cohomology (and so automatically commute with $Q$ and $S$). 
%Thus we consider the following index
%\begin{equation}
%\mc{I} (t_1,t_2,r,q) \define \tr_{\mc{H}} (-1)^F t_1^{h_1} t_2^{h_2} r^h q^\Delta .
%\end{equation}
%After tracing over $\mc{H}$ one can identify the superconformal index with the partition function of the model on a space which is topologically equivalent to~$S^{5} \times S^1$.



%\[
%(\C^3 - 0) / \sim  \; \simeq \; S^5 \times S^1 .
%\]
%The perspective of the holomorphic twist allows us to holomorphic theory agrees with the partition function on the product of spheres is basically goes by the process of `radial quantization'. 
%Consider the restriction of the theory to $\C^3 - 0 \subset \C^3$ and its dimensional reduction to quantum mechanics along
%\[
%|-| : \C^3 - 0 \to \RR_{>0} .
%\]
%The fiber of this map over a point is $S^5$. 
%By the nature of holomorphic QFT, we can extract from the OPE in the radial direction a canonical associative ($A_\infty$) algebra $\cA_{\mf{u}(1)} = \int_{\C^3 - 0} \Obs$ which is roughly the value of the theory on $S^5$. 
%There is a canonical boundary condition of the quantum mechanics theory at radius $r = 0$ given by the local operators $\Obs(0)$ at $0 \in \C^3$ which, in turn, has the structure of a $\cA$-module. 
%By standard arguments placing this quantum mechanics theory on circle $S^1$ results in the trace of the $\cA$-module $\Obs(0)$ 
%\[
%Z(S^{5} \times S^1) = {\rm Tr}_{\cA} (\Obs(0)) .
%\]
%From the trace on the right-hand side we can recover the character as defined above. 
%Indeed, the $E(3|6)$-module structure on local operators factors through a map 
%$E(3|6) \to \cA$ since we wrote down the explicit Hamiltonians above in \eqref{eq:ham1}, for instance. 

%\subsection{Comparison to `states'}


%\subsection{Categorifying the index for free theories}
%
%In the case of both membranes and fivebranes we constructed a particular restriction of the local $L_\infty$ algebra $\mc{L}_{sugra}$ to the respective worldvolume theories which we denoted by $\Bar{\pi}_* \mc{L}_{sugra}$. 
%There are two important sub local $L_\infty$ algebras 
%\[
%\begin{tikzcd}
%& \Bar{\pi}_*\mc{L}_{sugra} & \\
%\Bar{\pi}_*\mc{L}_{sugra}^{(-1)} \ar[ur] & & \Bar{\pi}_*\mc{L}_{sugra}^{(0)} \ar[ul] .
%\end{tikzcd}
%\]
%This diagram induces a diagram of factorization algebras
%\[
%\begin{tikzcd}
%& \left(\Obs_{sugra}|_Z\right)^! & \\
%\clie_\bullet(\Bar{\pi}_*\mc{L}_{sugra,c}^{(-1)}) \ar[ur] & & \clie_\bullet(\Bar{\pi}_*\mc{L}_{sugra,c}^{(0)}) \ar[ul].
%\end{tikzcd}
%\]

%\parsec[s:sugraops]
%
%By the usual methods of the BV formalism the action functional $S_{sugra}$ described above endows the parity shift of the fields $\mc{L}_{sugra} = \Pi \mc{F}_{sugra}$ with the structure of a holomorphic-topological local $\Z/2$ graded $L_\infty$ algebra. 
%
%On $\C^5 \times \R$ we can describe this super Lie algebra structure explicitly. 
%First, by the Dolbeault and de Rham Poincar\'e lemmas it is easy that the even part of the super Lie algebra $\mc{L}(\C^5 \times \R)$ is equivalent to a one-dimensional central summand $\C$ plus the Lie algebra of divergence-free vector fields on $\C^5$:
%\[
%\Vect_0 (\C^5) = \{X \in \Vect(\C^5) \; | \; \div X = 0\} .
%\]
%The odd part of the super Lie algebra $\mc{L}(\C^5 \times \R)$ is equivalent to the space of holomorphic one-forms on $\C^5$ modulo exact one-forms
%\[
%\Omega^{1,hol}(\C^5) / {\rm Im}(\del) 
%\]
%which is, of course, equivalent to the space of closed holomorphic two-forms $\Omega^{2,hol}_{cl}(\C^5)$. 
%
%\begin{thm}[\cite{RSW}[Theorem 2.1]]
%The Taylor expansion map determines a map of $\Z/2$ graded $L_\infty$ algebras
%\[
%j_\infty : \mc{L}_{sugra}(\C^5 \times \R) \to L_{sugra} .
%\]
%Furthermore, $L_{sugra}$ is equivalent as a $\Z/2$ graded $L_\infty$ algebra to $\Hat{E(5|10)}$. 
%\end{thm} 
%
%As an immediate corollary of this result we obtain by Lemma \ref{lem:localops} the following.
%
%\begin{cor}
%\label{cor:sugraops}
%Let $\Obs_{sugra}$ be the factorization algebra on $\C^5 \times \R$ of classical observables of the minimal twist of eleven-dimensional supergravity.
%There is a quasi-isomorphism of commutative dg algebras
%\[
%\Obs_{sugra} (0) \simeq \clie^\bullet \left( \Hat{E(5|10)} \right) .
%\]
%\end{cor}

%\end{document}
