\documentclass[../main.tex]{subfiles}

\begin{document} 

\section{$E(1|6)$ modules from gravitons on $AdS_4\times S^7$}\label{sec:e16}

Having justified that the spaces of supergravity states constructed in the previous subsection are in fact counting gravitons on $AdS_4\times S^7$ and $AdS_7\times S^4$ respectively, we turn to studying representation theoretic properties of these state spaces.

A feature of the physical $AdS_4\times S^7$ and $AdS_7\times S^4$ backgrounds is that they have as isometries, the 3d $\mc N=8$ and 6d $\mc N=(2,0)$ superconformal algebras respectively.
In each case, the complex form of the superconformal algebra is $\lie{osp}(8|4)$ (though their real forms differ).
In both the $AdS_4$ and $AdS_7$ cases there is a map of super Lie algebras 
\[
\phi \colon \mf {siso}_{11d} \to \mf {osp}(8|4).
\]
Further, it is observed in \cite{SWsuco2} that if $Q\in \mf{siso}_{11d}$ is the odd square-zero element used to define the minimal twist of eleven-dimensional supergravity, then the $\phi(Q)$-cohomology of the superconformal algebra is 
\[ H^\bu \left ( \mf {osp}(8|4), [\phi(Q), - ]\right )\cong \mf {osp}(6|2).\]

The super Lie algebra $\mf{osp}(6|2)$ plays the role of the residual isometries of the twisted $AdS$ background. 
The bosonic part of $\mf{osp}(6|2)$ is the direct sum Lie algebra $\mf{sl}(4) \oplus \mf{sl}(2)$. The odd part of the algebra $\mf{osp}(6|2)$ is $\wedge^4 W \otimes R$ where $W$ is the fundamental $\mf{sl}(4)$ representation and $R$ is the fundamental $\mf{sl}(2)$ representation.
The super Lie bracket is determined by the natural invariant symmetric-symmetric and anti-symmetric-anti-symmetric pairings.

In this section we will focus on $AdS_4$.
In \cite{RSW} we have witnessed the super Lie algebra $\lie{osp}(6|2)$ as symmetries of our eleven-dimensional model on the eleven-manifold obtained by subtracting the location of the brane
\beqn
\til Z = \R \times \C^5 - (\R \times \C \times 0) .
\eeqn
We have further checked that these symmetries extend to symmetries in the theory $\mc E^N_{AdS_4}$ for nonzero flux $N \ne 0$; thus providing evidence that $\lie{osp}(6|2)$ is a global symmetry of twisted $AdS$ space.

As a consequence, the space of twisted supergravity states $\cH_{AdS_4}$ as a representation for $\lie{osp}(6|2)$.
In this section we argue that the symmetries of twisted $AdS$ space enhance to an infinite-dimensional symmetry algebra called $E(1|6)$, which is also of exceptional type \cite{KacClass}.
As a consequence, we understand $\cH_{AdS_4}$ as a representation for this infinite-dimensional enhancement of the three-dimensional twisted superconformal algebra.

%In this section, we focus on the case of gravitons on $AdS_4\times S^7$, using the description of the state space afforded by proposition \ref{prop:altstates} which describes it as the costalk of a factorization envelopes of the boundary conditions $\Omega^\bullet _\R\otimes \Omega^{0,\bullet}_\C (\mc L^{{r=0}}_{AdS_4} )$.

To this end, we construct a $\C^\times$-action on our eleven-dimensional model on twisted $AdS_4$ space.
The homogenous components of this $\C^\times$-action should be understood as a twisted version of excited gravitational states at a particular charge.
The weight zero part of this $\C^\times$-action is endowed with the structure of a Lie algebra, and we will see that it is precisely $E(1|6)$.
This action readily gives a decomposition of the state space $\mc H_{AdS_4}$ into $E(1|6)$-representations.
We explicitly characterize the summands of this decomposition with their module structures and give closed form expressions for their characters.

%As is usual in the perturbative BV formalism, the BV action functional equips the parity shift of the fields $\Pi \cE$ with the structure of an $L_\infty$ algebra.
%Consider the eleven-dimensional model that we introduced in the first section describing the putative holomorphic twist of supergravity.
%On flat space $\til Z = \R \times \C^5$ we have shown in \cite{RSW} that there is a quasi-isomorphism of $\Pi \cE (\R \times \C^5)$ with a central extension of the exceptional super Lie algebra $E(5|10)$.
%In this section we will deduce the analogous symmetry algebra on the twisted analogs of AdS space.

\subsection{The graviton decomposition of twisted $AdS_4\times S^7$}

Recall the boundary condition at zero in twisted $AdS$ space $\cA^\bu(\cL_{AdS_4}^{r=0})$ that we introduced in section \ref{sec:transversebc}.
Recall that this boundary condition makes sense for any three-manifold $M$ equipped with a THF.
The parity shift of these boundary fields $\cA^\bu(\Pi \cL_{AdS_4}^{r=0})$ is equipped with an $L_\infty$ structure inherited from the BV action of the eleven-dimensional theory.

%We define a decomposition of the space of states $\mc H_{AdS_4}$.
%It is induced by a decomposition of the boundary fields $\cA^\bu_{M} (\mc L^{{r=0}}_{AdS_4} )$ introduced in section \ref{sec:transversebc}  arising from a $\C^\times$ action which mixes fiberwise rescalings on spacetime with a fiberwise rescaling of the space of fields. 
%We first consider a general three-manifold $M$ equipped with a THF; we specialize to $M = \R \times \C$ shortly.

Define the $\C^\times$-action on holomorphic sections of $\cL_{AdS_4}^{r=0}$:
\begin{itemize}
\item On the sections 
\[
\mu(w_a,z) \in \C[w_1, \cdots, w_4]\{\del_{w_a}\} \otimes \cO_M \oplus \C[w_1, \cdots, w_4]\otimes T_M \] 
the action is
\[
\lambda \cdot \mu(w_a,z) = \mu(\lambda w_a , z).
\]
\item On the fields $\nu(w_a,z) \in \C[w_1, \cdots, w_4] \otimes \cO_M$ the action is
\[
\lambda \cdot \nu(w_a,z) = \nu(\lambda w , z).
\]
\item On the fields $\beta(w_a,z) \in  \C[w_1, \cdots, w_4] \otimes \cO_M$ the action is
\[
\lambda \cdot \beta(w_a,z) = \lambda^{-2} \beta(\lambda w_a , z).
\]
\item On the fields 
\[
\gamma(w_a,z) \in  \C[w_1, \cdots, w_4] \{\d w_a\} \otimes \cO_M \oplus  \C[w_1, \cdots, w_4] \otimes \Omega^1_M
\] 
the action is
\[
\lambda \cdot \gamma(w_a,z) = \lambda^{-2} \gamma(\lambda w _a, z).
\]
\end{itemize}

This $\C^\times$-action on holomorphic sections extends to an action on the de Rham--Dolbeault complex $\cA^\bu_M(\cL_{AdS_7}^{r=0})$ in such a way that preserves the de Rham--Dolbeault operator.
In fact, this $\C^\times$-action preserves the full local $L_\infty$ structure present on the parity shift of this complex of vector bundles.

\begin{prop}\label{prop:ads4decomp}
The $L_\infty$ structure on $\Pi \cA_M^\bu(\mc L^{r=0}_{AdS_4} )$ identified in section \ref{sec:transversebc} is equivariant for this $\C^\times$ action.
\end{prop}

This result induces a product decomposition 
\[
 \cA_M^\bu(\mc L^{r=0}_{AdS_4} ) = \prod _{n\geq -2} \mc F^{(n)}_{M}
\]
where $\cF^{(n)}_M \subset \cA_M^\bu(\mc L^{r=0}_{AdS_4} )$ is the weight $n$ eigenspace with respect to the above $\C^\times$ action thought of as a complex of super vector bundles on $M$. 
In particular, we see that $\mc F^{(0)}_{M}$ is itself a local dg Lie algebra, for which  every $\mc F^{(n)}_{M}$ is a module.

\subsection{The lowest pieces and minimally twisted rank 1 ABJM}\label{sec:BLG}
\parsec[] The first nontrivial case is the weight ${(-2)}$ piece. We have the following

\begin{lem}
There is an isomorphism of elliptic complexes \[\mc F^{(-2)}_{M} \simeq \Omega^\bullet_{M}.\]
\end{lem}
\begin{proof}
The only sections which contribute are those of type $\beta$ or $\gamma$ with no form components along the fiber directions. Therefore, we see directly that \[\mc F^{(-2)}_{M} \simeq \cA^\bu_M \xrightarrow{\del} \cA^\bu_M(\Omega^1) = \Omega^\bu_M.\] 
\end{proof}

\parsec[] The next nontrivial case is the weight ${(-1)}$ piece. 

\begin{lem}
There is an isomorphism of elliptic complexes 
\[ \mc F^{(-1)}_{M} \cong \cA^\bu_M \left (K^{1/4}\otimes (\C^4)^* \oplus \Pi K^{3/4}\otimes \C^4 \right ) \]
\end{lem}
\begin{proof}
The sections of this specified weight are:
\begin{itemize}
\item sections of type $\mu$ of the form $\mu_a(z) \del_{w_a}$. 
As the $w_a$ are fiber coordinates on $K^{1/4}$, these fields transform as sections of $K^{1/4}$.
\item sections of type $\gamma$ of the form $\gamma_a(z) \d w_a$. 
These fields transform as sections of $K^{3/4}$. 
\end{itemize}
\end{proof}

\parsec[]
We wish to flag an appearance of $\mc F^{(-1)}_{\R\times \C}$ in supersymmetric physics in three-dimensions. There is a highly supersymmetric Chern-Simons-matter theory discovered by \cite{} which describes the low-energy dynamics of a single M2 brane probing the origin in $\C^4$. The aptly named ABJM theory has a manifest $\mc N=6$ superconformal symmetry, and admits a holomorphic-topological twist that was characterized by Garner in \cite{Garner2022vds}. Non-perturbatively, it has an $\mc N=8$ superconformal symmetry, the enhancement to which is furnished by monopole operators. 

The sheaf of complexes $\mc F^{(-1)}_{\R\times \C}$ matches the field contents of the holmorphic-topological twist of the rank 1 ABJM theory at level 1. In work-in-progress with Garner and Williams, we explicate the action of $\mc F^{(0)}_{\R\times \C}$ on $\mc F^{(-1)}_{\R\times \C}$ at the level of local operators in terms of currents. As we describe in the next subsection, the infinity jets at the origin of $\mc F^{(0)}_{\R\times \C}$ contain a copy of the minimally twisted three-dimensional $\mc N=8$ superconformal algebra. In keeping with the fact that the rank 1 level 1 ABJM theory only admits three-dimensional $\mc N=8$ superconformal symmetry after taking into account monopoles, we see that the currents realizing the $\mc F^{(0)}_{\R\times \C}$ action involves monopoles. The fact that the action is manifest from the perspective of this paper articulates a sense in which perturbative gravity may contain information about non-perturbative brane dynamics.

\subsection{The zero-th piece: A local version of $E(1|6)$}
\parsec[] The next nontrivial case is the weight zero piece $\cF^{(0)}_M$. 
This factor is special because it carries the induced structure of a local $L_\infty$ algebra on $M$. 
We will prove that it is equivalent to a local Lie algebra version of the exceptional super Lie algebra $E(1|6)$. 

We first recall the definition of this super Lie algebra \cite{KacBible}. 
The even part $E(1|6)_0$ is the semidirect product Lie algebra 
\beqn
\Vect(\Hat{D}) \ltimes \left( \cO(\Hat{D}) \otimes \mf{sl}(4) \right )
\eeqn
where $\Vect(\Hat{D})$ and $\cO(\Hat{D})$ denote vector fields and functions on the formal disk.

The odd part is the (unique) nontrivial extension of $E(1|6)_0$-modules 
\[0\to \Sym^2 (\C^4)\otimes \Gamma(\Hat{D}, K^{1/2}_{\Hat{D}}) \to E(1|6)_1 \to \wedge^2 (\C^4) \otimes \Gamma(\Hat{D}, K^{-1/2}_{\Hat{D}}) \to 0 
\]
where $K^{\pm 1/2}_{\Hat{D}}$ are positive and negative square roots of the canonical bundle on $\Hat{D}$.

To characterize this super Lie algebra we must define the following brackets.
\begin{itemize}
\item Given elements $A \otimes f \d z^{1/2}\in \Sym^2 (\C^4)\otimes K^{1/2}_\C$ and $B\otimes g \del_z^{1/2}\in \wedge^2 (\C^4)\otimes K^{-1/2}_\C$, we have that
\[
[A \otimes f \d z^{1/2}, B\otimes g \del_z^{1/2}] = A \star B \otimes fg \in \mf{sl}(4)\otimes \mc O.
\]
Here, $\star$ refers to the hodge star of $B$ and we are viewing $A$ and $*B$ as symmetric and skew-symmetric $4\times 4$ matrices respectively; their product is traceless. 

\item The final nontrivial bracket is more complicated. 
Given elements $B \otimes f \del_z^{1/2}, B'\otimes g \del_z^{1/2} \in \Gamma( \hat D, \wedge^2 (\C^4) \otimes K^{1/2}_\C)$, we have
\begin{align*}
[B\otimes f \d z^{-1/2} , B' \otimes g \d z^{-1/2} ] & = \tr (B \star B') \otimes fg \del_z \\ & + \frac12 (B \star B')_0 \otimes \left (\del (f \d z^{-1/2} ) g \d z^{-1/2} + f \d z^{-1/2} \del (g \d z^{-1/2} ) \right ) \\
& \in \Gamma (\hat D, T) \ltimes \left (\mf {sl}(4) \otimes \Gamma (\hat D, \mc O) \right ) .
\end{align*}
where again $\star$ denotes the Hodge star and the subscript of zero denotes projection to the traceless part. 
\brian{simplify last line so it is obviously a vector field}
\surya{isnt the last line an $\mf{sl}_4$ current ?}
\brian{Choose $\d z^{-1/2}$ or $\del_z^{1/2}$}
\end{itemize}

The relationship between this super Lie algebra and our decomposition is established through the following result.

\begin{prop}
Consider the case $M = \R \times \C$.
There is a quasi-isomorphism of super Lie algebras
\[
J^\infty_0\mc F^{(0)}_{\R\times \C} \simeq E(1|6).
\]
where the left hand side denotes the $\infty$-jets at $0\in \R\times \C$ of the local super-Lie algebra $\mc F^{(0)}_{\R\times \C}$. 
\end{prop}
\begin{proof}
We will characterize the local $L_\infty$-algebra $\mc F^{(0)}_{M}$ for general $M$. 
We will show that it is quasi-isomorphic to a local version of $E(1|6)$ where sections over the formal disk can be upgraded to sections over any open set in the THF manifold $M$.

The underlying complex of vector bundles $\mc F^{(0)}_{M}$ is the de Rham--Dolbeault complex of the complex of bundles 
\begin{equation}
\begin{tikzcd}
\ul{even} & \ul{odd} \\
\C\{w_a\del_{w_b}\}\otimes \mc O\ar[r, "\del^W_\Omega"]  & \mc O \\ 
\T \ar[ur] & \\
\Sym^2 (\C^4)\ar[r, "\del_W"]\ar[dr] & \C\{w_a\d w_b\} \otimes K^{-1/2}\ \\
& \Sym^2 (\C^4)\otimes \Omega^1\otimes K^{-1/2}
\end{tikzcd}
\end{equation}
Of course, the differentials are just appropriate components of the divergence operator and holomorphic deRham operator. We can compute cohomology by way of a spectral sequence whose first page is the cohomology with respect to $\del^W_\Omega + \del_W$. We see that the differential $\del^W$ maps surjectively onto functions and its kernel is isomorphic to $\mf{sl}(4)\otimes \mc O$. Likewise, the differential $\del_W$ is the canonical inclusion of $\mf{sl}(4)$ representations $\Sym^2 (\C^4)\to \C^4\otimes \C^4$. 
Its cokernel is a copy of $\wedge^2 \C^4$. 

Thus, we see that this page of the spectral sequence is given by
\begin{equation}
\mc E_M(1|6) \define \cA^\bu_M \left (
\begin{tikzcd}
\ul{even} & \ul{odd} \\
\T & \wedge^2 (\C^4) \otimes K^{-1/2} \\
\mf {sl}(4)\otimes \mc O & \Sym^2 (\C^4)\otimes K^{1/2}
\end{tikzcd} \right)
\end{equation}
and there are no non-zero differentials so the spectral sequence degenerates.

To see that the Lie structure induced from the $L_\infty$-structure on $\cA_M^\bu (\mc L^{r=0}_{AdS_4} )$ is in fact given by the same formulae as the brackets on $E(1|6)$ in equation \ref{defn:e(1|6)}, we provide an explicit quasi-isomorphism $\Psi^{(0)} \colon \mc E_M(1|6) \to \mc F^{(0)}_{M}$.
This quasi-isomorphism arises from a map on holomorphic sections.

\begin{itemize}
\item Given a section $\mu$ of $\T_M$ define
\begin{align*}
\Psi^{(0)} (\mu) &= \mu - \frac 14 (\del_\Omega \mu) w_a\del_{w_a} 
\end{align*}
where we view the right hand side as a section of $\T \oplus \C\{w_a \del_{w_a}\} \otimes \cO$.
\item Given a section $C = (C_{ab})$ of $\mf{sl}(4)\otimes \mc O$ define
\[
\Psi^{(0)} (C) = C_{ab}w_a\del_{w_b}
\]
which we view as a section of $\C\{w_a \del_{w_a}\} \otimes \cO$.
\item Given a section $B = (B_{ab})$ of $\wedge^2 (\C^4) \otimes K^{-1/2}$ define
\[
\Psi^{(0)} (B) = B_{ab} w_a \d w_b .
\]
which we view as a section of $\C\{w_{[a} \d w_{b]}\} \otimes K^{-1/2} = \wedge^2 \C^4 \otimes K^{-1/2}$.
\item Given a section 
$A=(A_{ab})$ of $\Sym^2 (\C^4) \otimes K^{1/2}$ define
\begin{align*}
\Psi^{(0)} (A_{ab}) = A_{ab} w_a w_b 
\end{align*}
which we view as a section of $\Sym^2 (\C^4) \otimes \Omega^1 \otimes K^{-1/2}$.
\end{itemize}
It is easy to see that $\Psi^{(0)}$ is a quasi-isomorphism and a straightforward if lengthy check confirms that it preserves brackets. 

The result about $\infty$-jets at $0$ in affine space $M = \R \times \C$ follows immediately.
\end{proof}

%\begin{rmk}
%We note that the map $i_{M2}$ from lemma \ref{lem:m2emb} in fact defines a Lie map from $\mf{osp}(6|2)$ to the sections of the boundary condition $\cA_{\R \times \C}^\bu (\mc L^{r=0}_{AdS_4} )$ over every open set containing the origin. 
%The image of the map lands exactly in the step $\mc F^{(0)}_{\R\times \C}$ of the decomposition from proposition \ref{prop:ads4decomp}. 
%Therefore we see that $E(1|6)$ contains $\mf{osp}(6|2)$ as a finite dimensional subalgebra.  
%\end{rmk}

\subsection{General summands and $E(1|6)$-modules}
We now move on to giving an explicit description of the general summand $\mc F^{(j)}_{\R\times \C}$ for $j \geq 1$. By proposition \ref{prrop:ads4decomp} the costalks $\mc F^{(j)}_{\R\times \C, c} (0)$ are $E(1|6)$-modules. A full classification of irreducible $E(1|6)$-modules is incomplete. Nevertheless, we expect the modules we find are irreducible.

\surya{i think this is probably pretty quick just using $\mf{sl}_4$-rep theory}

We first fix some notation for irreducible highest weight representations of $\mf{sl}(4)$. Let $\mf {h}\subset \mf{sl}$ be the Cartan given by diagonal matrices and let $L_i\in \mf {h}^*$ be the linear functional that picks out the $i$-th diagonal entry. We may accordingly write $\mf h^* = \C \{L_1, L_2, L_3, L_4\}/(L_1+\cdots + L_4)$. We will write $\Gamma_{a_1,a_2, a_3}$ for the irreducible representation of $\mf {sl}(4)$ of highest weight $(a_1+a_2+a_3)L_1+(a_2+a_3)L_2 + a_3L_3$. 

\begin{prop}
Let $j\geq 1$. The complex of vector bundles $\mc F^{(j)}_{\R\times \C}$ is quasi-isomorphic to 
\begin{equation}
\Omega^\bullet_\R\otimes \Omega^{0,\bullet}_\C \left (
\begin{tikzcd}
\ul{even} & \ul{odd} \\
\Gamma_{j,1,0} \otimes K^{-j/4} & \Sym^{j+2}(\C^4) \otimes \Omega^1 \otimes K^{-(j+2)/4} \\
\Sym^j (\C^4) \otimes \T \otimes K^{-j/4} & \Gamma_{j+1, 0, 1} \otimes K^{-(j+2)/4}
\end{tikzcd} \right )
\end{equation}


\end{prop}
\begin{proof}
We begin by noting that we can explicitly describe $\mc F_{\R\times \C}^{(j)}$ as $\Omega^\bullet _\R \otimes \Omega^{0,\bullet}_\C (F^{(j)})$ where $F^{(j)}$ denotes the following dg-vector bundle:
\begin{equation}
\begin{tikzcd}
\ul{even} & \ul{odd} \\
\Sym^{j+1} (\C^4) \otimes (\C^4)^* \otimes K^{-j/4} \ar[dr, "\del^W_\Omega"] \\ & \Sym^j (\C^4) \otimes K^{-j/4} \\ 
\Sym^j (\C^4) \otimes \T \otimes K^{-j/4} \ar[ur, "\del^V_\Omega"'] & \\
& \Sym^{j+2}(\C^4) \otimes \T^*\otimes K^{-(j+2)/4}\\ 
\Sym^{j+2} (\C^4) \otimes K^{-(j+2)/4}\ar[ur, "\del_V"] \ar[dr, "\del_W"'] \\
& K^{-(j+2)/4}\otimes \Sym^{j+1}(\C^4) \otimes \C^4 . 
\end{tikzcd}
\end{equation}

Note that the differentials here are all $\mf{sl}(4)$ equivariant maps, tensored with a differential operator acting on sections of a bundle on $\C$. In particular

\begin{itemize}
\item The differential $\del^W_\Omega$ involves the canonical projection 
\[\Sym^{j+1} (\C^4)\otimes (\C^4)^* \to \Sym^j (\C^4).\] Its kernel is precisely the irreducible highest weight representation $\Gamma_{j+1, 0 ,1}$.

\item The differential $\del_W$ is the canonical inclusion 
\[\Sym^{j+2} (\C^4) \to \Sym^{j+1}(\C^4) \otimes \C^4.\] Its cokernel is the irreducible highest weight representation $\Gamma_{j, 1, 0}$.
\end{itemize}

We can compute the cohomology using a spectral sequence whose first page is given by the cohomology with respect to $\del^W_\Omega + \del_W$. There are no further differentials on this page so the result follows. 
\end{proof}

\subsection{Characters of $E(1|6)$-modules}

Note that the decomposition of the state space $\Sym (\mc H_{AdS_4} ) = \prod_{j\geq -2} \mc U (\mc F^{(j)} _{\R\times \C} )(0)$ gives a product formula for the characters computed in proposition \ref{prop:ads4index}


\[\chi \left (\Sym (\mc H_{AdS_4} ) \right )= \prod_{j\geq -2} \chi \left ( \mc U (\mc F^{(j)} _{\R\times \C} )(0)\right ).\]

We end this section by computing each of the characters $\chi \left ( \mc U (\mc F^{(j)} _{\R\times \C} )(0)\right )$. We will express our characters in terms of characters of highest weight respresentations of $\mf{sl(4)}$ which we denote $\chi^{\mf{sl}(4)}(\Gamma_{a_1,a_2, a_3} )$.

\parsec[]
From the characterization in \ref{}, the lowest step of the decomposition $\mc F^{(-2)}$ is just given by the deRham complex on $\R\times \C$, and accordingly the character of $\mc F^{(-2)}_{c}(0)$ is the constant function 1.

\parsec[]
We proceed to the next step of the decomposition, using the characterization in \ref{}.

\begin{prop}
The character $\chi \left ( \mc U(\mc F_{\R\times \C}^{(-1)})(0)\right )$ is given by the plethystic exponential of the following expression:
\begin{equation}
f_{-1}(t_1, t_2, t_3, q) = \frac{q\left (q^{-3/4}(t_1+ t_2+t_3 + t_1^{-1} t_2^{-1} t_3^{-1} )-q^{-1/4}(t_1^{-1} + t_2^{-1}+t_3^{-1} + t_1t_2t_3)\right )}{(1-q)}
\end{equation}
\end{prop}
\begin{proof}
The proof proceeds by the same trick as in the proof of proposition \ref{prop:altstates}. To describe the costalk, we wish to compute a limit of sections of $\mc F^{(-1)}_{\R\times \C,c}$ on open sets of the form $I\times D$ containing the origin in $\R\times \C$. Using ellipticity, we can describe such sections as a module over the ring generated by holomorphic derivatives of the delta function. 

Accordingly, we have contributions from the following summands:
\begin{itemize}
\item An even copy of $\C^4\otimes \C\{\d z^{3/4}\} \otimes \C[\del_z]\delta_{z=0}$. The character of this summand is
\[
\frac{q\left (q^{-3/4}\chi^{\mf{sl}(4)}(\Gamma_{1,0,0}) \right )}{(1-q)} =\frac{q\left (q^{-3/4}(t_1+ t_2+t_3 + t_1^{-1} t_2^{-1} t_3^{-1} )\right )}{(1-q)}
\]
\item An odd copy of $\C^4\otimes \C\{\d z^{1/4}\} \otimes \C[\del_z]\delta_{z=0}$. The character of this summand is
\[
\frac{-q\left (q^{-1/4}\chi^{\mf{sl}(4)}(\Gamma_{0,0,1}) \right )}{(1-q)} =\frac{-q\left (q^{-3/4}(t_1^{-1} + t_2^{-1} +t_3^{-1}  + t_1 t_2 t_3)\right )}{(1-q)}
\]
\end{itemize}
\end{proof}

Note that under the change of fugacities in \ref{}, this matches exactly with the single particle index for the theory on a single M2 brane \cite[Eq. (2.32)]{Bhattacharya:2008zy}.

\parsec[]
We continue to the next step of the decomposition given by $\mc{F}^{(0)}_{\R\times \C}$. 

Arguing similarly as in the proof of the previous proposition, we have the following.
\begin{prop}
The character $\chi \left ( \mc U (\mc{F}^{(0)}_{\R\times \C})(0)\right )$ is given by the plethystic exponential of the following expression:
\begin{equation}
f_0(t_1, t_2, t_3, q) = \frac{q}{(1-q)}\left ( q^{1/2}\chi^{\mf{sl}(4)}(\Gamma_{0,1,0})  + q^{-1/2}\chi^{\mf{sl}(4)}(\Gamma_{2,0,0})  - q - \chi^{\mf{sl}(4)}(\Gamma_{1,0,1}) \right)
\end{equation}
\end{prop}

\parsec[]
Finally, we continue to the general step of the decomposition.

\begin{prop}
Let $j\geq 1$. The character $\chi \left ( \mc U (\mc F^{(j)}_{\R\times \C} ) (0)\right )$ is the plethystic exponential of the following expression:
\begin{equation}
f_j(t_1, t_2, t_3, q) = \frac{q}{(1-q)}\left (\begin{aligned} q^{(j-2)/4}\chi^{\mf{sl}(4)}(\Gamma_{j+2,0,0})  &+ q^{(j+2)/4}\chi^{\mf{sl}(4)}(\Gamma_{j+1,0,1}) \\ - q^{j/4}\chi^{\mf{sl}(4)}(\Gamma_{j,1,0}) & - q^{(j+1)/4}\chi^{\mf{sl}(4)}(\Gamma_{j,0,0}) \end{aligned}\right)
\end{equation}
\end{prop}

\parsec[]
As a consequence, of the above we have that $f_{AdS_4} (t_i, q) = \sum_{j\geq -2} f_j (t_i, q)$, or explicitly:
\begin{align*}
& \frac{q\left (\begin{aligned} q^{1/4}(t_1+ t_2 + t_3+t_1^{-1}t_2^{-1}t_3^{-1}) &+ q^{-1} \\- q^{-1/4}(t_1^{-1}+ t_2^{-1} + t_3^{-1}+t_1t_2t_3) &- q   \end{aligned}\right)}{(1-q)(1-q^{1/4}t_1)(1-q^{1/4}t_2)(1-q^{1/4}t_3)(1-q^{1/4}t_1^{-1}t_2^{-1}t_3^{-1})}  \\ 
& =  1 + \frac{q}{1-q} \left (\begin{aligned} & q^{-3/4}\chi^{\mf{sl}(4)}(\Gamma_{1,0,0}) - q^{-1/4}\chi^{\mf{sl}(4)}(\Gamma_{0,0,1}) \\+ & q^{1/2}\chi^{\mf{sl}(4)}(\Gamma_{0,1,0})  + q^{-1/2}\chi^{\mf{sl}(4)}(\Gamma_{2,0,0})  - q - \chi^{\mf{sl}(4)}(\Gamma_{1,0,1}) \end{aligned}\right ) \\
& + \frac{q}{1-q} \sum_{j\geq 1} \left (\begin{aligned} q^{(j-2)/4}\chi^{\mf{sl}(4)}(\Gamma_{j+2,0,0})  &+ q^{(j+2)/4}\chi^{\mf{sl}(4)}(\Gamma_{j+1,0,1}) \\ - q^{j/4}\chi^{\mf{sl}(4)}(\Gamma_{j,1,0}) & - q^{(j+1)/4}\chi^{\mf{sl}(4)}(\Gamma_{j,0,0}) \end{aligned}\right ) 
\end{align*}

In \cite[Eq. (2.15, 2.16)]{Bhattacharya:2008zy}, the index counting gravitons on $f_{AdS_4}$ is expressed as a sum of characters of irreducible representations of the 3d $\mc N = 8$ superconformal algebra that the authors call \textit{graviton representations}. In \cite[Table 9]{cordova2016multiplets} these representations are examples of short representations labeled $B_1(0)_{\Delta}^{(R_1,\cdots, R_4)}$. Here the superscripts and subscript denote the $\mf{so}(8)_R$ weights and scaling dimensions of a superconformal primary respectively. 

\surya{we should pin down what values of $\Delta, R_i$ appear for us}

Comparison with the above expansion suggests the following conjecture

\begin{conj}\label{conj:e16gravitonrep}
For $j\geq -1$, the minimal twist of the $(j+2)$-weight graviton representation in \cite[Eq. (2.15, 2.16)]{Bhattacharya:2008zy} is exactly $\mc F^{(j)}_{\R\times \C, c}(0)$. 
\end{conj}

\begin{rmk}\label{rmk:e16enhance}
This conjecture implies that the minimal twist of these graviton representations, which is a priori a module for the minimally twisted 3d $\mc N=8$ superconformal algebra $\mf{osp}(6|2)$, is in fact a module for the larger infinite dimensional super-Lie algebra $E(1|6)$. This can be thought of as analogous to the enhancement of conformal symmetries to the action of the Witt algebra of vector fields in 2d chiral conformal field theory. Such symmetry enhancements in 3 dimensions is the topic of joint work in progress with Garner and Williams.
\end{rmk}

\end{document}