%\documentclass[11pt]{amsart}
%
%%\usepackage{../macros-master}
%\usepackage{macros-fivebrane}
%
%\begin{document}



%\parsec[s:membraneobs]
%
%
%\parsec[s:fivebraneobs]
%
%Let $Z$ be a complex threefold. 
%In \ref{s:single} we have described the holomorphic twist of a theory on a single fivebrane within the (degenerate) BV formalism in terms of the (abelian) local $L_\infty$ algebra $\mc{L}_{single}$ on $Z$. 
%The factorization algebra of classical observables of the holomorphic twist of the single fivebrane theory assigns to an open set $U \subset Z$
%the cochain complex
%\[
%\clie^\bullet\left(\mc{L}_{single}(U)\right) .
%\]
%We will denote the entire factorization algebra on $Z$ of classical observables by~$\clie^\bullet(\mc{L}_{single})$.
\section{$E(1|6)$ modules from gravitons on $AdS_4\times S^7$}\label{sec:e16}

Having justified that the spaces of supergravity states constructed in the previous subsection are in fact counting gravitons on $AdS_4\times S^7$ and $AdS_7\times S^4$ respectively, we turn to studying representation theoretic properties of these state spaces. In this section, we focus on the case of gravitons on $AdS_4\times S^7$, using the description of the state space afforded by proposition \ref{prop:altstates} which describes it as the costalk of a factorization envelopes of the boundary conditions $\Omega^\bullet _\R\otimes \Omega^{0,\bullet}_\C (\mc L^{{r=0}}_{AdS_4\times S^7} )$.

We construct a certain $\C^\times$-action on the boundary fields $\Omega^\bullet _\R\otimes \Omega^{0,\bullet}_\C (\mc L^{{r=0}}_{AdS_4\times S^7} )$ equipped with the $L_\infty$ structure from remark \label{rmk:nottransferred} with the feature that the zeroth weight spaces are a local version of another exceptional linearly compact super-Lie algebras, $E(1|6)$. This in particular readily gives a decomposition of the state space $\mc H_{AdS_4\times S^7}$ into $E(1|6)$-modules. We explicitly characterize the summands of this decomposition with their module structures and give closed form expressions for their characters.

\subsection{The graviton decomposition of twisted $AdS_4\times S^7$}
\parsec{}
We consider a particular decomposition of the space of states $\mc H_{AdS_4\times S^7}$. It is induced by a decomposition of the boundary fields $\Omega^\bullet_\R\otimes \Omega^{0,\bullet}_{\C} (\mc L^{{r=0}}_{AdS_4\times S^7} )$ introduced in section \ref{sec:transversebc}  The decomposition is induced by a $\C^\times$ action on the boundary fieldsboundary fields $\Omega^\bullet_\R \otimes \Omega^{0,\bullet}_\C (\mc L^{{r=0}}_{AdS_4\times S^7} )$ that mixes fiberwise rescalings on spacetime with a fiberwise rescaling of the space of fields. 

Explicitly, the action is given as follows 
\begin{itemize}
\item On the fields 
\[
\mu(t; w_a,z) \in \C[w_1, \cdots, w_4]\{\del_{w_a}\} \otimes \Omega^\bullet_\R(I) \otimes \Omega^{0,\bullet}_{\C} (U) \oplus \C[w_1, \cdots, w_4]\otimes \Omega^\bullet_\R(I)\otimes \Omega^{0,\bullet}_{\C}(U, T)\] 
the action is
\[
\lambda \cdot \mu(t; w_a,z) = \mu(t; \lambda w_a , z).
\]
\item On the fields $\nu(t; w_a,z) \in \C[w_1, \cdots, w_4] \otimes \Omega^{\bullet}_\R(I) \otimes \Omega^{0,\bullet}_{\C}(U)$ the action is
\[
\lambda \cdot \nu(t; w_a,z) = \nu(t; \lambda w , z).
\]
\item On the fields $\beta(t; w_a,z) \in  \C[w_1, \cdots, w_4] \otimes \Omega^{\bullet}_\R(I) \otimes \Omega^{0,\bullet}_{\C}(U)$ the action is
\[
\lambda \cdot \beta(t; w_a,z) = \lambda^{-2} \beta(t; \lambda w_a , z).
\]
\item On the fields 
\[
\gamma(t; w_a,z) \in  \C[w_1, \cdots, w_4] \{\d w_a\} \otimes \Omega^{\bullet}_\R(I) \otimes\Omega^{0,\bullet}_{\C} (U) \oplus  \C[w_1, \cdots, w_4] \otimes \Omega^{\bullet}_\R(I) \otimes\Omega^{0,\bullet}_{\C}(U, \Omega^1)
\] 
the action is
\[
\lambda \cdot \gamma(t; w_a,z) = \lambda^{-2} \gamma(t; \lambda w _a, z).
\]
\end{itemize}

The following result is a straightforward if lengthy computation. We state it without proof.

\begin{prop}\label{prop:ads4decomp}
The $L_\infty$ structure on $\Pi\Omega^\bullet_\R\otimes \Omega^{0,\bullet}_\C (\mc L^{r=0}_{AdS_4\times S^7} )$ identified in section \ref{sec:transversebc} is equivariant for this $\C^\times$ action.
\end{prop}

This result induces a product decomposition 
\[
\Omega^\bullet_\R\otimes \Omega^{0,\bullet}_\C (\mc L^{r=0}_{AdS_4\times S^7} ) = \prod _{n\geq 2} \mc F^{(n)}_{\R\times \C}
\]

where for each open set $I\times U\subset\R\times \C$, we have that \[\mc F_{\R\times \C}^{(n)} (I\times U) \subset \Omega^\bullet_\R (I) \otimes \Omega^{0,\bullet}_\C (U, \mc L^{r=0}_{AdS_4\times S^7} )\] is the weight $n$ eigenspace with respect to the above $\C^\times$ action. In particular, we see that $\mc F^{(0)}_{\R\times \C}$ is itself a local dg-Lie algebra, for which  every $\mc F^{(n)}_{\R\times \C}$ is a module.

\subsection{The lowest piece: the holomorphic-topological twist of the 3d $\mc N=8$ BLG theory}\label{sec:BLG}
\parsec[] The first nontrivial case is the weight ${(-2)}$ piece. We have the following

\begin{lem}
There is a quasi-isomorphism \[\mc F^{(-2)}_{\R\times \C} \cong \Omega^\bullet_{\R\times \C}.\]
\end{lem}
\begin{proof}
The only sections which contribute are those of type $\beta$ or $\gamma$ with no form components along the fiber directions. Therefore, we see directly that \[\mc F^{(-2)}_{\R\times \C} \cong \Omega^\bullet _\R \otimes \Omega^{0,\bullet}_\C ( \mc O \xrightarrow{\del} \Omega^1 ).\] 
\end{proof}

\parsec[] The next nontrivial case is the weight ${(-1)}$ piece. 

\begin{lem}
There is a quasi-isomorphism \[ \mc F^{(-1)}_{\R\times \C} \cong \Omega^\bullet _\R \otimes \Omega^{0,\bullet}_\C \left (K^{1/4}\otimes (\C^4)^* \oplus \Pi K^{3/4}\otimes \C^4 \right ) \]
\end{lem}
\begin{proof}
On an open set of the form $I\times U$, the sections of the specified weight are:
\begin{itemize}
\item fields of type $\mu$ of the form $\mu_a(t; z) \del_{w_a}$. As the $w_a$ are fiber coordinates on $K^{1/4}_\C$, these fields transform as sections of $K^{1/4}_\C$.
\item fields of type $\gamma$ of the form $\gamma_a(t;z) \d w_a$. These fields transform as sections of $K^{3/4}_\C$. 
\end{itemize}
\end{proof}

\parsec[]
We wish to flag an appearance of $\mc F^{(-1)}_{\R\times \C}$ in supersymmetric physics in three-dimensions. There is a highly supersymmetric Chern-Simons-matter theory discovered independently by Bagger-Lambert \cite{Bagger_2007}, \cite{Bagger:2007jr} and Gustavsson \cite{Gustavsson:2007vu}. The aptly named BLG theory has $\mc N=8$ superconformal symmetry, and admits a holomorphic-topological twist that was computed by Garner in \cite{Garner2022vds}. 

The sheaf of complexes $\mc F^{(-1)}_{\R\times \C}$ matches the field contents of the holmorphic-topological twist of the BLG theory, and as such, it can be equipped with an $L_\infty$ structure under which it is perturbatively equivalent to the twisted BLG theory. In work-in-progress with Garner and Williams, we show that the action of $\mc F^{(0)}_{\R\times \C}$ on $\mc F^{(-1)}_{\R\times \C}$ in fact preserves this $L_\infty$-structure. 

\subsection{The zero-th piece: A local version of $E(1|6)$}
\parsec[] The next nontrivial case is the weight ${(0)}$ piece. This factor is special because it carries the induced structure of a local $L_\infty$ algebra on $\R\times \C$. We will prove that it is equivalent to a local Lie algebra version of the exceptional super-Lie algebra $E(1|6)$. 

We first recall the definition of this super-Lie algebra \cite{KacBible}
\begin{defn}\label{defn:e(1|6)}
Let $E(1|6)$ be the following super-Lie algebra. 
\begin{itemize}
\item The even part of $E(1|6)_0$ given by the semidirect product Lie algebra $\Gamma (\hat D, T) \ltimes \left( \Gamma (\hat D, \mc O) \otimes \mf{sl}(4) \right )$
\item The odd part $E(1|6)_1$ is given by the (unique) nontrivial extension of $E(1|6)_0$-modules 
\[0\to \Sym^2 (\C^4)\otimes K^{1/2}_\C \to E(1|6)_1 \to \wedge^2 (\C^4) \otimes K^{-1/2}_\C \to 0.\]

\end{itemize}
The only remaining bracket to be specified, the odd bracket, is given as follows.
\begin{itemize}
\item Given sections $A \otimes f \d z^{1/2}\in \Sym^2 (\C^4)\otimes K^{1/2}_\C$ and $B\otimes g \del_z^{1/2}\in \wedge^2 (\C^4)\otimes K^{-1/2}_\C$, we have that
\[
[A \otimes f \d z^{1/2}, B\otimes g \del_z^{1/2}] = A*B \otimes fg \in \mf{sl}(4)\otimes \mc O.
\]
Here, $*$ refers to the hodge star of $B$ and we are viewing $A$ and $*B$ as symmetric and skew-symmetric $4\times 4$ matrices respectively; their product is traceless. 

\item Given sections $A \otimes f \del_z^{1/2}, B\otimes g \del_z^{1/2} \in \Gamma( \hat D, \wedge^2 (\C^4) \otimes K^{1/2}_\C)$, we have that 
\begin{align*}
[A\otimes f \d z^{-1/2} , B \otimes g \d z^{-1/2} ] & = \tr (A * B) \otimes fg \del_z + \frac12 (A*B)_0 \otimes \left (\del (f \d z^{-1/2} ) g \d z^{-1/2} + f \d z^{-1/2} \del (g \d z^{-1/2} ) \right ) \\
& \in \Gamma (\hat D, T) \ltimes \left (\mf {sl}(4) \otimes \Gamma (\hat D, \mc O) \right ) .
\end{align*}
where again $*$ denotes the Hodge star and the subscript of zero denotes projection to the traceless part. 
\end{itemize}
\end{defn}

The relationship between this super-Lie algebra and our decomposition is established through the following result.

\begin{prop}
There is an equivalence of super-Lie algebras
\[
\mc F^{(0)}_{\R\times \C, c}  (0) \cong E(1|6).
\]
\end{prop}
\begin{proof}
We will begin by trying to characterize the local $L_\infty$-algebra $\mc F^{(0)}_{\R\times \C}$. We claim that it is quasi-isomorphic to a local version of $E(1|6)$. 

Indeed, it is easy to see that the weight zero sections consists of the following cochain complex 
\begin{equation}
\Omega^\bullet _\R \otimes \Omega^{0,\bullet}_\C \left (
\begin{tikzcd}
\ul{even} & \ul{odd} \\
\C[w_a\del_{w_b}]\otimes \mc O\ar[r, "\del^W_\Omega"]  & \mc O \\ 
\T \ar[ur] & \\
\Sym^2 (\C^4)\ar[r, "\del_W"]\ar[dr] & \C[w_adw_b] \otimes K^{-1/2}\ \\
& \Sym^2 (\C^4)\otimes \Omega^1\otimes K^{-1/2}
\end{tikzcd} \right)
\end{equation}
Of course, the differentials are just appropriate components of the divergence operator and holomorphic deRham operator. We can compute cohomology by way of a spectral sequence whose first page is the cohomology with respect to $\del^W_\Omega + \del_W$. We see that the differential $\del^W$ maps surjectively onto functions and its kernel is isomorphic to $\mf{sl}(4)\otimes \mc O$. Likewise, the differential $\del_W$ is the canonical inclusion of $\mf{sl}(4)$ representations $\Sym^2 (\C^4)\to \C^4\otimes \C^4$. Its cokernel is a copy of $\wedge^2 \C^4$. 

Thus, we see that this page of the spectral sequence is given by
\begin{equation}
\mc E(1|6) \define \Omega^\bullet _\R \otimes \Omega^{0,\bullet}_\C \left (
\begin{tikzcd}
\ul{even} & \ul{odd} \\
\T & \wedge^2 (\C^4) \otimes K^{-1/2} \\
\mf {sl}(4)\otimes \mc O & \Sym^2 (\C^4)\otimes K^{1/2}
\end{tikzcd} \right)
\end{equation}
and there are no non-zero differentials so the spectral sequence degenerates.

To see that the Lie structure induced from the $L_\infty$-structure on $\Omega^\bullet_\R\otimes \Omega^{0,\bullet}_\C (\mc L^{r=0}_{AdS_4\times S^7} )$ is in fact given by the same formulae as the brackets on $E(1|6)$ in equation \ref{defn:e(1|6)}, it will be useful to provide an explicit quasi-isomorphism $\Psi^{(0)} : \mc E(1|6) \to \mc F^{(0)}_{\R\times \C}$. On an open set $I\times D\subset \R\times \C$, this is defined as follows

\begin{itemize}
\item Given a section $g(t; z) \del_z \in \Omega^\bullet_\R (I)\otimes \Omega^{0,\bullet}_\C (D, T)$ where $g(t;z)$ is a mixed deRham-Dolbeault form on $I \times D$, we define
\begin{align*}
\Psi^{(0)} (g(t;z)\del_z) &= g(t;z)\del_z - \frac 14 (\del_z g(t; z)) w_a\del_{w_a} \\
&\in \Omega^\bullet_\R (I) \otimes \Omega^{0,\bullet}_\C \left ( D, T \oplus \C\{w_a\del _{w_b} \} \right )
\end{align*}
\item Given a section $A_{ab} \otimes g(t; z) \in \Omega^\bullet_\R (I)\otimes \Omega^{0,\bullet}_\C \left (D, \mf{sl}(4)\otimes \mc O \right )$ where $g(t;z)$ is a mixed deRham-Dolbeault form on $I \times D$ and $A_{ab}\in \mf {sl}(4)$
\begin{align*}
\Psi^{(0)} (A_{ab} \otimes g(t;z)) &= g(t;z)A_{ab}w_a\del_{w_b} \\
&\in \Omega^\bullet_\R (I) \otimes \Omega^{0,\bullet}_\C \left ( D, \C\{w_a\del _{w_b}\} \right )
\end{align*}

\item Given a section $A_{ab} \otimes g(t; z)\d z^{-1/2} \in \Omega^\bullet_\R (I)\otimes \Omega^{0,\bullet}_\C \left (D, \wedge^2 (\C^4) \otimes K^{-1/2}\right )$
where $g(t;z)$ is a mixed deRham-Dolbeault form on $I\times D$ and $A_{ab} \in \wedge^2 (\C^4)$ we define
\begin{align*}
\Psi^{(0)} (A_{ab} \otimes g(t; z)\d z^{-1/2}) &= g(t;z)  A_{ab} w_a \del_{w_b} \\
&\in \Omega^\bullet_\R (I) \otimes \Omega^{0,\bullet}_\C \left ( D, \C\{w_a\del _{w_b} \}\otimes K^{-1/2} \right )
\end{align*}


\item Given a section 
$A_{ab} \otimes g(t; z)\d z^{1/2} \in \Omega^\bullet_\R (I)\otimes \Omega^{0,\bullet}_\C \left (D, \Sym^2 (\C^4) \otimes K^{1/2}\right )$ where $g(t;z)$ is a mixed deRham-Dolbeault form on $I\times D$ and $A_{ab} \in \Sym^2 (\C^4)$ we define
\begin{align*}
\Psi^{(0)} (A_{ab} \otimes g(t; z)\d z^{1/2}) &= g(t;z)  A_{ab} w_a w_b \d z \\
&\in \Omega^\bullet_\R (I) \otimes \Omega^{0,\bullet}_\C \left ( D, \Sym^2 (\C^4) \otimes \Omega^1 \otimes K^{-1/2} \right )
\end{align*}
\end{itemize}
It is easy to see that $\Psi^{(0)}$ is a quasi-isomorphism and a straightforward if lengthy check confirms that it preserves Lie brackets. The result then follows from computing the limit of $\mc E(1|6)_c (I\times D)$ over open sets containing the origin. 
\end{proof}

\begin{rmk}
We note that the map $i_{M2}$ from lemma \ref{lem:m2emb} in fact defines a Lie map from $\mf{osp}(6|2)$ to the sections of the boundary condition $\Pi \Omega^\bullet_\R\otimes \Omega^{0,\bullet}_{\C} (\mc L^{r=0}_{AdS_4\times S^7} )$ over every open set containing the origin. The image of the map lands exactly in the step $\mc F^{(0)}_{\R\times \C}$ of the decomposition from proposition \ref{prop:ads4decomp}. Therefore we see that $E(1|6)$ contains $\mf{osp}(6|2)$ as a finite dimensional subalgebra.  
\end{rmk}

\subsection{General summands and $E(1|6)$-modules}
We now move on to giving an explicit description of the general summand $\mc F^{(j)}$ for $j \geq 1$. 

We first fix some notation for irreducible highest weight representations of $\mf{sl}(4)$. Let $\mf {h}\subset \mf{sl}$ be the Cartan given by diagonal matrices and let $L_i\in \mf {h}^*$ be the linear functional that picks out the $i$-th diagonal entry. We may accordingly write $\mf h^* = \C \{L_1, L_2, L_3, L_4\}/(L_1+\cdots + L_4)$. We will write $\Gamma_{a_1,a_2, a_3}$ for the irreducible representation of $\mf {sl}(4)$ of highest weight $(a_1+a_2+a_3)L_1+(a_2+a_3)L_2 + a_3L_3$. 

\begin{prop}
Let $j\geq 1$. The complex of vector bundles $\mc F^{(j)}_{\R\times \C}$ is quasi-isomorphic to 
\begin{equation}
\Omega^\bullet_\R\otimes \Omega^{0,\bullet}_\C \left (
\begin{tikzcd}
\ul{even} & \ul{odd} \\
\Gamma_{j,1,0} \otimes K^{-j/4} & \Sym^{j+2}(\C^4) \otimes \Omega^1 \otimes K^{-(j+2)/4} \\
\Sym^j (\C^4) \otimes \T \otimes K^{-j/4} & \Gamma_{j+1, 0, 1} \otimes K^{-(j+2)/4}
\end{tikzcd} \right )
\end{equation}


\end{prop}
\begin{proof}
We begin by noting that we can explicitly describe $\mc F_{\R\times \C}^{(j)}$ as $\Omega^\bullet _\R \otimes \Omega^{0,\bullet}_\C (F^{(j)})$ where $F^{(j)}$ denotes the following dg-vector bundle:
\begin{equation}
\begin{tikzcd}
\ul{even} & \ul{odd} \\
\Sym^{j+1} (\C^4) \otimes (\C^4)^* \otimes K^{-j/4} \ar[dr, "\del^W_\Omega"] \\ & \Sym^j (\C^4) \otimes K^{-j/4} \\ 
\Sym^j (\C^4) \otimes \T \otimes K^{-j/4} \ar[ur, "\del^V_\Omega"'] & \\
& \Sym^{j+2}(\C^4) \otimes \T^*\otimes K^{-(j+2)/4}\\ 
\Sym^{j+2} (\C^4) \otimes K^{-(j+2)/4}\ar[ur, "\del_V"] \ar[dr, "\del_W"'] \\
& K^{-(j+2)/4}\otimes \Sym^{j+1}(\C^4) \otimes \C^4 . 
\end{tikzcd}
\end{equation}

Note that the differentials here are all $\mf{sl}(4)$ equivariant maps, tensored with a differential operator acting on sections of a bundle on $\C$. In particular

\begin{itemize}
\item The differential $\del^W_\Omega$ involves the canonical projection 
\[\Sym^{j+1} (\C^4)\otimes (\C^4)^* \to \Sym^j (\C^4).\] Its kernel is precisely the irreducible highest weight representation $\Gamma_{j+1, 0 ,1}$.

\item The differential $\del_W$ is the canonical inclusion 
\[\Sym^{j+2} (\C^4) \to \Sym^{j+1}(\C^4) \otimes \C^4.\] Its cokernel is the irreducible highest weight representation $\Gamma_{j, 1, 0}$.
\end{itemize}

We can compute the cohomology using a spectral sequence whose first page is given by the cohomology with respect to $\del^W_\Omega + \del_W$. There are no further differentials on this page so the result follows. 
\end{proof}

\subsection{Characters of $E(1|6)$-modules}
Note that the decomposition of the state space $\Sym (\mc H_{AdS_4\times S^7} ) = \prod_{j\geq -2} \mc U (\mc F^{(j)} _{\R\times \C} )(0)$ gives a product formula for the characters computed in proposition \ref{} 

\[\chi \left (\Sym (\mc H_{AdS_4\times S^7} ) \right )= \prod_{j\geq -2} \chi \left ( \mc U (\mc F^{(j)} _{\R\times \C} )(0)\right ).\]

We end this section by computing each of the characters $\chi \left ( \mc U (\mc F^{(j)} _{\R\times \C} )(0)\right )$. We will express our characters in terms of characters of highest weight respresentations of $\mf{sl(4)}$ which we denote $\chi^{\mf{sl}(4)}(\Gamma_{a_1,a_2, a_3} )$.

\parsec[]
From the characterization in \ref{}, the lowest step of the decomposition $\mc F^{(-2)}$ is just given by the deRham complex on $\R\times \C$, and accordingly the character of $\mc F^{(-2)}_{c}(0)$ is the constant function 1.

\parsec[]
We proceed to the next step of the decomposition, using the characterization in \ref{}.

\begin{prop}
The character $\chi \left ( \mc U(\mc F_{\R\times \C}^{(-1)})(0)\right )$ is given by the plethystic exponential of the following expression:
\begin{equation}
f_{-1}(t_1, t_2, t_3, q) = \frac{q\left (q^{-3/4}(t_1+ t_2+t_3 + t_1^{-1} t_2^{-1} t_3^{-1} )-q^{-1/4}(t_1^{-1} + t_2^{-1}+t_3^{-1} + t_1t_2t_3)\right )}{(1-q)}
\end{equation}
\end{prop}
\begin{proof}
The proof proceeds by the same trick as in the proof of proposition \ref{prop:altstates}. To describe the costalk, we wish to compute a limit of sections of $\mc F^{(-1)}_{\R\times \C,c}$ on open sets of the form $I\times D$ containing the origin in $\R\times \C$. Using ellipticity, we can describe such sections as a module over the ring generated by holomorphic derivatives of the delta function. 

Accordingly, we have contributions from the following summands:
\begin{itemize}
\item An even copy of $\C^4\otimes \C\{\d z^{3/4}\} \otimes \C[\del_z]\delta_{z=0}$. The character of this summand is
\[
\frac{q\left (q^{-3/4}\chi^{\mf{sl}(4)}(\Gamma_{1,0,0}) \right )}{(1-q)} =\frac{q\left (q^{-3/4}(t_1+ t_2+t_3 + t_1^{-1} t_2^{-1} t_3^{-1} )\right )}{(1-q)}
\]
\item An odd copy of $\C^4\otimes \C\{\d z^{1/4}\} \otimes \C[\del_z]\delta_{z=0}$. The character of this summand is
\[
\frac{-q\left (q^{-1/4}\chi^{\mf{sl}(4)}(\Gamma_{0,0,1}) \right )}{(1-q)} =\frac{-q\left (q^{-3/4}(t_1^{-1} + t_2^{-1} +t_3^{-1}  + t_1 t_2 t_3)\right )}{(1-q)}
\]
\end{itemize}
\end{proof}

Note that under the change of fugacities in \ref{}, this matches exactly with the single particle index for the theory on a single M2 brane \cite[Eq. (2.32)]{Bhattacharya:2008zy}.

\parsec[]
We continue to the next step of the decomposition given by $\mc{F}^{(0)}_{\R\times \C}$. 

Arguing similarly as in the proof of the previous proposition, we have the following.
\begin{prop}
The character $\chi \left ( \mc U (\mc{F}^{(0)}_{\R\times \C})(0)\right )$ is given by the plethystic exponential of the following expression:
\begin{equation}
f_0(t_1, t_2, t_3, q) = \frac{q}{(1-q)}\left ( q^{1/2}\chi^{\mf{sl}(4)}(\Gamma_{0,1,0})  + q^{-1/2}\chi^{\mf{sl}(4)}(\Gamma_{2,0,0})  - q - \chi^{\mf{sl}(4)}(\Gamma_{1,0,1}) \right)
\end{equation}
\end{prop}

\parsec[]
Finally, we continue to the general step of the decomposition.

\begin{prop}
Let $j\geq 1$. The character $\chi \left ( \mc U (\mc F^{(j)}_{\R\times \C} ) (0)\right )$ is the plethystic exponential of the following expression:
\begin{equation}
f_j(t_1, t_2, t_3, q) = \frac{q}{(1-q)}\left (\begin{aligned} q^{(j-2)/4}\chi^{\mf{sl}(4)}(\Gamma_{j+2,0,0})  &+ q^{(j+2)/4}\chi^{\mf{sl}(4)}(\Gamma_{j+1,0,1}) \\ - q^{j/4}\chi^{\mf{sl}(4)}(\Gamma_{j,1,0}) & - q^{(j+1)/4}\chi^{\mf{sl}(4)}(\Gamma_{j,0,0}) \end{aligned}\right)
\end{equation}
\end{prop}

\parsec[]
As a consequence, of the above we have that $f_{AdS_4\times S^7} (t_i, q) = \sum_{j\geq -2} f_j (t_i, q)$, or explicitly:
\begin{align*}
& \frac{q\left (\begin{aligned} q^{1/4}(t_1+ t_2 + t_3+t_1^{-1}t_2^{-1}t_3^{-1}) &+ q^{-1} \\- q^{-1/4}(t_1^{-1}+ t_2^{-1} + t_3^{-1}+t_1t_2t_3) &- q   \end{aligned}\right)}{(1-q)(1-q^{1/4}t_1)(1-q^{1/4}t_2)(1-q^{1/4}t_3)(1-q^{1/4}t_1^{-1}t_2^{-1}t_3^{-1})}  \\ 
& =  1 + \frac{q}{1-q} \left (\begin{aligned} & q^{-3/4}\chi^{\mf{sl}(4)}(\Gamma_{1,0,0}) - q^{-1/4}\chi^{\mf{sl}(4)}(\Gamma_{0,0,1}) \\+ & q^{1/2}\chi^{\mf{sl}(4)}(\Gamma_{0,1,0})  + q^{-1/2}\chi^{\mf{sl}(4)}(\Gamma_{2,0,0})  - q - \chi^{\mf{sl}(4)}(\Gamma_{1,0,1}) \end{aligned}\right ) \\
& + \frac{q}{1-q} \sum_{j\geq 1} \left (\begin{aligned} q^{(j-2)/4}\chi^{\mf{sl}(4)}(\Gamma_{j+2,0,0})  &+ q^{(j+2)/4}\chi^{\mf{sl}(4)}(\Gamma_{j+1,0,1}) \\ - q^{j/4}\chi^{\mf{sl}(4)}(\Gamma_{j,1,0}) & - q^{(j+1)/4}\chi^{\mf{sl}(4)}(\Gamma_{j,0,0}) \end{aligned}\right ) 
\end{align*}

In \cite[Eq. (2.15, 2.16)]{Bhattacharya:2008zy}, the index counting gravitons on $f_{AdS_4\times S^7}$ is expressed as a sum of characters of irreducible representations of the 3d $\mc N = 8$ superconformal algebra that the authors call \textit{graviton representations}. Comparison with the above expansion suggests the following conjecture

\begin{conj}\label{conj:e16gravitonrep}
For $j\geq -1$, the minimal twist of the $j+2$nd graviton representation in \cite[Eq. (2.15, 2.16)]{Bhattacharya:2008zy} is exactly $\mc F^{(j)}_{\R\times \C, c}(0)$. 
\end{conj}

\begin{rmk}\label{rmk:e16enhance}
This conjecture implies that the minimal twist of these graviton representations, which is a priori a module for the minimally twisted 3d $\mc N=8$ superconformal algebra $\mf{osp}(6|2)$, is in fact a module for the larger infinite dimensional super-Lie algebra $E(1|6)$. This can be thought of as analogous to the enhancement of conformal symmetries to the action of the Witt algebra of vector fields in 2d chiral conformal field theory. Such symmetry enhancements in 3 dimensions is the topic of joint work in progress with Garner and Williams.
\end{rmk}