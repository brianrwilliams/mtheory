\section{Conjectures for indices of operators on fivebranes}
\label{s:finite}

In conjecture \ref{conj:ops} we have formulated a conjectural description of the space of local operators $\Obs_{N}(0)$ associated to the worldvolume theory on a stack of $N$ fivebranes in the holomorphic twist.
As we reviewed just in the previous section, the space of local operators is what categorifies the superconformal index that we study in this paper.
In this section we begin to provide some evidence for this description at the level of characters.
%Of course, this is just a small piece of the full algebra structure present in the local operators.
%The structure of a three-dimensional holomorphic factorization algebra induces algebraic operations on the algebra of local operators


For a stack of $N=1$ fivebranes, which corresponds to the abelian six-dimensional superconformal field theory, we find that our local character matches exactly with the expressions in the literature. 
This is not a surprise as we have shown that even at the level of factorization algebras $\clie_\bullet(\mc{G}_{1,c})$ is quasi-isomorphic to the classical limit of~$\Obs_1$, see Proposition~\ref{prop:factabelian}.

The main computation of this section is a closed formula for the local character of the factorization algebra $\clie_\bullet(\mc{G}_{N,c})$ for $N > 1$, see Theorem \ref{thm:finite}. 
Following conjecture \ref{conj:ops} and the general discussion of \S \ref{sec:sucaindex} we are led to hypothesize a closed formula for the superconformal index for the theory on a finite number of fivebranes (in flat space).
As far as the authors are aware of there is no closed formula for the refined superconformal index (with four independent fugacities) for the theory on a stack of $N > 1$ fivebranes.
For small values of $N$ we expand our closed formulas to low orders in the fugacity $q$ (which roughly counts instanton charge) to match exactly with expressions in the literature. 

\subsection{Operators on a single fivebrane}


In Proposition \ref{lem:single} we have shown that $\clie_\bullet (\mc{G}_{c}^{(-1)})$ is equivalent to the factorization algebra $\Obs^{cl}_{1}$ encoding the classical observables  of the holomorphic twist on a single fivebrane.
On $\C^3$, the global sections of the local Lie algebra $\mc{G}^{(0)}$ is closely related to $E(3|6)$---the $\infty$-jets of $\mc{G}^{(0)}$ at $0 \in \C^3$ is quasi-isomorphic to $E(3|6)$.
Combining these facts we see that $\Obs_{1}(0)$ is a module for $E(3|6)$. 
This module turns out to be a one-dimensional extension of an irreducible $E(3|6)$-module which was classified in~\cite{KR2}. 

%and compare our expression for the character to the character of a certain irreducible module for the exceptional super Lie algebra $E(3|6)$ considered in~\cite{KR2}.

%\parsec
%
%There are various degenerations, or specializations, of this character which are interesting to consider.
%These specializations involve restricting the character above to a subalgebra of the full Cartan that we considered above.
%
%One degeneration of this character involves specializing $t_1=t_2=1$ which results in the $U(1) \times SU(2)$ character:
%\begin{equation}
%f_{1}(r,q) = \frac{(r+r^{-1})q^{3/2} - 3 q^{2} + q^3}{(1-q)^3} .
%\end{equation}
%They compute the absolute (non-super) character of the module $I(0,0;1;-1)$ where they additionally specialize $t_1=t_2=r=1$. 
%In a similar method to the one used in \cite{KR1}, one can compute the specialized (super) character of $I(0,0;1;-1)$ to find
%\[
%\chi_{u(1)} (q,t_1=t_2=r=1) = \frac{2 q^{3/2} - 3 q^2 + q^3}{(1-q)^3} .
%\]

\parsec
There are various degenerations, or specializations, of this character which are interesting to consider.
A particularly meaningful one is related to two different deformations of the theory by elements in the (twisted) superconformal algebra and is known as the Schur limit of the index.

Recall that after performing the holomorphic twist the residual superconformal algebra is~$\mf{osp}(6|2)$.
We have recalled in \S\ref{s:global1} how the bosonic part of this algebra is represented by fields of the eleven-dimensional theory. 
There are two types of odd elements of~$\mf{osp}(6|2)$ that also have a natural interpretation in the eleven-dimensional theory.
The odd part of~$\mf{osp}(6|2)$ can be identified with the twelve-dimensional space
\[
\C^3 \otimes \C^2 \oplus \wedge^2(\C^3) \otimes \C^2 
\]
where $\C^3, \C^2$ are the fundamental $\mf{sl}(3)$ and $\mf{sl}(2)$ representations, respectively. 
The $\mf{gl}(1)$ factor in the bosonic part of $\mf{osp}(6|2)$ acts with weight $1/2$ on both summands. 

\begin{itemize}
\item The summand $\C^3 \otimes \C^2$ embeds into the ghosts of twisted supergravity via the $\gamma$-type fields which satisfy
\[
\del \gamma = \d w_a \d z_i .
\]
where $i=1,2,3$ and $a = 1,2$.
Note that $\gamma$ appears to be ambiguous up to a closed holomorphic one-form, but since there is a linear gauge symmetry which sends $\beta \mapsto \del \beta$, it implies that $\gamma$ is unique up to a BRST exact term. 
Since in our model all closed one-forms are rendered trivial in cohomology
\item The summand $\wedge^2(\C^3) \otimes \C^2$ embeds as another $\gamma$-type field which satisfies 
\[
\del \gamma = w_a \d z_i \d z_j .
\]
\end{itemize}

Both deformations break the global Cartan subalgebra down to $\mf{gl}(1) \times \mf{gl}(1)$ according to the specializations
\begin{equation}\label{eqn:special1}
y=1 , \quad y_3 = 1 .
\end{equation}
Notice that due to the constraint $y_1y_2y_3=1$ this forces $y_1 = y_2^{-1}$.
As one can easily check, this specialization yields the following single particle index
\[
f_{1}(y_1, y_1^{-1},y_3=1, y=1, q) = \frac{q}{1-q} 
\]
which recovers the single particle index of a single chiral boson on the Riemann surface $\Sigma = \C_{z_1}$. 
Notice that the dependence on the parameter $y_1$ has completely dropped out even though we have not specialized it to any value.
%Notice that although the Cartan subalgebra generated by the vector field $z_1 \del_{z_1} - z_2 \del_{z_2}$ is unbroken by this deformation, the dependence on its fugacity $t_1$ completely drops out of the expression.

\subsection{A conjectural description of operators on a stack of two fivebranes}

In \S\ref{sec:factsummary} we saw that the decomposition of the local $L_\infty$ algebra $\mc{G} = \mc{G}_Z$ on $Z$ induces a filtration of the factorization algebra $\clie_\bullet(\mc{G}_c)$. 
\[
\clie_{\bullet}(\mc{G}_{1,c}) \subset \clie_{\bullet}(\mc{G}_{2,c}) \subset \cdots .
\]
We now turn to the factorization algebra $\clie_{\bullet}(\mc{G}_{2,c})$.

Recall that $\mc{G}_{2}$ is the local $L_\infty$ algebra on $Z$ defined as $\mc{G}_{2} = \mc{G} / \mc{G}^{\geq 1}$. 
Since $\mc{G}$ is concentrated in weights $\geq -1$ we see that $\tilde {\mc{G}}_{2}$ is of the form
\[
\mc{G}_2 = \tilde {\mc{G}}_2 \ltimes \mc{G}_1 
\]
where $\mc{G}_1 = \mc{G}^{(-1)}$ is the weight $(-1)$ piece and $\tilde {\mc{G}}_2 = \mc{G}^{\geq 0} / \mc{G}^{\geq 1} = \mc{G}^{(0)}$.  
We focus mostly on the factorization algebra $\clie_\bullet(\tilde {\mc{G}}_{2,c})$.

We have already characterized the local dg Lie algebra $\tilde {\mc{G}}_{2} = \mc{G}^{(0)}$ as the weight zero part of $\mc{G}$ on on any threefold $Z$ in \S\ref{s:weight0}. 
We have also shown that $\mc{G}^{(0)}$ is equivalent to the local Lie algebra $\mc{E}(3|6)$. 
The even part of $\mc{E}(3|6)$ is
\[
\Omega^{0,\bullet}(Z, \T_Z) \oplus \Omega^{0,\bullet}(Z) \otimes \mf{sl}(2) 
\]
with its natural cohomological grading by Dolbeault form type. 
The odd part of $\mc{E}(3|6)$ is
\[
\Omega^{1,\bullet}(Z, K_Z^{-1/2}) \otimes \C^2 .
\]
The differential is $\dbar$ and the Lie bracket has been described in \S\ref{s:weight0}.

%On $Z = \C^3$ this local dg Lie algebra is related to the exceptional simple super Lie algebra $E(3|6)$ classified by Kac \cite{KacClass}. 
%Indeed, one can show (see the forthcoming work \cite{SW6d}) that the fiber of the $\infty$-jet bundle of $\mc{G}_2$ at $0 \in \C^3$ is quasi-isomorphic to $E(3|6)$. 

\parsec

We continue by computing the character of local operators associated to the factorization algebra $\clie_\bullet(\mc{G}_{2,c})$ using Lemma~\ref{lem:envelope}.
For simplicity we will use the fugacities $y_i, y, q$.



Combining these expressions we obtain the following.

\begin{prop} \label{prop:6dtwo}
The character of local operators of the factorization algebra $\clie_\bullet(\tilde {\mc{G}}_{2,c})$ on $\C^3$ is given by the plethystic exponential of the following expression
\begin{equation}\label{eqn:6dtwo}
\tilde f_{2} (y_i,y,q) = \frac{q^4(y_1+y_2+y_3) + q^2 (y^2 + q + q^2 y^{-2}) - q^{3} (y + q y^{-1})(y_1^{-1} + y_2^{-1} + y_3^{-1})}{(1-y_1q) (1-y_2 q) (1-y_3 q)}.
\end{equation}
%\begin{equation}\label{eqn:6dtwo}
%f_{2} (t_1,t_2,r,q) = \frac{q^4(t_1^{-1} + t_1 t_2^{-1}  + t_2) + q^3 (r^2 + r^{-2} + 1) - q^{7/2} (r + r^{-1})(t_1 + t_1^{-1} t_2 + t_2^{-1})}{(1-t_1^{-1}q) (1-t_1 t_2^{-1} q) (1-t_2 q)} .
%\end{equation}
\end{prop}

Recall that our conjecture for the space of local operators associated to the holomorphic twist of the six-dimensional worldvolume theory on a stack of two fivebranes is $\Obs_2 (0) \simeq \clie_\bullet(\mc{G}_{2,c})(0) \mc{O}ng \clie_\bullet(\mc{G}_{1,c})(0) \otimes \clie_\bullet(\tilde {\mc{G}}_{2,c})(0)$. 
And after removing the center of mass degrees of freedom, our conjecture is $\tilde \Obs_2 \simeq \clie_\bullet(\tilde {\mc{G}}_{2,c})$.

Just as in the abelian case, the local operators $\tilde \Obs_2(0)$ form a module over $E(3|6)$.
It turns out that this module is irreducible~\cite{KR2}.

We can now state a decategorified version of conjecture \ref{conj:ops} at the level of superconformal indices, or local characters.

\begin{conj}\label{conj:6dtwo}
The superconformal index of the six-dimensional superconformal theory of type $A_1$ is given by
\[
\tilde \chi_{2} (y_i,y,q) = {\rm PExp} \left[\tilde f_2(y_i,y,q) \right] .
\]
where $\tilde f_2(y_i,y,q)$ is as in \eqref{eqn:6dtwo}.
\end{conj}

Similarly, the index associated to the $\mf{gl}(2)$ theory, which is the local character of $\clie_\bullet(\mc{G}_{2,c})$, is conjectured to be simply the product 
\[
\chi_{2} (y_i,y,q) = \chi_{2} (y_i,y,q) \cdot \chi_{1}(y_i,y,q)
\]
where the character $\chi_{1}$ for the $\mf{gl}(1)$ theory is given in proposition~\ref{prop:6done}
Equivalently, $\chi_2$ is the plethystic exponential of $f_2 = f_1 + \tilde f_2$. 

%\parsec[]
%
%The specialization of this index $t_1=t_2=r=1$ yields the single particle index
%\[
%\frac{3q^4 + 3 q^3 - 6 q^{7/2}}{(1-q)^3}. 
%\]

\parsec[]

The Schur limit $y=1, y_3=1$ of $\tilde f_2$ in \eqref{eqn:special1} yields 
\[
\tilde f_{2}(y_1, y_2, y_3=1, y=1, q) = \frac{q^2}{1-q} 
\]
which is the single particle index of Virasoro vacuum module on the Riemann surface $\Sigma = \C_{z_1}$. 

\subsection{A closed formula for the finite $N$ index}

Before exhibiting the general formula for the local character of the factorization algebra $\clie_\bullet(\mc{G}_{N,c})$ on $\C^3$ we set up some notation. 
As above, we let $\chi_k^{\mf{sl}(2)}$ and $\chi^{\mf{sl}(3)}_{[k,l]}$ denote the highest weight $\mf{sl}(2)$ and $\mf{sl}(3)$ characters. 
We also define the following expression which appears in the denominator in all of our characters
\begin{equation}
d(y_i,y,q) = (1-y_1 q)(1-y_2q)(1-y_3q) .
\end{equation} 
To simplify formulas, we will temporarily denote the single particle character for the $N=1$ theory $\Obs_1$ by 
\begin{equation}
g_{-1} (y_i,y,q) = f_1(y_i,y,q)
\end{equation}
where $f_1(y_i,y,q)$ is as in equation \eqref{eqn:6done1} and also denote by 
\begin{equation}
g_0 (y_i,y,q) = \tilde f_2(y_i,y,q)
\end{equation}
where $\tilde f_2(y_i,y,q)$ is as in equation \eqref{eqn:6dtwo}. 
Thus $g_2$ is the single particle local character of $\clie_\bullet(\tilde {\mc{G}}_{2,c}) = \clie_{\bullet}(\mc{G}_c^{(0)})$.
Finally, for $k \geq 1$ let
%\begin{align*}
%f_k (y_1,y_2,y_3,y,q) & \define q^{3k/2} \left(q \chi^{\mf{sl}(2)}_{k-2}(q^{-1/2} y)(y_1 + y_2 + y_3) + \chi^{\mf{sl}(2)}_k(q^{-1/2} y) \right. \\
%& \left.  - q \chi^{\mf{sl}(2)}_{k-3}(q^{-1/2}y) - \chi^{\mf{sl}(2)}_{k-1} (q^{-1/2} y) (y_1^{-1} + y_2^{-1} + y_3^{-1} ) \right) .
%\end{align*}
%\begin{align*}
%g_k (y_i,y,q) & \define q^{3} \left(q^{1 + 3 (k-2)/2} \chi^{\mf{sl}(2)}_{k-2}(q^{-1/2} y)(y_1 + y_2 + y_3) + q^{3(k-2)/2} \chi^{\mf{sl}(2)}_k(q^{-1/2} y) \right. \\
%& \frac{\left.  - q^{3(k-1)/2} \chi^{\mf{sl}(2)}_{k-3}(q^{-1/2}y) - q^{-1 + 3(k-1)/2} \chi^{\mf{sl}(2)}_{k-1} (q^{-1/2} y) (y_1^{-1} + y_2^{-1} + y_3^{-1} ) \right)}{d(y_i,y,q)} .
%\end{align*}
\begin{equation}
\label{eqn:gk}
\begin{array}{lllll}
g_k (y_i,y,q) \define & q^{3} \left(q^{1 + 3 k/2} \chi^{\mf{sl}(2)}_{k}(q^{-1/2} y)\chi^{\mf{sl}(3)}_{[1,0]}(y_i) + q^{3k/2} \chi^{\mf{sl}(2)}_{k+2}(q^{-1/2} y) \right. \\
&\displaystyle \frac{\left.  - q^{3(k+1)/2} \chi^{\mf{sl}(2)}_{k-1}(q^{-1/2}y) - q^{-1 + 3(k+1)/2} \chi^{\mf{sl}(2)}_{k+1} (q^{-1/2} y) \chi^{\mf{sl}(3)}_{[0,1]}(y_i) \right)}{d(y_i,y,q)} .
\end{array}
\end{equation}
%and hence the conjectural single particle index for the superconformal theory associated to the Lie algebra $\mf{sl}(2)$. 

\begin{thm}
\label{thm:finite}
Let $N \geq 3$. 
The local character of the factorization algebra $\clie_{\bullet}(\mc{G}_{N,c})$ is
\begin{equation}
\chi_{N}(y_1,y_2,y_3,y,q) = \text{PExp}\left[\sum_{k=-1}^{N-2} g_k(y_1,y_2,y_3,y,q)\right].
\end{equation}
Similarly, the local character of the factorization algebra $\clie_\bullet(\tilde{\mc{G}}_{N,c})$ is 
\begin{equation}
\tilde{\chi}_{N}(y_1,y_2,y_3,y,q) = \text{PExp}\left[\sum_{k=0}^{N-2} g_k(y_1,y_2,y_3,y,q)\right].
\end{equation}
\end{thm}
\begin{proof}
By Lemma~\ref{lem:envelope} the character of $\clie_\bullet (\mc{G}_{N,c})$ is given by 
\begin{equation}
\chi_N = \text{PExp} \left[f_N\right]
\end{equation}
where $f_N$ is the single particle local character.
Thus, it suffices to show that $f_N = \sum_{k = -1}^{N-2} g_k$.
Recall that from the description \eqref{eqn:gN} we have, as local Lie algebras:
\begin{equation}
\mc{G}_N = \mc{G} / \mc{G}^{(\geq N-2)} ,
\end{equation} 
for $N \geq 1$. 
In particular, as a super vector bundle on the threefold $Z = \C^3$ we have
\[
\mc{G}_N = \mc{G}^{(-1)} \oplus \mc{G}^{(0)} \oplus \cdots \oplus \mc{G}^{(N-2)} .
\]
%\begin{equation}
%\clie_{\bullet}(\mc{G}_{2,c}) \leftarrow \clie_{\bullet}(\mc{G}_{3,c}) \leftarrow \cdots \leftarrow \clie_\bullet (\mc{G}_{N,c}) .
%\end{equation}
So, it suffices to observe that $g_k$ is the single particle index of the factorization algebra $\clie_\bullet(\mc{G}^{(k)}_c)$, which is a direct observation using the description of $\mc{G}^{(k)}$ we have given in Proposition \ref{prop:Vj}.
\end{proof}

We thus arrive at the following conjecture for the index of the worldvolume theory on a stack of a finite number of fivebranes which we phrase in terms of the six-dimensional superconformal theory associated to the Lie algebra $\mf{sl}(N)$.

\begin{conj} 
The superconformal index of the six-dimensional superconformal theory associated to the Lie algebra of type $A_{N-1}$ is $\tilde \chi_{N}(y_1,y_2,y_3,y,q)$. 
\end{conj}

We proceed to give some concrete evidence for this conjecture.
First, we show that when we take the limit as $N \to \infty$ that we recover the index computed from the gravitational side.

\parsec

It follows from the limit description in \eqref{eqn:lim} that the large $N$ limit of $\chi_N$ is precisely the multiparticle supergravity index we computed in proposition~\ref{prop:sugraindex1}. 
Alternatively, we have the following direct proof of this fact. 

\begin{prop}
One has
\begin{equation}
\chi_{sugra}(y_i, y, q) = \lim_{N \to \infty} \chi_N(y_i,y,q)
\end{equation}
\end{prop}

\begin{proof}
It suffices to show that at the level of single particle indices one has
\begin{equation}
f_{sugra}(y_i, y, q) = \lim_{N \to \infty} f_N(y_i,y,q) ,
\end{equation}
where $f_N = \sum_{k = -1}^{N-2} g_k$. 

We will use the following identity 
\begin{equation}
\sum_{k=0}^\infty q^{3k/2} \chi_{k}^{\mf{sl}(2)}(q^{-1/2}y) = \frac{1}{(1-q y)(1-q^2 y^{-1})} .
\end{equation}
We will denote this expression by $S(y,q)$.

Using this identity one can directly see that the result reduces to observing that
\begin{multline}
\left(q^4 (y_1+y_2+y_3) + 1 - q^6 - q^2 (y_1^{-1} + y_2^{-1} + y_3^{-1})\right)S(y,q) - 1 + q^3= \\
\left(q^4(y_1+y_2+y_3)-q^2(y_1^{-1} + y_2^{-1} + y_3^{-1})+(1-q^3)(yq + y^{-1} q^2) \right) S(y,q) .
\end{multline}


%\begin{itemize}
%\item $q^4 \sum_{k=0}^\infty q^{3k/2} \chi_{k}^{\mf{sl}(2)}(q^{-1/2} y)(y_1+y_2+y_3) = \frac{q^4(y_1+y_2+y_3)}{(1-q y)(1-q^2 y^{-1})}$. 
%\end{itemize}

\end{proof}

As an immediate corollary we have the following result.
\begin{cor}
For any $N \geq 1$ one has
\begin{equation}
\chi_{sugra}(y_i,y,q) = \tilde{\chi}_N(y_i,y,q) \mod q^{N+1} .
\end{equation}
\end{cor}
\begin{proof}
This follows from observing that at the level of single particle states $f_N$ is of order $q^{N}$.
\end{proof}