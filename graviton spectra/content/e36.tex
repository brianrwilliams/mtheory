\documentclass[../main.tex]{subfiles}

\begin{document} 

\section{$E(3|6)$-modules from gravitons on $AdS_7\times S^4$}\label{sec:e36}

We now repeat the analysis of the previous subsection for gravitons on $AdS_7\times S^4$ respectively, making use of the description of supergravity states on $AdS_7\times S^4$ as the costalk of the factorization envelopes of the boundary condition and $\Omega^{0,\bullet}_{\C^3} (\mc L^{{r=0}}_{AdS_7\times S^4})$

As before, we construct certain $\C^\times$ actions on the boundary fields $\Omega^{0,\bullet}_{\C^3} (\mc L^{{r=0}}_{AdS_7\times S^4})$; we find that the zeroth weight space is a local version of another exceptional linearly compact super-Lie algebra $E(3|6)$. The decomposition of $\mc H_{AdS_7\times S^4}$ as a direct sum of $E(3|6)$ modules incidentally turns out to be very closely related to a decomposition of $E(5|10)$ into $E(3|6)$ modules studied by Cheng-Kac \cite{}.

\subsection{The graviton decomposition of twisted $AdS_7\times S^4$}
We wish to consider a particular decomposition of the space of states $\mc H_{AdS_7\times S^4}$. It is induced by a decomposition of the boundary fields $\Omega^{0,\bullet}_{\C^3} (\mc L^{{r=0}}_{AdS_7\times S^4} )$ introduced in section \ref{sec:transversebc}. 

%For simplicity let us consider the eleven-dimensional theory on flat space $\R \times \C^5$ with some number of fivebranes supported on
%\[
%0 \times 0 \times \C^3 \subset \C^5 .
%\]
%is that (after taking into account the backreaction) the observables on a large number of fivebranes is Koszul dual to the factorization algebra 
%
%\parsec[s:flatdecomp]
%
%%In the simple case where $Z = \C^3$ and we identify the total space of $K^{1/2}_Z \otimes \C^2$ with $\C^5$ then the manifold obtained by removing the locus of the brane is homeomorphic to
%%\[
%%\C^3 \times (\R \times \C^2 - 0) .
%%\]
%%
%%Let $\pi : \R \times \C^5 - (0 \times \C^3) \to \C^3 \times \R_{>0}$ be the projection map whose fibers are homeomorphic to the sphere $S^4$ which links the location of the fivebranes.
%%We restrict the factorization algebra of the eleven dimensional theory $\Obs_{sugra}$ to the open set obtained by removing the locus of the brane.
%
%In the simple case that $Z = \C^3$ and we identify the total space of $K^{1/2}_Z \otimes \C^2$ with $\C^5$ there is a more direct construction of the factorization algebra $\Obs_{sugra}|_{Z}$. 
%
%Let $\pi : \R \times \C^5 \to \C^3$ be the projection map.
%Then, via $\pi$ we can pushforward the factorization algebra associated to the eleven-dimensional theory to obtain a factorization algebra
%\[
%\pi_* \Obs_{sugra} 
%\]
%on $\C^3$.
%This factorization algebra is not the factorization algebra associated to an ordinary sort of field theory on $\C^3$. 
%Nevertheless there is a subfactorization algebra which admits a natural grading so that each filtered component can be understood as such.

Let $U\subset \C^3$ be an open; explicitly, the $\C^\times$ action on $\Omega^{0,\bullet}_{\C^3} (U, \mc L^{{r=0}}_{AdS_7\times S^4} )$ is given as follows.
\begin{itemize}
\item On the fields $\mu(w_a,z_i) \in \C[w_1, w_2]\{\del_{w_a}\} \otimes \Omega^{0,\bullet}_{\C^3} (U) \oplus \C[w_1, w_2]\otimes \Omega^{0,\bullet}_{\C^3}(U, T)$ the action is
\[
\lambda \cdot \mu(w_a,z_i) = \mu(\lambda w_a , z_i).
\]
\item On the fields $\nu(w_a,z_i) \in \C[w_1, w_2] \otimes \Omega^{0,\bullet}_{\C^3}(U)$ the action is
\[
\lambda \cdot \nu(w_a,z_i) = \nu(\lambda w , z).
\]
\item On the fields $\beta(w_a,z_i) \in \C[w_1, w_2] \otimes \Omega^{0,\bullet}_{\C^3}(U)$ the action is
\[
\lambda \cdot \beta(w_a,z_i) = \lambda^{-1} \beta(\lambda w_a , z_i).
\]
\item On the fields $\gamma(w_a,z_i) \in \C[w_1, w_2] \{\d w_a\} \otimes \Omega^{0,\bullet}_{\C^3} (U) \oplus \C[w_1, w_2]\otimes \Omega^{0,\bullet}_{\C^3}(U, \Omega^1)$ the action is
\[
\lambda \cdot \gamma(w_a,z_i) = \lambda^{-1} \gamma(\lambda w _a, z_i).
\]
\end{itemize}

The following proposition is a straightforward if lengthy computation - we state it without proof.
\begin{prop}\label{prop:ads7decomp}
The $L_\infty$ structure on $\Omega^{0,\bullet}_{\C^3} (\mc L^{{r=0}}_{AdS_7\times S^4} )$ identified in remark \ref{} is equivariant for this $\C^\times$ action. 
\end{prop}

%\begin{proof}
%The only nontrivial bracket to check compatibility with the $\C^\times$ action is the one which takes two $\gamma$-type sections to a $\mu$-type section of the form
%\[
%[\gamma_1, \gamma_2] = \Omega^{-1} \vee (\del \gamma_1 \wedge \del \gamma_2) 
%\]
%where $\Omega \vee (-)$ is the operation which takes a four-form 
%
%It suffices to check equivariance in local coordinates; there are three cases.
%Suppose $\gamma_1(t;w,z) = f^i(t;w,z) \d z_i$ and $\gamma_2(t;w,z) = g^j(t;w,z) \d z_j$where $f^i,g^j$ are de Rham--Dolbeault forms for $i,j=1,2,3$. 
%Then $\lambda \cdot \gamma_1 = \lambda^{-1} f^i(\lambda t;\lambda w, z) \d z_i$ and $\lambda \cdot \gamma_2 = \lambda^{-1} g^j(\lambda t;\lambda w, z) \d z_j$. 
%Thus, if the $\mu$-type field is expanded as $[\gamma_1,\gamma_2] = F^k(t;w,z) \del_{z_k} + G^a (t;w,z) \del_{w_a}$ for some de Rham--Dolbeault forms $F^k,G^a$, $k=1,2,3,a=1,2$, it suffices to show that 
%\begin{equation}
%\begin{array}{llll}
%\label{eqn:lambda1} \lambda \cdot \left(F^k(t;w,z) \del_{z_k} \right) & = \lambda^{-2} F^k(\lambda t;\lambda w,z) \del_{z_k} \\
%\lambda \cdot \left(G^a(t;w,z) \del_{w_a} \right) & = \lambda^{-2} G^a(\lambda t;\lambda w,z) \del_{w_a} .
%\end{array}
%\end{equation}
%We expand $\del \gamma_1 \wedge \del \gamma_2$ to three terms
%\begin{multline}
%\del_{w_a} f^i(t;w,z) \del_{z_l} g^j(t;w,z) \d w_a \d z_i \d z_l \d z_j + \del_{z_k} f^i(t;w,z) \del_{w_b} g^j(t;w,z) \d z_k \d z_i \d w_b \d z_j \\  + \del_{w_a} f^i(t;w,z) \del_{w_b} g^j(t;w,z) \d w_a \d z_i \d w_b \d z_j .
%\end{multline}
%Thus in the notation above we have 
%\begin{align*}
%F^k (t;w,z) & = \epsilon_{ab} \epsilon_{ijk} \del_{w_a} f^i(t;w,z) \del_{w_b} g^j(t;w,z) \\
%G^a(t;w,z) & = \epsilon_{ilj} \epsilon_{ab} \del_{w_a} f^i(t;w,z) \del_{z_l} g^j(t;w,z)
%+ \epsilon_{kij} \epsilon_{ab} \del_{z_k} f^i(t;w,z) \del_{w_a} g^j(t;w,z)  .
%\end{align*}
%We compute the action of $\lambda \in \C^\times$ on $\mu = F^k \del_{z_k}$
%\[
%\lambda \cdot \left(F^k \del_{z_k} \right) = \epsilon_{ab} \epsilon_{ijk} \lambda^{-1} \del_{w_a} f^i(\lambda t;\lambda w,z) \lambda^{-1} \del_{w_b} g^j(\lambda t;\lambda w,z) \del_{z_k} .
%\]
%Thus the first part of \eqref{eqn:lambda1} is satisfied. 
%Similarly we observe that $\lambda \cdot \left(G^a(t;w,z) \del_{w_a} \right)  = \lambda^{-2} G^a(\lambda t;\lambda w,z) \del_{w_a}$. 
%
%The next case is when $\gamma_1(t;w,z) = f^i(t;w,z) \d z_i$ and $\gamma_2(t;w,z) = g^a(t;w,z) \d w_a$where $f^i,g^a$ are de Rham--Dolbeault forms for $i=1,2,3$ and $a=1,2$. 
%Then $\lambda \cdot \gamma_1 = \lambda^{-1} f^i(\lambda t;\lambda w, z) \d z_i$ and $\lambda \cdot \gamma_2 = g^a(\lambda t;\lambda w, z) \d w_a$. 
%Thus, if the $\mu$-type field is expanded as $[\gamma_1,\gamma_2] = F^k(t;w,z) \del_{z_k} + G^a (t;w,z) \del_{w_a}$ for some de Rham--Dolbeault forms $F^k,G^a$, $k=1,2,3,a=1,2$, it suffices to show that 
%\begin{equation}
%\begin{array}{llll}
%\label{eqn:lambda2} \lambda \cdot \left(F^k(t;w,z) \del_{z_k} \right) & = \lambda^{-1} F^k(\lambda t;\lambda w,z) \del_{z_k} \\
%\lambda \cdot \left(G^a(t;w,z) \del_{w_a} \right) & = \lambda^{-1} G^a(\lambda t;\lambda w,z) \del_{w_a} .
%\end{array}
%\end{equation}
%We expand $\del \gamma_1 \wedge \del \gamma_2$ to three terms
%\begin{multline}
%\del_{w_b} f^i(t;w,z) \del_{z_j} g^a(t;w,z) \d w_b \d z_i \d z_j \d w_a + \del_{z_k} f^i(t;w,z) \del_{w_b} g^a(t;w,z) \d z_k \d z_i \d w_b \d w_a \\  + \del_{z_j} f^i(t;w,z) \del_{z_k} g^a(t;w,z) \d z_j \d z_i \d z_k \d w_a .
%\end{multline}
%Thus in the notation above we have 
%\begin{align*}
%F^k (t;w,z) & = \epsilon_{ab} \epsilon_{ijk} \del_{w_a} f^i(t;w,z) \del_{z_l} g^b(t;w,z) + \epsilon_{ab} \epsilon_{kij} \del_{z_k} f^i(t;w,z) \del_{w_a} g^b(t;w,z) \\
%G^a(t;w,z) & = \epsilon_{ab} \epsilon_{jik} \del_{z_j} f^i(t;w,z) \del_{z_k} g^b(t;w,z) .
%\end{align*}
%We compute the action of $\lambda \in \C^\times$ on $\mu = G^a \del_{z_a}$
%\[
%\lambda \cdot \left(G^a \del_{w_a} \right) = \epsilon_{ab} \epsilon_{jik} \del_{z_j} f^i(t;w,z) \del_{z_k} g^b(t;w,z) \lambda^{-1} \del_{w_a} .
%\]
%Thus the second part of \eqref{eqn:lambda2} is satisfied. 
%Similarly we observe that $\lambda \cdot \left(F^k(t;w,z) \del_{z_k} \right)  = \lambda^{-2} F^k(\lambda t;\lambda w,z) \del_{z_k}$. 
%
%The last case is when $\gamma_1(t;w,z) = f^a(t;w,z) \d w_a$ and $\gamma_2(t;w,z) = g^b(t;w,z) \d w_b$where $f^a,g^b$ are de Rham--Dolbeault forms for and $a,b=1,2$.
%The argument is nearly identical so we omit the details. 
%\end{proof}

 
This result induces a product decomposition 
\begin{equation}
\label{eqn:Gdecomp}
\Omega^{0,\bullet}_{\C^3} (\mc L^{{r=0}}_{AdS_7\times S^4}) = \prod_{n \geq -1} \mc{G}_{\C^3}^{(n)}
\end{equation}
where for each open set $U \subset \C^3$
\[
\mc{G}_\C^3(U)^{(n)}\subset \Omega^{0,\bullet}_{\C^3} (U, \mc L^{{r=0}}_{AdS_7\times S^4} )
\]
is the weight $n$ eigenspace with respect to the above $\C^\times$ action.  In particular, we see that $\mc{G}^{(0)}_{\C^3}$ is itself a local dg-Lie algebra (that we will soon describe). 
Moreover, every $\mc{G}_{\C^3}^{(n)}$, $n \geq -1$ is a (local) module for this local dg-Lie algebra.

%The $\C^\times$ action is compatible with the $\Z/2$ graded $L_\infty$ structure on $\mc{L}_{sugra}$. 
%Since the $n$th eigenspace is trivial when $n < -1$, we see that the product
%\[
%\mc{G}_Z (U)  \prod_{n \geq -1} (\pi_*\mc{L}_{sugra})(U)^{(n)}
%\]
%is equipped with the structure of a $\Z/2$ graded $L_\infty$ algebra.
%In this way, the assignment 
%\[
%\Bar{\pi}_* \mc{L}_{sugra} : U \mapsto (\Bar{\pi}_* \mc{L}_{sugra})(U) 
%\]
%defines a sheaf of $\Z/2$ graded $L_\infty$ algebras on $Z$. 

%\begin{lem}
%\label{lem:technical}
%For each $n$, $\mc{G}
%\[
%U \mapsto \Omega^{0,\bullet}(U, \mc{V}_{fivebrane}^{(n)})
%\]
%for some finite rank super holomorphic vector bundle $\mc{V}_{fivebrane}^{(n)}$ on $Z$.
%For each $n$, the differential is of the form $\dbar + Q^{hol}$.
%In particular, this endows $\Bar{\pi}_* \mc{L}_{sugra}$ with the structure of a pro-vector bundle on $Z$. 
%\end{lem}
%
%\begin{proof}
%In \S \ref{s:Lsugra} we introduced a holomorphic vector bundle $L_X$ defined on any Calabi--Yau five-fold $X$ whose sheaf of holomorphic sections was equipped with a $\Z/2$ graded $L_\infty$ structure. 
%Here, we consider the Calabi--Yau fivefold $X = X_Z$ as defined above.
%For any open subset $U \subset Z$ of the threefold the value of $\mc{L}_{sugra}(\pi^{-1}U)$ as a sheaf of super vector spaces is 
%\[
%\Omega^\bullet(\R) \otimes \Omega^{0,\bullet}(p^{-1} U, L_{X_Z}),
%\]
%where $p : X_Z \to Z$ is the rank two bundle over $Z$. 
%The one-ary structure map, or differential, is of the form $\d_{dR} + \dbar + Q^{hol}$ where $Q^{hol}$ is some odd holomorphic differential operator acting on sections of $L_{X_Z}$. 
%
%Since the de Rham complex of $\R$ is contractible we have a quasi-isomorphism of $L_\infty$ algebras
%\[
%\mc{L}_{sugra}(\pi^{-1}U) \simeq \Omega^{0,\bullet}(p^{-1} U, L_{X_Z})
%\]
%for every $U \subset Z$. 
%Without loss of generality, we can assume that $U$ is a coordinate chart for the vector bundle $X_Z$ so that $X_Z|_U \cong U \times \C^2$ and where the bundle $L_{X_Z}$ splits as 
%\[
%L_{X_Z}|_{\pi^{-1} U} \cong L' \boxtimes L'' 
%\]
%where $L'$ is a holomorphic super vector bundle on $U$ and $L''$ is.a holomorphic vector on the fiber $\C^2$.
%Then, we can further identify this with a Dolbeault complex of the form
%\[
%\Omega^{0,\bullet} \left(U, L' \otimes \Omega^{0,\bullet}(\C^2, L'') \right) . 
%\]
%By the Dolbeault Poincar\'e lemma applied to the holomorphic vector bundle $L''$ over $\C^2$,
%\[
%\Omega^{0,\bullet} \left(U, L' \otimes M\right) 
%\]
%where we now interpret $L' \otimes M$ as an infinite rank bundle over $U \subset Z$---
%the remaining differential is $\dbar_U + Q^{hol}$ where $Q^{hol}$ is a holomorphic differential operator.
%
%With these simplifications, it suffices to show that the weight $n$ eigenspace of $L' \otimes M$ is finite rank over $U$ \brian{almost finished}
%%we can fix a quasi-isomorphism $M \simeq \Omega^{0,\bullet}(\C^2, L'')$ where $M$ is a 
%\end{proof}


\subsection{The lowest piece: the holomorphic twist of the abelian 6d $\mc N=(2,0)$ tensor multiplet}\label{subsec:g-1}

The first non trivial case is the weight $(-1)$ piece.

\begin{lem}
There is an equivalence of abelian local Lie algebras 
\[
\mc{G}_{\C^3} ^{(-1)}\cong \Omega^{0,\bullet}_{\C^3} \left ( 
\begin{tikzcd}
\ul{+} & \ul{-} \\
\C^2\otimes K^{1/2}_{\C^3} & \\ 
\mc{O}_{\C^3} \ar[r, "\del"] & \Omega^1_{\C^3} 
\end{tikzcd}
\right )
\] 
\end{lem}
\begin{proof}
We readily see that the fields of weight $-1$ include
\begin{itemize}
\item 
fields of type $\mu$ of the form $\mu_a (z_i)\del_{w_a}$. As $w_a$ are fiber coordinates on $K^{1/2}_{\C^3}$, these fields transform as sections of $K^{1/2}_{\C^3}$. 
\item 
fields of type $\beta$ with no $w_a$-dependence. These fields constitute a copy of $\mc {O}_{\C^3}$.
\item 
fields of type $\gamma$ of the form $\gamma_i(z_i)\d z_i$. These fields constitute a copy of $\Pi \Omega^1_{\C^3}$. 
\end{itemize}
Since $\del$ is weight zero for this $\C^\times$ action, the fields of the last two type combine to give the complex of sheaves
\[
\mc{O}_{\C^3}  \xto{\del} \Pi \Omega^{1}_{\C^3}  .
\]

\end{proof}

\parsec[]
In \cite{SWtensor} Saberi and Williams, the authors studied the minimal twist of the 6d $\mc N=(2,0)$ abelian tensor multiplet. The twist is a free theory and can be defined on any complex three-fold admitting a square root of its canonical bundle. On $\C^3$, the $\Z\times \Z/2$ graded sheaf of complexes $\mc E_{tens}$ encoding its field content is given by 

\begin{equation}
\begin{tikzcd}
\ul{-1} & \ul{0} \\
\Pi \C^2\otimes \Omega^{0,\bullet}_{\C^3} \otimes  K_{\C^3}^{1/2}  & \\
\Omega^{2,\bullet}_{\C^3} \ar[r,"\del"] & \Omega^{3,\bullet}_{\C^3} 
\end{tikzcd} 
\end{equation}
Here we recall in the $\Z \times \Z/2$ bigrading the differential has bidegree $(1,0)$. 

We observe the following:
\begin{prop}
\label{prop:factabelian}
There is a quasi-isomorphism of factorization algebras valued in $\Z/2$ graded commutative dg algebras on $\C^3$
\[
\mc U (\mc G^{(-1)}_{\C^3} )\xto{\simeq} \clie^\bullet(\Pi \mc E_{tens})
\]
\end{prop}

\begin{proof}
Recall that the factorization algebra $\mc U(\mc{G}_{\C^3}^{(-1)})$ assigns to an open set $U\subset \C^3$ the graded symmetric algebra on the complex
\begin{equation}\label{eqn:weight-1}
\begin{tikzcd}
\ul{-} & \ul{+}\\
\Omega_{\C^3, c}^{0,\bullet}(U) \ar[r,"\del"] & \Omega_{\C^3, c}^{1,\bullet}(U) \\
\Omega_{\C^3, c}^{0,\bullet}(U, \C^2\otimes K^{1/2}) & 
\end{tikzcd}
\end{equation}
On the other hand, if we totalize the $\Z\times\Z/2$-grading on $\mc E_{tens}$ to a $\Z/2$-grading, the factorization algebra $\clie^\bullet (\Pi \mc E_{tens})$ assigns to an open set $U\subset \C^3$ the symmetric algebra on the complex 
\begin{equation}
\begin{tikzcd}
\ul{-} & \ul{+}\\
\Omega_{\C^3}^{2,\bullet}(U)^\vee \ar[r,"\del"] & \Omega_{\C^3}^{3,\bullet}(U)^\vee \\
& \Omega_{\C^3}^{0,\bullet}(U, \C^2\otimes K^{1/2})^\vee \\
\end{tikzcd}
\end{equation}

Here the superscript refers to the topological dual, which is described in terms of compactly supported distributional sections of the Serre dual vector bundle. Thus, we see that the above complex is the same as 

\begin{equation}
\begin{tikzcd}
\ul{-} & \ul{+}\\
\overline \Omega_{\C^3, c}^{0,\bullet}(U) \ar[r,"\del"] & \overline \Omega_{\C^3, c}^{1,\bullet}(U) \\
\overline \Omega_{\C^3, c}^{0,\bullet}(U, \C^2\otimes K^{1/2}) & 
\end{tikzcd}
\end{equation}

where the degree shift is coming from Serre duality. The result then follows from the fact that by ellipticity, the natural inclusion $\Omega^{0,\bullet}_{\C^3,c}\to\overline \Omega^{0,\bullet}_{\C^3,c}$ is a quasi-isomorphism.
\end{proof}

\subsection{The zero-th piece: a local version of $E(3|6)$}

As before, the weight zero summand $\mc{G}_{\C^3} ^{(0)}$ is special because it carries the induced structure of a local $L_\infty$-algebra on ${\C^3} $ inherited from the $L_\infty$ structure on $\Pi\Omega^{0,\bullet} (\mc L^{r = 0}_{AdS_7\times S^4} )$ identified in section \ref{sec:transversebc}. We will prove that it is equivalent to a local Lie algebra version of the exceptional super-Lie algebra $E(3|6)$ \cite{KacBible}.

We first recall the definition of this super-Lie algebra. 

\begin{defn}\label{defn:e(3|6)}
Let $E(3|6)$ be the following super-Lie algebra.
\begin{itemize}
\item The even part, $E(3|6)_0$ is given by the semidirect product Lie algebra $\Gamma(\widehat D, T) \ltimes \left ( \mf{sl}(2) \otimes \Gamma (\widehat D, \mc O) \right )$. 
\item The odd part, $E(3|6)_1$ is given by $\C^2\otimes \Gamma (\widehat D, \Omega^1(K^{-1/2} ))$. 
\end{itemize}
The remaining brackets to be specified, are as follows:
\begin{itemize}
\item The action of $E(3|6)_0$ on $E(3|6)_1$ is given by the Lie derivative, along with the fundamental action of $\mf{sl}(2)$. 
\item The bracket between two odd elements is given by 
\begin{align*}
[v_1\otimes f_i \d z_i \otimes (\del_{z_1}\del_{z_2}\del_{z_3})^{1/2} &, v_2\otimes g_j \d z_j \otimes (\del_{z_1}\del_{z_2}\del_{z_3})^{1/2} ]  \\ 
 = & \  \omega (v_1, v_2) \eps^{ijk} f_i g_j \del_{z_k}  \\
 + & \  (v_1 \odot v_2) \left ( \del (f_i \d z_i ) g_j \d z_j  - f_i \d z_i \del (g_j \d z_j)  \right )\vee (\del_{z_1}\del_{z_2}\del_{z_3} )
\end{align*}
where $\omega$ denotes a symplectic form on $\C^2$ and $\odot: \C^2\otimes \C^2 \to \mf{sl}(2)$ is the canonical $\mf {sl}(2)$-equivariant projection. 
\end{itemize}
\end{defn}

The relationship between this super-Lie algebra and our decomposition is established through the following result.

\begin{prop}\label{prop:g0e36}
There is an equivalence of super-Lie algebras
\[
\mc G^{(0)}_{\C^3, c} (0) \cong E(3|6)
\]
\end{prop}
\begin{proof}
We begin by characterizing the local $L_\infty$-algebra $\mc G^{(0)}_{\C^3}$. We claim that it is quasi-isomorphic to a local version of $E(3|6)$. 

Indeed, we readily see that the weight zero sections consist of the following cochain complex

\begin{equation}
\Omega^{0,\bullet}_{\C^3} \left (
\begin{tikzcd}
\ul{even} & \ul{odd} \\
\C\{w_a\del_{w_b}\}\otimes \mc O\ar[r, "\del^W_\Omega" description]  & \mc O \\ 
\T \ar[ur, "\del^Z_\Omega" description] & \\
\C^2\otimes \mc O \ar[r, "\del_W" description]\ar[dr, "\del_Z" description] & \C\{\d w_a\} \otimes K^{-1/2}\ \\
& \C^2 \otimes \Omega^1\otimes K^{-1/2}
\end{tikzcd} \right)
\end{equation}

The differentials are again components of the divergence operator and holomorphic deRham operator. We can compute cohomology by way of a spectral sequence whose first page is the cohomology with respect to $\del^W_\Omega + \del_W$. We see that the differential $\del^W_\Omega$ maps surjectively onto functions and its kernel is isomorphic to $\mf {sl}(2) \otimes \mc O$. Likewise, the differential $\del_W$ is just the identity map between $\C^2$ and $\C\{\d w_a\}$. 

Thus we see that this page of the spectral sequence is given by 
\begin{equation}
\mc E(3|6) \define \Omega^{0,\bullet}_{\C^3} \left (
\begin{tikzcd}
\ul{even} & \ul{odd} \\
\T & \C^2\otimes \Omega^1_{\C^3} (K^{-1/2}) \\
\mf {sl}(2)\otimes \mc O
\end{tikzcd} \right)
\end{equation}
and there are no non-zero differentials so the spectral sequence degenerates.

To see that the Lie structure induced from the $L_\infty$-structure on $\Omega^{0,\bullet}_{\C^3} (\mc L^{r=0}_{AdS_7\times S^4} )$ is in fact given by the same formulae as the brackets on $E(3|6)$ given in \ref{defn:e(3|6)}, it will be useful to provide an explicit quasi-isomorphism $\Phi^{(0)} : \mc E(3|6) \to \mc G^{(0)}_{\C^3}$. On an open set $U\subset \C^3$, this is defined as follows.

\begin{itemize}
\item Given a section $g_i(z)\del_{z_i}\in \Omega^{0,\bullet}_{\C^3} (U, T)$ where $g_i(z)$ is a Dolbeault form on $U$, we define
\begin{align*}
\Phi^{(0)} (g_i (z) \del_{z_i} ) &= g_i (z) \del_{z_i} - \frac 12(\del_{z_i} g_i (z)) w_a \del_{w_a} \\
&\in \Omega^{0,\bullet}_{\C^3} \left ( U, T \oplus \C \{w_a \del_{w_b} \} \otimes \mc O \right ).  
\end{align*}

\item Given a section $A\otimes g(z)\in \Omega^{0,\bullet}_{\C^3} (U,\mf {sl}(2) \otimes \mc O)$ where $g(z)$ is a Dolbeault form on $U$, and $A_{ab}\in \mf {sl}(2)$ we define
\begin{align*}
\Phi^{(0)} (A_{ab}\otimes g(z) ) &=  g(z) A_{ab}w_a\del_{w_b}\\
&\in \Omega^{0,\bullet}_{\C^3} \left ( \C \{w_a \del_{w_b} \} \otimes \mc O \right ).  
\end{align*}

\item Given a section $v\otimes g_i(z)\d z_i (\del_{z_1}\del_{z_2}\del_{z_3} ) \in \Omega^{0,\bullet}_{\C^3} (U, \C^2\otimes \Omega^1\otimes K^{-1/2})$ where $g_i(z)$ is a Dolbeault form on $U$ and $v\in \C^2$, we define
\begin{align*}
\Phi^{(0)} (v\otimes g_i(z)\d z_i (\del_{z_1}\del_{z_2}\del_{z_3} ) ) &= (w_1(v)w_1+w_2(v)w_2) \otimes g_i(z)  \\
&\in \Omega^{0,\bullet}_{\C^3} \left ( \C^2\otimes \Omega^1\otimes K^{-1/2}  \right ).  
\end{align*}
\end{itemize}

The result then follows from computing the limit of $\mc E(3|6)_c (D^3)$ over open sets containing the origin.
\end{proof}

\begin{rmk}
We note that the map $i_{M5}$ from lemma \ref{lem:m5emb} in fact defines a Lie map from $\mf{osp}(6|2)$ to the sections of the boundary condition $\Pi \Omega^{0,\bullet}_{\C^3} (\mc L^{r=0}_{AdS_7\times S^4} )$ over every open set containing the origin. The image of the map lands exactly in the step $\mc G^{(0)}_{\C^3}$ of the decomposition from proposition \ref{prop:ads7decomp}. Therefore we see that $E(3|6)$ contains $\mf{osp}(6|2)$ as a finite dimensional subalgebra.  
\end{rmk}
%\begin{rmk}
%In \cite{KacClass} an embedding of super Lie algebras from $E(3|6)$ into $E(5|10)$ is constructed.
%Recall that in the case $Z = \C^3$, the $\infty$-jets at $0 \in \C^3$ of the local Lie algebra $\mc{E}(3|6)$ is precisely the exceptional super Lie algebra $E(3|6)$.
%Similarly, the $\infty$-jets at $0 \in \C^3$ of the local $L_\infty$ algebra $\mc{G}_{\C^3}$ is $\widehat{E(5|10)}$.
%The embedding of local $L_\infty$ algebras on $Z$ from $\mc{E}(3|6)$ into $\mc{G}_Z$ that we just described agrees with the embedding of \cite{KacClass} upon taking $\infty$-jets.
%(Note that the central term in $\widehat{E(5|10)}$ plays no role herem since it sits in $\C^\times$-weight $-1$.)
%\end{rmk}

%\begin{itemize}
%\item For $\Vect^{hol}(Z)$ there is the standard commutator of holomorphic vector fields. 
%This acts on the sections of $\mf{sl}(2) \otimes \mc{O}_Z$ and $\Pi\Omega^{1,hol}(Z, K^{-1/2}_Z \otimes \C^2)$ by Lie derivative. 
%\item On sections of $\mf{sl}(2) \otimes \mc{O}_Z$ there is the matrix commutator. 
%This acts on the odd part $\Pi\Omega^{1,hol}(Z, K^{-1/2}_Z \otimes \C^2)$ where we view $\C^2$ as the fundamental $\mf{sl}(2)$ representation. 
%\item Finally, and most interestingly, there is a bracket of the form
%\[
%\Pi\Omega^{1,hol}(Z, K^{-1/2}_Z \otimes \C^2) \times \Pi\Omega^{1,hol}(Z, K^{-1/2}_Z \otimes \C^2) \to \Vect^{hol}(Z) \oplus \mf{sl}(2) \otimes \mc{O}_Z 
%\]
%given in coordinates by
%\begin{multline}
%[f^{ai}(z) w_a \d z_i, g^{bj} (z) w_b \d z_j] = \epsilon_{ijk} \epsilon_{ab} f^{ai} (z) g^{bj}(z) \del_{z_k} \\ + \epsilon_{ijk} \epsilon_{bc}  \del_{z_k} f^{ai}(z) g^{bj}(z) w_a \del_{w_c} + \epsilon_{ijk} \epsilon_{ac} f^{ai}(z) \del_{z_k} g^{bj}(z) w_b \del_{w_c} .
%\end{multline}
%The top line is a holomorphic vector field on~$Z$ and the bottom line is a $\mf{sl}(2)$-valued holomorphic function on~$Z$.
%\end{itemize}


%\parsec[s:mainfact]
%
%Since $\mc{G}_Z$ is a pro-vector bundle with a compatible $L_\infty$ structure, the assignment 
%\[
%U\subset Z \mapsto \clie^\bullet\left(\Bar{\pi}_* \mc{L}_{sugra}(U)\right) 
%\]
%has the structure of a factorization algebra on the three-fold $Z$. 
%We denote this factorization algebra on $Z$ by $\Obs_{sugra}|_Z$. 

%\begin{prop}
%Let $Z$ be a Calabi--Yau three-fold and $\pi : Z \times \C^2 \times \R \to Z$ be the projection. 
%There is a pro local Lie algebra $\Bar{\pi}_*\mc{L}_{sugra}$ and a factorization algebra $\Bar{\pi}_* \Obs_{sugra} = \clie^\bullet(\Bar{\pi}_* \mc{L}_{sugra})$ on $Z$ such that:
%\begin{itemize}
%\item[(1)] there is a natural inclusion of of factorization algebras on $Z$
%\[
%\Bar{\pi}_* \Obs_{sugra} \hookrightarrow \pi_* \Obs_{sugra}
%\]
%which is dense at the level of cohomology. 
%\item[(2)] there is a weight grading on $\mc{L}_{\pi,sugra}$ which is concentrated in degrees $\geq -1$ and gives rise to a decomposition of vector bundles
%\begin{equation}\label{eqn:decomp3}
%\Bar{\pi}_{*} \mc{L}_{sugra} = \prod_{n \geq -1} \mc{V}_{n} 
%\end{equation}
%\item[(3)]
%In weight zero, there is an equivalence of local Lie algebras on $Z$ 
%\[
%\mc{V}_0 \simeq \mc{E}(3|6)|Z 
%\]
%where $\mc{E}(3|6)|Z$ is a local Lie algebra enhancement of the exceptional simple super Lie algebra $E(3|6)$. 
%\end{itemize}
%\end{prop}

%We want to argue that $\Obs_{sugra} |_{\C^3} \cong \Bar{\pi}_* \Obs_{sugra}$. 

\subsection{General summands and $E(3|6)$-modules}

We move on to give the following general description of the weight $j$ component $\mc{G}_{\C^3} ^{(j)}$.
Since we have already described $j = -1,0$ we focus on $j \geq 1$.

\begin{prop}
\label{prop:Vj}
Let $j \geq 1$. 
The complex of vector bundles $\mc{G}^{(j)}$ is quasi-isomorphic to
\begin{equation}
\label{eqn:Gj}
\Omega^{0,\bullet}_{\C^3} \left (\begin{tikzcd}
\ul{even} & \ul{odd} \\
\Sym^{j}(\C^2) \otimes \T \otimes K^{-j/2} & \Sym^{j-1}(\C^2) \otimes K^{-(j+1)/2}\\
\Sym^{j+2}(\C^2) \otimes K^{-j/2} & \Sym^{j+1}(\C^2) \otimes \T^* \otimes K^{-(j+1)/2} .
\end{tikzcd}\right )
\end{equation}
\end{prop}

\begin{proof}
We begin by noting that we can explicitly describe the weight $j$ component $\mc G^{(j)}_{\C^3}$ as $\Omega^{0,\bullet}_{\C^3} (G^{(j)})$ where $G^{(j)}$ is the following dg-vector bundle
\begin{equation}
\begin{tikzcd}
\ul{even} & \ul{odd} \\
\Sym^{j+1}(\C^2) \otimes \C^2 \ar[dr, "\del^W_\Omega"'] \otimes K^{-j/2}\\ 
& \Sym^j (\C^2) \otimes K^{-j/2} \\ 
\Sym^j (\C^2) \otimes  \T \otimes K^{-j/2} \ar[ur, "\del^Z_\Omega"]  & \\
& \Sym^{j+1}(\C^2)  \otimes  \Omega^1 \otimes K^{-(j+1)/2} \\ 
\Sym^{j+1}(\C^2) \otimes K^{-(j+1)/2} \ar[ur, "\del_Z"] \ar[dr, "\del_W"'] \\
& \Sym^j(\C^2) \otimes \C^2\otimes K^{-(j+1)/2} 
\end{tikzcd}
\end{equation}

Note that the differentials here are $\mf{sl}(2)$-equivariant maps, tensored with a differential operator acting on sections of a bundle on $\C^3$. In particular

\begin{itemize}
\item The differential $\del^W_\Omega$ is the canonical projection \[\Sym^{j+1}(\C^2) \otimes \C^2 \cong \Sym^{j+2} (\C^2) \oplus \Sym ^j (\C^2) \twoheadrightarrow \Sym^j (\C^2)\] tensored with the identity acting on $K^{-j/2}$. 

\item The differential $\del_W$ is the canonical inclusion \[\Sym^{j+1}(\C^2) \hookrightarrow \Sym^{j-1}(\C^2) \oplus \Sym^{j+1}(\C^2) \cong \Sym^j(\C^2) \otimes \C^2 .\] tensored with the identity acting on $K^{-(j+1)/2}$.
\end{itemize}

There is a spectral sequence whose first term is computed by the $\del^W_\Omega +\del_W$-cohomology. The result is the complex of sheaves in equation \ref{eqn:Gj}. There are no further differentials so the spectral sequence collapses at this page and the result follows.
\end{proof}

\iffalse
\parsec[]
We wish to explicate the $E(3|6)$-module structure on each of the costalks $\Omega^{0,\bullet}_{\C^3, c} (G^{(j)} )(0)$. To do so, it will be useful to explicate the quasi-isomorphism whose existence was determined in the proof of the preceding proposition. To this end, we describe an explicit quasi-isomorphism 
\[\Phi: \Omega^{0,\bullet}_{\C^3} (G^{(j)} ) \to \mc G^{(j)}_{\C^3}\] 
as follows. 

Let $U\subset \C^3$ be open. 
\begin{itemize}
\item Given a section
\[
f(w) \otimes g_i(z) \del_{z_i}\in \Omega^{0,\bullet}_{\C^3} \left (U, \Sym^j(\C^2) \otimes \T \otimes K^{-j/2} \right )
\] where $f(w) \in \Sym^j (\C^2)$ is a homogenous degree $j$ polynomial in $w_1,w_2$ and the $g_i(z)$'s are Dolbeault forms on $\C^3$, we define
\begin{align*}
\Phi(f(w) \otimes g_i(z) \del_{z_i}) &= f(w) g(z) \del_{z_i} - \frac{1}{j+2} \left(\del_{z_i} g_i(z)\right) f(w) w_a\del_{w_a} \\
&\in \Omega^{0,\bullet}_{\C^3} \left ( U, (\Sym^j (\C^2) \otimes T\otimes K^{-j/2} )\oplus ( \Sym^{j+1}(\C^2) \otimes \C^2 \otimes K^{-j/2} ) \right )
\end{align*}

\item Given a section
\[
f(w) \otimes g(z)\in \Omega^{0,\bullet}_{\C^3} \left (U, \Sym^{j+2} \otimes K^{-j/2} \right)
\] where $f(w) \in \Sym^{j+2} (\C^2)$ is a homogenous degree $j+2$ polynomial in the variables $w_1,w_2$ and $g(z)$ is a Dolbeault form on $\C^3$, we define
\begin{align*}
\Phi ( f(w) \otimes g(z) ) &= g(z) (\del _{w_1} f(w) \del_{w_2} - \del _{w_2} f(w) \del_{w_1} ) \\
&\in \Omega^{0,\bullet}_{\C^3} \left ( U, \Sym^{j+1} (\C^2) \otimes T \otimes K^{-j/2}\right )
\end{align*}

\item Given a section
\[
f(w) \otimes g(z)  \in \Omega^{0,\bullet}_{\C^3} \left (U, \Sym^{j-1}(\C^2) \otimes K^{-(j+1)/2} \right)
\]
where $f(w) \in \Sym^{j-1} (\C^2)$ is a homogenous degree $j-1$ polynomial in the variables $w_1,w_2$ and $g(z)$ is a Dolbeault form on $Z$ we define
\begin{align*}
\Phi( f(w) \otimes g(z) ) &= \frac12 g(z)f(w)(w_1 \d w_2 - w_2 \d w_1)  \\
& \in \Omega^{0,\bullet}_{\C^3}\left  (U, \Sym^j (\C^2) \otimes \C^2 \otimes K^{-(j+1)/2} \right )
\end{align*}

\item
Given a section
\[
f(w) \otimes g^i(z)  \d z_i\in \Omega^{0,\bullet}_{\C^3} \left (U, \Sym^{j+1}(\C^2) \otimes T^* \otimes K^{-(j+1)/2} \right)
\] 
where $f(w) \in \Sym^{j+1} (\C^2)$ and the $g^i(z)$'s are Dolbeault forms on $\C^3$, we define 
\begin{align*}
\Phi(f(w) \otimes g^i(z) \d z_i) & = f(w) g^i (z) \d z_i \\
& \in \Omega^{0,\bullet}_{\C^3} \left (U, \Sym^{j+1} (\C^2) \otimes \Omega^1 \otimes K^{-(j+1)/2} \right )
\end{align*}
\end{itemize}

The following is immediate:

\begin{lem}\label{lem:qisG}
The map $\Phi: \Omega^{0,\bullet}_{\C^3} (G^{(j)} ) \to \mc G^{(j)}_{\C^3}$ is a quasi-isomorphism.
\end{lem}


\parsec[]
Recall that as an immediate corollary of proposition \ref{prop:ads7decomp}, the local super-Lie algebra $\mc E(3|6)$ acts on the abelian local Lie algebras $\mc G^{(j)} _{\C^3}$. By way of the explicit quasi-isomorphisms  $\mc{G}^{(0)} \simeq \mc{E}(3|6)$ and $\mc{G}^{(j)} \simeq \Omega^{0,\bullet}_{\C^3}( G^{(j)})$ we can explicate this module structure.

Recall that $\mc{E}(3|6)$ was given by  $\Omega^{0,\bullet}_{\C^3} (G^{(0)})$ where 
\begin{equation}
G^{(0)} = 
\begin{tikzcd}
\ul{even} & \ul{odd} \\
\T & \C^2\otimes \Omega^1_{\C^3} (K^{-1/2}) \\
\mf {sl}(2)\otimes \mc O
\end{tikzcd}
\end{equation}
The local Lie algebra structure on $\mc{E}(3|6)$ arises from a Lie algebra structure on the sheaf of holomorphic sections of $G^{(0)}$. Likewise, the $\mc{E}(3|6)$-module structure on $\mc{G}^{(j)} \simeq \Omega^{0,\bullet}_{\C^3} (G^{(j)})$ arises from an action of the sheaf of holomorphic sections of $G^{(0)}$ on the sheaf of holomorphic sections of $G^{(j)}$, where the action is expressed through holomorphic differential operators.

Let us explicate this action. It is easy to see how holomorphic sections of the even part of the super-vector bundle $G^{(0)}$ will act. Sections of $T$ are holomorphic vector fields and act by Lie derivative. Likewise, sections of $\mf{sl}(2) \otimes \mc O$ act via the corresponding $\mf{sl}(2)$-representation.

We need only explain how holomorphic sections of the odd component $\C^2 \otimes \T^* \otimes K^{-1/2}$ of $G^{(0)}$ act. We first give a global description of this action, and then we will write down the explicit formula in local coordinates. In local coordinates, we may express such a section as  $w_a g^i(z) \d z_i$ where $g^i(z)$ is a holomorphic function. The action is as follows:
 
\begin{itemize}
\item The odd part of $\mc{V}^{(0)}$ acts on the component $S^{j}(\C^2) \otimes \T_Z \otimes K^{-j/2}_Z$ through the composition
\begin{equation}
\begin{tikzcd}
\left(\T^*_Z \otimes K^{-1/2}_Z \otimes \C^2\right) \otimes \left(S^{j}(\C^2) \otimes \T_Z \otimes K^{-j/2}_Z\right) \ar[r,"\cong"] & \left(S^{j-1}(\C^2) \oplus S^{j+1}(\C^2)\right) \otimes \left(\T_Z \otimes \T^*_Z \otimes K^{-(j+1)/2}_Z\right) \ar[dl] \ar[d] \\
S^{j-1}(\C^2) \otimes K^{-(j+1)/2}_Z & S^{j+1}(\C^2) \otimes \T^*_Z \otimes K^{-(j+1)/2}_Z 
\end{tikzcd}
\end{equation}
Here, the leftmost downward arrow is the evident $\mf{sl}(2)$ projection together with the canonical pairing between sections of $\T_Z$ and $\T^*_Z$.
The rightmost downward arrow is the other $\mf{sl}(2)$ projection together with the Lie derivative of holomorphic one-forms.  
Given a local section $f(w) \otimes h_i(z) \del_{z_i}$ of $S^j(\C^2) \otimes \T_Z \otimes K_Z^{-j/2}$ an explicit formula for this action is
\begin{align*}
(w_a g^i(z) \d z_i) \cdot (f(w) \otimes h_k(z) \del_{z_k}) & = \epsilon_{ab} (\del_{w_b} f(w)) (g^i h_i)(z)  \\ & + w_a f(w) L_{h_k \del_{z_k}} (g^i \d z_i) .
\end{align*}
\item The odd part of $\mc{V}^{(0)}$ acts on the component $S^{j+2}(\C^2) \otimes K^{-j/2}_Z$ through the composition
\begin{equation}
\begin{tikzcd}
\left(\T^*_Z \otimes K^{-1/2}_Z \otimes \C^2\right) \otimes \left(S^{j+2}(\C^2) \otimes K^{-j/2}_Z\right) \ar[r,"\cong"] & \left(S^{j+1}(\C^2) \oplus S^{j+3}(\C^2)\right) \otimes \left(\T^*_Z \otimes K^{-(j+1)/2}_Z\right) \ar[d] \\
& S^{j+1}(\C^2) \otimes \T^*_Z \otimes K^{-(j+1)/2}_Z
\end{tikzcd}
\end{equation}
where the downward arrow is induced by the evident $\mf{sl}(2)$ projection.
\item The odd part of $\mc{V}^{(0)}$ acts on the component $S^{j-1}(\C^2) \otimes K^{-(j+1)/2}_Z$ through the composition
\begin{equation}
\begin{tikzcd}
\left(\T^*_Z \otimes K^{-1/2}_Z \otimes \C^2\right) \otimes \left(S^{j-1}(\C^2) \otimes K^{-j/2}_Z\right) \ar[r,"\cong"] & \left(S^{j-2}(\C^2) \oplus S^{j}(\C^2)\right) \otimes \left(\T^*_Z \otimes K^{-1}_Z K^{-j/2}_Z\right) \ar[d] \\
& S^{j}(\C^2) \otimes \T_Z \otimes K^{-j/2}_Z
\end{tikzcd}
\end{equation}
where the downward arrow is induced by the evident $\mf{sl}(2)$ projection together with the holomorphic de Rham operator taking holomorphic one-forms to holomorphic two-forms. 
\item Finally, the odd part of $\mc{V}^{(0)}$ acts on the component $S^{j+1}(\C^2) \otimes \T^*_Z \otimes K^{-(j+1)/2}_Z$ through the composition
\begin{equation}
\begin{tikzcd}
\left(\T^*_Z \otimes K^{-1/2}_Z \otimes \C^2\right) \otimes \left(S^{j+1}(\C^2) \otimes \T^*_Z \otimes K^{-j/2}_Z\right) \ar[r,"\cong"] & \left(S^{j}(\C^2) \oplus S^{j+2}(\C^2)\right) \otimes \left(\T^*_Z \otimes \T^*_Z \otimes K^{-1}_Z K^{-j/2}_Z\right) \ar[dl] \ar[d] \\
S^j (\C^2) \otimes \T_Z \otimes K_Z^{-j/2} & S^{j+2}(\C^2)  \otimes K^{-j/2}_Z
\end{tikzcd}
\end{equation}
where the leftmost downward arrow is induced by the evident $\mf{sl}(2)$ projection.
The rightmost downward arrow is induced by the remaining $\mf{sl}(2)$ projection together with the holomorphic de Rham operator taking holomorphic two-forms to holomorphic three-forms.
\end{itemize}
\fi

\parsec[s:kacrelation]

The decomposition of $\Omega^{0,\bullet}_{\C^3} (\mc L^{r=0}_{AdS_7\times S^4} )$ in equation \eqref{eqn:Gdecomp} is closely related to a decomposition of the exceptional simple super Lie algebra $E(5|10)$ studied in \cite{KR2}. In \ref{thm:global}, we showed that the global sections of the parity shifted fields of our eleven-dimensional theory on flat space is quasi-isomorphic to a Lie 2-extension of $E(5|10)$, which we denoted $\widehat{E(5|10)}$. More precisely, we found a Lie 2-extension of a version of $E(5|10)$ built out of polynomials rather than Taylor series. Given that the $L_\infty$ structure on $\Pi \Omega^{0,\bullet}_{\C^3} (\mc L^{r=0}_{AdS_7\times S^4} )$ is given by the same formulas as that on $\Pi \mc E$, it is easy to see that the space of $\infty$-jets of $\Pi \Omega^{0,\bullet}_{\C^3} (\mc L^{r=0}_{AdS_7\times S^4} )$ at the origin, which following lemma \ref{} has underlying vector space $\mc H_{AdS_7\times S^4}$, is quasi-isomorphic to $\widehat{E(5|10)}$. 

In \cite{KR2} the following weight decomposition of $E(5|10)$ is constructed. Splitting $\C^5 = \C^2_{w_a}\times \C^3_{z_i}$ as we have been doing, we stipulate that
\begin{itemize} 
\item the coordinate $z_i$ has weight zero, $\wt(z_i) = 0$. 
\item the coordinate $w_a$ has weight $+1$, $\wt(w_a) = +1$. 
\item the parity of an element carries an additional weight of $-1$. 
Thus, for example, the odd element $[\d w_1 \d z_1] \in \Omega^{2,cl}(\widehat{D}^5)$ carries weight $+1 - 1 = 0$. Viewing the odd part as the space of closed two-forms, then equivalently this grading translates to the one-form symbol $\d(-)$ as carrying weight $-1/2$.
\end{itemize} 

It is straightforward to verify that this weight grading is compatible with the super Lie algebra structure on $E(5|10)$. Moreover, we see that similarly to the decomposition of $\mc H_{AdS_7\times S^4}$ induced by our $\C^\times$ action in \ref{prop:ads7decomp}, the weight grading is concentrated in degrees $\geq -1$.  In particular, there is a decomposition of super vector spaces
\begin{equation}\label{eqn:decomp1}
E(5|10) = \tilde U_{-1} \times \prod_{j \geq 0} U_j 
\end{equation}
Further, this decomposition also has the property that the 0-th piece $U_0$ is isomorphic to $E(3|6)$. As such, each $U_j$ is an $E(3|6)$-module; Kac characterizes these modules explicitly and identifies them as certain irreducible $E(3|6)$-modules. In the notation of \cite{KR2} we have that $U_{-1} = I(0,0;1;-1)^*$ and for $j\geq 1$, $U_j = I(0,0;j-1;j+1)^*$. 

The decomposition of $\mc H_{AdS_7\times S^4}$ afforded by proposition \ref{prop:ads7decomp} induces a weight grading of $\widehat{E(5|10)}$ which extends the one on $E(5|10)$ that we have just described by declaring that the central term have weight $-1$.
In this way, we get a related decomposition of super $L_\infty$ algebras
\begin{equation}\label{eqn:decomp2}
\widehat{E(5|10)} = \prod_{j \geq -1} U_j            
\end{equation}                      
were $U_{-1}$ is a $\C$-extension of $\tilde U_{-1}$ defined in the decomposition \eqref{eqn:decomp1} and for $j \geq 0$ the $U_j$'s are the same as in the non centrally extended case. As a corollary, we see that each of the $E(3|6)$ modules which we have identified as the costalk at 0 $\mc G^{(j)}_{\C^3, c} (0)$ is in fact irreducible.

\subsection{Characters of $E(3|6)$-modules}\label{e36char}
The decomposition of the state space $\Sym \left (\mc H_{AdS_7\times S^4}\right ) = \prod_{j\geq -1} \mc U (\mc G^{(j)})(0)$ gives a product formula for the characters computed in proposition \ref{prop:sugraindex1}

\[ \chi \left ( \Sym \mc H_{AdS_7\times S^4} \right ) = \prod_{j\geq -1} \chi \left (\mc U (\mc G^{(j)})(0) \right ) \]

We end this section by computing each of the characters $\chi \left (\mc U (\mc G^{(j)})(0) \right )$. We will express our characters in terms of characters of highest weight representations of $\mf{sl}(2)$ and $\mf {sl}(3)$, which we denote by $\chi_k^{\mf{sl}(2)}$ and $\chi^{\mf{sl}(3)}_{[k,l]}$.

\parsec[]
We begin with the lowest step of the decomposition, using the characterization given in \ref{subsec:g-1}. 

\begin{prop}
\label{prop:6done}
The character $\chi \left ( \mc U (\mc G^{(-1)}_{\C^3} ) (0)\right )$ is given by the plethystic exponential of the following expression:
\begin{equation}\label{eqn:6done}
g_{-1} (t_1,t_2,r,q) = \frac{q^{3/2}(r + r^{-1}) - q^2(t_1 + t_1^{-1} t_2 + t_2^{-1} ) + q^3}{(1-t_1^{-1}q) (1-t_1 t_2^{-1} q) (1-t_2 q)} .
\end{equation}
\end{prop}
\begin{proof}
Note that in light of proposition \ref{prop:factabelian} we can equivalently compute the character of the costalk at the origin of $\clie^\bullet (\Pi \mc E_{tens})$. We first give a more explicit description of the costalk as a cochain complex. Proceeding exactly analogously to the proof of lemma \ref{lem:ads4states}, we see that the costalk is given by the symmetric algebra on the following cochain complex
\[
\begin{tikzcd}
\ul{-} & \ul{+}\\
\C\{\d z_i\d z_j\} \otimes \C [\del_{z_1}, \del_{z_2}, \del_{z_3}]\delta_{z_i = 0} \ar[r,"\del"] & \C\{\d z_1\d z_2\d z_3\} \otimes \C [\del_{z_1}, \del_{z_2}, \del_{z_3}]\delta_{z_i = 0} \\
& \C^2\otimes \C\{\d z_i^{1/2}\} \otimes \C [\del_{z_1}, \del_{z_2}, \del_{z_3}]\delta_{z_i = 0} 
\end{tikzcd}
\]

Computing summand-by-summand, we see:
\begin{itemize}
\item the odd summand $\C\{\d z_i\d z_j\} \otimes \C [\del_{z_1}, \del_{z_2}, \del_{z_3}]\delta_{z_i = 0}$ contributes
\[
- q^2 \frac{\chi^{\mf{sl}(3)}_{[0,1]}(t_1,t_2)}{(1-t_1^{-1}q) (1-t_1 t_2^{-1} q) (1-t_2 q)} = - q^2 \frac{t_1  + t_1^{-1} t_2  + t_2^{-1} }{(1-t_1^{-1}q) (1-t_1 t_2^{-1} q) (1-t_2 q)}.
\]
where $\chi^{\mf{sl}(3)}_{[1,0]}(t_1,t_2)$ is the $\mf{sl}(3)$ character of highest weight $[1,0]$.
\item the even summand $\C [\del_{z_1}, \del_{z_2}, \del_{z_3}]\delta_{z_i = 0}$ contributes
\[
q^3 \frac{1}{(1-t_1^{-1}q) (1-t_1 t_2^{-1} q) (1-t_2 q)}.
\]
\item the even summand $\C^2\otimes \C\{\d z_i^{1/2}\} \otimes \C [\del_{z_1}, \del_{z_2}, \del_{z_3}]\delta_{z_i = 0}$ contributes
\[
q^{3/2} \frac{\chi_{1}^{\mf{sl}(2)} (r)}{(1-t_1^{-1}q) (1-t_1 t_2^{-1} q) (1-t_2 q)} = q^{3/2}\frac{(r + r^{-1})}{(1-t_1^{-1}q) (1-t_1 t_2^{-1} q) (1-t_2 q)}.
\]
where $\chi_{1}^{\mf{sl}(2)} (r)$ is the $\mf{sl}(2)$ character of highest weight one.
\end{itemize}
\end{proof}

In terms of the parameters $y_1,y_2,y_3,y,q$ introduced in \ref{} this single particle character reads
\begin{equation}
\label{eqn:6done1}
g_{-1} (y_i,y,q) = \frac{qy + q^2y^{-1} - q^2(y^{-1}_1+y^{-1}_2+y^{-1}_3) + q^3}{(1-y_1q) (1-y_2 q) (1-y_3 q)} .
\end{equation}

The expression matches exactly with the index of the abelian six-dimensional superconformal theory. For example, compare with \cite[Eq. (3.1)]{Kim:2013nva} or \cite[Eq. (3.35)]{Bhattacharya:2008zy}.

From now on, we will give all formulas for the index in terms of the parameters $y_1,y_2,y_3,y,q$.

\parsec[]
We continue to the next step of the decomposition, which is given by  $\mc G^{(0)}_{\C^3}$.

\begin{prop}
The character $\chi \left( \mc U(\mc G^{(0)} _{\C^3}) (0)\right )$  is the plethystic exponential of following expression:
\begin{equation}\label{eqn:6dtwo}
g_{0} (y_i,y,q) = \frac{q^4(y_1+y_2+y_3) + q^2 (y^2 + q + q^2 y^{-2}) - q^{3} (y + q y^{-1})(y_1^{-1} + y_2^{-1} + y_3^{-1})}{(1-y_1q) (1-y_2 q) (1-y_3 q)}.
\end{equation}
\end{prop}
\begin{proof}
As usual, we wish to describe the costalk $\mc G^{(0)}_{\C^3, c} (0)$  more explicitly. By the same argument as in the proofs of propositions \ref{}, \ref{}, we may use elliptic regularity to describe the compactly supported smooth sections on a disc in terms of derivatives of the delta function at the origin in $\C^3$. 

Accordingly, we have contributions from the following summands.
\begin{itemize}
\item An even copy of $\C\{\del_{z_i} \} \otimes \C[\del_{z_1}, \del_{z_2}, \del_{z_3}] \delta_{z_i=0}$. The character of this summand is
\[
q^4 \frac{\chi^{\mf{sl}(3)}_{[1,0]}(y_i)}{(1-y_1q) (1-y_2 q) (1-y_3 q)}  = q^4 \frac{y_1 + y_2 + y_3}{(1-y_1q) (1-y_2 q) (1-y_3 q)}.
\]
\item An even copy of $\mf{sl}(2) \otimes \C[\del_{z_1}, \del_{z_2}, \del_{z_3}] \delta_{z_i=0}$. The character of this summand is
\[
q^3 \frac{\chi_2^{\mf{sl}(2)} (q^{-1/2}y)}{(1-y_1q) (1-y_2 q) (1-y_3 q)}  = \frac{q^2 y^2 + q^3 + q^4 y^{-2}}{(1-y_1q) (1-y_2 q) (1-y_3 q)}.
\]
\item An odd copy of $\C^2\otimes \C\{\d z_i \} \otimes \C[\del_{z_1}, \del_{z_2}, \del_{z_3}] \delta_{z_i=0}$. The character of this summand is 
\[
q^{7/2}\frac{\chi^{\mf{sl}(2)}_{1}(q^{-1/2} y) \, \chi_{[0,1]}^{\mf{sl}(3)} (y_i)}{(1-y_1q) (1-y_2 q) (1-y_3 q)} = q^{3}\frac{(y + q y^{-1})(y_1^{-1} + y_2^{-1} + y_3^{-1})}{(1-y_1q) (1-y_2 q) (1-y_3 q)} .
\]
\end{itemize}
\end{proof}

\parsec[]
Finally, we continue to the general step of the decomposition.

\begin{prop}
Let $j\geq 1$. The character $\chi \left( \mc U(\mc G^{(j)} _{\C^3}) (0)\right )$  is the plethystic exponential of following expression:
\begin{equation}\label{eqn:6dtwo}
g_{j} (y_i,y,q) =  \frac{ q^{3} \left( \begin{aligned} q^{1 + 3 j/2} \chi^{\mf{sl}(2)}_{j}(q^{-1/2} y)\chi^{\mf{sl}(3)}_{[1,0]}(y_i) + \  & q^{3j/2} \chi^{\mf{sl}(2)}_{j+2}(q^{-1/2} y)  \\
 - q^{3(j+1)/2} \chi^{\mf{sl}(2)}_{j-1}(q^{-1/2}y) - \ & q^{-1 + 3(j+1)/2} \chi^{\mf{sl}(2)}_{j+1} (q^{-1/2} y) \chi^{\mf{sl}(3)}_{[0,1]}(y_i) \end{aligned} \right)}{(1-y_1 q)(1-y_2q)(1-y_3q)}.
\end{equation}
\end{prop}
\begin{proof}
We proceed exactly analogously to all the previous cases. Using elliptic regularity on sections of $\mc G^{(j)}_{\C^3}$ over a disc containing the origin, we see that $\mc U (\mc G^{(j)}_{\C^3} )(0)$ is a symmetric algebra on a cochain complex with the following summands

\begin{itemize}
\item An even copy of $\Sym^j (\C^2) \otimes \C\{\del_{z_i}\}\otimes \C[\del_{z_1}, \del_{z_2}, \del_{z_3}]\delta_{z_i=0}\otimes (\del_{z_1}\del_{z_2}\del_{z_3})^{-j/2}$ which contributes 
\begin{equation}
\frac{ q^{3} \left(  q^{1 + 3 j/2} \chi^{\mf{sl}(2)}_{j}(q^{-1/2} y)\chi^{\mf{sl}(3)}_{[1,0]}(y_i) \right)}{(1-y_1 q)(1-y_2q)(1-y_3q)}
\end{equation} 

\item An even copy of $\Sym^{j+2} (\C^2) \otimes \C[\del_{z_1}, \del_{z_2}, \del_{z_3}]\delta_{z_i=0}\otimes (\del_{z_1}\del_{z_2}\del_{z_3})^{-j/2}$ which contributes 
\begin{equation}
\frac{ q^{3} \left( q^{3j/2} \chi^{\mf{sl}(2)}_{j+2}(q^{-1/2} y)  \right)}{(1-y_1 q)(1-y_2q)(1-y_3q)}
\end{equation} 

\item An odd copy of $\Sym^{j-1} (\C^2) \otimes \C[\del_{z_1}, \del_{z_2}, \del_{z_3}]\delta_{z_i=0}\otimes (\del_{z_1}\del_{z_2}\del_{z_3})^{-(j+1)/2}$ which contributes 
\begin{equation}
\frac{ - q^{3} \left( q^{3(j+1)/2} \chi^{\mf{sl}(2)}_{j-1}(q^{-1/2}y) \right)}{(1-y_1 q)(1-y_2q)(1-y_3q)}
\end{equation} 

\item An odd copy of $\Sym^{j+1} (\C^2) \otimes \C\{\d z_i\}\otimes \C[\del_{z_1}, \del_{z_2}, \del_{z_3}]\delta_{z_i=0}\otimes (\del_{z_1}\del_{z_2}\del_{z_3})^{-(j+1)/2}$ which contributes 
\begin{equation}
\frac{ - q^{3} \left( q^{-1 + 3(j+1)/2} \chi^{\mf{sl}(2)}_{j+1} (q^{-1/2} y) \chi^{\mf{sl}(3)}_{[0,1]}(y_i) \right)}{(1-y_1 q)(1-y_2q)(1-y_3q)}
\end{equation} 

\end{itemize}
\end{proof}

\parsec[]
As a consequence, we have that $f_{AdS_7\times S^4} (y_i, y, q) = \sum_{j\geq -1} g_j (y_i, y, q)$, or explicitly:
\begin{align*}
& \frac{q^4(y_1+y_2+y_3)-q^2(y_1^{-1} + y_2^{-1} + y_3^{-1})+ (1-q^3)(yq + y^{-1} q^2)}{(1-y_1 q)(1-y_2 q)(1-y_3 q)(1-yq)(1-y^{-1} q^2)}  \\ 
& =  \frac{qy + q^2y^{-1} - q^2(y^{-1}_1+y^{-1}_2+y^{-1}_3) + q^3}{(1-y_1q) (1-y_2 q) (1-y_3 q)} \\
& +  \frac{q^4(y_1+y_2+y_3) + q^2 (y^2 + q + q^2 y^{-2}) - q^{3} (y + q y^{-1})(y_1^{-1} + y_2^{-1} + y_3^{-1})}{(1-y_1q) (1-y_2 q) (1-y_3 q)} \\
& + \sum_{j\geq 1} \frac{ q^{3} \left( \begin{aligned} q^{1 + 3 j/2} \chi^{\mf{sl}(2)}_{j}(q^{-1/2} y)\chi^{\mf{sl}(3)}_{[1,0]}(y_i) + \  & q^{3j/2} \chi^{\mf{sl}(2)}_{j+2}(q^{-1/2} y)  \\
 - q^{3(j+1)/2} \chi^{\mf{sl}(2)}_{j-1}(q^{-1/2}y) - \ & q^{-1 + 3(j+1)/2} \chi^{\mf{sl}(2)}_{j+1} (q^{-1/2} y) \chi^{\mf{sl}(3)}_{[0,1]}(y_i) \end{aligned} \right)}{(1-y_1 q)(1-y_2q)(1-y_3q)}
\end{align*}

In \cite[Eq. (3.22, 3.23)]{Bhattacharya:2008zy}, the index counting gravitons on $f_{AdS_7\times S^4}$ is expressed as a sum of characters of irreducible representations of the 6d $\mc N = (2,0)$ superconformal algebra. In \cite[Table 24]{cordova2016multiplets} these representations are labeled as $\mc D_1[0,0,0]^{(0,m)}_{2m}$ where $m \geq 1$. The characters of these modules have been computed (see for example \cite[Eq. (166)]{Arai_2020} and match exactly with $g_{m-2}(y_i, y, q)$ after a suitable change of variables. Thus, we conjecture the following

\begin{conj}
For $j\geq -1$, the minimal twist of $\mc D_1[0,0,0]^{(0,j+2)}_{2(j+2)}$ is exactly $\mc G^{(j)}_{\C^3, c}(0)$.
\end{conj}

\begin{rmk}\label{rmk:e36enhance}
As we remarked in \ref{rmk:e16enhance}, this conjecture implies that the minimal twist of $\mc D_1[0,0,0]^{(0,j+2)}_{2(j+2)}$ which is a priori a module for the minimally twisted 6d $\mc N=(2,0)$ superconformal algebra $\mf{osp}(6|2)$, is in fact a module for the larger infinite dimensional super-Lie algebra $E(3|6)$. This can be thought of as analogous to the enhancement of conformal symmetries to the action of the Witt algebra of vector fields in 2d chiral conformal field theory.
\end{rmk}

\end{document}

%The subalgebra $\mc{G}^{(\geq k)} = \oplus_{j \geq k} \mc{G}^{(j)}$ is an ideal for every $k$. 
%This sequence of ideals induces a limit diagram of vector bundles
%\begin{equation}\label{eqn:lim}
%0 = \mc{G} / \mc{G}^{(\geq -1)} \leftarrow \mc{G} / \mc{G}^{(\geq 0)} \leftarrow \mc{G} / \mc{G}^{(\geq 1)} \leftarrow \cdots ,
%\end{equation}
%where $0$ is the zero vector bundle.
%For each $k$, we point out that $\mc{G} / \mc{G}^{(\geq k)}$ is genuinely a finite rank vector bundle on the worldvolume.
%The resulting filtration of the $!$-dual
%\begin{equation}
%\left(\Bar{\pi}_*\Obs_{sugra}\right)^! = \clie_\bullet(\mc{G}_{c})
%\end{equation} 
%of the factorization algebra \eqref{eqn:factgrad} is of the form
%\begin{equation}\label{eqn:fil1}
%\clie_\bullet (\mc{G}_{1,c}) \subset \clie_\bullet (\mc{G}_{2,c}) \subset \cdots \subset \clie_\bullet(\mc{G}_c) .
%\end{equation}
%Similarly, there is a filtration on the factorization algebra $\clie_\bullet (\tilde \mc{G}_c)$ of the form
%\begin{equation}
%\label{eqn:fil2}
%\clie_\bullet(\tilde \mc{G}_{2,c}) \subset \clie_\bullet(\tilde \mc{G}_{3,c}) \subset \cdots \subset \clie_\bullet(\tilde \mc{G}_c).
%\end{equation}



%\end{document}
