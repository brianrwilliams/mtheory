\documentclass[../main.tex]{subfiles}

\begin{document}

\section{Twisted supergravity states}
\label{sec:states}

%The first entry of the AdS/CFT dictionary in traditional treatments is a matching between \textit{supergravity states} and local operators in the CFT.
%The goal of this section is to provide constructions of spaces of twisted supergravity states in our eleven-dimensional model, via geometric quantization. 
%The state space on the twisted version of $AdS_{7}\times S^{4}$ has a remarkable property---it is naturally a module for a certain infinite-dimensional exceptional super Lie algebra. 
%We conclude the section by computing characters for these modules and comparing them with large $N$ indices for fivebranes and membranes in the literature.

In this section, we pursue a check of conjectures \ref{conj:ads4} and \ref{conj:ads7}. We enumerate supergravity states in the twists of the $AdS_4\times S^7$ and $AdS_7\times S^4$ backgrounds and compare them with expressions in the literature enumerating gravitons in these geometries. 

We begin by giving a definition of supergravity states suited to our context. 
The definition is meant to codify the following situation. 
Suppose we have a bulk gravitational theory defined on a background of the form $AdS_{d+1} \times S^{d^\prime}$ where the conformal boundary of $AdS_{d+1}$ is a $d$-dimensional manifold $M$. 
We can compactify the theory, retaining all the Kaluza-Klein harmonics, to get a theory on $AdS_{d+1}$. 
A gravitational state is traditionally defined to be a solution to linearized equations of motion in this compactified theory with a given boundary value on $M^{d}$ \cite{WittenAdS}. Often, this definition is made in situations where the relevant boundary value problem has a unique solution, in which case one may label states by the corresponding boundary values. Moreover, one may think of such boundary values as arising from modifications of a vacuum boundary condition at a point.

For twists of supergravity, we procedurally implement this as follows. First, we wish to describe a model for sphere compactification on twisted $AdS$ backgrounds. 
As we saw in the last section, our proposal for describing twists backgrounds of the form $AdS_{d+1}\times S^{d^\prime}$ involve elliptic complexes defined on certain manifolds of the form 
\beqn
X = \operatorname{Tot }(V\to Z)\setminus 0(Z)
\eeqn
Note that $Z$ can be written as a sphere bundle over $\R_{>0}\times Z$ - the sphere compactification of our theory on $X$ is given by the pushforward to $\R_{>0}\times Z$. 
Such sphere compactifications can be described using a method for computing pushforwards of modules for Lie algebroids associated to foliations \cite[section 4.2]{KormanThesis}, \cite{KamberTondeur}.

The compactified theory will admit a natural boundary condition at $\{\infty\}\times Z\subset \R_{>0}\times Z$.
Such a boundary condition also admits a perturbative description in terms of some elliptic complex $\mc L$ on $Z$.
In fact, the presence of extra differentials involving bracketing with the flux sourced by branes wrapping $Z$ will induce an odd Poisson structure on $\mc L$.
Given an elliptic complex $\cL$ underlying a perturbative classical field theory, the algebra of observables $\cO(\cL)$ carries the structure of a $\mb{P}_0$-factorization algebra \cite{CG1}, \cite{CG2}. 
Associated to a factorization algebra, we can define a space of local operators, which is the costalk at a point of the underlying cosheaf.
The sought-after spaces of supergravity states will be a space of local operators associated to $\cO(\cL)$. 

\begin{rmk}
Crucially, we will only implement this procedure at the level of the free limits of the theories defined in \ref{defn:ads4}, \ref{defn:ads7} - this will suffice for the purposes of extracting spaces of states and will allow us to forgo discussion involving homotopy transfer of $L_\infty$ structure. As such, the boundary conditions we specify at $\{\infty\}\times Z$ will in fact be abelian local Lie algebras. On the other hand, in subsection \ref{sec:transversebc} we will identify boundary conditions for the pushforward of the \textit{interacting} theories that exist after turning off the flux, which yield the same spaces of supergravity states. 
\end{rmk}

Let us now carry out this construction for the theories defined in \ref{defn:ads4}, \ref{defn:ads7}.

%\parsec[s:brkevin]
%As a way to highlight the key aspects of the construction, we detail the ingredients in the simplified model of Costello's twisted $M$ theory. The relevant local calculation can be found in the appendix of \cite{}; our goal here is to simply identify the salient global features that allow one to reduce to said local calculation.

%We consider the theory on $M = \text{Tot} (\R\oplus K_{C})$, with some number of twisted `fivebranes' wrapping the zero section
%\[
%0 \times C \subset \R \times \T^* C .
%\]
%Denote by $t$ the real coordinate and by $w$ the fiber coordinate in $\T^* C$. We wish to describe the complement of the zero section $\mathring M = M - 0 \times C$.

%Note that the bundle $\R\oplus K_{C}$ is equipped with a partially flat connection - this data equips the total space $M$ with the data of a transversely holomorphic foliation (THF) \cite{DuchampKalka}.

%If we choose a fiberwise partially hermitian metric on the bundle $\R \oplus K_C$ we obtain a projection $p: \R \times \T^*C \to \R_{+} \times C$ which combines the fiberwise norm with the natural bundle projection. The restriction $p| \mathring M$ equips $\mathring M$ with the structure of an $S^{2}$-bundle over $\R_{>0}\times C$. Moreover, the partial flat connection on $\R\oplus K_{C}$ induces a partially flat connection on $\mathring M$. As part of this data, each of the fiber spheres is equipped with a complex foliation of rank 1.

%Compactification amounts to pushing forward a local $L_{\infty}$-algebra along $p| \mathring M$. The result is a theory with infinitely many Kaluza--Klein modes along the fiber spheres. In the holomorphic-topological setting, the Kaluza-Klein modes will be modeled by a variant of Cauchy-Riemann cohomology.

%Moreover, including the flux sourced by the brane deforms this structure. The lowest lying Kaluza-Klein modes in the deformed theory are equivalent to three-dimensional Chern-Simons theory.

%For sake of analogy, we think of the resulting deformation as being a twisted version of $AdS_3 \times S^2$.\footnote{It is an interesting question if this corresponds the actual twist of a five--dimensional supersymmetric background of this form.}
%We proceed to describe the twisted version of states at the boundary of this version of $AdS$.
%We first proceed before turning on the flux sourced by the brane.

%The theory admits a natural `vacuum' boundary condition at $r=0$.
%In local coordinates, these are fields $\alpha(t,z,w)$ on the complement to the brane which extend to regular functions along the brane.

%The `supergravity states' are, by definition, fields which satisfy the linearized equations of motion and satisfy the vacuum boundary condition except at a single point.
%The linearized equations of motion are simply $(\d_{dR} + \dbar) \alpha = 0$.
%Thus, up to equivalence, all solutions to the linearized equations of motion are constant in the real variable $t$, and holomorphic in $z,w$.

%Modifications of the boundary condition at the point~$z = 0$ on the boundary take the form
%\[
%\alpha = f(w) \delta^{(r)}_{z=0}
%\]
%where $f$ is some holomorphic function.
%Here $\delta^{(r)}_{z=0}$ denotes the $r$th derivative of the $\delta$-function at $z=0$.
%It is convenient to parameterize such boundary modifications algebraically by expressions of the form
%\[
%\alpha_{k,r} = w^k \delta^{(r)}_{z=0} .
%\]
%Linear combinations of such states form a dense subspace of all possible modifications at the boundary.

%The reason that the boundary modifications take this form can be seen by understanding in more explicit terms the vacuum boundary condition.
%The phase space at the boundary $C$ can be identified with the following cohomology
%\[
%\Omega^{0,\bullet}(C) \otimes \mc{A}^{0;\bullet}(\R \times \C - 0) [1]
%\]
%where $\mc{A}^{0;\bullet}$ denotes the mixed de Rham--Dolbeault cohomology of $\R \times \C - 0$ as a manifold equipped with a transversely holomorphic foliation \cite{DuchampKalka}.
%We refer to the section below for a reminder on this geometric structure.

%The phase space is equipped with a natural symplectic form given by
%\[
%\int_C \d z \oint_{S^2} \d w \, \alpha \wedge \alpha' .
%\]
%There is a natural Lagrangian inside of the phase space which consists of linear combinations of elements $\alpha(z) \otimes f(t,w)$ where $\alpha(z) \in \Omega^{0,\bullet}(C)$ and $f(t,w)$ is a smooth function on $\R \times \C - 0$ which extends to zero.
%The linearized equations of motion simply say that $\alpha$ is holomorphic, $f$ is independent of $t$ and depends holomorphically on $w$
\subsection{States on twisted $AdS_4\times S^7$}\label{s:ads4states}
We begin more generally with the input being a three-manifold $M$ equipped with a THF structure of codimension one.
As in \S \ref{s:??} the elliptic complex $\cA^\bu_M$ computes the THF cohomology of $M$.
We denote by $K$ the holomorphic canonical bundle of $M$ which has a local frame given by $\d z$.
The case of $AdS_4 \times S^7$ corresponds to taking $M = \R \times \C$, which we do throughout this section.

We will also fix a fourth-root $K^{1/4}$ of this bundle.
Following the above prescription, we consider the $S^7$ bundle 
\beqn\label{eqn:geomads4}
\begin{tikzcd}
S^7 \ar[r] & \operatorname{Tot} (K^{1/4} \otimes \C^4 \to M) \setminus 0(M) \ar[d, "p"] \\ & \R_{>0}\times M
\end{tikzcd}
\eeqn
We wish to describe the free limit of the pushforward $p_* \mc E^N_{AdS_4}$ as a sheaf of cochain complexes on $\R_{>0}\times M$. 

\begin{prop}\label{prop:pushads4}
The pushforward is given by the sheaf of complexes $\Omega^\bullet_{\R_{>0}} \otimes \cA^{\bullet}_{M} (\mc V^N)$ on $\R_{>0}\times M$, where $\mc V^N$ is the following complex of vector bundles on $M$:

\begin{equation}
  \begin{tikzcd}[row sep = 1 ex]
    \ul{-} & \ul{+} \\ 
H^\bullet (\C^{4}\setminus 0, \T) \otimes \mc{O}\ar[r, "\del^W_\Omega" description] & H^\bullet (\C^{4}\setminus 0) \otimes \mc O \\
H^\bullet (\C^{4}\setminus 0)  \otimes \T\ar[ur, "\del^M_\Omega" description] \\
H^\bullet (\C^{4}\setminus 0)\otimes \mc{O}\ar[r, "\del_M" description]\ar[dr, "\del_W" description] & H^\bullet (\C^{4}\setminus 0) \otimes \Omega^1 \\ & H^\bullet (\C^{4}\setminus 0, \Omega^1)  \otimes \mc O
\end{tikzcd}
\end{equation}
where the differentials are as follows:
\begin{itemize}
\item The differentials $\del_\Omega^M$ and $\del_\Omega^W$ are the divergence operators along the base and fiber respectively.
\item The differentials $\del_M$ and $\del_W$ are components of the holomorphic de Rham differentials along the base and fiber respectively.
\item  Internal to each summand is a differential given by bracketing with the flux $NF_{M2}$, see equation \eqref{eqn:FM2}. 
\end{itemize}
\end{prop} 

Before proceeding with the proof, we explicate the internal differential in the third item above. Recall that for $\mc{F} = \mc{O}, \T$, or $\Omega^1$, the cohomology $H^\bullet(\C^4 \setminus 0, \mc{F})$ is concentrated in degrees $0$ and $3$. We will make use of the following dense embeddings.
\begin{align*}
\C[w_1,\ldots, w_4] & \hookrightarrow H^0(\C^4 \setminus 0, \mc{O}) \\ 
\C[w_1,\ldots, w_4] \{\partial_{w_a}\} & \hookrightarrow H^0(\C^4 \setminus 0, \T) \\
\C[w_1,\ldots, w_4] \{\d w_a\} & \hookrightarrow H^0(\C^4 \setminus 0, \Omega^1) 
\end{align*}
and
\begin{align*}
(w_1\cdots w_4)^{-1} \C[w_1^{-1},\ldots, w_4^{-1}] & \hookrightarrow H^3(\C^4 \setminus 0, \mc{O}) \\ 
(w_1\cdots w_4)^{-1} \C[w_1^{-1},\ldots, w_4^{-1}] \{\partial_{w_a}\} & \hookrightarrow H^3(\C^4 \setminus 0, \T) \\
(w_1\cdots w_4)^{-1} \C[w_1^{-1},\ldots, w_4^{-1}] \{\d w_a\} & \hookrightarrow H^3(\C^4 \setminus 0, \Omega^1) .
\end{align*}

The flux $NF_{M2}$ of lemma \ref{lem:m2flux} is represented by the section $N (w_1\cdots w_4)^{-1}\del_z \in H^3 (\C^4\setminus 0)\otimes \T$ and acts on each summand by Lie derivative along $z$ and multiplying by $(w_1\cdots w_4)^{-1}$.

\begin{proof}
To compute the pushforward of this sort of elliptic complex associated to a holomorphic-topological field theory along a map of THF manifolds, we can use a result of \cite[section 4.2]{KormanThesis}, \cite{KamberTondeur} for describing direct images of Lie algebroid modules along maps of Lie algebroids. Schematically, if we have a proper submersion $p : X\to Z$ of THF manifolds, and a sheaf of complexes $\mc E$ on $X$ which resolves sections of some bundle flat along the leaves of the foliation on $X$, then the pushforward $p_*\mc E$ has a model as a partially flat bundle on $Z$. The fiber of this partially flat bundle on $Z$ is the THF cohomology of the fiber of $p$, with respect to the induced foliation on the fiber, with coefficients in the pullback of the bundle to the fiber. In the case of a holomorphic submersion, this recovers the usual construction of the Gauss-Manin connection for instance. 

In our case, the pushforward $p_* \mc E_{AdS_4}$ is a complex of $\Omega^\bullet_{\R_{>0}} \otimes \cA^\bu_M$- modules given by 

\begin{equation}
  \label{eqn:ads4ss} 
  \begin{tikzcd}[row sep = 1 ex]
    \ul{-} & \ul{+} \\ 
H^\bullet_{\mr{THF}}(S^7, \T) \otimes \mc{O}\ar[r, "\del^W_\Omega" description] & H^\bullet_{\mr{THF}} (S^7) \otimes \mc O \\
H^\bullet_{\mr{THF}} (S^7)  \otimes \T\ar[ur, "\del^X_\Omega" description] \\
H^\bullet_{\mr{THF}} (S^7)\otimes \mc{O}\ar[r, "\del_X" description]\ar[dr, "\del_W" description] & H^\bullet_{\mr{THF}} (S^7) \otimes \Omega^1
 \\ & H^\bullet_{\mr{THF}} (S^7, \Omega^1)  \otimes \mc O
 \end{tikzcd}
\end{equation}

Then the proposition follows from the fact that the map $S^7\to \C^4\setminus \{0\}$ induces an isomorphism in THF cohomology.
\end{proof}

\parsec{}
We proceed to define a natural boundary condition of the twisted $AdS_4$ theory at $\{\infty\} \times M \subset \R_{>0} \times M$. 
Such a boundary condition is specified by a Lagrangian in the phase space whose symplectic structure uses Serre duality on the THF manifold $M$ together with the higher residue pairing along the THF structure of the sphere $S^7$. 
The restriction $(p_* \mc E^N_{AdS_4})|_{\{\infty\}\times \C}$ describes the phase space and is easily seen to be the complex of vector bundles $\cA_M^\bu (\mc V^N)$. 
We wish to describe a shifted Lagrangian therein.

We begin by rewriting the phase space in the following form. 
Using the holomorphic volume form, there is a higher residue pairing 
\[H^0(\C^4\setminus 0, \cO) \otimes H^3(\C^4\setminus 0, \cO) \to \C;\]
together with the natural pairings between $T, \Omega^1$ and the integration along $M$, this equips $\cA_M^\bu (\mc V^N)$ with a local even-shifted symplectic structure.

Via this symplectic structure, we can identify the phase space with a twisted cotangent bundle
\beqn\label{eqn:cotm2}
\cA_M^\bu (\mc V^N) = T^* _{\alpha} \left (\begin{tikzcd}[row sep = 1 ex]
 \ul{-} & \ul{+} \\ 
\C[w_1,\ldots, w_4] \{\partial_{w_a}\} \otimes \cA_M^\bu \ar[r, "\del^W_\Omega" description] & \C[w_1,\ldots, w_4] \otimes \cA_M^\bu  \\
\C[w_1,\ldots, w_4] \otimes \cA_M^\bu (\T) \ar[ur, "\del^M_\Omega" description] \\
\C[w_1,\ldots, w_4] \otimes \cA_M^\bu \ar[r, "\del_M" description]\ar[dr, "\del_W" description] & \C[w_1,\ldots, w_4] \otimes \cA_M^\bu(\Omega^1)\\ & \C[w_1,\ldots, w_4] \{\d w_a\} \otimes \cA_M^\bu  \end{tikzcd} \right) 
\eeqn
For now, the subscript of $\alpha$ is just meant to indicate that the extra differential given by bracketing with the flux constitutes a deformation of the cotangent bundle. 

Thus we see that a natural Lagrangian in the phase space is given by the sheaf of cochain complexes $\cA^\bu_M(\mc L^N_{AdS_4})$ where $\mc L^N_{AdS_4}$ is the following complex of holomorphic vector bundles on $M$:
 
 \begin{equation}
 \begin{tikzcd}[row sep = 1 ex]
       \ul{+} & \ul{-} \\ 
(w_1\cdots w_4)^{-1} \C[w_1^{-1},\ldots, w_4^{-1}] \{\partial_{w_a}\} \otimes \mc{O}\ar[r, "\del^W_\Omega" description] & (w_1\cdots w_4)^{-1} \C[w_1^{-1},\ldots, w_4^{-1}]  \otimes \mc O \\
(w_1\cdots w_4)^{-1} \C[w_1^{-1},\ldots, w_4^{-1}]   \otimes \T\ar[ur, "\del^M_\Omega" description] \\
(w_1\cdots w_4)^{-1} \C[w_1^{-1},\ldots, w_4^{-1}] \otimes \mc{O}\ar[r, "\del_M" description]\ar[dr, "\del_W" description] & (w_1\cdots w_4)^{-1} \C[w_1^{-1},\ldots, w_4^{-1}] \otimes \Omega^1
\\ & (w_1\cdots w_4)^{-1} \C[w_1^{-1},\ldots, w_4^{-1}] \{\d w_a\}  \otimes \mc O \end{tikzcd}
\end{equation}

We denote by $\Obs_{AdS_4}$ the factorization algebra on $M = \R\times \C$ which assigns to an open set $U \subset \R \times \C$ the cochain complex
\beqn
 \Obs^N_{AdS_4} (U) = \cO\left(\cA^\bu(U,\mc L^N_{AdS_4}) \right ) .
\eeqn
%Here, $\cA^\bu(U, \cL^N_{AdS_4}$ 

\begin{defn}\label{defn:ads4states}
The \emph{space of supergravity states} $\cS^N_{AdS_4}$ on twisted $AdS_4\times S^7$ is given by the cochain complex given by the costalk at zero of $\Obs_{AdS_4}^N$:
\beqn
\cS^N_{AdS_4} \define \Obs^N_{AdS_4} (0) .
\eeqn
\end{defn}

\parsec{}
We explicate definition \ref{defn:ads4states} and characterize the cochain complex underlying the space of supergravity states.

\begin{lem}\label{lem:ads4states}
The space of supergravity states on twisted $AdS_4\times S^7$ is the symmetric algebra 
\beqn
\cS^N_{AdS_4} = \Sym (\mc H_{AdS_4})
\eeqn
where $\mc H_{AdS_4}$ is the cochain complex
 \begin{equation} 
 \begin{tikzcd}[row sep = 1 ex]
    \ul{-} & \ul{+} \\ 
\C[w_1,\ldots, w_4] \{\partial_{w_a}\} \otimes \C[\del_z]\delta_{z=0}\ar[r, "\del^W_\Omega" description] & \C[w_1,\ldots, w_4]  \otimes \C[\del_z]\delta_{z=0} \\
\C[w_1,\ldots, w_4] \del_z  \otimes \C[\del_z]\delta_{z=0}\ar[ur, "\del^M_\Omega" description] \\
\C[w_1,\ldots, w_4] \otimes \C[\del_z]\delta_{z=0}\ar[r, "\del_Z" description]\ar[dr, "\del_W" description] & \C[w_1,\ldots, w_4] \d z \otimes \C[\del_z]\delta_{z=0} \\ & \C[w_1,\ldots, w_4] \{\d w_a\}  \otimes \C[\del_z]\delta_{z=0}
\end{tikzcd}
\end{equation}
\end{lem}

\begin{proof}
To characterize the costalk at $0 \in \R \times \C$ of a factorization algebra, we consider a nested sequence of open sets containing the origin and compute the limit of the value of the factorization algebra over this sequence. Consider open sets in $\R\times \C$ of the form $I\times D$ where $I\subset \R$ is an interval and $D\subset \C$ is a disc. The sections of the sheaf of cochain complexes $\cA^\bu_{\R \times \C}(\mc L^N_{AdS_4\times S^7} ) $ over this open set is given by \begin{equation}
\begin{tikzcd}[row sep = 1 ex]
    \ul{+} & \ul{-} \\
(w_1\cdots w_4)^{-1} \C[w_1^{-1},\ldots, w_4^{-1}] \{\partial_{w_a}\} \otimes \mc{O}(D)\ar[r, "\del^W_\Omega" description] & (w_1\cdots w_4)^{-1} \C[w_1^{-1},\ldots, w_4^{-1}]  \otimes \mc O(D) \\
(w_1\cdots w_4)^{-1} \C[w_1^{-1},\ldots, w_4^{-1}]   \otimes \Gamma (D, \T)\ar[ur, "\del^M_\Omega" description] \\
(w_1\cdots w_4)^{-1} \C[w_1^{-1},\ldots, w_4^{-1}] \otimes \mc{O}(D)\ar[r, "\del_Z" description]\ar[dr, "\del_W" description] & (w_1\cdots w_4)^{-1} \C[w_1^{-1},\ldots, w_4^{-1}] \otimes \Gamma (D, \Omega^1) \\ & (w_1\cdots w_4)^{-1} \C[w_1^{-1},\ldots, w_4^{-1}] \{\d w_a\}  \otimes \mc O(D)
\end{tikzcd}
\end{equation}

Now note that there is a canonical map $\mc O(D) \to \C[\![z]\!] $ given by taking the Taylor expansion at the origin. Given a functional on the fields that only depends on the value of their derivatives at the origin, then the functional must factor through the Taylor expansion. Therefore, we have that the costalk of our factorization algebra is given by 

\begin{equation}
\cO\left (\begin{tikzcd}[row sep = 1 ex]
        \ul{+} & \ul{-} \\ 
(w_1\cdots w_4)^{-1} \C[w_1^{-1},\ldots, w_4^{-1}] \{\partial_{w_a}\} \otimes \C[\![z]\!] \ar[r, "\del^W_\Omega" description] & (w_1\cdots w_4)^{-1} \C[w_1^{-1},\ldots, w_4^{-1}]  \otimes \C[\![z]\!] \\
(w_1\cdots w_4)^{-1} \C[w_1^{-1},\ldots, w_4^{-1}]   \otimes \C[\![z]\!]\del_z\ar[ur, "\del^M_\Omega" description] \\
(w_1\cdots w_4)^{-1} \C[w_1^{-1},\ldots, w_4^{-1}] \otimes \C[\![z]\!]\ar[r, "\del_Z" description]\ar[dr, "\del_W" description] & (w_1\cdots w_4)^{-1} \C[w_1^{-1},\ldots, w_4^{-1}] \otimes \C[\![z]\!]\d z \\ & (w_1\cdots w_4)^{-1} \C[w_1^{-1},\ldots, w_4^{-1}] \{\d w_a\}  \otimes \C[\![z]\!]
\end{tikzcd} \right )
\end{equation}

The definition of algebraic functions $\cO(-)$ involves the continuous linear dual of a chain complex of topological vector spaces. 
The duals of each of the tensor factors are as follows:

\begin{itemize}
\item There is an isomorphism between the continuous linear dual of $\C[\![ z]\!]$ and $\C[\del_z]\delta_{z=0}$: every continuous linear functional on $\C[\![ z]\!]$ is given by a derivative of the $\delta$-function at zero. 
\item The higher residue pairing lets us identify the continuous linear dual of $(w_1\cdots w_4)^{-1} \C[w_1^{-1},\ldots, w_4^{-1}]$ with $\C[w_1,\cdots, w_4]$.
\item The tensor factors involving one-forms and vector fields are dual to each other in the obvious way.
\end{itemize}

Thus, we see that $\mc H_{AdS_4}$ is indeed as claimed.
\end{proof}

\parsec{}
We proceed to computing a local character for the factorization algebra defined in \ref{defn:ads4states}; thanks to lemma \ref{lem:ads4states} we can compute this as a character of $\Sym(\mc H_{AdS_4})$. 

We first observe the following action of $\mf{sl}(4)\oplus \mf {sl}(2)$ on $\mc H_{AdS_4}$. The $\mf{sl}(4)$ summand acts on the tensor factor $\C[w_1,w_2,w_3,w_4]$ in the obvious way - it's the symmetric algebra on the fundamental representation. The $\mf{sl}(2)$ summand acts by bracketing with the vector fields 
\[
\frac{\del}{\del z} ,\quad z \frac{\del}{\del z} - \frac14 \sum_{a=1}^4 w_a \frac{\del}{\del w_a} , \quad z \left(z \frac{\del}{\del z} - \frac12 \sum_{a=1}^4 w_a \frac{\del}{\del w_a} \right).
\]

We choose the following explicit generators for a choice of Cartan as follows:
\begin{itemize}
\item $t_1, t_2, t_3$ denote generators for the Cartan of $\mf {sl}_4$ which is spanned by the vector fields \[h_1 = w_1\frac{\del}{\del w_1}- w_4\frac{\del}{\del w_4}, \ \ \ \ \ h_2 = w_2\frac{\del}{\del w_2}- w_4\frac{\del}{\del w_4}, \ \ \ \ h_3 = w_3\frac{\del}{\del w_3}- w_4\frac{\del}{\del w_4}\]
\item $q$ denotes a generator for the Cartan of $\mf{sl}_2$ which is spanned by the vector field \[\Delta = \frac14\sum_{a=1}^4 w_a \frac{\del }{\del w_a}-z\frac{\del}{\del z}.\]
\end{itemize}

The weights of $\mc H_{AdS_4}$ with respect to the generators of this Cartan subalgebra are entirely determined by the weights of the holomorphic coordinates $z, w_a, a= 1, \cdots, 4$, which we summarize in table \ref{tbl:sugraM2}

\begin{table}
\begin{center}
\begin{tabular}{c c c c c c}
  & $z$ & $w_1$ & $w_2$ & $w_{3}$ & $w_{4}$ \\
  \hline
  $t_{1}$ & 0 & 1 & $0$ & 0 & -1 \\
  $t_{2}$ & 0 & 0 &  1 & 0 & -1 \\
  $t_{3}$ & 0 & 0 & 0 & 1 & $-1$ \\
  $q$ & $-1$ & $\frac14$ & $\frac14$ & $\frac14$ & $\frac14$
\end{tabular}
\caption{Fugacities for the fields of the holomorphic twist of eleven-dimensional supergravity for the geometry $\R \times \C^5 \setminus (\R\times \C^2)$.}
\label{tbl:sugraM2}
\end{center}
\end{table}

With this in hand, we wish to compute the character of the space of supergravity states $\Sym (\mc H_{AdS_4})$. Note that the space of supergravity states was defined to be a symmetric algebra - therefore its character can be computed using plethystic exponentiation of the character of $\mc H_{AdS_4}$ - the latter may be referred to as a \textit{single particle index} and is defined by 

\[
f_{AdS_4} (t_1, t_2, t_3, q) = \tr_{\mc H_{AdS_4}} (-1)^F q^\Delta t_1^{h_1} t_2^{h_2} t_3^{h_3}.
\]

\begin{prop}\label{prop:ads4index}
The single particle index of the space of supergravity states $\mc H_{AdS_4}$ is given by 
\[
f_{AdS_4} (t_1, t_2, t_3, q)  = \frac{q\left (\begin{aligned} q^{1/4}(t_1+ t_2 + t_3+t_1^{-1}t_2^{-1}t_3^{-1}) &+ q^{-1} \\- q^{-1/4}(t_1^{-1}+ t_2^{-1} + t_3^{-1}+t_1t_2t_3) &- q   \end{aligned}\right)}{(1-q)(1-q^{1/4}t_1)(1-q^{1/4}t_2)(1-q^{1/4}t_3)(1-q^{1/4}t_1^{-1}t_2^{-1}t_3^{-1})}
\]
\end{prop}
\begin{proof}
The two summands not involving holomorphic vector fields or forms appear with opposite parity, so their contributions to the character cancel. For the remaining summands, It is straightforward to compute the character of each tensor factor:
\begin{itemize}
\item The factor $\C[\del_z]\delta_{z=0}$ contributes a factor of \[\frac{q}{1-q}.\]
\item The tensor factor $\C[w_1,\cdots, w_4]$ contributes a factor of \[\frac{1}{(1-q^{1/4}t_1)(1-q^{1/4}t_2)(1-q^{1/4}t_3)(1-q^{1/4}t_1^{-1}t_2^{-1}t_3^{-1})}.\]
\item The tensor factors involving vector fields and forms contribute a factor of \[-q^{-1/4}(t_1^{-1}+ t_2^{-1} + t_3^{-1}+t_1t_2t_3) + q^{1/4}(t_1+ t_2 + t_3+t_1^{-1}t_2^{-1}t_3^{-1}) - q + q^{-1}.\]
\end{itemize}
\end{proof}

\parsec[] Upon subtracting one and making the substitution
\[
q = x^2, \ \ \ \ \ t_1 = (y_2y_3)^{1/2}/y_1^{1/2}, \ \ \ \ \ t_2 = (y_1y_3)^{1/2}/y_2^{1/2}, \ \ \ \ \ t_3 = (y_1y_2)^{1/2}/y_3^{1/2}
\]
this character matches the expression in \cite[equation (2.17)]{Bhattacharya:2008zy}. The discrepancy of one is accounted for by a zero mode that we have introduced in writing our theory in such a way that M2 branes couple electrically. Indeed, this is an avatar of the central element in the central extension $\widehat{E(5|10)}$ of section \ref{thm:global}.

\subsection{States on twisted $AdS_7\times S^{4}$}
\label{s:ads7states}
Next, we consider the sphere reduction of $\mc E^N_{AdS_7}$.
We begin more generally with the case that $X$ is an arbitrary complex three-fold equipped with a square-root of its canonical bundle.
The case of twisted $AdS_7 \times S^4$ corresponds to $X = \C^3$.
As before, we consider the $S^4$ bundle 
\beqn\label{eqn:geomads7}
\begin{tikzcd}
S^4 \ar[r] & \operatorname{Tot} (\R\oplus K^{1/2}\otimes \C^2\to X) \setminus 0(X) \ar[d, "p"] \\ & \R_{>0}\times X
\end{tikzcd}
\eeqn

We wish to describe the free limit of the pushforward $p_* \mc E_{AdS_7}$ as a sheaf of cochain complexes on $\R_{>0}\times X$. 

\begin{prop}\label{prop:ads7push}
The pushforward $p_* \mc E^N_{AdS_7}$ is given by the sheaf of cochain complexes $\Omega^\bullet_{\R_{>0}} \otimes \Omega^{0,\bullet}_{X} (\mc V^N_X)$ where $\mc V^N_{X}$ is the following complex of vector bundles on~$X$:

\begin{equation}
  \label{eqn:ads7push} 
  \begin{tikzcd}[row sep = 1 ex]
    \ul{-} & \ul{+} \\
H^\bullet_{\mr{THF}} \left ((\R\times \C^2)\setminus 0, \T \right ) \otimes \mc{O}\ar[r, "\del^W_\Omega" description]\ar[ddr, dotted] & H^\bullet_{\mr{THF}} \left ((\R\times \C^2)\setminus 0 \right ) \otimes \mc O \\
H^\bullet_{\mr{THF}} \left ((\R\times \C^2)\setminus 0 \right )  \otimes \T\ar[ur, "\del^X_\Omega" description] \\
H^\bullet_{\mr{THF}} \left ((\R\times \C^2)\setminus 0 \right )\otimes \mc{O}\ar[r, "\del_X" description]\ar[dr, "\del_W" description] & H^\bullet_{\mr{THF}} \left ((\R\times \C^2)\setminus 0 \right ) \otimes \Omega^1 \\ & H^\bullet_{\mr{THF}} \left ((\R\times \C^2)\setminus 0, \Omega^1 \right )  \otimes \mc O\ar[uul, dotted]
\end{tikzcd}
\end{equation}

where the differentials are as follows:

\begin{itemize}
\item The differentials $\del_\Omega^X$ and $\del_\Omega^W$ are the divergence operators along the base and fiber respectively.
\item The differentials $\del_X$ and $\del_W$ are components of the holomorphic deRham differentials along the base and fiber respectively.
\item The dotted arrows are $N$ dependent differentials roughly given by bracketing with the flux $N F_{M5}$, see \eqref{eqn:FM5}, and are explicated below. 
\end{itemize}
\end{prop}

Before proceeding with the proof, it will again be useful to explicate the internal differentials above. The THF cohomology of $(\R\times \C^2) \setminus 0$ possibly with coefficients in a sheaf $\mc F$ equipped with a partial flat connection along the leaves of the THF can be described as the cohomology of the following quotient of the deRham complex 
\[\Omega^\bullet \left ((\R\times \C^2) \setminus 0 \right) / (\d w_1, \d w_2) .\] 

The cohomology is accordingly concentrated in degrees zero and two. We will make use of the dense embeddings
\begin{align*}
\C[w_1,w_2] \hookrightarrow & H^0 \bigg( \Omega^\bullet\left(\C^2_w \times \R \setminus 0 \right) \, / \, (\d w_1, \d w_2) \bigg) \\
w_{1}^{-1} w_2^{-1} \C[w_1^{-1},w_2^{-1}] \hookrightarrow & H^2 \bigg( \Omega^\bullet\left(\C^2_w \times \R \setminus 0 \right) \, / \, (\d w_1, \d w_2) \bigg)
\end{align*}
along with the analogous versions with coefficients in the sheaf $\mc F = \Omega^1, T$. 

The flux $NF_{M5}$ is then represented by a class of the form $N(w_1w_2)^{-1} \d w_1 \d w_2 \in (w_1w_2)^{-1} \C[w_1^{-1}, w_2^{-1}]\otimes \Omega^2$. The dotted differentials in equation \ref{eqn:ads7push} are then explicitly given by maps 
\begin{align*}
\C[w_1,w_2]\{\del_{w_a}\} \otimes \mc O &\to (w_1w_2)^{-1}\C[w_1^{-1}, w_2^{-1}] \{\d w_a\}\otimes \mc O \\
\C [w_1, w_2] \otimes \Omega^1 &\to (w_1w_2)^{-1}\C[w_1^{-1}, w_2^{-1}] \otimes \T  
\end{align*}
where the first map is given by contracting with $\d w_1 \d w_2$ and multiplying by $(w_1w_2)^{-1}$, while the second map is given by applying $\del_X$, wedging with the $\d w_1 \d w_2$ and contracting with the inverse of the holomorphic volume form on $\C^5$ to get a vector field along $X$. 

\begin{proof}
The proof is exactly analogous to that of proposition \ref{prop:pushads4} . Using results from \cite[section 4.2]{KormanThesis}, \cite{KamberTondeur}  to compute the pushforward, we find a sheaf of $\Omega^\bullet_{\R_{>0}}\otimes \Omega^{0,\bullet}_{X}$-modules whose sections have a tensor factor given by the THF cohomology of $S^4$. Next, we use the isomorphism in THF cohomology afforded by the deformation retraction of $(\R\times \C^2)\setminus 0$ onto $S^4$. 
\end{proof}

\parsec{}

As in the previous subsection, we ask for a boundary condition we can place on the fields at $\{\infty\}\times X\subset \R_{>0}\times X$. The phase space, given by $(p_* \mc E^N_{AdS_7})|_{\{\infty\} \times X}$, is seen to be $\Omega^{0,\bullet}_{X} (\mc V_{X} )$, and we search for a shifted Lagrangian therein. 

Completely analogously to before, we may rewrite the phase space as a shifted cotangent bundle. The higher residue pairing, the natural pairing between $T, \Omega^1$, and Serre duality along $X$, all conspire to give the phase space an even-shifted symplectic structure. Together with this, we may once again identify the phase space with a twisted cotangent bundle 

\beqn\label{eqn:cotm5}
\Omega^{0,\bullet}_{X} (\mc V_{X}) = T^* _{\alpha_N} \left (\begin{tikzcd}[row sep = 1 ex] 
\ul{-} & \ul{+} \\
\C[w_1,w_2] \{\partial_{w_a}\} \otimes \Omega^{0,\bullet}_{X} \ar[r, "\del^W_\Omega" description] & \C[w_1,w_2] \otimes \Omega^{0,\bullet}_{X}  \\
\C[w_1,w_2] \otimes \Omega^{0,\bullet}_{X} (\T)\ar[ur, "\del^X_\Omega" description] \\
\C[w_1,w_2] \otimes \Omega^{0,\bullet}_{X} \ar[r, "\del_X" description]\ar[dr, "\del_W" description] & \C[w_1,w_2]\otimes \Omega^{0,\bullet}_{X} (\Omega^1) \\ &  \C[w_1,w_2] \{\d w_a\} \otimes \Omega^{0,\bullet}_{X} 
\end{tikzcd} \right) 
\eeqn
Analogously to the case of $AdS_4\times S^7$, $\alpha_N$ here denotes the extra $N$-dependent differential induced by bracketing with the flux, which deforms the cotangent bundle. 

A natural Lagrangian in the phase space is thus given by $\Omega^{0,\bullet}_{X} (\mc L_{AdS_7})$ where $\mc L^N_{AdS_7}$ is the complex of vector bundles given by  
\begin{equation}
\begin{tikzcd}[row sep = 1 ex]
    \ul{-} & \ul{+} \\
(w_1w_2)^{-1} \C[w_1^{-1}, w_2^{-1}] \{\partial_{w_a}\} \otimes \mc{O}\ar[r, "\del^W_\Omega" description] & (w_1w_2)^{-1} \C[w_1^{-1}, w_2^{-1}] \otimes \mc O \\
(w_1w_2)^{-1} \C[w_1^{-1}, w_2^{-1}] \otimes \T\ar[ur, "\del^X_\Omega" description] \\
(w_1w_2)^{-1} \C[w_1^{-1}, w_2^{-1}] \otimes \mc{O}\ar[r, "\del_X" description]\ar[dr, "\del_W" description] & (w_1w_2)^{-1} \C[w_1^{-1}, w_2^{-1}]  \otimes \Omega^1 \\ & (w_1w_2)^{-1} \C[w_1^{-1}, w_2^{-1}] \{\d w_a\} \otimes \mc O
\end{tikzcd}
\end{equation}

We denote by $\Obs_{AdS_7}$ the factorization algebra on $X = \C^3$ which assigns to an open set $U \subset \C^3$ the cochain complex
\beqn
 \Obs_{AdS_7}^N (U) = \cO\left(\Omega^{0,\bu}(U,\mc L^N_{AdS_7}) \right ) .
\eeqn

%Here, $\cA^\bu(U, \cL^N_{AdS_4}$ 

\begin{defn}\label{defn:ads7states}
The \emph{space of supergravity states} $\cS^N_{AdS_7}$ on twisted $AdS_7\times S^4$ is given by the cochain complex given by the costalk at zero of the factorization algebra $\cF_{AdS_7}$:
\beqn
\cS_{AdS_7}^N \define \cF_{AdS_7} (0) .
\eeqn
\end{defn}

\parsec{}
An argument exactly analogous to the proof of lemma \ref{lem:ads4states} gives us the following

\begin{lem}
The space of supergravity states $\cS_{AdS_7}^N$ on twisted $AdS_7\times S^4$ is the symmetric algebra $\Sym (\mc H_{AdS_7} )$ where $\mc H_{AdS_7}$ is given by the cochain complex

 \begin{equation} 
 \begin{tikzcd}[row sep = 1 ex]
    + & - \\ \hline
\C[w_1,w_2] \{\partial_{w_a}\} \otimes \C[\del_{z_1}, \del_{z_2}, \del_{z_3}]\delta_{z_i=0}\ar[r, "\del^W_\Omega" description] & \C[w_1,w_2]  \otimes \C[\del_{z_1}, \del_{z_2}, \del_{z_3}]\delta_{z_i=0} \\
\C[w_1,w_2]  \{\del_{z_i}\}  \otimes \C[\del_{z_1}, \del_{z_2}, \del_{z_3}]\delta_{z_i=0}\ar[ur, "\del^X_\Omega" description] \\
\C[w_1,w_2] \otimes \C[\del_{z_1}, \del_{z_2}, \del_{z_3}]\delta_{z_i=0}\ar[r, "\del_X" description]\ar[dr, "\del_W" description] & \C[w_1,w_2] \{\d z_a\} \otimes \C[\del_{z_1}, \del_{z_2}, \del_{z_3}]\delta_{z_i=0} \\ & \C[w_1, w_2] \{\d w_a\} \otimes \C[\del_{z_1}, \del_{z_2}, \del_{z_3}]\delta_{z_i=0}
\end{tikzcd}
\end{equation}
\end{lem}

\parsec{}
As before, we may compute the local character of the factorization algebra defined in \ref{defn:ads7states} as a character of $\Sym (\mc H_{AdS_7})$. 

We will use an action of $\mf{sl}(3)\oplus \mf {sl}(2)\oplus \mf{gl}(1)$ on $\mc H_{AdS_7}$ which we may explicitly realize as follows:
\begin{itemize}\label{eqn:im5}
\item
The subalgebra $\mf{sl}(3)$ acts as vector fields rotating the plane $\C^3_z$
\begin{equation}
\sum_{ij} A_{ij} z_i \frac{\del}{\del z_j} \quad (A_{ij}) \in \mf{sl}(3) .
\end{equation}
\item 
The subalgebra $\mf{sl}(2)$ acts by the triple of vector fields
\begin{equation}
 w_1 \frac{\del}{\del w_2}, \quad w_2 \frac{\del}{\del w_1}, \quad \frac{1}{2}\left (w_1\frac{\del}{\del w_1}-w_2\frac{\del}{\del w_2}\right).
\end{equation}
\item The subalgebra $\mf{gl}(1)$ acts as the vector field
\begin{equation}\label{eqn:Delta}
        \Delta = \sum_{i=1}^3 z_i\frac{\del}{\del z_i} - \frac 32\sum_{a=1}^2 w_a\frac{\del}{\del w_a}.
\end{equation}
\end{itemize}

%\begin{rmk}
%In the classification of simple super Lie algebras, Kac makes use of a weight grading $\oplus_{j \geq -2} \fg_j$ of the exceptional Lie algebra $E(3|6)$ for which the finite-dimensional subalgebra above is the weight zero piece
%\cite{KacClass}.
%We will make use of this grading in \S \ref{s:kr}.
%\end{rmk}

The character will be a function on a Cartan in $\mf {sl}(3)\oplus \mf {sl}(2)\oplus \mf {gl}(1)$; we choose one whose generators are given as follows. 
\begin{itemize}
  \item $t_{1}, t_{2}$ denote generators for the Cartan of $\mf{sl}(3)$ which is spanned by the vector fields
  \begin{equation}
  h_1 = z_1 \frac{\del}{\del {z_1}} - z_2 \frac{\del}{\del{z_2}} , \quad h_2 = z_2 \frac{\del}{\del{z_2}} - z_3 \frac{\del}{\del{z_3}}.
  \end{equation}
  \item $r$ denotes a generator for the Cartan of a $\mf{sl}(2)$ which is generated by the element 
  \begin{equation}
  \label{eqn:hCartan}
  h = \frac12 \left(w_1 \frac{\del}{\del w_1} - w_2 \frac{\del}{\del w_2}\right) .
  \end{equation}
\item $q$ denotes a generator for the Cartan of the~$\mf{gl}(1)$ which is generated by the element $\Delta$ from equation~$\eqref{eqn:Delta}$. 
\end{itemize}

The weights of twisted supergravity states with respect to the generators of the Cartan subalgebra above are completely determined by the weights of the holomorphic coordinates $w_1, w_2, z_1, z_2, z_3$, which we summarize in table \ref{tbl:sugraM5}.

\begin{table}
\begin{center}
\begin{tabular}{c c c c c c}
  & $z_{1}$ & $z_{2}$ & $z_{3}$ & $w_{1}$ & $w_{2}$ \\
  \hline
  $t_{1}$ & $1$ & 0 & $-1$ & 0 & 0 \\
  $t_{2}$ & 0 & 1 & $-1$ & 0 & 0 \\
  $r$ & 0 & 0 & 0 & 1 & $-1$ \\
  $q$ & $-1$ & $-1$ & $-1$ & $\frac{3}{2}$ & $\frac{3}{2}$
\end{tabular}
\caption{Fugacities for the fields of the holomorphic twist of eleven-dimensional supergravity for the geometry $\R \times \C^5 \setminus \C^3$.}
\label{tbl:sugraM5}
\end{center}
\end{table}

We enumerate single particle supergravity states via computing the super trace of the operator $q^\Delta t_1^{h_1} t_2^{h_2} r^h$ acting on $\mc{H}_{AdS_7}$:
\begin{equation}
f_{AdS_7}(t_1,t_2,r,q) = \tr_{\mc{H}_{AdS_7}} (-1)^F q^\Delta t_1^{h_1} t_2^{h_2} r^h  .
\end{equation}

\begin{prop}
\label{prop:sugraindex1}
The single particle index of the space of twisted supergravity states~$\mc{H}_{AdS_7}$ is given by the following expression
\begin{equation}
\label{eqn:sugra_index}
f_{AdS_7} (t_1,t_2, r, q) = \frac{q^4(t_1^{-1}+t_1t_2^{-1}+t_2)-q^2(t_1+t_1^{-1}t_2+t_2^{-1})+(q^{3/2}-q^{9/2})(r+r^{-1})}{(1-t_{1}^{-1}q)(1-t_{2}q)(1-t_{1}t_{2}^{-1}q)(1-rq^{3/2})(1-r^{-1}q^{3/2})}.
\end{equation}
The full (multiparticle) index is defined to be the plethystic exponential 
\begin{equation}
{\rm PExp}\left[f_{AdS_7}(t_1,t_2,r,q)\right] .
\end{equation}
\end{prop}

\begin{proof}
As before, the two summands not involving holomorphic vector fields or forms appear with opposite parity, so their contributions to the character will cancel. For the remaining summands, it is again straightforward to compute the character of each tensor factor.
\begin{itemize}
\item The factor $\C[\del_{z_1}, \del_{z_2}, \del_{z_3}]\delta_{z_i=0}$ contributes a factor of \[ \frac{q^3}{(1-t_{1}^{-1}q)(1-t_{2}q)(1-t_{1}t_{2}^{-1}q)}.\]
\item The factor of $\C[w_1, w_2]$ contributes a factor of \[ \frac{1}{(1-rq^{3/2})(1-r^{-1}q^{3/2})}.\]
\item The tensor factors involving vector fields and forms contribute a factor of \[ q(t_1^{-1}+t_1t_2^{-1}+t_2)-q^{-1}(t_1+t_1^{-1}t_2+t_2^{-1})+(q^{-3/2}-q^{3/2})(r+r^{-1})\] 
\end{itemize}
\end{proof}

\parsec{}
To simplify the form of this index we can introduce a different parametrization of the Cartan of $\mf{sl}(3) \oplus \mf{sl}(2) \oplus \mf{gl}(1)$.
First, we can parameterize the Cartan of $\mf{sl}(3)$ by the vector fields
  \begin{equation}\label{eqn:ys}
  -(\log y_1) z_1 \frac{\del}{\del {z_1}} - (\log y_2) z_2 \frac{\del}{\del{z_2}} - (\log y_3) z_3 \frac{\del}{\del{z_3}} .
  \end{equation}
where $y_1,y_2,y_3$ are parameters which satisfy the single constraint
\begin{equation}
y_1 y_2 y_3 = 1 .
\end{equation}
In terms of the variables $t_1,t_2$ used above we have
\begin{equation}
y_1 = t_1^{-1},\quad y_2 = t_1 t_2^{-1}, \quad y_3 = t_2 .
\end{equation}
Second, we can parametrize the Cartan of the remaining subalgebra $\mf{sl}(2) \oplus \mf{gl}(1)$ by the two vector fields
\begin{equation}
\tilde{h} = h + \frac12 \Delta \quad \text{and} \quad \Delta
\end{equation}
where $\Delta$ is as in equation \eqref{eqn:Delta} and $h$ is as in \eqref{eqn:hCartan}.
We denote by $y$ the generator of the Cartan corresponding to the vector field $\tilde{h}$ and by $q$ (as above) the generator corresponding to~$\Delta$.
In terms of the variable~$r$ used above we have 
\begin{equation}
y = q^{1/2} r .
\end{equation}

Using the parametrization of the Cartan given by the variables $y_i, y,\Delta$ we obtain the equivalent expression for the index \eqref{eqn:sugra_index} as 
\begin{equation}
\label{eqn:Kim_sugra}
f_{AdS_7} (y_i, y, q) = \frac{q^4(y_1+y_2+y_3)-q^2(y_1^{-1} + y_2^{-1} + y_3^{-1})+(1-q^3)(yq + y^{-1} q^2)}{(1-y_1 q)(1-y_2 q)(1-y_3 q)(1-yq)(1-y^{-1} q^2)},
\end{equation}
We note that this matches exactly with the index computed in \cite[equation (3.23)]{Kim:2013nva} with the change of variables.
%\begin{equation}
%t_1 = y_1^{-1} , \quad t_2 = y_3, \quad r = q^{-1/2} y 
%\end{equation} 

Our formula \eqref{eqn:sugra_index} also matches with \cite[equation (3.24)]{Bhattacharya:2008zy} where we use the change of variables
\begin{equation}
q = x^4, \quad t_1 = y_2, \quad t_2 = y_1, \quad r^2 = z .
\end{equation}
(Notice the variables $y_1,y_2$ used in \cite{Bhattacharya:2008zy} differ from the variables we introduced in \eqref{eqn:ys}.)
%We record a few specializations of this index which we will remark on further in~\S\ref{s:??}.

\parsec \label{eqn:winfty}
We consider the specialization of this index 
\begin{equation}
q=r^2, t_2=1 
\end{equation}
which is known as the Schur limit.
Applying this limit to \eqref{eqn:sugra_index} yields the plethystic exponential of the following single particle index
\[
f_{AdS_7}(q, t_1, t_2=1, r = q^{1/2}) = \frac{q}{(1-q)^2} 
\]
This plethystic exponential yields the MacMahon function, which is the character of the vacuum module of the $W_{1+\infty}$-algebra. We will revisit this observation in section \ref{sec:holspec}.

\subsection{Transverse boundary conditions}\label{sec:transversebc}

In the previous subsections, we discussed boundary conditions in the phase spaces of the sphere compactifications $p_* \mc E^N_{AdS_4}$ and $p_* \mc E^N_{AdS_7}$ viewed as free theories, that exist for generic values of $N$. However, there are distinguished boundary conditions that exist for $N = 0$ that we will use in the sequel. Moreover, these distinguished boundary conditions are in fact boundary conditions for the interacting theory. 

Indeed, recall that in equations \eqref{eqn:cotm2}, \eqref{eqn:cotm5} we wrote the phase spaces as twisted cotangent bundles, where the nontrivial Poisson tensor was induced by the terms in the differential coming from bracketing with the flux. When we specialize $N=0$, this Poisson tensor vanishes, and there is an additional Lagrangian given by the zero section.
Since the back reactions do not appear, these boundary conditions make sense for arbitrary THF three-manifold $M$ in \eqref{eqn:geomads4} and arbitrary threefold $X$ in \eqref{eqn:geomads7}.
We will specialize to $M = \R \times \C$ and $X =\C^3$ again momentarily.

Explicitly these Lagrangians in the phase space are described as follows.
\begin{itemize}
\item Let $\mc L_{AdS_4}^{r=0}$ denote the following complex of vector bundles on $M$
 \begin{equation}
 \begin{tikzcd}[row sep = 1 ex]
   \ul{-} & \ul{+} \\
\C[w_1,\ldots, w_4] \{\partial_{w_a}\} \otimes \mc{O}\ar[r, "\del^W_\Omega" description] &\C[w_1,\ldots, w_4]  \otimes \mc O \\
\C[w_1,\ldots, w_4]   \otimes \T\ar[ur, "\del^M_\Omega" description] \\
\C[w_1,\ldots, w_4] \otimes \mc{O}\ar[r, "\del_M" description]\ar[dr, "\del_W" description] & \C[w_1,\ldots, w_4]\otimes \Omega^1 \\ & \C[w_1,\ldots, w_4] \{\d w_a\}  \otimes \mc O
\end{tikzcd}
\end{equation}
The desired Lagrangian in $(p_*\mc E^{N=0}_{AdS_4})|_{\{\infty\} \times M}$ is given by the complex of bundles $\cA^\bu_{M}(\mc L_{AdS_4}^{r=0})$ on $M$.

\item Let $\mc L_{AdS_7}^{r=0}$ denote the following complex of vector bundles on $X$
 \begin{equation}
 \begin{tikzcd}[row sep = 1 ex]
   \ul{-} & \ul{+} \\
\C[w_1, w_2] \{\partial_{w_a}\} \otimes \mc{O}\ar[r, "\del^W_\Omega" description] &\C[w_1,w_2]  \otimes \mc O \\
\C[w_1, w_2]   \otimes \T\ar[ur, "\del^X_\Omega" description] \\
\C[w_1, w_2] \otimes \mc{O}\ar[r, "\del_X" description]\ar[dr, "\del_W" description] & \C[w_1,w_2]\otimes \Omega^1 \\ & \C[w_1, w_2] \{\d w_a\}  \otimes \mc O
\end{tikzcd}
\end{equation}
The desired Lagrangian in $(p_*\mc E^{N=0}_{AdS_7})|_{\{\infty\} \times X}$ is given by $\Omega^{0,\bullet}_{X} (\mc L_{AdS_7}^{r=0})$.
\end{itemize}

These Lagrangians exist in the fully interacting eleven-dimensional model, which we can encode perturbatively in terms of an $L_\infty$ structure.
In particular, the parity shift of these Lagrangians are equipped with $L_\infty$ structures compatible with the interacting bulk theory in the sense that the morphisms
\begin{align*}
\cA_M^\bu (\mc L^{r=0}_{AdS_4} ) & \to (p_* \mc E^{N=0} _{AdS_4})|_{\{\infty\} \times M}\\
\Omega^{0,\bullet}_{X} (\mc L^{r=0}_{AdS_7} ) & \to (p_* \mc E^{N=0} _{AdS_7})|_{\{\infty\} \times X}
\end{align*}
preserve the $L_\infty$ brackets even taking into account the potential for additional higher brackets on the target coming from homotopy transfer. 
%Indeed, such brackets must necessarily involve classes in $H^2_{\mr{THF}} \left ( (\R\times \C^2)\setminus 0 \right )$

\begin{rmk}
These boundary conditions have a very natural physical interpretation. Recall that we constructed our avatars of the $AdS_4\times S^7$ and $AdS_7\times S^4$ backgrounds by codifying their appearance as backreactions of M2 and M5 branes respectively. In the absence of the fluxes sourced by these branes, we may ask that the supergravity fields extend over the former locations of these branes. As such, we can think of the boundary conditions defined above as finite-type models for the restriction of the fields of the eleven-dimensional theory to the location of branes. 
\end{rmk}

\parsec[]

We restrict to the case $M = \R \times \C$ and $X = \C^3$.
The following lemma illustrates that the state space of definitions \ref{defn:ads4states}, \ref{defn:ads7states} can also be computed using these alternate boundary conditions.
These Lagrangians afford an alternative description of twisted supergravity states, which will be used to investigate their representation theoretic properties.

Given any local $L_\infty$ algebra $\cL$ we can consider the factorization algebra $\clie_\bu(\cL_c)$ which assigns to an open set $U$ the cochain complex computing the Lie algebra homology of the $L_\infty$ algebra of compactly supported sections $\clie_\bu (\cL_c(U))$.

\begin{prop}\label{prop:altstates}
There are isomorphisms of cochain complexes
\begin{align*}
 \left ( \clie_\bu(\cA^\bu_c (\mc L_{AdS_4}^{r=0} )) \right )(0) & \cong \Sym( \mc {H}_{AdS_4} ) \\
\left ( \clie_\bu ( \Omega^{0,\bullet}_{c}(\mc L_{AdS_7}^{r=0})) \right )(0) & \cong \Sym( \mc {H}_{AdS_7} )
\end{align*}
\end{prop}
\begin{proof}
For a local $L_\infty$ algebra $\mc L$, its factorization envelope $\mc U(L)$ is defined to be $\clie_\bullet (\mc L_c)$ the Lie algebra chains on the cosheaf of compactly supported sections. Therefore, it suffices to show that in each case, the costalk of the cosheaf of comapctly supported sections is quasi-isomorphic to $\mc H_{AdS_4}$ and $\mc H_{AdS_7}$ respectively. This is a consequence of the following observations.

Note that by ellipticity, there are quasi-isomorphisms 
\[\overline \Omega^{0,\bullet}_{\C,c} (D) \to \Omega^{0,\bullet}_{\C,c} (D), \ \ \ \ \overline \Omega^{0,\bullet}_{\C^3,c} \to \Omega^{0,\bullet}_{\C^3,c} (D^3).\]
coming from the inclusion of compactly supported distributional sections into compactly supported smooth sections. Now contracting the Dolbeault resolution, there are quasi-isomorphisms
\[\C[\del_z]\delta_{z=0} \to \overline \Omega^{0,\bullet}_{\C,c} (D), \ \ \ \ \C[\del_{z_1}, \del_{z_2}, \del_{z_3}]\delta_{z=0} \to \overline \Omega^{0,\bullet}_{\C^3,c} (D^3).\]
These results apply equally as well for sections of holomorphic bundles. Now, computing the limit in the definition of the costalk over a collection of open sets containing the origin gives the result.
\end{proof}

\end{document}

%\parsec
%
%The specialization $t_1=t_2=r=1$ yields the single particle index
%\[
%f_{sugra}^{6d} (q, t_1=t_2=r=1) = \frac{3 q^4 - 3 q^2 + 2 q^{3/2} - 2 q^{9/2}}{(1-q)^3 (1-q^{3/2})^2} .
%\]
