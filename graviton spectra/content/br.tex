\documentclass[../main.tex]{subfiles}

\begin{document} 
\section{Twisted AdS space}
\label{sec:ads}

In eleven-dimensional supergravity the $AdS_7 \times S^4$ and $AdS_4\times S^7$ backgrounds are obtained by backreacting a number of M5 branes and M2 branes in flat space \cite{Maldacena:1997re,WittenAdS}. In this section, we will give an account of this procedure at the level of our twisted theory in eleven-dimensions. 
The main outcome is a geometric definition of twisted analogs of $AdS$ spaces. 
Before describing the specific examples of interest, we begin with some generalities.

Suppose we have a theory of gravity on the total space of a vector bundle. In this thesis, we are interested in holomorphic-topological field theories, and in this context, the bundle projection is a map of THF manifolds, and the gravitational theory is a local moduli problem that describes, in part, deformations of the THF structure on the total space. Operationally, producing the theory in the backreacted geometry is the output of the following two-step procedure. 

\begin{itemize}
  \item Restrict the theory on the complement of the zero section. 
  If the theory is defined on flat space $\R^d$ to begin with, and the brane lives a long a coordinate plane $\R^{n} \subset \R^d$, then this amounts to restricting the theory on $\R^d \setminus \R^n$.
  \item Deform the theory on the complement of the zero section by a certain Maurer--Cartan element, thought of as the flux sourced by branes wrapping the zero section. More rigorously, the zero section determines a certain curved Maurer-Cartan equation, and the desired Maurer-Cartan element is a solution to this equation. 
\end{itemize}

This procedure is implemented at the level of the $\Omega$-deformed nonminimal twist on flat space in the appendix of \cite{CostelloM5}, and in \cite{raghavendran2022holographic} the procedure is carried about for M5 branes in our eleven-dimensional model in some global generality. 
As another example, in \cite{CGhol} the authors show that carrying out this procedure for branes in the topological $B$-model on $\C^3$ results in the deformed conifold.
For the purposes of this thesis, we will mostly content ourselves with examples on flat space.

\subsection{Twisted supergravity}

The eleven-dimensional supersymmetry algebra admits two types of twists characterized by the number of invariant directions that the supercharge determines:
\begin{itemize}
\item 
The $SU(5)$ twist. 
This twist leaves six real directions invariant and is stabilized by the subgroup $SU(5)$ of the eleven-dimensional Lorentz group.\footnote{We always work in Euclidean signature.}
\item 
The $G_2$ twist.
This twist leaves nine real directions invariant and is preserved by the subgroup $SU(2) \times G_2$ of the Lorentz group. 
\end{itemize}

In \cite{RSW} we have proposed a mathematical model for the $SU(5)$-twist of eleven-dimensional supergravity.
Many tests of this proposal were performed, including a consistency check that the dimensional reduction of the model agrees with proposals for twists of type IIA supergravity given in terms of topological strings~\cite{CLsugra}.

The eleven-dimensional theory exists on any manifold which is locally of the form 
\beqn\label{eqn:local}
\R \times Z
\eeqn
where $Z$ is a Calabi--Yau fivefold.
The model is topological along $\R$ in the sense that translations in this direction act in a homotopically trivial way on the theory.
Likewise, locally on the Calabi--Yau fivefold, all anti-holomorphic translations act in a homotopically trivial way---in this sense, the theory is {\em holomorphic} along $Z$ \cite{BWhol}.
More generally, the eleven-dimensional theory can be constructed on any eleven-manifold equipped with a transversely holomorphic foliation which is equipped with an appropriate holomorphic volume form on the leaves of the foliation.

As examples, we point out two classes of THF eleven-manifolds which we will consider in this paper, depending on whether we consider stack of M2 or M5 branes respectively:
\begin{itemize}
\item Let $M$ be a three-manifold equipped with a THF of real codimension two. 
Locally such a manifold is of the form $\R \times \C$.
Let $K_M$ be the canonical bundle of $M$ and suppose we have fixed a fourth-root $K^{1/4}$.\footnote{If $\cF$ is a THF of codimension two then $\T_M / \T_\cF \otimes_\R \C = Q \oplus \bar{Q}$ for the bundle $Q$ which is locally spanned by $\del_z$.
We let $K = Q^\vee$.}
Then we consider the THF eleven-manifold 
\beqn\label{eqn:thfm2}
\text{Tot} \left(K_M^{1/4} \otimes \C^4 \to M\right) .
\eeqn
The M2 branes will wrap the zero section of this total bundle.
Notice that the power of $1/4$ is to guarantee the Calabi--Yau condition.
\item Let $X$ be a complex threefold equipped with $K_X^{1/2}$.
We can consider the THF eleven-manifold
\beqn\label{eqn:thfm5}
\R \times \text{Tot} \left(K_X^{1/2} \otimes \C^2 \to X \right) .
\eeqn
The M5 branes will wrap the zero section and lie at a point in $\R$.
Notice that in this case the manifold is globally of the form $\R \times Z$ where $Z$ is a Calabi--Yau fivefold.
\end{itemize}


In \cite{RSW,RWindex} we describe how physical fields in supergravity are manifest in the twisted model.
We review a small part of this.
If $\til{Z}$ is an eleven-manifold equipped with a THF structure as above, the fields will live in a space of the form
\beqn
\cA^\bu(\til{Z}, \til{V})
\eeqn
where
\begin{itemize}
\item The graded bundle $\til{V}$ is holomorphic with respect to the THF structure.
\item $\cA^\bu (\til{Z}, \til{V})$ denotes the THF cohomology of $\til{Z}$ with coefficients in $\til{V}$. \brian{elaborate}
\end{itemize}
To simplify discussion in the remainder of this section we will only work in the local model~\eqref{eqn:local} where $\til{Z} = \R \times Z$ for $Z$ a Calabi--Yau fivefold.
Then, the space of fields is of the form
\beqn
\Omega^\bu(\R) \otimes \Omega^{0,\bu}(Z, V)
\eeqn
where $V$ is some holomorphic vector bundle on $Z$.\footnote{The tensor product $\otimes$ is the completed tensor product with respect to natural topologies present in the space of sections of a smooth vector bundle.}
Since the theory is of topological/holomorphic nature, there is a linear part of the BRST differential of the form $\d + \dbar$ where $\d$ is the de Rham differential on $\R$ and $\dbar$ corresponds to the holomorphic structure on the bundle $V$.

Deformations of the background metric which survive the twist can be identified with Beltrami differentials, that is, sections of 
\[
\Bar{\T}_Z^* \otimes \T_Z \cong \Bar{\T}^*_Z \otimes \Bar{\T}_Z^* . 
\]
More generally, the fields of this model include sections of the holomorphic tangent bundle with coefficients in Dolbeault forms on $Z$ of arbitrary Dolbeault type and de Rham forms on $\R$, which we can write in superfield notation as 
\[
\mu = \mu^{\Bar{I}}_j (t;z,\zbar) \d \zbar_{\Bar{I}} \otimes \del_{z_j} + \mu^{\Bar{I}}_{t,j} (t;z,\zbar) \d t \d \zbar_{\Bar{I}} \otimes \del_{z_j}
\]
where the sum over the multi-index $\Bar{I}$ ranges over subsets of $\{1,\ldots,5\}$.
We view this as a section
\beqn
\mu \in \Omega^\bu(\R) \otimes \Omega^{0,\bu}(Z, \Pi \T_Z) .
\eeqn
In this notation, the component $\mu^{\Bar{I}}_j$ is {\em odd} if $|\Bar{I}|=0,2,4$ is even and {\em even} if $|I| = 1,3,5$ is odd.
The parity shift of the holomorphic tangent bundle is to ensure that deformations of the background metric which survive the twist are viewed as even fields.
In particular we have odd fields $\mu_j(t;z,\zbar) \del_{z_j}$ which play the role of ghosts for infinitesimal changes of coordinates.

Part of the linear equations of motion in our model imply that the field $\mu$ satisfies a condition that it be {\em divergence-free} for the Calabi--Yau structure on the fivefold~$Z$.
This condition is imposed, cohomologically, by adding another field which is simply a differential form
\beqn
\nu \in \Omega^\bu(\R) \otimes \Omega^{0,\bu}(Z) 
\eeqn
and a linear part of the differential $\delta \nu = \div \mu$ encoding the divergence-free condition.
Another piece of the linear equations of motion imply that it $\mu$ be constant along~$\R$. 

Ordinarily, for $\mu$ to describe a deformation of complex structure in the $Z$ direction we would require that it satisfy the Maurer--Cartan equation.
For twisted supergravity, however, we find a modification of this equation which involves the twisted analog of another familiar field---the higher form gauge field in supergravity.

In the BV formalism one will see all types of differential forms corresponding to a tower of ghosts for this higher gauge field together with antifields and antighosts.
In local coordinates, the only components of this tower of differential forms which survives the twist are
\begin{align*}
\beta & = \beta^{\Bar{I}} \d \zbar_{\Bar{I}} + \beta_t^{\Bar{J}} \d t \d \zbar_{\Bar{J}} \in \Omega^{\bu}(\R) \otimes \Omega^{0,\bu}(Z) \\
\gamma & = \gamma^{i \Bar{I}} \d z_i \d \zbar_{\Bar{I}} + \gamma_t^{j \Bar{J}} \d t \d z_j \d \zbar_{\Bar{J}} \in \Omega^{\bu} (\R) \otimes \Omega^{1,\bu}(Z).
\end{align*}
%When we want to be specific by the form type we use the notation $\beta^{k;q}(t;z,\zbar)$ for the $(k;0,q)$ component of the three-form and $\gamma^{k;q}(t;z,\zbar)$ for the $(k;1,q)$ component of the three-form.
Fields $\beta^{\Bar{I}}, \gamma_t^{i \Bar{I}}$ with $|\Bar{I}| = $odd (resp. even) are {\em even} (resp. {\em odd}) and fields $\beta_t^{\Bar{I}}, \gamma^{i \Bar{I}}$ with $|\Bar{I}| = $odd (resp. even) are {\em odd} (resp. {\em even}). 
The three-forms $\beta^{\Bar{i}\Bar{j}\Bar{k}} \d \zbar_{\Bar{i}} \d \zbar_{\Bar{j}} \d \zbar_{\Bar{k}}$, $\beta_t^{\Bar{i}\Bar{j}} \d t \ \zbar_{\Bar{i}} \d \zbar_{\Bar{j}}$, $\gamma^{i \Bar{j} \Bar{k}} \d z_i \d \zbar_{\Bar{j}} \d \zbar_{\Bar{k}}$, and $\gamma_t^{i \Bar{j}} \d t \d z_i \d \zbar_{\Bar{j}}$ comprise components of the supergravity three-form $C$ which survive the $SU(5)$ twist. 

The most important equation of motion of the eleven-dimensional theory involves both the fields $\mu$ and $\gamma$. 
When $\div \mu = 0$ it takes the form
\beqn\label{eqn:eom2}
\dbar \mu + \frac12 [\mu,\mu] = \del \gamma \del \gamma .
\eeqn
Because of the term on the right hand side this equation is not exactly the usual Beltrami equation for deformations of complex structures.
On the left hand side we are implicitly using an identification between the holomorphic tangent bundle $\T_Z$ and the bundle $\wedge^4 \T^*_Z$ granted by the holomorphic volume form on $Z$.

In \cite{RSW} we find the above equation of motion together with its gauge symmetries by formulating the theory within the BV formalism.
We briefly recount what this entails.
In the BV formalism, the space of BV fields $\cE$ is given as the sections of some graded vector bundle on spacetime. 
The grading is typically by the abelian group $\Z$, which is usually referred to as the cohomological grading.
In cohomological degree zero of $\cE$ sit the usual (sometimes called {\em physical}) fields.
The ghosts comprise the cohomological degree $-1$ part of $\cE$. 
In cohomological degree one sit the antifields, etc..
Another key piece of structure is a skew symmetric pairing $\omega_{BV}$ of cohomological degree $-1$ on $\cE$ which is concretely a pairing between the fields and antifields (and ghosts and antighosts).
The full BV action functional $S_{BV}$ is a functional on the space of fields which encodes not only the original action functional on fields, but through its dependence on ghosts and antifields it also encodes the gauge symmetries (and gauge symmetries for gauge symmetries, etc.) of the theory. 

For our eleven-dimensional model the BV action functional takes the following form
\begin{multline}
S_{BV} = \int_{\R \times Z} \bigg[\beta \wedge (\dbar + \d) \nu + \gamma \wedge (\dbar + \d) \mu +  \beta \wedge \div \mu  \bigg] \\
+ \frac{1}{2} \int_{\R \times Z}  \mu^2 \vee \del \gamma + \int_{\R \times Z} \gamma \del \gamma \del \gamma + \cdots .
\end{multline}
Here:
\begin{itemize}
\item the first line is the free part of the action.
The operator $\dbar$ is the $\dbar$-operator on $Z$, the operator $\d$ is the de Rham operator on $\R$, and the operator $\div$ is the holomorphic divergence operator on $Z$ corresponding to the holomorphic volume form.
\item We implicitly use the holomorphic volume form in the above integrals.
\item The term 
\beqn\label{eqn:J}
J(\gamma) \define \int_{\R \times Z} \gamma \del \gamma \del \gamma .
\eeqn
is the holomorphic avatar of the famous Chern--Simons term of the higher form gauge field present in supergravity.
\item The $\cdots$ indicate additional terms needed so that this functional satisfies the classical master equation (in this presentation there are, in fact, infinitely many terms).
They take a form which is reminiscent of the genus zero BCOV action for the fivefold $Z$ (we make a further comment on this below).
\end{itemize}

In what follows we will define $S_{BF}$ as
\beqn
S_{BV} = S_{BF} + J ,
\eeqn
with $J$ the Chern-Simons term \eqref{eqn:J}.
We will not need an explicit formula for $S_{BF}$ but we remark that reason for the terminology $S_{BF}$ is that this term in the action is homotopy equivalent to a standard BF type action, see \cite{RSW}.

In \cite{RSW} we have performed a number of consistency checks that this model indeed describes the $SU(5)$ twist of supergravity.
Famously, $M$-theory on a circle is expected to be equivalent to type IIA string theory where the length of the circle is proportional to the string coupling constant.
At the level of supergravity, the $S^1$-reduction of eleven-dimensional supergravity is ten-dimensional supergravity of type IIA.
Here we utilize a description of the twist of type IIA strings using a version of Kodaira--Spencer theory developed in \cite{CLbcov1,CLsugra}.

%From the point of view of the $SU(5)$ twist there are essentially two types of circles one can consider.
%The first is a `holomorphic circle'.
%This is present when we assume our Calabi--Yau fivefold is of the form $Z = \C^\times \times Y$ with $Y$ a Calabi--Yau fourfold.
%We assume that $\C^\times$ is equipped with the standard volume form $\d z$.
%The $S^1$ reduction along a circle in $\C^\times$ results in a twist of type IIA supergravity with global symmetry $SU(4)$ on the manifold
%\beqn
%\R \times \R \times Y .
%\eeqn
%More generally, this twist can be defined when $\R^2$ is replaced by a smooth oriented two-dimensional manifold $M^2$.
%In \cite{CLsugra}, Costello and Li have given a conjectural description of this particular twist of type IIA (and many other twists) using methods of topological string theory.
%In this case, the conjecture of Costello and Li is that the twist is equivalent to a mixed topological string theory which is given by the A-model along $M^2$ and the B-model along $Y$.
%In perturbation theory, this admits a description in terms of Kodaira--Spencer theory along the fourfold $Y$.
%Via an explicit dimensional reduction of our model we find an exact match with this expectation \cite{RSW}.
%The dimensional reduction of our eleven-dimensional model on
%\beqn
%\R \times \C \times Y
%\eeqn
%along $S^1 \subset \C$ is equivalent to the mixed topological A-model along $\R^2$ and B-model along $Y$ as considered in \cite{CLsugra}.
%
%The other type of reduction one can consider is that along a `topological' circle.
%This corresponds to placing our eleven-dimensional model on
%\beqn
%S^1 \times Z
%\eeqn
%and performing reduction along $S^1 \times Z \to Z$.
%What results is a theory with $SU(5)$ global symmetry.
%We conjecture that this is the $SU(5)$ twist of type IIA supergravity.
%There is no expected description of the $SU(5)$ twist in terms of topological strings like in the $SU(4)$ or $SU(2)$ cases.

%
%There are two bosonic fields of particular importance.
%First is a section
%\beqn
%\mu^{0;0,1} \in \Omega^0(\R) \otimes \Omega^{0,1}(X, \T_X) .
%\eeqn
%Here $\T_X$ is the holomorphic tangent bundle on $X$.
%The linear part of the BRST differential requires this field be constant in the real direction.
%Thus, this field has the same form as a Beltrami differential on $X$.
%There are particular field configurations where the non-linear equations of motion for $\mu$ imply that it is a complex structure deformation of $X$, but the full equations of motion are more complicated.
%Another bosonic field of importance is a section
%\beqn
%\gamma^{0;0,1} \in \Omega^0(\R) \otimes \Omega^{1,1}(X) .
%\eeqn
%
%The field $\mu$ satisfies a condition that it be {\em divergence-free} for the Calabi--Yau structure on the fivefold~$X$.
%This means that it is required to satisfy the equation
%\beqn
%\div \mu = 0
%\eeqn
%where $\div(-)$ is the divergence with respect to the holomorphic volume form $\Omega$ on $X$.

%\subsection{The $G_2$ twist}
%
%We will give a description to of the $G_2$ twist of eleven-dimensional supergravity.
%Independent derivations of this twist have appeared in \cite{CostelloM5, EagerHahner}.
%The theory is defined on any manifold of the form
%\beqn
%M \times Y
%\eeqn
%where $M$ is a manifold with $G_2$ holonomy and $Y$ is a manifold with $SU(2)$ holonomy.
%Unfortunately, the perturbative description we are about to give depends in no way on the $G_2$ structure and can be defined when $M$ is any smooth seven-dimensional manifold.
%To see the full $G_2$ structure appearing one must take into account non-perturbative effects (much like in the target space description of the topological A-model).
%
%The fundamental field is simply a differential form
%\beqn
%\alpha \in \Omega^\bu(M) \otimes \Omega^{0,\bu}(Y) [1] .
%\eeqn
%The linear BRST differential is simply $\d + \dbar$ where $\d$ is the de Rham differential on $M$ and $\dbar$ is the Dolbeault operator for $Y$.
%
%Being a manifold with $SU(2)$ holonomy, the algebra of holomorphic fnuctions on $Z$ is equipped with a holomorphic Poisson bracket $\{-,-\}$.
%Since this bracket involves only holomorphic differntial operators, it can be extended to a Poisson bracket on the dg algebra $\Omega^{0,\bu}(Y)$.
%The theory utilizes this bracket in an essential way.
%Indeed, the full BV action is
%\beqn
%\frac12 \int_{M \times Y} \Omega \, \alpha \d \alpha + \frac16 \int_{M \times Y} \Omega \, \alpha \{\alpha, \alpha\} .
%\eeqn
%where $\Omega \in \Omega^{2,0}(Y)$ is the holomorphic volume form.
%This description appeared in \cite{RYsduality} and was independently derived from a further twist of the $SU(5)$ twist in \cite{RSW}. 
%
%Like the $SU(5)$ twist, this theory is only $\Z/2$ graded.
%Nevertheless, one convenient way to think about this theory is as a higher dimensional Chern--Simons theory for a particular (dg) Lie algebra.
%Given any odd-dimensional smooth manifold $M$ and a Lie algebra $\lie{g}$ equipped with an invariant non-degenerate symmetric inner product one can consider a $\Z/2$ graded version of Chern--Simons theory whose BV action is given by the usual formula $\int_M CS(A)$.
%It is only in the case that $\dim M = 3$ that this theory admits a $\Z$-grading.
%The theory we are considering here can be interpreted as $\Z/2$ graded Chern--Simons theory on the seven-manifold $M$ for the Lie algebra of holomorphic functions on $Y$ equipped with the holomorphic Poisson bracket.
%Here, the invariant pairing is given by integrating against the holomorphic volume form $\Omega$.

\subsection{The ${\mr AdS}_4 \times S^7$ background}

In this section we introduce the analog of the ${\mr AdS}_4 \times S^7$ background in our conjectural description of the minimal twist of eleven-dimensional supergravity. 

\parsec[]

We begin by viewing the eleven-dimensional manifold
\[
\operatorname{Tot}(K^{1/4}_M \otimes \C^4\to M)
\]
as in \eqref{eqn:thfm2}.
Even when we restrict ourselves to flat space $M = \R \times \C$ this way of presenting the eleven-manifold $\R \times \C^5$ is a convenient mechanism to record weights under natural scaling actions. 
We will use $w_a$ to denote holomorphic fiber coordinates on $K^{1/4}_M \otimes \C^4$.

We carry out the above procedure. 
Consider a stack of $N$ twisted M2 branes wrapping the zero section $M$.
Our ansatz for this coupling is heuristically of the form
\[
I_{M2}(\gamma) = N\int_{M} \gamma + \cdots
\] 
which is nonzero only on the component of $\gamma$ which is a top form along $M$.
We have only indicated the lowest order coupling, the $\cdots$ indicate higher-order couplings which will be higher order in the fields of the eleven-dimensional theory and explicitly involve the fields in the worldvolume theory. 

This coupling is justified by comparison with the physical theory and by dimensional reduction. 
Indeed, as discussed in the previous section, the component of $\gamma$ which participates in the above coupling is a component of the three-form $C$-field of eleven dimensional supergravity. 
Thus, the proposal mirrors electric couplings of M2 branes in the physical theory, which simply involves integrating the $C$-field over the worldvolume of the brane. 

Moreover, reducing on a circle transverse to the M2 brane yields the $SU(4)$ twist of type IIA supergravity with $N$ $D2$ branes wrapping $M$. 
As is shown in \cite{CLsugra}, an electric coupling of D2 branes to the $SU(4)$ twist of type IIA supergravity is given by 
\[
I_{D2}(\gamma) = N \int_{M} \gamma + \cdots
\] 
where $\gamma$ now denotes the 1-form field of the $SU(4)$ twist of type IIA supergravity. It is immediate that the pullback of $I_{M2}$ along the map in the proof of proposition \ref{prop:dimred} recovers $I_{D2}$. 

\parsec[sec:m2backreact]

The backreacted geometry will be given by a solution to the equations of motion upon deforming the eleven-dimensional action by the interaction $I_{M2}(\gamma)$.
For the twisted version of $AdS$ space we should start with the three-manifold $M$ being just $\R \times \C$, so that the resulting theory is defined originally just on flat space.

Varying the deformed action with respect to $\gamma$,
we obtain the equation of motion
\begin{equation}\label{eqn:ads4eom1}
\dbar \mu + \frac12 [\mu, \mu] + \partial\gamma\partial\gamma = N \Omega^{-1} \delta_{w=0} .
\end{equation}
Here $[-,-]$ is the Schouten bracket. 
Varying $\beta$, we obtain the equation of motion
\begin{equation}\label{eqn:adseom2}
\div \mu = 0 .
\end{equation}

\begin{lem}\label{lem:m2flux}
Let
\beqn\label{eqn:FM2}
 F_{M2} = \Omega^{-1} \frac{6}{(2\pi i)^4} \frac{\sum_{a=1}^4 \bar{w}_a \d \bar{w}_1 \cdots \widehat{\d \bar{w}_a} \cdots \d \bar{w}_4}{\|w\|^{8}} \d w_1 \d w_2 \d w_3 \d w_4 .
\eeqn
Then the background where $\mu = N F_{M2}$ and $\gamma = 0$
satisfies the above equations of motion in the presence of a stack of $N$ M2 branes:
\begin{align*}
\dbar (N F_{M2}) + \frac12 [N F_{M2}, N F_{M2}] & = N \Omega^{-1} \delta_{w=0} \\
\div (N F_{M2}) & = 0  .
\end{align*}
Here we set all components of the field $\gamma$ equal to zero (as well as the fields $\nu,\beta$). 
\end{lem}

\begin{proof}
Upon specializing $\gamma = 0$, the last term in the first equation above vanishes. The equation $\dbar F_{M2} = \Omega^{-1} \delta_{w=0}$ characterizes the Bochner--Martinelli kernel representing the residue class on $\C^4 \, \setminus \, 0$. 
It is clear that $\div F_{M2} = 0$ and 
\[
[F_{M2}, F_{M2}] = 0
\] 
by simple type reasons. 
\end{proof}

We summarize the output of our computation with a definition.

\begin{defn}\label{defn:ads4}
Let $\mc E^N_{AdS_4}$ denote the classical BV theory on 
\beqn\label{eqn:totalm2}
\operatorname{Tot} (K_\C^{1/4}\otimes \C^4 \to \R\times \C_z)\setminus 0(\R\times \C)
\eeqn
given by the sheaf of cochain complexes $\mc E |_{(\R\times \C)\times (\C^4\setminus \{0\} )}$, with BV pairing induced from $\mc E$, deformed by the interaction \[S_{BF,\infty}(\mu + NF_{M2}, \nu, \beta, \gamma) + J(\gamma).\]
\end{defn}


\begin{rmk}
By definition, the theory $\cE_{AdS_4}^N$ is a deformation of our eleven-dimensional model on \eqref{eqn:totalm2} by the $N$-dependent action functional $S_{BF, \infty}(\mu = N F_{M2})$.
Upon expanding this action around $NF_{M2}$, the cubic term in $S_{BF,\infty}$ will contribute a differential which acts on $\gamma$ and $\mu$ by bracketing with $NF_{M2}$. We accordingly denote this differential $[NF_{M2}, - ] $, and we see that that linearized BRST complex underlying $\mc E_{AdS_4}$ is \[\left ( \mc E |_{(\R\times \C)\times (\C^4\setminus \{0\} ) }, \delta^{(1)} + [NF_{M2},-] \right )\] where $\delta^{(1)}$ denotes the original linearized BRST differential of the eleven-dimensional theory defined on flat space.
\end{rmk}

\begin{conj}\label{conj:ads4}
The minimal twist of eleven-dimensional supergravity on the $AdS_4\times S^7$ background with $N$ units of M2 brane flux supported on $S^7$ is perturbatively equivalent to $\mc E^N_{AdS_4}$.
\end{conj}

To verify this conjecture, we should directly twist eleven-dimensional supergravity on the $AdS_4\times S^7$ spacetime. Doing so seems difficult - while it is likely not hard to identify the covariantly constant nilpotent spinors which define the twist, it seems more difficult to establish a perturbative equivalence with our description above. A modification of the pure spinor superfield formalism to symmetric spaces such as cosets for the superconformal group might make such checks more feasible. In lieu of such, we will instead pursue other consistency checks in the following two sections{}. 

\subsection{The ${\mr AdS}_7 \times S^4$ background}

We similarly introduce an analog of the ${\mr AdS}_7 \times S^4$ background in our description of the minimal twist of eleven-dimensional supergravity.
We begin with our twisted theory defined on
\[
\R \times \operatorname{Tot}(K_{X}^{1/2}\otimes \C^2 \to X)
\]
with $X$ a complex threefold as in \eqref{eqn:thfm5}.
As above, we will momentarily be only concerned with flat space which means taking $X = \C^3$.
We once again will use $w_a$ to denote holomorphic fiber coordinates on $K^{1/2}_{X}\otimes \C^2$, and we use $t$ to denote a fiber coordinate on $\R$.

\parsec[sec:m5coupling]

To repeat the procedure in the previous subsection, we begin by defining our eleven-dimensional theory couples to M5 branes (to first order).
Consider a stack of $N$ M5 branes wrapping the zero section $X$. 
We consider the heuristic nonlocal coupling 
\[
I_{M5} = N\int_{X} \div^{-1}\mu \vee \Omega +\cdots 
\]
Note that this expression is only nonzero on the component of $\mu$ in $\PV^{1,3}(X)$. 
We argue that this coupling is consistent with expectations from the physical theory and from dimensional reduction. 

The twisted field $\mu^{1,3}$ is a component of the Hodge star of the metric flux in the physical theory.
In the physical theory, M5 branes magnetically couple to the $C$-field; the coupling involves choosing a primitive for the Hodge star of the $G$-flux and integrating it over the M5 worldvolume. Our twist contains no fields corresponding to components of such a primitive; hence such a magnetic coupling is reflected in the appearance of $\div^{-1}$. 

We may once again justify this coupling by dimensional reduction to IIA supergravity. 
When $X = \C^2 \times \C$ we reduce on the circle along the last copy of $\C$ that the M5 branes wrap to get the $SU(4)$ invariant twist of type IIA supergravity on $\C^4 \times \R^2$ with $N$ $D4$ branes wrapping $\C^2 \times \R$.

In \cite{CLsugra}, it is shown that the magnetic coupling of $D4$ branes to the $SU(4)$ twist of IIA is of the form
\[
N \int _{\C^2 \times \R} \div^{-1} \mu \vee \Omega_{\C^4} + \cdots .
\]
Again, we have only explicitly indicated the first-order piece of the coupling. 

\parsec[s:m5backreact]

The backreacted geometry will be given by a solution to the equations of motion upon deforming the eleven-dimensional action by the interaction $I_{M5}(\mu)$.

Varying the potential $\div^{-1} \mu$, we obtain the following equation of motion involving the field $\gamma$:
\begin{equation}\label{eqn:m5eom1}
\dbar \del \gamma + \div \left(\frac{1}{1-\nu} \mu\right) \wedge \del \gamma = N \delta_{w_1=w_2=t=0} .
\end{equation}
Notice that there is an extra derivative compared to the equation of motion arising from varying the field $\mu$. 
This equation only depends on $\gamma$ through its field strength $\del \gamma$.

Varying $\gamma$ we obtain the equation of motion 
\begin{equation}\label{eqn:m5eom2}
(\dbar + \d_\R) \mu + \del \gamma \del \gamma = 0 .
\end{equation} 
Again, this only depends on $\gamma$ through its field strength $\del \gamma$.
It is at this point that we restrict ourselves to the case $X = \C^3$.

\begin{lem}
\label{lem:ads7flux}
Let
\beqn\label{eqn:FM2}
F_{M5} = \frac{1}{(2\pi i)^3} \frac{\bar{w}_1 \d \bar{w}_2 \wedge \d t - \bar{w}_2 \d \bar{w}_1 \wedge \d t + t \d \bar{w}_1 \wedge \d \bar{w}_2}{(\|w\|^2 + t^2)^{5/2}} \wedge \d w_1 \wedge \d w_2
\eeqn
Then, $\del\gamma = N F_{M5}$, $\mu = 0$, and $\nu = 0$ satisfies the equations of motion in the presence of a stack of $N$ M5 branes sourced by the term $N \delta_{w_1=w_2=t=0}$:
\begin{align*}
\dbar (NF_{M5}) + \d_{\R} (NF_{M5}) & = N \delta_{w_1=w_2=t=0}  \\ 
(NF_{M5}) \wedge (NF_{M5}) & = 0 .
\end{align*}
Here, we set all components of the field $\mu$ equal to zero (as well as the fields $\nu,\beta$). 
\end{lem}

\begin{proof}
The first equation,
\[
\dbar F + \d_{\R} F = N \delta_{w_1=w_2=t=0},
\]
characterizes the kernel representing $N$ times the residue class for a four-sphere in 
\[
(\C^2 \times \R) \setminus 0 \simeq S^4 \times \R .
\] 
That is
\[
\oint_{S^4} N F = N 
\]
for any four-sphere centered at $0 \in \C^2 \times \R$.

The second equation $F \wedge F = 0$ follows by simple type reasons. 
\end{proof}

Once again, we summarize the lemma above with a definition.

\begin{defn}\label{defn:ads7}
Let $\mc E^N_{AdS_7}$ denote the classical BV theory on 
\beqn
\operatorname{Tot}(\R \oplus K_{\C^3}^{1/2}\otimes \C^2 \to \C^3_z)\setminus 0(\C^3_z)
\eeqn
given by the sheaf of cochain complexes $\mc E |_{\C^3\times (\R\times \C^2\setminus \{0\} )}$, with BV pairing induced from that on $\mc E$, deformed by the interaction \[S_{BF,\infty}(\mu, \nu, \beta, \gamma +N\del^{-1} F_{M5}) + J(\gamma + N \del^{-1} F_{M5}).\]
\end{defn}

\begin{rmk}
Note that both terms in the action only depend on $\gamma$ through its holomorphic derivatives so the above expression for the action is indeed well-defined. 

As before, upon expanding the interactions around $NF_{M5}$, the cubic terms in both $S_{BF,\infty}$ and $J$ will contribute differentials. From $S_{BF,\infty}$, we get a differential which takes a $\mu$ type field to the Schouten bracket $N[F_{M5}, \mu]$ and from $J$, we get a differential which acts as $\gamma\mapsto N F_{M5} \wedge \del \gamma$. We accordingly denote this differential $[NF_{M5}, - ] $, and linear BRST complex is
\[
\left (\mc E |_{\C^3\times (\R\times \C^2\setminus \{0\} )}, \delta^{(1)} + [NF_{M5},-] \right )
\] where $\delta^{(1)}$ denotes the original linearized BRST differential.
\end{rmk}

\begin{conj}\label{conj:ads7}
The minimal twist of eleven-dimensional supergravity on the $AdS_7\times S^4$ background with $N$ units of M5 brane flux supported on $S^4$ is perturbatively equivalent to $\mc E^N_{AdS_7}$. 
\end{conj}

%Part of our goal in the remainder of the paper is to justify conjectures \ref{conj:ads4} and \ref{conj:ads7}.
\end{document}
