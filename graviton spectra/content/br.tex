\section{Twisted Backreactions}
\label{sec:ads}

As remarked above, in eleven-dimensional supergravity, the $AdS_7 \times S^4$ and $AdS_4\times S^7$ backgrounds are obtained by backreacting a number of M5 branes and M2 branes in flat space \cite{Maldacena:1997re,WittenAdS}. In this section, we wish to give an account of this procedure at the level of our twisted theory in eleven-dimensions. Before describing the specific examples of interest, we begin with some generalities.

Suppose we have a theory of gravity on the total space of a vector bundle. In this thesis, we are interested in holomorphic-topological field theories, and in this context, the bundle projection is a map of THF manifolds, and the gravitational theory is a local moduli problem that describes, in part, deformations of the THF structure on the total space. Operationally, producing the theory in the backreacted geometry is the output of the following two-step procedure. 

\begin{itemize}
  \item Place the theory on the complement of the zero section. 
  \item Deform the theory on the complement of the zero section by a certain Maurer--Cartan element, thought of as the flux sourced by branes wrapping the zero section. More rigorously, the zero section determines a certain curved Maurer-Cartan equation, and the desired Maurer-Cartan element is a solution to this equation. 
\end{itemize}

This procedure is implemented at the level of the $\Omega$-deformed nonminimal twist on flat space in the appendix of \cite{CostelloM5}, and in \cite{raghavendran2022holographic} the procedure is carried about for M5 branes in our eleven-dimensional model in some global generality. For the purposes of this thesis however, we will content ourselves with examples on flat space. 

\subsection{The ${\mr AdS}_4 \times S^7$ background}

In this section we introduce the analog of the ${\mr AdS}_4 \times S^7$ background in our conjectural description of the minimal twist of eleven-dimensional supergravity. 

\parsec[]

We begin by viewing the eleven-dimensional manifold $\R\times \C^5$ as 
\[
\operatorname{Tot}(K^{1/4}_\C\otimes \C^4\to \R\times \C_z)
\]
where we have abusively used $K^{1/4}_\C\otimes \C^4$ to denote its pullback along the natural projection $\R\times \C\to \C$. Thinking of flat space in this way is simply a way to record weights under natural scaling actions. We will use $w_a$ to denote holomorphic fiber coordinates on $K^{1/4}_\C\otimes \C^4$.

We carry out the above procedure. Consider a stack of $N$ M2 branes wrapping the zero section $\R\times \C_z$. A natural interaction to consider is 
\[
I_{M2}(\gamma) = N\int_{\C_z} \gamma + \cdots
\] 
which is nonzero only on the component of $\gamma$ in $\Omega^1(\R)\otimes \Omega^{1,1}(\C^5)$. We have only indicated the lowest order coupling, the $\cdots$ indicate higher-order couplings which will be higher order in the fields of the eleven-dimensional theory and explicitly involve the fields in the worldvolume theory. 

This coupling is justified by comparison with the physical theory and by dimensional reduction. 
Indeed, as discussed in~\S\ref{s:components}, the component of $\gamma$ which participates in the above coupling is a component of the $C$-field of eleven dimensional supergravity. Thus, the proposal mirrors electric couplings of M2 branes in the physical theory, which simply involves integrating the $C$-field over the worldvolume of the brane. 

Moreover, reducing on a circle transverse to the M2 brane yields the $SU(4)$ twist of type IIA supergravity on $\R^2\times \C_z\times \C^3$ with $N$ $D2$ branes wrapping $\R\times \C_z$. As is shown in \cite{CLsugra}, an electric coupling of D2 branes to the $SU(4)$ twist of type IIA supergravity is given by 
\[
I_{D2}(\gamma) = N \int_{\R\times\C_z} \gamma + \cdots
\] 
where $\gamma$ now denotes the 1-form field of the $SU(4)$ twist of type IIA supergravity. It is immediate that the pullback of $I_{M2}$ along the map in the proof of proposition \ref{prop:dimred} recovers $I_{D2}$. 

\parsec[sec:m2backreact]

The backreacted geometry will be given by a solution to the equations of motion upon deforming the eleven-dimensional action by the interaction $I_{M2}(\gamma)$. 
Varying the deformed action with respect to $\gamma$,
we obtain the equation of motion
\begin{equation}\label{eqn:ads4eom1}
\dbar \mu + \frac12 [\mu, \mu] + \partial\gamma\partial\gamma = N \Omega^{-1} \delta_{w=0} .
\end{equation}
Here $[-,-]$ is the Schouten bracket. 
Varying $\beta$, we obtain the equation of motion
\begin{equation}\label{eqn:adseom2}
\div \mu = 0 .
\end{equation}

\begin{lem}\label{lem:m2flux}
Let
\[
 F_{M2} = \frac{6}{(2\pi i)^4} \frac{\sum_{a=1}^4 \bar{w}_a \d \bar{w}_1 \cdots \widehat{\d \bar{w}_a} \cdots \d \bar{w}_4}{\|w\|^{8}} \partial_z .
\]
Then the background where $\mu = N F_{M2}$ and $\gamma = 0$
satisfies the above equations of motion in the presence of a stack of $N$ M2 branes:
\begin{align*}
\dbar (N F_{M2}) + \frac12 [N F_{M2}, N F_{M2}] & = N \Omega^{-1} \delta_{w=0} \\
\div (N F_{M2}) & = 0  .
\end{align*}
Here we set all components of the field $\gamma$ equal to zero (as well as the fields $\nu,\beta$). 
\end{lem}

\begin{proof}
Upon specializing $\gamma = 0$, the last term in the first equation above vanishes. The equation $\dbar F_{M2} = \Omega^{-1} \delta_{w=0}$ characterizes the Bochner--Martinelli kernel representing the residue class on $\C^4 \, \setminus \, 0$. 
It is clear that $\div F_{M2} = 0$ and 
\[
[F_{M2}, F_{M2}] = 0
\] 
by simple type reasons. 
\end{proof}

We summarize the output of our computation with a definition. 

\begin{defn}\label{defn:ads4}
Let $\mc E^N_{AdS_4\times S^7}$ denote the classical BV theory on $\operatorname{Tot} (K_\C^{1/4}\otimes \C^4 \to \R\times \C_z)\setminus 0(\R\times \C)$ given by the sheaf of cochain complexes $\mc E |_{(\R\times \C)\times (\C^4\setminus \{0\} )}$, with BV pairing induced from $\mc E$, deformed by the interaction \[S_{BF,\infty}(\mu + NF_{M2}, \nu, \beta, \gamma) + J(\gamma).\]
\end{defn}

\begin{rmk}
Note that upon expanding the interaction around $NF_{M2}$, the cubic term in $S_{BF,\infty}$ will contribute a differential which acts on $\gamma$ and $\mu$ by bracketing with $NF_{M2}$. We accordingly denote this differential $[NF_{M2}, - ] $, and we see that the sheaf of cochain complexes underlying $\mc E_{AdS_4\times S^7}$ is in fact \[\left ( \mc E |_{(\R\times \C)\times (\C^4\setminus \{0\} ) }, \delta^{(1)} + [NF_{M2},-] \right )\] where $\delta^{(1)}$ denotes the original linearized BRST differential.
\end{rmk}

\begin{conj}\label{conj:ads4}
The minimal twist of eleven-dimensional supergravity on the $AdS_4\times S^7$ background with $N$ units of M2 brane flux supported on $S^7$ is perturbatively equivalent to $\mc E^N_{AdS_4\times S^7}$.
\end{conj}

To verify this conjecture, we should directly twist eleven-dimensional supergravity on the $AdS_4\times S^7$ spacetime. Doing so seems difficult - while it is likely not hard to identify the covariantly constant nilpotent spinors which define the twist, it seems more difficult to establish a perturbative equivalence with our description above. A modification of the pure spinor superfield formalism to symmetric spaces such as cosets for the superconformal group might make such checks more feasible. In lieu of such, we will instead pursue other consistency checks in the following two sections{}. 

\subsection{The ${\mr AdS}_7 \times S^4$ background}

We similarly introduce an analog of the ${\mr AdS}_7 \times S^4$ background in our description of the minimal twist of eleven-dimensional supergravity. As before, we begin by viewing our eleven-dimensional manifold $\R\times \C^5$ as 
\[
\operatorname{Tot}(\R \oplus K_{\C^3}^{1/2}\otimes \C^2 \to \C^3_z)
\]
to record weights under natural scaling actions. We once again will use $w_a$ to denote holomorphic fiber coordinates on $K^{1/2}_{\C^3}\otimes \C^2$, and we use $t$ to denote a fiber coordinate on the trivial bundle $\R \to \C^3_z$.  

\parsec[sec:m5coupling]

To repeat the procedure in the previous subsection, we begin by discussing how the eleven-dimensional theory couples to M5 branes. 
Consider a stack of $N$ M5 branes wrapping the zero section $\C^3_z$. 

It is natural to consider the nonlocal interaction 
\[
I_{M5} = N\int_{\C^3_z} \div^{-1}\mu \vee \Omega +\cdots 
\]
Note that this expression is only nonzero on the component of $\mu$ in $\PV^{1,3}$. 
We argue that this coupling is consistent with expectations from the physical theory and from dimensional reduction. 

The twisted field $\mu^{1,3}$ is a component of the Hodge star of the $G$-flux in the physical theory (\S\ref{s:components}). 
In the physical theory, M5 branes magnetically couple to the $C$-field; the coupling involves choosing a primitive for the Hodge star of the $G$-flux and integrating it over the M5 worldvolume. Our twist contains no fields corresponding to components of such a primitive; hence such a magnetic coupling is reflected in the appearance of $\div^{-1}$. 

We may once again justify this coupling by dimensional reduction to IIA supergravity. Reducing on the circle along the directions the M5 branes wrap yields the $SU(4)$ invariant twist of type IIA supergravity on $\C^4 \times \R^2$ with $N$ $D4$ branes wrapping $\C^2 \times \R$. 

In \cite{CLsugra}, it is shown that the magnetic coupling of $D4$ branes to the $SU(4)$ twist of IIA is of the form
\[
N \int _{\C^2 \times \R} \div^{-1} \mu \vee \Omega_{\C^4} + \cdots .
\]
Again, we have only explicitly indicated the first-order piece of the coupling. 

\parsec[s:m5backreact]

The backreacted geometry will be given by a solution to the equations of motion upon deforming the eleven-dimensional action by the interaction $I_{M5}(\mu)$. 

Varying the potential $\div^{-1} \mu$, we obtain the following equation of motion involving the field $\gamma$:
\begin{equation}\label{eqn:m5eom1}
\dbar \del \gamma + \div \left(\frac{1}{1-\nu} \mu\right) \wedge \del \gamma = N \delta_{w_1=w_2=t=0} .
\end{equation}
Notice that there is an extra derivative compared to the equation of motion arising from varying the field $\mu$. 
This equation only depends on $\gamma$ through its field strength $\del \gamma$. 

Varying $\gamma$ we obtain the equation of motion 
\begin{equation}\label{eqn:m5eom2}
(\dbar + \d_\R) \mu + \del \gamma \del \gamma = 0 .
\end{equation} 
Again, this only depends on $\gamma$ through its field strength $\del \gamma$.


\begin{lem}
\label{lem:ads7flux}
Let
\[
F_{M5} = \frac{1}{(2\pi i)^3} \frac{\bar{w}_1 \d \bar{w}_2 \wedge \d t - \bar{w}_2 \d \bar{w}_1 \wedge \d t + t \d \bar{w}_1 \wedge \d \bar{w}_2}{(\|w\|^2 + t^2)^{5/2}} \wedge \d w_1 \wedge \d w_2
\]
Then, $\del\gamma = N F_{M5}$, $\mu = 0$, and $\nu = 0$ satisfies the equations of motion in the presence of a stack of $N$ M5 branes sourced by the term $N \delta_{w_1=w_2=t=0}$:
\begin{align*}
\dbar (NF_{M5}) + \d_{\R} (NF_{M5}) & = N \delta_{w_1=w_2=t=0}  \\ 
(NF_{M5}) \wedge (NF_{M5}) & = 0 .
\end{align*}
Here, we set all components of the field $\mu$ equal to zero (as well as the fields $\nu,\beta$). 
\end{lem}

\begin{proof}
The first equation,
\[
\dbar F + \d_{\R} F = N \delta_{w_1=w_2=t=0},
\]
characterizes the kernel representing $N$ times the residue class for a four-sphere in 
\[
(\C^2 \times \R) \setminus 0 \simeq S^4 \times \R .
\] 
That is
\[
\oint_{S^4} N F = N 
\]
for any four-sphere centered at $0 \in \C^2 \times \R$.

The second equation $F \wedge F = 0$ follows by simple type reasons. 
\end{proof}

Once again, we summarize our findings in a definition. 
\begin{defn}\label{defn:ads7}
Let $\mc E^N_{AdS_7\times S^4}$ denote the classical BV theory on $\operatorname{Tot}(\R \oplus K_{\C^3}^{1/2}\otimes \C^2 \to \C^3_z)\setminus 0(\C^3_z)$ given by the sheaf of cochain complexes $\mc E |_{\C^3\times (\R\times \C^2\setminus \{0\} )}$, with BV pairing induced from that on $\mc E$, deformed by the interaction \[S_{BF,\infty}(\mu, \nu, \beta, \gamma +N\del^{-1} F_{M5}) + J(\gamma + N \del^{-1} F_{M5}).\]
\end{defn}

\begin{rmk}
Note that both terms in the action only depend on $\gamma$ through its holomorphic derivatives so the above expression for the action is indeed well-defined. 

As before, upon expanding the interactions around $NF_{M5}$, the cubic terms in both $S_{BF,\infty}$ and $J$ will contribute differentials. From $S_{BF,\infty}$, we get a differential which takes a $\mu$ type field to the Schouten bracket $N[F_{M5}, \mu]$ and from $J$, we get a differential which acts as $\gamma\mapsto N F_{M5} \wedge \del \gamma$. We accordingly denote this differential $[NF_{M5}, - ] $, and the sheaf of cochain complexes underlying $\mc E_{AdS_7\times S^4}$ is in fact \[\left (\mc E |_{\C^3\times (\R\times \C^2\setminus \{0\} )}, \delta^{(1)} + [NF_{M5},-] \right )\] where $\delta^{(1)}$ denotes the original linearized BRST differential.
\end{rmk}

\begin{conj}\label{conj:ads7}
The minimal twist of eleven-dimensional supergravity on the $AdS_7\times S^4$ background with $N$ units of M5 brane flux supported on $S^4$ is perturbatively equivalent to $\mc E_{AdS_7\times S^4}$. 
\end{conj}
