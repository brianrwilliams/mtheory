\documentclass[11pt]{amsart}

\usepackage{macros-mtheory,amsaddr}

\def\vep{\varepsilon}

\addbibresource{cfs.bib}

%\linespread{1.2} %for editing
%\usepackage{mathpazo}

\begin{document}

\section{Localization to a 5-dimensional theory} 

In the last section we described a further twist 

\subsection{The superconformal deformation}

The localization of the physical 11-dimensional supergravity theory is with respect to a supercharge of the form 
\[
{\bf Q} = Q + S 
\]
where $Q$ is the supertranslation defining the holomorphic twist. 
The element $S$ does not live in the algebra of supertranslations, but rather it lives in the superconformal algebra...

We consider the holomorphic two-form
\[
S(a,b,c) = a \, w_1 \d w_2 \wedge \d w_3 + b \, w_2 \d w_3 \wedge \d w_1 + c \, w_3 \d w_1 \wedge \d w_2 .
\]
In order for $S(a,b,c)$ to be $\del$-closed we have the following constraint on the variables $a,b,c$:
\beqn\label{eqn:abc}
a + b + c = 0 .
\eeqn
Using this equation, we can eliminate a single variable. 
We reparameterize using two new variables $a = \ep$, $b=\delta$, $c = -\ep - \delta$ and denote
\[
S(\vep, \delta) \define \ep \, w_3 \d w_1 \wedge \d w_2 + \delta \, w_2 \d w_3 \wedge \d w_1 - (\ep + \delta) \, w_1 \d w_2 \wedge \d w_3 .
\]

\subsection{Localized global symmetry algebra}

Up to a copy of constant functions, the global symmetry algebra of the holomorphic twist of the 11-dimensional theory is the super Lie algebra $E(5,10)$.
From this point of view, the global symmetry algebra of the $S$-deformed theory is given by deformation of this super Lie algebra by the Maurer--Cartan element 
\[
S(\vep, \delta) \in \Omega^{2}_{cl}(\CC^5) .
\]
We recall that the space of closed two-forms on $\CC^5$ is precisely the odd part of the super Lie algebra $E(5,10)$. 

We compute the cohomology of $E(5,10)$ with respect to the differential which is bracketing with this Maurer--Cartan element. 
Recall that we are using the holomorphic coordinates $(z_1,z_2,w_1,w_2,w_3)$ on $\CC^5$. 

We record the following brackets in $E(5,10)$:
\begin{align*}
[f_i \del_{z_i}, S(\vep,\delta)] & = 0 \\
[g_a \del_{w_a}, S(\vep, \delta)] & = \vep w_1 (\del g_2 \wedge \d w_3 - \del g_3 \wedge \d w_2) 
+ \delta w_2 (\del g_3 \wedge \d w_1 - \del g_1 \wedge \d w_3) \\
& - (\vep + \delta) w_3 (\del g_1 \wedge \d w_2 - \del g_2 \wedge \d w_1) \\
[h \d z_1 \wedge \d z_2 , S(\vep, \delta)] & = h \left(\vep w_1 \del_{w_1} + \delta w_2 \del_{w_2} - (\ep + \delta) w_3 \del_{w_3} \right) \\
[h^{ia} \d z_i \wedge \d w_a, S(\vep, \delta)] & = \vep \ep_{ij} h^{i1} w_1 \del_{z_j} + \delta \ep_{ij} h^{i2} w_2 \del_{z_j} - (\vep + \delta) \ep_{ij} h^{i3} w_3 \del_{z_j} .
\end{align*} 

From these formulas, we can read off the elements which are in the kernel of $[S,-]$:
\begin{itemize}
\item $f_i \del_{z_i}$ where $f_i \in \cO(\CC^5)$. 
\item $\lambda_a \del_{w_a}$, where $\lambda_a$ is a constant for $a=1,2,3$.  
\end{itemize}
In other words, the annihilator of $[S,-]$ in $E(5,10)$ is, up to completions, isomorphic to 
\[
\CC[z_i, w_a] \{\del_{z_i}\} \oplus \CC\{\del_{w_a}\} .
\]

\subsection{Deformation of $M2$ and $M5$ branes}

We argue that when restricted to $M2$ and $M5$ branes that the supercharge $S(\ep, \delta)$ becomes explicit superconformal elements which live in the holomorphic twist of the relevant $3d$ and $6d$ superconformal algebras, respectively. 

\parsec[]

Let's first consider the $M2$ brane case. 
In the notation of this section, let us take the $M2$ brane to live along $z_1 = z_2 = w_1 = w_2 = 0$:
\[
\{0\} \times \{0\} \times \CC_{w_3} \times \RR_t \subset \CC^2_{z_i} \times  \CC^3_{w_a} \times \RR_t .
\]

Recall that the odd part of the 3-dimensional superconformal algebra, which can be identified with $\lie{osp}(6|1)$, is 
\[
\wedge^2 W \otimes R 
\]
where $W$ is the defining $\lie{sl}(4)$ representation and $R$ is the defining $\lie{sl}(2)$ representation. 
In \ref{s:m2embedding} we polarized $R$ and decomposed this as $(\wedge^2 W)_{+1} \oplus (\wedge^2 W)_{-1}$. 

Along the brane, the first summand $(\wedge^2 W)_{+1}$ corresponds to the residual supertranslations in the twist of the 3d $\cN=8$ supersymmetry algebra. 
The second summand $(\wedge^2 W)_{-1}$ corresponds to residual superconformal transformations. 
It is spanned by elements of the form 
\[
w_3 \otimes (v_\alpha \wedge v_\beta)
\]
where $w_3$ is the holomorphic coordinate along the $M2$ brane and $\{v_\alpha\}$ is a basis for $W \cong \CC^4$.
In the notation of the 11-dimensional theory, $W$ is spanned by translation invariant holomorphic one-forms $\{\d z_1, \d z_2, \d w_1, \d w_2\}$. 

Restricting the element $S(\ep,\delta)$ to the $M2$ brane we obtain precisely an element of this form: 
\[
S(\ep,\delta)|_{M2} = \ep w_3 \otimes (v_3 \wedge v_4).
\]
Notice that the dependence on $\delta$ has completely dropped since $\delta$ is proportional to terms involving the linear functions $w_1,w_2$ which are set to zero along the brane. 


\parsec[]
Next, let's consider the restriction of the element $S(\ep, \delta)$ to an $M5$ brane. 

We take the $M5$ brane to live along $z_2 = w_3 = t = 0$:
\[
\CC_{z_1} \times \{0\} \times \CC_{w_1} \times \CC_{w_2} \times \{0\} \subset \CC^2_{z_i} \times  \CC^3_{w_a} \times \RR_t .
\] 

\brian{I'm using $R_{6d} = \CC.\d z_2 \oplus \CC. \d w_3$.}
Recall that the odd part of twist of the 6-dimensional superconformal algebra (which is also $\lie{osp}(6|1)$) can be identified with 
\[
L \otimes R \oplus \wedge^2 L \otimes R 
\]
where $L \cong \CC^3$ is the defining $\lie{sl}(3)$ representation and $R \cong \CC^2$ is the defining $\lie{sl}(2)$ representation. 
A basis for $L$ is given by the constant holomorphic one-forms along the brane, which in this notation is $\{\d z_1, \d w_1, \d w_3\}$. 
A basis $\{r_1,r_2\}$ for $R$ corresponds in 11-dimensions to the constant holomorphic one-forms $\{\d z_1, \d w_3\}$. 

The first summand $L \otimes R$ corresponds to residual supertranslations of the 6d $\cN=(2,0)$ superconformal algebra after the holomorphic twist. 
The second summand $\wedge^2 \otimes R$ corresponds to residual superconformal elements. 

Following our embedding from \S \ref{s:m5embedding} we find that the restriction of $S(\ep, \delta)$ is precisely one of these superconformal elements:
\[
S(\ep, \delta) = - \left(\delta w_2 \d w_1 + (\ep + \delta) w_1 \d w_2 \right) \otimes r_2 .
\]


\subsection{Relationship to the $\Omega$-background}


In \cite{CostelloOmega}, Costello defines the $\Omega$-background for supergravity. 
The main example treated is a certain $\Omega$-background for the non-minimal twist of 11-dimensional supergavity on 
\[
\CC^2 \times TN_k \times \RR^3 .
\]
Here, $TN_k$ is a Taub-NUT manifold with an $A_k$-singularity at the origin. 
One treats $\CC^2$ as a manifold with holonomy in $SU(2)$ and $TN_k \times \RR^3$ as a manifold with holonomy in $G_2$.

In the language of the minimal twist on 
\[
\CC^5 \times \RR = \CC^2_{z} \times \CC^3_w \times \RR_t 
\]
recall that the {\em nonminimal} twist arises from a background where the field strength of the field $\gamma$ takes the value 
\[
\del \gamma_{nm} = \d z_1 \wedge \d z_2 .
\]
This breaks the holonomy from $SU(5)$ to $SU(2)$.
Furthermore, we have shown that this background renders the theory topological along $\CC^3_w$. 

\brian{equivariant...}

Therefore, we want to find a field $\gamma(a,b,c)$ such that
\beqn\label{eqn:euler}
[\gamma_{nm} , \gamma(a,b,c)] = a w_1 \del_{w_1} + b w_2 \del _{w_2} + c w_3 \del_{w_3} .
\eeqn
Recall that this bracket only depends on the field strengths of the fields. 

Using $\del \gamma_{nm} = \d z_1 \wedge \d z_2$ we can easily find a solution for the field strength of $\gamma(a,b,c)$:
\[
\del \gamma(a,b,c) = a \, w_1 \d w_2 \wedge \d w_3 + b \, w_2 \d w_3 \wedge \d w_1 + c \, w_3 \d w_1 \wedge \d w_2 .
\]
In order for the expression for $\del \gamma(a,b,c)$ to be $\del$-closed, so that such a holomorphic one-form $\gamma(a,b,c)$ exists, we have the condition
\beqn\label{eqn:abc}
a + b + c = 0 .
\eeqn
Notice that we can add to this $\gamma(a,b,c)$ any holomorphic one-form on $\CC^2_z$ and it will still satisfy \eqref{eqn:euler}. 

 
\subsection{Localization to 7d}

Consider
\[
S(\ep) = \ep (w_1 \d w_2 \wedge \d w_3 + w_2 \d w_1 \wedge \d w_3) .
\]

The annihilator of $[S(\ep),-]$ in $E(5,10)$ is linearly spanned by the following elements
\begin{itemize}
\item Divergence-free vector fields of the form $f_i(z_1,z_2, w_3) \del_{z_i}$. 
\item Vector fields of the form $(w_1 w_2)^n \del_{w_3}$. 
\item $\cdots$
\end{itemize}


\appendix 

\section{Branes}

\subsection{$\Omega$-deformed symmetry algebra}

On flat space, we have 



\subsection{M2 branes in the $\Omega$-background} 

\parsec[]

In the twisted $\Omega$-background a stack of $N$ M2 branes wraps 
\beqn\label{eqn:m2omega}
\{0\} \times \RR \subset \CC^2 \times \RR .
\eeqn

The field which sources this brane is a distributional differential form $F$.
The linear equation of motion for $F$ is the distributional equation 
\[
\d^2 z \wedge \dbar F = N \delta_{z=0} 
\]
where $\delta_{z=0}$ is the $\delta$-distribution for the submanifold \eqref{eqn:m2omega}.  
In what follows we denote $\til{\delta}_{z=0}$ the distributional $(0,2)$ form defined by contracting $\delta_{z=0}$ with the Calabi--Yau form~$\d^2 z$. 

The field $F$ is a smooth Dolbeault form of type $(0,1)$ away from the locus of the brane $\CC^2 \times \RR_t \setminus \{0\} \times \RR \cong (\CC^2 \setminus 0) \times \RR_t$. 
Explicitly, we can characterize it as follows. 

\begin{prop}
The field $F \in \Omega^{0,1} (\CC^2 \setminus 0) \otimes \Omega^{0} (\RR) $ defined by 
\[
F = \# N \frac{\zbar_1 \d \zbar_2 - \zbar_2 \d \zbar_1}{\|z\|^4} 
\] 
satisfies the non-linear equation of motion 
\[
\dbar F + \frac12 F \star_c F = N  \til{\delta}_{z=0} .
\]
\end{prop}
\begin{proof}
The equation $\dbar F = N  \til{\delta}_{z=0}$ characterizes the Bochner--Martinelli kernel representing the residue class in $\CC^2 \setminus 0$. 
Since $F$ is a one-form $F \star_c F = 0$ is satisfied by anti-symmetry. 
\end{proof}

\parsec[]

We can interpret $F$ geometrically in the following way. 
Let's first consider just the complex geometric situation, forgetting about the $\RR$-direction. 
 
On any hyper K\"ahler manifold $X$ there is an exact sequence of sheaves of Lie algebras
\[
0 \to \ul\CC \to \cO^{hol}(X) \to \Vect_0^{hol} (X) \to 0
\]
where $\Vect^{hol}_0(X)$ is the Lie algebra of holomorphic divergence-free vector fields and $\cO^{hol}(X)$ is equipped with the Poisson bracket induced by the holomorphic symplectic form. 
The map $\cO^{hol}(X) \to \Vect_0(X)$ is given by the holomorphic de Rham operator $\del \colon \cO^{hol}(X) \to \Omega^{1,hol}_{cl}(X)$ followed by the isomorphism 
$\Omega^{1,hol}_{cl} (X) \cong \Vect_0(X)$ given by the holomorphic symplectic form. 

Extending this exact sequence to the level of Dolbeault resolutions, we see that we can view $F_0 \in \Omega^{0,1}(\CC^2 \setminus 0)$ as a $(0,1)$ Dolbeault valued divergence-free vector field $\til{F}_0$ on $\CC^2 \setminus 0$. 
Explicitly, if $F_0 = f^i (z,\zbar) \d \zbar_i$ then this $(0,1)$-valued vector field is
\[
\til{F}_0 = \ep_{jk} \partial_{z_j} f^i (z,\zbar) \d \zbar_i \partial_{z_k} .
\]

In the case at hand, $F \in \Omega^{0,1}(\CC^2 \setminus 0) \otimes \Omega^0(\RR)$ is given by the restriction along 
\[
(\CC^2 \setminus 0) \times \RR \to \CC^2 \setminus 0
\]
of the $(0,1)$-form
\[
F_0 = \# N \frac{\zbar_1 \d \zbar_2 - \zbar_2 \d \zbar_1}{\|z\|^4} .
\] 
The corresponding $(0,1)$-valued vector field on $\CC^2 \setminus 0$ is
\[
\til{F}_0 = \#' N \frac{\zbar_1 \d \zbar_2 - \zbar_2 \d \zbar_1}{\|z\|^6} \left(\zbar_1 \del_{z_2} - \zbar_2 \del_{z_1}\right) .
\] 

\end{document}