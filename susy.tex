\section{Residual supersymmetry} 
\label{sec:susy}
%m2brane

In this section we consider the minimal twist of 11-dimensinoal supersymmetry explicitly. 
We compute the residual supersymmetry algebra given by taking the cohomology of the 11-dimensional supersymmetry algebra with respect to the minimal twisting supercharge. 
In order for this to be a symmetry of the 11-dimensional theory that we proposed in Section \ref{sec:dfn} it is necessary to perform a central extension of the 11-dimensional supersymmetry algebra by the $M2$ brane.

\parsec[sec:susy]

The (complexified) eleven-dimensional supertranslation algebra is a complex super Lie algebra of the form
\[
  \ft_{11d} = V \oplus \Pi S
\]
where $S$ is the (unique) spin representation and $V \cong \CC^{11}$ the complex vector representation, of~$\lie{so}(11, \CC)$. 
The bracket is the unique surjective $\lie{so}(11,\CC)$-equivariant map from the symmetric square of~$S$ to~$V$;
this decomposes into three irreducibles, 
\beqn\label{eqn:decomp}
  \Sym^2(S) \cong V \oplus \wedge^2 V \oplus \wedge^5 V.
\eeqn
Denote by $\Gamma_{\wedge^1}, \Gamma_{\wedge^2}, \Gamma_{\wedge^5}$ the projections onto each of the summands above. 
The bracket in $\ft_{11d}$ is defined using the first projection
\[
[\psi, \psi'] = \Gamma_{\wedge^1} (\psi, \psi') .
\]
The super Poincar\'{e} algebra is
\[
  \lie{siso}_{11d} = \lie{so}(11 , \CC) \ltimes \ft_{11d} .
\]
The $R$-symmetry is trivial in 11-dimensional supersymmetry. 


\parsec[sec:m2brane]

Extensions of the supersymmetry algebra correspond to the existence of branes in the original theory of supergravity \brian{good reference?}. 
In 11-dimensional supersymmetry, there are two such extensions corresponding to the $M2$ brane and the M5 brane.
We begin by describing a dg Lie algebra model for the $M2$ brane algebra, then we will discuss its relationship to other descriptions as a Lie 3-algebra \cite{Basu_2005,Bagger_2007,fiorenza2015super}. 

The $M2$ brane algebra is dg Lie algebra extension of the super Poincar\'e algebra $\lie{siso}_{11d}$.
Introduce the cochain complex $\Omega^{\bu}(\RR^{11})$ of (complex valued) differential forms on $\RR^{11}$ equipped with the de Rham differential $\d$.  
 
The super dg Lie algebra $\m2$ is the extension of $\lie{siso}_{11d}$ by the cocycle
  \[
    c_{M2} \in \clie^2\left(\lie{siso}_{11d} \; ; \; \Omega^\bu (\RR^{11})[2]\right)
  \]
  defined by the formula 
  \[c_{M2} (\psi, \psi') = \Gamma_{\wedge^2}(\psi, \psi') \in \Omega^2(\RR^{11})\]
  where $\Gamma_{\wedge^2}$ is the projection onto $\wedge^2 V$, thought of as the space of constant coefficient two-forms, as in the decomposition \eqref{eqn:decomp}.

We view $\m2$ as a $\ZZ \times \ZZ/2$-graded dg Lie algebra where the bracket is bidegree $(0,+)$ and the differential is bidegree $(1,+)$.

\parsec[sec:m2branetwist]

Fix a rank one supercharge $Q \in S$ satisfying $Q^2 = 0$.
This supercharge defines the holomorphic twist of 11-dimensional supergravity. 
\brian{cite \cite{SWspinor}}
We characterize the cohomology of the algebra $\m2$ with respect to this supercharge. 

Such a supercharge defines a maximal isotropic subspace $L \subset V$. 
We can decompose the algebra into $\lie{sl}(L) = \lie{sl}(5)$ representations by
\deq{
  V = L \oplus L^\vee \oplus \CC_t, \qquad S = \wedge^\bu L^\vee.
}
In the expression for $S$, we are omitting factors of $\det(L)^{\frac12}$ for simplicity. 
Also, $\lie{so}(11, \CC)$ decomposes as
\[
\lie{sl}(5) \oplus \wedge^2 L \oplus \wedge^2 L^\vee \oplus L \oplus L^\vee .
\]
Furthermore, the spinorial representation can be identified with
\[
S = \wedge^\bu (L^\vee) = \CC \oplus L^\vee \oplus \wedge^2 L^\vee \oplus \wedge^3 L^\vee \oplus \wedge^4 L^\vee \oplus \wedge^5 L^\vee .
\]
The element $Q$ lives in the first summand.
Let ${\rm Stab}(Q) \subset \lie{so}(11,\CC)$ be the stabilizer of $Q$. 
This is a parabolic subalgebra whose Levi factor is $\lie{sl}(5)$.

\begin{prop}\label{prop:model}
Fix a holomorphic supercharge $Q \in S$ and let $\m2^Q$ be the dg Lie algebra with bracket that on $\m2$ and with the differential $\d + [Q,-]$. 
As a $\ZZ/2$ graded space, the cohomology $H^\bu(\m2^Q)$ is
\[
    L^\vee \oplus {\rm Stab}(Q) \oplus \Pi \left(\wedge^3 L^\vee\right) \oplus \CC
  \]
whose elements we denote by $(v, m, \psi, c)$.

The transferred $L_\infty$ structure on $H^\bu(\fg) \cong H^\bu(\m2^Q)$ is a Lie-3 algebra given by the extension of ${\rm Stab}(Q)$ together with the brackets
\begin{align*}
[\psi, \psi']_2 & = \psi \wedge \psi' \in \wedge^4 L \cong L^\vee_v \\
[v, v', \psi]_3 & = \<v \wedge v', \psi\> \in \CC_c .
\end{align*}
\end{prop}

\parsec[]

As a consequence of the above proposition, we see that the dg Lie algebra $(\m2,Q)$ is {\em not} formal; there is a 3-ary $L_\infty$ bracket present in cohomology. 
Instead of this $L_\infty$ description, there is the following minimal model of this dg Lie algebra, which we will use to relate to the 11-dimensional theory. 
This lemma implies Proposition \ref{prop:model}.

\begin{lem}
\label{lem:gmodel}
Let $\fg$ denote the following $\ZZ/2$ graded dg Lie algebra which as a cochain complex is
\[
H^\bu(\m2^Q) \oplus (\Pi L \xto{\id} L)  .
\]
Denote the elements of the second summand by $(\tilde{\lambda}, \lambda)$. 
The nontrivial Lie brackets are
\begin{align*}
[v,\lambda] & = \<v, \til\lambda\> \in \CC_c \\ 
[v,\psi] & = \<v, \psi\> \in \Pi L_{\Tilde{\lambda}} \\
[\psi, \psi'] & = \psi \wedge \psi' \in L^\vee_v  .
\end{align*}
There is an $L_\infty$ map 
\[
\fg \rightsquigarrow \m2^Q
\] 
which is a quasi-isomorphism of cochain complexes.  
\end{lem}
\begin{proof}
The supercharge $Q$ is odd and of cohomological degree zero.
Since the differential on $\m2$ is even of cohomological degree $+1$, only a totalized $\ZZ/2$ grading makes the differential $\d + [Q,-]$ homogenous. 

The cohomology of the non-centrally extended algebra was computed in \cite{SWspinor}, we briefly recall the result. 
The element $Q$ only acts nontrivially on the summands $\wedge^4 L^\vee$ and $\wedge^5 L^\vee$ in $S$. 
The image of $\wedge^4 L^\vee \cong L$ trivializes the antiholomorphic translations while the image of $\wedge^5 L^\vee$ trivializes the time translation.
So, of the translations, only the holomorphic ones $L^\vee$ survive.
The map 
\[
[Q,-] \colon \lie{so}(11,\CC) \to S 
\] 
is the projection onto $\wedge^0 L^\vee \oplus \wedge^1 L^\vee \oplus \wedge^2 L^\vee$. 
The kernel of $[Q,-]$ is the stabilizer ${\rm Stab}(Q)$.

In summary, the space of odd translations which survive cohomology is $\wedge^3 L^\vee \cong \wedge^2 L$.
This completes the calculation of the cohomology. 

We embed $\fg$ into $\m2^Q$ in the following way: ${\rm Stab}(Q)$ and $L^\vee$ sit inside in the evident way.
The central element maps to $c \mapsto \pm 1 \in \Omega^0(\RR^{11})$.
The summand $L_\lambda$ is mapped to the linear functions in $\Omega^0(\RR^{11})$ and $\Pi L_{\Tilde{\lambda}}$ is sent to the constant coefficient one-forms in $\Pi \Omega^1(\RR^{11})$. 
It remains to declare where $\psi \in \wedge^2 L$ is mapped.

Define the map
\[
H \colon \Omega^2 (\RR^{11}) \to \Omega^1(\RR^{11})
\]
which sends a two-form $\alpha$ to the one-form $H \alpha$ defined by the formula
\[
(H \alpha) (x) = \int_0^x \alpha .
\]

Notice that if $\alpha$ is $\d$-closed then $\d (H \alpha) = \alpha$. 
It follows that any element $\psi \in \wedge^2 L \subset S$ can be lifted to a closed element at the cochain level in $\m2^Q$ by the formula
\[
\Tilde{\psi} = \psi - H \Gamma_{\wedge^2} (Q, \psi) \in \Pi S \oplus \Pi \Omega^1 .
\]
Thus, sending $\psi \mapsto \Tilde{\psi}$ defines a cochain map $\fg \to \m2^Q$. 

The Lie bracket $[\Tilde{\psi}, \Tilde{\psi}']$ agrees with $[\psi, \psi']$. 
On the other hand, in $\m2^Q$ there is the Lie bracket 
\[
[v,\Tilde{\psi}] = - L_v (H \Gamma_{\wedge^2} (Q, \psi)) = -\<v, \Gamma_{\wedge^2}(Q, \psi) - \d \<v, H \Gamma_{\wedge^2}(Q, \psi) .
\]
The first term agrees with the bracket $[v, \psi]_{\fg}$ in $\fg$. 
The other term is exact in $\m2^Q$ and can hence be corrected by the following bilinear  
\[
v \otimes \psi \mapsto \<v, H \Gamma_{\wedge^2} \> \in L_\lambda .
\] 
Together with the cochain map described above, this bilinear term prescribes the desired $L_\infty$ map. 

\end{proof}

\parsec[s:residual]
Consider now the super $L_\infty$ algebra $\cL$ underlying the eleven-dimensional theory on $\CC^5 \times \RR$. 

\begin{prop}
There is a linear map of super $L_\infty$ algebras 
\[
\fg \to \cL 
\]
where $\fg$ is the model for the $Q$-cohomology of the super Lie algebra $\m2$ from Proposition \ref{lem:gmodel}. 
In particular, the $Q$-twisted algebra $\m2^Q \simeq \fg$ is a symmetry of 11-dimensional theory on $\CC^5 \times \RR$. 
\end{prop}
\begin{proof}
Recall the model $\fg$ takes the following form
\beqn 
\begin{tikzcd}
\ul{even} & \ul{odd} & \ul{even} \\
 L_1 & \wedge^2 L_2 & L^\vee \\
\wedge^2 L_1 & & \\
\lie{sl}(5) \\
L_2 \ar[r, "\id"] & L_3 \\ 
\CC_c & .
\end{tikzcd}
\eeqn
The subscripts in $L_1, L_2, L_3$ are used to distinguish between the various copies of $L$.

The map from the dg model $\fg$ to the fields of the twisted $11$-dimensional supergravity theory is as follows. 
\begin{align*}
 L_1 & \mapsto 0 \\
 \wedge^2 L_1 & \mapsto 0 \\
z_i \wedge z_j \in \wedge^2 L_2 & \mapsto \frac12 (z_i \d z_j - z_j \d z_i) \in \Omega^{1,0} (\CC^5) \hotimes \Omega^0 (\RR) \\
A_{ij} \in \lie{sl}(5) & \mapsto \sum_{ij} A_{ij} z_i \in \PV^{1,0}(\CC^5) \hotimes \Omega^0(\RR) \\\frac{\partial}{\partial z_j} \in L^\vee & \mapsto
\frac{\partial}{\partial z_i} \in \PV^{1,0} (\CC^5) \hotimes \Omega^0 (\RR^5) \\ z_i \in L_2 & \mapsto z_i \in \Omega^{0,0}(\CC^5) \hotimes \Omega^0 (\RR) \\
z_i \in L_3 & \mapsto \d z_i \in \Omega^{1,0}(\CC^5) \hotimes \Omega^0 (\RR) \\
c \in \CC_c & \mapsto c \in \Omega^{0,0}(\CC^5) \hotimes \Omega^0 (\RR) .
\end{align*}

It is immediate to check that this is a map of cochain complexes where one equips $\cL$ with the linear BRST differential \eqref{eqn:linearBRST}.
Since $\nu$ does not appear in the image of this map, to check that the higher Lie brackets are preserved one only needs to check that the $2$-ary bracket is preserved and this is immediate. 
\end{proof}

Because this map preserves differentials, it descends to a map in cohomology. 
We have already computed the cohomology of $\cL$ on $\CC^5 \times \RR$, it is the trivial one-dimensional central extension of $E (5,10)$. 
The cohomology of $\fg$ is described in Proposition \ref{prop:model}. 
The map
\[
L^\vee \oplus {\rm Stab}(Q) \oplus \Pi \left(\wedge^3 L^\vee\right) \oplus \CC_c \to E (5,10) \oplus \CC_{c'}
\]
is defined by very similar formulas as above
\begin{align*}
 L_1 & \mapsto 0 \\
 \wedge^2 L_1 & \mapsto 0 \\
z_i \wedge z_j \in \wedge^2 L_2 & \mapsto \d z_i \wedge \d z_j \in \Omega^{2}_{cl} (\CC^5) \\
A_{ij} \in \lie{sl}(5) & \mapsto \sum_{ij} A_{ij} z_i \in \Vect_0(\CC^5) \\ \frac{\partial}{\partial z_j} \in L^\vee & \mapsto
\frac{\partial}{\partial z_i} \in \Vect_0(\CC^5) \\
c \in \CC_c & \mapsto c \in \CC_{c'} .
\end{align*}

\parsec[]

In this section we compare to the work of \brian{Linfty refs}.
In these references, the algebra $\m2$ is defined as an $L_\infty$  
central extension of $\lie{siso}_{11d}$. 

Recall that given two spinors $\psi, \psi' \in S$ we can form the constant coefficient two-form $\Gamma_{\wedge^2} (\psi, \psi')$. 
Using this two-form we can define the following four-linear expression
\[
\mu_2 (\psi, \psi',v,v') = \<v \wedge v', \Gamma(\psi, \psi')\> .
\]
This expression is symmetric on the spinors and anti-symmetric on the vectors, therefore it defines an element in $\clie^4(\ft_{11d})$. 
In \cite{??} it is shown that this cocycle $\mu_2$ is 

%For clarity we adjust notation for $\lie{sl}(5)$-representations. Denote by $V^{1,0} = L^\vee$ the space of holomorphic translations on $\CC^5$ and $V^{\vee 1,0}$ the translation invariant holomorphic one-forms on $\CC^5$. 