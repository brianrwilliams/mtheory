\documentclass[11pt]{amsart}

\usepackage{macros-master,amsaddr,physics,amssymb}

\addbibresource{refs}

\renewcommand{\div}{\del_{\Omega}}
\renewcommand{\PV}{\mathrm{PV}}
\renewcommand{\op}{\operatorname}
\newcommand{\CB}{\mathbb{C}\mathbb{B}}
\newcommand{\SB}{\mathbb{S}\mathbb{B}}

\setcounter{tocdepth}{1}

\begin{document}
\title{Notes on agt}
\maketitle


This is a test. Simply a test.

\section{The twisted brane construction}

We will use the fact that the minimal twist of $M$-theory exists on geometries of the form
\beqn
Z \times \R
\eeqn
where $Z$ is a Calabi--Yau fivefold.

We consider familiar extended objects, which in the twisted setting have the following flavor.
\begin{itemize}
\item[(M5)] Twisted fivebranes wrap an embedded threefold
\beqn
X \subset Z
\eeqn
with the property that $K_X \simeq \wedge^2 N_{X \subset Z}$ where $N_{X \subset Z}$ is the normal bundle to $X$ in $Z$.
\item[(M2)] Twisted membranes wrap the real line times an embedded curve $\Sigma \subset Z$:
\beqn
\Sigma \times \R \subset Z \times \R
\eeqn
with the property that $K_\Sigma \simeq \wedge^4 N_{\Sigma \subset Z}$.
\end{itemize}



Some special cases of such twisted geometries are worth addressing.

\begin{itemize}
\item[(1)] First, consider the following local fivefold geometry
\beqn
Z = \op{Tot} \begin{pmatrix} W \\ \downarrow \\ X \end{pmatrix} 
\eeqn
where $W$ is a rank two vector bundle on a threefold $X$, which satisfies $\wedge^2 W = K_X$.
The case $X = \C^3$ and $W = K^{1/2}_{\C^3} \otimes \C^2$.
The worldvolume theory on $\C^3$ in this geometry is the (minimal) holomorphic twist of six-dimensional $\cN=(2,0)$ supersymmetry.
More generally, if a threefold $X$ admits a square root of its canonical bundle $K^{1/2}_X$, then we can consider the geometry 
\beqn
Z = \op{Tot} \begin{pmatrix} K^{1/2}_X \otimes \C^2 \\ \downarrow \\ X \end{pmatrix} ,
\eeqn
with fivebranes wrapping the zero section $X \subset Z$.
\item[(2)]
Notice that in the above example there is a global $SU(2)$ symmetry.
We can break this to $U(1)$ by assuming that $W = L_1 \oplus L_2$ where $L_1,L_2$ are line bundles on $X$ satisfying $L_1 \otimes L_2 = K_X$.
Note that necessarily we have $L_2 = L^{-1} \otimes K_X$, $L=L_1$.
\item[(3)] Suppose that $Y$ is a Calabi--Yau threefold with an embedded holomorphic curve $C$, and $S$ is a Calabi--Yau surface.
Then $Z = Y \times S$ determines a twisted $M$-theory geometry with fivebranes that wrap 
\beqn
X = C \times S \subset Y \times S \times \R = Z \times \R .
\eeqn
\item[(4)]
As a special case of (3), consider the Calabi--Yau threefold
\beqn
Y = \op{Tot} \begin{pmatrix} \C \oplus \T^*_C \\ \downarrow \\ C \end{pmatrix} \simeq \C \times \T^* C ,
\eeqn
where $C$ is an arbitrary Riemann surface.
To phrase this as a special case of (2) one takes $W$ to be the rank two bundle obtained by restricting $\C \oplus \T^*_C$ along $C \times S \to C$.
In this geometry, one can allow for $C$ to have punctures.
The four-dimensional theory one obtains by compactifying the six-dimensional worldvolume theory of fivebranes along
\beqn
X = C \times S \to S 
\eeqn
in this geometry gives rise to the holomorphic twist of four-dimensional theories of class $\cS$ \cite{GaiottoS,GMN}.
\item[(5)] 
Again, as a special case of (3) (but more general than the last example), consider the Calabi--Yau threefold
\beqn
Y = \op{Tot} \begin{pmatrix} V \\ \downarrow \\ C \end{pmatrix} ,
\eeqn
where $V$ is an $SU(2)$ bundle over $C$.
Compactification along $X = C \times S \to S$ gives rise to holomorphic twists of four-dimensional superconformal theories \cite{BeemM5}.
This fits as a special case of (1) by taking $W$ to be the restriction of the bundle $V$ along the projection $X \to C$.
\end{itemize}


\section{Twisted fivebranes}

We will work in the generality of the geometry in (1) where the fivefold is of the form $\op{Tot}(W \to X)$ with $\wedge^2 W = K_X$.

The theory of a twisted fivebrane in this geometry admits the following description.
In the Poisson BV formalism its sheaf of fields is of the form $\Omega^{0,\bu}(X,\bV)$ where $\bV$ is the following complex of holomorphic super vector bundles
\beqn\label{eqn:fivebranes}
\bV \;\;\;\left\{
\begin{tikzcd}
& \ul{-1} & \ul{0} \\
\ul{\text{even}} & \Omega^2_X \ar[r,"\del"] & \Omega^3_X \\
\ul{\text{odd}} & \Pi W  
\end{tikzcd} \right.
\eeqn
We will write Dolbeault valued fields of the bundle in the first line with the letter as~$\beta$ and in the bottom line as $\phi$. 
We emphasize that there is both a $\Z/2$ and a $\Z$ grading; for example, the $(0,1)$ component of the field $\phi$ is of odd parity and of cohomological degree zero.


The Poisson BV structure is determined by\brian{finish}

\section{Five-dimensional Yang--Mills theory}

In this section we describe the twist of $\cN=2$ supersymmetric Yang--Mills theory which is relevant to our analysis of holomorphic fivebranes.
We then describe how the compactification of Yang--Mills theory along a circle leads to ...

\subsection{The twisted phase space}

Let $S$ be a complex surface and $W$ a rank two vector bundle on $S$ satisfying $\wedge^2 W = K_S$.
For now, we assume that $S$ is projective.
Given any Lie algebra~$\lie{g}$ equipped with a non-degenerate invariant pairing, we consider the following $\Z \times \Z/2$ graded five-dimensional gauge theory which we refer refer to as twisted five-dimensional $\cN=2$ Yang--Mills theory.
Here, $\Z$ is the `cohomological' grading and $\Z/2$ is parity.

In the perturbative BV formalism, the fields are:
\begin{itemize}
\item Hybrid $BF$ theory with fields
\begin{align*}
A & \in \Omega^{\bu}(\R) \otimes \Omega^{0,\bu}(S) \otimes \lie{g} [1] \\
B & \in \Omega^{\bu}(\R) \otimes \Omega^{2,\bu}(S)\otimes \lie{g} [1] .
\end{align*}
Note that these are the fields present in the twist of pure five-dimensional $\cN=1$ gauge theory on $\R \times S$ \cite{ESW}.
\item Hypermultiplet fields
\beqn
\varphi \in \Omega^\bu(\R) \otimes \Omega^{0,\bu}(S,\Pi W) \otimes \lie{g} [1] .
\eeqn
In particular, the $(0;0,1)$ component of $\varphi$ is an odd field of cohomological degree zero in this description.
\end{itemize}
The action functional, which is obtained as the twist of the physical $\cN=2$ supersymmetric action takes the simple form
\beqn
\int_{\R \times S} (B , F_A) + \frac12 \int_{\R \times S} (\varphi, \d_A \varphi) .
\eeqn
Note that $\d_A$ acting on a $k$-form in $\Omega^\bu(\R) \otimes \Omega^{0,\bu}(S,\Pi W) \otimes \lie{g} [1]$ will produce a $(k+1)$-form in
\beqn
\Omega^\bu(\R) \otimes \Omega^{0,\bu}(S,\Pi W) \otimes \lie{g} [1] \oplus \Omega^\bu(\R) \otimes \Omega^{1,\bu}(S,\Pi W) \otimes \lie{g} [1] .
\eeqn
Only components of $\d_A \varphi$ which live in the first summand appear in the action functional above.
Similarly, only the components of $F_A$ which live in $\Omega^\bu(\R) \otimes \Omega^{0,\bu}(S) \otimes \lie{g}[1]$ appear in the first term of the action.
This is all to say that no \textit{holomorphic} derivatives along $S$ appear in the action functional above.

Alternatively, this theory is hybrid Chern--Simons theory associated to the cyclic, holomorphic, Lie superalgebra
\beqn
\lie{g} \otimes \wedge^\bu W ,
\eeqn
where $\wedge^0, \wedge^2$ are even and $\wedge^1$ is odd.
This Lie superalgebra lives entirely in cohomological degree zero.
Here, we view $\wedge^\bu W$ as a bundle of commutative super algebras.
Explicitly, the nonzero brackets $[-,-]$ in this Lie superalgebra is of the form
\begin{align*}
\lie{g} \otimes \wedge^0 W \times \lie{g} \otimes \wedge^0 W & \to \lie{g} \otimes \wedge^0 W  \\
\lie{g} \otimes \wedge^0 W \times \lie{g} \otimes \wedge^1 W & \to \lie{g} \otimes \wedge^1 W \\
\lie{g} \otimes \wedge^1 W \times \lie{g} \otimes \wedge^1 W & \to \lie{g} \otimes \wedge^2 W .
\end{align*}

For a fixed time slice, the perturbative phase space on $S$ is the $\Z \times \Z/2$ even symplectic space
\beqn
\T^* \left( \lie{g} \otimes \Omega^{0,\bu}(S)[1] \right) \oplus \Pi \left(\lie{g} \otimes \Omega^{0,\bu}(S,W)[1] \right) .
\eeqn
A global moduli theoretic interpretation of this is as follows.
%Consider the moduli stack $\op{Bun}_G(S)$ of holomorphic $G$-bundles on $S$.
%Let $\T^*_{\op{Bun}_G(S)}$ be its cotangent complex.
Consider the moduli of triples
\beqn\label{eqn:pertphase1}
(P, B, \varphi)
\eeqn
where $P$ is a holomorphic $G$-bundle on $S$, $\eta \in \Omega^{2,1}(S, \op{ad} P)$, and $\varphi \in \Omega^{0,1}(S,\op{ad}(P) \otimes W)$ which satisfy 
\begin{align*}
\dbar \varphi = 0 \\
\dbar B + \frac12 [\varphi,\varphi] = 0 .
\end{align*}
The perturbative phase space in \eqref{eqn:pertphase1} is the tangent complex to this moduli stack at the trivial $G$-bundle (and $B = \varphi = 0$).

Let us now assume we have a polarization of $W = L_1 \oplus L_2$ into a sum of two line bundles on $S$ satisfying $L_1 \otimes L_2 = K_S$.
Without loss of generality we set $L = L_1$ and $L_2 = L^{-1} \otimes K_S$.
This allows us to consider a tilted (regraded) perturbative phase space\brian{We should explain $\Pi W[1] \to L_2[2] \oplus L_1$.}
\beqn
\begin{tikzcd}
\ul{-2} & \ul{-1} & \ul{0} \\
& \Omega^{0,\bu}(S, \lie{g}) & \\
\Omega^{2,\bu}(S,\lie{g} \otimes L^{-1}) & & \Omega^{0,\bu}(S, \lie{g} \otimes L) \\
& \Omega^{2,\bu}(S, \lie{g}) & .
\end{tikzcd}
\eeqn
Then, the perturbative phase space of generalized, twisted, five-dimensional $\cN=2$ supersymmetric Yang--Mills theory can be written as the moduli stack
\beqn
\op{Higgs}_G (S,L) .
\eeqn
of quadruples $(P,f, f^*,B)$ where $P$ is a holomorphic $G$-bundle, and
\begin{align*}
\varphi_1 & \in \Gamma(S, \op{ad}(P) \otimes L) \\
\varphi_2 & \in \Omega^{2,2}(S, \op{ad}(P) \otimes L^{-1}) \\
B & \in \Omega^{2,1}(S, \op{ad} P) 
\end{align*}
which satisfy
\begin{align*}
\dbar \varphi_1 = \dbar \varphi_2^* & = 0 \\
\dbar B + [\varphi_1,\varphi_2] & = 0 .
\end{align*}
We can furthermore write this as the cotangent bundle to the moduli stack $\cM_{G}(S,L)$ of pairs $(P,\varphi_1)$ where $P$ is a holomorphic principal $G$-bundle on $S$ and $\varphi_1 \in \Gamma(S, \lie{g} \otimes L)$ satisfying $\dbar \varphi_1 = 0$.

In this example, the Hilbert space of states coming from geometric quantization is functions on the base of this cotangent bundle
\beqn
\cO \left( \cM_{G}(S,L) \right) .
\eeqn 
As a special case, note that when $L = K_S$ we have $\cM_{G}(S,L) \simeq \T[1] \op{Bun}_G(S)$, so that the Hilbert space is the algebra of de Rham forms on the moduli stack of principal $G$-bundles
\beqn
\cO \left( \T[1] \op{Bun}_G(S) \right) \simeq \Omega^{\bu}(\op{Bun}_G(S)) .
\eeqn

\subsection{Circle compactification}

We consider the twisted six-dimensional theory on the threefold
\beqn
X = \C^\times \times S 
\eeqn
where $S$, as above, is a projective surface.
The compactification of this theory along $X \to \R_{>0} \to S$, induced by the radial projection $\C^\times \to \R_{>0}$, yields a five-dimensional theory with an infinite-dimensional space of fields.

Recall that the data which defines the six-dimensional theory on $X$ is a rank two vector bundle $W$ satisfying $\wedge^2 W = K_X$.
We will assume first that $W$ takes the specific form
\beqn
W = \cO_C \boxtimes L_1 \oplus K_C \otimes L_2
\eeqn
where $L_1,L_2$ are line bundles on $S$ with the property that $L_1 \otimes L_2 = K_S$.

The compactification of the twisted free hypermultiplet field is easy to describe.
For this choice of $W$, the fundamental fields are
\begin{align*}
\phi_1 & \in \Omega^{0,\bu}(\C^\times) \hotimes \Omega^{0,\bu}(S, \Pi L_1) [1] \\
\phi_2 \frac{\d z}{2 \pi \im z} & \in \Omega^{1,\bu}(\C^\times) \hotimes \Omega^{0,\bu}(S, \Pi L_2) [1] .
\end{align*}
The six-dimensional action is simply $\int_{C \times S} \frac{\d z}{2 \pi \im z} \phi_1 \dbar \phi_2 $.

We use polar coordinates for $\C^\times$, $(r = |z|, \theta)$.
The six-dimensional fields $\phi_i$ can be written, in these coordinates, as
\beqn
\phi_i(r,\theta) = \sum_n \varphi_{i,n} (r) e^{2 \pi \im n \theta} , 
\eeqn
where we are viewing $\varphi_{i,n} (r) \in \Omega^{\bu}(\R_{>0}) \hotimes \Omega^{0,\bu}(S, \Pi L_i)[1]$ as a form-valued section over $\C^\times \times S$ by restriction along $\C^\times \times S \to \R_{>0} \times S$.

The free action functional for these fields can be written, in five-dimensional language, as
\begin{align*}
\sum_{n,m} \int_{\C^\times \times S} \frac{\d z}{2 \pi \im z} \varphi_{1,n} \dbar \varphi_{2,m} & =  ...\\ & = \sum_{n} \int_{\R_{>0} \times S} \left(\varphi_{1,n} \d \varphi_{2,-n} + n \varphi_{1,n} \varphi_{2,-n}\frac{\d r}{r} \right) .
\end{align*}

We can formally view the collections $\{\varphi_{i,n}\}_{n \in \Z}$ as Laurent series-valued fields
\beqn
\varphi_{i}(t) = \sum_{n} \varphi_{i,n} t^n \in \Omega^{\bu}(\R) \hotimes \Omega^{0,\bu}(S, \Pi L_i) [1] \otimes \C((t)) .
\eeqn
The action then takes the simple form
\beqn
\int_{\R \times S} \op{Res}_t \frac{\d t}{t} \left( \varphi_1 (t) \d \varphi_2 (t) + \varphi_1 (t) t \frac{\del}{\del t} \varphi_2 (t) \frac{\d r}{r}  \right) .
\eeqn

\subsection{Chern--Simons deformation}

Let $\omega \in \Omega^{1,1}(S)$ be a $(1,1)$-form on $S$ with the property that $\dbar \omega = 0$ and $\del \omega = 0$.
In other words, $\omega$ defines an element in $H^1(S, \Omega^{1,cl})$.
Then, consider the deformation of generalized twisted five-dimensional $\cN=2$ Yang--Mills theory defined by
\beqn
S_\theta = \frac12 \int_{\R \times S} \omega \wedge (A, \del A) .
\eeqn
Locally, we can find a primitive $\omega = \del \theta$ and this deformation takes the more symmetric form $\frac12 \int \theta \wedge (\del A, \del A)$.

Turning on this deformation leads to the deformed perturbative phase space
\beqn
\begin{tikzcd}
\ul{-2} & \ul{-1} & \ul{0} \\
& \Omega^{0,\bu}(S, \lie{g}) \ar[dd,dotted,"\omega \wedge \del"] & \\
\Omega^{2,\bu}(S,\lie{g} \otimes L^{-1}) & & \Omega^{0,\bu}(S, \lie{g} \otimes L) \\
& \Omega^{2,\bu}(S, \lie{g}) & .
\end{tikzcd}
\eeqn
A more invariant way to construct this perturbative phase space is the following.
The non-degenerate invariant pairing determines a four-form $\kappa$ on $BG$.
This pulls back, along the evaluation map
\beqn
\op{ev} \colon S \times \op{Bun}_G(S) \to BG 
\eeqn
to a four-form on $S \times \op{Bun}_G(S)$.
Integration of the six-form $\omega \wedge \op{ev}^*\kappa$ along $S$ yields a two-form $\omega_{\kappa} = \int_S \omega \wedge \op{ev}^*\kappa$ on $\op{Bun}_G(S)$.
We can, in turn, further restrict this to a two-form on $\cM_{G}(S,L)$ that we will abusively still denote by $\omega_{\kappa}$.
The phase space is then the twisted cotangent bundle

\beqn\label{eqn:5dphase}
\T_{\omega_{\kappa}}^* \left(\cM_{G}(S,L) \right) .
\eeqn

\section{Vertex algebras from six dimensions}

\subsection{Twisted compactification}

We will start with the twist of the six-dimensional theory on a threefold of the form 
\beqn
X = C \times S ,
\eeqn
where $S$ is projective.
Recall that the input data for the twisted fivebrane theory is a rank two holomorphic vector bundle $W$ on $X$ satisfying $\wedge^2 W$.
We consider two cases:
\begin{itemize}
\item[(1)] $W = \cO_C \boxtimes L_1 \oplus K_C \otimes L_2$ where $L_1,L_2$ are line bundles on $S$ with the property that $L_1 \otimes L_2 = K_S$.
\item[(2)] $W = K^{1/2}_C \boxtimes V$ where $V$ is a rank two vector bundle on $S$ which satisfies $\wedge^2 V = K_S$. Note that $W$ depends on the choice of a spin structure on $C$.
\end{itemize}

The compactification of the twisted six-dimensional theory along $S$ will yield a two-dimensional chiral conformal field theory on $C$.
In each of the cases we can describe the CFT explicitly.
The associated vertex algebras are vacuum modules for the associative algebra of modes.
Furthermore, local operators are identified, explicitly, with the Hilbert space of the five-dimensional theory obtained by the twisted circle compactification $\C^\times \to \R_{>0}$ described in section \ref{s:circle1}.

\subsection{Nakajima, Grojnowksi, and the Heisenberg vertex algebra}

We start with a familiar choice of bundle data to define the twisted six-dimensional theory.
Let $S$ be a projective surface as before and suppose
\beqn
W = \cO_C \boxtimes K_S \oplus K_C \boxtimes \cO_S .
\eeqn
and further regrade using the fact that $SU(2)$ has been broken to $U(1)$. \brian{explain this}
This six-dimensional theory is of the Dolbeault complex of the $\Z$-graded complex of holomorphic vector bundles
\beqn
\begin{tikzcd}
\ul{-2} & \ul{-1} & \ul{0} \\
& \Omega^2_{S \times C} \ar[r,"\del"] & \Omega^3_{C \times S} \\
\Omega^1_C \boxtimes \cO_S & & \cO_C \boxtimes \Omega^2_S .
\end{tikzcd} 
\eeqn
The BV quantization of this theory yields a factorization algebra on $S \times C$ that we will denote $\Obs_{S}^{NG}$.
The compactification, or factorization homology, along $S$
\beqn
\int_S \Obs^{NG}_{S}
\eeqn
is a holomorphic factorization algebra on $C$.
On $C = \C$ this determines a vertex algebra that we denote $\VV_{S}^{NG}$.

To describe this vertex algebra, we introduce some familiar notations.
Suppose that~$V$ is a graded vector space equipped with a graded symmetric pairing of total degree zero.
Let $\{e_i\}$ be a basis for~$V$.
There is a vertex algebra $\CB[V]$, called the `chiral boson' vertex algebra, generated by fields $e_i(z)$ of spin one and with OPEs
\beqn
e_i(z) e_j(w) \simeq \frac{(e_i,e_j)}{(z-w)^2} .
\eeqn
This is a graded vertex algebra with grading inherited from that of $V$.
Notice when $V = \C$ with the obvious pairing this is simply the chiral boson vertex algebra.
Of course, this vertex algebra does not depend on the chosen basis.

Next, we introduce the $\beta\gamma$ vertex algebra.
If $V$ is any graded vector space with basis $\{e_i\}$ then $\beta\gamma[V]$ is the vertex algebra generated by fields $\gamma_i(z)$ of spin zero and $\beta^j(z)$ of spin $1$ with OPEs
\beqn
\beta^j(z) \gamma_i(w) \simeq \frac{\delta_{i}^j}{z-w} .
\eeqn
This vertex algebra is also independent of the chosen basis.

\begin{prop}
The vertex algebra $\VV_{S}^{NG}$ is quasi-isomorphic to
\beqn
\CB[H^\bu(S,\Omega^1_S)] \otimes \beta\gamma[H^\bu(S,K)] \otimes T
\eeqn 
where $T$ is the vector space
\beqn
\Sym(H^\bu(S,K_S)[1])
\eeqn
equipped with the trivial vertex algebra structure.
The character of this vertex algebra (whose fugacities we explain in the proof) recovers G\"ottsche's formula for the Hodge-Poincar\'e polynomial of the Hilbert schemes of $S$:
\beqn
\chi(\VV_S^{NG}) = \sum_{n \geq 0}\sum_{i,j=0}^{2n} q^n h^{i,j}(S^{[n]}) (-x)^i (-y)^j = \prod_{m \geq 1} \prod_{k,\ell =0}^2 \left(1 - (-1)^{k+\ell} x^{k+m-1} y^{\ell + m - 1} q^m \right)^{(-1)^{k+\ell+1} h^{k,\ell}(S)} .
\eeqn
\end{prop}
\begin{proof}
Let $\pi \colon C \times S \to C$ be the projection.
Notice that since $S$ is projective there is a quasi-isomorphism of sheaves on $C$:
\begin{align*}
\pi_*\Omega^{2,\bu}_{C \times S} & \simeq  \Omega^{0,\bu}_C \otimes H^{\bu} (S, K_S)  \oplus \Omega^{1,\bu}_C \otimes H^\bu(S, \Omega^1_S)  \\
\pi_* \Omega^{3,\bu}_{C \times S} & \simeq \Omega^{1,\bu}_C \otimes H^{\bu} (S, K_S) \\
\pi_* \Omega^{0,\bu}_{C} \otimes \Omega^{2,\bu}_S & \simeq \Omega^{0,\bu}_C \otimes H^\bu(S,K_S) \\
\pi_* \Omega^{1,\bu}_{C} \otimes \Omega^{0,\bu}_S & \simeq \Omega^{1,\bu}_C \otimes H^\bu(S,\cO_S) .
\end{align*}

Let $\til q$ be the fugacity associated to the conformal weight operator $L_0$.
Let $u$ be the fugacity for cohomological degree.
Let $v$ be the `flavor' fugacity, where $\gamma$ is of weight one and $\beta$ is opposite.
Finally, let $w$ be a fugacity corresponding to the rather trivial symmetry which scales the generators of the commutative part of the vertex algebra with weight one.

Then the single particle characters of each of the fields is computed directly as follows (here $h^{k,\ell} = h^{k,\ell}(S)$).
\begin{itemize}
\item The $\gamma$ field in the $\beta\gamma$ system valued contributes
\beqn
\frac{v}{1-\til q} \left(h^{2,0} - u h^{2,1} + u^2 h^{2,2}\right) .
\eeqn
\item The $\beta$ field in the $\beta\gamma$ system valued contributes
\beqn
\frac{v^{-1} \til q}{1- \til q} \left(u^{-2} h^{0,0} - u^{-1} h^{0,1} + h^{0,2}\right) .
\eeqn
\item The chiral boson field contributes
\beqn
\frac{\til q}{1-\til q} \left(-u^{-1} h^{1,0} + h^{1,1} - u h^{1,2} \right) .
\eeqn
\item The commutative part of the vertex algebra contributes
\beqn
w \left(-u^{-1} h^{2,0} + h^{2,1} - u h^{2,2}\right) .
\eeqn
\end{itemize}
To recover G\"ottsche's formula we specialize
\beqn
\til q = xy q, \quad u = y, \quad v = x y^{-1}, \quad w = x .
\eeqn
\end{proof}

\brian{I think the following is true by some boson-boson correspondence}
\begin{prop}
There is an isomorphism of vertex algebras
\beqn
\VV_{S}^{NG} \simeq \CB\left[H_{dR}^\bu(S)\right] .
\eeqn
\end{prop}

This vertex algebra was associated to a specific choice of rank two vector bundle \eqref{eqn:ng1}.
More generally, we can consider the following vector bundle
\beqn
W = \cO_C \boxtimes L \oplus K_C \boxtimes L^!
\eeqn
where $L^! = L^* \otimes K_S$.
We will denote the associated vertex algebra by $\VV_{S,L}^{NG}$.
In the same way as we just did for $L = K_S$, we arrive at the following description.

\begin{prop}
The vertex algebra $\VV_{S,L}^{NG}$ is quasi-isomorphic to
\beqn
\CB[H^\bu(S,\Omega^1_S)] \otimes \beta\gamma[H^\bu(S,L)] \otimes T
\eeqn 
where $T$ is the vector space
\beqn
\Sym(H^\bu(S,K_S)[1])
\eeqn
equipped with the trivial vertex algebra structure.

The character of this vertex algebra is
\begin{multline}
\chi(\VV_{S,L}^{NG}) = \frac{(1-(xy^{-1})^{h^{2,0}}) (1- (xy)^{h^{2,2}})}{1-x^{h^{2,1}}} \prod_{m \geq 1} \prod_{\ell =0}^2 \left(1 - (-1)^{\ell+1} x^{m} y^{\ell + m - 1} q^m \right)^{(-1)^{\ell} h^{1,\ell}(S)} \\
\times \left[(1-x^{m+1}y^{m+\ell-1} q^m)(1-x^m y^{m-\ell+2} q^{m+1})\right]^{(-1)^{\ell+1} h_L^\ell} .
\end{multline}
\end{prop}
\begin{proof}
We adjust the computation of the character to this case, the remainder of the proof is a straightforward generalization.
The only bit that needs modification is the character of the $\beta\gamma$ system.
In this case, the $\beta\gamma$ system is based on the graded vector space $H^\bu(S,L)$.
Let $h^{i}_L$ denote the dimensions of this cohomology.
Then:
\begin{itemize}
\item The $\gamma$ field contributes
\beqn
\frac{v}{1-\til q} (h^{0}_L - u h^{1}_L + u^2 h^{2}_L) .
\eeqn
\item The $\beta$ field contributes
\beqn
\frac{v^{-1} \til q}{1 - \til q} (u^{-2} h^{2}L - u^{-1} h^1_L + h^0_L) .
\eeqn
\end{itemize}
\end{proof}



\subsection{A spin version}

Assume that $S$ is a projective surface equipped with a rank two vector bundle $V$ satisfying $\wedge^2 V = K_S$.
Consider the resulting twisted six-dimensional theory on $X = S \times C$ where 
\beqn
W = K_C^{1/2} \boxtimes V.
\eeqn
Here, we are assuming that $C$ is equipped with a spin structure.
In the Poisson BV formalism its sheaf of fields is of the form $\Omega^{0,\bu}(S \times C,\bV)$ where $\bV$ is the following super complex of holomorphic vector bundles
\beqn
\bV \colon \;\;\;\left\{
\begin{tikzcd}
& \ul{-1} & \ul{0} \\
\ul{\text{even}} & \Omega^2_{C \times S} \ar[r,"\del"] & \Omega^3_{C \times S} \\
\ul{\text{odd}} & \Pi p^* V  
\end{tikzcd} \right.
\eeqn
We will write Dolbeault valued fields of the bundle in the first line with the letter as~$\beta$ and in the bottom line as $\phi$, as we did above.
The BV quantization of this theory yields a factorization algebra on $S \times C$ that we will denote $\Obs_{S,V}$.
Its compactification, or factorization homology, along $S$
\beqn
\int_S \Obs_{S,V} 
\eeqn
is a holomorphic factorization algebra on $C$.
On $C = \C$ this determines a vertex algebra that we denote $\VV_{S,V}$.

We describe this vertex algebra explicitly.
Suppose that $U$ is any $\Z \times \Z/2$ vector space equipped with a graded totally anti-symmetric pairing $\<-,-\>$ of total degree zero.
Then we can consider the symplectic boson vertex algebra valued in $U$ that we denote $\SB[U]$.
If $\{f_i\}$ is a basis for $U$ then the vertex algebra is generated by free fields $f_i(z)$ of spin $1/2$ and with OPEs
\beqn
f_i(z) f_j(w) \simeq \frac{\<f_i,f_j\>}{z-w} .
\eeqn
This vertex algebra inherits a $\Z \times \Z/2$ grading from $U$.

Notice that since $S$ is projective there is a quasi-isomorphism of sheaves on $C$:
\begin{align*}
\pi_*\Omega^{2,\bu}_{C \times S} & \simeq  \Omega^{0,\bu}_C \otimes H^{\bu} (S, K_S)  \oplus \Omega^{1,\bu}_C \otimes H^\bu(S, \Omega^1_S)  \\
\pi_* \Omega^{3,\bu}_{C \times S} & \simeq \Omega^{1,\bu}_C \otimes H^{\bu} (S, K_S) \\
\pi_* \Omega^{0,\bu}_{C \times S}( K_{C}^{1/2}\boxtimes V) & \simeq \Omega^{0,\bu}_C(K^{1/2}_C) \otimes H^\bu(S, V) .
\end{align*}


\begin{prop}
The vertex algebra $\VV_{S,V}$ is quasi-isomorphic to the vertex algebra
\beqn
\CB[H^\bu(S,\Omega^1_S)] \otimes \SB\left[\Pi H^\bu(S,V)[1]\right] \otimes T
\eeqn 
where $T$ is the vector space
\beqn
\Sym(H^\bu(S,K_S)[1])
\eeqn
equipped with the trivial vertex algebra structure.
\end{prop}

\section{AGT statements}

Let $S$ be a projective surface and $L_1,L_2,L_3$ line bundles on $S$ satisfying 
\beqn
L_1 \otimes L_2 \otimes L_3 = K_S .
\eeqn
We consider a special case of the geometry (2) where the threefold that twisted fivebranes $X$ is of the form
\beqn
X = \op{Tot} \begin{pmatrix} L_3 \\ \downarrow \\ S \end{pmatrix},
\eeqn
and where $W = L_1 \oplus L_2$ (or rather, the restrictions of these line bundles to $X$).
In this case, the local Calabi--Yau fivefold geometry is $Z = \op{Tot}(L_1 \oplus L_2 \oplus L_3).$

We will consider a variant of the factorization algebra $\Obs_{S, W}$ in the case where $L_3$ is possibly a nontrivial line bundle. Indeed, BV quantization of the fields \eqref{eqn:fivebrane} of a twisted fivebrane on $X$ yields a holomorphic factorization algebra $\Obs_{X, W}$ that specializes to $\Obs_{S, W}$ in the case where $L_3$ is the trivial line bundle on $S$ \surya{introduce earlier}

From the factorization algebra $\Obs_{X, W}$, we will produce a pair of an associative algebra and a module. To do so, we choose a Hermitian metric on $L_3$. Doing so gives us a norm map

\beqn
X\to X\times_S X \to S\times \C
\eeqn

which factors through the trivial $\R_{\geq 0}$-bundle on $S$ since a Hermitian metric is in particular positive definite. Let $p$ denote the composition of this map with the projection onto $\R_{\geq 0}$. 

The geometry of the situation can be summarized in the following diagram

\[
\begin{tikzcd}
X\setminus S \ar[d,"\mathring p"'] \ar[r,hook] & X \ar[d,"p"] & \ar[l,hook',"0"'] S \ar[d] \\
\R_{>0} \ar[r,hook] & \R_{\geq 0} & \ar[l,hook'] \{0\}.
\end{tikzcd}
\]

The pushforward
\beqn
p_* \Obs_{X,W}
\eeqn

is therefore a factorization algebra on $\R_{\geq 0}$. 

We may extract from this data, an associative algebra and a module as follows. 

\begin{itemize}
\item Let $(a,b)\subset \R_{\geq 0}$. The sections $(p_* \Obs_{X, W})(a,b)$ are an associative algebra, which we will denote as $\AA(X, W)$. 

This associative algebra can equivalently be computed by restricting away from $S$, and pushing forward along $\mathring p$. The fiber of this map at $r \in \R_{>0}$ is a copy of $S_r(L_3)\subset X$, the total space of the radius $r$ circle bundle in $L_3$. Notice that in the case when $L_3$ is trivial, this associative algebra is exactly the modes algebra of the vertex algebra $\VV_{S,W}$. 

\item The costalk at zero $(p_*\Obs{X,W})_0$ is a module, which we denote as $\MM (X, W)$. 
\end{itemize}

The AGT relation asserts that $\MM(X,W)$ may equivalently be computed as the geometric quantization of \eqref{eqn:5dphase}

\end{document}
