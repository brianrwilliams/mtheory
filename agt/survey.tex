\documentclass[11pt]{amsart}

\usepackage{macros-master,amsaddr,physics,amssymb}

\addbibresource{refs}

\renewcommand{\div}{\del_{\Omega}}
\renewcommand{\PV}{\mathrm{PV}}
\renewcommand{\op}{\operatorname}

\setcounter{tocdepth}{1}

\begin{document}
\title{Notes on agt}
\maketitle


\section{The twisted brane construction}

We will use the fact that the minimal twist of $M$-theory exists on geometries of the form
\beqn
Z \times \R
\eeqn
where $Z$ is a Calabi--Yau fivefold.

We consider familiar extended objects, which in the twisted setting have the following flavor.
\begin{itemize}
\item[(M5)] Twisted fivebranes wrap an embedded threefold
\beqn
X \subset Z
\eeqn
with the property that $K_X \simeq \wedge^2 N_{X \subset Z}$ where $N_{X \subset Z}$ is the normal bundle to $X$ in $Z$.
\item[(M2)] Twisted membranes wrap the real line times an embedded curve $\Sigma \subset Z$:
\beqn
\Sigma \times \R \subset Z \times \R
\eeqn
with the property that $K_\Sigma \simeq \wedge^4 N_{\Sigma \subset Z}$.
\end{itemize}



Some special cases of such twisted geometries are worth addressing.

\begin{itemize}
\item[(1)] First, consider the following local fivefold geometry
\beqn
Z = \op{Tot} \begin{pmatrix} W \\ \downarrow \\ X \end{pmatrix} 
\eeqn
where $W$ is a rank two vector bundle on a threefold $X$, which satisfies $\wedge^2 W = K_X$.
The case $X = \C^3$ and $W = K^{1/2}_{\C^3} \otimes \C^2$.
The worldvolume theory on $\C^3$ in this geometry is the (minimal) holomorphic twist of six-dimensional $\cN=(2,0)$ supersymmetry.
More generally, if a threefold $X$ admits a square root of its canonical bundle $K^{1/2}_X$, then we can consider the geometry 
\beqn
Z = \op{Tot} \begin{pmatrix} K^{1/2}_X \otimes \C^2 \\ \downarrow \\ X \end{pmatrix} ,
\eeqn
with fivebranes wrapping the zero section $X \subset Z$.
\item[(2)]
Notice that in the above example there is a global $SU(2)$ symmetry.
We can break this to $U(1)$ by assuming that $W = L_1 \oplus L_2$ where $L_1,L_2$ are line bundles on $X$ satisfying $L_1 \otimes L_2 = K_X$.
Note that necessarily we have $L_2 = L^{-1} \otimes K_X$, $L=L_1$.
\item[(3)] Suppose that $Y$ is a Calabi--Yau threefold with an embedded holomorphic curve $C$, and $S$ is a Calabi--Yau surface.
Then $Z = Y \times S$ determines a twisted $M$-theory geometry with fivebranes that wrap 
\beqn
X = C \times S \subset Y \times S \times \R = Z \times \R .
\eeqn
\item[(4)]
As a special case of (3), consider the Calabi--Yau threefold
\beqn
Y = \op{Tot} \begin{pmatrix} \C \oplus \T^*_C \\ \downarrow \\ C \end{pmatrix} \simeq \C \times \T^* C ,
\eeqn
where $C$ is an arbitrary Riemann surface.
To phrase this as a special case of (2) one takes $W$ to be the rank two bundle obtained by restricting $\C \oplus \T^*_C$ along $C \times S \to C$.
In this geometry, one can allow for $C$ to have punctures.
The four-dimensional theory one obtains by compactifying the six-dimensional worldvolume theory of fivebranes along
\beqn
X = C \times S \to S 
\eeqn
in this geometry gives rise to the holomorphic twist of four-dimensional theories of class $\cS$ \cite{GaiottoS,GMN}.
\item[(5)] 
Again, as a special case of (3) (but more general than the last example), consider the Calabi--Yau threefold
\beqn
Y = \op{Tot} \begin{pmatrix} V \\ \downarrow \\ C \end{pmatrix} ,
\eeqn
where $V$ is an $SU(2)$ bundle over $C$.
Compactification along $X = C \times S \to S$ gives rise to holomorphic twists of four-dimensional superconformal theories \cite{BeemM5}.
This fits as a special case of (1) by taking $W$ to be the restriction of the bundle $V$ along the projection $X \to C$.
\end{itemize}


\section{Twisted fivebranes}

We will work in the generality of the geometry in (1) where the fivefold is of the form $\op{Tot}(W \to X)$ with $\wedge^2 W = K_X$.

The theory of a twisted fivebrane in this geometry admits the following description.
In the Poisson BV formalism its sheaf of fields is of the form $\Omega^{0,\bu}(X,\bV)$ where $\bV$ is the following super complex of holomorphic vector bundles
\beqn
\bV \;\;\;\left\{
\begin{tikzcd}
& \ul{-1} & \ul{0} \\
\ul{\text{even}} & \Omega^2_X \ar[r,"\del"] & \Omega^3_X \\
\ul{\text{odd}} & \Pi W  
\end{tikzcd} \right.
\eeqn
We will write Dolbeault valued fields of the bundle in the first line with the letter as~$\beta$ and in the bottom line as $\phi$. 

The Poisson BV structure is determined by\brian{finish}

\subsection{Five-dimensional supersymmetric Yang--Mills theory}

Let $S$ be a complex surface and $W$ a rank two vector bundle on $S$ satisfying $\wedge^2 W = K_S$.
Given any Lie algebra~$\lie{g}$ equipped with a non-degenerate invariant pairing, we consider the following $\Z \times \Z/2$ graded five-dimensional gauge theory which we refer to as generalized twisted five-dimensional $\cN=2$ Yang--Mills theory.

The fields are
\begin{itemize}
\item Hybrid $BF$ theory with fields
\begin{align*}
A & \in \Omega^{\bu}(\R) \otimes \Omega^{0,\bu}(S) \otimes \lie{g} [1] \\
B & \in \Omega^{\bu}(\R) \otimes \Omega^{2,\bu}(S)\otimes \lie{g} [1] .
\end{align*}
\item Hypermultiplet fields
\beqn
\phi \in \Omega^\bu(\R) \otimes \Omega^{0,\bu}(S,\Pi W) \otimes \lie{g} [1] .
\eeqn
In other words, $\phi^{0,0}$ is an odd field of cohomological degree $-1$.
\end{itemize}
The action functional is
\beqn
\int_{\R \times S} (B , F_A) + \frac12 \int_{\R \times S} (\phi, \dbar_A \phi) .
\eeqn

Alternatively, this theory is hybrid Chern--Simons theory associated to the cyclic, holomorphic, local Lie superalgebra
\beqn
\lie{g} \otimes \wedge^\bu W ,
\eeqn
where $\wedge^0, \wedge^2$ are even and $\wedge^1$ is odd.
Here, we view $\wedge^\bu W$ as a bundle of commutative super algebras.
Explicitly, the nonzero brackets $[-,-]$ in this Lie superalgebra is of the form
\begin{align*}
\lie{g} \otimes \wedge^0 W \times \lie{g} \otimes \wedge^0 W & \to \lie{g} \otimes \wedge^0 W  \\
\lie{g} \otimes \wedge^0 W \times \lie{g} \otimes \wedge^1 W & \to \lie{g} \otimes \wedge^1 W \\
\lie{g} \otimes \wedge^1 W \times \lie{g} \otimes \wedge^1 W & \to \lie{g} \otimes \wedge^2 W .
\end{align*}

The perturbative phase space on $S$ is the $\Z \times \Z/2$ even symplectic space
\beqn
\T^* \left( \lie{g} \otimes \Omega^{0,\bu}(S)[1] \right) \oplus \Pi \left(\lie{g} \otimes \Omega^{0,\bu}(S,W)[1] \right) .
\eeqn
A moduli theoretic interpretation of this is as follows.
%Consider the moduli stack $\op{Bun}_G(S)$ of holomorphic $G$-bundles on $S$.
%Let $\T^*_{\op{Bun}_G(S)}$ be its cotangent complex.
Consider the moduli of triples
\beqn\label{eqn:pertphase1}
(P, B, \phi)
\eeqn
where $P$ is a holomorphic $G$-bundle on $S$, $\eta \in \Omega^{2,1}(S, \op{ad} P)$, and $\phi \in \Omega^{0,1}(S,\op{ad}(P) \otimes W)$ which satisfy 
\begin{align*}
\dbar \phi = 0 \\
\dbar B + \frac12 [\phi,\phi] = 0 .
\end{align*}
The perturbative phase space in \eqref{eqn:pertphase1} is the tangent complex to this moduli stack at the trivial $G$-bundle (and $B = \phi = 0$).

Let us now assume we have a polarization of $W = L_1 \oplus L_2$ into a sum of two line bundles on $S$ satisfying $L_1 \otimes L_2 = K_S$.
Without loss of generality we set $L = L_1$ and $L_2 = L^{-1} \otimes K_S$.
This allows us to consider a tilted (regraded) perturbative phase space\footnote{We should explain $\Pi W[1] \to L_2[2] \oplus L_1$.}
\beqn
\begin{tikzcd}
\ul{-2} & \ul{-1} & \ul{0} \\
& \Omega^{0,\bu}(S, \lie{g}) & \\
\Omega^{2,\bu}(S,\lie{g} \otimes L^{-1}) & & \Omega^{0,\bu}(S, \lie{g} \otimes L) \\
& \Omega^{2,\bu}(S, \lie{g}) & .
\end{tikzcd}
\eeqn
Then, the perturbative phase space of generalized, twisted, five-dimensional $\cN=2$ supersymmetric Yang--Mills theory can be written as the moduli stack
\beqn
\op{Higgs}_G (S,L) .
\eeqn
of quadruples $(P,\varphi, \varphi^*,B)$ where $P$ is a holomorphic $G$-bundle, and
\begin{align*}
\varphi & \in \Gamma(S, \op{ad}(P) \otimes L) \\
\varphi^* & \in \Omega^{2,2}(S, \op{ad}(P) \otimes L^{-1}) \\
B & \in \Omega^{2,1}(S, \op{ad} P) 
\end{align*}
which satisfy
\begin{align*}
\dbar \varphi = \dbar \varphi^* & = 0 \\
\dbar B + [\varphi, \varphi^*] & = 0 .
\end{align*}
We can furthermore write this as the cotangent bundle to the moduli stack $\cM_{G}(S,L)$ of pairs $(P,\varphi)$ where $P$ is a holomorphic principal $G$-bundle on $S$ and $\varphi \in \Gamma(S, \lie{g} \otimes L)$ satisfying $\dbar \varphi = 0$.

In this example, the Hilbert space of states coming from geometric quantization is functions on the base of this cotangent bundle
\beqn
\cO \left( \cM_{G}(S,L) \right) .
\eeqn 
Notice that when $L = K_S$ we have the Hilbert space
\beqn
\cO \left( \T[1] \op{Bun}_G(S) \right) \simeq \Omega^{\bu}(\op{Bun}_G(S)) .
\eeqn

\subsection{Chern--Simons deformation}

Let $\omega \in \Omega^{1,1}(S)$ be a $(1,1)$-form on $S$ with the property that $\dbar \omega = 0$ and $\del \omega = 0$.
In other words, $\omega$ defines an element in $H^1(S, \Omega^{1,cl})$.
Then, consider the deformation of generalized twisted five-dimensional $\cN=2$ Yang--Mills theory defined by
\beqn
S_\theta = \frac12 \int_{\R \times S} \omega \wedge (A, \del A) .
\eeqn
Locally, we can find a primitive $\omega = \del \theta$ and this deformation takes the more symmetric form $\frac12 \int \theta \wedge (\del A, \del A)$.

Turning on this deformation leads to the deformed perturbative phase space
\beqn
\begin{tikzcd}
\ul{-2} & \ul{-1} & \ul{0} \\
& \Omega^{0,\bu}(S, \lie{g}) \ar[dd,dotted,"\omega \wedge \del"] & \\
\Omega^{2,\bu}(S,\lie{g} \otimes L^{-1}) & & \Omega^{0,\bu}(S, \lie{g} \otimes L) \\
& \Omega^{2,\bu}(S, \lie{g}) & .
\end{tikzcd}
\eeqn
A more invariant way to construct this perturbative phase space is the following.
By transgression, we can view the invariant form as a one-form on the moduli stack $\op{Bun}_G(S)$ of cohomological degree $-1$.
We can pull this one-form back to get a one-form on $S \times \op{Bun}_G(S)$ also of cohomological degree $-1$.
Integration of this one-form against $\omega$ along the fibers of $S \times \op{Bun}_G(S) \to \op{Bun}_G(S)$ yields a two-form on $\op{Bun}_G(S)$ (of cohomological degree zero).
We can, in turn, further restrict this to a two-form on $\cM_{G}(S,L)$ that we will abusively still denote by $\omega$.
The phase space is then the twisted cotangent bundle
\beqn
\T_{\omega}^* \left(\cM_{G}(S,L) \right) .
\eeqn

\section{The abelian case}

We will work in the generality of the geometry in (1) where the fivefold is of the form $\op{Tot}(W \to X)$ with $\wedge^2 W = K_X$.

\subsection{Nakajima, Grojnowksi, and the Heisenberg vertex algebra}

Assume that $S$ is a projective surface equipped with a rank two vector bundle $V$ satisfying $\wedge^2 W = K_S$.
Consider the resulting twisted fivebrane worldvolume theory on $X = S \times C$ where $W = V \boxtimes K_C^{1/2}$ (here, we are assuming that $C$ is equipped with a spin structure).
In the Poisson BV formalism its sheaf of fields is of the form $\Omega^{0,\bu}(S \times C,\bV)$ where $\bV$ is the following super complex of holomorphic vector bundles
\beqn
\bV \colon \;\;\;\left\{
\begin{tikzcd}
& \ul{-1} & \ul{0} \\
\ul{\text{even}} & \Omega^2_{S \times C} \ar[r,"\del"] & \Omega^3_{S \times C} \\
\ul{\text{odd}} & \Pi p^* W  
\end{tikzcd} \right.
\eeqn
We will write Dolbeault valued fields of the bundle in the first line with the letter as~$\beta$ and in the bottom line as $\phi$, as we did above.
The BV quantization of this theory yields a factorization algebra on $S \times C$ that we will denote $\Obs_{S,W}$.
Its compactification, or factorization homology, along $S$
\beqn
\int_S \Obs_{S,W} 
\eeqn
is a holomorphic factorization algebra on $C$.
On $C = \C$ this determines a vertex algebra that we denote $\VV_{S,W}$.

We describe this vertex algebra explicitly.
Notice that since $S$ is projective there is a quasi-isomorphism of sheaves on $C$:
\begin{align*}
\Omega^{2,\bu}_{S \times C} & \simeq H^{\bu} (S, K_S) \otimes \Omega^{0,\bu}_C  \oplus H^\bu(S, \Omega^1_S) \otimes \Omega^{1,\bu}_C \\
\Omega^{3,\bu}_{S \times C} & \simeq H^{\bu} (S, K_S) \otimes \Omega^{1,\bu}_C  \\
\Omega^{0,\bu}_{S \times C}(W \boxtimes K_{C}^{1/2}) & \simeq H^\bu(S, W) \otimes \Omega^{0,\bu}_C(K^{1/2}_C) .
\end{align*}

Suppose that $V$ is a graded vector space equipped with a graded symmetric pairing of total degree zero.
Let $\{e_i\}$ be a basis for $V$.
There is a vertex algebra $\cC\cB(V)$ generated by fields $e_i(z)$ of spin one and with OPEs
\beqn
e_i(z) e_j(w) \simeq \frac{(e_i,e_j)}{(z-w)^2} .
\eeqn
This is a graded vertex algebra with grading inherited from that of $V$.
Notice when $V = \C$ with the obvious pairing this is simply the chiral boson vertex algebra.

Similarly, suppose that $U$ is a $\Z \times \Z/2$ vector space equipped with a graded totally anti-symmetric pairing $\<-,-\>$ of total degree zero.
Then we can consider the symplectic boson vertex algebra valued in $U$ that we denote $\cS\cB(U)$.
If $\{f_i\}$ is a basis for $U$ then the vertex algebra is generated by free fields $f_i(z)$ of spin $1/2$ and with OPEs
\beqn
f_i(z) f_j(w) \simeq \frac{\<f_i,f_j\>}{z-w} .
\eeqn
This vertex algebra inherits a $\Z \times \Z/2$ grading from $U$.

\begin{prop}
The vertex algebra $\VV_{S,W}$ is quasi-isomorphic to the vertex algebra
\beqn
\cC\cB(H^\bu(S,\Omega^1_S)) \otimes \cS\cB(\Pi H^\bu(S,W)[1]) \otimes T
\eeqn 
where $T$ is the vector space
\beqn
\Sym(H^\bu(S,K_S)[1])
\eeqn
equipped with the trivial vertex algebra structure.
\end{prop}

\section{AGT statements}

Let $S$ be a projective surface and $L_1,L_2,L_3$ line bundles on $S$ satisfying 
\beqn
L_1 \otimes L_2 \otimes L_3 = K_S .
\eeqn
We consider a special case of the geometry (2) where the threefold that twisted fivebranes $X$ is of the form
\beqn
X = \op{Tot} \begin{pmatrix} L_3 \\ \downarrow \\ S \end{pmatrix},
\eeqn
and where $W = L_1 \oplus L_2$ (or rather, the restrictions of these line bundles to $X$).
In this case, the local Calabi--Yau fivefold geometry is $Z = \op{Tot}(L_1 \oplus L_2 \oplus L_3).$
\end{document}
