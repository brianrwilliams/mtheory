% Created 2020-08-18 Tue 22:24
% Intended LaTeX compiler: pdflatex
\documentclass[11pt]{article}
\usepackage[utf8]{inputenc}
\usepackage[T1]{fontenc}
\usepackage{graphicx}
\usepackage{grffile}
\usepackage{longtable}
\usepackage{wrapfig}
\usepackage{rotating}
\usepackage[normalem]{ulem}
\usepackage{amsmath}
\usepackage{textcomp}
\usepackage{amssymb}
\usepackage{capt-of}
\usepackage{hyperref}
\usepackage{macros}
\usepackage{tikz,tikz-cd}
\newcommand{\surya}[1]{(\textcolor{red}{Surya: #1})}
\newcommand{\brian}[1]{(\textcolor{blue}{Brian: #1})}
%brian's macros
\newcommand\bu{\bullet}
\newcommand\dbar{\overline{\partial}}

\date{\today}
\title{Holomorphic M-theory and the $\SU(4)$-invariant twist of type $\IIA$ }
\hypersetup{
 pdfauthor={Surya Raghavendran},
 pdftitle={BCOV with potentials},
 pdfkeywords={},
 pdfsubject={},
 pdfcreator={Emacs 26.3 (Org mode 9.4)}, 
 pdflang={English}}
\begin{document}

\maketitle

BCOV with potentials refers to a modification of minimal BCOV theory where we impose certain constraints on the fields so as to make the Poisson BV structure of the theory invertible. These constraints amount to requiring that certain fields lie in the image of the divergence operator \(\del\), or better yet replacing $\del$-closed fields in a summand \(\PV^{d,\bullet}\) with all of \(\PV^{d,\bullet}\) and using a fixed choice of splitting of \(\del: \PV^{d,\bullet}\to \PV^{d-1,\bullet}\) to rewrite \(\PV^{d,\bullet}\cong\im\del\oplus\ker\del\).

Under the conjectures of Costello-Li that describe twisted type II supergravity in terms of BCOV theory, these primitives correspond to certain components of Ramond-Ramond fields, which are chosen as potentials for Ramond-Ramond field strengths.

\section{Warm-up: Kodaira--Spencer theory on a Calabi--Yau surface}

Let $X$ be a Calabi--Yau surface. 
Minimal Kodaira--Spencer theory is a $\mb Z/2$-graded theory described by two fundamental sets of fields:
\begin{itemize}
\item An odd field given by a divergence-free holomorphic vector field $\mu^1$. 
\item An even field given by a holomorphic function $\mu^0$.
\end{itemize}
These fields combine to define a $\mb Z/2$-graded sheaf $\mc E^{\rm hol}$ on $X$. 
There is a Lie algebra structure the parity reversed sheaf $\Pi \mc E^{\rm hol}$ using the Lie bracket of holomorphic vector fields together with the natural action of holomorphic vector fields on holomorphic functions. 

The sheaf $\mc E^{\rm hol}$ admits the following locally free description:
\[
\begin{tikzcd}
\ul{\rm odd} & \ul{\rm even} \\
 & \PV^{0,\bu} \\
 \PV^{1,\bu} \ar[r, "\partial"] & \PV^{0,\bu} .
\end{tikzcd}
\]
We refer to this locally free description by $\mc E$. 
The Lie bracket on $\Pi \mc E^{\rm hol}$ described above extends to a Lie bracket on $\Pi \mc E$.
Together with the differential this gives $\Pi \mc E$ the structure of a local dg Lie algebra. 

The bundle $\mc E$ is equipped with an odd Poisson tensor defined by 
\[
\Pi = (\partial \otimes 1) \delta_{\rm Diag} .
\]

We introduce another theory on the Calabi--Yau surface $X$ that we call minimal Kodaira--Spencer theory {\em with potentials}.
The underlying vector bundle is 
\[
\begin{tikzcd}
\ul{\rm odd} & \ul{\rm even} \\
 & \PV^{0,\bu} \\
 \PV^{2,\bu}  .
\end{tikzcd}
\]
We will denote the resulting $\mb Z/2$-graded sheaf of cochain complexes by $\mc E_{\rm pot}$. 

We interpret this as the theory of  ``potentials"  of minimal Kodaira--Spencer theory in the following way. 
There is a map of bundles $\Phi : \mc E_{\rm Pot} \to \mc E$ which is the identity on $\PV^{0,\bu}$ and given by $\partial : \PV^{2,\bu} \to \PV^{1,\bu}$ on the remaining component. 
It is immediate to see that $\Phi$ defines a map of sheaves of cochain complexes. 

In fact, the parity shifted bundle $\Pi \mc E_{\rm pot}$ also has the structure of a local Lie algebra, and the map $\Phi$ intertwines these local Lie algebra structures. 

To describe the local Lie algebra structure on minimal Kodaira--Spencer theory with potentials we use the Calabi--Yau form $\Omega$ to identify $\mc E_{\rm Pot}$ with the sheaf of cochain complexes
\[
\begin{tikzcd}
\ul{\rm odd} & \ul{\rm even} \\
 & \Omega^{2,\bu} \\
 \Omega^{0,\bu}  .
\end{tikzcd}
\]

Now, note that any Calabi--Yau surface comes equipped with a holomorphic symplectic structure and there is a Poisson bracket defined on the sheaf of holomorphic functions.
Since the bracket is defined in terms of holomorphic differential operators, it extends to a bracket on the Dobleault complex $\Omega^{0,\bu}(X)$.

This further extends to a local Lie algebra structure on the semi-direct product 
\[
\Omega^{0,\bu}(X) \ltimes \Pi \Omega^{2,\bu}(X)
\]
which describes the local Lie structure on $\Pi \mc E_{\rm Pot}$. 
It is immediate to verify that the map $\Phi : \mc E_{\rm pot} \to \mc E$ intertwines the two $L_\infty$-structures. 

%Using the isomorphism $\Omega : \PV^{2, \bu} (X) \cong \Omega^{0,\bu}(X)$ we thus obtain a Lie bracket on $\PV^{2,\bu}(X)$ defined by the equation
%\[
%[\alpha, \alpha']_{\PV^2} = \Omega^{-1} [\Omega \alpha, \Omega \alpha'] .
%\]
%This endows $\PV^{2,\bu}(X)$ 
%
%The sheaf $\PV^{0,\bu}(X)$ is a module for 

Finally, the theory $\mc E_{\rm pot}$ is a non-degenerate BV theory with BV pairing defined by the wedge-and-integrate pairing
\[
\alpha, \beta \mapsto \int \alpha \wedge \beta  .
\]

\begin{prop}
The map $\Phi$ determines a map of $\mb P_0$-factorization algebras on $X$:
\[
\Phi^* : {\rm Obs}_{\mc E} \to {\rm Obs}_{\mc E_{\rm Pot}} .
\]
\end{prop}

\section{BCOV theory with potentials on a CY4}
\label{sec:org3d6a090}
Let \(X\) be a Calabi-Yau 4 fold. Minimal Kodaira-Spencer theory on $X$ is a $\Z/2$-graded theory with the following fundamental fields:
\begin{itemize}
\item The even fields are a holomorphic function $\mu^0$ and a $\del$-closed holomorphic bivector $\mu^2$.
\item The odd fields are a divergence-free holomorphic vector field $\mu^1$ and a $\del$-closed holomorphic section $\mu^3$ of $\wedge^3 T_X$. 
\end{itemize}

%As before, these fields combine to define a $\Z/2$-graded sheaf $\mc E^{\rm hol}$ on $X$. The parity reversed sheaf $\Pi\mc E^{\rm hol}$ has a graded Lie algebra structure given by the Schouten-Nijenhuis bracket of holomorphic polyvector fields.

The space of fields admits a locally free description obtained by including the ``descendants". 
The descendants of the field $\mu^j$ will be denoted $u^k \mu^j$ where $k = 1,\ldots, j$.
Here, $u^k \mu^j$ is a section of $\PV^{j - k, \bu}$. 
The sheaf of cochain complexes $\mc E$ underlying minimal Kodaira--Spencer theory on $X$ is 
\[
\begin{tikzcd}
\ul{\rm odd} & \ul{\rm even} & \ul{\rm odd} & \ul{\rm even} \\
& & & \mu^0 \in \PV^{0,\bu} \\
& & \sum u^k \mu^1 \in \PV^{1,\bu} \ar[r, "u \partial"] & u \PV^{0,\bu} \\
& \sum u^k \mu^2 \in  \PV^{2,\bu} \ar[r,"u \del"] & u \PV^{1,\bu} \ar[r, "u \partial"] & u^2 \PV^{0,\bu} \\ 
\sum u^k \mu^3 \in \PV^{3,\bu} \ar[r, "u \del"]& u \PV^{2,\bu} \ar[r,"u \del"] & u^2 \PV^{1,\bu} \ar[r, "u \partial"] & u^3\PV^{0,\bu} \\ 
\end{tikzcd}
\]
The differential on this sheaf of cochain complexes is given by $\dbar + u \partial$. 

There is a local Lie algebra structure on $\Pi\mc E$ using the Schouten-Nijenhuis bracket $[-,-]_{\rm Sch}$ on polyvector fields.
On the fields (including the descendants) it is defined by the formula
\[
[u^k \mu^i , u^\ell \mu^j] = u^{k+\ell} [\mu^i, \mu^j]_{\rm Sch} .
\]

Furthermore, the sheaf $\mc E$ is equipped with an odd Poisson tensor given by $\Pi = (\del\otimes 1)\delta_{\rm Diag}$. Together, this data equips $\mc E$ with the structure of a $\Z/2$-graded Poisson BV theory.

As in the surface case, there is a closely related BV theory describing the "potentials" of minimal Kodaira--Spencer theory.
The underlying sheaf of cochain complexes is 
\[
\begin{tikzcd}
\ul{\rm odd} & \ul{\rm even} & \ul{\rm odd} & \ul{\rm even} \\
& & & \mu^{0}\in\PV^{0,\bu} \\
& & \sum u^k \mu^{1}\in\PV^{1,\bu} \ar[r, "\partial"] & \mu^{c}\in\PV^{0,\bu} \\
u^{-1} \gamma^3 + \gamma^3  \in u^{-1} \PV^{4,\bu} \ar[r, "u \partial"] & \PV^{3,\bu} & &  \\
\gamma^{4} \in\PV^{4,\bu} &  &  & \\
\end{tikzcd}
\]
We will again refer to this sheaf as $\mc E_{\rm pot}$.

There is a local Lie algebra structure described as follows.
\brian{It's mostly the Schouten bracket, but there is the additional ``weird" bracket
\[
[\gamma^3, \gamma^{3'}] = [\partial \gamma^3 , \gamma^{3'}] \pm [\gamma^3, \partial \gamma^{3'}] .
\]
}

There is a map of sheaves of cochain complexes $\Phi: \mc E_{\rm pot}\to \mc E$ given by the identity map on $\PV^{1,\bu}$ and $\PV^{0,\bu}$ and the $\del$ operator on $\PV^{3,\bu}$ and $\PV^{4,\bu}$. 
Explicitly, in formulas
\[
\begin{array}{ccccccccc}
\Phi(t^k \mu^{i}) & = & u^k \mu^i & \in & u^k \PV^{i, \bu} & i = 0,1 \\
\Phi(u^{-1} \gamma^3) & = & 0 \\
\Phi(\gamma^3) & = & \partial \gamma^3 & \in & \PV^{2,\bu} \\
\Phi(\gamma^4) & = & \partial \gamma^4 & \in & \PV^{3,\bu} .
\end{array} 
\]

%We may equip the parity shifted bundle $\Pi\mc E_{\rm pot}$ with the structure of a local dg Lie algebra such that $\Phi$ is a map of Lie algebras. \surya{Will add explicit description of the brackets}. 

Together with the wedge and integrate pairing, $\mc E_{\rm pot}$ has the structure of a nondegenerate BV theory.

In fact, we have the following, analogous to the case of a Calabi-Yau surface:
\begin{prop}
The map $\Phi$ determines a map of $\mb P_0$-factorization algebras on $X$:
\[
\Phi^* : {\rm Obs}_{\mc E} \to {\rm Obs}_{\mc E_{\rm Pot}} .
\]
\end{prop} 

%\iffalse
Let \(X\) be a CY4. BCOV theory on \(X\) with potentials will be the \(\mb{Z}/2\) graded BV theory defined as follows. Fix splittings \(C_1: (\im\del\subset\PV^{3,\bullet})\to \PV^{4,\bullet}\) of  \[0\to\ker\del\to \PV^{4,\bullet}\to(\im\del\subset\PV^{3,\bullet}\to 0\] and \(C_2: (\im\del\subset\PV^{2,\bullet})\to\PV^{3,\bullet}\) of \[0\to\ker\del\to\PV^{3,\bullet}\to(\im\del\subset\PV^{2,\bullet}(X))\to 0.\] And let \(\phi_{C_1} : \PV^{4,\bullet}\cong(\ker\del\subset\PV^{4,\bullet})\oplus(\im\del\subset\PV^{3,\bullet})\), \(\phi_{C_2}: \PV^{3,\bullet}\cong(\ker\del\subset\PV^{3,\bullet})\oplus(\im\del\subset\PV^{2,\bullet})\) be the resulting isomorphisms.
\begin{itemize}
\item The fields of the theory are
 \[\begin{aligned} \mc E_{mBCOV}^{C_1,C_2} = \PV^{0,\bullet} \oplus (\ker\del\subset\PV^{1,\bullet})\oplus & (\im\del\subset\PV^{2,\bullet} \oplus\ker\del\subset\PV^{3,\bullet})\\ \oplus &(\im\del\subset\PV^{3,\bullet}\oplus\ker\del\subset\PV^{4,\bullet})\end{aligned}\]
\item The Poisson kernel is given by $(\del\otimes 1)\delta_{Diag}$
\item The \(L_\infty\) structure is defined as follows
\begin{itemize}
\item \(\ell_1=\delbar\)
\item \(\ell_2\) is a certain modification of the Schouten bracket, defined as follows. Let \([-,-]\) denote the usual Schouten bracket of polyvector fields. Then

\begin{enumerate}
\item \(\mu\in\PV^{0,\bullet}, \nu\in\ker\del\subset\PV^{1,\bullet}\), \(\ell_2(\mu,\nu)=[\mu,\nu]\).
\item \(\mu\in\PV^{0,\bullet}, \nu\in\im\del\subset\PV^{2,\bullet}\), \(\ell_2(\mu,\nu)=(-1)^{|\mu|-1}\del[\mu,\phi^{_1}_{C_1}\nu]\) and \(\ell_2(\nu,\mu)=\del[\mu,\phi^{-1}_{C_1}\nu]\)
\item \(\mu\in\PV^{0,\bullet}, \nu\in(\im\del\subset\PV^{3,\bullet})\oplus(\ker\del\subset\PV^{4,\bullet})\), \(\ell_2(\mu,\nu)=(-1)^{|\mu|-1}\phi_{C_1}[\mu, \phi^{-1}_{C_2}\nu]\) and \(\ell_2(\nu, \mu)=\phi_{C_1}[\mu, \phi^{-1}_{C_2}\nu]\).
\item \(\mu,\nu\in\ker\del\subset\PV^{1,\bullet}\), \(\ell_2(\mu,\nu)=[\mu,\nu]\).
\item \(\mu\in\ker\del\subset\PV^{1,\bullet}, \nu\in(\im\del\subset\PV^{2,\bullet})\oplus(\ker\del\subset\PV^{3,\bullet})\), \(\ell_2(\mu,\nu)=(-1)^{|\mu|-1}\phi_{C_1}[\mu, \phi^{-1}_{C_1}\nu]\) and \(\ell_2(\nu, \mu)=\phi_{C_1}[\phi^{-1}_{C_1}\nu, \mu]\)
\item \(\mu\in\ker\del\subset\PV^{1,\bullet}, \nu\in(\im\del\subset\PV^{3,\bullet})\oplus(\ker\del\subset\PV^{5,\bullet})\), \(\ell_2(\mu,\nu)=(-1)^{|\mu|-1}\phi_{C_2}[\mu, \phi^{-1}_{C_2}\nu]\) and \(\ell_2(\nu, \mu)=\phi_{C_2}[\phi^{-1}_{C_2}\nu, \mu]\)
\item All other brackets vanish for degree reasons.
\end{enumerate}
\end{itemize}
\end{itemize}

\begin{prop}
The above in fact defines a (shifted) $L_\infty$-structure.
\end{prop}

\begin{rmk}
\begin{itemize}
\item Note that the complex underlying $\mc E_{mBCOV}^{C_1,C_2}$ does not arise as sections of a graded vector bundle due to the presence of the constraints.
\item Recall that minimal BCOV theory has fields $\mc E_{mBCOV}=\oplus_{i\leq 3} (\ker\del\subset\PV^{\i,\bullet}$. As we have mentioned above, we may view the underlying complex of $\mc E_{mBCOV}^{C_1,C_2}$ as gotten by replacing $\ker\del\subset\PV^{2/3,\bullet}\subset\mc E_{mBCOV}$ with $\PV^{3/4,\bullet}$ in the same degree, and using the isomorphisms $\phi_{C_1,C_2}$. The $L_\infty$ structure is gotten by simply transporting the ordinary Schouten-Nijenhuis bracket on \[\PV^{0,\bullet}\oplus(\ker\del\subset\PV^{1,\bullet}\oplus\PV^{3,\bullet}[]\oplus\PV^{4,\bullet}[]\] and applying the fact that since $\del$ is a derivation of $[-,-]$, for $\mu\in\ker\del$, $[\mu,\del\gamma]=(-1)^{|\mu|-1}\del[\mu,\gamma]$ and $[\del\gamma,\mu]=\del[\gamma,\mu]$. Hopefully this demystifies the above formulas.
\item Suppose we were to naively try to define a BV pairing $\omega$ on $\mc E_{mBCOV}^{C_1,C_2}$ by $\omega(-,-)=\int (-)\wedge\del^{-1}(-)$ and write an "action functional" using the above $L_\infty$-structure. Then the resulting action functional would be equivalent to one coming from an honest BV pairing and an $L_\infty$-structure involving a composition of $\del$ and the ordinary Schouten-Nijenhuis bracket. Note that we could have accomplished the same thing by choosing splittings of $\del$ from any $\im\del\subset \PV^{d_1,\bullet}$, $\im\del\subset\PV^{d_2,\bullet}$ such that $d_1+d_2\neq d-1$. However, it seems like these two choices of splittings are favored in a sense (articulated below). 
\end{itemize}
\end{rmk}
%\fi

\section{Dimensional Reduction}
Let's consider the holomorphic twist of M-theory on $\R\times\C^\times\times\C^4$. We may decompose the fields as
\begin{itemize}
\item \[(\mu^{1}, \mu^{0})=(\mu_{01}+\mu_{10}, \mu^{0})\in\Omega^\bullet(\R)\otimes\left (\begin{aligned} &( \PV^{0,\bullet}(\C^\times)\otimes\PV^{1,\bullet}(\C^4)) \\ \oplus & (\PV^{1,\bullet}(\C^\times)\otimes\PV^{0,\bullet}(\C^4))\end{aligned} \to\PV^{0,\bullet}(\C^\times\times\C^4)\right ).\]
\item $\gamma=\gamma_{01}+\gamma_{10}\in \Omega^\bullet(\R)\otimes(\Omega^{0,\bullet}(\C^\times)\otimes\Omega^{1,\bullet}(\C^4)\oplus\Omega^{1,\bullet}(\C^\times)\otimes\Omega^{0,\bullet}(\C^4))$.
\end{itemize}

\begin{prop}
There is a homomorphism of $L_\infty$-algebras from the $\delbar_{\C^\times}$-cohomology of M theory on $\R\times\C^\times\times\C$ to $\Omega^\bullet(\R^2)\otimes\mc E_{Pot}$ given by
\begin{itemize}
\item $[\gamma_{10}]\mapsto\mu^0\in\PV^{0,\bullet}\subset\mc E_{Pot}$.
  \item $[\mu_{01}]\mapsto\mu^1\in\PV^{1,\bullet}(\C^4)\subset\mc E_{Pot}$.
\item $[\mu^{0}]\mapsto\mu^{c}\in\PV^{0, \bullet}(\C^{4})\subset\mc E_{Pot}$.
\item $[\gamma_{01}]\mapsto\mu^3 = \gamma_{01}\vee\Omega^{-1}_{\C^4}\in\PV^{3,\bullet}\subset\mc E_{Pot}$ where $\Omega_{\C^4}$ denotes the holomorphic volume form on $\C^4$.
\item $[\mu_{10}]\mapsto\mu^4 =\mu_{10}\vee\Omega^{-1}_{\C^4}\in\PV^{4,\bullet}\subset\mc E_{Pot}$ .


\end{itemize}
preserving the relevant pairings.
\end{prop}

That is, the reduction of the holomorphic M theory on a holomorphic circle should be the $\SU(4)$ invariant twist of $\IIA$.

\begin{rmk}
Note that the field $\mu^{c}\in \mc E_{\mr Pot}$ is nonpropating. It must pair against a field in $\PV^{3}$. In order for the field $\mu^{0}$ in M-theory to propagate, it must pair against a field $\gamma^{0}\in \Omega^{0,\bullet}$ which is the ghost for the gauge transformation for $\gamma^{1}$. This suggests an enlargement of $\mc E_{\mr Pot}$ where we add an additional field in $\PV^{3}$ which will be the image of the ghost of $\gamma^{1}$ in M-theory under dimensional reduction.
\end{rmk}

\surya{ When we discussewd dimensional reduction on 9/22 over Zoom, I accidentally switched the roles of $\gamma_{10}$ and $\mu_{10}$. I realized this was wrong - the resulting map cannot be a Lie map. Indeed, there is certainly a nontrivial bracket between $\gamma_{10}$ and $\gamma_{10}$ in M-theory, from expanding the $\gamma\del\gamma\del\gamma$ term. However, there is no nontrivial bracket between $\PV^{3,\bullet}$ and $\PV^{4,\bullet}$ in $\mc E_{\rm Pot}$ }

\end{document}
