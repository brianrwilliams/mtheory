% Created 2020-08-18 Tue 22:24
% Intended LaTeX compiler: pdflatex
\documentclass[11pt]{article}
\usepackage[utf8]{inputenc}
\usepackage[T1]{fontenc}
\usepackage{graphicx}
\usepackage{grffile}
\usepackage{longtable}
\usepackage{wrapfig}
\usepackage{rotating}
\usepackage[normalem]{ulem}
\usepackage{amsmath}
\usepackage{textcomp}
\usepackage{amssymb}
\usepackage{capt-of}
\usepackage{hyperref}
\usepackage{macros}
\usepackage{tikz,tikz-cd}
\newcommand{\surya}[1]{(\textcolor{red}{Surya: #1})}
\newcommand{\brian}[1]{(\textcolor{blue}{Brian: #1})}
%brian's macros
\newcommand\bu{\bullet}
\newcommand\dbar{\overline{\partial}}
\newcommand\CC{{\mb C}}
\newcommand \RR{{\mb R}}
\def\Hat{\widehat}

\date{\today}
\title{Holomorphic M-theory and the $\SU(4)$-invariant twist of type $\IIA$ }
\hypersetup{
 pdfauthor={Surya Raghavendran},
 pdftitle={BCOV with potentials},
 pdfkeywords={},
 pdfsubject={},
 pdfcreator={Emacs 26.3 (Org mode 9.4)}, 
 pdflang={English}}
\begin{document}

\maketitle

BCOV with potentials refers to a modification of minimal BCOV theory where we impose certain constraints on the fields so as to make the Poisson BV structure of the theory invertible. These constraints amount to requiring that certain fields lie in the image of the divergence operator \(\del\), or better yet replacing $\del$-closed fields in a summand \(\PV^{d,\bullet}\) with all of \(\PV^{d,\bullet}\) and using a fixed choice of splitting of \(\del: \PV^{d,\bullet}\to \PV^{d-1,\bullet}\) to rewrite \(\PV^{d,\bullet}\cong\im\del\oplus\ker\del\).

Under the conjectures of Costello-Li that describe twisted type II supergravity in terms of BCOV theory, these primitives correspond to certain components of Ramond-Ramond fields, which are chosen as potentials for Ramond-Ramond field strengths.

\section{Warm-up: Kodaira--Spencer theory on a Calabi--Yau surface}

On a Calabi--Yau surface $X$, minimal Kodaira--Spencer theory is the $\mb Z/2$-graded theory described by two fundamental sets of fields:
\begin{itemize}
\item An odd field given by a divergence-free holomorphic vector field $\mu^1$. 
\item An even field given by a holomorphic function $\mu^0$.
\end{itemize}
%These fields combine to define a $\mb Z/2$-graded sheaf $\mc E^{\rm hol}$ on $X$. 
%There is a Lie algebra structure the parity reversed sheaf $\Pi \mc E^{\rm hol}$ using the Lie bracket of holomorphic vector fields together with the natural action of holomorphic vector fields on holomorphic functions. 

%The sheaf $\mc E^{\rm hol}$ admits the following locally free description:

In the BV formalism, minimal Kodaira--Spencer theory on $X$ is a (degenerate) Poisson BV theory with space of fields given by
\[
\begin{tikzcd}
\ul{\rm odd} & \ul{\rm even} \\
 & \PV^{0,\bu} \\
 \PV^{1,\bu} \ar[r, "u \partial"] & u \PV^{0,\bu} .
\end{tikzcd}
\]
We will denote this sheaf of $\mb Z/2$-graded cochain complexes by $\mc E_{\rm KS}$.

There is a local (dg) Lie algebra structure on the parity shifted object $\Pi \mc E_{\rm KS}$. 
The Lie bracket is defined using the Schouten-Nijenhuis bracket $[-,-]_{\rm NS}$ on polyvector fields and is given by the formula
\[
[u^k \alpha , u^\ell \beta] = u^{k+\ell} [\alpha, \beta]_{\rm NS} .
\]
where $k, \ell = 0,1$.
Together with the differential this equips the parity shifted sheaf of cochain complexes $\Pi \mc E_{\rm KS}$ with the structure of a local (dg) Lie algebra. 

The fields of minimal Kodaira--Spencer theory $\mc E_{\rm KS}$ is equipped with an odd Poisson tensor defined by 
\[
\Pi_{\rm KS} = (\partial \otimes 1) \delta_{\rm Diag} .
\]

We introduce another theory on the Calabi--Yau surface $X$ that we call minimal Kodaira--Spencer theory {\em with potentials}.
The underlying vector bundle is 
\[
\begin{tikzcd}
\ul{\rm odd} & \ul{\rm even} \\
 & \PV^{0,\bu} \\
 \PV^{2,\bu}  .
\end{tikzcd}
\]
We will denote the resulting $\mb Z/2$-graded sheaf of cochain complexes by $\mc E_{\rm pot}$. 

We interpret this as the theory of  ``potentials"  of minimal Kodaira--Spencer theory in the following way. 
There is a map of sheaves of cochani complexes 
\[
\Phi : \mc E_{\rm Pot} \to \mc E_{\rm KS}
\]
which is the identity on $\PV^{0,\bu}$ and given by $\partial : \PV^{2,\bu} \to \PV^{1,\bu}$ on the remaining component. 
It is immediate to see that $\Phi$ defines a map of sheaves of cochain complexes. 
The theory $\mc E_{\rm pot}$ is equipped with a non-degenerate BV pairing defined by the wedge-and-integrate pairing
\[
\omega_{\rm pot} (\alpha, \beta) = \int \alpha \wedge \beta  .
\]
It is immediate to verify that $\Phi$ intertwines the resulting bivector $\omega_{\rm pot}^{-1}$ and the Kodaira--Spencer Poisson bivector $\Pi_{\rm KS}$.

In fact, the parity shifted bundle $\Pi \mc E_{\rm pot}$ also has the structure of a local Lie algebra, and the map $\Phi$ intertwines these local Lie algebra structures. 

To describe the local Lie algebra structure on minimal Kodaira--Spencer theory with potentials we use the Calabi--Yau form $\Omega$ to identify $\mc E_{\rm Pot}$ with the sheaf of cochain complexes
\[
\begin{tikzcd}
\ul{\rm odd} & \ul{\rm even} \\
 & \Omega^{2,\bu} \\
 \Omega^{0,\bu}  .
\end{tikzcd}
\]

Now, note that any Calabi--Yau surface comes equipped with a holomorphic symplectic structure and there is a Poisson bracket defined on the sheaf of holomorphic functions.
Since the bracket is defined in terms of holomorphic differential operators, it extends to a bracket on the Dobleault complex $\Omega^{0,\bu}(X)$.

This further extends to a local Lie algebra structure on the semi-direct product 
\[
\Omega^{0,\bu}(X) \ltimes \Pi \Omega^{2,\bu}(X)
\]
which describes the local Lie structure on $\Pi \mc E_{\rm Pot}$. 
It is immediate to verify that the map $\Phi : \mc E_{\rm pot} \to \mc E$ intertwines the two $L_\infty$-structures. 

This equipped $\mc E_{\rm pot}$ with the structure of an interacting (non-degenerate) BV theory. 
It's relationship to (minimal) Kodaira--Spencer theory can be summarized as follows. 

%Using the isomorphism $\Omega : \PV^{2, \bu} (X) \cong \Omega^{0,\bu}(X)$ we thus obtain a Lie bracket on $\PV^{2,\bu}(X)$ defined by the equation
%\[
%[\alpha, \alpha']_{\PV^2} = \Omega^{-1} [\Omega \alpha, \Omega \alpha'] .
%\]
%This endows $\PV^{2,\bu}(X)$ 
%
%The sheaf $\PV^{0,\bu}(X)$ is a module for 

\begin{prop}
The map $\Phi : \mc E_{\rm pot} \to \mc E_{\rm KS}$ determines a map of $\mb P_0$-factorization algebras on $X$:
\[
\Phi^* : {\rm Obs}_{\mc E_{\rm KS}} \to {\rm Obs}_{\mc E_{\rm Pot}} .
\]
\end{prop}

\section{BCOV theory with potentials on a CY4}
\label{sec:org3d6a090}
Let \(X\) be a Calabi-Yau 4 fold. Minimal Kodaira-Spencer theory on $X$ is a $\Z/2$-graded theory with the following fundamental fields:
\begin{itemize}
\item The even fields are a holomorphic function $\mu^0$ and a $\del$-closed holomorphic bivector $\mu^2$.
\item The odd fields are a divergence-free holomorphic vector field $\mu^1$ and a $\del$-closed holomorphic section $\mu^3$ of $\wedge^3 T_X$. 
\end{itemize}

%As before, these fields combine to define a $\Z/2$-graded sheaf $\mc E^{\rm hol}$ on $X$. The parity reversed sheaf $\Pi\mc E^{\rm hol}$ has a graded Lie algebra structure given by the Schouten-Nijenhuis bracket of holomorphic polyvector fields.

The space of fields admits a locally free description obtained by including the ``descendants". 
The descendants of the field $\mu^j$ will be denoted $u^k \mu^j$ where $k = 1,\ldots, j$.
Here, $u^k \mu^j$ is a section of $\PV^{j - k, \bu}$. 
The sheaf of cochain complexes $\mc E$ underlying minimal Kodaira--Spencer theory on $X$ is 
\[
\begin{tikzcd}
\ul{\rm odd} & \ul{\rm even} & \ul{\rm odd} & \ul{\rm even} \\
& & & \mu^0 \in \PV^{0,\bu} \\
& & \sum u^k \mu^1 \in \PV^{1,\bu} \ar[r, "u \partial"] & u \PV^{0,\bu} \\
& \sum u^k \mu^2 \in  \PV^{2,\bu} \ar[r,"u \del"] & u \PV^{1,\bu} \ar[r, "u \partial"] & u^2 \PV^{0,\bu} \\ 
\sum u^k \mu^3 \in \PV^{3,\bu} \ar[r, "u \del"]& u \PV^{2,\bu} \ar[r,"u \del"] & u^2 \PV^{1,\bu} \ar[r, "u \partial"] & u^3\PV^{0,\bu} \\ 
\end{tikzcd}
\]
The differential on this sheaf of cochain complexes is given by $\dbar + u \partial$. 

There is a local Lie algebra structure on $\Pi\mc E$ using the Schouten-Nijenhuis bracket $[-,-]_{\rm Sch}$ on polyvector fields.
On the fields (including the descendants) it is defined by the formula
\[
[u^k \mu^i , u^\ell \mu^j] = u^{k+\ell} [\mu^i, \mu^j]_{\rm NS} .
\]

The space of fields of minimal Kodaira--Spencer theory $\mc E_{\rm KS}$ is equipped with an odd Poisson tensor defined by 
\[
\Pi_{\rm KS} = (\partial \otimes 1) \delta_{\rm Diag} .
\]
Together with the local Lie algebra structure, this data equips $\mc E$ with the structure of a $\Z/2$-graded Poisson BV theory.

As in the surface case, there is a closely related BV theory describing the "potentials" of minimal Kodaira--Spencer theory.
The underlying sheaf of cochain complexes is 
\[
\begin{tikzcd}
\ul{\rm odd} & \ul{\rm even} & \ul{\rm odd} & \ul{\rm even} \\
& & & \mu^{0}\in\PV^{0,\bu} \\
& & \sum u^k \mu^{1}\in\PV^{1,\bu} \ar[r, "u \partial"] & u \PV^{0,\bu} \\
u^{-1} \gamma^3 + \gamma^3  \in u^{-1} \PV^{4,\bu} \ar[r, "u \partial"] & \PV^{3,\bu} & &  \\
\gamma^{4} \in\PV^{4,\bu} &  &  & \\
\end{tikzcd}
\]
We will again refer to this sheaf as $\mc E_{\rm pot}$.

There is a local Lie algebra structure described as follows.
\begin{itemize}
\item 
There is a self bracket between the $\mu$-fields defined by
\[
[u^k \mu^i, u^\ell \Tilde{\mu}^j] = u^{k+j} [\mu^i, \Tilde{\mu}^j]_{\rm NS} .
\]
\item Next, there is a self-bracket between the $\gamma$-fields defined by 
\[
[\gamma^3, \Tilde{\gamma}^3] = [\partial \gamma^3 , \Tilde{\gamma}^{3}]_{\rm NS} \pm [\gamma^3, \partial \Tilde{\gamma}^{3}]_{\rm NS} \in {\rm PV}^{4,\bu} 
\]
\item Finally, there are mixed terms in the bracket defined by
\[
[u^{-k} \gamma^i , u^\ell \mu^j] = u^{-k + \ell} [\partial \gamma^i, \mu^j]_{\rm NS}
\]
\end{itemize}

Together with the wedge and integrate pairing, $\mc E_{\rm pot}$ has the structure of a nondegenerate BV theory.

Like in the case of Kodaira--Spencer theory on a complex surface, there is a  map of sheaves of cochain complexes 
\[
\Phi: \mc E_{\rm pot}\to \mc E_{\rm KS} .
\] 
It is given by the identity map on $\PV^{1,\bu}$ and $\PV^{0,\bu}$ and the $\del$ operator on $\PV^{3,\bu}$ and $\PV^{4,\bu}$. 
Explicitly, in formulas
\[
\arraycolsep=1.4pt\def\arraystretch{2.2}
\begin{array}{ccccccccc}
\Phi(u^k \mu^{i}) & = & u^k \mu^i & \in & u^k \PV^{i, \bu} & i = 0,1 \\
\Phi(u^{-1} \gamma^3) & = & 0 \\
\Phi(\gamma^3) & = & \partial \gamma^3 & \in & \PV^{2,\bu} \\
\Phi(\gamma^4) & = & \partial \gamma^4 & \in & \PV^{3,\bu} .
\end{array} 
\]

%We may equip the parity shifted bundle $\Pi\mc E_{\rm pot}$ with the structure of a local dg Lie algebra such that $\Phi$ is a map of Lie algebras. \surya{Will add explicit description of the brackets}. 

In fact, we have the following result, in analogy with the case of a Calabi-Yau surface.

\begin{prop}
The map $\Phi$ intertwines the local Lie algebra structures and Poisson BV structures on $\mc E_{\rm KS}$ and $\mc E_{\rm pot}$. 
So, it induces a map of $\mb P_0$-factorization algebras:
\[
\Phi^* : {\rm Obs}_{\mc E} \to {\rm Obs}_{\mc E_{\rm Pot}} .
\]
\end{prop} 

\iffalse
Let \(X\) be a CY4. BCOV theory on \(X\) with potentials will be the \(\mb{Z}/2\) graded BV theory defined as follows. Fix splittings \(C_1: (\im\del\subset\PV^{3,\bullet})\to \PV^{4,\bullet}\) of  \[0\to\ker\del\to \PV^{4,\bullet}\to(\im\del\subset\PV^{3,\bullet}\to 0\] and \(C_2: (\im\del\subset\PV^{2,\bullet})\to\PV^{3,\bullet}\) of \[0\to\ker\del\to\PV^{3,\bullet}\to(\im\del\subset\PV^{2,\bullet}(X))\to 0.\] And let \(\phi_{C_1} : \PV^{4,\bullet}\cong(\ker\del\subset\PV^{4,\bullet})\oplus(\im\del\subset\PV^{3,\bullet})\), \(\phi_{C_2}: \PV^{3,\bullet}\cong(\ker\del\subset\PV^{3,\bullet})\oplus(\im\del\subset\PV^{2,\bullet})\) be the resulting isomorphisms.
\begin{itemize}
\item The fields of the theory are
 \[\begin{aligned} \mc E_{mBCOV}^{C_1,C_2} = \PV^{0,\bullet} \oplus (\ker\del\subset\PV^{1,\bullet})\oplus & (\im\del\subset\PV^{2,\bullet} \oplus\ker\del\subset\PV^{3,\bullet})\\ \oplus &(\im\del\subset\PV^{3,\bullet}\oplus\ker\del\subset\PV^{4,\bullet})\end{aligned}\]
\item The Poisson kernel is given by $(\del\otimes 1)\delta_{Diag}$
\item The \(L_\infty\) structure is defined as follows
\begin{itemize}
\item \(\ell_1=\delbar\)
\item \(\ell_2\) is a certain modification of the Schouten bracket, defined as follows. Let \([-,-]\) denote the usual Schouten bracket of polyvector fields. Then

\begin{enumerate}
\item \(\mu\in\PV^{0,\bullet}, \nu\in\ker\del\subset\PV^{1,\bullet}\), \(\ell_2(\mu,\nu)=[\mu,\nu]\).
\item \(\mu\in\PV^{0,\bullet}, \nu\in\im\del\subset\PV^{2,\bullet}\), \(\ell_2(\mu,\nu)=(-1)^{|\mu|-1}\del[\mu,\phi^{_1}_{C_1}\nu]\) and \(\ell_2(\nu,\mu)=\del[\mu,\phi^{-1}_{C_1}\nu]\)
\item \(\mu\in\PV^{0,\bullet}, \nu\in(\im\del\subset\PV^{3,\bullet})\oplus(\ker\del\subset\PV^{4,\bullet})\), \(\ell_2(\mu,\nu)=(-1)^{|\mu|-1}\phi_{C_1}[\mu, \phi^{-1}_{C_2}\nu]\) and \(\ell_2(\nu, \mu)=\phi_{C_1}[\mu, \phi^{-1}_{C_2}\nu]\).
\item \(\mu,\nu\in\ker\del\subset\PV^{1,\bullet}\), \(\ell_2(\mu,\nu)=[\mu,\nu]\).
\item \(\mu\in\ker\del\subset\PV^{1,\bullet}, \nu\in(\im\del\subset\PV^{2,\bullet})\oplus(\ker\del\subset\PV^{3,\bullet})\), \(\ell_2(\mu,\nu)=(-1)^{|\mu|-1}\phi_{C_1}[\mu, \phi^{-1}_{C_1}\nu]\) and \(\ell_2(\nu, \mu)=\phi_{C_1}[\phi^{-1}_{C_1}\nu, \mu]\)
\item \(\mu\in\ker\del\subset\PV^{1,\bullet}, \nu\in(\im\del\subset\PV^{3,\bullet})\oplus(\ker\del\subset\PV^{5,\bullet})\), \(\ell_2(\mu,\nu)=(-1)^{|\mu|-1}\phi_{C_2}[\mu, \phi^{-1}_{C_2}\nu]\) and \(\ell_2(\nu, \mu)=\phi_{C_2}[\phi^{-1}_{C_2}\nu, \mu]\)
\item All other brackets vanish for degree reasons.
\end{enumerate}
\end{itemize}
\end{itemize}

\begin{prop}
The above in fact defines a (shifted) $L_\infty$-structure.
\end{prop}

\begin{rmk}
\begin{itemize}
\item Note that the complex underlying $\mc E_{mBCOV}^{C_1,C_2}$ does not arise as sections of a graded vector bundle due to the presence of the constraints.
\item Recall that minimal BCOV theory has fields $\mc E_{mBCOV}=\oplus_{i\leq 3} (\ker\del\subset\PV^{\i,\bullet}$. As we have mentioned above, we may view the underlying complex of $\mc E_{mBCOV}^{C_1,C_2}$ as gotten by replacing $\ker\del\subset\PV^{2/3,\bullet}\subset\mc E_{mBCOV}$ with $\PV^{3/4,\bullet}$ in the same degree, and using the isomorphisms $\phi_{C_1,C_2}$. The $L_\infty$ structure is gotten by simply transporting the ordinary Schouten-Nijenhuis bracket on \[\PV^{0,\bullet}\oplus(\ker\del\subset\PV^{1,\bullet}\oplus\PV^{3,\bullet}[]\oplus\PV^{4,\bullet}[]\] and applying the fact that since $\del$ is a derivation of $[-,-]$, for $\mu\in\ker\del$, $[\mu,\del\gamma]=(-1)^{|\mu|-1}\del[\mu,\gamma]$ and $[\del\gamma,\mu]=\del[\gamma,\mu]$. Hopefully this demystifies the above formulas.
\item Suppose we were to naively try to define a BV pairing $\omega$ on $\mc E_{mBCOV}^{C_1,C_2}$ by $\omega(-,-)=\int (-)\wedge\del^{-1}(-)$ and write an "action functional" using the above $L_\infty$-structure. Then the resulting action functional would be equivalent to one coming from an honest BV pairing and an $L_\infty$-structure involving a composition of $\del$ and the ordinary Schouten-Nijenhuis bracket. Note that we could have accomplished the same thing by choosing splittings of $\del$ from any $\im\del\subset \PV^{d_1,\bullet}$, $\im\del\subset\PV^{d_2,\bullet}$ such that $d_1+d_2\neq d-1$. However, it seems like these two choices of splittings are favored in a sense (articulated below). 
\end{itemize}
\end{rmk}
\fi

\section{Dimensional Reduction}
Let's consider the $11$-dimensional theory on the manifold
\[
\R\times\C^\times\times\C^4 .
\]

The fields decompose as follows.
For the ``base'' direction, fields we labeled by $\mu$, we have
\[
\mu = \begin{pmatrix} \mu_{01} \\ \mu_{10} \end{pmatrix} + u \mu_{11d}^0 \in \left (\begin{aligned} & \PV^{0,\bullet}(\C^\times)\otimes\PV^{1,\bullet}(\C^4) \\ & \PV^{1,\bullet}(\C^\times)\otimes\PV^{0,\bullet}(\C^4)\end{aligned} \to u \PV^{0,\bullet}(\C^\times\times\C^4)\right ) \Hat{\otimes} \Omega^\bullet(\R) [1] .
\]

For the ``fiber'' direction, fields we labeled by $\gamma$, we have
\[
\gamma = u^{-1} \gamma_{11d}^0 + \begin{pmatrix} \gamma_{01} \\ \gamma_{10} \end{pmatrix} \in \left (u^{-1} \Omega^{0,\bullet}(\C^\times\times\C^4) \to \begin{aligned} & \Omega^{0,\bullet}(\C^\times)\otimes\Omega^{1,\bullet}(\C^4)\\ & \Omega^{1,\bullet}(\C^\times)\otimes\Omega^{0,\bullet}(\C^4)\end{aligned} \right ) \Hat{\otimes} \Omega^\bullet(\R).
\]

We consider the dimensional reduction of the theory along the circle $S^1 \subset \mb C^\times$.
The dimensional reduction has the affect of replacing $\PV^{i,\bu}(\mb C^\times)$ and $\Omega^{1,\bu}(\mb C^\times)$ by the de Rham along the radial direction in $\mb C^\times$, see Proposition 1.59 of \cite{ESW}. 
Denote the space of fields of the dimensional reduction by $\oint_{S^1} \mc E_{11d}$. 

We will use the same symbols for the fields in the dimensionally reduced ten-dimensional theory: for the dimensionally reduced $\mu$-fields we have
\[
\mu = \begin{pmatrix} \mu_{01} \\ \mu_{10} \end{pmatrix} + u \mu_{11d}^0 \in \left (\begin{aligned} & \PV^{1,\bullet}(\C^4) \\ & \PV^{0,\bullet}(\C^4)\end{aligned} \to u \PV^{0,\bullet}(\C^4)\right ) \Hat{\otimes} \Omega^\bullet(\R^2) [1] .
\]
and for the dimensionally reduced $\gamma$-fields we have
\[
\gamma = u^{-1} \gamma_{11d}^0 + \begin{pmatrix} \gamma_{01} \\ \gamma_{10} \end{pmatrix} \in \left (u^{-1} \Omega^{0,\bullet}(\C^4) \to \begin{aligned} & \Omega^{1,\bullet}(\C^4)\\ & \Omega^{0,\bullet}(\C^4)\end{aligned} \right ) \Hat{\otimes} \Omega^\bullet(\R^2).
\]

So far, we have only described how the fields of the $11$-dimensional theory behave upon dimensional reduction. 
One can rename the fields of the dimensionally reduced theory to match precisely with the fields of the conjectural minimal twist of  Type IIA supergravity with potentials $\mc E_{\rm pot}(\mb C^4) \Hat{\otimes} \Omega^\bu(\mb R^2)$. 
In fact, we have the following stronger result, which identifies the interacting BV theories of the dimensionally reduced theory with the conjectural twist of Type IIA supergravity with potentials. 

\begin{prop}
There is an equivalence of interacting classical BV theories on $\mb C^2 \times \mb R^2$
\[
\Psi : \oint_{S^1} \mc E_{11d} \to \mc E_{\rm pot} (\CC^4) \Hat{\otimes} \Omega^\bu(\mb R^2) 
\] 
defined on the fields by
\[
\arraycolsep=1.4pt\def\arraystretch{2.2}
\begin{array}{cccccccc}
\Psi (\mu_{01}) & = & \mu_{01} & \in & \PV^{1,\bu}(\mb C^4) \Hat{\otimes} \Omega^\bu(\mb R^2) \\
\Psi (\mu_{10}) & = & \Omega^{-1}_{\mb C^4} \vee \mu_{10} & \in & \PV^{4,\bu}(\mb C^4) \Hat{\otimes} \Omega^\bu(\mb R^2) \\ 
\Psi(u \mu^0_{11d}) & = & u \mu^0_{11d} & \in & u \PV^{0,\bu} (\CC^4) \Hat{\otimes} \Omega^\bu (\RR^4) \\ 
\Psi (u^{-1} \gamma_{11d}^0) & = & u^{-1} \Omega^{-1}_{\CC^4} \vee \gamma_{11d}^0 & \in & u^{-1} \PV^{4,\bu}(\CC^4) \Hat{\otimes} \Omega^\bu(\RR^2) \\ \Psi(\gamma_{01}) &  = & \Omega^{-1}_{\CC^4} \vee \gamma_{01} & \in & \PV^{3,\bu}(\CC^4) \Hat{\otimes} \Omega^\bu(\RR^2) \\
\Psi (\gamma_{10}) & = & \gamma_{10} & \in & \PV^{0,\bu}(\CC^4) \Hat{\otimes} \Omega^\bu(\RR^2) .
\end{array} .
\]
\end{prop}  

%\[
%\gamma=\gamma_{01}+\gamma_{10}\in \Omega^\bullet(\R)\otimes(\Omega^{0,\bullet}(\C^\times)\otimes\Omega^{1,\bullet}(\C^4)\oplus\Omega^{1,\bullet}(\C^\times)\otimes\Omega^{0,\bullet}(\C^4)) .
%\]

%
%\begin{prop}
%There is a homomorphism of $L_\infty$-algebras from the $\delbar_{\C^\times}$-cohomology of M theory on $\R\times\C^\times\times\C$ to $\Omega^\bullet(\R^2)\otimes\mc E_{Pot}$ given by
%\begin{itemize}
%\item $[\gamma_{10}]\mapsto\mu^0\in\PV^{0,\bullet}\subset\mc E_{Pot}$.
%  \item $[\mu_{01}]\mapsto\mu^1\in\PV^{1,\bullet}(\C^4)\subset\mc E_{Pot}$.
%\item $[\mu^{0}]\mapsto\mu^{c}\in\PV^{0, \bullet}(\C^{4})\subset\mc E_{Pot}$.
%\item $[\gamma_{01}]\mapsto\mu^3 = \gamma_{01}\vee\Omega^{-1}_{\C^4}\in\PV^{3,\bullet}\subset\mc E_{Pot}$ where $\Omega_{\C^4}$ denotes the holomorphic volume form on $\C^4$.
%\item $[\mu_{10}]\mapsto\mu^4 =\mu_{10}\vee\Omega^{-1}_{\C^4}\in\PV^{4,\bullet}\subset\mc E_{Pot}$ .
%
%
%\end{itemize}
%preserving the relevant pairings.
%\end{prop}

That is, the reduction of the holomorphic M theory on a holomorphic circle should be the $\SU(4)$ invariant twist of $\IIA$.

%\begin{rmk}
%Note that the field $\mu^{c}\in \mc E_{\mr Pot}$ is nonpropating. It must pair against a field in $\PV^{3}$. In order for the field $\mu^{0}$ in M-theory to propagate, it must pair against a field $\gamma^{0}\in \Omega^{0,\bullet}$ which is the ghost for the gauge transformation for $\gamma^{1}$. This suggests an enlargement of $\mc E_{\mr Pot}$ where we add an additional field in $\PV^{3}$ which will be the image of the ghost of $\gamma^{1}$ in M-theory under dimensional reduction.
%\end{rmk}

%\surya{ When we discussewd dimensional reduction on 9/22 over Zoom, I accidentally switched the roles of $\gamma_{10}$ and $\mu_{10}$. I realized this was wrong - the resulting map cannot be a Lie map. Indeed, there is certainly a nontrivial bracket between $\gamma_{10}$ and $\gamma_{10}$ in M-theory, from expanding the $\gamma\del\gamma\del\gamma$ term. However, there is no nontrivial bracket between $\PV^{3,\bullet}$ and $\PV^{4,\bullet}$ in $\mc E_{\rm Pot}$ }

\section{Compactification to five dimensions} 

Now, consider placing the $11$-dimensional theory on the manifold
\[
X \times Y \times \mb R 
\]
where $X$ is compact Calabi--Yau three-fold and $Y$ is a Calabi--Yau surface.
We consider the compactification of the theory along the projection map 
\[
\pi : X \times Y \times \mb R \to Y \times \mb R .
\]
Recall, the space of fields of the $11$-dimensional theory is
\begin{equation}\label{eqn:cotangent1}
T^* [-1] \bigg(\PV^{\leq 1,\bu}(X \times Y) \Hat{\otimes} \Omega^\bu(\mb R) [1] \bigg) .
\end{equation}

Since $X$ is compact and Calabi--Yau, we have a sequence of quasi-isomorphisms 
\[
\PV^{j, \bu}(X) \cong_{\Omega_X} \Omega^{3-j, \bu} (X) \simeq H^{3-j, \bu}(X) .
\]
The first isomorphism is simply contraction with the Calabi--Yau form $\Omega_X \in \Omega^{3,hol}(X)$ and the second quasi-isomorphism follows from formality of $X$. 

In particular, at the level of sheaves one has a quasi-isomorphism
\[
\pi_* \bigg(\PV^{\leq 1,\bu}(X \times Y) \Hat{\otimes} \Omega^\bu(\mb R) [1] \bigg) \simeq H^{\geq 2, \bu}(X) \otimes \PV^{\leq 1, \bu} (Y) \Hat{\otimes} \Omega^\bu (\mb R) [1] .
\]
This describes the ``base" direction, fields we labeled by $\mu$, of the space of fields (\ref{eqn:cotangent1}) upon compactification. 

We extract the piece of the above sheaf which involves the top cohomology $H^{3,3}(X) \cong H^6(X) \cong \mb C$. 
This is the sheaf
\begin{equation}\label{eqn:pair1}
\PV^{\leq 1, \bu}(Y) \Hat{\otimes} \Omega^\bu (\mb R) [1] .
\end{equation}
(Recall, we are only working with sheaves of $\mb Z/2$-graded cochain complexes.)

Likewise, the ``fiber" direction of (\ref{eqn:cotangent1}) becomes, after compactification:
\[
\pi_* \bigg(\Omega^{\leq 1, \bu}(X \times Y) \Hat{\otimes} \Omega^\bu(\mb R) \bigg) \simeq H^{\leq 1, \bu}(X) \otimes \Omega^{\leq 1, \bu} (X) \Hat{\otimes} \Omega^\bu (\mb R)  .
\]
Under the BV pairing, the piece of this sheaf of cochain complexes with (\ref{eqn:pair1}) is the part involving $H^0(X) \cong \mb C$.
This is precisely
\[
\Omega^{\leq 1}(X) \Hat{\otimes} \Omega^\bu (\mb R) .
\] 



\end{document}
