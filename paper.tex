% Created 2020-10-14 Wed 14:51
% Intended LaTeX compiler: pdflatex
\documentclass[11pt]{amsart}
\usepackage[utf8]{inputenc}
\usepackage[T1]{fontenc}
\usepackage{graphicx}
\usepackage{grffile}
\usepackage{longtable}
\usepackage{wrapfig}
\usepackage{rotating}
\usepackage[normalem]{ulem}
\usepackage{amsmath}
\usepackage{textcomp}
\usepackage{amssymb}
\usepackage{capt-of}
\usepackage{hyperref}
\usepackage{amsfonts}
\usepackage{macros-mtheory}

\author{Surya and Brian}
\date{\today}
\title{Twisted \(M\)-theory and its perturbative quantization.}
\hypersetup{
 pdfauthor={Surya and Brian},
 pdftitle={Twisted \(M\)-theory and its perturbative quantization.},
 pdfkeywords={},
 pdfsubject={},
 pdfcreator={Emacs 27.1 (Org mode 9.4)}, 
 pdflang={English}}

\def\pv{{\rm PV}}
\def\PV{{\rm PV}}
\def\T{{\rm T}}

\begin{document}

\maketitle
\tableofcontents


\section{Definition of twisted supergravity}

\label{sec:org24653eb}
\subsection{The classical BV theory}
\label{sec:org5c9d94b}
In this section we define the central theory of study, within the Batalin--Vilkovisky formalism.
The theory will be defined on any eleven-dimensional manifold of the form $X \times L$ where $X$ is a Calabi--Yau five-fold with volume form $\Omega$ and $L$ is a smooth oriented one-manifold.

One of the fundamental geometric objects will be the sheaf of divergence-free holomorphic vector fields on $X$. 
We will utilize a convenient semi-free resolution of this sheaf in the category of $C^\infty$-modules. 

If $j$ is an integer, we let $\Omega^{0,j}(X, V)$ denote the space of Dolbeault forms of type $j$ on with values in $V$.
The $\dbar$ operator for $V$ is of the form $\dbar : \Omega^{0,j}(X, V)\to \Omega^{0,j+1}(X)$.
This operator defines the Dolbeault complex of $V$
\[
  \Omega^{0,\bu}(X, V) = \left(\Omega^{0,j}(X, V)[-j] \; , \; \dbar\right)
\]
which is a free resolution for the sheaf of holomorphic sections of $V$.
Applying this to the holomorphic tangent bundle $V = \T_X$, we obtain a resolution of the sheaf of holomorphic vector fields. 

%Denote by $TX$ the holomorphic tangent bundle of $X$.
%Then, the Lie bracket of holomorphic vector fields extends to a Lie bracket on the Dolbeault $\Omega^{0,\bu}(X, TX)$ thus endowing it with the structure of a dg Lie algebra.
%
%By definition, the Dolbeault complex for the trivial bundle $\Omega^{0,\bu}(X)$ is a dg module for the dg Lie algebra $\Omega^{0,\bu}(X, TX)$.
%We can thus consider the following semi-direct product dg Lie algebra
%\beqn\label{eqn:div1}
%  \Omega^{0,\bu}(X, TX) \ltimes \Omega^{0,\bu}(X) [-1] .
%\eeqn
%Notice that $\Omega^{0,\bu}(X)$ has been shifted to a seemingly unnatural cohomological degree, with $(0,j)$-forms sitting in degree $j+1$.
%We will see the reason for this shift momentarily. 

Now, suppose $X$ is a Calabi--Yau manifold with holomorphic volume form $\Omega$.
The holomorphic divergence operator extends to the Dolbeault complex to take the form
\[
  \partial_\Omega : \Omega^{0,\bu}(X, \T_X) \to \Omega^{0,\bu}(X)
\]
and is defined by the formula
\[
  \partial_\Omega (\mu) \wedge \Omega = L_\mu (\Omega)
\]
where, on the right hand side, we mean the Lie derivative of $\Omega$ with respect to $\mu$.

Using the divergence operator, we define the double complex of sheaves
\[
\Omega^{0,\bu}(X, \T_X) \xto{\partial_\Omega} \Omega^{0,\bu}(X) [-1] 
\]
where the horizontal differential is the divergence operator and the vertical differential is the $\dbar$ operator (which we have left implicit). 
We denote the totalization of this double complex of sheaves $\cS$. 

It is immediate to check that $\cS$ provides a free resolution of the sheaf of holomorphic divergence-free vector fields on $X$.
Furthermore, there is a natural Lie bracket on $\cS$ given by the Schouten--Nijenhous bracket $\{-,-\}_{\rm NS}$ which is compatible with the usual Lie bracket of holomorphic vector fields. 
In other words, $\cS$ is a sheaf of dg Lie algebras and the inclusion of holomorphic divergence-free vector fields is a quasi-isomorphism of sheaves of Lie algebras.

%Using the holomorphic divergence operator $\partial_\Omega$ we can consider the following deformation of the dg Lie algebra (\ref{eqn:div1}).
%The Lie bracket remains the same but the differential is deformed from $\dbar$ to $\dbar + \partial_\Omega$.
%Here, $\partial_{\Omega}$ denotes the holomorphic divergence operator thought of as a degree one operator from $\Omega^{0,\bu}(X, TX)$ to $\Omega^{0,\bu}(X)$.
%We will denote the resulting dg Lie algebra by $\cS(X)$.



%We will mostly consider the sub dg Lie algebra of $(\ref{eqn:pvlie1})$ consisting of the polyvector fields of type $i\leq 1$:
%\beqn\label{eqn:pvlie2}
%  \pv^{\leq 1, \bu} (X)[1] =
%\eeqn


%We note that this deformation is only in the $\ZZ/2$-graded sense.

%For $X$ a Calabi--Yau manifold let $\cS^{\bu,\bu}(X)$ be the $\ZZ/2$-graded dg Lie algebra whose underlying $\ZZ/2$-graded vector space is $\pv^{\bu,\bu}(X)[1]$ (with gradings defined modulo $2$), whose differential is
%\beqn\label{differential}
%\dbar + \partial_{\Omega}
%\eeqn
%and the bracket is the Nijenhuis--Schouten bracket $\{\cdot,\cdot\}_{\rm NS}$.
%We notice that in the original $\ZZ$-grading on $\pv^{\bu,\bu}(X)$ the operator $\partial_{\Omega}$ has cohomological degree $-1$, which is why the above deformation is only defined in a $\ZZ/2$-graded sense.

%The differential (\ref{differential}) clearly preserves the subspace $\pv^{\leq 1, \bu}(X)$.
%We let $\cS^{\leq 1,\bu}(X) \subset \cS^{\bu,\bu}(X)$ denote the $\ZZ/2$-graded sub dg Lie algebra consisting of polyvector fields of type $i\leq 1$ equipped with the differential $\dbar + \partial_{\Omega}$.

%\begin{rmk}
%  There is a way to regrade $\cS^{\leq 1,\bu}(X)$ while keeping the differential the same in such a way that it has the structure of a $\ZZ$-graded dg Lie algebra.
%  Explicitly, this is
%  \[
%    \begin{array}{ccccc}
%      \ul{0} & & \ul{1} \\
%      \pv^{1,\bu}(X) &\xto{\partial_{\Omega}} & \pv^{0,\bu}(X) .
%    \end{array}
%  \]
%  This is a $\ZZ$-graded dg Lie algebra which agrees with $\cS^{\leq 1, \bu}(X)$ as a $\ZZ/2$- graded dg Lie algebra.
%  This dg Lie algebra is a free resolution for the Lie algebra of holomorphic divergence-free vector fields on $X$.
%  That is, holomorphic vector fields which preserve the holomorphic volume form $\Omega$.
% \end{rmk}


The space of differential forms on any smooth manifold $L$ is a commutative dg algebra $\Omega^{\bu}(L) = \left(\oplus_{k}\Omega^{k}(L)[-k], \d_{\dR}\right)$.
By taking the tensor product with the dg Lie algebra $\cS(X)$ we obtain the dg Lie algebra
\beqn\label{localLie}
  \cS(X) \; \Hat{\otimes} \; \Omega^{\bu}(L)
\eeqn
where $X$ is a Calabi--Yau manifold and $L$ is a smooth manifold.
The differential in this dg Lie algebra is of the form $\dbar \otimes 1 + \partial_\Omega \otimes 1 + 1 \otimes \d_{\rm dR}$, which we will abbreviate by $\dbar + \partial_\Omega + \d_{\rm dR}$.
The bracket is defined using the bracket on $\cS(X)$ together with the wedge product of forms on $\Omega^\bu(L)$ by the formula
\[
  [\mu \otimes \alpha, \nu \otimes \beta] = \{\mu, \nu\}_{\rm NS} \otimes (\alpha \wedge \beta) .
\]


We remark that the structure maps of this Lie algebra, namely the differential and bracket, are given by differential and bidifferential operators, respectively.
In this sense, it fits the definition of a {\em local dg Lie algebra} on the product manifold $X \times L$, see Definition ?? \ref{CG2}.

Associated to any local dg Lie algebra is a classical field theory in the BV formalism that one refers to as ``$BF$-theory'' and it is defined as follows.
Denote the sheaf of sections of the local dg Lie algebra by $\cL$.
The fields of the associated BV theory are pairs
\[
  (A, B) \in \cL[1] \oplus \cL^{!}[-2]
\]
and the action functional reads $\int B F_{A} = \int B (\d_{\cL}A + \frac12 [A,A])$.
Here $\d_{\cL}$ denotes the differential on $\cL$ and $[\cdot,\cdot]$ is the Lie bracket.
Notice that the equations of motion for the $A$-field is simply the Maurer--Cartan equation $\d_\cL A + \frac12 [A,A] = 0$.
For the $B$-field the equations of motion read $\d_\cL B + [A, B] = 0$.

For the case at hand, we consider the $BF$ theory associated to the local dg Lie algebra  $\cS(X) \otimes \Omega^\bu(L)$ defined on the product $X \times L$ where $X$ is a Calabi--Yau manifold and $L$ is a smooth manifold.
Denote by $d \in \ZZ$ the sum $\dim_{\CC}(X) + \dim_{\RR}(L)$.

The fields decompose as
\[
(\mu, \gamma) = (\mu^1 + \mu^0, \gamma^0 + \gamma^1)
\]
where
\[
  \begin{array}{ccl}
    \mu = \mu^1+\mu^0 & \in & \left(\Omega^{0,\bu}(X, \T_X)[1] \oplus \Omega^{0,\bu}(X) \right) \; \Hat{\otimes} \; \Omega^{\bu}(L) \\
    \gamma = \gamma^0 + \gamma^1 & \in & \left(\Omega^{0,\bu}(X)[-1] \oplus \Omega^{1,\bu}(X)[-2] \right) \; \Hat{\otimes}\; \Omega^{\bu}(L)[d] .
  \end{array}
\]
Notice that what we called the ``$A$''-field in the general definition of $BF$ theory we are now calling $\mu$ and what we called the ``$B$''-field we are now calling $\gamma$.
The action of this version of $BF$ theory simply reads 
\[
\int_{X \times L} \left[\gamma \vee \left(\dbar \mu + \partial_\Omega \mu + \d_{\dR} \mu + \frac12 [\mu, \mu] \right) \right] \wedge \Omega .
\]
This theory is defined on any geometry $X \times L$ where $X$ is a Calabi--Yau manifold and $L$ is a one-dimensional smooth manifold.
There is a deformation of this theory that we will be most concerned with.

%By force of habit, we will still refer to the theory with these particular fields as $BF$-theory.
%Notice that while the definition of $\mu$  of the dimension, the field $\gamma$ 

%This is in order for the BV antibracket to be of the appropriate degree.
%Additionally, we have utilized the holomorphic volume form in the definition of the field $\gamma$.
%Indeed, in this presentation the BV antipairing on fields reads
%\[
%  \int_{X \times L} (\mu \vee \gamma) \wedge \Omega = \int_{X \times L} (\mu^1 \vee \gamma^1 + \mu^0 \gamma^0) \wedge \Omega .
%\]

\subsubsection{Chern--Simons deformation}
The theory we have just described admits a deformation by a higher Chern--Simons term.
We explain how this relates to the familiar Chern--Simons term in eleven-dimensional supergravity in Section \ref{sec:evidence}.

We will be mostly interested in the case that $X$ is a Calabi--Yau five-fold and $L$ is a real one-dimensional manifold.
In the notation of the last section, $d = 6$, so the field $\gamma$ takes the form
\[
  \gamma \in \left(\Omega^{0,\bu}(X)[6] \oplus \Omega^{1,\bu}(X)[5] \right) \otimes \Omega^\bu(L).
\]

What is important for us is not simply BF theory for the dg Lie algebra $\cS(X) \otimes \Omega^\bu(L)$, but a certain deformation of it.
We consider a theory whose fields are that of the BF theory but whose action functional is of the form $S_{\rm BF} + J$, where $J$ is a local functional that purely depends on the fields $\gamma$.
It is defined by the following formula
\[
J(\gamma) = \frac16 \int_{X\times L} \gamma \partial \gamma \partial \gamma .
\]

First off, we notice an inconsistency in the cohomological degrees of the functionals $S_{\rm BF}$ and $J$.
Being a classical action functional $S_{\rm BF}$ is, of course, ghost degree zero.
However, we observe that $J$ is of ghost degree $+6$.
In order to make sense of the deformed theory, therefore, we must consider the space of fields as a $\ZZ/2$-graded object rather than a $\ZZ$-graded one.

So, in order to make sense of the action functional $S_{\rm BF} + J$ as a homogenously even functional, the $\ZZ /2$-graded space of fields takes the form
\[
  \begin{array}{ccl}
    \mu = \mu^1+\mu^0 & \in & \left(\pv^{1,\bu}(X)[1] \oplus \pv^{0,\bu}(X) \right) \; \Hat{\otimes} \; \Omega^{\bu}(L) \\
    \gamma = \gamma^0+ \gamma^1 & \in & \left(\Omega^{0,\bu}(X)[1] \oplus \Omega^{1,\bu}(X) \right) \; \Hat{\otimes}\; \Omega^{\bu}(L) .
  \end{array}
\]
Here, we have used the following convention for $\ZZ/2$-graded complexes.
If $V$ is a $\ZZ/2$-graded vector space with even part $V^0$ and odd part $V^1$, then $V[1]$ is a $\ZZ/2$-graded vector space with even part $V^1$ and odd part $V^0$.

With the issue of gradings out of the way, we turn to the question of whether this is a well-defined classical BV theory.
In order for this to be true, we must see that the total action functional $S_{\rm BF} + J$ satisfies the classical master equation
\[
  \{S_{\rm BF} + J , S_{\rm BF} + J\} = 0 .
\]
Of course, we know the undeformed BF-theory satisfies the classical master equation $\{S_{\rm BF}, S_{\rm BF}\} = 0$.
Additionally, since $\gamma$ does not pair with itself under the antibracket, we also see that $\{J,J\}=0$.
What remains is to see that $J$ is closed for the differential $\{S_{\rm BF}, \cdot\}$.

\begin{lem}
The functional $J = \frac16\int \gamma \partial \gamma \partial \gamma$ is closed for the differential $\{S_{\rm BF},\cdot\}$.
In particular, $S_{\rm BF} + J$ is a well-defined classical BV action.
\end{lem}

\begin{proof}
It is clear that $J$ is closed for the part of the differential arising from the kinetic term $\{S_{\rm free},\cdot\}$.
It remains to see that\brian{finish}
\end{proof}

\begin{rmk}
  We have actually shown something slightly stronger here.
  Namely, for any constant $c$ the functional $S_{\rm BF} + c J$ satisfies the classical master equation.
  This dependence on a coupling constant will be important later on in this paper.
\end{rmk}

\begin{dfn}\label{dfn:11dtheory}
Let $X$ be a Calabi--Yau five-fold and $L$ a one-dimensional smooth manifold.
Define the $\ZZ/2$-graded BV theory $\cT_{X \times L}$ (or simply just $\cT$) on
\[
  X \times L
\]
to be the deformation of BF theory for the $\ZZ/2$-graded local Lie algebra $\cS(X) \Hat{\otimes} \Omega^{\bu}(L)$ by the local functional $J (\gamma) = \frac16 \int_{X \times L} \gamma \partial \gamma \partial \gamma.$
\end{dfn}

%Unpacking this a bit, we see that the BV action functional for the theory $\cT$ splits as a sum
%\begin{equation}\label{11daction}
%  S_{\rm BF} + J
%\end{equation}
%where $J$ is as in the definition above and $S_{BF}$ is the action functional for the underlying BF theory which takes the form
%\[
%  S_{\rm BF} = \int \gamma \vdash \left(\dbar + \d_{\rm dR} + \partial_{\Omega}\right) \mu \wedge \Omega + \frac12 \int \gamma \vdash \{\mu,\mu\}_{\rm NS} \wedge \Omega .
%\]
%We will furthermore decompose $S_{\rm BF} = S_{\rm free}+I_{\rm BF}$ into its free and interacting pieces.

\subsubsection{A generalization of the theory}

We briefly discuss a generalization of this deformation of BF theory for the dg Lie algebra $\cS(X) \otimes \Omega^\bu(L)$ when $X$ is any {\em odd dimensional} Calabi--Yau manifold of dimension at least three and $L$ is a smooth one-dimensional manifold.

Suppose $X$ is a Calabi--Yau manifold of complex dimension $2m+1$ where $m > 0$.
Then, the fields of BF theory for the dg Lie algebra $\cS(X) \otimes \Omega^{\bu}(L)$ are of the form
\[
  \begin{array}{ccl}
    \mu = \mu^1+\mu^0 & \in & \left(\pv^{1,\bu}(X)[1] \oplus \pv^{0,\bu}(X) \right) \; \Hat{\otimes} \; \Omega^{\bu}(L) \\
    \gamma = \gamma^0 + \gamma^1 & \in & \left(\Omega^{0,\bu}(X) \oplus \Omega^{1,\bu}(X)[-1] \right) \; \Hat{\otimes}\; \Omega^{\bu}(L)[depends on m] .
  \end{array}
\]

The deformation we consider is of the form
\[
  J_m (\gamma) = \frac{1}{m+1} \int_{X \times L} \gamma \underbrace{\partial \gamma \cdots \partial \gamma}_{m} .
\]
That is, this deformation is $(m+1)$-linear and contains $m$ holomorphic derivatives.
As in the $m=2$ ($X$ is a five-fold) case that we discussed above, we see that $J_m$ is closed for the differential $\{S_{\rm BF},\cdot\}$ defining the BF theory.

\begin{lem}
  The functional $J_m$ is closed for the differential $\{S_{\rm BF}, \cdot \}$.
\end{lem}

A simple count reveals that the ghost degree of the functional $J_m$ is $2m^2 - 2$.
In particular, when $m \ne 1$, the above lemma implies that the action functional $S_{\rm BF} + J_m$ only defines a $\ZZ/2$-graded classical BV theory.

\subsubsection{The case of a CY three-fold}

When $m=1$, hence $X$ is a CY three-fold, the action $S_{\rm BF} + J_1$ actually defines a $\ZZ$-graded BV theory.
Another feature of the case $m=1$ is that the deformed theory is actually a {\em topological} field theory.
It is instructive to spell out this case in more detail.

Using the isomorphism
\[
 (-) \vee \Omega : \PV^{i,\bu} (X) \cong \Omega^{3-i, \bu}(X)
\]
we can identify the $\mu$-fields of the theory in terms of differential forms
\[
  \alpha^2 + \alpha^3 \in \left(\Omega^{2,\bu}(X)[1] \oplus \Omega^{3,\bu}(X)\right) \; \Hat{\otimes} \; \Omega^\bu(L)
\]
by the formula $\alpha^{3-i} = \mu^{i} \vee \Omega$, for $i=0,1$.
We will also write $\alpha^{j} = \gamma^{j}$ for $j=0,1$, so the remaining fields read
\[
  \alpha^0 + \alpha^1 \in \left(\Omega^{0,\bu}(X)[3] \oplus \Omega^{1,\bu}(X)[2] \right) \; \Hat{\otimes} \; \Omega^\bu (L).
\]
Let us write the theory as $S_{\rm BF} + c J_1$ where $c$ is a constant.
We denote by $\cT_c [X \times L]$ the resulting BV theory which depends on this coupling constant.
Also, let $\cM_c[X]$ denote the moduli space of solutions on the Calabi--Yau three-fold.
Using this decomposition of the fields we can write the underlying {\em free} theory by the following action
\[
  \int_{X\times L} \left(\alpha^0 (\dbar + \d_{\rm dR}) \alpha^3 + \alpha^1 (\dbar + \d_{\rm dR}) \alpha^2 \right) + \int_{X \times L} \alpha^0 \partial \alpha^2 + \frac{c}{2} \int_{X \times L} \alpha^1 \partial \alpha^1
\]
Here, as above, $\partial$ denotes the holomorphic de Rham operator on $X$.

Now, for $c \ne 0$, we observe that this free action is equivalent to abelian seven-dimensional Chern--Simons theory on $X \times L$.
Indeed, by the above formula the linearized BV complex of the theory is given by the following cochain complex
\[
  \bigg(\Omega^{\bu, \bu} (X) \; \Hat{\otimes} \; \Omega^{\bu}(L) [3] \; , \; \dbar + \Hat{\partial} + \d_{\rm dR} + c \partial_{\Omega^{1} \to \Omega^{2}} \bigg)
\]
Here, $\Hat{\partial}$ denotes the components of the holomorphic de Rham differential $\Omega^{0,\bu}(X) \to \Omega^{1,\bu}(X)$ and $\Omega^{2,\bu}(X) \to \Omega^{3,\bu}(X)$.
Also, $\partial_{\Omega^1 \to \Omega^2}$ denotes the component $\Omega^{1,\bu}(X) \to \Omega^{2,\bu}(X)$.
When $c \ne 0$ we see that this cochain complex is isomorphic to the full (shifted) de Rham complex $\Omega^\bu(X \times L)[3]$.

In the BV formalism, seven-dimensional Chern--Simons theory is described by the fields $\alpha^i \in \Omega^{i,\bu}(X) \Hat{\otimes} \Omega^\bu(L)[3-i]$, $i=0, \ldots 3$ which we can now decompose as a de Rham type field $C^\bu \in \Omega^\bu(X \times L)[3]$ so that $C^j$ has total de Rham degree $j$.
The BV antipairing is the wedge and integrate pairing $\int C \wedge C'$ and the action reads
\[
  \frac{1}{2} \int_{X \times L} C \d C,
\]
where $\d$ is the full de Rham differential on $X \times L$.
Sometimes, this is referred to as ``three-form Chern--Simons theory'' since its fundamental field is a three-form $C^3$ and the equations of motion read $\d C^3 = 0$.

We have seen that when $c \ne 0$, the free theory underlying $\cT_c[X \times L]$ is equivalent to seven-dimensional Chern--Simons theory.
In particular, the moduli space of solutions $\cM_c[X]$, for $c \ne 0$, has as its free limit the moduli space of circle $3$-bundles on $X$ with connection.

We can consider a different limit of $\cM_c[X]$, namely the one where $c \to 0$.
This is precisely the cotangent bundle to the moduli space of Calabi--Yau structures $\cM_{CY}[X]$ on the three-fold $X$:
\[
  \cM_0[X] = T^* \left(\cM_{\rm CY} [X]\right) .
\]

In summary, we have seen that $\cM_c[X]$ provides a roof of deformations between the moduli space of Calabi--Yau structures on $X$ and the moduli of circle $3$-bundles:
\[
  \begin{tikzcd}
    & \cM_c[X] \ar[dr, "c \ne 0"', "{\rm free\;limit}"] \ar[dl, "c \to 0"'] & \\ T^* \left(\cM_{\rm CY} [X] \right) & & \left\{ \begin{array}{ccc} \mbox{circle\;3-bundles} \\ {\rm on} \; X \end{array} \right\} .
  \end{tikzcd}
\]


\subsection{Equations of motion}
\label{sec:org2973a4d}
We return to the case of the theory $\cT$ on $X\times L$ where $X$ is a Calabi--Yau five-fold and $L$ is a one-manifold.
In this section we analyze the equations of motion for the eleven-dimensional theory and provide partial moduli theoretic interpretation of the solution space.

The action functional of the theory in Definition \ref{dfn:11dtheory} on $X \times L$ is of the form
\[
  \int_{X \times L} \gamma \vdash \left(\dbar + \d_{\rm dR} + \partial_{\Omega}\right) \mu \wedge \Omega + \frac12 \int_{X \times L} \gamma \vdash \{\mu,\mu\}_{\rm NS} \wedge \Omega + \frac{c}{3} \int_{X \times L} \gamma \partial \gamma \partial \gamma .
\]
Notice that just as in the last section we have included a coupling constant $c$ into the term which deforms the classical BF type theory.

From this, we read off the equations of motion as
\begin{equation}\label{eqn:eom}
  \begin{array}{rccc}
    \left(\dbar + \partial_\Omega + \d_{\rm dR} \right) \mu + \frac12 \{\mu, \mu\}_{\rm NS} + c \partial \gamma \partial \gamma \vdash \Omega & = & 0 \\
    \left(\dbar + \partial_\Omega + \d_{\rm dR} \right) \gamma + L_\mu (\gamma) & = & 0 .
  \end{array}
\end{equation}
Here, $L_\mu(\gamma)$ denotes the Lie derivative of $\gamma$ with respect to the vector field $\mu$.

On $L = \RR_t$ we proceed to describe the classical phase space of the model  which is the space of solutions to the above equations at $t = 0$.
Upon taking into account the gauge transformations, we will denote the resulting moduli space by $\cM_c[X]$.
When $X$ is compact, the BV pairing in the bulk eleven-dimensional theory induces a symplectic structure on this moduli space.

Notice that just as in the case of the phase space of the $7$-dimensional theory on a Calabi--Yau three-fold, at $c=0$ this moduli space is precisely the cotangent bundle to the moduli space of Calabi--Yau five-folds $\cM_0 [X] = T^*(\cM_{\rm CY}(X))$.

In the general case, we begin to decompose the two equations of motion above according to their form degree.
When $c = 0$, the field $\mu^{1,1}$ is a Beltrami differential which provides the parameter describing the deformation of complex structure on the Calabi--Yau $X$.
For $c \ne 0$, it makes an appearance in the following equation
\begin{equation}\label{eqn:mu11}
  \dbar \mu^{1,1} + \frac12 \{\mu^{1,1}, \mu^{1,1}\}_{\rm NS} + c \partial \gamma^{1,0} \partial \gamma^{1,2} \vdash \Omega = 0 .
\end{equation}
(Notice that the term $\partial \gamma^{1,1} \partial \gamma^{1,1} = 0$ by form type reasons.)

The field $\gamma^{1,0}$ also appears in the second class of equations of motion in (\ref{eqn:eom}) as
\[
  \dbar \gamma^{1,0} + \partial \gamma^{0,1} + L_{\mu^{1,1}} \gamma^{1,0} + L_{\mu^{1,0}} \gamma^{0,1} = 0 .
\]
We choose a gauge so that $\gamma^{0,1} = 0$, \brian{Why is this allowed.} so that this equation reads
\begin{equation}\label{eqn:gamma10}
  \dbar \gamma^{1,0} + L_{\mu^{1,1}} \gamma^{1,0} = 0 .
\end{equation}

Denote by $\beta^{2,0} = \partial \gamma^{1,0}$ the $(2,0)$-form obtained from $\gamma^{1,0}$ and consider the sequence of maps of holomorphic vector bundles
\[
  \begin{tikzcd}
    \wedge^{2,0} T^{*1,0} \ar[r, "\beta^{2,0} \wedge"] & \wedge^4 T_X^{*1,0} \ar[r, "\Omega", "\cong"'] & T^{1,0}_X .
  \end{tikzcd}
\]
The image of this map is a holomorphic distribution $F \subset T_X^{1,0}$.
It follows from the fact that $\partial \beta = 0$ that $F$ defines a holomorphic foliation on $X$.

Using the foliation $F$, the holomorphic tangent bundle splits as $T^{1,0} = F \oplus Q$ where $Q = T^{1,0}_X / F$ is the quotient bundle.
Using this splitting, we can split the components of $\mu^{1,1}$ according to
\[
  T^{1,0}_X \otimes T^{*0,1}_X = (F \otimes\Bar{F}^*) \oplus (F \otimes \Bar{Q}^*) \oplus (Q \otimes \Bar{F}^*) \oplus (Q \otimes \Bar{Q}^*) .
\]

There is a component $\mu_{Q\Bar{Q}}^{1,1}$ of $\mu^{1,1}$ which is a Beltrami differential for the quotient bundle $Q$.
At the level of the quotient bundle Equation (\ref{eqn:mu11}) becomes
\[
  \dbar \mu^{1,1}_{Q \Bar{Q}} + \frac12 \{\mu^{1,1}_{Q \Bar{Q}} , \mu^{1,1}_{Q \Bar{Q}}\}_{\rm NS} = 0 .
\]
In other words $\mu^{1,1}_{Q \Bar{Q}}$ is a Beltrami differential for this quotient bundle and hence determines a deformation of complex structure of $Q = T^{1,0}_X / F$.

So, after fixing the value of $\beta \partial \gamma^{1,0}$.
The remaining components of $\mu^{1,1}$ determine deformations of complex structure of the bundle $F \subset T^{1,0}_X$.
For instance, the $\mu^{1,1}_{F \Bar{F}}$ determines a deformation of the complex structure on $F$.
Furthermore, by Equation (\ref{eqn:gamma10}) we see that $\gamma^{1,0}$, hence $\beta = \partial \gamma^{1,0}$, is holomorphic for this deformed complex structure.


\subsection{Evidence for twisted supergravity}\label{sec:evidence}

\brian{state conjecture about twist.}

In parts of the remainder of the paper we will provide evidence, and consistency checks, that the eleven-dimensional theory that we have just defined is a candidate for the minimal twist of supergravity.
In this section, we discuss a more direct relationship by exhibiting the fields of the theory we have outlined as components of the supergravity multiplet that are expected to survive in the twist.

We recall that our eleven-dimensional theory is only $\ZZ/2$-graded.
This is consistent with the minimal twist of supergravity.
Indeed, since the $R$-symmetry group is trivial there is no way to regrade the twisted theory in such a way that the BRST differential and twisting supercharge are of homogenous $\ZZ$-degree.

The conjecture implies that the even fields of our eleven dimensional theory should correspond to certain components of the fundamental fields supergravity that survive the minimal twist.
The even components of the field $\mu$ and $\gamma$ are of the form
\[
  \mu^{0, i ; j} \; , \; \mu^{1, i ; j} \; , \; \gamma^{0, i ; j} \; , \; \gamma^{1, i ; j}
\]
where $i+j$ is an even integer and $i = 0,\ldots 5$ and $j=0,1$.
Here, as in the previous section, we have used the following index notation:
\[
  \begin{array}{ccccc}
    \mu^{k, i ; j} & \in & \pv^{k,i} (\CC^5) \; \Hat{\otimes} \; \Omega^j(\RR) \\
    \gamma^{k, i ; j} & \in & \Omega^{k,i}(\CC^5) \; \Hat{\otimes} \; \Omega^j(\RR).
  \end{array}
\]

Let us now turn to the field content of eleven-dimensional supergravity.
In Euclidean signature, in the flat background, the fundamental fields of eleven-dimensional supergravity are
\[
  \begin{array}{ccll}
    e & \in & \Omega^1(\RR^{11}) \otimes \CC^{11} & \mbox{vielbien} \\
    \omega & \in & \Omega^1(\RR^{11}) \otimes \fs \fo(11) & \mbox{spin connection} \\
    C & \in & \Omega^3(\RR^{11}) & \mbox{supergravity $3$-form} \\
    \psi & \in & \Pi \Omega^1(\RR^{11}) \otimes S & \mbox{gravitino} .
  \end{array}
\]
We note that the fields above have been complexified from the usual presentation of supergravity in terms of real fields.
Here, $S$ denotes the $128$ dimensional complex spin representation of $\fs \fo(11)$.

\subsubsection{Residual supersymmetry}

\def\m2{\mathfrak{m}2\mathfrak{brane}}
\def\susy{\mathfrak{susy}}
\def\so{\mathfrak{so}}
\def\siso{\mathfrak{siso}}
\def\fsl{\mathfrak{sl}}

Recall, the (complexified) eleven-dimensional supertranslation algebra is a complex super Lie algebra of the form
\[
  \ft_{11d} = V \oplus \Pi S
\]
where $V$ is the fundamental representation of $\so(11 , \CC)$ and $S$ is the irreducible (complex) spin representation.
The only non-trivial Lie bracket is
\[
  [Q, Q'] = \Gamma_{\Omega^1}(Q \otimes Q')
\]
where
\[
  \Gamma_{\Omega^1} : \Sym^2(S) \to V 
\]
is the unique $\so(11, \CC)$-equivariant \brian{finish}
The super Poincar\'{e} algebra is
\[
  \siso_{11d} = \so(11 , \CC) \ltimes \ft_{11d} .
\]

Introduce the cochain complex $\Omega^{\bu}(\RR^{11})$ of (complex valued) differential forms on $\RR^{11}$. 
Let
\[
  \Gamma_{\Omega^2} : \Sym^2(S) \to \wedge^2 V \subset \Omega^2(\RR^{11}) 
\]
be the \brian{finish}. 

\begin{dfn}
  The super dg Lie algebra $\m2$ is the central extension of $\siso_{11d}$ by the cocycle
  \[
    c_{M2} \in \clie^2\left(\siso_{11d} \; ; \; \Omega^\bu (\RR^{11})[2]\right)
  \]
  defined by the formula $c_{M2} (Q, Q') = \Gamma_{\Omega^2}(Q \otimes Q') \in \Omega^2(\RR^{11})$.
\end{dfn}

Notice that $\m2$ is a $\ZZ \times \ZZ/2$-graded Lie algebra and the differential is of bidegree $(1,0)$.

\begin{lem}
  Fix a holomorphic supercharge $Q \in S$ and let $\m2^Q$ be the $Q$-cohomology of $\m2$.
  There is an isomorphism of $\ZZ/2$-graded Lie algebras
  \[
    \m2^Q \simeq \fsl(5;\CC) \ltimes ???
  \]
\end{lem}

Consider now the eleven-dimensional theory $\cT = \cT_{\CC^5 \times \RR}$ defined on $\CC^5 \times \RR$. 
The BV action induces the structure of a dg Lie algebra on $\cT[1]$. 

\begin{prop}
The assignment
\brian{??} defines a map of $\ZZ/2$-graded dg Lie algebras 
\[
\m2 \to \cT[1] .
\]
In particular, the $Q$-twisted algebra $\m2^Q$ is a symmetry of eleven-dimensional theory on $\CC^5 \times \RR$. 
\end{prop}


\section{Quantization of holomorphic-topological theories}
\label{sec:org297a559}
\subsection{Effective renormalization}
\label{sec:orga5cd4ed}
\subsection{The \(11\)-dimensional theory}
\label{sec:org8b189fc}
\subsection{The moduli space of quantizations}
\label{sec:orgf278b4e}
\section{Relationship to the Type IIA string}
\label{sec:org355a726}
In the remainder of the paper, we wish to establish various consistency checks corroborating the claim that the 11d theory of interest describes a twist of 11d supergravity. In this section, we demonstrate that this claim is consistent with a conjectural description of a twist of Type IIA supergravity due to \cite{CLsugra}.

Before recalling this conjecture, we first fix some notation which is useful to describe the state space for the Type IIA string. \surya{move earlier?}
For $i,j$ integers define the space of $(i,j)$-polyvector fields to be
\[
  \pv^{i,j}(X) \define \Omega^{0,j}(X, \wedge^{i} TX)
\]
where $TX$ is the holormorphic tangent bundle.
Using the $\dbar$ operator for the holomorphic bundle $\wedge^{i}TX$ we obtain a cochain complex $\pv^{i,\bu}(X) = \left(\oplus_{j} \pv^{i,j}(X) [-j], \dbar \right)$ which provides a free resolution of the sheaf of holomorphic polyvector fields $\PV^{i}_{\rm hol}(X)$ of type $i$.

There is a bracket on the space of holomorphic polyvector fields called the Nijenhuis--Schouten bracket.
This bracket is defined using holomorphic pvdifferential operators, so extends to a bracket on the Dolbeault complex to define a bracket of the form
\[
  \{\cdot, \cdot\}_{\rm NS} \colon \pv^{i,j}(X) \times \pv^{k,\ell}(X) \to \PV^{i+k-1, j+\ell}(X) .
\]
This bracket endows the total complex
\beqn\label{eqn:pvlie1}
\pv^{\bu,\bu}(X)[1] = \left(\oplus_{i,j} \pv^{i,j}(X)[-i-j+1] , \dbar \right)
\eeqn
with the structure of a dg Lie algebra.
Here, we note that in this dg Lie algebra the space $\pv^{i,j}(X)$ lies in cohomological degree $i+j-1$.

When $X$ is Calabi--Yau of complex dimension $n$, the holomorphic volume form $\Omega$ defines an isomorphism
\[
  \vdash \Omega : \pv^{i,j}(X) \cong \Omega^{n-i, j}(X).
\]
In turn, the holomorphic de Rham operator $\partial : \Omega^{p, j}(X) \to \Omega^{p+1,j}$ defines a holomorphic differential operator
\[
  \partial_{\Omega} : \pv^{i,j}(X) \to \pv^{i-1,j}(X) .
\]
This is the holomorphic analog of the divergence operator with respect to $\Omega$.

There is compelling evidence \cite{CLsugra} for the existence of a twist of the Type IIA string on the ten-manifold $M \times X$ where $M$ is a real surface and $X$ is a Calabi--Yau four-fold.
Roughly, the twist behaves like the A-model topological string along $M$ and the $B$-model topological string along $X$.
The conjectural state space is
\[
  \Omega^\bu(M) \; \Hat{\otimes} \; \PV^{\bu, \bu} (X) .
\]
The linear BRST differential on this state space is given by $\d_{\rm dR} + \dbar$.
The fields of the closed string are given by the $S^1$-equivariant states, which is modeled \cite{CLbcov} by
\[
  \Omega^\bu(M) \; \Hat{\otimes} \; \PV^{\bu, \bu} (X) [\![u]\!]
\]
where $u$ is the equivariant parameter and the new linear BRST operator is $\d_{\rm dR} + \dbar + u \partial_\Omega$.

With this informal description of the twist of the Type IIA closed string in hand, one can deduce the following low energy limit of the description which gives a conjectural description of Type IIA supergravity.

\begin{conj}
  [Costello-Li \cite{CLsugra}]
  Let $(M,h_{M})$ be a Kahler surface and $(X, h_{X})$ be a Calabi-Yau 4-fold.
  Consider perturbative type IIA supergravity on $M\times X$ around a background where:
  \begin{itemize}
    \item the graviton is set to $h_{M}+ h_{X}$,
    \item the bosonic ghost for local supersymmetries is set to a covariantly constant square-zero spinor $Q$ that is invariant for the $\rm SU(4)$ subgroup of isometries preserving $X\subset M\times X$,
    \item all other background fields are set to zero.
  \end{itemize}
  In this background, the resulting theory is equivalent to the Kodaira--Spencer type theory whose fields are
  \[
    \Omega^{\bullet}(M) \; \Hat{\otimes} \; \PV^{\bullet, \bullet}(X)[\![u]\!]
  \]
  where $u$ is a parameter of cohomological degree $2$.
  \brian{this theory is only $\ZZ/2$-graded, right?}
  %$L_{\infty}$-structure given by $\ell_{1} = d\otimes 1+ 1\otimes (\dbar+t\partial)$, $\ell_{2} = \wedge\otimes \{-,-\}_{\rm NS}$ and Poisson kernel given by $(1\otimes \partial)\delta_{\Delta(X)}\delta_{\Delta (M)}$.
\end{conj}

Type II supergravity around such backgrounds where the bosonic ghost takes a nonzero VEV is what is referred to as {\em twisted supergravity} in \cite{CLsugra}.
In fact, the authors further conjecture that type IIA superstring theory on $M\times X$ around a particular Ramond-Ramond background is equivalent to a topological string theory given by the A-model on $M$ and the B-model on $X$.
This implies the above conjecture at the level of closed string field theory.
Indeed the above description of twisted supergravity is a combination of Kodaira--Spencer theory on $X$ with the zero-winding sector of Kahler gravity on $M$.

There is further evidence for this claim through studying the residual supersymmetry.
Namely, they construct an $L_{\infty}$ map from the $Q$-cohomology of the 10d $\cN = (1,1)$ algebra to the fields of the above theory \cite{CLsugra}, realizing the former as a collection of ghosts in the latter. This allows us to make sense of further twists of the above twist.

The goal of this section is to identify the above conjectural twist of IIA supergravity with an $S^{1}$ reduction of our 11d theory. In order to do so, we will need to introduce a slight modification of the Kodaira-Spencer theory. We turn to this in the next subsection.

\subsection{Kodaira--Spencer theory with potentials}
\label{KSPot}

One property of the description of the twist of type II supergravity in terms of Kodaira--Spencer theory is that the Ramond-Ramond fields of type II do not correspond to fundamental fields in Kodaira--Spencer theory. Rather it is certain components of the Ramond-Ramond field strengths that appear. However, in the identification of type IIA supergravity as the $S^{1}$ reduction of 11d supergravity, certain components of the C-field become components of the Ramond-Ramond 2-form. This suggests that in order to match the dimensional reduction of our 11d theory with a twist of type IIA, we must modify the description of the twist of type IIA to include certain components of Ramond-Ramond fields as potentials for those components of Ramond-Ramond field strengths that one finds in the twisted theory.

\subsubsection{Warm-up: four-dimensional Kodaira--Spencer theory}
\label{sec:org91dc4ca}
In the BV formalism, minimal Kodaira--Spencer theory on $X$ is a (degenerate) Poisson BV theory with space of fields given by
\[
\begin{tikzcd}
\ul{\rm odd} & \ul{\rm even} \\
 & \PV^{0,\bu} \\
 \PV^{1,\bu} \ar[r, "u \partial"] & u \PV^{0,\bu} .
\end{tikzcd}
\]
We will denote this sheaf of $\ZZ /2$-graded cochain complexes by $\cE_{\rm KS}$.

There is a local (dg) Lie algebra structure on the parity shifted object $\Pi \cE_{\rm KS}$.
The Lie bracket is defined using the Schouten-Nijenhuis bracket $[-,-]_{\rm NS}$ on polyvector fields and is given by the formula
\[
[u^k \alpha , u^\ell \beta] = u^{k+\ell} [\alpha, \beta]_{\rm NS} .
\]
where $k, \ell = 0,1$.
Together with the differential this equips the parity shifted sheaf of cochain complexes $\Pi \cE_{\rm KS}$ with the structure of a local dg Lie algebra.

The fields of minimal Kodaira--Spencer theory $\cE_{\rm KS}$ is equipped with an odd Poisson tensor defined by
\[
\Pi_{\rm KS} = (\partial \otimes 1) \delta_{\rm Diag} .
\]

We introduce another theory on the Calabi--Yau surface $X$ that we call minimal Kodaira--Spencer theory {\em with potentials}.
The underlying vector bundle is
\[
\begin{tikzcd}
\ul{\rm odd} & \ul{\rm even} \\
 & \PV^{0,\bu} \\
 \PV^{2,\bu}  .
\end{tikzcd}
\]
We will denote the resulting $\ZZ /2$-graded sheaf of cochain complexes by $\cE_{\rm Pot}$.

We interpret this as the theory of  ``potentials"  of minimal Kodaira--Spencer theory in the following way.
There is a map of sheaves of cochani complexes
\[
\Phi : \cE_{\rm Pot} \to \cE_{\rm KS}
\]
which is the identity on $\PV^{0,\bu}$ and given by $\partial : \PV^{2,\bu} \to \PV^{1,\bu}$ on the remaining component.
It is immediate to see that $\Phi$ defines a map of sheaves of cochain complexes.
The theory $\cE_{\rm pot}$ is equipped with a non-degenerate BV pairing defined by the wedge-and-integrate pairing
\[
\omega_{\rm pot} (\alpha, \beta) = \int \alpha \wedge \beta  .
\]
It is immediate to verify that $\Phi$ intertwines the resulting bivector $\omega_{\rm pot}^{-1}$ and the Kodaira--Spencer Poisson bivector $\Pi_{\rm KS}$.

In fact, the parity shifted bundle $\Pi \cE_{\rm pot}$ also has the structure of a local Lie algebra, and the map $\Phi$ intertwines these local Lie algebra structures.

To describe the local Lie algebra structure on minimal Kodaira--Spencer theory with potentials we use the Calabi--Yau form $\Omega$ to identify $\cE_{\rm Pot}$ with the sheaf of cochain complexes
\[
\begin{tikzcd}
\ul{\rm odd} & \ul{\rm even} \\
 & \Omega^{2,\bu} \\
 \Omega^{0,\bu}  .
\end{tikzcd}
\]

Now, note that any Calabi--Yau surface comes equipped with a holomorphic symplectic structure and there is a Poisson bracket defined on the sheaf of holomorphic functions.
Since the bracket is defined in terms of holomorphic differential operators, it extends to a bracket on the Dobleault complex $\Omega^{0,\bu}(X)$.

This further extends to a local Lie algebra structure on the semi-direct product
\[
\Omega^{0,\bu}(X) \ltimes \Pi \Omega^{2,\bu}(X)
\]
which describes the local Lie structure on $\Pi \cE_{\rm Pot}$.
It is immediate to verify that the map $\Phi : \cE_{\rm pot} \to \cE$ intertwines the two $L_\infty$-structures.

This equips $\cE_{\rm pot}$ with the structure of an interacting (non-degenerate) BV theory.
Its relationship to (minimal) Kodaira--Spencer theory can be summarized as follows.

\begin{prop}
  The map $\Phi : \cE_{\rm pot} \to \cE_{\rm KS}$ is a map of Poisson BV theories.
  In particular, it determines a map of $\PP_0$-factorization algebras on $X$:
\[
\Phi^* : {\rm Obs}_{\cE_{\rm KS}} \to {\rm Obs}_{\cE_{\rm Pot}} .
\]
\end{prop}

\subsubsection{Eight-dimensional Kodaira--Spencer theory}
\label{sec:orgeb2dd4d}

Let \(X\) be a Calabi-Yau 4 fold. Minimal Kodaira-Spencer theory on $X$ is a $\ZZ/2$-graded theory with the following fundamental fields:
\begin{itemize}
\item The even fields are a holomorphic function $\mu^0$ and a $\partial$-closed holomorphic bivector $\mu^2$.
\item The odd fields are a divergence-free holomorphic vector field $\mu^1$ and a $\partial$-closed holomorphic section $\mu^3$ of $\wedge^3 T_X$.
\end{itemize}

The space of fields admits a locally free description obtained by including the ``descendants". The descendants of the field $\mu^j$ will be denoted $u^k \mu^j$ where $k = 1,\ldots, j$.
Here, $u^k \mu^j$ is a section of $\PV^{j - k, \bu}$.
The sheaf of cochain complexes $\cE$ underlying minimal Kodaira--Spencer theory on $X$ is
\[\begin{tikzcd}
\ul{\rm odd} & \ul{\rm even} & \ul{\rm odd} & \ul{\rm even} \\
& & & \mu^0 \in \PV^{0,\bu} \\
& & \sum u^k \mu^1 \in \PV^{1,\bu} \ar[r, "u \partial"] & u \PV^{0,\bu} \\
& \sum u^k \mu^2 \in  \PV^{2,\bu} \ar[r,"u \partial"] & u \PV^{1,\bu} \ar[r, "u \partial"] & u^2 \PV^{0,\bu} \\
\sum u^k \mu^3 \in \PV^{3,\bu} \ar[r, "u \partial"]& u \PV^{2,\bu} \ar[r,"u \partial"] & u^2 \PV^{1,\bu} \ar[r, "u \partial"] & u^3\PV^{0,\bu} \\
\end{tikzcd}
\]
The differential on this sheaf of cochain complexes is given by $\dbar + u \partial$.

There is a local Lie algebra structure on $\Pi\cE$ using the Schouten-Nijenhuis bracket $[-,-]_{\rm Sch}$ on polyvector fields.
On the fields (including the descendants) it is defined by the formula
\[
[u^k \mu^i , u^\ell \mu^j] = u^{k+\ell} \{\mu^i, \mu^j\}_{\rm NS}.
\]

The space of fields of minimal Kodaira--Spencer theory $\cE_{\rm KS}$ is equipped with an odd Poisson tensor defined by
\[
\Pi_{\rm KS} = (\partial \otimes 1) \delta_{\rm Diag} .
\]
Together with the local Lie algebra structure, this data equips $\cE_{\rm KS}$ with the structure of a $\ZZ/2$-graded Poisson BV theory.

As in the surface case, there is a closely related BV theory describing the "potentials" of minimal Kodaira--Spencer theory. The underlying sheaf of cochain complexes is
\[
\begin{tikzcd}
\ul{\rm odd} & \ul{\rm even} & \ul{\rm odd} & \ul{\rm even} \\
& & & \eta^{0}\in\PV^{0,\bu} \\
& & \mu^{1}+u\mu^{0}\in\PV^{1,\bu} \ar[r, "u \partial"] & u \PV^{0,\bu} \\
u^{-1} \gamma^4 + \gamma^3  \in u^{-1} \PV^{4,\bu} \ar[r, "u \partial"] & \PV^{3,\bu} & &  \\
\beta^{4} \in\PV^{4,\bu} &  &  & \\
\end{tikzcd}
\]
We will again refer to this sheaf as $\cE_{\rm pot}$. Note that we can identify \[\cE_{\rm pot}\cong \Pi T^{*}(\PV^{0,\bu}\oplus (\PV^{1,\bu}\to u\PV^{0,\bu})).\]

\begin{dfn}
  Let $X$ be a Calabi-Yau 4-fold, and let $\cL$ denote the local dg Lie algebra $\PV^{0,\bu}\oplus (\PV^{1,\bu}\to u\PV^{0,\bu})$ with Lie bracket given by the Schouten bracket. Kodaira-Spencer with potentials on $X$ is the $\ZZ/2$-graded BV theory given by deforming the BF theory of $\cL$ by the local functional \[ \frac{1}{3}\int_{X}(\nu^{0}\partial\gamma^{3}\partial\gamma^{3})\]\end{dfn}

To facillitate the calculations in this section, it will be useful to explicate the Lie structure. We have the following brackets:

\begin{align}
  \PV^{1,\bu}\times\PV^{0,\bu}\to \PV^{0,\bu},\ \ \ & [\mu^{1},\nu^{0}] = \{\mu^{1},\nu^{0}\}_{NS}\\
  \PV^{1,\bu}\times\PV^{1,\bu}\to\PV^{1,\bu},\ \ \ & [\mu^{1}, \mu^{1}] = \{\mu^{1}, \mu^{1}\}_{NS}\\
  \PV{1,\bu}\times u\PV^{0,\bu}\to u\PV^{0,\bu}, \ \ \ & [\mu^{1}, u\mu^{0}]=u\{\mu^{1},\mu^{0}\}_{NS} \\
  \PV^{3,\bu}\times \PV^{3,\bu}\to \PV^{4,\bu}, \ \ \ & [\gamma^3, \Tilde{\gamma}^3] = \frac{1}{2}(\{\partial \gamma^3 , \Tilde{\gamma}^{3}\}_{\rm NS} +(-1)^{|\gamma^{3}|} \{\gamma^3, \partial \Tilde{\gamma}^{3}\}_{\rm NS}) \\
  \PV^{4,\bu}\times \PV^{0,\bu}\to \PV^{3,\bu}, \ \ \ & [\beta^{4}, \nu^{0}] = \{\beta^{4}, \nu^{0}\}_{\rm {NS}}\\
  \PV^{3,\bu}\times \PV^{0,\bu}\to \PV^{1,\bu}, \ \ \ & [\gamma^{3}, \nu^{0}] = \{\partial\gamma^{3}, \nu^{0}\}_{\rm NS}\\
  \PV^{3,\bu}\times \PV^{1,\bu}\to \PV^{3,\bu}, \ \ \ & [\gamma^{3}, \mu^{1}] = \{\gamma^{3}, \mu^{1}\}_{\rm NS}\\
  u^{-1}\PV^{4,\bu}\times u\PV^{0,\bu}\to \PV^{3,\bu}, \ \ \ & [u^{-1}\gamma^{4}, u\mu^{0}] = \{\gamma^{4},\mu^{0}\}_{\rm NS}
\end{align}

Together with the wedge and integrate pairing, $\cE_{\rm pot}$ has the structure of a nondegenerate BV theory.

Like in the case of Kodaira--Spencer theory on a complex surface, there is a  map of sheaves of cochain complexes $\Phi^{(1)}: \cE_{\rm pot}\to \cE_{\rm KS}$ given by

\[\begin{pmatrix}\nu^{0} \\ \mu^{1}+u\mu^{0} \\ u^{-1}\gamma^{4}+\gamma^{3} \\ \beta^{4}\end{pmatrix}\mapsto \begin{pmatrix}\nu^{0} \\ \mu^{1}+u\mu^{0} \\ \partial\gamma^{3} \\ \partial\beta^{4}\end{pmatrix}\]

This fails to be a map of local dg Lie algebras, but the induced map on $\partial$-cohomology is indeed a map of local Lie algebras. Therefore, we still have the following:

\begin{prop}
There is a local $L_{\infty}$ map $\cE_{\rm pot}\to \cE_{\rm KS}$.
\end{prop}
\begin{proof}
  Consider the collection of linear maps $\{\Phi^{(n)}: \cE_{\rm Pot}^{\otimes n}\to \cE_{\rm KS}[n-1]\}$ where $\Phi^{(1)}$ is given as above and $\Phi^{{(2)}}$ is given by \[\begin{pmatrix}\nu^{0} \\ \mu^{1}+u\mu^{0} \\ u^{-1}\gamma^{4}+\gamma^{3} \\ \beta^{4}\end{pmatrix}\otimes \begin{pmatrix}\tilde{\nu}^{0} \\ \tilde{\mu}^{1}+u\tilde{\mu}^{0}\\ u^{-1}\tilde{\gamma}^{4}+\tilde{\gamma}^{3} \\ \tilde{\beta}^{4} \end{pmatrix}\to \begin{pmatrix} 0 \\ 0 \\ [\mu^{0},\tilde{\gamma}^{3}]_{\rm NS}\pm [\tilde{\mu}^{0},\gamma^{3}] \\ 0\end{pmatrix}.\]

  We claim that the $\Phi^{(n)}$ define an $L_{\infty}$ map. To this end, there are three kinds of conditions to check.
  \begin{itemize}
    \item A 2-linear condition which expresses that $\Phi^{(2)}$ renders the failure of $\Phi^{(1)}$ to preserve the brackets homotopically trivial. That is, for $a,b\in\cE_{\rm Pot},$ \[\Phi^{(1)}([a,b])-[\Phi^{(1)}(a),\Phi^{(1)}(b)]=d_{\rm KS}\Phi^{(2)}(a,b)-\Phi^{(2)}(d_{\rm Pot}a, b)-(-1)^{|a|}\Phi^{(2)}(a,d_{\rm Pot}b).\] Note that $\Phi^{(1)}$ in fact preserves all brackets other than the brackets $\PV^{3,\bu}\otimes \PV^{1,\bu}\to \PV^{3,\bu}$ and $u^{-1}\PV^{4,\bu}\otimes u\PV^{0,\bu}\to \PV^{3,\bu}$
      \surya{to finish}
  \end{itemize}
\end{proof}

\subsection{The Type IIA topological string}
\label{IIAtop}
\subsection{Reduction of twisted supergravity}
\label{sec:orgcf7b6a4}
\subsubsection{Calabi--Yau compactifications}
\label{sec:org16a2c98}
\subsection{Twisted supergravity on a three-fold}
\label{sec:org774abb4}
\section{The non-minimal G2 twist}
\label{sec:org590ab85}

\brian{some general background on G2 twist, state main result}

We consider the eleven-dimensional theory on the space
\[
  X \times Z \times \RR
\]
where $Z$ is a hyper-K\"{a}hler surface and $X$ is a Calabi--Yau three-fold.
Denote by $\Omega_Z$ the holomorphic volume form on $Z$ and $\Omega_X$ the holomorphic volume form on $X$.
We will work in a background where the $(1,0)$-component of the field $\gamma$ satisfies the following equations:
\begin{equation}\label{eqn:background}
  \begin{array}{rcccc}
    \partial \gamma^{1,0} & = & \Omega_Z \\
    \dbar \gamma^{1,0} + \d_{\rm dR} \gamma^{1,0} & = & 0 .
  \end{array}
\end{equation}
The second equation implies that $\gamma^{1,0}$ is constant along $\RR$ and holomorphic along $X \times Z$.
The first equation says that $\gamma^{1,0}$ is a trivialation of the holomorphic volume form on $Z$.

To see that this is a consistent background of the eleven-dimensional theory we must check that $\gamma^{1,0}$ satisfies the appropriate equations of motion.

\begin{lem}
  Any field $\gamma^{1,0}$ satisfying the equations in (\ref{eqn:background}) satisfies the Maurer--Cartan equation of the eleven-dimensional theory.
\end{lem}

\begin{proof}
  We must show that $\gamma^{1,0}$ satisfies the following Maurer--Cartan equation
  \[
    Q_{\rm BRST} \gamma^{1,0} + \frac12 [\gamma^{1,0} ,\gamma^{1,0}] = 0
  \]
  where $Q_{\rm BRST}$ is the linear BRST operator and $[-,-]$ is the Lie bracket defining the interacting piece of the eleven-dimensional theory.

Recall, the linear BRST differential is of the form
  \[
    \dbar + \partial_{\Omega} + \partial_{\Omega^0 \to \Omega^1} + \d_{\rm dR}
  \]
  Here, $\partial_\Omega$ is the divergence operator which only acts on the $\mu$-type fields and $\partial_{\Omega^0 \to \Omega^1}$ is the first piece of the holomorphic de Rham operator acting on $\Omega^{0,\bu}(X \times Z)$.
  Since $(\dbar + \d_{\rm dR}) \gamma^{1,0} = 0$ by assumption, we see that $\gamma^{1,0}$ is closed for the linear BRST operator.

The only component of the Lie bracket involving two fields of type $\gamma$ is of the form
  \[
    [\gamma, \gamma] = (\partial \gamma \wedge \partial \gamma) \vee (\Omega_X \wedge \Omega_Z) \in \PV^{1,\bu}(X \times Z) \; \Hat{\otimes} \; \Omega^\bu(L) .
  \]
  Since $\partial \gamma^{1,0}$ is a $(2,0)$ form along $Z$, we see that $\partial \gamma^{1,0} \wedge \partial \gamma^{1,0} =0$.
  We conclude that $\gamma^{1,0}$ satisfies the Maurer--Cartan equation.
\end{proof}

Next, we expand the action functional near the background where $(1,0)$ component of $\gamma$ takes the value $\gamma^{1,0}$ where $\gamma^{1,0}$ satisfies equations (\ref{eqn:background}).
This will generate new kinetic and interacting terms which we can extract by inserting a formal parameter $\delta$ and expressing the action functional in terms of the deformed field $\Tilde{\gamma} = \gamma + \delta \gamma^{1,0}$.

There are two interaction terms in the original theory.
The first is
\begin{equation}\label{eqn:int1}
  \frac12 \int_{X\times Z \times L} \left(\gamma \vee \{\mu, \mu\} \right) \wedge (\Omega_Z \wedge \Omega_X)
\end{equation}
and the second is
\begin{equation} \label{eqn:int2}
  \frac16\int_{X \times Z \times L} \gamma \partial \gamma \partial \gamma .
\end{equation}

We can integrate Equation (\ref{eqn:int1}) by parts to put it in the form $\frac12 \int_{X \times Z \times L} \left[(\partial \gamma) \vee (\mu \wedge \mu) \right]$ where $\mu \wedge \mu$ is the wedge product of polvector fields.\brian{there might be some factors I'm being sloppy with here}
Expanding this expression around $\gamma \to \Tilde{\gamma} = \gamma + \delta \gamma^{1,0}$ we obtain
\[
  \frac12 \int \left(\gamma \vee \{\mu, \mu\} \right) \wedge (\Omega_Z \wedge \Omega_X) + \frac{\delta}{2} \int \left[\Omega_Z \vee (\mu \wedge \mu) \right] \wedge (\Omega_Z \wedge \Omega_X) .
\]
Here, we have used the identity $\partial \gamma^{1,0} = \Omega_Z$.

Next, we expand Equation (\ref{eqn:int2}) around $\gamma \to \Tilde{\gamma} = \gamma + \delta \gamma^{1,0}$.
This becomes
\[
  \frac16 \int \gamma \partial \gamma \partial \gamma + \frac{\delta}{2} \int \left(\gamma \partial \gamma\right) \wedge \Omega_Z .
\]
Notice that there are no $\delta^2$ terms since $\partial \gamma^{1,0} \partial \gamma^{1,0} = 0$.

So far, we have written everything in terms of action functionals.
There is a completely equivalent statement in terms of the resulting dg Lie algebra structure describing the eleven-dimensional theory in this background which we summarize as follows.

\begin{lem} \label{lem:background}
  The dg Lie algebra describing the eleven-dimensional theory placed in a background where the $(1,0)$ component of $\gamma$ takes the value $\gamma^{1,0}$ satisfying (\ref{eqn:background}) is isomorphic to the the dg Lie algebra whose differential is
  \begin{equation}\label{eqn:newdiff}
    Q_{\rm BRST} + \frac{1}{2} \left\{ \int \left[\Omega_Z \vee (\mu \wedge \mu) \right] \wedge (\Omega_Z \wedge \Omega_X) + \int \left(\gamma \partial \gamma\right) \wedge \Omega_Z , \; - \; \right\}
  \end{equation}
  and whose Lie bracket is unchanged.
  Here, $Q_{\rm BRST}$ is the original linear BRST differential of the eleven-dimensional theory.
\end{lem}

This lemma followed directly from our analysis of the way the action functional of the theory decomposed around this particular background.
The next result we state is an equivalence with a theory which exists on any product of manifolds $Z \times M$ where $Z$ is a hyper-K\"{a}hler surface as above but now $M$ is any smooth seven-dimensional manifold.

\begin{prop}
  The eleven-dimensional theory on $X \times Z \times L$ placed in the background where the $(1,0)$ component of $\gamma$ takes the value $\gamma^{1,0}$ satisfying (\ref{eqn:background}) is equivalent to the $\ZZ/2$-graded theory whose fields are
    $\alpha \in \Omega^{0,\bu}(Z) \; \Hat{\otimes} \; \Omega^\bu(X \times L) [1]$
  and whose action functional reads
  \[
    \frac12 \int \alpha \d\alpha + \frac16 \int \alpha \{\alpha, \alpha\}
  \]
  where $\{-,-\}$ is the Poisson bracket on $\Omega^{0,\bu}(Z)$.
\end{prop}

As stated in this proposition, it is clear that the theory does not depend on the complex structure on the Calabi--Yau three-fold $X$ even though this is not {\em a priori} obvious from the description in Lemma \ref{lem:background}. 
The theory only depends on the smooth structure on $M = X \times L$.
\brian{cite Surya and Phil, Chris and B, Kevin}

\begin{proof}
To unpack this new differential (\ref{eqn:newdiff}) in Lemma \ref{lem:background} we need to reidentify the fields of the eleven-dimensional theory.
Using the holomorphic volume forms $\Omega_X$ and $\Omega_Z$ we make the following identifications
\begin{equation}\label{eqn:decomposevf}
\begin{array}{ccccc}
  \PV^{1,\bu}(X \times Z) & = & \PV^{1,\bu}(X) \; \Hat{\otimes} \; \PV^{0,\bu}(Z) & \oplus & \PV^{0,\bu}(X) \; \Hat{\otimes} \; \PV^{1,\bu}(Z) \\ & \cong & \Omega^{2,\bu}(X) \; \Hat{\otimes} \; \Omega^{0,\bu} (Z) & \oplus & \Omega^{0,\bu}(X) \; \Hat{\otimes} \; \Omega^{1,\bu}(Z) 
\end{array}
\end{equation}


From Lemma \ref{lem:background} we see that the new terms in the linear BRST differential arise from the quadratic terms in the action
\begin{equation}\label{eqn:newterms}
  \int \left[\Omega_Z \vee (\mu \wedge \mu) \right] \wedge (\Omega_Z \wedge \Omega_X) + \int \left(\gamma \partial \gamma\right) \wedge \Omega_Z .
\end{equation}

Recall that the linear BRST complex of the original eleven-dimensional theory is obtained by taking the tensor product of the cochain complexes $\PV^{1,\bu}(X \times Z) \xto{\partial_\Omega} \PV^{0,\bu}(X \times Z)$ and $\Omega^{0,\bu}(X \times Z) \xto{\partial} \Omega^{1,\bu}(X \times Z)$ with the de Rham complex $\Omega^\bu(L)$ on the one-manifold $L$.
Using the decomposition (\ref{eqn:decomposevf}) we see that the deformed linear BRST complex is the tensor product of the de Rham complex $\Omega^{\bu}(L)$ with the cochain complex
\[
  \begin{tikzcd}
  & \Omega^{0,\bu}(X) \; \Hat{\otimes} \; \Omega^{1,\bu}(Z) \ar[dr, "\partial_Z"] \\
  & & \Omega^{3,\bu}(X) \; \Hat{\otimes} \; \Omega^{2,\bu}(Z) \\
 & \Omega^{2,\bu}(X) \; \Hat{\otimes} \; \Omega^{0,\bu}(Z) \ar[ur, "\partial_X"'] & \\
\;_{\cong}  & & \Omega^{1,\bu}(X) \; \Hat{\otimes} \; \Omega^{0,\bu}(Z) \ar[ul, dashed, bend left = 10, "\partial_X"]\\
 & \Omega^{0,\bu}(X) \; \Hat{\otimes} \; \Omega^{0,\bu}(Z) \ar[ur, "\partial_X"] \ar[dr,"\partial_Z"'] \\
  & & \Omega^{0,\bu}(X) \; \Hat{\otimes} \; \Omega^{1,\bu}(Z)
\ar[dashed, rounded corners, to path={ -- ([yshift=-2ex]\tikztostart.south) -| ([xshift=-1.5ex]\tikztotarget.west) -- (\tikztotarget)}, uuuuul]
  %\ar[uuuuul, start anchor =  {[yshift = 0ex, xshift=0ex]}, end anchor = {[yshift=1.0ex, xshift=-5ex]}, bend left = 90, dotted] .
  \end{tikzcd}
\]
Here, the dashed arrow along the outside of the diagram corresponds to the BV antibracket with the first term in (\ref{eqn:newterms}).
Note that this is the identity morphism on $\Omega^{0,\bu}(X) \; \Hat{\otimes} \; \Omega^{1,\bu}(Z)$.
The other dashed arrow corresponds to the BV antibracket with the second term in (\ref{eqn:newterms}).
It is given by the holomorphic de Rham operator $\partial_X : \Omega^{1,\bu}(X) \to \Omega^{2,\bu}(X)$.

\brian{cochain homotopy finish}
Since the outer dashed arrow is an isomorphism, we see that this cochain map is a quasi-isomorphism.
It is immediate to check that this quasi-isomorphism intertwines the Lie brackets.


\end{proof}

\end{document}


TO DO:

-- m2 brane twist

-- intro to renormalization section

-- email Kevin about moduli of quantizations. 

-- CY3 compactification

-- G2 compactification. 
