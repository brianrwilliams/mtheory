% Created 2020-10-14 Wed 14:51
% Intended LaTeX compiler: pdflatex
\documentclass[11pt]{amsart}
\usepackage[utf8]{inputenc}
\usepackage[T1]{fontenc}
\usepackage{graphicx}
\usepackage{grffile}
\usepackage{longtable}
\usepackage{wrapfig}
\usepackage{rotating}
\usepackage[normalem]{ulem}
\usepackage{amsmath}
\usepackage{textcomp}
\usepackage{amssymb}
\usepackage{capt-of}
\usepackage{hyperref}

\usepackage{macros-mtheory}

\author{Surya and Brian}
\date{\today}
\title{Twisted \(M\)-theory and its perturbative quantization.}
\hypersetup{
 pdfauthor={Surya and Brian},
 pdftitle={Twisted \(M\)-theory and its perturbative quantization.},
 pdfkeywords={},
 pdfsubject={},
 pdfcreator={Emacs 27.1 (Org mode 9.4)}, 
 pdflang={English}}

\def\pv{{\rm PV}}

\begin{document}

\maketitle
\tableofcontents


\section{Definition of twisted supergravity}

\label{sec:org24653eb}
\subsection{The classical BV theory}
\label{sec:org5c9d94b}
In this section we define the central theory of study, within the Batalin--Vilkovisky formalism.
The theory will be defined on any eleven-dimensional manifold of the form $X \times L$ where $X$ is a Calabi--Yau five-fold with volume form $\Omega$ and $L$ is a smooth oriented one-manifold.

Before stating the definition, we recall some general objects in complex geometry that will play a central role.
For now, let $X$ be any complex manifold and let $V$ be a holomorphic vector bundle on $X$.
If $j$ is an integer, we let $\Omega^{0,j}(X, V)$ denote the space of Dolbeault forms of type $j$ on with values in $V$.
The $\dbar$ operator for $V$ is of the form $\dbar : \Omega^{0,j}(X, V)\to \Omega^{0,j+1}(X)$.
This operator defines the Dolbeault complex of $V$
\[
  \Omega^{0,\bu}(X, V) = \left(\Omega^{0,j}(X, V)[-j] \; , \; \dbar\right)
\]
which is a free resolution for the sheaf of holomorphic sections of $V$.

Denote by $TX$ the holomorphic tangent bundle of $X$.
Then, the Lie bracket of holomorphic vector fields extends to a Lie bracket on the Dolbeault $\Omega^{0,\bu}(X, TX)$ thus endowing it with the structure of a dg Lie algebra.

By definition, the Dolbeault complex for the trivial bundle $\Omega^{0,\bu}(X)$ is a dg module for the dg Lie algebra $\Omega^{0,\bu}(X, TX)$.
We can thus consider the following semi-direct product dg Lie algebra
\beqn\label{eqn:div1}
  \Omega^{0,\bu}(X, TX) \oplus \Omega^{0,\bu}(X) [-1] .
\eeqn
Notice that $\Omega^{0,\bu}(X)$ has been shifted to a seemingly unnatural cohomological degree, with $(0,j)$-forms sitting in degree $j+1$.

Now, suppose $X$ is a Calabi--Yau manifold with holomorphic volume form $\Omega$.
The holomorphic divergence operator with respect to $\Omega$ is an operator of the form
\[
  \partial_\Omega : \Omega^{0,\bu}(X, TX) \to \Omega^{0,\bu}(X)
\]
defined by the formula
\[
  \partial_\Omega (\mu) \wedge \Omega = L_\mu (\Omega)
\]
where, on the right hand side, we mean the Lie derivative of $\Omega$ with respect to $\mu$.

Using the holomorphic divergence operator $\partial_\Omega$ we can consider the following deformation of the dg Lie algebra (\ref{eqn:div1}).
The Lie bracket remains the same but the differential is deformed from $\dbar$ to $\dbar + \partial_\Omega$.
Here, $\partial_{\Omega}$ denotes the holomorphic divergence operator thought of as a degree one operator from $\Omega^{0,\bu}(X, TX)$ to $\Omega^{0,\bu}(X)$.
We will denote the resulting dg Lie algebra by $\cS(X)$.

It is easy to see that $\cS(X)$ provides a free resolution of the sheaf of holomorphic divergence free vector fields on $X$.

%We will mostly consider the sub dg Lie algebra of $(\ref{eqn:pvlie1})$ consisting of the polyvector fields of type $i\leq 1$:
%\beqn\label{eqn:pvlie2}
%  \pv^{\leq 1, \bu} (X)[1] =
%\eeqn


%We note that this deformation is only in the $\ZZ/2$-graded sense.

%For $X$ a Calabi--Yau manifold let $\cS^{\bu,\bu}(X)$ be the $\ZZ/2$-graded dg Lie algebra whose underlying $\ZZ/2$-graded vector space is $\pv^{\bu,\bu}(X)[1]$ (with gradings defined modulo $2$), whose differential is
%\beqn\label{differential}
%\dbar + \partial_{\Omega}
%\eeqn
%and the bracket is the Nijenhuis--Schouten bracket $\{\cdot,\cdot\}_{\rm NS}$.
%We notice that in the original $\ZZ$-grading on $\pv^{\bu,\bu}(X)$ the operator $\partial_{\Omega}$ has cohomological degree $-1$, which is why the above deformation is only defined in a $\ZZ/2$-graded sense.

%The differential (\ref{differential}) clearly preserves the subspace $\pv^{\leq 1, \bu}(X)$.
%We let $\cS^{\leq 1,\bu}(X) \subset \cS^{\bu,\bu}(X)$ denote the $\ZZ/2$-graded sub dg Lie algebra consisting of polyvector fields of type $i\leq 1$ equipped with the differential $\dbar + \partial_{\Omega}$.

%\begin{rmk}
%  There is a way to regrade $\cS^{\leq 1,\bu}(X)$ while keeping the differential the same in such a way that it has the structure of a $\ZZ$-graded dg Lie algebra.
%  Explicitly, this is
%  \[
%    \begin{array}{ccccc}
%      \ul{0} & & \ul{1} \\
%      \pv^{1,\bu}(X) &\xto{\partial_{\Omega}} & \pv^{0,\bu}(X) .
%    \end{array}
%  \]
%  This is a $\ZZ$-graded dg Lie algebra which agrees with $\cS^{\leq 1, \bu}(X)$ as a $\ZZ/2$- graded dg Lie algebra.
%  This dg Lie algebra is a free resolution for the Lie algebra of holomorphic divergence-free vector fields on $X$.
%  That is, holomorphic vector fields which preserve the holomorphic volume form $\Omega$.
% \end{rmk}


The space of differential forms on any smooth manifold $L$ is a commutative dg algebra $\Omega^{\bu}(L) = \left(\oplus_{k}\Omega^{k}(L)[-k], \d_{\dR}\right)$.
By taking the tensor product with the dg Lie algebra $\cS(X)$ we obtain the dg Lie algebra
\beqn\label{localLie}
  \cS(X) \; \Hat{\otimes} \; \Omega^{\bu}(L)
\eeqn
where $X$ is a Calabi--Yau manifold and $L$ is a smooth manifold.
We remark that the structure maps of this Lie algebra, namely the differential and bracket, are given by differential and bidifferential operators, respectively.
In this sense, it fits the definition of a {\em local dg Lie algebra} on the product manifold $X \times L$, see Definition ?? \ref{CG2}.

Associated to any local dg Lie algebra is a classical field theory in the BV formalism that one refers to as ``$BF$-theory''.
Denote the sheaf of sections of the local dg Lie algebra by $\cL$.
The fields of the associated BV theory are pairs
\[
(A, B) \in \cL[1] \oplus \cL^{!}[-1]\]
and the action functional reads $\int B F_{A} = \int B (\d_{\cL}A + \frac12 [A,A])$.
Here $\d_{\cL}$ denotes the differential on $\cL$ and $[\cdot,\cdot]$ is the Lie bracket.
Notice that the equations of motion for the $A$-field is simply the Maurer--Cartan equation $\d_\cL A + \frac12 [A,A] = 0$.
For the $B$-field the equations of motion read $\d_\cL B + [A, B] = 0$.

For the case at hand, we consider the BF theory associated to the local dg Lie algebra (\ref{localLie}) defined on the product $X \times L$ where $X$ is a Calabi--Yau manifold and $L$ is a smooth manifold.
Denote by $d \in \ZZ$ the sum $\dim_{\CC}(X) + \dim_{\RR}(L)$.

The fields decompose as follows
\[
(\mu, \gamma) = (\mu^{1,\bu} + \mu^{0,\bu}, \gamma^{0,\bu}+\gamma^{1,\bu})
\]
where
\[
  \begin{array}{ccl}
    \mu = \mu^{1,\bu}+\mu^{0,\bu} & \in & \left(\pv^{1,\bu}(X)[1] \oplus \pv^{0,\bu}(X) \right) \; \Hat{\otimes} \; \Omega^{\bu}(L) \\
    \gamma = \gamma^{0,\bu} + \gamma^{1,\bu} & \in & \left(\Omega^{0,\bu}(X) \oplus \Omega^{1,\bu}(X)[-1] \right) \; \Hat{\otimes}\; \Omega^{\bu}(L)[d] .
  \end{array}
\]
Notice that what we called the ``$A$''-field in the general definition of BF theory we are now calling $\mu$ and what we called the ``$B$''-field we are now calling $\gamma$.
By force of habit, we will still refer to the theory with these particular fields as $BF$-theory.

While the definition of $\mu$ is independent of the dimension, the field $\gamma$ depends on $d$.
This is in order for the BV antibracket to be of the appropriate degree.
Additionally, we have utilized the holomorphic volume form in the definition of the field $\gamma$.
Indeed, in this presentation the BV antipairing on fields reads
\[
  \int_{X \times L} (\mu \vdash \gamma) \wedge \Omega = \int_{X \times L} (\mu^{1,\bu} \vdash \gamma^{1,\bu} + \mu^{0,\bu} \vdash \gamma^{0,\bu}) \wedge \Omega .
\]

\subsubsection{}
We will be mostly interested in the case that $X$ is a Calabi--Yau five-fold and $L$ is a real one-dimensional manifold.
In this case $d = 6$ so the field $\gamma$ takes the form
\[
  \gamma \in \left(\Omega^{0,\bu}(X)[6] \oplus \Omega^{1,\bu}(X)[5] \right) \otimes \Omega^\bu(L).
\]

What is important for us is not simply BF theory for the dg Lie algebra $\cS(X) \otimes \Omega^\bu(L)$, but a certain deformation of it.
We consider a theory whose fields are that of the BF theory but whose action functional is of the form $S_{\rm BF} + J$, where $J$ is a local functional that purely depends on the fields $\gamma$.
It is defined by the following formula
\[
J(\gamma) = \frac13 \int_{X\times L} \gamma \partial \gamma \partial \gamma .
\]

First off, we notice an inconsistency in the cohomological degrees of the functionals $S_{\rm BF}$ and $J$.
Being a classical action functional $S_{\rm BF}$ is, of course, ghost degree zero.
However, we observe that $J$ is of ghost degree $??$.
In order to make sense of the deformed theory, therefore, we must consider the space of fields as a $\ZZ/2$-graded object rather than a $\ZZ$-graded one.

So, in order to make sense of the action functional $S_{\rm BF} + J$ as a homogenously even functional, the $\ZZ/2$-graded space of fields takes the form
\[
  \begin{array}{ccl}
    \mu = \mu^{1,\bu}+\mu^{0,\bu} & \in & \left(\pv^{1,\bu}(X)[1] \oplus \pv^{0,\bu}(X) \right) \; \Hat{\otimes} \; \Omega^{\bu}(L) \\
    \gamma = \gamma^{0,\bu} + \gamma^{1,\bu} & \in & \left(\Omega^{0,\bu}(X) \oplus \Omega^{1,\bu}(X)[1] \right) \; \Hat{\otimes}\; \Omega^{\bu}(L) .
  \end{array}
\]
Here, we have used the following convention for $\ZZ/2$-graded complexes.
If $V$ is a $\ZZ/2$-graded vector space with even part $V^0$ and odd part $V^1$, then $V[1]$ is a $\ZZ/2$-graded vector space with even part $V^1$ and odd part $V^0$.

With the issue of gradings out of the way, we turn to the question of whether this is a well-defined classical BV theory.
In order for this to be true, we must know that the total action functional $S_{\rm BF} + J$ satisfies the classical master equation $\{S_{\rm BF} + J , S_{\rm BF} + J\} = 0$.
Of course, we know the undeformed BF-theory satisfies the classical master equation $\{S_{\rm BF}, S_{\rm BF}\} = 0$.
Additionally, since $\gamma$ does not pair with itself under the antibracket, we also see that $\{J,J\}=0$.
What remains is to see that $J$ is closed for the differential $\{S_{\rm BF}, \cdot\}$.

\begin{lem}
The functional $J = \frac13 \int \gamma \partial \gamma \partial \gamma$ is closed for the differential $\{S_{\rm BF},\cdot\}$.
In particular, $S_{\rm BF} + J$ is a well-defined classical BV action.
\end{lem}

\begin{proof}
It is clear that $J$ is closed for the part of the differential arising from the kinetic term $\{S_{\rm free},\cdot\}$.
It remains to see that
\end{proof}

\begin{dfn}
Let $X$ be a Calabi--Yau five-fold and $L$ a one-dimensional smooth manifold.
Define the $\ZZ/2$-graded BV theory $\cT$ on
      \[
        X \times L
      \]
to be the deformation of BF theory for the $\ZZ/2$-graded local Lie algebra $\cS^{\leq 1,\bu}(X) \Hat{\otimes} \Omega^{\bu}(L)$ by the local functional

\end{dfn}

The BV action functional for the theory splits as a sum
\[
  S_{\rm BF} + J
\]
where $J$ is as in the definition above and $S_{BF}$ is the action functional for the underlying BF theory which takes the form
\[
  S_{\rm BF} = \int \gamma \vdash \left(\dbar + \d_{\rm dR} + \partial_{\Omega}\right) \mu \wedge \Omega + \frac12 \int \gamma \vdash \{\mu,\mu\}_{\rm NS} \wedge \Omega .
\]
We will furthermore decompose $S_{\rm BF} = S_{\rm free}+I_{\rm BF}$ into its free and interacting pieces.

For this deformation to be a well-defined BV theory, we must know that the total action functional $S_{\rm BF} + J$ satisfies the classical master equation $\{S_{\rm BF} + J , S_{\rm BF} + J\} = 0$.
Of course, we know the undeformed BF-theory satisfies the classical master equation $\{S_{\rm BF}, S_{\rm BF}\} = 0$.
Additionally, simply by type reasons do we see that $\{J,J\}=0$.
What remains is to see that $J$ is closed for the differential $\{S_{\rm BF}, \cdot\}$.

\begin{lem}
  The functional $J = \frac13 \int \gamma \partial \gamma \partial \gamma$ is closed for the differential $\{S_{\rm BF},\cdot\}$.
  In particular, $S_{\rm BF} + J$ is a well-defined classical BV action.
\end{lem}
\begin{proof}
  It is clear that $J$ is closed for the part of the differential arising from the kinetic term $\{S_{\rm free},\cdot\}$.
  It remains to see that :
\end{proof}

There is a similar deformation of BF theory for the dg Lie algebra $\cS(X) \otimes \Omega^\bu(L)$ when $X$ is any {\em odd dimensional} Calabi--Yau manifold and $L$ is a smooth one-dimensional manifold.


\subsection{Equations of motion}
\label{sec:org2973a4d}
In this section we analyze the equations of motion and provide a moduli theoretic interpretation of the solution space.

\subsection{Evidence for twisted supergravity}

In parts of the remainder of the paper we will provide evidence, and consistency checks, that the eleven-dimensional theory that we have just defined is a candidate for the minimal twist of supergravity.
In this section, we discuss a more direct relationship by exhibiting the fields of the theory we have outlined as components of the supergravity multiplet that are expected to survive in the twist.

\section{Quantization of holomorphic-topological theories}
\label{sec:org297a559}
\subsection{Effective renormalization}
\label{sec:orga5cd4ed}
\subsection{The \(11\)-dimensional theory}
\label{sec:org8b189fc}
\subsection{The moduli space of quantizations}
\label{sec:orgf278b4e}
\section{Relationship to the Type IIA string}
\label{sec:org355a726}

Let $TX$ denote the holomorphic tangent bundle and define
\[
\pv^{i,j}(X) \define \Omega^{0,j}(X, \wedge^{i} TX) .\]
This is the space of $i$-pvvectors of Dolbeault type $j$.
Using the $\dbar$ operator for the holomorphic bundle $\wedge^{i}TX$ we obtain a cochain complex $\pv^{i,\bu}(X) = \left(\oplus_{j} \pv^{i,j}(X) [-j], \dbar \right)$ which provides a free resolution of the sheaf of holomorphic polyvector fields of type $i$.


There is a bracket on the space of holomorphic polyvector fields called the Nijenhuis--Schouten bracket.
This bracket is defined using holomorphic pvdifferential operators, so extends to a bracket on the Dolbeault complex to define a bracket of the form
\[
  \{\cdot, \cdot\}_{\rm NS} \colon \pv^{i,j}(X) \times \pv^{k,\ell}(X) \to \PV^{i+k-1, j+\ell}(X) .
\]
This bracket endows the total complex
\beqn\label{eqn:pvlie1}
\pv^{\bu,\bu}(X)[1] = \left(\oplus_{i,j} \pv^{i,j}(X)[-i-j+1] , \dbar \right)
\eeqn
with the structure of a dg Lie algebra.
Here, we note that in this dg Lie algebra the space $\pv^{i,j}(X)$ lies in cohomological degree $i+j-1$.


So far, $X$ has been an arbitrary complex manifold.
When $X$ is Calabi--Yau of complex dimension $n$, the holomorphic volume form $\Omega$ defines an isomorphism
\[
  \vdash \Omega : \pv^{i,j}(X) \cong \Omega^{n-i, j}(X).
\]
In turn, the holomorphic de Rham operator $\partial : \Omega^{p, j}(X) \to \Omega^{p+1,j}$ defines a holomorphic differential operator
\[
  \partial_{\Omega} : \pv^{i,j}(X) \to \pv^{i-1,j}(X) .
\]
This is the holomorphic analog of the divergence operator with respect to $\Omega$.

\subsection{Kodaira--Spencer theory with potentials}
\label{sec:org5e18f8e}
\subsubsection{Warm-up: four-dimensional Kodaira--Spencer theory}
\label{sec:org91dc4ca}
\subsubsection{Eight-dimensional Kodaira--Spencer theory}
\label{sec:orgeb2dd4d}
\subsection{The Type IIA topological string}
\label{sec:org1f7e793}
\subsection{Reduction of twisted supergravity}
\label{sec:orgcf7b6a4}
\subsubsection{Calabi--Yau compactifications}
\label{sec:org16a2c98}
\subsection{Twisted supergravity on a three-fold}
\label{sec:org774abb4}
\section{The non-minimal G2 twist}
\label{sec:org590ab85}
Does math mode \(work here = e^x\)?
\end{document}
