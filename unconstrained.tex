\documentclass[11pt]{amsart}

\pdfoutput = 1

\usepackage{macros,slashed,amsaddr}
%\usepackage[dvipsnames]{xcolor}

\linespread{1.5}

\setcounter{tocdepth}{2}
\numberwithin{equation}{section}
\newcommand{\nocontentsline}[3]{}
\newcommand{\tocless}[2]{\bgroup\let\addcontentsline=\nocontentsline#1{#2}\egroup}
\newcommand{\changelocaltocdepth}[1]{%
  \addtocontents{toc}{\protect\setcounter{tocdepth}{#1}}%
  \setcounter{tocdepth}{#1}%
}
\setcounter{tocdepth}{1}

\def\brian{\textcolor{blue}{BW: }\textcolor{blue}}

\newcommand{\defterm}[1]{\textbf{\emph{#1}}}

\def\PV{{\rm PV}}

\begin{document}

\title{Quantization of the unconstrained theory}

\maketitle

Consider the $\ZZ/2$-graded theory on $\CC^5 \times \RR$ with two sets of fundamental fields:
\begin{align*}
\mu & \in \Pi \Omega^\bu (\RR) \, \Hat{\otimes} \, \PV^{1,\bu}(\CC^5) \\
\gamma & \in \Omega^{\bu}(\RR) \, \Hat{\otimes} \, \Omega^{1,\bu}(\CC^5) .
\end{align*}
We will denote the full space of fields of the theory by $\cE$. 

This is a non-degenerate BV theory with pairing defined by
\[
\int \gamma \wedge (\mu \vdash \Omega) .
\] 

The action is 
\[
S = \int \gamma \bigg(\d \mu + \frac12 [\mu, \mu] \bigg) \vdash \Omega + \frac13 \int \gamma \partial \gamma \partial \gamma .
\]
where $[\cdot, \cdot]$ denotes the Lie bracket of vector fields. 

The first term describes a theory of BF type based on the local Lie algebra 
\[
\cL = \Omega^\bu(\RR) \, \Hat{\otimes} \, \PV^{1,\bu}(\CC^5) .
\]
With the term $\frac13\int \gamma \partial \gamma \partial \gamma$ we will view this theory as a deformation of a BF type theory. 

Let $\oloc(\cE)$ denote the cochain complex of local functionals.
It is equipped with the classical BRST operator 
\[
\{S, -\} = \d + \d_{\rm CE} + \delta 
\]
where $\d_{\rm CE}$ denotes (roughly) the Chevalley--Eilenberg differential for holomorphic vector fields and $\delta$ is the operator $\left\{\frac13 \int \gamma \partial \gamma \partial \gamma \;, \; \cdot \; \right\}$. 

The anomaly to satisfying the quantum master equation is a class in $H^1 (\oloc(\cE))$. 

There is a decreasing filtration on $\oloc(\cE)$ whose $p$th layer $F^p \oloc(\cE)$ consists of local functionals which are at least $p$-linear in the field $\gamma$. 
\brian{I'm worried that this isn't a complete filtration.}

Roughly, the $p$th layer in the associated graded of this filtration is the local cohomology of holomorphic vector fields on $\CC^5$ with values in a particular local module corresponding to $p$-linear local functionals of the field $\gamma$. 

\begin{itemize}
\item At the zeroth layer $p=0$, the associated graded is the local Lie algebra cochains $\cloc^\bu(\cL)$. 
By an argument similar to \brian{Brian and Chris}, we see that at this stage there the class of the anomaly can be  identified with a class in  $H^{12} (\fw_5)$ where $\fw_5$ is the Lie algebra of vector fields on the formal $5$-disk.
This cohomology is trivial $H^{12}(\fw_5) = H^{13}({\rm BU}(5)) = 0$. 
\brian{check this first isomorphism, there might be a connecting differential}

\item \brian{at the next layer it seems like we should be looking at $H^\bu(\fw_5, \fw_5)$.}
\end{itemize}

\end{document}



zshhjknbhucvvgbbnhhggFDGFGRTRYYRTDFSXR465768787YHGBCVBNBVBCBVNBCGFH,J.LK;K,HNGDGFgyggcbgnhHKJLK;OLLJGjhhjghuykjogfytcjfukkh




CGFTHFHXCVXBGFVFSGDYGFHDGYGdvffdfgdrt544465y7hgff

 dgfhjgyfhghj dfdgfsdgfdgfgdff 
 dhgngsfhffdgfgdtd
 
 
 tdtdg
 f
 hy
 ftttegfdtrfgdfdfdgfgdffgdtrgdfdfrredfr4wertryutrt45
 6
 
 
 Laszlofuyryjgfghfjhhfgth
 dfxgcfdtfhgyugdghjgytfghdfgnhghytfdhghdttrhykuscvbghkiotgf


