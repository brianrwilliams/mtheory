\documentclass[11pt]{amsart}

\usepackage{macros-mtheory,amsaddr}

\addbibresource{cfs.bib}

%\linespread{1.2} %for editing
%\usepackage{mathpazo}

\begin{document}

\title{Twisted 11-dimensional supergravity, I}
\author{Surya Raghavendran}
\address{Perimeter Institute for Theoretical Physics \\ 31 Caroline Street North \\ 
Waterloo, Ontario N2L 2Y5\\ Canada}
\email{??}
\author{Ingmar Saberi}
\address{Ludwig-Maximilians-Universit\"at M\"unchen \\ Fakult\"at f\"ur Physik \\ Theresienstra\ss{}e 37 \\ 80333 M\"unchen \\ Deutschland}
\email{i.saberi@physik.uni-muenchen.de}
\author{Brian R. Williams}
\address{School of Mathematics \\ University of Edinburgh \\ Edinburgh EH9 3FD \\ Scotland}
\email{brian.williams@ed.ac.uk}
\begin{abstract}
to do
\end{abstract}
\maketitle

\newpage 

\section{A family of moduli spaces} 

In this section we define the central theory of study, within the Batalin--Vilkovisky formalism.
The theory will be defined on any eleven-dimensional manifold of the form $X \times L$ where $X$ is a Calabi--Yau five-fold and $L$ is a smooth oriented one-manifold.

\subsection{Divergence-free vector fields} 

\subsubsection{}
\label{sec:divfree}
We set up some notations and conventions in the context of complex geometry. 
Let $V$ be a holomorphic vector bundle on a complex manifold $X$. 
If $j$ is an integer, we let $\Omega^{0,j}(X, V)$ denote the space of anti-holomorphic Dolbeault forms of type $j$ on with values in $V$.
The $\dbar$ operator for $V$ is $\dbar \colon \Omega^{0,j}(X, V)\to \Omega^{0,j+1}(X)$ defines the Dolbeault complex of $V$
\[
  \Omega^{0,\bu}(X, V) = \left(\Omega^{0,j}(X, V)[-j] \; , \; \dbar\right) .
\]
This is a free resolution for the sheaf of holomorphic sections of $V$.

Suppose $X$ is a Calabi--Yau manifold with holomorphic volume form $\Omega$.
The divergence $\div(\mu)$ of a holomorphic vector field $\mu$ is defined by the formula
\[
\div (\mu) \wedge \Omega = L_\mu (\Omega)
\]
where, on the right hand side, we mean the Lie derivative of $\Omega$ with respect to $\mu$.

Let $\T_X$ denote the holomorphic tangent bundle and consider its Dolbeault complex $\Omega^{0,\bu}(X , \T_X)$ resolving the sheaf of holomorphic vector fields. 
The divergence operator extends to the Dolbeault complex to yield a map of cochain complexes 
\[
\div \colon \Omega^{0,\bu}(X , \T_X) \to \Omega^{0,\bu}(X) .
\]

The kernel of this map resolves the sheaf of holomorphic divergence-free vector fields $\Vect_0^{hol}(X)$.
In fact, we can form the complex 
\beqn\label{eqn:cplx1}
\begin{tikzcd}
\ul{0} & \ul{1} \\
\Omega^{0,\bu}(X , \T_X) \ar[r, "\div"] & \Omega^{0,\bu}(X) .
\end{tikzcd}
\eeqn
The $\dbar$ operator, as always, is left implicit. 
By the holomorphic Poincar\'e lemma, the embedding of the sheaf $\Vect^{hol}_0(X)$ into the degree zero piece of this complex is a quasi-isomorphism. 

There is a direct way to extend the Lie bracket of vector fields to the complex \eqref{eqn:cplx1}. 
Denote by $\mu$ an element of $\Omega^{0,\bu}(X , \T_X)$ and $\nu$ an element of $\Omega^{0,\bu}(X)$ (for simplicity in notation, we will not expand the anti-holomorphic dependence). 
The Lie bracket defined by the formulas
\begin{align*}
[\mu, \mu'] & = L_\mu \mu' \\
[\mu, \nu] & = L_\mu \nu 
\end{align*}
is compatible with $\div$ and endows \eqref{eqn:cplx1} with the structure of a sheaf of dg Lie algebras.
We will refer to this sheaf by the symbol $\cL_0(X)$, or just $\cL_0$ if $X$ is understood. 

The sheaf $\cL_0$ has the structure of a {\em local} dg Lie algebra, see see \cite[??]{CG2}.
This means that as a graded sheaf, $\cL_0$ is the smooth sections of a graded vector bundle and the differential and Lie bracket are given by differential and bidifferential operators, respectively.


\parsec[sec:Linfty]

Recall that an $L_\infty$ algebra is a $\ZZ$-graded vector space $\cL$ together with the data of a square-zero, degree $+1$ derivation $\delta$ of the free commutative graded algebra $\Sym\left(\cL^\vee [-1] \right)$. 
The Chevalley--Eilenberg cochain complex is 
\[
\left(\Sym\left(\cL^\vee [-1] \right), \delta\right) .
\]
The Taylor components of $\delta$ define higher brackets $\{[-]_k\}_{k=1,2,\ldots}$ where $[-]_k \colon \cL^{\times k} \to \cL[2-k]$. 

An $L_\infty$ morphism $\Phi: \cL \rightsquigarrow \cL'$ is the same datum as a map of commutative dg algebras 
\deq{
  \Phi^*: \clie^\bu(\cL') \to \clie^\bu(\cL)
}
between their respective Lie algebra cochains. It follows from this that \emph{any} automorphism $\Phi$ of the free commutative algebra on $\cL^\vee[-1]$ defines a new model of the $L_\infty$ algebra $\cL$, for which the Chevalley--Eilenberg differential is obtained by conjugating $\delta_\cL$ by~$\Phi$, and where $\Phi$ itself defines the $L_\infty$ isomorphism.

$\cL_0$ is the sheaf~\eqref{eqn:cplx1} resolving divergence-free vector fields equipped with the dg Lie algebra structure constructed in the previous section.
We consider the following automorphism of~$\Sym(\cL_0^\vee[-1])$, defined by its action on generators:\ingmar{notation for operators vs fields}
\deq[eq:newbase]{
%    \Psi_\infty:  \nu &\mapsto -\frac{\nu}{1-\nu}, \quad \mu \mapsto -\mu \\
    \Psi_\infty: \nu \mapsto 1 - e^{-\nu}, \quad \mu \mapsto e^{-\nu} \mu.
}
This map defines a new model of the $L_\infty$ algebra with underlying graded vector space the same as \eqref{eqn:cplx1}, which we will call $\cL_\infty$. 
The formulas for the automorphism above clearly arise from maps of vector bundles and hence endow $\cL_\infty$ with the structure of a local $L_\infty$ algebra, meaning all operations are given by polydifferential operators.  

The notation refers to the fact that this new model has nonvanishing $L_\infty$ brackets of every order. 
It is this new model that we will use to define the eleven-dimensional theory of twisted supergravity (as well as the family of analogous formal moduli problems on products of odd Calabi--Yau manifolds with~$\R$). 


%The previous proposition characterizes the $L_\infty$ model for divergence-free vector fields that we will use to define the 11-dimensional theory of twisted supergravity. 
%Hereon, we denote by $\cL_0 = \cE_0[-1]$ this local $L_\infty$ algebra. 

%We can unpack Proposition \ref{prop:Linfty} to describe the $L_\infty$ structure on $\cL_0$ explicitly. 
We can describe the $L_\infty$ structure on our new model $\cL_\infty$ more explicitly.
Recall that we have two types of elements: $\mu \in \PV^{1,\bu}$ and $\nu \in \PV^{0,\bu}[-1]$. 
The first few nonzero brackets are
\begin{align*}
[\mu]_1 & = \dbar \mu + \div \mu \\
[\mu_1,\mu_2]_2 & = \div (\mu_1 \mu_2) \\
[\nu, \mu_1,\mu_2]_3 & = \div(\nu \mu_1 \mu_2) \\
[\nu_1,\nu_2, \mu_1,\mu_2]_4 & = \# \div(\nu_1 \nu_2 \mu_1 \mu_2).
\end{align*}
\brian{coefficients}
For $k \geq 2$ the general formula for the $k$-ary bracket is 
\[
[\nu_1,\nu_2, \ldots, \nu_{k-2}, \mu_1,\mu_2]_{k} = \# \div(\nu_1 \cdots \nu_k \mu_1 \mu_2) .
\]

%We describe the explicit $L_\infty$ automorphism $\Psi \colon (\cL_0)^{L_\infty} \rightsquigarrow (\cL_0)^{strict}$ intertwining the strict dg Lie structure on $\cL_0$ and this $L_\infty$ structure.
%The linear term $\Psi^{(1)} = \id$ is the identity map. 
%The higher terms $\Psi^{(n)}$ are defined by 
%\brian{someone check me}
%\begin{align*}
%\Psi^{(n)} (\nu_1,\ldots, \nu_{n-k},\mu_1,\ldots, \mu_k) & = \delta_{k=1} \nu_1 \cdots \nu_{n-1} \mu_1 . \\
%\end{align*}


\parsec
There is yet another $L_\infty$ model for divergence-free vector fields that we remark on here.
\ingmar{I want to write the other automorphism later, I think; otherwise we have to write the composition, since we presented it on bcov}


\subsection{Theories of BF type}

\parsec
Suppose that $\cL$ is an $L_\infty$ algebra with $L_\infty$ operations $\{[-]^\cL_k\}_{k=1,2,\ldots}$ and that $(A, \d_A)$ is a commutative dg algebra. 
The graded vector space $\cL \otimes A$ is equipped with the natural structure of an $L_\infty$ algebra with operations $\{[-]_k\}_{k=1,2,\ldots}$ defined by
\begin{align*}
[x \otimes a]_1 & = [x]^\cL_1 \otimes a + (-1)^{|x|} x \otimes \d_A a \\
[x_1 \otimes a_1, \ldots , x_k \otimes a_k]_k & = [x_1,\ldots,x_k]^\cL_k \otimes (a_1 \cdots a_k), \qquad k \geq 2 .
\end{align*}

We apply this construction, taking $\cL$ to be the sheaf resolving divergence-free holomorphic vector fields on a Calabi--Yau manifold $X$ equipped with either the strict dg Lie algebra structure $\cL_0(X)$ or its non-strict $L_\infty$ structure $\cL_\infty (X)$. 
The algebra $A$ will be the smooth de Rham complex $(\Omega^\bu(L) , \d_L)$ where $L$ is any smooth manifold. 

We thus obtain the structure of an dg Lie algebra on $\cL_0(X) \otimes \Omega^{\bu}(L)$ or an $L_\infty$ algebra $\cL_\infty(X) \otimes \Omega^\bu(L)$.
These define equivalent local $L_\infty$ algebras on the product manifold $X \times L$. 

\subsubsection{}

Associated to any local $L_\infty$ algebra is a classical field theory in the BV formalism that one refers to as BF theory.
We recall the construction. 
Let $\cL$ be any local $L_\infty$ algebra on some manifold $M$, it is the sheaf of sections of some graded vector bundle $L$. 
For a section $A \in \cL$, introduce the `higher curvature map' defined by the formula
\[
\mathsf{F}_A = [A]_1 + \frac12 [A,A]_2 + \frac{1}{3!} [A,A,A]_3 + \cdots .
\]

The fields of the associated BV theory are pairs
\[
  (A, B) \in \cL[1] \oplus \cL^{!}[-2] .
\]
Here $\cL^!$ denotes the sheaf of sections of the bundle $L^* \otimes {\rm Dens}$, where ${\rm Dens}$ is the bundle of densities. 
The shifted symplectic BV pairing is the obvious integration pairing between $A$ and $B$. 

The action functional reads $S_{\rm BF} = \int_M B \, \mathsf{F}_{A}$ which leads to the equations of motion $\mathsf{F}_{A} = 0$ and $\mathsf{D}_A B= 0$ where $\mathsf{D}_A$ is the higher covariant derivative along $A$. 

We thus obtain a theory in the BV formalism on the product manifold $X \times L$ associated to both local $L_\infty$ algebras $\cL_0(X) \otimes \Omega^{\bu}(L)$ and $\cL_\infty(X) \otimes \Omega^\bu(L)$.

\parsec
For concreteness, we spell out the fields of the theories we have constructed on $X \times L$.
In both cases, the space of fields equipped with the linear BRST operator is
\begin{equation}
  \label{eq:sympfields} 
  \begin{tikzcd}[row sep = 1 ex]
    -n & -n + 1 & -1 & 0 \\ \hline
    \Omega^{0,\bu}(X;L) \ar[r, "\del"] & \Omega^{1,\bu}(X;L) & 
     \PV^{1,\bu}(X; L) \ar[r, "\div"] & \PV^{0,\bu}(X; L).
\end{tikzcd}
\end{equation}
We denote the fields $(\beta,\gamma,\mu,\nu)$ respectively.
We are using the shorthand notation $\PV^{i}(X;L) = (\PV^{i,\bu}(X)\otimes \Omega^\bu(L), \dbar + \d_{L}))$.

The natural pairing between $\PV^i(X;L)$ and~$\Omega^i(X;L)$ is of degree $-\dim_\C(X) -\dim_\R(L)$. 
As such, the $\Z$-grading indicated in~\eqref{eq:sympfields} equips the sheaf of fields with a $(-1)$-shifted symplectic structure, provided that we choose the shift to be given by
\deq{
  n = \dim_\C(X) + \dim_\R(L) - 1.
}

We have constructed two equivalent descriptions of the BF theory which share the linear BRST complex \eqref{eq:sympfields}.

Explicitly, the action functional for BF theory associated to the local dg Lie algebra $\cL_0(X) \otimes \Omega^{\bu}(L)$ is
\deq{
  S_0 =  \beta (\dbar + \d_L) \nu +  \gamma (\dbar + \d_L) \mu +  \beta\partial_\Omega \mu + \frac{1}{2}  \gamma [\mu,\mu] +  \beta[\mu,\nu].
}
As in the Lie algebra structure of this strict model, notice that the Schouten--Nijenhuis bracket appears explicitly. 

The action functional of BF theory associated to $\cL_\infty(X) \otimes \Omega^{\bu}(L)$ is of infinite order. 
(In fact, it is related to the BCOV action functional via a procedure we outline below.)\ingmar{or somewhere} 
Explicitly, this action functional is
\deq{
  S_\infty =  \beta (\dbar + \d_L) \nu +  \gamma (\dbar + \d_L) \mu +   \beta \partial_\Omega \mu + \frac12 \frac{1}{1-\nu} (\partial \gamma) \mu^2.
}

We demonstrated above that the two local $L_\infty$ algebras on which these BF theories are based are equivalent. As such, the BF theories are also equivalent; the map~\eqref{eq:newbase} extends uniquely to an automorphism of BV theories.
Explicitly, the automorphism is
\begin{multline}
  \mu \mapsto e^{-\nu} \mu, \qquad \nu \mapsto 1-e^{-\nu} \\
  \beta \mapsto (\beta - \mu \gamma) e^{\nu},\qquad \gamma \mapsto e^{\nu} \gamma .
\end{multline}
\subsection{A family of deformations} 

\parsec
%check CME
We now specialize to the case where $X$ is an odd Calabi--Yau manifold and $L$ an odd-dimensional smooth manifold. 
In this case, there exists a consistent deformation of the BF theory we have constructed by a further term which is not of BF type. That is, we cannot define this deformation at the level of the local $L_\infty$ algebra resolving divergence-free holomorphic vector fields; it is defined only at the level of the full BF theory.

Recall that such deformations are governed by the classical master equation: the parameterized family of actions $S + g J$ is sensible only if 
\deq{
  \{S + g J, S + g J \} = 0.
}
Since this must hold for all $g$, and since the undeformed action $S$ is already a solution to the classical master equation, this reduces to the two conditions 
\deq[eq:2cond]{
  \{S,J\} = \{J,J\} = 0.
}
\begin{thm}
  Let $X$ be a Calabi--Yau manifold of complex dimension $2k+1$ and $L$ a smooth manifold of real dimension $2\ell + 1$, and consider the BV theory $(\cE, S_\infty)$ on $X \times L$ constructed above. The local functional 
  \deq{
    J = \gamma (\del \gamma)^k,
  }
  where $\gamma \in \Omega^1(X;L)$, defines a deformation of~$(\cE,S_\infty)$ as a $\Z/2\Z$-graded BV theory.
\end{thm}
\begin{proof}
  First off, we consider grading issues. In the $\Z$-grading on the free theory, the component $\gamma^{1,i;j}$ sits in degree $-2k-2\ell -2+i+j$. The local functional $J$ includes terms that are degree-$(k+1)$ polynomials in the components of $\gamma$, for which $\sum i = 2k+1$ and $\sum j = 2\ell + 1$. As such, we have
  \deq{
    \deg(J) =-  (2k+ 2\ell+ 2)(k+1) + 2k+2\ell + 2 = - k(2k+2\ell+2).
  }
  Thus $J$ defines an even local functional; of course, it does not define a local functional of degree zero.

  We then consider the classical master equation, in the form of the conditions~\eqref{eq:2cond}. It is immediate from the form of the shifted Poisson bracket that $\{J,J\} = 0$, since $J$ depends only on the $\gamma$ field. It remains to check that $\{I_\infty,J\} = 0$. For the quadratic term, we note that 
  \deq{
    \{\beta\div\mu, J\} = (k+1) \del\beta (\del \gamma)^k = 0,
  }
  because total derivatives are equivalent to zero as local functionals. The cubic term takes the form
  \deq{
    \left\{ \frac{1}{1-\nu} \del\gamma \mu^2, \gamma(\del\gamma)^k \right\} = \frac{2(k+1)}{1-\nu} (\del\gamma\vee\mu) (\del\gamma)^k.
}
This expression is zero for symmetry reasons. Recall that $\del\gamma$ is a two-form, and that the expression must be a totally symmetric local functional in $(k+1)$ copies of this two-form. We can ask whether such a  contraction exists just at the level of $\lie{sl}(2k+1)$ representation theory by computing the decomposition of the two-form  
\end{proof}

\subsection{An alternative description} 
So far, we have described a family of BV theories on $X \times L$ where $X$ is an odd-dimensional Calabi--Yau manifold and $L$ is an odd-dimensional smooth manifold. 
The space of fields \eqref{eq:sympfields} is equipped with an odd symplectic pairing. 


There is a related, alternative description of this formal moduli space which is almost equivalent. 
Consider the resolution of holomorphic closed two-forms 
\beqn\label{eqn:twoform}
\Omega_{cl}^{2,hol}(X) \simeq \Omega^{2,\bu}(X) \xto{\del} \Omega^{3,\bu}(X)[-1] \to \cdots 
\eeqn
where we have left the $\dbar$-differential implicit, as always. 

There is a map from the two-term complex 
\[
\Omega^{0,\bu}(X)[1] \to \Omega^{1,\bu}(X)
\]
to the resolution \eqref{eqn:twoform} defined by the holomorphic de Rham operator $\del$. 
This map is almost a quasi-isomorphism of sheaves; it differs by copy of constant functions $\CC[1]$ in cohomological degree $-1$. 

Similarly, we can tensor with the dg algebra of de Rham forms on $L$ to obtain a map of $\ZZ/2$-graded cochain complexes
\[
\bigg(\Pi \Omega^{0,\bu}(X;L) \xto{\del} \Omega^{1,\bu}(X;L) \bigg) \xto{\del} \bigg(\Omega^{2,\bu}(X;L) \xto{\del} \Pi \Omega^{3,\bu}(X;L) \to \cdots \bigg) .
\]
Again, this map is a quasi-isomorphism up to a copy of $\Pi \CC$.
The left-hand side contains the fields $(\beta, \gamma)$. 
This map has the effect of sending $\beta \mapsto 0$ and $\gamma \mapsto \partial \gamma \in \Omega^{2,\bu}(X;L)$. 

The basic idea is to replace the complex where $(\beta,\gamma)$ live by this resolutoin of closed two-forms. 
The new space of fields is 
\begin{equation}
  \label{eq:poissfields} 
  \begin{tikzcd}[row sep = 1 ex]
     - & + & - & + \\ \hline
     \PV^{1,\bu}(X; L) \ar[r, "\div"] & \PV^{0,\bu}(X; L) \\
     & \Omega^{2,\bu}(X;L) \ar[r, "\del"] & \Omega^{3,\bu}(X;L) \ar[r] & \cdots 
\end{tikzcd}
\end{equation}

\begin{prop}
The complex \eqref{eq:poissfields} is equipped with the structure of an interacting Poisson BV theory.
Let $\cG = \cG(X \times L)$ be the resulting $L_\infty$ algebra...
\end{prop}

Heuristically, we can write the action in the following non-local form
\[
\frac{1}{1-\nu} \mu^2 \eta + (\del^{-1} \eta) \eta^2 .
\]
\brian{are there more terms involving higher forms?}

The first few nonzero brackets are
\begin{align*}
[\mu]_1 & = \dbar \mu + \div \mu \\
[\eta]_1 & = \dbar \eta + \del \eta \\ 
[\mu_1,\mu_2]_2 & = \div (\mu_1 \wedge \mu_2) \\
[\mu, \eta]_2 & = \mu \vee \del \eta \\
[\eta_1,\eta_2]_2 & = \# \Omega^{-1} \vee (\eta_1 \wedge \eta_2)  \\
[\nu, \mu_1,\mu_2]_3 & = \div(\nu \mu_1 \mu_2) \\
[\nu, \mu,\eta]_3 & = \nu \mu \del \eta
\end{align*}
\brian{coefficients}
For $k \geq 2$ the general formula for the $k$-ary bracket is 
\begin{align*}
[\eta_1,\eta_2]_2 & = \# \Omega^{-1} \vee (\eta_1 \wedge \eta_2)  \\
[\nu_1,\nu_2, \ldots, \nu_{k-2}, \mu_1,\mu_2]_{k} & = \# \div(\nu_1 \cdots \nu_{k-2} \mu_1 \wedge \mu_2) \\ 
[\nu_1,\nu_2, \ldots, \nu_{k-2}, \mu,\eta]_{k} & = \# \nu_1 \cdots \nu_{k-2} (\mu \vee \del \eta) .
\end{align*}

\parsec[]
Consider the theory on $\CC^3 \times \RR$. 
There is a quasi-isomorphism of $L_\infty$ Lie algebras $\CC \simeq \cG (\CC^3 \times \RR)$. 

\parsec[]
Consider the theory on $\CC^5 \times \RR$. 

\begin{prop}
There is a quasi-isomorphism of super $L_\infty$ algebras 
\[
E(5,10) \xto{\simeq} \cG(\CC^5 \times \RR) .
\]
This lifts to map of super $L_\infty$ algebras $E(5,10) \to \cL(\CC^5 \times \RR)$ whose kernel is $\CC$.
\end{prop}

\subsection{Global symmetry algebra}

\subsection{Further equivalences}

\subsection{Equations of motion}

%whatever we can say about the formal moduli space

\subsection{Boundary conditions}

%foreshadow the Horava--Witten picture

\subsection{One-loop quantization} 

%cite result on thf quantization.



\subsubsection{}

Define the action by the group $\CC^\times$ on the complex $\til{\cE}_{KS}$ as follows.
Declare the first line in \eqref{eqn:tilks} is weight zero and the second line is weight $+1$. 
%Declare $\mu, \nu$ have $\CC^\times$ weight zero and $\beta,\gamma$ have $\CC^\times$ weight $+1$. 
The linear BRST operator is clearly weight zero and the shifted Poisson structure $\Pi_{KS}$ is of weight $+1$. 
Thus, the shifted Poisson bracket $\{-,-\}_{KS}$ is of $\CC^\times$ weight $+1$.

\begin{lem}
The restricted action $\til{I}_{BCOV}$ has $\CC^\times$ weight $-1$. 
In particular, the non-linear BRST operator 
\beqn\label{eqn:KSbrst}
\delta_{BCOV} = Q_{KS} + \{\til{I}_{BCOV},-\}_{KS}
\eeqn
is $\CC^\times$ weight zero.
\end{lem}
\begin{proof}
%The map $\Phi \colon \cE_{pot} \to \til{\cE}_{KS}$ preserves $\CC^\times$ weight, so it suffices to prove that $\til{I}_{BCOV}$ is weight $+1$. 
We will make use of two gradings on $\til{\cE}_{KS}$. 
The first is holomorphic polyvector field type, and the second is descendant degree.
The summand $u^l \PV^{k,\bu}$ is polyvector type $k$ and descendant degree $l$. 

We introduce notation in this proof that won't be used later on. 
Let $\alpha$ denote a super field living in the first line of \eqref{eqn:tilks} and let $\beta$ denote a super field living in the second line of \eqref{eqn:tilks}.
If a field of type $\alpha$ has polyvector type $k$ then it has descendant degree $1-k$, $k=0,1$. 
If a field of type $\beta$ has polyvector type $l$ then it has descendant degree $d-2-l$.

A homogenous polynomial degree term in the BCOV action $I_{BCOV}$ is a linear combination of functionals of the form
\[
 \alpha_1 \wedge \cdots \wedge \alpha_m \wedge \beta_1 \wedge \cdots \wedge \beta_n  .
\]
Let $k_i$ be the polyvector field type of $\alpha_i$ and let $l_j$ be the polyvector field type of $\beta_j$. 
In order for this expression to contribute nontrivially to the BCOV action two constraints must hold:
\begin{align*}
\sum_{i=1}^m k_i + \sum_{j=1}^n l_j & = d \\
\sum_{i=1}^m (1-k_i) + \sum_{j=1}^n (d-2-l_j) & = m+n-3 .
\end{align*}
The first constraint ensures that the integrand is of top polyvector degree. 
The second constraint is on the descendant degree, see \brian{ref above}. 

Simplifying these two equations, one finds the condition
\[
m+n-3 = m+(d-2)n - d 
\]
which implies $n=1$, as desired. 
\end{proof}

Consider the cochain complex $\left(\oloc(\til{\cE}_{KS}) \, , \, \delta_{KS} \right)$ of all local functionals equipped with the non-linear BRST differential \eqref{eqn:KSbrst}.  
The $\CC^\times$ weight zero subcomplex is 
\[
\big(\oloc(\til\cE_{KS})^{(0)} \, , \, \delta_{KS} \big) = \big(\oloc(\cE_0) \, , \, \delta_{KS} \big) 
\]
where $\cE_0 = \PV^{1,\bu}[1] \oplus u \PV^{0,\bu}$ is the subcomplex of the fields $\til\cE_{KS}$ comprising the first line of \eqref{eqn:tilks}.

\begin{prop}
\label{prop:Linfty}
The differential $\delta_{KS}$ acting on the weight zero fields induces a local $L_\infty$ algebra structure on the complex of vector bundles 
\[
\cE_0 [-1] = \PV^{1,\bu} \oplus \PV^{0,\bu}[-1]
\]
with $[-]_1 = \dbar + \div$. 
Furthermore, this $L_\infty$ algebra structure is equivalent to the strict dg Lie algebra structure on the resolution of divergence-free vector fields defined in \S \ref{sec:divfree}. 
\end{prop}

\section{Relationship to physical supergravity}

\subsection{Component fields}

%Can we try to roughly say where each of our twisted fields come from in the untwisted theory? I imagine this section being pretty chill and mostly to orient physicists. 

\subsection{Residual supersymmetry} 

%m2brane

\parsec[sec:susy]

The (complexified) eleven-dimensional supertranslation algebra is a complex super Lie algebra of the form
\[
  \ft_{11d} = V \oplus \Pi S
\]
where $S$ is the (unique) spin representation and $V \cong \CC^{11}$ the complex vector representation, of~$\lie{so}(11, \CC)$. 
The bracket is the unique surjective $\lie{so}(11,\CC)$-equivariant map from the symmetric square of~$S$ to~$V$;
this decomposes into three irreducibles, 
\beqn\label{eqn:decomp}
  \Sym^2(S) \cong V \oplus \wedge^2 V \oplus \wedge^5 V.
\eeqn
Denote by $\Gamma_{\wedge^1}, \Gamma_{\wedge^2}, \Gamma_{\wedge^5}$ the projections onto each of the summands above. 
The bracket in $\ft_{11d}$ is defined using the first projection
\[
[\psi, \psi'] = \Gamma_{\wedge^1} (\psi, \psi') .
\]
The super Poincar\'{e} algebra is
\[
  \lie{siso}_{11d} = \lie{so}(11 , \CC) \ltimes \ft_{11d} .
\]
The $R$-symmetry is trivial in 11-dimensional supersymmetry. 


\parsec[sec:m2brane]

Extensions of the supersymmetry algebra correspond to the existence of branes in the original theory of supergravity \brian{good reference?}. 
In 11d, there are two such extensions corresponding to the M2 brane and the M5 brane.
We begin by describing a dg Lie algebra model for the M2 brane algebra, then we will discuss its relationship to other descriptions as a Lie 3-algebra \cite{Basu_2005,Bagger_2007,fiorenza2015super}. 

The M2 brane algebra is dg Lie algebra extension of the super Poincar\'e algebra $\lie{siso}_{11d}$.
Introduce the cochain complex $\Omega^{\bu}(\RR^{11})$ of (complex valued) differential forms on $\RR^{11}$ equipped with the de Rham differential $\d$.  
 
The super dg Lie algebra $\m2$ is the extension of $\lie{siso}_{11d}$ by the cocycle
  \[
    c_{M2} \in \clie^2\left(\lie{siso}_{11d} \; ; \; \Omega^\bu (\RR^{11})[2]\right)
  \]
  defined by the formula 
  \[c_{M2} (\psi, \psi') = \Gamma_{\wedge^2}(\psi, \psi') \in \Omega^2(\RR^{11})\]
  where $\Gamma_{\wedge^2}$ is the projection onto $\wedge^2 V$, thought of as the space of constant coefficient two-forms, as in the decomposition \eqref{eqn:decomp}.

We view $\m2$ as a $\ZZ \times \ZZ/2$-graded dg Lie algebra where the bracket is bidegree $(0,+)$ and the differential is bidegree $(1,+)$.

\parsec[sec:m2branetwist]

Fix a rank one supercharge $Q \in S$ satisfying $Q^2 = 0$.
This supercharge defines the holomorphic twist of 11-dimensional supergravity. 
\brian{cite \cite{SWspinor}}
We characterize the cohomology of the algebra $\m2$ with respect to this supercharge. 

Such a supercharge defines a maximal isotropic subspace $L \subset V$. 
We can decompose the algebra into $\lie{sl}(L) = \lie{sl}(5)$ representations by
\deq{
  V = L \oplus L^\vee \oplus \CC_t, \qquad S = \wedge^\bu L^\vee.
}
In the expression for $S$, we are omitting factors of $\det(L)^{\frac12}$ for simplicity. 
Also, $\lie{so}(11, \CC)$ decomposes as
\[
\lie{sl}(5) \oplus \wedge^2 L \oplus \wedge^2 L^\vee \oplus L \oplus L^\vee .
\]
Furthermore, the spinorial representation can be identified with
\[
S = \wedge^\bu (L^\vee) = \CC \oplus L^\vee \oplus \wedge^2 L^\vee \oplus \wedge^3 L^\vee \oplus \wedge^4 L^\vee \oplus \wedge^5 L^\vee .
\]
The element $Q$ lives in the first summand.
Let ${\rm Stab}(Q) \subset \lie{so}(11,\CC)$ be the stabilizer of $Q$. 
This is a parabolic subalgebra whose Levi factor is $\lie{sl}(5)$.

\begin{prop}\label{prop:model}
Fix a holomorphic supercharge $Q \in S$ and let $\m2^Q$ be the dg Lie algebra with bracket that on $\m2$ and with the differential $\d + [Q,-]$. 
As a $\ZZ/2$ graded space, the cohomology $H^\bu(\m2^Q)$ is
\[
    L^\vee \oplus {\rm Stab}(Q) \oplus \Pi \left(\wedge^3 L^\vee\right) \oplus \CC
  \]
whose elements we denote by $(v, m, \psi, c)$.

The transferred $L_\infty$ structure on $H^\bu(\fg) \cong H^\bu(\m2^Q)$ is a Lie-3 algebra given the obvious extension of ${\rm Stab}(Q)$ together with the brackets
\begin{align*}
[\psi, \psi']_2 & = \psi \wedge \psi' \in \wedge^4 L \cong L^\vee_v \\
[v, v', \psi]_3 & = \<v \wedge v', \psi\> \in \CC_c .
\end{align*}
\end{prop}

\parsec[]

As a consequence of the above proposition, we see that the dg Lie algebra $(\m2,Q)$ is {\em not} formal; there is a 3-ary $L_\infty$ bracket present in cohomology. 
Instead of this $L_\infty$ description, there is the following minimal model of this dg Lie algebra, which we will use to relate to twist of 11d supergravity. 

\begin{prop}
\label{prop:gmodel}
Let $\fg$ denote the following $\ZZ/2$ graded dg Lie algebra which as a cochain complex is
\[
H^\bu(\m2^Q) \oplus (\Pi L \xto{\id} L)  .
\]
Denote the elements of the second summand by $(\tilde{\lambda}, \lambda)$. 
The nontrivial Lie brackets are
\begin{align*}
[v,\lambda] & = \<v, \til\lambda\> \in \CC_c \\ 
[v,\psi] & = \<v, \psi\> \in \Pi L_{\Tilde{\lambda}} \\
[\psi, \psi'] & = \psi \wedge \psi' \in L^\vee_v  .
\end{align*}
There is an $L_\infty$ map 
\[
\fg \rightsquigarrow \m2^Q
\] 
which is a quasi-isomorphism of complexes.  
\end{prop}
\begin{proof}
The supercharge $Q$ is odd and of cohomological degree zero.
Since the differential on $\m2$ is even of cohomological degree $+1$, only a totalized $\ZZ/2$ grading makes the differential $\d + [Q,-]$ homogenous. 

The cohomology of the non-centrally extended algebra was computed in \cite{SWspinor}, we briefly recall the result. 
The element $Q$ only acts nontrivially on the summands $\wedge^4 L^\vee$ and $\wedge^5 L^\vee$ in $S$. 
The image of $\wedge^4 L^\vee \cong L$ trivializes the antiholomorphic translations while the image of $\wedge^5 L^\vee$ trivializes the time translation.
So, of the translations, only the holomorphic ones $L^\vee$ survive.
The map 
\[
[Q,-] \colon \lie{so}(11,\CC) \to S 
\] 
is the projection onto $\wedge^0 L^\vee \oplus \wedge^1 L^\vee \oplus \wedge^2 L^\vee$. 
The kernel of $[Q,-]$ is the stabilizer ${\rm Stab}(Q)$.

In summary, the space of odd translations which survive cohomology is $\wedge^3 L^\vee \cong \wedge^2 L$.
This completes the calculation of the cohomology. 

We embed $\fg$ into $\m2^Q$ in the following way: ${\rm Stab}(Q)$ and $L^\vee$ sit inside in the evident way.
The central element maps to $c \mapsto \pm 1 \in \Omega^0(\RR^{11})$.
The summand $L_\lambda$ is mapped to the linear functions in $\Omega^0(\RR^{11})$ and $\Pi L_{\Tilde{\lambda}}$ is sent to the constant coefficient one-forms in $\Pi \Omega^1(\RR^{11})$. 
It remains to declare where $\psi \in \wedge^2 L$ is mapped.

Define the map
\[
H \colon \Omega^2 (\RR^{11}) \to \Omega^1(\RR^{11})
\]
which sends a two-form $\alpha$ to the one-form $H \alpha$ defined by the formula
\[
(H \alpha) (x) = \int_0^x \alpha .
\]

Notice that if $\alpha$ is $\d$-closed then $\d (H \alpha) = \alpha$. 
It follows that any element $\psi \in \wedge^2 L \subset S$ can be lifted to a closed element at the cochain level in $\m2^Q$ by the formula
\[
\Tilde{\psi} = \psi - H \Gamma_{\wedge^2} (Q, \psi) \in \Pi S \oplus \Pi \Omega^1 .
\]
Thus, sending $\psi \mapsto \Tilde{\psi}$ defines a cochain map $\fg \to \m2^Q$. 

The Lie bracket $[\Tilde{\psi}, \Tilde{\psi}']$ agrees with $[\psi, \psi']$. 
On the other hand, in $\m2^Q$ there is the Lie bracket 
\[
[v,\Tilde{\psi}] = - L_v (H \Gamma_{\wedge^2} (Q, \psi)) = -\<v, \Gamma_{\wedge^2}(Q, \psi) - \d \<v, H \Gamma_{\wedge^2}(Q, \psi) .
\]
The first term agrees with the bracket $[v, \psi]_{\fg}$ in $\fg$. 
The other term is exact in $\m2^Q$ and can hence be corrected by the following bilinear  
\[
v \otimes \psi \mapsto \<v, H \Gamma_{\wedge^2} \> \in L_\lambda .
\] 
Together with the cochain map described above, this bilinear term prescribes the desired $L_\infty$ map. 

\end{proof}

\parsec[]
Consider now the $\ZZ/2$ graded dg Lie algebra $\cL$ underlying the eleven-dimensional theory on $\CC^5 \times \RR$. 

\begin{prop}
There is a map of $\ZZ/2$-graded dg Lie algebras 
\[
\fg \to \cL 
\]
where $\fg$ is the model for the $Q$-cohomology of the super Lie algebra $\m2$ from Proposition \ref{prop:gmodel}. 
In particular, the $Q$-twisted algebra $\m2^Q \simeq \fg$ is a symmetry of eleven-dimensional theory on $\CC^5 \times \RR$. 
\end{prop}
\begin{proof}
Recall the model $\fg$ takes the following form
\beqn 
\begin{tikzcd}
\ul{even} & \ul{odd} & \ul{even} \\
 L_a & \wedge^2 L_b & L^\vee \\
\wedge^2 L_a & & \\
\lie{sl}(5) \\
L_b \ar[r, "\id"] & L_c .
\end{tikzcd}
\eeqn
The subscripts in $L_a, L_b, L_c$ are used to distinguish between the various copies of $L$.

The map from the dg model $\fg$ to the fields of the twisted $11$-dimensional supergravity theory is as follows. 
\begin{align*}
 L_a & \mapsto 0 \\
 \wedge^2 L_a & \mapsto 0 \\
z_i \wedge z_j \in \wedge^2 L_b & \mapsto \frac12 (z_i \d z_j - z_j \d z_i) \in \Omega^{1,0} (\CC^5) \hotimes \Omega^0 (\RR) \\
A_{ij} \in \lie{sl}(5) & \mapsto \sum_{ij} A_{ij} z_i \in \PV^{1,0}(\CC^5) \hotimes \Omega^0(\RR) \\\frac{\partial}{\partial z_j} \in L^\vee & \mapsto
\frac{\partial}{\partial z_i} \in \PV^{1,0} (\CC^5) \hotimes \Omega^0 (\RR^5) \\ z_i \in L_b & \mapsto z_i \in \Omega^{0,0}(\CC^5) \hotimes \Omega^0 (\RR) \\
z_i \in L_c & \mapsto \d z_i \in \Omega^{1,0}(\CC^5) \hotimes \Omega^0 (\RR)
\end{align*}
One immediately verifies that this is a map of dg Lie algebras. 
\end{proof}

\parsec[]

In this section we compare to the work of \brian{Linfty refs}.
In these references, the algebra $\m2$ is defined as an $L_\infty$  
central extension of $\lie{siso}_{11d}$. 

Recall that given two spinors $\psi, \psi' \in S$ we can form the constant coefficient two-form $\Gamma_{\wedge^2} (\psi, \psi')$. 
Using this two-form we can define the following four-linear expression
\[
\mu_2 (\psi, \psi',v,v') = \<v \wedge v', \Gamma(\psi, \psi')\> .
\]
This expression is symmetric on the spinors and anti-symmetric on the vectors, therefore it defines an element in $\clie^4(\ft_{11d})$. 
In \cite{??} it is shown that this cocycle $\mu_2$ is 

%For clarity we adjust notation for $\lie{sl}(5)$-representations. Denote by $V^{1,0} = L^\vee$ the space of holomorphic translations on $\CC^5$ and $V^{\vee 1,0}$ the translation invariant holomorphic one-forms on $\CC^5$. 


\subsection{Superconformal algebras}

%6d suco 
%what about 3d N=8 suco?

\section{Dimensional reduction and compactifications} 

\subsection{Relationship to the topological string}

%conjecture for minimal twist of IIA
%check the SU(4) twist

\subsection{The theory on a three-fold}

%count hypers and vectors in 5d N=1 sugra relate to Kevin and my work with Chris

\section{The twisted $\Omega$-backround} 

\section{The non-minimal twist} 

\appendix

\subsection{Relation to BCOV theory}

\parsec[sec:pv] 
\ingmar{a section defining polyvector fields; not sure where this belongs}
Equipped with this structure, the complex we have written maps in an obvious way into the complex of polyvector fields on~$X$. Recall that one defines
\deq{
  \PV^{i,j}(X) = \Omega^{0,j}(X, \wedge^i \T_X).
}
\ingmar{Wedges look super fucked up}
The complex $\PV^{\bu,\bu}$ is equipped with two natural differentials: the Dolbeault operator $\dbar$, of $(i,j)$-degree $(0,1)$, and the holomorphic divergence operator $\div$, which carries $(i,j)$-degree $(-1,0)$. Assigning total degree $- i + j$ to $\PV^{i,j}$ thus gives the total polyvector fields the structure of a bicomplex. We will use the shorthand notation $\PV^i = (\PV^{i,\bu}, \dbar )$ for the Dolbeault resolution of holomorphic $i$-polyvector fields. 

\subsubsection{}
It will be convenient for us to cast this dg Lie algebra in a slightly different way.
We construct a different model for the Lie algebra of divergence-free vector fields whose underlying cochain complex is the same as \eqref{eqn:cplx1}. 
The distinguishing feature is that this model is no longer a dg Lie algebra, but an $L_\infty$ algebra. 

The model we use is motivated by the topological string, specifically the description of the closed topological string in terms of Kodaira--Spencer theory \cite{BCOV}.
We recall the requisite background, but refer to \cite{CLbcov1,CLbcov2,CLtypeI} for more details. 

Suppose $X$ be a Calabi--Yau manifold of dimension $d$. 
Let $\PV^{k,\bu}(X)$ denote the Dolbeault complex of the holomorphic vector bundle $\wedge^k \T_X$; we will omit $X$ in this section and write $\PV^{k,\bu}$ for simplicity.
In particular, $\PV^{k,j}$ is the space of smooth sections of the bundle $\wedge^j \T_X^* \otimes \wedge^k \T_X$. 
With this notation, the dg Lie algebra in \eqref{eqn:cplx1} is $\PV^{1,\bu} \oplus \PV^{0,\bu}[-1]$ with differential $\dbar + \div$. 

Introduce a formal parameter $u$ of cohomological degree $+2$ and consider the complex 
\beqn\label{eqn:pv1}
\PV^{\bu,\bu} [[u]] [1]
\eeqn
with differential $\dbar + u \div$ which we will often denote just by $Q_{KS}$ and refer to as the linear BRST differential. 
With our conventions, notice that $u^l \PV^{k,j}$ sits in cohomological degree $2l +k + j - 1$. 

The Schouten--Nijenhuis bracket 
\[
[-,-] \colon \PV^{k,j} \times \PV^{p,q} \to \PV^{p+k-1,j+q} 
\]
extends $u$-linearly to endow \eqref{eqn:pv1} with the structure of a dg Lie algebra.
We refer to this as the Kodaira--Spencer, or strict, dg Lie algebra structure.
We refer to the further cohomological shift 
\[
\cE_{KS} = \PV^{\bu,\bu} [[u]] [2]
\]
as the space of fields of Kodaira--Spencer theory.

Notice that the complex resolving divergence-free vector fields \eqref{eqn:cplx1} sits inside $\cE_{KS} [-1]$ as a sub dg Lie algebra as $\PV^{1,\bu} \oplus u \PV^{0,\bu}[1]$. 

\subsubsection{}

\brian{bv bracket}

\subsubsection{}

Together with the linear BRST differential $Q_{KS}$, the bracket $\{-,-\}_{KS}$ induces the structure of a dg Lie algebra the cohomological shift of the space of local functionals $\oloc(\cE_{KS})[-2d+5]$. 

\begin{thm}[\cite{CLbcov1}]
The BCOV action 
\[
I_{BCOV} \in \oloc \left(\cE_{KS} \right) 
\]
satisfies the classical master equation 
\[
Q_{KS} I_{BCOV} + \frac12 \left\{I_{BCOV}, I_{BCOV}\right\}_{KS} = 0 .
\]
In other words, $I_{BCOV}$ determines a Maurer--Cartan element in the dg Lie algebra $\oloc(\cE_{KS})[-2d+5]$.
\end{thm}

This action induces the square-zero operator 
\[
\delta_{BCOV} = Q_{KS} + \{I_{BCOV}, -\} 
\]
acting on observables of the Kodaira--Spencer fields. 
In other words, it defines a non-linear BRST operator; in turn it
induces an $L_\infty$ structure $\{[-]_k\}_{k =1,2,3,\ldots}$ on the complex \eqref{eqn:pv1} whose linear operation is $[-]_1 = Q_{KS} = \dbar + u\div$. 

This $L_\infty$ structure is clearly not identical as the strict dg Lie algebra structure (it has operations of arbitrary high order). 
Nevertheless, it is equivalent to the strict dg Lie model: there is an $L_\infty$ automorphism which exchanges the two structures.

It is easiest to describe this automorphism at the level of observables.
Use the notation $\Sigma$ for a linear observable of the Kodaira--Spencer fields $\cE_{KS}$. 
Then, the Kodaira--Spencer dg Lie algebra structure induces the non-linear BRST operator $\delta_{KS}$ given by
\[
\delta_{KS} (\Sigma) = Q_{KS} \Sigma + \frac12 [\Sigma,\Sigma] .
\]

The non-linear change of coordinates relating the two structures is defined by
\[
\Psi \colon \Sigma \mapsto \left[u (e^{\Sigma/u} -1)\right]_+
\]
where $[-]_+$ projects onto the non-negative powers of $u$.  

\subsubsection{}

The complex resolving the (shift of) divergence-free vector fields 
\[
\left(\PV^{1,\bu}[1] \oplus u \PV^{0,\bu}[2] \, , \, \dbar + u \div\right) 
\]
is a subcomplex of $\cE_{KS}$.
If $d = \dim_\CC(X)=3$ then the shifted Poisson bivector $\Pi_{BCOV}$ restricts to a shifted Poisson bivector on this subcomplex. 
In particular, the action $I_{BCOV}$ restricts to a solution to the classical master equation for this subcomplex. 

If $d \ne 3$ then there is the following subcomplex $\til{\cE}_{KS}\subset \cE_{KS}$ defined by:
\beqn\label{eqn:tilks}
\begin{tikzcd}
\ul{-1} & \ul{0} & \cdots & \ul{d-4} & \ul{d-3} & \cdots    \\
\PV^{1,\bu} \ar[r] & u\PV^{0,\bu} & & & & &  \\
& & & \PV^{d-2, \bu} \ar[r] & u \PV^{d-3, \bu} \ar[r] & \cdots &   .
\end{tikzcd}
\eeqn
And one can check that the shifted Poisson bivector restricts to a shifted Poisson bivector on this subcomplex. 
Thus, we have the following.

\begin{prop}
\label{prop:tilbcov}
The BCOV action $I_{BCOV}$ restricts to a solution of the classical master equation on $\til{\cE}_{KS}$ that we denote by $\til{I}_{BCOV}$. 
\end{prop}
%Applying this to the $k$th exterior power of the holomorphic tangent bundle $V = \wedge^k \T_X$, we obtain a resolution of the sheaf of $k$-linear polyvector fields which we denote by $\PV^{k,\bu} (X) = \Omega^{0,\bu}(X, \wedge^k \T_X)$. 

%\subsubsection{}
%
%Consider the following cochain complex $\cE_{pot}$:
%\beqn\label{eqn:E}
%\begin{tikzcd}
%\ul{-1} & \ul{0} & \cdots & \ul{d-5} & \ul{d-4} &   \\
%\PV^{1,\bu} \ar[r,"\div"] & \PV^{0,\bu} & \cdots & & \\
%& & & \Omega^{0, \bu} \ar[r,"\del"] & \Omega^{1, \bu} 
%\end{tikzcd}
%\eeqn
%We will refer to the components of the fields using the notation
%\begin{align*}
%(\mu, \nu) & \in \Pi \PV^{1,\bu}(X)[1] \oplus \PV^{0,\bu}(X) \\
%(\beta, \gamma) & \in \Omega^{0,\bu}(X)[d-5] \oplus \Omega^{1,\bu}(X) [d-4].
%\end{align*}
%The expressions $\mu,\nu,\beta,\gamma$ are still super fields in the sense that they have components in all anti-holomorphic Dolbeault degree. 
%
%Let 
%\[
%\begin{array}{rccc}
% \colon \Omega_c^{0,\bu}(X) & \to & \CC [??] \\
%\alpha & \mapsto & \int_{X} \Omega \wedge \alpha 
%\end{array}
%\]
%be integration along the holomorphic volume form. 
%Define the pairing 
%\[
%\omega \colon \cE_{pot,c} \times \cE_{pot} \to \CC [??] 
%\]
%by the formula $ \gamma \vee \mu +  \beta \nu.$
%
%\brian{defines Poisson bivector $\Pi$.}
%
%\subsubsection{}

%Define the map of cochain complexes $\Phi \colon \cE_{pot} \to \til{\cE}_{KS}$
%by the formulas
%\[
%\Phi (\mu,\nu) = \mu + u \, \nu,\qquad \Phi(\beta, \gamma) = \div \left( \gamma \vee \Omega^{-1} \right)  .
%\]
%In the last expression we have used the isomorphism $\Omega^{-1} \colon \Omega^{1,\bu} \xto{\cong} \PV^{d-1,\bu}$
%so that $\div(\gamma \vee \Omega^{-1}) \in \PV^{d-2,\bu} \subset \til{\cE}_{KS}$.
%
%\begin{prop}
%\label{prop:pot}
%The map $\Phi \colon  \cE_{pot} \to \til{\cE}_{KS}$ intertwines the linear BRST differentials and the shifted Poisson bivectors $\Phi_*\Pi = \til{\Pi}_{KS}$. 
%In particular, it defines a map of dg Lie algebras
%\[
%\Phi^* \colon \big(\oloc(\cE_{KS}),\{-,-\}_c, Q_{KS} \big) \to \big(\oloc(\cE_{pot}),\{-,-\}, Q \big)
%\]
%\end{prop}
%
%\begin{proof}
%prove this
%\end{proof}
%
%As a corollary of Proposition \ref{prop:tilbcov} and Proposition \ref{prop:pot} we have the following result. 
%
%\begin{cor}
%The BCOV action $\til{I}_{BCOV}$ restricts along $\Phi$ to a solution of the classical master equation for $\cE_{pot}$:
%\begin{align*}
%I_{pot} & = \Phi^* \til{I}_{BCOV} \\
%Q I_{pot} + \frac12 \{I_{pot},I_{pot}\} & = 0 .
%\end{align*} 
%\end{cor} 


\end{document}
