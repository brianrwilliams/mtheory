\documentclass[11pt]{amsart}

\usepackage{macros-mtheory,amsaddr,tikz-feynman}

\addbibresource{references.bib}

%\linespread{1.2} %for editing
%\usepackage{mathpazo}

\begin{document}

\title{Twisted eleven-dimensional supergravity}
\author{Surya Raghavendran}
\address{Perimeter Institute for Theoretical Physics \\ 31 Caroline Street North \\ 
Waterloo, Ontario N2L 2Y5\\ Canada}
\email{sraghavendran@perimeterinstitute.ca}
\author{Ingmar Saberi}
\address{Ludwig-Maximilians-Universit\"at M\"unchen \\ Fakult\"at f\"ur Physik \\ Theresienstra\ss{}e 37 \\ 80333 M\"unchen \\ Deutschland}
\email{i.saberi@physik.uni-muenchen.de}
\author{Brian R. Williams}
\address{School of Mathematics \\ University of Edinburgh \\ Edinburgh EH9 3FD \\ Scotland}
\email{brian.williams@ed.ac.uk}
\begin{abstract}
to do
\end{abstract}
\maketitle

\vfill\eject

\setcounter{tocdepth}{1}
\tableofcontents

%\documentclass[11pt]{amsart}
%
%%\usepackage{../macros-master}
%\usepackage{macros-fivebrane}
%
%\begin{document}

\section{Introduction}
Superconformal field theories admit a plethora of exactly computable, protected quantities, giving them a distinguished role in supersymmetric physics. Since it is believed that all supersymmetric field theories flow to superconformal fixed points, such quantities provide robust invariants of supersymmetric field theories. Examples of such a quantities are superconformal indices, which are generating functions for the R-charges of BPS local operators. Crucial steps towards a more microscopic understanding of superconformal indices were taken in \cite{}, articulated through the construction of twisting.

Introduced by Witten \cite{} and further developed by Costello \cite{}, twisting refers to a fixed-point construction for field theories equipped with the action of a supersymmetry algebra. Operationally, one modifies the BRST differential of the theory by a nilpotent supercharge - the result is a theory on which the translations in the image of the supercharge act homotopically trivially. Every supersymmetric field theory admits a so-called minimal twist; the resulting theory is a holomorphic-topological field theory that is holomorphic in the maximal number of spacetime directions.

One of the insights of \cite{} was that superconformal indices count exactly local operators in the minimal twist - accordingly, we may think of the algebra of local operators in the minimal twist as \textit{categorifying} the superconformal index. Moreover, the algebra of local operators is part of the richer structure of a \textit{factorization algebra}. Whilst the former governs the behavior of observables supported at points, the latter organizes observables that are supported on any open set.

A principal goal of the present paper is to initiate a study of factorization algebras associated to the minimal twists of the six-dimensional $\cN=(2,0)$ and three-dimensional $\cN=8$ superconformal field theories. However, the six-dimensional $\cN=(2,0)$ supersymmetric theory remains quite elusive - it is non-lagrangian, and outside of the abelian case, is not known to admit a field-realization. In light of this, we propose to access the minimal twist and its local operators via \textit{twisted holography}.

\subsection{Twisted Holography}
Introduced by Costello and Li in \cite{CLsugra}, the twisted holography proposal posits an avatar of the AdS/CFT correspondence that holds at the level of twists. The main objects involved are factorization algebras associated to a gravitational theory and to the worldvolume theory of a number of branes therein.

A more concrete, albeit informal, statement is as follows. Let $X$ be a smooth manifold, and let $\Obs_{grav}$ denote a factorization algebra on $X$ that we view as the observables of some bulk gravitational theory. Suppose we in addition have a stack of $N$ branes, wrapping a submanifold $\iota: Y\to X$, whose worldvolume theory has a factorization algebra of observables $\Obs^{N}_{brane}$.

\begin{expect}[Twisted Holographic Principle]
  There is a map of factorization algebras
  \[
        (\iota^{*}\Obs_{grav})^{!}\to \Obs^{N}_{brane}
  \]

      that becomes an equivalence in the large $N$ limit.
\end{expect}

    This statement has been tested in many examples \cite{}.


\subsection{Infinite dimensional symmetry enhancement by exceptional simple lie superalgebras}

\subsection{Outline}

The paper is organized as follows. The first section is the present introduction.

In section 2 we begin by recalling descriptions of minimal twists of eleven dimensional supergravity \cite{} and the abelian six-dimensional $\cN=(2,0)$ \cite{} and three-dimensional $\cN=8$ \cite{} superconformal field theories.

%\end{document}
 

\section{The minimal twist of 11-dimensional supergravity} 
\label{s:dfn}

In this section we define the central theory of study, within the Batalin--Vilkovisky formalism.
The theory will be defined on any eleven-dimensional manifold of the form $X \times L$ where $X$ is a Calabi--Yau five-fold and $L$ is a smooth oriented one-manifold.

\subsection{Divergence-free vector fields} 

\subsubsection{}
\label{sec:divfree}
We set up some notations and conventions in the context of complex geometry. 
Let $V$ be a holomorphic vector bundle on a complex manifold $X$. 
If $j$ is an integer, we let $\Omega^{0,j}(X, V)$ denote the space of anti-holomorphic Dolbeault forms of type $j$ on with values in $V$.
The $\dbar$ operator for $V$ is $\dbar \colon \Omega^{0,j}(X, V)\to \Omega^{0,j+1}(X)$ defines the Dolbeault complex of $V$
\[
  \Omega^{0,\bu}(X, V) = \left(\Omega^{0,j}(X, V)[-j] \; , \; \dbar\right) .
\]
This is a free resolution for the sheaf of holomorphic sections of $V$.

Suppose $X$ is a Calabi--Yau manifold with holomorphic volume form $\Omega$.
The divergence $\div(\mu)$ of a holomorphic vector field $\mu$ is defined by the formula
\[
\div (\mu) \wedge \Omega = L_\mu (\Omega)
\]
where, on the right hand side, we mean the Lie derivative of $\Omega$ with respect to $\mu$.

Let $\T_X$ denote the holomorphic tangent bundle and consider its Dolbeault complex $\Omega^{0,\bu}(X , \T_X)$ resolving the sheaf of holomorphic vector fields. 
The divergence operator extends to the Dolbeault complex to yield a map of cochain complexes 
\[
\div \colon \Omega^{0,\bu}(X , \T_X) \to \Omega^{0,\bu}(X) .
\]

The kernel of this map resolves the sheaf of holomorphic divergence-free vector fields $\Vect_0 (X)$.
In fact, we can form the complex 
\beqn\label{eqn:cplx1}
\begin{tikzcd}
\ul{0} & \ul{1} \\
\Omega^{0,\bu}(X , \T_X) \ar[r, "\div"] & \Omega^{0,\bu}(X) .
\end{tikzcd}
\eeqn
The $\dbar$ operator, as always, is left implicit. 
By the holomorphic Poincar\'e lemma, the embedding of the sheaf $\Vect_0(X)$ into the degree zero piece of this complex is a quasi-isomorphism. 

There is a direct way to extend the Lie bracket of vector fields to the complex \eqref{eqn:cplx1}. 
Denote by $\mu$ an element of $\Omega^{0,\bu}(X , \T_X)$ and $\nu$ an element of $\Omega^{0,\bu}(X)$ (for simplicity in notation, we will not expand the anti-holomorphic dependence). 
The Lie bracket defined by the formulas
\begin{align*}
[\mu, \mu'] & = L_\mu \mu' \\
[\mu, \nu] & = L_\mu \nu 
\end{align*}
is compatible with $\div$ and endows \eqref{eqn:cplx1} with the structure of a sheaf of dg Lie algebras.
We will refer to this sheaf by the symbol $\cL_0(X)$, or just $\cL_0$ if $X$ is understood. 

The sheaf $\cL_0$ has the structure of a {\em local} dg Lie algebra, see see \cite[??]{CG2}.
This means that as a graded sheaf, $\cL_0$ is the smooth sections of a graded vector bundle and the differential and Lie bracket are given by differential and bidifferential operators, respectively.


\parsec[sec:Linfty]

Recall that an $L_\infty$ algebra is a $\ZZ$-graded vector space $\cL$ together with the data of a square-zero, degree $+1$ derivation $\delta_\cL$ of the free commutative graded algebra $\Sym\left(\cL^\vee [-1] \right)$. 
The Chevalley--Eilenberg cochain complex is 
\[
\left(\Sym\left(\cL^\vee [-1] \right), \delta_\cL\right) .
\]
The Taylor components of $\delta_\cL$ define higher brackets $\{[-]_k\}_{k=1,2,\ldots}$ where $[-]_k \colon \cL^{\times k} \to \cL[2-k]$. 
The condition that the differential is square-zero $\delta_\cL \circ \delta_\cL = 0$ is equivalent to the higher Jacobi relations.

An $L_\infty$ morphism $\Phi: \cL \rightsquigarrow \cL'$ is the same datum as a map of commutative dg algebras 
\deq{
  \Phi^*: \clie^\bu(\cL') \to \clie^\bu(\cL)
}
between their respective Lie algebra cochains. It follows from this that \emph{any} automorphism $\Phi$ of the free commutative algebra on $\cL^\vee[-1]$ defines a new model of the $L_\infty$ algebra $\cL$, for which the Chevalley--Eilenberg differential is obtained by conjugating $\delta_\cL$ by~$\Phi$, and where $\Phi$ itself defines the $L_\infty$ isomorphism.

$\cL_0$ is the sheaf~\eqref{eqn:cplx1} resolving divergence-free vector fields equipped with the dg Lie algebra structure constructed in the previous section.
We consider the following automorphism of~$\Sym(\cL_0^\vee[-1])$, defined by its action on generators:
\deq[eq:newbase]{
%    \Psi_\infty:  \nu &\mapsto -\frac{\nu}{1-\nu}, \quad \mu \mapsto -\mu \\
    \Psi_\infty: \nu \mapsto 1 - e^{-\nu}, \quad \mu \mapsto e^{-\nu} \mu.
}
This map defines a new model of the $L_\infty$ algebra with underlying graded vector space the same as \eqref{eqn:cplx1}, which we will call $\cL_\infty$.\footnote{We are being slightly abusive and using the symbols $\nu,\mu$ dually as coordinates, or operators, on the graded linear space $\cL[1]$.}
The formulas for the automorphism above clearly arise from maps of vector bundles and hence endow $\cL_\infty$ with the structure of a local $L_\infty$ algebra, meaning all operations are given by polydifferential operators.  

The notation refers to the fact that this new model has nonvanishing $L_\infty$ brackets of every order. 
It is this new model that we will use to define the eleven-dimensional theory of twisted supergravity (as well as the family of analogous formal moduli problems on products of odd Calabi--Yau manifolds with~$\R$). 


%The previous proposition characterizes the $L_\infty$ model for divergence-free vector fields that we will use to define the 11-dimensional theory of twisted supergravity. 
%Hereon, we denote by $\cL_0 = \cE_0[-1]$ this local $L_\infty$ algebra. 

%We can unpack Proposition \ref{prop:Linfty} to describe the $L_\infty$ structure on $\cL_0$ explicitly. 
We can describe the $L_\infty$ structure on our new model $\cL_\infty$ explicitly.
Recall that we have two types of elements: $\mu \in \PV^{1,\bu}$ and $\nu \in \PV^{0,\bu}[-1]$. 
The first few nonzero brackets are
\begin{align*}
[\mu]_1 & = \dbar \mu + \div \mu \\
[\mu_1,\mu_2]_2 & = \div (\mu_1 \wedge \mu_2) \\
[\nu, \mu_1,\mu_2]_3 & = \div(\nu \mu_1 \wedge \mu_2) 
%\\
%[\nu_1,\nu_2, \mu_1,\mu_2]_4 & = \# \div(\nu_1 \nu_2 \mu_1 \mu_2).
\end{align*}
For $k \geq 2$ the general formula for the $k$-ary brackets are 
\begin{align*}
[\nu_1, \ldots, \nu_{k-2}, \mu_1,\mu_2]_{k} & = \div(\nu_1 \cdots \nu_k \mu_1 \wedge \mu_2) \\
[\nu_1,\ldots, \nu_{k-3}, \mu_1,\mu_2,\gamma]_k & = \nu_1 \cdots \nu_{k-3} (\mu \wedge \mu') \vee \del \gamma .\\
[\nu_1,\ldots,\nu_{k-2}, \mu, \gamma]_k & = \nu_1 \cdots \nu_{k-2} \mu \vee \del \gamma .
\end{align*}

%We describe the explicit $L_\infty$ automorphism $\Psi \colon (\cL_0)^{L_\infty} \rightsquigarrow (\cL_0)^{strict}$ intertwining the strict dg Lie structure on $\cL_0$ and this $L_\infty$ structure.
%The linear term $\Psi^{(1)} = \id$ is the identity map. 
%The higher terms $\Psi^{(n)}$ are defined by 
%\brian{someone check me}
%\begin{align*}
%\Psi^{(n)} (\nu_1,\ldots, \nu_{n-k},\mu_1,\ldots, \mu_k) & = \delta_{k=1} \nu_1 \cdots \nu_{n-1} \mu_1 . \\
%\end{align*}


\parsec
There is yet another $L_\infty$ model for divergence-free vector fields that we remark on here.
\ingmar{I want to write the other automorphism later, I think; otherwise we have to write the composition, since we presented it on bcov}


\subsection{Theories of BF type}

\parsec
Suppose that $\cL$ is an $L_\infty$ algebra with $L_\infty$ operations $\{[-]^\cL_k\}_{k=1,2,\ldots}$ and that $(\cA, \d_\cA)$ is a commutative dg algebra. 
The graded vector space $\cL \otimes \cA$ is equipped with the natural structure of an $L_\infty$ algebra with operations $\{[-]_k\}_{k=1,2,\ldots}$ defined by
\begin{align*}
[x \otimes a]_1 & = [x]^\cL_1 \otimes a + (-1)^{|x|} x \otimes \d_\cA a \\
[x_1 \otimes a_1, \ldots , x_k \otimes a_k]_k & = [x_1,\ldots,x_k]^\cL_k \otimes (a_1 \cdots a_k), \qquad k \geq 2 .
\end{align*}

We apply this construction, taking $\cL$ to be the sheaf resolving divergence-free holomorphic vector fields on a Calabi--Yau manifold $X$ equipped with either the strict dg Lie algebra structure $\cL_0(X)$ or its non-strict $L_\infty$ structure $\cL_\infty (X)$. 
The algebra $\cA$ will be the smooth de Rham complex $(\Omega^\bu(S) , \d_S)$ where $S$ is a smooth manifold (we will specialize the dimension of this smooth manifold shortly, but the constructions in this section make sense in any dimension). 

We thus obtain the structure of an dg Lie algebra on $\cL_0(X) \otimes \Omega^{\bu}(S)$ or an $L_\infty$ algebra $\cL_\infty(X) \otimes \Omega^\bu(S)$.
These define equivalent local $L_\infty$ algebras on the product manifold~$X \times S$. 

\parsec[s:bf]

Associated to any local $L_\infty$ algebra is a classical field theory in the BV formalism.
Let $\cL$ be a local $L_\infty$ algebra on some manifold $M$, it is the sheaf of sections of some graded vector bundle $L$. 
For a section $A \in \cL$, introduce the `higher curvature map' defined by the formula
\[
\mathsf{F}_A = [A]_1 + \frac12 [A,A]_2 + \frac{1}{3!} [A,A,A]_3 + \cdots .
\]

The fields of the associated BV theory are pairs
\[
  (A, B) \in \cL[1] \oplus \cL^{!}[-2] .
\]
Here $\cL^!$ denotes the sheaf of sections of the bundle $L^* \otimes {\rm Dens}$, where ${\rm Dens}$ is the bundle of densities. 
The shifted symplectic BV pairing is the obvious integration pairing between $A$ and $B$. 

The action functional reads $S_{\rm BF} = \int_M B \, \mathsf{F}_{A}$ which leads to the equations of motion $\mathsf{F}_{A} = 0$ and $\mathsf{D}_A B= 0$ where $\mathsf{D}_A$ is the higher covariant derivative along $A$. 
We refer to this as the ``BF theory'' associated to $\cL$.

We thus obtain a theory in the BV formalism on the product manifold $X \times S$ associated to both local $L_\infty$ algebras $\cL_0(X) \otimes \Omega^{\bu}(S)$ and $\cL_\infty(X) \otimes \Omega^\bu(S)$.

\parsec
For concreteness, we spell out the fields of the theories we have constructed on $X \times S$.
In both cases, the space of fields equipped with the linear BRST operator is
\begin{equation}
  \label{eq:sympfields} 
  \begin{tikzcd}[row sep = 1 ex]
    -n & -n + 1 & -1 & 0 \\ \hline
    \Omega^{0}(X;S) \ar[r, "\del"] & \Omega^{1}(X;S) & 
     \PV^{1}(X; S) \ar[r, "\div"] & \PV^{0}(X; S).
\end{tikzcd}
\end{equation}
We denote the fields $(\beta,\gamma,\mu,\nu)$ respectively.
We are using the shorthand notation
\begin{align*}
\Omega^{i}(X;S) & = \Omega^{i , \bu;\bu}(X;S) \\
 & = \oplus_{j,k} \PV^{i,j}(X) \otimes \Omega^k(S) [-j-k] .
\end{align*}
which is equipped with the $\dbar + \d_S$ operator and similarly for $\PV^{i}(X;S)$. 

The natural pairing between $\PV^i(X;S)$ and~$\Omega^i(X;S)$ is of degree $-\dim_\C(X) -\dim_\R(S)$. 
As such, the $\Z$-grading indicated in~\eqref{eq:sympfields} equips the sheaf of fields with a $(-1)$-shifted symplectic structure, provided that we choose the shift to be given by
\deq{
  n = \dim_\C(X) + \dim_\R(S) - 1.
}

We have constructed two equivalent descriptions of the BF theory which share the linear BRST complex \eqref{eq:sympfields}.
Explicitly, the action functional for BF theory associated to the local dg Lie algebra $\cL_0(X) \otimes \Omega^{\bu}(S)$ is
\deq{
  S_{BF,0} =  \beta \wedge (\dbar + \d_S) \nu +  \gamma \wedge (\dbar + \d_S) \mu +  \beta \wedge \partial_\Omega \mu + \frac{1}{2} [\mu,\mu] \vee \gamma +  [\mu,\nu] \beta .
}
As in the Lie algebra structure of this strict model, notice that the Schouten--Nijenhuis bracket appears explicitly. 

The action functional of BF theory associated to $\cL_\infty(X) \otimes \Omega^{\bu}(S)$ is non polynomial. 
(In fact, it is related to the BCOV action functional via a procedure we outline below.)\ingmar{or somewhere} 
Explicitly, this action functional is
\deq{
  S_{BF,\infty} =  \beta \wedge (\dbar + \d_S) \nu +  \gamma \wedge (\dbar + \d_S) \mu +   \beta \wedge \partial_\Omega \mu + \frac12 \frac{1}{1-\nu} \mu^2 \vee \del \gamma .
}

We demonstrated above that the two local $L_\infty$ algebras on which these BF theories are based are equivalent. As such, the BF theories are also equivalent; the map~\eqref{eq:newbase} extends uniquely to an automorphism of BV theories.
Explicitly, the automorphism is
\begin{multline}\label{eqn:auto1}
  \mu \mapsto e^{-\nu} \mu, \qquad \nu \mapsto 1-e^{-\nu} \\
  \beta \mapsto (\beta - \mu \vee \gamma) e^{\nu},\qquad \gamma \mapsto e^{\nu} \gamma .
\end{multline}

\parsec[]

In what follows, we specialize to the case that $X$ is a Calabi--Yau five-fold and that $S$ is a one-dimensional smooth orientable manifold. 
In this case, with $n = 5 + 1 - 1 = 5$ the theories described in this section are $\ZZ$-graded in the BV formalism.
Momentarily, we consider a new term in the action which will break this grading, so this integer shift will not play an essential role.

\subsection{A deformation of BF theory} 

Let $X$ be a Calabi--Yau five-fold and $S$ be a smooth oriented one-dimensional real manifold. 
We will break the $\ZZ$-grading present in BF theory discussed in the previous section to a $\ZZ/2$ grading. 
For reference, this means that linear BRST complex of fields of the model now take the following form. 

\begin{equation}
  \label{eq:sympfields} 
  \begin{tikzcd}[row sep = 1 ex]
    {\rm odd} & {\rm even} & {\rm odd} & {\rm even} \\ \hline
    \Omega^{0}(X;S)_\beta \ar[r, "\del"] & \Omega^{1}(X;S)_\gamma & 
     \PV^{1}(X; S)_\mu \ar[r, "\div"] & \PV^{0}(X; S)_\nu.
\end{tikzcd}
\end{equation}

\parsec
%check CME

To define our classical field theory on $X \times S$ we consider  a deformation of BF theory $S_{BF}$ (this refers to either the presentation as $S_{BF,0}$ or $S_{BF,\infty}$). 
Such deformations are governed by the classical master equation: the parameterized family of actions 
\beqn\label{eqn:defaction}
S_{BF} + g J
\eeqn
defines a consistent theory in the BV formalism if and only if
\deq{
  \{S_{BF} + g J, S_{BF} + g J \} = 0.
}
Since this must hold for all $g$, and since the undeformed action $S$ is already a solution to the classical master equation, this reduces the condition
\deq[eq:2cond]{
  \{S_{BF},J\} + \frac12 \{J,J\} = 0.
}

The form of $J$ depends on which presentation we use for BF theory.
To begin, we will use the presentation of BF theory $S_{BF, \infty}$ which uses the the non-strict $L_\infty$ structure on divergence-free holomorphic vector fields.
The deformation $J$ does not make reference to the Calabi--Yau structure explicitly, but it does involve the holomorphic de Rham operator $\del$ on $X$. 

The main result of this section is the following. 

\begin{thm}
\label{thm:dfn}
Let $X$ be a Calabi--Yau five-fold and $S$ a smooth one-dimensional manifold, and consider the BV theory $(\cE, S_{BF,\infty})$ on $X \times S$ defined above. The local functional 
  \deq{
    J = \frac16 \gamma \wedge \del \gamma \wedge \del \gamma ,
  }
  where $\gamma \in \Omega^{1,\bu}(X;S)$, defines a deformation of~$(\cE,S_{BF,\infty})$ as a $\Z/2$-graded BV theory.
\end{thm}

\parsec[]

First off, we remark on grading issues. 
In the original $\Z$-grading on the BF theory given in \eqref{eq:sympfields} with $n=5$, the component 
\[
\gamma^{1,i;j} \in \Omega^{1,i}(X) \otimes \Omega^j(S) 
\]
sits in degree $-4+i+j$. 
Thus, we see that in the original $\Z$-grading on BF theory one has
  \deq{
    \deg(J) = 6.
  }
Thus $S_{BF} + J$ is not of homogenous $\ZZ$ grading.

This is completely reasonable from the point of view of twisting supersymmetry in 11-dimensions. 
Indeed, the $R$-symmetry group is trivial, and there is not a way to regrade the fields of the twisted theory using twisting data \label{CosHol,ESW}. 

Nevertheless, if we break to the obvious $\ZZ/2$ grading, the functional $S_{BF} + g J$ defines an even action functional.
Unless otherwise stated, we will work with this $\ZZ/2$ grading for the remainder of this section.

\parsec[]
We proceed to show that $S_{BF,\infty} + g J$ solves the classical master equation.
By apparent field type reasons, the equation \eqref{eq:2cond} is equivalent to the following two equations
\[
\{S_{BF,\infty}, J\} = 0, \qquad \{J,J\} = 0.
\] 
It is furthermore immediate from the form of the BV bracket that $\{J,J\} = 0$, since $J$ depends only on the $\gamma$ field. 

It remains to check that $\{S_{BF,\infty},J\} = 0$. 
For the quadratic term in the BF action, we note that 
  \deq{
    \{\beta \wedge \div\mu, J\} = \frac12 \del\beta \wedge \del \gamma \wedge \del \gamma = 0,
  }
  because total derivatives are equivalent to zero as local functionals. 
  
The contribution from the remaining BF action takes the form
\begin{multline}
    \left\{ \frac12 \frac{1}{1-\nu} \del\gamma \vee \mu^2, \frac16 \gamma \wedge \del\gamma \wedge \del \gamma \right\} \\ = \frac12 \frac{1}{1-\nu} \bigg[ (\del\gamma\vee\mu) \wedge \del\gamma \wedge \del \gamma \pm \del\gamma \wedge (\del\gamma\vee\mu) \wedge \del \gamma \pm \del\gamma \wedge \del \gamma \wedge (\del\gamma\vee\mu) \bigg].
\end{multline}
This expression is zero for symmetry reasons. Recall that $\del\gamma$ is a two-form, and that the expression must be a totally symmetric local functional which is cubic in this two-form. We can ask whether such a  contraction exists just at the level of $\lie{sl}(5)$ representation theory by computing the decomposition of the two-form  
\ingmar{finish}

\parsec[s:coupling]

We make note of the dependence on the coupling constant $g$ in the definition of the deformed action $S_{BF,\infty} + g J$. 

When $g = 0$ we return to BF theory for the $L_\infty$ algebra $\cL_\infty(\CC^5) \otimes \Omega^\bu(\RR)$. 
For any $g \ne 0$ the theories are essentially equivalent in perturbation theory. 
Indeed, if $g \ne 0$ we can make the following field redefinition 
\[
\gamma \mapsto \sqrt{g} \gamma, \quad \beta \mapsto \sqrt{g} \beta 
\]
to write the action as 
\[
\frac{1}{\sqrt{g}} \left(S_{BF,\infty} + J \right)  .
\]

In perturbation theory, this has the affect of modifying the quantization parameter $\hbar$ to $\hbar / \sqrt{g}$.
Thus, after modifying $\hbar$ and making the above field redefinition, the perturbative expansion of any theory is equivalent to the one with $g = 1$. 

\parsec[s:altdfn]

We remark on an alternative, equivalent, description of the deformed theory which involves the strict dg Lie algebra structure on divergence-free holomorphic vector fields.

We can replace $S_{BF,\infty}$ by $S_{BF,0}$ via applying the field automorphism \eqref{eqn:auto1}.
Doing this we see that $J$ becomes 
\[
\til{J} = \frac16 e^\nu \gamma \wedge \del (e^\nu \gamma) \wedge \del(e^\nu \gamma) .
\]
Since this automorphism preserves the odd BV bracket, the actions $S_{BF,\infty} + g J$ and $S_{BF, 0} + g \til{J}$ are both solutions to the classical master equations and are equivalent as~$\ZZ/2$ graded BV theories.

\subsection{Equations of motion} \label{s:eom}

Soon, we will provide a series of justifications for the assertion that the deformed theory $S_{BF, \infty} + g J$ is the minimal twist of $11$-dimensional supergravity on flat space $X \times S = \CC^5 \times \RR$ where $\CC^5$ is equipped with its flat Calabi--Yau form. 
For the moment, we briefly read off the equations of motion of the general theory on $X \times S$.
Let $\Omega$ denote the Calabi--Yau form on $X$. 

We consider the action $S_{BF, \infty} + gJ$.
The equation of motion obtained by varying $\beta$ is especially simple, in fact linear, since it only appears in the action via a quadratic term. 
It is
\beqn\label{eqn:eombeta}
\dbar \nu + \d_S \nu + \div \mu = 0 .
\eeqn
Varying $\gamma$ we obtain the equation of motion
\beqn\label{eqn:eomgamma}
\dbar \mu + \d_S \mu + \frac12 \frac{1}{1-\nu} \div (\mu^2) + \frac12 (\del \gamma \wedge \del \gamma) \vee (g \Omega^{-1}) = 0 .
\eeqn
The last term represents the contraction of an element of $\Omega^{4,\bu}(X;S)$ with the nonvanishing section $\Omega^{-1} \in \PV^{5,\bu}(X;S)$ to yield an element of $\PV^{1,\bu}(X;S)$. 
If we vary the $\mu$ we obtain 
\beqn\label{eqn:eommu}
(\dbar + \d_S) \gamma + \del \beta + \frac{1}{1-\nu} (\mu \vee \del \gamma) = 0 .
\eeqn
Finally, if we vary $\nu$ we obtain
\beqn\label{eqn:eomnu}
(\dbar + \d_S) \beta + \frac12 \frac{1}{(1-\nu)^2} \mu^2 \vee \del \gamma = 0 .
\eeqn

The equation of motion must hold for any inhomogenous superfields.
We can get a better sense of the equations if we expand in components of these fields. 
The component fields of the 11-dimensional theory on $X \times S$ have the following form: 
\begin{itemize}
\item $\mu = \sum_{i,j} \mu^{i;j}$ is a superfield where
\[
\mu^{i;j} \in \PV^{1,i}(X) \otimes \Omega^j(\RR) ,\quad i=0,\ldots, 5, \quad j=0,1.
\]
The component $\mu^{i;j}$ has parity $i+j+1 \mod 2$. 
\item $\nu = \sum_{i,j} \nu^{i;j}$ is a superfield where
\[
\nu^{i;j} \in \PV^{0,i}(X, \T_X) \otimes \Omega^j(\RR) ,\quad i=0,\ldots, 5, \quad j=0,1.
\]
The component $\nu^{i;j}$ has parity $i+j \mod 2$. 
%The linear equations of motion state that $\mu$ is constant along $\RR$ and holomorphic divergence-free as a vector field on $\CC^5$. 
\item 
$\gamma = \sum_{i,j} \gamma^{i;j}$ is a superfield where
\[
\gamma^{i;j} \in \Omega^{1,i}(X) \otimes \Omega^j(\RR) ,\quad i=0,\ldots, 5, \quad j=0,1.
\]
The component $\gamma^{i;j}$ has parity $i+j$. 
\item 
\item $\beta = \sum_{i,j} \beta^{i;j}$ is a superfield where
\[
\beta^{i;j} \in \Omega^{0,i}(X) \otimes \Omega^j(\RR) ,\quad i=0,\ldots, 5, \quad j=0,1.
\]
The component $\beta^{i;j}$ has parity $i+j+1 \mod 2$. 
\end{itemize}

We look closely at the geometric meaning of \eqref{eqn:eombeta}. 
Let's make the simplifying assumption that all components of $\mu$ are divergence-free that all fields are locally constant along $S$: $\div \mu = 0$ and $\d_S \mu = \d_S \gamma = 0$.
Then, $\nu = 0$ is a solution to \eqref{eqn:eombeta} and we can assume that all fields are functions, or zero-forms, along $S$. 
Then, there is a component of \eqref{eqn:eomgamma} which can be written as 
\beqn\label{eqn:eomgamma1}
\dbar \mu^{1;0} + \frac12 [\mu^{1;0},\mu^{1;0}] + \left(\frac12 \del \gamma^{1;0} \wedge \del \gamma^{1;0} + \del \gamma^{2;0} \wedge \del \gamma^{0;0}\right) \vee (g \Omega^{-1}) = 0 
\eeqn
where now $[-,-]$ stands for the Schouten bracket.


%The most geometrically relevant component is the case where 
%\[
%\mu^{1;0} \in \PV^{1,1}(X) \otimes \Omega^0(S) 
%\]
%which is an even field in our $\ZZ/2$ graded BV theory.
%The superscript denotes anti-holomorphic; de Rham form type. 
%For this component of $\mu$, the only components of $\gamma$ which appear in the above equations of motion are $\gamma^{0;0}$, $\gamma^{1;0}$, and $\gamma^{2;0}$.
%Expanding these components out, we obtain
%\beqn\label{eqn:eomgamma1}
%\dbar \mu^{1;0} + \frac12 [\mu^{1;0},\mu^{1;0}] + (\del \gamma^{0;0} \wedge \del \gamma^{2;0}) \vee (g \Omega^{-1}) + \frac12 (\del \gamma^{1;0} \wedge \del \gamma^{1;0}) \vee (g \Omega^{-1}) = 0 .
%\eeqn

\subsection{Local character}

We consider the 11-dimensional theory on the manifold $\CC^5 \times \RR$ where $\CC^5$ is equipped with its standard Calabi--Yau structure. 
On this background, the theory is manifestly $SU(5)$ invariant. 
In this section, we compute the corresponding character of the local operators at the origin. 

The local character is only sensitive to the free limit of the theory.
Furthermore, the linear BRST operator is an $SU(5)$-invariant deformation of the $(\dbar + \d_{\RR})$ operator. 
Therefore, to compute the character it suffices to compute the $SU(5)$-equivariant character of the $\dbar$ cohomology. 

The solutions to the $(\dbar + \d_{\RR})$-equations of motion simply say that all fields are holomorphic along $\CC^5$ and constant along $\RR$. 
Thus, the solutions can be identified with 
\begin{align*}
\mu^{i}\partial_{z_i} & \in \Vect(\CC^5) \cong \cO(\C^5)\partial_{z_i},\quad 
\nu \in \cO (\C^5) \\
\beta & \in \cO (\C^5), \quad \gamma^{i} \d z_i \in \Omega^{1}(\CC^5) \cong \cO (\C^5)\d z_i 
\end{align*}
where $z_i$ is a holomorphic coordinate on $\CC^5$. 

Corresponding to each of the above, we have a tower of linear local operators labeled by $(m_j) = (m_1, m_2, m_3, m_4, m_5)\in \Z^5_{\geq 0}$; these are given by
\begin{align*}
 \mu^{i}_{(m_j)} &: \mu^{i}\mapsto \partial_{z_1}^{m_1}\partial_{z_2}^{m_2}\partial_{z_3}^{m_3}\partial_{z_4}^{m_4}\partial_{z_5}^{m_5}\mu^{i} (0) \\
\nu_{(m_j)} &: \nu\mapsto \partial_{z_1}^{m_1}\partial_{z_2}^{m_2}\partial_{z_3}^{m_3}\partial_{z_4}^{m_4}\partial_{z_5}^{m_5}\nu (0) \\
\gamma^{i}_{(m_j)} &: \gamma^{i}\mapsto \partial_{z_1}^{m_1}\partial_{z_2}^{m_2}\partial_{z_3}^{m_3}\partial_{z_4}^{m_4}\partial_{z_5}^{m_5}\gamma^{i} (0) \\
 \beta_{(m_j)} &: \beta\mapsto \partial_{z_1}^{m_1}\partial_{z_2}^{m_2}\partial_{z_3}^{m_3}\partial_{z_4}^{m_4}\partial_{z_5}^{m_5}\beta (0) \\
\end{align*}

We choose generators of the Cartan subgroup for the $SU(5)$ action with the following weights:

\[\begin{array}{|c|c|c|c|c|c|}
& z_1 & z_2 & z_3 & z_4 & z_5 \\
\hline
q & & & & 1 & -1 \\
t_1 & 1 & -1 & & & \\
t_2 & & 1 & -1 & & \\
y & -1 & -1 & -1 &\frac 3 2 & \frac 3 2
\end{array}\]
We are choosing the weights in this way as a matter of convenience for the later sections. 
For instance, upon performing the further twist of the 11-dimensional theory, the $SU(5)$ symmetry is broken to an $SU(3)\times SU(2)$ symmetry; the Cartan of the unbroken symmetries corresponds to the fugacities $q, t_1, t_2$ in the above table. 

We can now readily compute the characters.
\begin{prop}\label{prop:locchar}
The $SU(5)$ local character of the holomorphic twist of the 11-dimensional theory on flat space is
\begin{multline}
\chi_{SU(5)} = 
\prod_ {(m_j)\in \Z^5_{\geq 0}} \frac{1- t_1^{-m_1+m_2}t_2^{-m_2+m_3}q^{-m_4+m_5}y^{m_1+m_2+m_3-\frac 3 2 (m_4+m_5)}}{1- t_1^{-m_1+m_2}t_2^{-m_2+m_3}q^{-m_4+m_5}y^{m_1+m_2+m_3-\frac 3 2 (m_4+m_5)} }
\\ 
\times \frac{t_1^{-1}y + t_1t_2^{-1}y + t_2y + q^{-1}y^{-\frac 3 2} + qy^{-\frac 3 2}}{t_1y^{-1} + t_1^{-1}t_2y^{-1} + t_2^{-1}y^{-1} + q^{-1}y^{\frac 3 2} + qy^{\frac 3 2}}.
\end{multline}
\end{prop}
\begin{proof}
We compute the local character as the plethystic exponential of the character of linear local operators. 

Note that the linear local operators $ \nu_{(m_j)}$ and $\beta_{(m_j)}$ are of the same weight but opposite parity so contribute to the character with opposite sign. These contributions therefore cancel. Next, there is a summand of the linear local operators of the form 
\[
\bigoplus _{(m_j)\in \Z^5_{\geq 0 }} \Pi \left ( \C\mu^{z_1}_{(m_j)}\partial_{z_1}\oplus \C\mu^{z_2}_{(m_j)}\partial_{z_2}\oplus \C\mu^{z_3}_{(m_j)}\partial_{z_3}\oplus\C\mu^{w_1}_{(m_j)}\partial_{w_1}\oplus \C\mu^{w_2}_{(m_j)}\partial_{w_2}\right).
\] 
This contributes 
\[
\sum_{(m_j)\in \Z^5_{\geq 0 }}- t_1^{-m_1+m_2}t_2^{-m_2+m_3}q^{-m_4+m_5}y^{m_1+m_2+m_3-\frac 3 2 (m_4+m_5)}\left (t_1^{-1}y + t_1t_2^{-1}y + t_2y + q^{-1}y^{-\frac 3 2} + qy^{-\frac 3 2} \right).
\] 
Finally, there is a summand of the form 
\[
\bigoplus _{(m_i;n_j)\in \Z^5_{\geq 0 }} \Pi \left ( \C\gamma^{z_1}_{(m_i; n_j)}d{z_1}\oplus \C\gamma^{z_2}_{(m_i; n_j)}d{z_2}\oplus\C\gamma^{z_3}_{(m_i; n_j)}d{z_3}\oplus\C\gamma^{w_1}_{(m_i; n_j)}d{w_1}\oplus \C\gamma^{w_2}_{(m_i; n_j)}d {w_2}\right).
\] 
This likewise contributes 
\[
\sum_{(m_j)\in \Z^5_{\geq 0 }}t_1^{-m_1+m_2}t_2^{-m_2+m_3}q^{-n_1+n_2}y^{m_1+m_2+m_3-\frac 3 2 (m_4+m_5)}\left (t_1y^{-1} + t_1^{-1}t_2y^{-1} + t_2^{-1}y^{-1} + q^{-1}y^{\frac 3 2} + qy^{\frac 3 2} \right).
\] 
In sum, the character of linear local operators is the geometric series 
\[
\sum_{(m_j)\in \Z^5_{\geq 0 }}-t_1^{-m_1+m_2}t_2^{-m_2+m_3}q^{-n_1+n_2}y^{m_1+m_2+m_3-\frac 3 2 (m_4+m_5)}\left (\begin{aligned}t_1y^{-1} + t_1^{-1}t_2y^{-1} + t_2^{-1}y^{-1} + q^{-1}y^{\frac 3 2} + qy^{\frac 3 2} \\  - (t_1^{-1}y + t_1t_2^{-1}y + t_2y + q^{-1}y^{-\frac 3 2} + qy^{-\frac 3 2})\end{aligned}\right).
\] 
The plethystic exponential returns the desired expression.
%\[
%\prod_ {(m_j)\in \Z^5_{\geq 0}}\frac{1- t_1^{-m_1+m_2}t_2^{-m_2+m_3}q^{-n_1+n_2}y^{m_1+m_2+m_3-\frac 3 2 (m_4+m_5)}\left (t_1^{-1}y + t_1t_2^{-1}y + t_2y + q^{-1}y^{-\frac 3 2} + qy^{-\frac 3 2} \right)}{1- t_1^{-m_1+m_2}t_2^{-m_2+m_3}q^{-m_4+m_5}y^{m_1+m_2+m_3-\frac 3 2 (m_4+m_5)}\left (t_1y^{-1} + t_1^{-1}t_2y^{-1} + t_2^{-1}y^{-1} + q^{-1}y^{\frac 3 2} + qy^{\frac 3 2} \right)}.
%\]
\end{proof}

%The linear equations of motion for the superfield $\mu$ read 
%\[
%\dbar \mu^{i;j} + \d_{S} \mu^{i+1, j-1} = 0, \quad \div \mu^{i,j} = 0
%\]
%for all $i,j$. 
%The operator $\dbar$ is the anti-holomorphic Dolbeault operator acting on $\CC^5$, $\d_S$ is the de Rham operator on $S$, and $\div$ is the divergence with respect to the fixed holomorphic volume form on $\CC^5$. 
%In particular, this states that $\mu^{i;0}$ is locally constant along $S$ and holomorphic divergence-free as a Dolbeault valued vector field on $\CC^5$.

\subsection{One-loop quantization}

In \cite{GRWthf} an existence result for one-loop quantizations of mixed topological-holomorphic theories was established. 
We apply this to the 11-dimensional model at hand. 

The 11-dimensional theory is a mixed topological-holomorphic theory.
On flat space $\CC^5_z \times \RR_t$, this means that the theory is translation invariant and that the following act homotopically trivially:
\begin{itemize}
\item the vector fields $\del_{\zbar_1}, \ldots, \del_{\zbar_{5}}$ corresponding to infinitesimal anti-holomorphic translations,
\item the vector field $\partial_t$ corresponding to infinitesimal translations in the $\RR_t$ direction. 
\end{itemize}

Recall that the action functional of the 11-dimensional theory is $S_{BF, \infty} + c J$. 
Since the cubic and higher interactions only involve holomorphic derivatives, we obtain the following directly from the main result of \cite{GRWthf}. 

\begin{thm}
There exists a gauge fixing condition for the 11-dimensional theory on $\CC^5 \times \RR$ which renders its one-loop quantization finite and anomaly-free. 
\end{thm} 

When $c=0$, this result is actually exact
since there are no Feynman diagrams present past one-loop in this case. 
When $c \ne 0$, on the other hand, this result does not immediately imply the existence of a gauge invariant perturbative quantization to higher orders in $\hbar$. 
The presence of the functional $J = \frac16 \int \gamma \del \gamma \del \gamma$ allows one to construct Feynman graphs at arbitrary loop order. 

Upon performing the $\Omega$-background,
%see \S \ref{s:omega},
In \cite{CostelloM5}, Costello argues that the theory localizes to a five-dimensional theory on $\CC^2 \times \RR$. 
Via a cohomological argument, it is shown that this effective five-dimensional theory exhibits an essentially unique quantization in perturbation theory. 
We will return to the existence and uniqueness of a higher order quantization of the 11-dimensional theory in future work. 


\section{Twisted BFFS matrix model}

\subsection{Global symmetry algebra}
\label{sec:global}

In any field theory, the cohomology classes of states of odd ghost number have the structure of a Lie algebra. 
More generally, after shifting the cohomological degree by one the full cohomology of states with respect to the linear BRST operator results in a graded Lie algebra. 
If we forget the grading to a $\ZZ/2$ grading then the global symmetry algebra has the structure of a super Lie algebra. 

In general, taking cohomology loses information. 
If the dg Lie, or $L_\infty$ algebra, we start with is not formal then there exists higher order operations present in the linearized BRST cohomology. 
We will refer to this as the global symmetry algebra of the theory.

Before taking cohomology with respect to the linear BRST operator, we described the super $L_\infty$ structure on the parity shift of the 11-dimensional fields in the previous section. 
This is encoded by the full BV action of the 11-dimensional theory.
The cubic component of the full BV action induces the super Lie algebra structure present in the linearized BRST cohomology. 

Our main result is to relate the global symmetry algebra of the minimal twist of 11-dimensional supergravity on $\CC^5 \times \RR$ to a certain infinite-dimensional exceptional simple super Lie algebra studied by Kac \cite{KacClass,KR} called $E(5,10)$.
We recall the definition below. 

\begin{thm}\label{thm:global}
Let $\Pi\cE(\CC^5 \times \RR)$ be the parity shift of the fields of 11-dimensional supergravity on $\CC^5 \times \RR$ and denote by $\delta^{(1)}$ the linearized BRST operator. 
\begin{enumerate}
\item 
As a super Lie algebra, the $\delta^{(1)}$-cohomology of $\Pi\cE(\CC^5 \times \RR)$ is isomorphic to the trivial one-dimensional central extension of the super Lie algebra $E^{hol}(5,10)$.
\item 
The global symmetry algebra is equivalent, as a super $L_\infty$ algebra, to the non-trivial central extension of $E^{hol}(5,10)$ determined by the even cocycle \eqref{eqn:cocycle}. 
\end{enumerate}
\end{thm}
%defined by the even cocycle
%\begin{align*}
%E^{hol}(5,10) \times E^{hol}(5,10) \times E^{hol}(5,10) & \to \CC \\
%(\mu,\mu',\alpha) & \mapsto \<\mu \wedge \mu', \alpha\>|_{z=0} .
%\end{align*} 

%The parity of the functional in the theorem is odd, but it is also trilinear. 
%Thus, as a cocycle in the Chevalley--Eilenberg complex of $E^{hol}(5,10)$ it is of total even parity.

The super Lie algebra $E^{hol}(5,10)$ is very closely related to the super Lie algebra $E(5,10)$ studied by Kac; there is a dense embedding of super Lie algebras $E(5,10) \hookrightarrow E^{hol}(5,10)$. 

\subsection{Linearized BRST cohomology} 

We compute the linearized BRST cohomology of 11-dimensional supergravity.
Then, we will describe the induced structure of a super Lie algebra present in the parity shift of the cohomology thus proving part (1) of Theorem \ref{thm:global}.

\parsec[]

First, we recall the definition of the exceptional simple super Lie algebra $E(5,10)$. 
Recall that $\Vect_0 (\CC^5)$ is the Lie algebra of divergence-free holomorphic vector fields on $\CC^5$.
Let $\Omega^{2}_{cl} (\CC^5)$ be the module of holomoprhic $2$-forms which are closed for the holomorphic de Rham operator $\del$.

The even part of the super Lie algebra $E^{hol}(5,10)$ is the Lie algebra of divergence-free vector fields on $\CC^5$
\[
E^{hol}(5,10)_+ = \Vect_0(\CC^5) ,
\]
whose elements we continue to denote by $\mu$. 
The odd piece is the module 
\[
E^{hol}(5,10)_- = \Omega^{2}_{cl} (\CC^5) 
\]
whose elements we denote by $\alpha$. 
Besides the natural module structure, there is an odd $\times$ odd $\to$ even bracket. 
The bracket uses the isomorphism $\Omega^{-1} \vee (-) \colon \Omega^{4} \cong \Vect (\CC^5)$ induced by the standard Calabi--Yau form $\d^5 z$ and is defined by
\beqn\label{eqn:e510}
[\alpha, \alpha'] = \Omega^{-1} \vee (\alpha \wedge \alpha') .
\eeqn
One immediately checks that since both $\alpha, \alpha'$ are closed two-forms that the resulting vector field on the right hand side is divergence-free. 
In coordinates, if $f^{ij} \d z_i \wedge \d z_j$, $g^{kl} \d z_k \wedge \d z_l$ are two closed two-forms, their bracket is the vector field $\ep_{ijklm} f^{ij}g^{kl} \partial_{z_m}$. 

More precisely, Kac studied a dense sub Lie algebra $E(5,10)$ of $E^{hol}(5,10)$ consisting of those vector fields and two-forms which have polynomial coefficients.
It is this dense sub super Lie algebra which is simple.

\parsec[]

If $\cE$ is the space of fields of any theory in the BV or BRST formalism, the shift $\cL = \cE[-1]$ has the structure of a Lie, possibly $L_\infty$ algebra. 
In the $\ZZ/2$ graded world, the parity shifted object $\cL = \Pi \cE$ has the structure of a super $L_\infty$ algebra. 

In this section, we use the description of the 11-dimensional theory as the deformation of the BF action $S_{BF,\infty}$ by the functional $J$ of Theorem \ref{thm:dfn}. 
We will obtain the exact same results if we use the equivalent model of the 11-dimensional theory explained in \S \ref{s:altdfn}. 
We remark on this below in \S \ref{s:altglobal}. 

The full BRST operator is given by the BV bracket with the BV action. 
For us, this~is 
\[
\delta = \{S_{BF,\infty} + J, -\} .
\]
The linear BRST operator comes only from the quadratic summands in $S_{BF,\infty}$ and is of the form
\beqn\label{eqn:linearBRST}
\delta^{(1)} = \dbar + \d_{\RR} + \div |_{\mu \to \nu} + \del |_{\beta \to \gamma} .
\eeqn

To compute the cohomology with respect to $\delta^{(1)}$ we can use a spectral sequence, first taking the cohomology with respect to $\dbar + \d_{\RR}$ and then with respect to $\div$. 
By the $\dbar$ and de Rham Poincar\'e lemmas, the cohomology of the space of fields of the 11-dimensional theory on $\CC^5 \times \RR$ with respect $\dbar + \d_{\RR}$ results in the cochain complex
\begin{equation}
  \label{eq:lin1} 
  \begin{tikzcd}[row sep = 1 ex]
    - & + \\ \hline
    \Vect(\CC^5) \ar[r, "\div"] & \cO(\CC^5) \\ 
     \cO(\CC^5) \ar[r, "\del"] & \Omega^{1}(\CC^5).
\end{tikzcd}
\end{equation}
Recall that $\Vect(\CC^5), \cO(\CC^5)$, and $\Omega^1(\CC^5)$ denote the space of holomorphic vector fields, functions, and one-forms, respectively.

The cohomology with respect to the remaining linearized BRST operator consists of the space of triples $(\mu, [\gamma], b)$ where:
\begin{itemize}
\item $\mu$ is a divergence-free holomorphic vector field on $\CC^5$, which is constant along $\RR$
\[
\mu = \mu \otimes 1 \in \Pi \Vect_0(\CC^5) \otimes \Omega^0(\RR) .
\]
Note that $\mu$ is a ghost in the $\ZZ/2$ graded theory. 
\item $[\gamma]$ is an equivalence class of a holomorphic one-form modulo exact holomorphic one-forms along $\CC^5$, which are also constant along $\RR$
\[
[\gamma] = [\gamma] \otimes 1 \in \left(\Omega^{1}(\CC^5) / \d \cO(\CC^5) \right) \otimes \Omega^0(\RR) .
\]
\item A constant function $b \in \Pi \CC$ on $\CC^5 \times \RR$.
This is a $\beta$-type field in the 11-dimensional theory, any constant function is closed for the de Rham differential. 
This element is also a ghost in the $\ZZ/2$-graded theory. 
\end{itemize}

\parsec[]

After parity shifting, we've identified the solutions to the linear equations of motion with triples
\[
(\mu, [\gamma], b) \in \Vect_0(\CC^5) \oplus \Pi \Omega^{1}(\CC^5) / \del \cO(\CC^5) \oplus \CC .
\]
The bracket induced by the cubic component of $S_{BF, \infty}$ in the classical BV action is the usual bracket on divergence-free vector fields together with the module structure on holomorphic one-forms by Lie derivative.
Notice that the Lie derivative commutes with the $\del$ operator, so this action descends to equivalence classes as above. 
The elements $b$ are central. 

The final term in the BV action $J = \frac16\int \gamma \wedge \del \gamma \wedge \del \gamma$ induces the following Lie bracket on the solutions to the linearized equations of motion
\beqn\label{eqn:eqb}
\big[[\gamma], [\gamma'] \big] = \Omega^{-1} \vee (\del \gamma \wedge \del \gamma') \in \Vect_0(\CC^5) .
\eeqn
where $\Omega^{-1}$ denotes the section of $\PV^{5,hol}(\CC^5)$ which is inverse to the Calabi--Yau form $\Omega$ on $\CC^5$. 
Notice that this bracket is well-defined as it does not depend on the particular equivalence classes and that the resulting vector field is automatically divergence-free.

\parsec[]

We complete the proof of the first part of Theorem \ref{thm:global}. 

The relationship of the $\mu$-elements in $E(5,10)$ and the 11-dimensional theory is apparent.

Next, we need to relate the equivalence classes $[\gamma]$ with the closed two-forms $\alpha$ in $E(5,10)$. 
On flat space, any closed differential form is exact (this is a holomorphic version of the Poincar\'e lemma). 
In other words, there is an isomorphism
\[
\del \colon \Omega^1 (\CC^5) / \d \cO(\CC^5) \xto{\cong} \Omega^{2}_{cl}(\CC^5)
\]
induced by the holomorphic de Rham differential.
This gives the relationship between the equivalence class $[\gamma]$ in the 11-dimensional theory and a closed two-form in $E^{hol}(5,10)$ by $\alpha = \del \gamma$. 
It is clear from Equations \eqref{eqn:e510} and \eqref{eqn:eqb} that this assignment intertwines the Lie brackets in $E^{hol}(5,10)$ and the twist of 11-dimensional supergravity. 
This completes the proof of part (1) of Theorem \ref{thm:global}.

\subsection{(Non)formality and homotopy transfer} 

We produce the following homotopy data:
\begin{equation}
\begin{tikzcd}
\arrow[loop left]{l}{K}(\Pi \cE , \delta^{(1)})\arrow[r, shift left, "q"] &(E^{hol}(5,10) \oplus \CC_b \, , \, 0)\arrow[l, shift left, "i"] \: ,
\end{tikzcd}
\end{equation}

\begin{itemize}
\item On the $\nu$'s we take $K$ be any operator $K \colon \cO \to \Vect$ such that $\div K \nu = \nu$. 
On the $\gamma$'s we take $K$ be any operator $K \colon \Omega^1 \to \Omega^0$ such that $\del K(\gamma) = \gamma$. 
Also, introduce the auxiliary operator $\til{K} \colon \Omega^2_{cl} \to \Omega^1$ which satisfies the homotopy relation
\beqn\label{eqn:htpy1}
\til{K} \del \gamma + \del K \gamma = \gamma . 
\eeqn
The precise form of each of these operators will not be needed.
The existence of such operators is guaranteed by the holomorphic Poincar\'e lemma.
The operator $K$ annihilates fields $\beta$ and $\mu$. 
\item 
The map $q$ is described as follows. 
First $q(\mu) = \mu - K \div (\mu)$.
Notice that $q(\mu)$ is automatically divergence-free.
Next, $q(\gamma) = [\gamma]$, the equivalence class in $\Omega^1 / \d$. 
If $\beta$ is a holomorphic function, then $q(\beta) = \beta (z=0)$.
\item 
The map $i$ embeds $\mu$ and $b$ in the obvious way.
On the equivalence class $[\gamma] \in \Omega^1 / \d$ we define $i([\gamma]) = \gamma - \til{K} \del \gamma$. 
Notice that this is independent of the choice of representative $\gamma$. 
\end{itemize}

It is straightforward to check that this comprises well-defined homotopy data, the only nontrivial thing to check is the relation $\id - i \circ q = \delta^{(1)} K - K \delta^{(1)}$. 
Plugging in the field $\gamma$ we see that we must check that
\[
\gamma - \til{K} \del \gamma = \del K \gamma 
\]
which is precisely \eqref{eqn:htpy1}. 

Given this homotopy data, we can compute the homotopy transferred $L_\infty$ structure on the linearized BRST cohomology following \brian{??}. 
Since $\nu$ does not survive to cohomology and the fact that there are no nontrivial Lie brackets involving the field $\beta$, this transferred structure is easy to compute. 

There is a single diagram which contributes to the transferred structure, it is given by
\brian{do it}

This diagram leads to a new $3$-ary bracket on $E^{hol}(5,10) \oplus \CC_b$
\[
\big[\mu,\mu',[\gamma]\big]_3 = \varphi(\mu,\mu',[\gamma])
\]
where $\varphi \in \clie^{even} (E^{hol}(5,10))$ is the even Lie algebra cocycle defined by the formula
\beqn
\begin{array}{rclr}
\varphi \colon E^{hol}(5,10) \times E^{hol}(5,10) \times E^{hol}(5,10) & \to & \CC_b \\
\varphi(\mu,\mu',\alpha) & = & \<\mu \wedge \mu', \alpha\>|_{z=0} .
\label{eqn:cocycle}
\end{array}
\eeqn
Since $b$ is central, this cocycle defines a central extension of $E^{hol}(5,10)$. 

\parsec[]
We briefly remark on Lie algebra cohomology for super Lie algebras.
The Lie algebra cohomology $\clie^{\bu,\bu}(\cL)$ of any super Lie algebra $\cL$ is graded by $\ZZ \times \ZZ/2$. 
The first grading is by the symmetric degree in the Chevalley--Eilenberg complex.
The second grading is the internal parity of the super Lie algebra $\cL$. 
The Chevalley--Eilenberg differential is degree $(1,+)$. 

The cocycle $\varphi$ has homogenous bigrading $(3,-)$.
In the above discussion we forgot the bigrading to a totalized $\ZZ/2$ grading where 
\begin{align*}
\clie^{even} (\cL) & = \clie^{2\bu , +} (\cL) \oplus \clie^{2\bu+1, -}(\cL) \\
\clie^{odd} (\cL) & = \clie^{2\bu , -} (\cL) \oplus \clie^{2\bu+1, +}(\cL) .
\end{align*}
With this totalization, $\varphi$ is an even cocycle and hence determines a super $L_\infty$ central extension by the one-dimensional even vector space $\CC$. 

\parsec[s:altglobal]

In \S \ref{s:altdfn} we gave an equivalent description of the 11-dimensional theory as a deformation of the BF action $S_{BF,0}$ by the functional $\til J$. 

\subsection{Twisted matrix model}

\section{Residual supersymmetry} 
\label{sec:susy}
%m2brane

In this section we consider the minimal twist of 11-dimensinoal supersymmetry explicitly. 
We compute the residual supersymmetry algebra given by taking the cohomology of the 11-dimensional supersymmetry algebra with respect to the minimal twisting supercharge. 
In order for this to be a symmetry of the 11-dimensional theory it is necessary to perform a central extension of the 11-dimensional supersymmetry algebra by the $M2$ brane.
We will see how this central extension is compatible, upon twisting by the minimal supercharge, with the central extension of $E(5,10)$ we found as the global symmetry algebra in the previous section. 

\subsection{Supersymmetry in 11 dimensions}
\label{sec:11dsusy}

The (complexified) eleven-dimensional supertranslation algebra is a complex super Lie algebra of the form
\[
  \ft_{11d} = V \oplus \Pi S
\]
where $S$ is the (unique) spin representation and $V \cong \CC^{11}$ the complex vector representation, of~$\lie{so}(11, \CC)$. 
The bracket is the unique surjective $\lie{so}(11,\CC)$-equivariant map from the symmetric square of~$S$ to~$V$;
this decomposes into three irreducibles, 
\beqn\label{eqn:decomp}
  \Sym^2(S) \cong V \oplus \wedge^2 V \oplus \wedge^5 V.
\eeqn
Denote by $\Gamma_{\wedge^1}, \Gamma_{\wedge^2}, \Gamma_{\wedge^5}$ the projections onto each of the summands above. 
The bracket in $\ft_{11d}$ is defined using the first projection
\[
[\psi, \psi'] = \Gamma_{\wedge^1} (\psi, \psi') .
\]
The super Poincar\'{e} algebra is
\[
  \lie{siso}_{11d} = \lie{so}(11 , \CC) \ltimes \ft_{11d} .
\]
The $R$-symmetry is trivial in 11-dimensional supersymmetry. 

\subsection{Central extensions of the supersymmetry algebra} 
\label{sec:m2brane}

Extensions of the supersymmetry algebra correspond to the existence of extended objects, such as branes, in the supergravity theory \brian{good reference?}. 
In 11-dimensional supersymmetry, there are two such extensions corresponding to the $M2$ brane and the $M5$ brane.
We begin by describing a less standard dg Lie algebra model for the $M2$ brane algebra.
In the next section we will explain the relationship to other descriptions in terms of $L_\infty$ algebras \cite{Basu_2005,Bagger_2007,fiorenza2015super}. 

Our model for the $M2$ brane algebra is a dg Lie algebra extension of the super Poincar\'e algebra $\lie{siso}_{11d}$.
Introduce the cochain complex $\Omega^{\bu}(\RR^{11})$ of (complex valued) differential forms on $\RR^{11}$ equipped with the de Rham differential $\d$.
 
The $M2$ brane algebra arises as a central extension of $\lie{siso}_{11d}$ by the cochain complex $\Omega^\bu(\RR^{11})[2]$ and is defined by a cocycle
\[
    c_{M2} \in \clie^{2,+} \left(\lie{siso}_{11d} \; ; \; \Omega^\bu (\RR^{11})[2]\right) .
\]
The formula is
  \[c_{M2} (\psi, \psi') = \Gamma_{\wedge^2}(\psi, \psi') \in \Omega^2(\RR^{11})\]
  where $\Gamma_{\wedge^2}$ is the projection onto $\wedge^2 V$, thought of as the space of constant coefficient two-forms, as in the decomposition \eqref{eqn:decomp}.
  
Here, we are using a bigrading by $\ZZ \times \ZZ/2$. 
The super Poincar\'e algebra is concentrated in zero integer grading and carries is natural $\ZZ/2$ grading as a super Lie algebra.
The complex $\Omega^{\bu}(\RR^{11})[2]$ is concentrated in integer degrees $[-2,9]$ and has even parity.

\begin{dfn}
The algebra $\m2$ is the $\ZZ \times \ZZ/2$-graded dg Lie algebra defined by the central extension of $\lie{siso}_{11d}$ by the cocycle $c_{M2}$.  
\end{dfn}

The bracket in $\m2$ is bidegree $(0,+)$ and the differential is bidegree $(1,+)$.

\subsection{The minimal twist}
\label{sec:mintwist}

Fix a rank one supercharge $Q \in S$ satisfying $Q^2 = 0$.
Such a supercharge defines the minimal twist of 11-dimensional supersymmetry. 
\brian{cite \cite{SWspinor}}
We characterize the cohomology of the algebra $\m2$ with respect to this supercharge. 

$Q$ defines a maximal isotropic subspace $L \subset V$. 
In turn, we will decompose the super Poincar\'e algebra into $\lie{sl}(L) = \lie{sl}(5)$ representations.
First, the defining and spinor representations decompose as
\deq{
  V = L \oplus L^\vee \oplus \CC_t, \qquad S = \wedge^\bu L^\vee.
}
In the expression for $S$, we are omitting factors of $\det(L)^{\frac12}$ for simplicity. 
Also, $\lie{so}(11, \CC)$ decomposes as
\[
\lie{sl}(5) \oplus \wedge^2 L \oplus \wedge^2 L^\vee \oplus L \oplus L^\vee \oplus \C .
\]
Furthermore, the spinorial representation can be identified with
\[
S = \wedge^\bu (L^\vee) = \CC \oplus L^\vee \oplus \wedge^2 L^\vee \oplus \wedge^3 L^\vee \oplus \wedge^4 L^\vee \oplus \wedge^5 L^\vee .
\]
The element $Q$ lives in the first summand.
Let ${\rm Stab}(Q) \subset \lie{so}(11,\CC)$ be the stabilizer of $Q$. 
This is a parabolic subalgebra whose Levi factor is $\lie{sl}(5)$.

\subsection{$Q$-cohomology of $\m2$}
\label{sec:m2branetwist}

Any element $Q \in S$ satisfying $Q^2 = 0$ determines a deformation of the dg Lie algebra $\m2$.
To deform $\d$ by $Q$ we must break the $\ZZ \times \ZZ/2$ bigrading.
The supercharge $Q$ is odd and of cohomological degree zero.
Recall, the original differential on $\m2$ is the de Rham differential $\d$ which just acts on the central summand and is even of cohomological degree $+1$.
Thus, only the totalized $\ZZ/2$ grading makes the differential $\d + [Q,-]$ homogenous. 

\begin{dfn}
The $Q$-twist $\m2^Q$ of $\m2$ is the super dg Lie algebra whose differential is $\d + [Q,-]$.
The bracket is unchanged.
\end{dfn}

We now assume that $Q$ is a rank one, or minimal, supercharge satisfying $Q^2 = 0$. 

\begin{prop}\label{prop:susycoh}
As a $\ZZ/2$ graded space, the cohomology of the $Q$-twist $\m2^Q$ is
\beqn\label{eqn:susycoh}
L^\vee \oplus {\rm Stab}(Q) \oplus \Pi \left(\wedge^2 L\right) \oplus \CC
\eeqn
whose elements we denote by $(v, m, \psi, c)$.

\begin{enumerate}
\item As a super Lie algebra, the cohomology of $\m2^Q$ is the natural extension of ${\rm Stab}(Q)$ together with the bracket
\beqn\label{eqn:susy2bra}
[\psi, \psi']_2 = \psi \wedge \psi' \in \wedge^4 L \cong L^\vee_v \\
\eeqn
\item 
$\m2^Q$ is not formal as a super dg Lie algebra.
As a super $L_\infty$ algebra, the $Q$-twist is equivalent to \eqref{eqn:susycoh} with $2$-brackets described in (1) where we additionally introduce the $3$-ary bracket 
\beqn\label{eqn:susy3bra}
[v, v', \psi]_3 = 4 \<v \wedge v', \psi\> \in \CC_c .
\eeqn
\end{enumerate}
\end{prop}

It will be useful to list the formulas for the brackets in terms of coordinates. 
Let $\{z_i\}$ denote a basis for $L$, which we will also think of as a linear coordinate on $\CC^5$. 
Let $\{\partial_{z_i}\}$ be a dual basis for $L^\vee$, which we will also think of as translation invariant vector fields.
The $2$-ary bracket above is 
\[
[z_i \wedge z_j, z_k \wedge z_l]_2 = \ep_{ijklm} \partial_{z_m} 
\]
and the $3$-ary bracket is
\[
[\partial_{z_i}, \partial_{z_j}, z_{k} \wedge z_{\ell}]_3 = 4 (\delta^i_k \delta^j_\ell - \delta^i_\ell \delta^j_k) .
\] 

\parsec[]

One outcome of this proposition is that the dg Lie algebra $\m2^Q$ is {\em not} formal; there is a 3-ary $L_\infty$ bracket present in cohomology. 
One way to prove the proposition above is to use homotopy transfer directly to $\m2^Q$, just as we did in \S \ref{s:ht} to deduce the form of the $3$-ary bracket. 
Instead, we will use the following minimal model for $\m2^Q$ to prove Proposition \ref{prop:susycoh}.
This minimal model also has the advantage of being more directly related to the 11-dimensional supergravity theory.

\begin{lem}
\label{lem:gmodel}
Let $\fg$ denote the following $\ZZ/2$ graded dg Lie algebra which as a cochain complex is
\[
H^\bu(\m2^Q) \oplus (L \xto{\id} \Pi L)  .
\]
Denote the elements of the second summand by $(\lambda, \til\lambda)$. 
The Lie structure extends the one on $H^\bu(\m2^Q)$ described in (1) of Proposition \ref{prop:susycoh} together with the brackets
\begin{align*}
[v,\lambda] & = \<v, \lambda\> \in \CC_c \\ 
[v,\psi] & = \<v, \psi\> \in \Pi L_{\Tilde{\lambda}}.
\end{align*}

There is an $L_\infty$ map 
\[
\fg \rightsquigarrow \m2^Q
\] 
which is a quasi-isomorphism of cochain complexes.  
\end{lem}
\begin{proof}

The cohomology of the non-centrally extended algebra was computed in \cite{SWspinor}, we briefly recall the result. 
The element $Q$ only acts nontrivially on the summands $\wedge^4 L^\vee$ and $\wedge^5 L^\vee$ in $S$. 
The image of $\wedge^4 L^\vee \cong L$ trivializes the antiholomorphic translations while the image of $\wedge^5 L^\vee$ trivializes the time translation.
So, of the translations, only the holomorphic ones $L^\vee$ survive.
The map 
\[
[Q,-] \colon \lie{so}(11,\CC) \to S 
\] 
is the projection onto $\wedge^0 L^\vee \oplus \wedge^1 L^\vee \oplus \wedge^2 L^\vee$. 
The kernel of $[Q,-]$ is the stabilizer~${\rm Stab}(Q)$.

In summary, the space of odd translations which survive cohomology is $\wedge^3 L^\vee \cong \wedge^2 L$.
This completes the calculation of the cohomology. 

We embed $\fg$ into $\m2^Q$ in the following way: ${\rm Stab}(Q)$ and $L^\vee$ sit inside in the evident way.
The central element maps to $c \mapsto - 1 \in \Omega^0(\RR^{11})$.
The summand $L_\lambda$ is mapped to the linear functions in $\Omega^0(\RR^{11})$ and $\Pi L_{\Tilde{\lambda}}$ is sent to the constant coefficient one-forms in $\Pi \Omega^1(\RR^{11})$. 
It remains to define where $\psi \in \wedge^2 L$ is mapped.

Notice that naively, $\psi \in \wedge^2 L$ is not $Q$-closed due to the presence of the central extension. 
To embed $\wedge^2 L$ we introduce the following operator
\[
H \colon \Omega^2 (\RR^{11}) \to \Omega^1(\RR^{11})
\]
which sends a two-form $\alpha$ to the one-form $H \alpha$ defined by the formula $(H \alpha) (x) = \int_0^x \alpha$
where we integrate over a straight line path from $0$ to $x$.

Notice that if $\alpha$ is $\d$-closed then $\d (H \alpha) = \alpha$. 
It follows that any element $\psi \in \wedge^2 L \subset S$ can be lifted to a closed element at the cochain level in $\m2^Q$ by the formula
\[
\Tilde{\psi} = \psi - H \Gamma_{\wedge^2} (Q, \psi) \in \Pi S \oplus \Pi \Omega^1 .
\]
Thus, sending $\psi \mapsto \Tilde{\psi}$ defines a cochain map $\fg \to \m2^Q$. 

The Lie bracket $[\Tilde{\psi}, \Tilde{\psi}']$ agrees with $[\psi, \psi']$. 
On the other hand, in $\m2^Q$ there is the Lie bracket 
\[
[v,\Tilde{\psi}] = - L_v (H \Gamma_{\wedge^2} (Q, \psi)) = -\<v, \Gamma_{\wedge^2}(Q, \psi)\> - \d \<v, H \Gamma_{\wedge^2}(Q, \psi)\> .
\]
The first term agrees with the bracket $[v, \psi]_{\fg}$ in $\fg$. 
The other term is exact in $\m2^Q$ and can hence be corrected by the following bilinear  
\[
v \otimes \psi \mapsto \<v, H \Gamma_{\wedge^2} (Q,\psi) \> \in L_\lambda .
\] 
Together with the cochain map described above, this bilinear term prescribes the desired $L_\infty$ map. 

\end{proof}

\parsec[]

Using the model $\fg$ the first part of Proposition \ref{prop:susycoh} follows immediately. 
We deduce the second part using homotopy transfer. 

Recall that we described the cohomology of $\m2^Q$ in \eqref{eqn:susycoh}.
Let $\delta$ denote the differential on $\fg$ which simply maps $\Pi L$ to $L$ by the identity map. 
We produce the homotopy data
\begin{equation}
\begin{tikzcd}
\arrow[loop left]{l}{K}(\fg , \delta)\arrow[r, shift left, "q"] &(H^\bu(\m2^Q) \, , \, 0)\arrow[l, shift left, "i"] \: ,
\end{tikzcd}
\end{equation}
as follows.
\begin{itemize}
\item The operator $K$ annihilates $H^\bu(\m2^Q)$ and is the identity map~$K \colon \Pi L_{\til \lambda} \to L_\lambda$. 
\item The map $q$ is the identity on $H^\bu(\m2^Q)$ and annihilates the summand~$L \to \Pi L$. 
\item The map $i$ embeds $H^\bu(\m2^Q)$ in the obvious way. 
\end{itemize}

It is immediate to verify this data prescribes valid homotopy data.
There is only a single term in the $L_\infty$ structure generated by homotopy transfer. 
It is determined by the following tree diagram. 
\brian{do it}
This recovers the formula in (2) of Proposition \ref{prop:susycoh}.

\subsection{Embedding supersymmetry into the 11-dimensional theory} \label{s:residual}

Consider now the super $L_\infty$ algebra $\cL$ underlying the eleven-dimensional theory on $\CC^5 \times \RR$. 

\begin{prop}
Endow the cohomology of $\m2^Q$ with the $L_\infty$ structure of Proposition \ref{prop:susycoh} and let $\cL(\CC^5 \times \RR)$ be the super $L_\infty$ algebra underlying 11-dimensional supergravity on $\CC^5 \times \RR$. 
There is a map of super $L_\infty$ algebras 
\[
H^\bu(\m2^Q) \rightsquigarrow \cL (\CC^5 \times \RR)
\]
where $\fg$ is the model for the $Q$-cohomology of the super Lie algebra $\m2$ from Proposition \ref{lem:gmodel}. 
In particular, the $Q$-twisted algebra $\m2^Q \simeq \fg$ is a symmetry of 11-dimensional theory on $\CC^5 \times \RR$. 
\end{prop}
\begin{proof}
Recall the cohomology of $\m2^Q$ takes the following form
\beqn 
\begin{tikzcd}
\ul{even} & \ul{odd} & \ul{even} \\
 L & \wedge^2 L_2 & L^\vee \\
\wedge^2 L_1 & & \\
\lie{sl}(5) && \CC_c  \\
%L_2 \ar[r, "\id"] & L_3 \\ 
 .
\end{tikzcd}
\eeqn
The lefthand column is ${\rm Stab}(Q)$. 
The subscripts in $\wedge^2 L_1, \wedge^2 L_2$ are used to distinguish between the two copies of $\wedge^2 L$.

The $L_\infty$ map from the dg Lie model $\fg$ to the fields of the twisted $11$-dimensional supergravity theory has a linear $\Phi^{(1)}$ and quadratic $\Phi^{(2)}$ piece.

Define the linear map $\Phi^{(1)} \colon \fg \to \cL$ as follows:
\begin{align*}
 L & \mapsto 0 \\
 \wedge^2 L_1 & \mapsto 0 \\
z_i \wedge z_j \in \wedge^2 L_2 & \mapsto \frac12 (z_i \d z_j - z_j \d z_i) \in \Omega^{1,0} (\CC^5) \hotimes \Omega^0 (\RR) \\
A_{ij} \in \lie{sl}(5) & \mapsto \sum_{ij} A_{ij} z_i \partial_{z_j} \in \PV^{1,0}(\CC^5) \hotimes \Omega^0(\RR) \\ \partial_{z_j} \in L^\vee & \mapsto
\partial_{z_i} \in \PV^{1,0} (\CC^5) \hotimes \Omega^0 (\RR^5) \\ %z_i \in %L_2 & \mapsto z_i \in \Omega^{0,0}(\CC^5) \hotimes \Omega^0 (\RR) \\
%z_i \in L_3 & \mapsto \d z_i \in \Omega^{1,0}(\CC^5) \hotimes \Omega^0 (\RR) \\
1 \in \CC_c & \mapsto 1 \in \Omega^{0,0}(\CC^5) \hotimes \Omega^0 (\RR) .
\end{align*}

It is immediate to check that this is a map of cochain complexes since all elements in the image of this map lie in the kernel of the linearized BRST operator \eqref{eqn:linearBRST}. 

This map also preserves the bracket between odd elements in $\wedge^2 L_2$. 
In the cohomology of $\m2^Q$ we have the bracket
\[
[z_i\wedge z_j , z_k \wedge z_l] = \ep_{ijklm} \partial_{z_m}
\]
which is precisely the bracket induced by the cubic term in the action $J = \frac16 \in \gamma \del \gamma \del \gamma$. 

This map does not preserve all of the brackets, however. 
Indeed, in the 11-dimensional theory $\cL(\CC^5 \times \RR)$ there is the bracket 
\[
\left[\partial_{z_i}, z_j \d z_k - z_k \d z_j\right] = \delta^i_j \d z_k - \delta^i_k \d z_j 
\]
arising from the cubic term in $\frac12 \int \frac{1}{1-\nu} \mu^2 \del \gamma$. 
To remedy the failure for $\Phi^{(1)}$ to preserve the brackets, we introduce the odd bilinear map $\Phi^{(2)} \colon \fg \times \fg \to \Pi \cL$ defined by 
\beqn\label{eqn:phi2}
\Phi^{(2)} \left(\partial_{z_i} , z_j \wedge z_k\right) = \frac12 (\delta^i_j z_k - \delta^i_k z_j) .
\eeqn
Notice that the field on the right hand side is of type $\beta$. 

The bilinear map $\Phi^{(2)}$ provides a homotopy trivialization for the failure for $\Phi^{(1)}$ to preserve the $2$-ary bracket: 
\[
[\Phi^{(1)} (\partial_{z_i}) , \Phi^{(1)}(z_j \wedge z_k)] = \del \Phi^{(2)}\left(\partial_{z_i} , z_j \wedge z_k\right).
\]
The lefthand side is $\frac12 (\delta_j^i \d z_k - \delta_k^i \d z_j)$ which is precisely the de Rham differential applied to \eqref{eqn:phi2}.

To define an $L_\infty$ morphism $\Phi^{(1)} + \Phi^{(2)}$ must satisfy additional higher relations. 
There is a single nontrivial cubic relation to verify:
\begin{multline} \label{eqn:cubicrln}
\Phi^{(1)}\left[\partial_{z_i}, \partial_{z_j}, z_k \wedge z_l\right]_3 = [\Phi^{(1)}(\partial_{z_i}), \Phi^{(1)}(\partial_{z_i}), \Phi^{(1)}(z_k \wedge z_l)]_3 \\ + [\del_{z_i}, \Phi^{(2)}(\partial_{z_j}, z_k \wedge z_l)] + [\del_{z_j}, \Phi^{(2)}(\del_{z_i}, z_k \wedge z_l)]
\end{multline}
where $[-]_3$ on the left hand side is the $3$-ary bracket defined in Proposition \ref{prop:susycoh} and $[-]_3$ on the right hand side is the $3$-ary bracket defined by the quartic part of the action $\frac12 \int \frac{1}{1-\nu} \mu^2 \vee \del \gamma$. 
The two terms in the second line of \eqref{eqn:cubicrln} cancel for symmetry reasons and the quartic term in the BV action induces precisely the correct $3$-ary bracket. 

%$[\Phi^{(1)}(z_i \wedge z_j), \Phi^{(1)}(z_k \wedge z_l), \Phi^{(1)} (z_m \wedge z_n)]_3 = \Phi^{(2)} ([z_i \wedge z_j, z_k\wedge z_l)], z_m \wedge z_n) + $permutations, where $[-]_3$ is the $3$-ary bracket arising from the 
\end{proof}

\parsec[]

Because this map preserves differentials, it descends to a map in cohomology. 
We have already computed the cohomology of $\cL$ on $\CC^5 \times \RR$, it is the trivial one-dimensional central extension of $E (5,10)$. 
The Lie algebra structure present in the cohomology of $\m2^Q$ is described in part (1) of Proposition \ref{prop:susycoh}. 
The map
\[
L^\vee \oplus {\rm Stab}(Q) \oplus \Pi \left(\wedge^3 L^\vee\right) \oplus \CC_c \to E (5,10) \oplus \CC_{c'}
\]
is defined by very similar formulas as above
\begin{align*}
 L_1 & \mapsto 0 \\
 \wedge^2 L_1 & \mapsto 0 \\
z_i \wedge z_j \in \wedge^2 L_2 & \mapsto \d z_i \wedge \d z_j \in \Omega^{2}_{cl} (\CC^5) \\
A_{ij} \in \lie{sl}(5) & \mapsto \sum_{ij} A_{ij} z_i \partial_{z_j} \in \Vect_0(\CC^5) \\ \partial z_i \in L^\vee & \mapsto
\partial z_i \in \Vect_0(\CC^5) \\
c \in \CC_c & \mapsto c \in \CC_{c'} .
\end{align*}

The relationship between the transferred $L_\infty$ structures can be described as follows. 
Recall, that the linear BRST cohomology of the parity shift of the fields of the 11-dimensional theory is equivalent to the super $L_\infty$ 
algebra $\Hat{E(5,10)}$ which is a central extension $E(5,10)$ by the cocycle \eqref{eqn:cocycle}.
Also, we described the $L_\infty$ structure present in the cohomology of $\m2^Q$ in part (2) of Proposition \ref{prop:susycoh}. 
Each of these $L_\infty$ structure involved introducing a single new $3$-ary bracket, which are easily seen to be compatible. 

\parsec[]

In this section we compare to the work of \brian{Linfty refs}.
In these references, the algebra $\m2$ is defined as an $L_\infty$  
central extension of $\lie{siso}_{11d}$. 

Recall that given two spinors $\psi, \psi' \in S$ we can form the constant coefficient two-form $\Gamma_{\wedge^2} (\psi, \psi')$. 
Using this two-form we can define the following four-linear expression
\[
\mu_2 (\psi, \psi',v,v') = \<v \wedge v', \Gamma(\psi, \psi')\> .
\]
This expression is symmetric on the spinors and anti-symmetric on the vectors, therefore it defines an element in $\clie^4(\ft_{11d})$. 
In \cite{fiorenza2015super} it is shown that $\mu_2$ defines a nontrivial class in $H^4(\ft_{11d})$ so defines a one dimensional central extension of $\ft_{11d}$ as a lie 3-algebra. Instead of working with a one-dimensional central extension by $\C[2]$, we work with a central extension by the resolution $\Omega^\bullet(\R^{11})[2]$. There is a quasi-isomorphism $\clie^4(\ft_{11d})\to \clie^4(\ft_{11d}, \Omega^\bullet (\R^11))$ such that the induced map on cohomology identifies $\mu_2$ with the $\Omega^\bullet(\R^{11})[2]$-valued 2-cocycle we use. 

%For clarity we adjust notation for $\lie{sl}(5)$-representations. Denote by $V^{1,0} = L^\vee$ the space of holomorphic translations on $\CC^5$ and $V^{\vee 1,0}$ the translation invariant holomorphic one-forms on $\CC^5$. 

\section{The non-minimal twist} 

We have provided numerous consistency checks that the 11-dimensional theory defined on a manifold with $SU(5)$ holonomy is a twist of supergravity. 
We have referred to this theory as minimal as it depends on the complex structure in the maximal number of directions. 
In this section we characterize a further twist of 11-dimensional supergravity from the lens of the holomorphic theory.  

On flat space, the further twist is essentially unique and renders the theory topological in seven directions, rather than just one as in the holomorphic twist. 
We will show that it is equivalent to a theory on $\CC^2 \times \RR^7$ that we call ``Poisson'' Chern--Simons theory. 
In the BV formalism, the theory $\ZZ/2$ graded and has fields given by
\[
A \in \Pi \Omega^{0,\bu}(\CC^2) \hotimes \Omega^\bu(\RR^7) ,
\]
where $\Pi$, as always, denotes parity shift.
The equations of motion are of the form
\[
\dbar A + \d_{\RR^7} A + \partial_{z_1} A \wedge \partial_{z_2} A = 0 .
\]
The action functional depends on the holomorphic symplectic structure on $\CC^2$ through the Poisson bracket on the algebra of holomorphic functions.
We give a precise definition below. 

The main result of this section is the following.

\begin{thm}
\label{thm:nonmin}
The non-minimal twist of the 11-dimensional theory is equivalent to Poisson Chern--Simons theory on 
\[
\CC^2 \times \RR^7 .
\]
\end{thm}

\subsection{Index matching}
\label{sec:indexcheck}

As a first consistency check, we can compare deformation invariants attached to the holomorphic twist and the $G2\times SU(2)$ twist. \brian{If we are going to say G2 we need to explain how G2 arises. I've been calling it the non minimal twist.}
We will find that the local character of the latter agrees with a specialization of the local character computed in proposition \ref{prop:locchar}

\begin{prop}
The  local character of the $G2\times SU(2)$ twist of 11d supergravity on flat space is given by
\[
\chi_{SU(3)\times SU(2)}(\Obs_0^{\cE_{G2\times SU(2)}}) = \prod _{(n_1,n_2)\in \Z^2_{\geq 0}} \frac{1}{1-q^{-n_1+n_2}}.
\] 
This agrees with the specializaiton of the local character computed in proposition \ref{prop:locchar} at $y=1$.
\end{prop}
\begin{proof}
The space of solutions to linearized equations of motion is parametrized by a holomorphic function $A$ on $\C^2_{w_j}$. The corresponding linear local operators are labeled by $(n_1,n_2)\in \Z^2_{\geq 0}$  and are given by 
\[
A_{(n_1,n_2)} : A \mapsto \partial_{w_1}^{n_1}\partial_{w_2}^{n_2} A (0).
\]

The character of the linear span of these is given by the geometric series
\[
\sum _{(n_1,n_2)\in \Z^2_{\geq 0}} q^{-n_1+n_2}
\] 
with plethystic exponential given by 
\[
\prod _{(n_1,n_2)\in \Z^2_{\geq 0}} \frac{1}{1-q^{-n_1+n_2}}.
\]

For the last part, it suffices to show that the characters of linear local operators agree upon evaluating $y=1$. Summing the geometric series giving the character of linear local operators in the holomorphic twist yields the rational function \[\frac{\left (\begin{aligned}t_1y^{-1} + t_1^{-1}t_2y^{-1} + t_2^{-1}y^{-1} + q^{-1}y^{\frac 3 2} + qy^{\frac 3 2} \\  - (t_1^{-1}y + t_1t_2^{-1}y + t_2y &+ q^{-1}y^{-\frac 3 2} + qy^{-\frac 3 2})\end{aligned}\right)}{(1-t_1^{-1}y)(1-t_1t_2^{-1}y)(1-t_2y)(1-q^{-1}y^{-3/2})(1-qy^{-3/2})}.\] Specializing $y=1$, the numerator cancels with the factors in the denominator involving the $t_i$. The result is \[\frac{1}{(1-q^{-1})(1-q)}\] which is exactly the sum of the geometric series we found

\end{proof}

%Notice that changing the values of $i,j$ just has the affect of permuting the holomorphic copies of $\CC^2$ leftover in the further twist. 

%The Lie algebra of gauge symmetries of this model on flat space is 
%\[
%\Omega^{0,\bu} (\CC_i \times \CC_j) \hotimes \Omega^\bu(\RR^7) 
%\]
%which is quasi-isomorphic to $\cO^{hol}(\CC^2)$ equipped with the Poisson bracket $\{-,-\}_{pb}$. 
%
%More generally, the twist can be defined \brian{finish}

\subsection{The non-minimal supercharge}

From the point of view of the untwisted theory, the non-minimal twist is defined by working in a background where the fermionic ghost in the physical theory is equal to a supertranslation of the form
\[
Q + Q_{nm} 
\]
where $Q$ is the supertranslation which defines the minimal twist, see \S \ref{sec:mintwist}.
The minimal twist of supergravity is obtained by setting a fermionic ghost equal to $Q$. 

In the language of the minimal twist, the supercharge $Q_{nm}$ determines a square-zero element in the $Q$-cohomology of the original supersymmetry algebra (which we will denote by the same letter). 
The characterization of this cohomology in Proposition \ref{prop:susycoh} implies that $Q_{nm}$ is an element 
\[
Q_{nm} \in \wedge^2 L 
\]
where $L \cong \CC^5$ is the defining $SU(5)$ representation. 
In other words, $Q$ is a translation invariant holomorphic two-form on $\CC^5$. 
The condition that $[Q_{nm}, Q_{nm}] = 0$ simply says that $Q_{nm}\wedge Q_{nm} = 0$ as a translation invariant four-form on $\CC^5$. 
By a linear change of coordinates, all such two-forms $Q$ are of the form $Q_{nm} = \d z_i \wedge \d z_j$ where $i,j=1,\ldots, 5$.

From hereon in this section we will rename coordinates by
\[
\CC^5 \times \RR = \CC^2_{z_i} \times \CC^3_{w_a} \times \RR
\]
which is most natural from the point of view of the non-minimal twist. 
We will fix the non-minimal supercharge 
\[
Q_{nm} = \d z_1 \wedge \d z_2 .
\]
Notice that this choice of supercharge breaks the holonomy of the 11-dimensional theory from $SU(5)$ to $SU(2)$. 

We constructed an embedding of the $Q$-cohomology of the supersymmetry algebra into the fields of our 11-dimensional theory on $\CC^5 \times \RR$. 
The further twist is obtained by working in a background where a certain field on $\CC^5 \times \RR$ takes nonzero value $Q_{nm}$. 
Explicitly, the element $Q_{nm} \in \wedge^2 L$ corresponds to the image under $\del$ of a $\gamma$-field of type $\Omega^{1,0}(\CC^5) \otimes \Omega^0(\RR)$. 
According to the embedding in \S \ref{s:residual} this is the $\gamma$-field 
\beqn\label{eqn:gammanm}
\gamma_{nm} = \frac12 (z_1 \d z_2 - z_2 \d z_1) \in \Omega^{1,0}(\CC^5) \otimes \Omega^0(\RR) 
\eeqn
Notice that $\del \gamma_{nm} = \d z_1 \wedge \d z_2$ as desired.

\parsec[sec:nmsymmetry]
Before proceeding to the proof of the theorem above, we perform a simple calculation of the global symmetry algebra present in the $Q_{nm}$-twisted theory. 

Recall that up to a copy of constant functions, the global symmetry algebra of the holomorphic twist of the 11-dimensional theory is the super Lie algebra $E(5,10)$.
From this point of view, the global symmetry algebra of the $Q_{nm}$-twisted theory is given by deformation of this super Lie algebra by the Maurer--Cartan element 
\[
\d z_1 \wedge \d z_2 \in \Omega^{2}_{cl}(\CC^5) .
\]
We recall that the space of closed two-forms on $\CC^5$ is precisely the odd part of the super Lie algebra $E(5,10)$. 

We compute the cohomology of $E(5,10)$ with respect to the differential which is bracketing with this Maurer--Cartan element. 
Recall that we are using the holomorphic coordinates $(z_1,z_2,w_1,w_2,w_3)$ on $\CC^5$. 

There are the following brackets in the super Lie algebra $E(5,10)$ 
\begin{align*}
[f_l \partial_{z_l} , \d z_1 \wedge \d z_2] & = \del f_i \wedge \d z_j - \del f_j \wedge \d z_i \\
[g_a \partial_{w_a} , \d z_1 \wedge \d z_2] & = 0 \\
[h^{ab} \d w_a \wedge \d w_b , \d z_1 \wedge \d z_2 ] & = \ep_{abc} h^{ab} \partial_{w_c} .
\end{align*}
where $f_l \partial_{z_l}$, $g_a \partial_{w_a}$ are divergence-free vector fields on $\CC^5$ and $h^{ab} \d w_a \wedge \d w_b$ is a closed two-form. 

From these relations, we see that the following elements are in the kernel of $[\d z_1 \wedge \d z_2, -]$:
\begin{itemize}
\item $f(z_i) \partial_{z_1} + g(z_i) \partial_{z_2}$ for holomorphic functions $f,g$ on $\CC_{z_1} \times \CC_{z_2}$ which satisfy 
\[
\del_{z_1} f + \del_{z_2} g = 0 .
\]
In other words, this is a divergence-free vector field on $\CC_{z_1} \times \CC_{z_2}$. 
\item $f_b(z_i, w_a) \partial_{w_b}$ for $f_b$ a holomorphic function on $\CC^5$ where $b=1,2,3$. 
\end{itemize}
It is immediate to check that these are the only nonzero elements in the kernel. 
Further, any element of the second type is clearly exact by the closed two-form $\ep^{ijklm} f \d z_l \d z_m$. 

Thus, the cohomology is the (purely bosonic) Lie algebra of divergence-free vector fields on $\CC^2 = \CC_i \times \CC_j$
\[
H^\bu\big(E(5,10), [\d z_1 \wedge \d z_2, -] \big) \simeq \Vect_0(\CC^2) .
\]

We proved in Theorem \ref{thm:global} that the global symmetry algebra of the 11-dimensional theory on $\CC^5 \times \RR$ is equivalent to a central extension $\Hat{E(5,10)}$ of the super Lie algebra~$E(5,10)$. 

The Lie algebra of divergence-free vector fields on $\CC^2$ also admits a central extension:
\beqn\label{eqn:centralvect}
0 \to \CC \to \cO (\CC^2) \to \Vect_0 (\CC^2) \to 0
\eeqn
where $\cO(\CC^2)$ is equipped with the Poisson bracket with respect to the symplectic form~$\d z_1 \wedge \d z_2$.
These two central extension are compatible. 

%Thus, the Lie algebra of gauge symmetries of Poisson Chern--Simons theory on $\CC_i \times \CC_j \times \RR^7$ is a trivial one-dimensional central extension of the cohomology of $E(5,10)$. 

\begin{prop}
Let $\Hat{E(5,10)}$ be the central extension of $E(5,10)$ which is equivalent to the global symmetry algebra of the 11-dimensional theory on $\CC^5 \times \RR$. 
Then, there is an isomorphism of Lie algebras 
\[
H^\bu \big(\Hat{E(5,10)} , [\d z_1 \wedge \d z_2, -] \big) \simeq \cO(\CC^2) .
\]
\end{prop}
\begin{proof}
The only thing to check is that, in cohomology, the cocycle defining the central extension of $E(5,10)$ is the cocycle exhibiting $\cO(\CC^2)$ as the central extension of divergence-free vector fields. 
Recall that the formula \eqref{eqn:cocycle} for the cocycle is 
\[
\varphi(\mu, \mu', \alpha) = \<\mu \wedge \mu', \alpha\>|_{z=0}.
\]

In cohomology, we obtain the cocycle for divergence-free vector fields by plugging in $\alpha = \d z_1 \wedge \d z_2$. 
This gives the cocycle on $\Vect_0(\CC^2)$ 
\[
(f_i \del_{z_i}, g_j \del_{z_j}) \mapsto (f_1 g_2 - f_2 g_1)(z_1=z_2=0) .
\]
This is the cocycle defining \eqref{eqn:centralvect}, as desired. 
\end{proof}.

This proposition implies that the global symmetry algebra of the non-minimal twist of 11-dimensional supergravity is the Lie algebra $\cO(\CC^2)$. 
We will see that this is compatible with the calculation of the non-minimal twist of the full BV theory. 

\subsection{The non-minimal twist of the 11-dimensional theory}

Now, we turn to deducing the action function of the non-minimal twist and hence the proof of Theorem \ref{thm:nonmin}. 
We will show that the eleven-dimensional theory on $\CC^5 \times \RR$ placed in the background where the $(1,0)$ component of $\gamma$ takes the value $\gamma_{nm}$ \eqref{eqn:gammanm} is equivalent to a theory with a purely Chern--Simons-like action functional that we referred to in the introduction to this section. 

Poisson Chern--Siimons theory is defined on any manifold of the form
\[
Z \times M
\]
where $Z$ is a hyper K\"ahler surface and $M$ is a smooth manifold of real dimension seven. 
The fundamental field of the theory is  
\[
\alpha \in \Pi \Omega^{0,\bu}(Z) \; \Hat{\otimes} \; \Omega^\bu(M)  .
\]
Just in our original 11-dimensional theory, this theory is also only $\ZZ/2$ graded. 

The holomorphic symplectic form $\omega_Z^{2,0}$ on $Z$ induces a Poisson bracket define on all Dolbeault forms $\Omega^{0,\bu}(Z)$ which we denote by $\{-,-\}_{pb}$. 
In local Darboux coordinates $(z_1,z_2)$, this bracket reads
\[
\{\alpha^I (z,\zbar) \d \zbar_I , \alpha'^J (z,\zbar) \d \zbar_J\}_{pb} = (\partial_{z_1} \alpha^I \partial_{z_2} \alpha^J \pm \partial_{z_2} \alpha^I \partial_{z_1} \alpha^J) \d \zbar_I \wedge \d \zbar_J . 
\]
The action functional of Poisson Chern--Simons theory is 
\beqn\label{eqn:pcsaction}
    \frac12 \int_{Z \times M} (\alpha \wedge \d\alpha) \wedge \omega^{2,0}_Z  + \frac16 \int_{Z\times M} \alpha \wedge \{\alpha, \alpha\}_{pb} \wedge \omega^{2,0}_Z
\eeqn
where $\{-,-\}$ is the Poisson bracket induced from the symplectic form $\omega_Z$ on $Z$. 

For simplicity, we will work only on flat space $\CC^5 \times \RR = \CC^2_z \times (\CC^3_w \times \RR)$, where we view $Z = \CC^2_z$ as a hyper K\"ahler manifold with its standard holomorphic symplectic form $\omega^{2,0} = \d^2 z$.

We will decompose the fields according to these coordinates. 
For example, we decompose the $\mu$-field as $\mu = \mu_z + \mu_w$ where
\begin{align*}
\mu_z  &\in \PV^{1,\bu}(\CC^2_z) \otimes \PV^{0,\bu}(\CC^3_w) \otimes \Omega^\bu(\RR) \\
\mu_w & \in \PV^{0,\bu}(\CC^2_z) \otimes \PV^{1,\bu}(\CC^3_w) \otimes \Omega^\bu(\RR)  .
\end{align*}
and similarly $\gamma = \gamma_z + \gamma_w$. 
We will also use the notation $\del^z$ for the holomorphic de Rham differential along $\CC_z^2$ and similarly $\del^w$ for the holomorphic de Rham differential along $\CC^3_w$. 

To twist, we expand near the background where the field $\gamma_z$ takes value $\gamma_{nm}$ as in \eqref{eqn:gammanm}. 
This will generate new kinetic and interacting terms. 
%which we can extract by inserting a formal parameter $\delta$ and expressing the action functional in terms of the deformed field $\Tilde{\gamma} = \gamma + \delta \gamma^{1,0}$.

There are two types of interactions in the original theory.
The first is
\begin{equation}\label{eqn:int1}
  \frac12 \int_{\CC^2 \times \CC^3 \times \RR} \frac{1}{1-\nu} \left(\del \gamma \vee \mu^2 \right) \wedge (\d^2 z \wedge \d^3 w)
\end{equation}
and the second is
\begin{equation} \label{eqn:int2}
  \frac16\int_{\CC^2 \times \CC^3 \times \RR} \gamma \partial \gamma \partial \gamma .
\end{equation}

%We can integrate Equation (\ref{eqn:int1}) by parts to put it in the form $\frac12 \int_{X \times Z \times L} \left[(\partial \gamma) \vee (\mu \wedge \mu) \right]$ where $\mu \wedge \mu$ is the wedge product of polvector fields.\brian{there might be some factors I'm being sloppy with here}
Expanding \eqref{eqn:int1} around the background where $\gamma$ takes value $\gamma_{nm}$, we obtain,
%\[
%  \frac12 \int \frac{1}{1-\nu} \left(\del \gamma \vee \mu^2 \right) \wedge (\omega_Z \wedge \Omega_W) + \frac{\delta}{2} \int \frac{1}{1-\nu} \left(\omega_Z \vee \mu^2 \right) \wedge (\omega_Z \wedge \Omega_W) .
%\]
%Here, we have used the equation of motion $\partial \gamma^{1,0} = \Omega_Z$.
%It will be convenient to further expand this into the components $\mu_W, \mu_Y, \gamma_W, \gamma_Y$:
\begin{multline}
 \int \frac{1}{1-\nu} \left(\frac12 \del^w \gamma_w \vee \mu_w^2  + \del^z \gamma_w \vee \mu_w \mu_z + \del^w \gamma_z \vee \mu_w\mu_z + \frac12 \del^z \gamma_z \vee \mu_z^2 \right) \wedge (\d^2 z \wedge \d^3 w) 
 \\
  + \frac{1}{2} \int \frac{1}{1-\nu} \left(\d^2 z \vee \mu_z^2 \right) \wedge (\d^2 z \wedge \d^3 w) .
  \label{eqn:delta1}
\end{multline}

We similarly expand (\ref{eqn:int2}),
%\[
%  \frac16 \int \gamma \partial \gamma \partial \gamma + \frac{\delta}{2} \int \left(\gamma \partial \gamma\right) \wedge \omega_Z .
%\]
%Notice that there are no $\delta^2$ terms since $\partial \gamma^{1,0} \partial \gamma^{1,0} = 0$.
%Again, we further expand this into holomorphic components along $X,Z$:
\beqn
\frac16 \int \left(\gamma_w \partial^z \gamma_w \partial^z \gamma_w +\gamma_w \partial^w \gamma_w \partial^z \gamma_z +  \gamma_w \partial^w \gamma_z \partial^w \gamma_z \right) + \frac{1}{2} \int \left(\gamma_w \partial^w \gamma_w \right) \wedge \d^2 z
\label{eqn:delta2}
\eeqn

The new terms in the non-minimally twisted linearized BRST differential arise from the quadratic terms in the action in Equations \eqref{eqn:delta1} and \eqref{eqn:delta2}:
\begin{equation}\label{eqn:newterms}
  \frac{1}{2} \int (\d^2 z \vee \mu_z^2) \wedge (\d^2 z \wedge \d^3 w) + \frac{1}{2} \int \left(\gamma_w \wedge \partial^w \gamma_w \right) \wedge \d^2 z .
\end{equation}
The non-minimally twisted linear BRST complex thus takes the form
\[
  \begin{tikzcd}
  & \PV^{1,\bu}_Z \hotimes \PV^{0,\bu}_W \ar[dr, "\div^z"] \ar[dashed, rounded corners, to path={ -- ([yshift=-2ex]\tikztostart.west) |- ([xshift=-1.5ex]\tikztotarget.west) -- (\tikztotarget)}, dddddr]\\
  & & \PV^{0,\bu}_Z \hotimes \PV^{0,\bu}_W \\
 & \PV^{0,\bu}_Z \hotimes \PV^{1,\bu}_W \ar[ur, "\div^w"'] & \\
\;_{\cong}  & & \Omega^{0,\bu}_Z \hotimes \Omega^{1,\bu}_W \ar[ul, dashed, bend left = 10, "\Omega^{-1}_W \partial^w"]\\
 & \Omega^{0,\bu}_Z \hotimes \Omega^{0,\bu}_W \ar[ur, "\partial^w"] \ar[dr,"\partial^z"'] \\
  & & \Omega^{1,\bu}_Z \hotimes \Omega^{0,\bu}_W
  %\ar[uuuuul, start anchor =  {[yshift = 0ex, xshift=0ex]}, end anchor = {[yshift=1.0ex, xshift=-5ex]}, bend left = 90, dotted] .
  \end{tikzcd}
\]
Here, we write $Z = \CC^2_z$ and $X = \CC^3_w$ for notational simplicity. 

Here, the dashed arrow along the outside of the diagram corresponds to the BV antibracket with the first term in (\ref{eqn:newterms}).
It is given by the isomorphism 
\[
\Omega^{1,\bu}_Z \hotimes \Omega^{0,\bu}_W \xto{\omega^{2,0}_Z \otimes \id} \PV^{1,\bu}_Z \hotimes \PV^{0,\bu}_W
\]
induced holomorphic symplectic form on $Z$. 
The other dashed arrow corresponds to the BV antibracket with the second term in (\ref{eqn:newterms}).
It is given by the composition
\[
\Omega^{0,\bu}_Z \hotimes \Omega^{1,\bu}_W \xto{\id \otimes \del^w} \Omega^{0,\bu}_Z \hotimes \Omega^{2,\bu}_W \xto{\id \otimes \Omega_W} \PV^{0,\bu}_Z \hotimes \PV^{1,\bu}_W
\]
given by applying the holomorphic de Rham operator along $X$ followed by contracting with the inverse holomorphic volume form along $X$. 
%\brian{introduce PCS}
%Explicitly, if $f,g$ are holomorphic functions on $\CC^2$ then
%\[
%\{f(z_1,z_2) , g(z_1,z_2)\}_{pb} = \partial_{z_i} f \partial_{z_j} g - \partial_{z_j} f \partial_{z_i} g .
%\]

We replace this linear BRST complex, up to quasi-isomorphism, with a smaller BRST complex. 
Consider the complex
\beqn
\Omega^{0,\bu}_Z \hotimes \Omega^{\bu,\bu}_X \hotimes \Omega^\bu_L = \oplus_{k =0}^3 \Omega^{0,\bu}_Z \hotimes \Omega^{k,\bu}_X \hotimes \Omega^\bu_L 
%[-k] 
\eeqn
which is equipped with the differential $\dbar^z + \dbar^w + \del^w + \d_{\RR}$. 
Write $\alpha = \alpha^0 + \cdots + \alpha^3$ for a field in this complex, using the decomposition on the right hand side. 

There is a map of linear BRST complexes from this one to the original one defined by the following equations 
\begin{multline}
\mu_z = (\del_{z_1} \wedge \del_{z_2}) \vee \del^z \alpha^0, \quad \mu_w = (\del_{w_1} \wedge \del_{w_2} \wedge \del_{w_3}) \vee \alpha^2, \quad \nu = \til{\alpha}^3 \\
\beta = \alpha^0 , \quad \gamma_w = \alpha^1 , \quad \gamma_z = 0 .
\label{eqn:g2map}
\end{multline}
In the above equation we have introduced the notation $\til{\alpha}^3 = \Omega_X^{-1} \vee \alpha^3$. 

The restriction of the kinetic terms $\int \gamma (\dbar + \d_{\RR}) \mu + \beta (\dbar + \d_{\RR}) \nu$ along \eqref{eqn:g2map} is
\beqn\label{eqn:kin1}
\int \sum_{k=0}^3 \alpha^k (\dbar + \d_{\RR}) \alpha^{3-k} 
\eeqn
The restriction of the kinetic term $\int \beta \div \mu$ along \eqref{eqn:g2map} is
\beqn\label{eqn:kin2}
\int \alpha^0 \del^w \alpha^2 . 
\eeqn
Finally, the restriction of the kinetic term $\gamma \del^w \gamma$ along \eqref{eqn:g2map} is 
\beqn\label{eqn:kin3} 
\int \frac12 \alpha^1 \del^w \alpha^1 . 
\eeqn
Together, \eqref{eqn:kin1}--\eqref{eqn:kin3} give the kinetic term in Poisson Chern--Simons theory. 

This shows that \eqref{eqn:g2map} is a map of linear BRST complexes.
Applying the obvious contracting homotopy, we see that this map is a quasi-isomorphism.
We will show that the full non-linear map intertwines the action functionals up to cohomologically exact terms, and hence defines an equivalence of BV theories.

%
%Restricting the action \eqref{eqn:delta1} along the map \eqref{eqn:g2map} we obtain
%\begin{multline}
%\frac12 \alpha^1 (\alpha^2)^2 \del^X \til{\alpha}^3 + \frac12 \del^X \alpha^1 (\alpha^2)^2 
%\end{multline}
%
%\begin{multline}
%\int \frac{\omega_Z}{1-\Omega^{-1}_X \alpha^3} \left(\frac12 \del^X \alpha^1  [ \alpha^2 \vee (\alpha^2 \vee \Omega_X^{-1})] + \alpha^2 \del^Z \alpha^1 \del^Z \alpha^0\right) \\
%+ \frac16 \int \frac{\omega_Z}{1-\Omega^{-1}_X \alpha^3} \alpha^1 \partial^Z \alpha^1 \partial^Z \alpha^1 \\ + \frac{\delta}{2} \int \frac{\omega_Z}{1-\Omega^{-1}_X \alpha^3} \alpha^3 \partial^Z \alpha^0 \partial^Z \alpha^0 + \frac{\delta}{2} \int \frac{\omega_Z}{1-\Omega^{-1}_X \alpha^3} \partial^X \alpha^1 (\alpha^1 \Omega^{-1}_X \alpha^3) .
%\end{multline}
%
%More invariantly, we can write the total BV action as
%\[
%\Tilde{I} = I_{CS} + I' 
%\]
%where $I_{CS} = \frac16 \int \alpha\{\alpha,\alpha\}$ and 
%\beqn
%I' = \frac16 \int \frac{\omega_Z}{1-\Omega^{-1}_X \alpha} \partial^X \alpha [\alpha \vee (\alpha \vee \Omega_X^{-1})] + \frac16 \int \frac{\omega_Z}{1-\Omega^{-1}_X \alpha} [\alpha \vee (\alpha \vee \Omega^{-1}_Z)] \{\alpha,\alpha\} .
%\label{eqn:Iprime}
%\eeqn
%
%Consider the odd functional
%\[
%K = \frac16 \int \frac{\omega_Z}{1-\Omega^{-1}_X \alpha} \alpha \wedge [\alpha \vee (\alpha \vee \Omega_X^{-1})] .
%\]
%
%\begin{lem}
%$Q K + \{I_{CS}, K\} = I'$ .
%\end{lem}
%
%The linear differential applied to $K$ is 
%\[
%\frac16 \int \frac{\omega_Z}{1-\Omega^{-1}_X \alpha} \partial^X \alpha \wedge [\alpha \vee (\alpha \vee \Omega_X^{-1})] \omega_Z .
%\]
%This is the first term in \eqref{eqn:Iprime}. 
%
%Next, we compute
%\[
%\{I_{CS}, K\} = \frac16 \int \frac{\omega_Z}{1-\Omega^{-1}_X \alpha} [\alpha \vee (\alpha \vee \Omega^{-1}_X)] \{\alpha,\alpha\} 
%\]
%which is the second term in \eqref{eqn:Iprime}. 


\def\im{{\rm i}}

\section{Dimensional reduction and 10-dimensional supergravity}
\label{sec:dimred}

In this section we demonstrate that our proposal for the action of minimally twisted 11-dimensional supergravity agrees with conjectural descriptions of twisted type IIA and type I supergravities due to Costello and Li. 

The original motivation for $M$-theory was as the strong coupling limit for type IIA string theory.
Roughly, the radius of the $M$-theory circle plays the role of this coupling constant. 
Additionally, at low energies $M$-theory is expected to be approximated by 11-dimensional supergravity in the same way that the low energy limit of type IIA/IIB string theory is type IIA/IIB supergravity. 
Combining these two pictures, various checks have been made that the dimensional reduction of 11-dimensional supergravity along the $M$-theory circle is type IIA supergravity. 

Motivated by the topological string, Costello and Li have laid out a series of conjectures for twists of type IIA/IIB supergravity \cite{CLsugra} and type I supergravity \cite{CLtypeI}. 
Their description was inspired by the model of the open and closed $B$-model topological string on a Calabi--Yau manifold. 
The open sector is holomorphic Chern--Simons theory \cite{WittenOpen} and the closed sector is called Kodaira--Spencer theory \cite{BCOV}. 
There are a few different versions of Kodaira--Spencer theory, but the shared characteristic is that they are all `gravitational' in nature; they describe fluctuations of the Calabi--Yau structure. 
From this point of view, Kodaira--Spencer theory is at the heart of the formulation of the various flavors of twisted 10-dimensional supergravity.

We begin by introducing certain variants of Kodaira--Spencer theory which will feature in the descriptions of twists of type IIA and type I supergravity.

\subsection{Kodaira--Spencer theory}
\label{s:BCOV}

Let $X$ be a Calabi--Yau manifold; for now it can be of arbitrary complex dimension $d$. 
Define
\deq{
  \PV^{i,j}(X) = \Omega^{0,j}(X, \wedge^i \T_X).
}
We will consider the graded space $\PV^{\bu,\bu}(X) = \oplus_{i,j} \PV^{i,j}(X)[-i-j]$ where the piece of type $(i,j)$ sits in degree $i+j$. 

For each fixed $i$, while we let $j$ vary, the $\dbar$ operator defines a cochain complex $\PV^{i,\bu}(X) = (\oplus_j\PV^{i,j}(X) [-j], \dbar)$ which provides a resolution for the sheaf of holomorphic polyvector fields of type $i$. 
The divergence operator extends to an operator of the form
\[
\div \colon \PV^{i,\bu}(X) \to \PV^{i-1,\bu}(X) .
\]

Motivated by the states of the topological $B$-model, one defines the fields of Kodaira--Spencer gravity on $X$ to be the cochain complex
\beqn\label{eqn:ks1}
\left(\PV^{\bu,\bu} (X)[[u]] [2] \, , \, \dbar + u \div\right) .
\eeqn 
Here, $u$ is a parameter of cohomological degree $+2$, which turns $\delta_{KS}^{(1)} = \dbar + u \div$ into an operator of homogenous degree $+1$. 
We also have performed an overall cohomological shift by $2$ so that $u^k \PV^{i,j}$ sits in degree $i+j+2k-2$. 
More precisely, this is a model for the $S^1$-equivariant cohomology of the states of the $B$-model on a closed disk. 
We refer to \cite{CLtypeI, CLsugra} for detailed justification for this ansatz. 

\parsec[s:poisson]
The original action for Kodaira-Spencer theory posited by \cite{BCOV} has a nonlocal kinetic term. In the BV formalism, this is codified by stipulating that the BV pairing is a degenerate odd Poisson tensor rather than an odd symplectic form. 
The Poisson kernel is given by the expression 
\[
(\div\otimes 1)\delta_{\Delta \subset X\times X} \in \left[\PV^{\bu,\bu}(X)\right]^{\hotimes 2} ,
\]
see \cite[{\S 1.4}]{CLbcov1}. 
Here, we view the $\delta$-distribution as a polyvector field using the Calabi--Yau form. 
Notice that the shifted Poisson tensor does not involve the parameter $u$ at all. 
For this reason, only the duals of a small number of fields pair nontrivially under the resulting odd BV bracket. 
%Explicitly, the above kernel pairs against the dual to 
%\[
%\Sigma^1 = \sum u^k\mu^1_k, \Sigma^2 = \sum u^k\mu^2_k\in \PV^{\bu,\bu}%(X)[[u]][2]
%\]
% by the formula 
%\[
%\int \overline{\mu}_0^1\div \overline{\mu}_0^2.
%\] 
%Here, the overlines denote the dual field in 
%\[
%(\PV^{\bu,\bu}(X)[[u]][2])^*=\PV_c^{d-\bu,d-\bu}(X)[[u]][2].
%\] 
%That is, further expanding the fields as 
%\[
%\mu_k = \sum_{i,j=1}^d\mu_k^{(i,j)}\in \PV^{i,j}
%\] 
%the expression pairs $\overline{\mu}_0^{(i,j)}$ with $\overline{\mu}_0^{(d-1-i, d-j)}$.

\parsec[s:ksaction] 

There is a natural local interaction which equips the complex \eqref{eqn:ks1} with the structure of $\Z/2$ graded Poisson BV theory. Explicitly, it is given by 
\beqn
I_{BCOV}(\Sigma) = {\rm Tr}_X \, \langle \exp \Sigma\rangle_0 = \sum_{n\geq 0} {\rm Tr}_X \, \langle\Sigma^{\otimes n}\rangle_0
\eeqn
where ${\rm Tr}_X \, \Phi = \int_X (\Phi \vee \Omega) \wedge \Omega$ and where $\langle - \rangle_0$ denotes the genus zero Gromov-Witten invariant with marked points
\beqn
\langle u^{k_1}\mu_1 \otimes \cdots \otimes u^{k_m}\mu_m\rangle_0 := \left (\int _{\overline {\cM}_{0,m}} \psi_1^{k_1}\cdots \psi_m^{k_m}\right ) \mu_1\cdots \mu_m = \binom{m-3}{ k_1,\cdots, k_m}  \mu_1\cdots \mu_m.
\eeqn

This interaction is extremely natural from the point of view of string field theory. Indeed, the B-model localizes to the space of constant maps into $X$, which factors as a product of $\overline{\cM}_{0,m}\times X$. This is in keeping with finding an interaction that factors as an integral over $X$ times an integral over $\overline{\cM}_{0,m}$. 

In \cite{BCOV} the authors show that the above interaction satisfies the classical master equation. Moreover, they show that the $L_\infty$ structure determined by the above action is equivalent to a natural dgla structure on the complex of fields with Lie bracket given by the Schouten bracket. Explicitly, the equivalence is given by the transcendental automorphism 
\[
\Sigma \mapsto [u(\exp (\Sigma/u)-1)]_+
\]
where $[-]_+$ denotes projection onto positive powers of $u$.

\parsec[s:minimalks]

We pointed out in \S\ref{s:poisson} that the majority of fields pair to zero under the Poisson tensor. Physically these correspond to closed string fields that do not propogate. In the supergravity approximation, the fields that survive are those closed string fields that propogate. In terms of our description of closed string field theory in terms of Kodaira-Spencer theory, this motivates us to consider the smallest cochain complex containing those fields thathave nonzero pairing under the Poisson tensor. This is referred to as minimal Kodaira-Spencer theory.

The fields of minimal Kodaira-Spencer theory are given by the subcomplex of \label{eqn:ks1}
\beqn
\left (\bigoplus_{i+j\leq d -1}u^i\PV^{j,\bu}(X)[2], \dbar + u \div\right).
\eeqn
We observe that the original odd Poisson tensor lives in this subcomplex. 
%here is a natural odd Poisson tensor on the above cochain complex such that the inclusion of this subcomplex into the fields of Kodaira-Spencer theory is a Poisson map. 
There is a natural action functional given by restricting $I_{BCOV}$ to this space.

\subsection{The $SU(4)$ twist of type IIA supergravity}\label{sec:SU(4)twist}

We recall the description of the $SU(4)$ twist of type IIA supergravity conjectured in \cite{CLsugra}. 
In principal, there is also a minimal, $SU(5)$ invariant, twist of type IIA supergravity but so far no description, even conjecturally, exists.
We turn to this in \S \ref{s:su5IIA}. 

Let $X$ be a Calabi-Yau manifold of complex dimension four. 
The $\ZZ/2$ graded complex of fields of minimal Kodaira--Spencer theory on $X$ takes the form
\beqn
\begin{tikzcd}
- & + & - & + \\ \hline
                        &                          &                                     & {\PV^{0,\bu}}    \\
                        &                          & {\PV^{1,\bu}} \arrow[r, "u\div"]    & {u\PV^{0,\bu}}   \\
                        & {\PV^{2,\bu}} \arrow[r, "u\div"]  & {u\PV^{1,\bu}} \arrow[r, "u\div"]   & {u^2\PV^{0,\bu}} \\
{\PV^{3,\bu}} \arrow[r, "u\div"] & {u\PV^{2,\bu}} \arrow[r, "u\div"] & {u^2\PV^{1,\bu}} \arrow[r, "u\div"] & {u^3\PV^{0,\bu}}
\end{tikzcd}.
\eeqn
Denote this complex by $\cE_{mKS}(X)$. 
Here, $u^\ell\PV^{k,i}$ is placed in parity $k + i -1 \mod 2$. 
%Denote this complex by $\cE_{mKS}(X)$. 
The classical BCOV action $I_{BCOV}$ follows from the general formula we gave above. 

With this in hand the conjecture of \cite{CLsugra} takes the following form.

\begin{conj}
The $SU(4)$-invariant twist of type IIA supergravity on $\RR^2\times \CC^4$ is the $\Z/2$-graded Poisson BV theory with fields 
\beqn\label{eqn:IIAfields}
\alpha = \sum_n \alpha_n u^n \in \cE_{mKS}(\CC^4) \otimes \Omega^{\bu}(\RR^2).
\eeqn
The classical interaction takes the form \[I_{IIA} = \int_{\CC^4 \times \RR^2} \alpha_0^3 + \cdots\]
%The $L_{\infty}$ structure is the natural one on the tensor product of the cdga $\Omega^{\bu}(M)$ with the $L_{\infty}$-algebra $\cE_{mKS}$.
\end{conj}

We will need a more detailed description of the classical action. 
For the moment, let us introduce some notations for the fields of this IIA model, as always we leave the internal Dolbeault degree implicit:
\begin{multline}
\eta \in \PV^{0,\bu}(\CC^4) \otimes \Omega^\bu (\RR^2), \quad \mu + u \nu \in \PV^{1, \bu}(\CC^4) \otimes \Omega^\bu (\RR^2) \oplus u \PV^{0,\bu} (\CC^4) \otimes \Omega^\bu (\RR^2) \\
\Pi \in \PV^{3,\bu}(\CC^4) \otimes \Omega^\bu(\RR^2), \quad \sigma \in \PV^{3,\bu}(\CC^4) \otimes \Omega^\bu (\RR^2) .
\end{multline}
We will not need an explicit notation for the remaining descendant fields. 

With this notation in hand, we have the more precise form of the action appearing in the conjecture:
\beqn\label{eqn:IIAaction}
I_{IIA} = \frac12 {\rm Tr}_{\CC^4 \times \RR^2} \frac{1}{1-\nu} \mu^2 \wedge \Pi + {\rm Tr}_{\CC^4 \times \RR^2} \frac{1}{1-\nu} \eta \wedge \mu \wedge \sigma + \frac12 {\rm Tr}_{\CC^4 \times \RR^2} \frac{1}{1-\nu} \eta \wedge \Pi^2 + \cdots 
\eeqn
where the $\cdots$ denote terms involving higher order descendants. 

\subsection{Reduction to IIA supergravity}
\label{s:su4red}

We now turn back to our 11-dimensional theory. 
The first goal is to compare the dimensional reduction of our 11-dimensional theory on $\CC^5 \times \RR$
%where $X$ is a Calabi-Yau 4-fold 
with the $SU(4)$ invariant twist of type IIA on $\R^{2}\times \CC^4$. 
Doing so will require a slight modification to the description of the $SU(4)$ twist of IIA supergravity recollected in \S \ref{sec:SU(4)twist}. 

\parsec[sec:IIApot]

Recall that in the physical theory, the components of the $C$-field in 11d that are not supported along the M-theory circle become the components of the Ramond--Ramond 2-form of type IIA. However, as noted in \cite{CLsugra} components of Ramond--Ramond fields do not appear as fields in Kodaira--Spencer theory; rather it is components of their field strengths that appear. 
We recalled in \S \ref{s:components} that components of the $C$-field become components of $\gamma_{11d}$ in $\cE$.
This suggests that we must modify our description of the twist of type IIA to include potentials for certain fields.

The fundamental fields of the $SU(4)$ twist of IIA supergravity were given in \eqref{eqn:IIAfields}. 
We modify the space of fields by introducing potentials for both the $\Pi$ and $\sigma$ fields. 
First, we introduce a field $\gamma \in \Omega^{1,\bu}(\CC^4) \otimes \Omega^\bu(\RR^2)$ (not to be confused, yet, with the $\gamma$ field in our 11-dimensional theory) which satisfies $\Pi \vee \Omega = \del \gamma$ where $\Omega$ is the Calabi--Yau form on $\CC^4$. 
This condition does not uniquely fix $\gamma$. 
There is a new linear gauge symmetry determined by $\gamma \to \gamma + \div \beta$ where $\beta$ is a ghost that we must also introduce. 
Similarly, we introduce a field $\theta \in \Omega^{0,\bu}(\CC^4) \otimes \Omega^\bu(\RR^2)$ which satisfies $\sigma \vee \Omega = \del \theta$, there is no extra gauge symmetry present in this condition.\footnote{Using the Calabi--Yau form we have normalized the potential fields $\gamma, \beta,\theta$ to be written as differential forms instead of polyvector fields.}

In diagrammatic detail, the potential theory we are considering has underlying cochain complex of fields
\beqn\label{eqn:IIApot}
\begin{tikzcd}
- & + \\ \hline
& {\PV^{0,\bu} (\CC^4) \otimes \Omega^\bu (\RR^2) }_\eta  \\
{\PV^{1,\bu} (\CC^4) \otimes \Omega^\bu (\RR^2)}_\mu \arrow[r, "u\div"] & u{\PV^{0,\bu} (\CC^4) \otimes \Omega^\bu (\RR^2)}_\nu \\
u^{-1}{\Omega^{0,\bu} (\CC^4) \otimes \Omega^\bu (\RR^2)}_\beta \arrow[r, "u\del"] & {\Omega^{1,\bu} (\CC^4) \otimes \Omega^\bu (\RR^2)}_\gamma  \\
{\Omega^{0,\bu} (\CC^4) \otimes \Omega^\bu (\RR^2)}_\theta &
\end{tikzcd}
\eeqn.

The original space of fields of the twist of IIA supergravity on $\CC^4 \times \RR^2$ was equipped with an odd Poisson bivector which was degenerate.
In other words, it did not define a theory in the conventional BV formalism. 
One of the key features of this new complex of fields, after we have taken these potentials, is that it is equipped with an odd non-degenerate pairing thus equipping it with the structure of a theory in the conventional BV formalism. 

The pairing is $\Res_u \frac{\d u}{u} \int^\Omega_{\CC^4 \times \RR^2} \alpha \vee \alpha'$ where $\alpha, \alpha'$ are two general fields in this potential theory on $\CC^4 \times \RR^2$. 
Explicitly, in the description of the fields in \eqref{eqn:IIApot} the pairing is 
\[
\int^\Omega_{\CC^4 \times \RR^2} \eta \theta + \int^\Omega_{\CC^4 \times \RR^2} \mu \vee \gamma + \int^\Omega_{\CC^4 \times \RR^2} \nu \beta .
\]
This pairing is compatible with the odd Poisson bracket present in the original theory on $\CC^4 \times \RR^2$.

The type IIA action completely determines the action of this theory with potentials. 
One simply takes the \eqref{eqn:IIAaction} and replaces all appearances of $\Pi$ with $\div \gamma$ and all appearances of $\sigma$ with $\div \theta$. 
This yields the interaction of the potential theory
\beqn\label{eqn:IIAactionpot}
\til I_{IIA} = \frac12 \int^\Omega_{\CC^4 \times \RR^2} \frac{1}{1-\nu} \mu^2 \vee \del \gamma + \int^\Omega_{\CC^4 \times \RR^2} \frac{1}{1-\nu} (\eta \wedge \mu) \vee \del \theta + \frac12 \int_{\CC^4 \times \RR^2} \frac{1}{1-\nu} \eta \wedge \del \gamma \wedge \del \gamma 
\eeqn
Notice that the terms involving higher descendants vanishes since these fields are set to zero in the potential theory.

%Note that there is a natural map of cochain complexes $\partial \cE_{pot}\to \cE_{mKS}$ given by the dotted arrows below:
%\beqn
%\begin{tikzcd}
%                                       & {\PV^{0,\bu}} \arrow[rrr, dotted, "\id"] &                          &                                     & {\PV^{0,\bu}}    \\
%{\PV^{1,\bu}} \arrow[r, "u\div"]       & {u\PV^{0,\bu}} \arrow[rr, dotted, "\id"] &                          & {\PV^{1,\bu}} \arrow[r, "u\div"]    & {u\PV^{0,\bu}}   \\
%{u^{-1}\PV^{4,\bu}} \arrow[r, "u\div"] & {\PV^{3,\bu}} \arrow[r, dotted, "\div"]  & {\PV^{2,\bu}} \arrow[r]  & {u\PV^{1,\bu}} \arrow[r, "u\div"]   & {u^2\PV^{0,\bu}} \\
%{\PV^{4,\bu}} \arrow[r, dotted, "\div"]        & {\PV^{3,\bu}} \arrow[r]          & {u\PV^{2,\bu}} \arrow[r] & {u^2\PV^{1,\bu}} \arrow[r, "u\div"] & {u^3\PV^{0,\bu}}
%\end{tikzcd}.
%\eeqn
%This is easily seen to be a Poisson map. We have that $I_{pot} = \partial^{*}I_{BCOV}$, so we see that $I_{pot}$ satisfies the classical master equation. Therefore, $I_{pot}$ determines an $L_{\infty}$ structure on $\cE_{pot}$, and hence one on $\Omega^{\bu}(\R^{2})\otimes \cE_{pot}$. In what follows, the $SU(4)$-invariant twist of IIA will refer to the BV theory $\Omega^{\bu}(\R^{2})\otimes \cE_{pot}$.

\parsec[-]

We turn to the proof of the main result of this section that the dimensional reduction of our 11-dimensional theory agrees with the twist of IIA supergravity just introduced. 

We recall the notion of dimensional along a holomorphic direction following \cite{ESW}. 
Suppose that $V_\RR$ is a real vector space and denote by $V$ its complexification. 
We consider a field theory defined on $M \times V$, which is holomorphic along $V$ (in particular, this means that the theory is translation invariant along $V$).  
We consider the dimensional reduction along the projection 
\beqn\label{eqn:dimred}
M \times V \to M \times V_\RR
\eeqn
induced by ${\rm Re} \colon V \to V_\RR$.
Most relevant for us is the case when $V = \CC$ and $M$ is $\CC^4 \times \RR$, but the explicit form of the theory along $M$ is not important at the moment.

For illustrative purposes, let us first assume that $M$ is a point and that the space of fields is of the form $\Omega^{0,\bu}(V) \otimes W$ for $W$ some graded vector space. 
As properly formulated in \cite{ESW}, it is shown that the dimensional reduction along $V \to V_\RR$ is equivalent to the theory whose fields are $\Omega^\bu(V_\RR) \otimes W$. 
In other words, the dimensional reduction of the holomorphic theory on $V$ is a topological theory on $V_\RR$. 

If we put $M$ back in, the result is similar. 
Suppose the original theory is of the form $\cE(M) \otimes \Omega^{0,\bu}(V) \otimes W$.
Then, the dimensional reduction along \eqref{eqn:dimred} is the theory whose space of fields is $\cE(M) \otimes \Omega^\bu(V_\RR) \otimes W$.

An explicit model for this reduction can be described as follows. 
Suppose $V \cong \CC^n$ and place the theory on $(\CC^\times)^{\times n} \subset \CC^n$. 
The dimensional reduction along $\CC^n \to \RR^n$ agrees with the compactification of the theory along $S^1 \times \cdots \times S^1$ where one throws away all nonzero winding modes around each circle.

\begin{prop}\label{prop:dimred}
The $SU(4)$ invariant twist of type IIA on $\CC^4 \times \RR^2$ is the dimensional reduction of the 11-dimensional theory along  
\[
\CC^4 \times \CC \times \RR_t \to \CC^4 \times \RR_x \times \RR_t \cong \CC^4 \times \RR^2 .
\]
\end{prop}
\begin{proof}
Let us denote the holomorphic coordinate we are reducing along by $z_5 = x + \im y$. 
We first read off the dimensional reduction of each component field of the 11-dimensional theory. 
Per the above discussion, this is obtained by taking all fields to be independent of $y$ and replacing $\d \zbar_5$ by $\d x$. 
To not confuse the notations of fields in 10 and 11 dimensions, we use the notation $\alpha_{11d}$ to denote an 11-dimensional field.

The reductions of the 11d fields $\nu_{11d}, \beta_{11d}$ are easy to describe. 
Recall that 
\[
\nu_{11d} \in \PV^{0,\bu}(\CC^5) \otimes \Omega^\bu(\RR) .
\]
The reduction of this field is a 10d $\nu$ field
\[
\nu (z_i,x,t) = \nu_{11d} (z_i, x, y=0, t) |_{\d \zbar_5 = \d x}  .
\]
Similarly, the reduction of $\beta_{11d}$ is a 10d $\beta$ field
\[
\beta (z_i,x,t) = \beta_{11d} (z_i, x, y=0, t) |_{\d \zbar_5 = \d x}  .
\]

The reduction of the 11d fields $\mu_{11d}$ and $\gamma_{11d}$ require a bit of massaging. 
We break the $SU(5)$ symmetry to $SU(4)$ to write
\[
\mu_{11d} = \mu^0_{11d} + \theta_{11d} \partial_{z_5} 
\]
where
\begin{align*}
\mu^0_{11d} & \in \PV^{1,\bu}(\CC^4) \otimes \Omega^{0,\bu}(\CC_{z_5}) \otimes \Omega^\bu(\RR_t) \\
\theta_{11d} & \in \Omega^{0,\bu}(\CC^4) \otimes \Omega^{0,\bu}(\CC_{z_5}) \otimes \Omega^\bu(\RR_t) .
\end{align*}
The dimensional reduction of $\mu^0_{11d}$ is a 10d $\mu$ field
\[
\mu(z_i,x,t) = \mu_{11d}^0 (z_i, x,y=0,t)|_{\d \zbar_5 = \d x} .
\]
The dimensional reduction of $\theta_{11d}$ is a $\theta$ field
\[
\theta(z_i,x,t) = \theta_{11d} (z_i, x,y=0,t)|_{\d \zbar_5 = \d x} .
\]

Finally, write the 11d field $\gamma_{11d}$ as
\[
\gamma_{11d} = \gamma_{11d}^0 + \eta_{11d} \d z_5
\]
where
\begin{align*}
\gamma^0_{11d} & \in \Omega^{1,\bu}(\CC^4) \otimes \Omega^{0,\bu}(\CC_{z_5}) \otimes \Omega^\bu(\RR_t) \\
\eta_{11d} & \in \PV^{0,\bu}(\CC^4) \otimes \Omega^{0,\bu}(\CC_{z_5}) \otimes \Omega^\bu(\RR_t) .
\end{align*}
The dimensional reduction of $\gamma^0_{11d}$ is a 10d $\gamma$ field
\[
\gamma(z_i,x,t) = \gamma_{11d}^0 (z_i, x,y=0,t)|_{\d \zbar_5 = \d x} .
\]
The dimensional reduction of $\eta_{11d}$ is an $\eta$ field
\[
\eta(z_i,x,t) = \eta_{11d} (z_i, x,y=0,t)|_{\d \zbar_5 = \d x} .
\]

Next, we read off the dimensional reduction of the 11d action. 
Let us first focus on the term present in BF theory which is
$\int^\Omega \frac{1}{1-\nu_{11d}} \mu_{11d}^2 \vee \del \gamma_{11d}$.
Upon reduction, this becomes 
\beqn\label{eqn:bfred}
\int^{\Omega_{\CC^4}}_{\CC^4 \times \RR^2} \frac{1}{1-\nu} \mu^2 \vee \del \gamma + \int^{\Omega_{\CC^4}}_{\CC^4 \times \RR^2} \frac{1}{1-\nu} (\theta \wedge \mu) \vee  \del \eta 
\eeqn

Next, consider the cubic term in the 11d action $J = \frac16 \int \gamma_{11d} \wedge \del \gamma_{11d} \wedge \del \gamma_{11d}$. 
Upon reduction, this becomes 
\beqn\label{eqn:jred}
\int_{\CC^4 \times \RR^2} \eta \wedge \del \gamma \wedge \del \gamma .
\eeqn

The sum of the action functionals \eqref{eqn:bfred} and \eqref{eqn:jred} does not precisely agree with the IIA action $\til I_{IIA}$. 
To relate the two actions we must make the following field redefinition:
\[
\til \theta = \frac{1}{1-\nu} \theta, \quad \til \eta = (1- \nu) \eta, \quad \til \beta = \beta + \frac{1}{1-\nu} \eta \wedge \theta .
\]
Notice that this change of coordinates is compatible with the odd symplectic pairing on the fields. 
Under this field redefinition the total dimensionally reduced action can be written as
\begin{multline}
\int^{\Omega_{\CC^4}}_{\CC^4 \times \RR^2} \frac{1}{1-\nu} \mu^2 \vee \del \gamma + \in^{\Omega_{\CC^4}}t_{\CC^4 \times \RR^2} \frac{1}{1-\nu} \til\eta \wedge \del \gamma \wedge \del \gamma + \int^{\Omega_{\CC^4}}_{\CC^4 \times \RR^2} (\til\theta \wedge \mu) \vee  \del \left(\frac{1}{1-\nu} \til\eta\right) \\ + \int_{\CC^4 \times \RR^2}^{\Omega_{\CC^4}} \frac{1}{1-\nu} (\til \eta \wedge \til \theta) \div \mu  .
\end{multline}
The first line comes from plugging in the new fields into the interactions \eqref{eqn:bfred} and \eqref{eqn:jred}.
The second line comes from plugging in the new fields into the kinetic term $\int \beta \div \mu$, which because of the non-linear change of coordinates now contributes to the interaction. 
We observe that the first two terms agree with the first and third terms in \eqref{eqn:IIAactionpot}. 

After integrating by parts, the remaining terms can be written as 
\[
- \int^{\Omega_{\CC^4}}_{\CC^4 \times \RR^2} \left(\frac{1}{1-\nu} \til\eta\right) \div (\til\theta \mu) + \int_{\CC^4 \times \RR^2}^{\Omega_{\CC^4}} \left(\frac{1}{1-\nu} \til \eta\right) \til \theta \div \mu .
\]
Applying the identity $\div (\til \theta \mu) = \til \theta \div \mu + \del (\til \theta) \vee \mu$, we see that this agrees exactly with the second term in \eqref{eqn:IIAactionpot}.
% but they are still not quite the same. 
%They are, however, cohomologous. 
%Integrating by parts and applying the BV relation, we can write the second line as
%\[
%\int^{\Omega_{\CC^4}}_{\CC^4 \times \RR^2} \left(\frac{1}{1-\nu} \til\eta\right)\wedge \div (\til\theta \wedge \mu)  = \int^{\Omega_{\CC^4}}_{\CC^4 \times \RR^2} \frac{1}{1-\nu} (\til\eta \wedge \mu) \vee \partial \til \theta + \int^{\Omega_{\CC^4}}_{\CC^4 \times \RR^2} \frac{1}{1-\nu} \til \eta \wedge \div \mu \wedge \til \theta .
%\]
%The first term in this equation agrees precisely with the second term in \eqref{eqn:IIAactionpot}. 
%The final term is cohomologically trivial via the odd Lagrangian $\int^\Omega \log(1-\nu) \wedge \til \eta \wedge \til \theta$.
%
%Let $\pi : \R\times \C^{\times}\times X \to \R^{2}\times X$ be the projection with fiber $S^{1}\subset \C^{\times}$. Note that there is an isomorphism \[\int: \pi^{*}(\Omega^{\bullet}(\R^{2})\otimes \cE_{pot})\to \cE\] given by \[(\eta, \mu, \nu, \beta,\gamma,\theta)\mapsto (\nu = \nu, \mu = \mu + \theta \vee \Omega_{X} \wedge\del_{z}, \beta = \beta, \gamma = \gamma\vee \Omega_{X} \eta dz ).\] It is clear that this isomorphism presrves the BV pairings; we need only check that $\int^{*} I$ is cohomologous to $I_{pot}$ in the deformation-obstruction complex of the free limit of $\Omega^{\bu}(\R^{2})\otimes \cE_{pot}$.
%
%We readily compute:
%\[\int^{*}I = \]
\end{proof}

\subsection{The twist of type I supergravity}

We now turn to a different type of redution of the 11-dimensional theory, this time involving type I supergravity. 
We begin by briefly recalling the description of type I supergravity following \cite{CLtypeI} which was motivated by the unoriented $B$-model. 
In \cite{SWspinor}, the second two authors verified the conjectural description of the space of fields recalled below using the pure spinor formalism. 
Unlike type IIA supergravity, there only exists an $SU(5)$ invariant twist of type I supergravity and it is holomorphic in the maximal number of dimensions.

Concretely, the space of fields of the $SU(5)$ twist of type I supergravity is a subspace of minimal Kodaira--Spencer theory on $\CC^5$. 
The $\ZZ/2$ graded space of field equipped with its linear BRST operator is 
\beqn\label{eqn:IIApot}
\begin{tikzcd}
- & + & - & +  \\ \hline
{\PV^{1,\bu}}(\CC^5) \arrow[r, "u\div"]    & {u\PV^{0,\bu}}(\CC^5) \\
{\PV^{3,\bu}} (\CC^5)\arrow[r, "u\div"] & {u\PV^{2,\bu}} (\CC^5)\arrow[r, "u\div"] & {u^2\PV^{1,\bu}}(\CC^5) \arrow[r, "u\div"] & {u^3\PV^{0,\bu}}(\CC^5)
\end{tikzcd}.
\eeqn

Let us give a description of the classical action. 
Introduce notations for the fields of this type I model:
\beqn\label{eqn:Ifields}
\mu + u \nu \in \PV^{1, \bu}(\CC^5) \oplus u \PV^{0,\bu} (\CC^5), \quad \sigma \in \PV^{3,\bu}(\CC^5) .
\eeqn
We will not need an explicit notation for the remaining descendant fields. 

\begin{conj}
The twist of type I supergravity on $\CC^5$ is the $\Z/2$-graded theory with fields $\mu+u\nu, \sigma$ as above and with classical action
\beqn\label{eqn:typeIaction}
I_{{\rm type\, I}} = {\rm Tr}_{\CC^5} \frac{1}{1-\nu} \mu^2 \vee \sigma + \cdots
\eeqn
where the $\cdots$ stands for terms involving the higher descendant fields. 
%The $L_{\infty}$ structure is the natural one on the tensor product of the cdga $\Omega^{\bu}(M)$ with the $L_{\infty}$-algebra $\cE_{mKS}$.
\end{conj}

\parsec[s:typeIpot]

Like in the type IIA discussion, there is a slight modification of the type I model above which is most directly related to 11-dimensional supergravity. 

This modification involves replacing the field $\sigma$ above by a potential $\til \gamma \in \Omega^{1,\bu}(\CC^5)$ which satisfies $\Omega \vee \sigma = \del \til \gamma$. 
This condition does not fix $\til \gamma$ uniquely, there is a gauge symmetry of the form $\til \gamma \to \til \gamma + \del \til \beta$. 

In detail, this potential theory we are considering has underlying cochain complex of fields
\begin{equation}
  \label{eq:Ipot} 
  \begin{tikzcd}[row sep = 1 ex]
    - & + & -\\ \hline
    \PV^{1,\bu}(\CC^5)_\mu \ar[r, "\div"] & \PV^{0,\bu}(\CC^5)_\nu  \\
         & \Omega^{0,\bu}(\CC^5)_{\til\beta} \ar[r, "\del"] & \Omega^{1,\bu}(\CC^5)_{\til\gamma} .
\end{tikzcd}
\end{equation} 
This space of fields is equipped with an odd non-degenerate pairing.
Like the eleven-dimensional theory, it is a classical BV theory in the $\ZZ/2$-graded sense. 

The type I action \eqref{eqn:typeIaction} completely determines the action of this theory with potentials. 
One simply takes the action and replaces all appearances of $\sigma$ with $\Omega^{-1} \vee \del \til\gamma$. 
This yields the interaction of the potential theory
\beqn\label{eqn:Iactionpot}
\til I_{\text{type I}} = \frac12 \int^\Omega_{\CC^5} \frac{1}{1-\nu} \mu^2 \vee \del \til\gamma .
\eeqn
Notice that the terms involving higher descendants vanishes since these fields are set to zero in the potential theory.

\subsection{Slab compactification}\label{s:Ired}

We consider placing twisted 11-dimensional supergravity on the manifold $\CC^5 \times [0,1]$. 
In order to do this, we must choose appropriate boundary conditions at $t=0$ and $t=1$.
Our 11-dimensional theory on such manifolds fits nicely into the formalism of \cite{BY,Eugene} in that it is topological in the direction transverse to the boundary.  

The phase space of the theory at $t=0$ or $t=1$ is 
\begin{equation}
  \label{eq:lin1} 
  \begin{tikzcd}[row sep = 1 ex]
    - & + \\ \hline
    \PV^{1,\bu}(\CC^5)_\mu \ar[r, "\div"] & \PV^{0,\bu}(\CC^5)_\nu \\ 
     \Omega^{0,\bu}(\CC^5)_\beta \ar[r, "\del"] & \Omega^{1,\bu}(\CC^5)_\gamma.
\end{tikzcd}
\end{equation}
The wedge an integrate pairing between the top and bottom lines induces an {\em even} symplectic structure on the phase space. 
Denote this phase space by $\cE_{\del}$ for the moment.

The phase space is equipped with the restriction of the linear BRST operator of the full 11-dimensional theory. 
There is also a non linear BRST operator, just like in the bulk theory.
The BV action induces a $L_\infty$ structure on the parity shift $\Pi\cE_{\del}$ whose cohomology is still a trivial central extension of $E(5,10)$. 

A boundary condition is given by a Lagrangian subspace of $\cE_{\del}$ with respect to this even symplectic structure. 
To make sense of the theory on $\CC^5 \times [0,1]$ we must make the choice of two separate boundary conditions 
\[
\cM_{t=0} , \cM_{t=1} \subset \cE_{\del} .
\]
Moreover, these boundary conditions carry non linear BRST operators endowing their parity shifts $\Pi \cM_{t=0} , \Pi\cM_{t=1}$ with the structures of $L_\infty$ algebras. 
These $L_\infty$ structures must be compatible with the one on the phase space.
In fact, in our context these boundary conditions are abstractly isomorphic. 
We will explain the explicit boundary conditions momentarily. 

An important thing to note is that the fields of the theory compactified on the slab is computed by the {\em derived} intersection of the two Lagrangians:
\[
\cM_{t=0} \overset{\LL}{\underset{\cE_{\del}}{\times}}\cM_{t=1}.
\]
To compute this derived intersection we must suitably resolve the boundary conditions.


\parsec[s:boundary]
At $t=0$, the boundary condition of the 11-dimensional theory is determined by declaring 
\[
\cM_{t=0}: \quad \gamma|_{t=0} = \beta|_{t=0} = 0 .
\]
We will place the theory on $\CC^5 \times [0,1]$ by imposing the same boundary condition at $t=1$:
\[
\cM_{t=1}: \quad \gamma|_{t=1} = \beta|_{t=1} = 0 .
\]

\begin{prop}
With these boundary conditions for the classical 11-dimensional theory on $\CC^5 \times [0,1]$, the dimensional reduction along 
\[
\CC^5 \times [0,1] \to \CC^5
\]
is equivalent to the twist of type I supergravity on $\CC^5$. 
\end{prop}
\begin{proof}
Notice that both $\cM_{t=0}$ and $\cM_{t=0}$ are abstractly isomorphic to the complex resolving divergence-free vector fields
\begin{equation}
  \label{eq:lin2} 
  \begin{tikzcd}[row sep = 1 ex]
    - & + \\ \hline
    \PV^{1,\bu}(\CC^5)_\mu \ar[r, "\div"] & \PV^{0,\bu}(\CC^5)_\nu  .
\end{tikzcd}
\end{equation}

To compute the derived intersection between the two Lagrangians at $t=0$ and $t=1$ we replace the Lagrangian morphism $\cM_{t=0} \hookrightarrow \cE_{\del}$. 
Consider the cochain complex $\til \cM_{t=0}$ defined by
\begin{equation}
  \label{eq:lin3} 
  \begin{tikzcd}[row sep = 1 ex]
    - & + & - \\ \hline
    \PV^{1,\bu}(\CC^5)_\mu \ar[r, "\div"] & \PV^{0,\bu}(\CC^5)_\nu \\ 
     \Omega^{0,\bu}(\CC^5)_\beta \ar[dr,dotted,"\id"]\ar[r, "\del"] & \Omega^{1,\bu}(\CC^5)_\gamma \ar[dr,dotted,"\id"] \\
     & \Omega^{0,\bu}(\CC^5)_{\til\beta} \ar[r, "\del"] & \Omega^{1,\bu}(\CC^5)_{\til\gamma} .
\end{tikzcd}
\end{equation}
Notice that as a graded vector space, this complex is of the form $\cE_{\del} \oplus (\Omega^{0,\bu} \oplus \Pi \Omega^{1,\bu})$. 
The $L_\infty$ structure on $\Pi \til \cM_{t=0}$ extends the one on $\cE_{\del}$ coming from the bulk BV action. 
Notice that the obvious embedding $\cM_{t=0} \hookrightarrow \til \cM_{t=0}$ is a quasi-isomorphism.

The projection map $\til \cM_{t=0} \twoheadrightarrow \cE_{\del}$ factors the original Lagrangian inclusion as
\[
\cM_{t=0} \hookrightarrow \til \cM_{t=0} \twoheadrightarrow \cE_{\del} .
\]
To compute the derived intersection of $\cM_{t=0}$ and $\cM_{t=1}$ we can compute the ordinary intersection of $\til \cM_{t=0}$ and $\cM_{t=1}$. 

Let $\mu_{t=1}$ and $\nu_{t=1}$ denote the fields present in the other boundary condition $\cM_{t=1}$. 
The intersection $\til \cM_{t=0} \times_{\cE_{\del}} \cM_{t=1}$ is computed by setting the fields $\beta, \gamma$ to zero and $\mu=\mu_{t=1}$, $\nu = \nu_{t=1}$. 
Thus, we are left with
\begin{equation}
  \label{eq:lin3} 
  \begin{tikzcd}[row sep = 1 ex]
    - & + & -\\ \hline
    \PV^{1,\bu}(\CC^5)_\mu \ar[r, "\div"] & \PV^{0,\bu}(\CC^5)_\nu  \\
         & \Omega^{0,\bu}(\CC^5)_{\til\beta} \ar[r, "\del"] & \Omega^{1,\bu}(\CC^5)_{\til\gamma} 
\end{tikzcd}
\end{equation}
This is precisely the underlying cochain complex of fields for the type I model with potentials. 
The odd non-degenerate pairing on this complex agrees with the one on this particular potential theory for the twist of type I supergravity. 
The $L_\infty$ structure on the parity shift of this complex is compatible with the one induced from the BV action in \eqref{eqn:Iactionpot}.
\end{proof}

\subsection{The $SU(5)$ twist of type IIA supergravity}
\label{s:su5IIA}

%\brian{Ingmar can you argue why this is the SU(5) twist on susy grounds?}

Thus, given that our 11-dimensional theory correctly describes the $SU(5)$-invariant twist of supergravity on $\CC^5 \times \RR$, to obtain the $SU(5)$ twist of type IIA supergravity we should reduce along the topological $\RR$ direction. 
This results in a $SU(5)$ invariant, holomorphic, theory on $\CC^5$. 

Let us briefly spell out the fields present in this dimensional reduction. 
The reduction is obtained by replacing $\Omega^\bu(\RR)$ with its translation invariant subalgebra $\CC[\ep] = \CC[\d t]$. 
Here, $\ep$ is an odd parameter playing the role of the translation invariant one-form $\d t \in \Omega^1(\RR)$. 
Equivalently, we are compactifying the theory along 
\[
\CC^5 \times S^1 \to \CC^5 .
\]

The 11-dimensional the field $\mu_{11d}$ is replaced by the field 
\[
\mu + \ep \mu' \in \Pi \PV^{1,\bu}(\CC^5) [\ep] .
\]
Notice that the lowest component of $\mu$ is odd (just like $\mu_{11d})$, but the lowest component of $\mu'$ is now even. 
Completely similarly, the remaining fields reduce as $\nu + \ep \nu'$, $\gamma + \ep \gamma'$, and $\beta + \ep \beta'$. 

In summary, the linear complex of fields of the dimensionally reduced theory on $\CC^5$ is
\begin{equation}
  \label{eqn:IIAsu5} 
  \begin{tikzcd}[row sep = 1 ex]
    {\rm odd} & {\rm even} & {\rm even} & {\rm odd} \\ \hline
    \PV^{1,\bu}(\CC^5)_{\mu} \ar[r, "\del"] & \PV^{0,\bu}(\CC^5)_\nu & \\ 
     & \ep\Omega^{0,\bu}(\CC^5)_{\beta'} \ar[r, "\div"] & \ep\Omega^{1,\bu}(\CC^5)_{\gamma'} . \\
     &  \ep \PV^{1,\bu}(\CC^5)_{\mu'} \ar[r, "\div"] & \ep \PV^{0,\bu}(\CC^5)_{\nu'} \\
     & & \Omega^{0,\bu}(\CC^5)_\beta \ar[r, "\div"] & \Omega^{1,\bu}(\CC^5)_\gamma.
\end{tikzcd}
\end{equation}

We can compute the dimensional reduction of the 11-dimensional action $S_{BF,\infty} + J$ in a similar way to how we have done in the past few sections. 
We arrive at the action functional described below. 

\begin{conj}
\label{conj:IIAsu5}
The $SU(5)$ twist of type IIA supergravity on $\CC^5$ is equivalent to the theory whose linear BRST complex of fields is displayed in \eqref{eqn:IIAsu5}. 
The full action functional is 
\begin{multline}
\label{eqn:su5action}
\int^\Omega_{\CC^5}\bigg(\beta' \wedge \dbar \nu + \beta \wedge \dbar \nu' + \gamma' \wedge \dbar \mu + \gamma \wedge \dbar \mu' +  \beta' \wedge \div \mu + \beta \wedge \div \mu' \bigg) \\
+ \int^\Omega_{\CC^5} \bigg( \frac12 \frac{1}{1-\nu} \mu^2 \vee \del \gamma' +  \frac{1}{1-\nu} (\mu \wedge \mu') \vee \del \gamma' + \frac12 \frac{\nu'}{(1-\nu)^2} \mu^2 \vee \del \gamma \bigg) \\
+ \frac12 \int_{\CC^5} \gamma' \wedge \del \gamma \wedge \del \gamma .
\end{multline} 
\end{conj} 

The first two lines in \eqref{eqn:su5action} arise from the reduction of the BF action $S_{BF,\infty}$. 
The final line arises from the reduction of $J = \frac16 \int \gamma_{11d} \del \gamma_{11d} \del \gamma_{11d}$. 

%\parsec[]
%
%
%We obtain further evidence that this is the $SU(5)$ invariant twist of type IIA supergravity by checking that it has the expected residual supersymmetries. 
%The type IIA superstring algebra $\lie{string}_{IIA}$ is a central extension of the 10d $\cN=(1,1)$ super Poincar\'e algebra, for a definition we refer to \cite{BH,FSS}. 
%
%A minimal twisting supercharge $Q_{hol}$ in the $\cN=(1,1)$ super Poincar\'e algebra determines a maximal isotropic $L \subset \CC^{10}$. 
%In a completely analogous way to the calculation for the twist of the algebra $\m2$ or the $SU(4)$ twist of the as in \cite{CLsugra} one can prove the following. 
%
%\begin{prop}
%As an $\lie{sl}(5)$ representation the $Q_{hol}$-cohomology of $\lie{string}_{IIA}$ is equivalent
%\end{prop}

\parsec[]\label{s:orbifold}

The slab compactification of the previous section was one way to implement the $S^1 / \ZZ/2$ reduction of the 11-dimensional theory. 
We offer another point of view of this $S^1 / \ZZ/2$ reduction. 

First off, there is the following $\ZZ/2$ action on the 11-dimensional theory on $\CC^5 \times S^1$ before compactifying. 
We obtain it by the following tensor product of $\ZZ/2$ actions. 
First, $\ZZ/2$ acts on $\Omega^\bu(S^1)$ by orientation reversing diffeomorphisms. 
Second, we declare that the eigenvalue of the $\ZZ/2$ action on $\PV^{k,\bu}(\CC^5)$, for $k=0,1$ is $+1$ and the eigenvalue of the $\ZZ/2$ action on $\Omega^{k,\bu}(\CC^5)$ for $k=0,1$ is $-1$. 
This determines a $\ZZ/2$ action on the full space of fields of the 11-dimensional theory. 

Upon $S^1$ compactification the $\ZZ/2$ action is easy to describe: $\mu,\nu$ both have eigenvalue $+1$, $\mu',\nu'$ both have eigenvalue $-1$, $\gamma,\beta$ both have eigenvalue $-1$, and $\gamma',\beta'$ both have eigenvalue $+1$. 
In particular, we see that the $\ZZ/2$ fixed points simply pick out the $\mu, \nu, \gamma', \beta'$ fields; this comprises the first two lines of \eqref{eqn:IIAsu5}. 

The fields match precisely with the fields in the twist of type I supergravity that we recalled in \S \ref{s:typeIpot} (Under the relabeling $\gamma' \leftrightarrow \til \gamma, \beta' \leftrightarrow \til \beta$). 
Furthermore, restricting the action in the above conjecture agrees precisely with the action of this twisted type I model. 

\subsection{Compactification along a CY3}\label{s:CY3}

In the first section we saw that the 11-dimensional theory can be defined on any manifold that is a product of a Calabi--Yau five-fold with a smooth oriented one-manifold. 
In this section, we investigate an important compactification of the 11-dimensional theory which involves the Calabi--Yau manifold $X \times \CC^2$ where $X$ is a simply connected compact Calabi--Yau three-fold.

The compactification of the theory along the three-fold $X$ 
\[
X \times \CC^2 \times \RR \to \CC^2 \times \RR
\]
yields an effective five-dimensional theory which is holomorphic along $\CC^2$ and topological along $\RR$. 
Upon compactification, we will find a match with a description of the twist of five-dimensional minimally supersymmetric supergravity. 

\begin{prop}
\label{prop:5dsugra}
The compactification of the 11-dimensional theory along a Calabi--Yau three-fold $X$ is equivalent to the twist of 5d $\cN=1$ supergravity with $h^{1,1}(X)-1$ vector multiplets and $h^{1,2}(X) + 1$ hyper multiplets. 
\end{prop}

\parsec[s:5dsugra]

We give a conjectural capitulation of the twist of 5d $\cN=1$ supergravity. 
Before twisting, a general 5d $\cN=1$ supergravity contains a gravity multiplet coupled to some number of vector and hyper multiplets. 
The twist of the vector and hyper multiplet has been computed in \cite{ESW}, and we recall it below. 
The twist of the gravity multiplet is less clear. 
A thorough computation of the twist has yet to appear, though some checks have been established by Elliott and the last author in \cite{EWpoisson}. 
We give a description of the twist now, but leave a detailed computation from first principles to future work. 

The gravity multiplet, see \cite{CCDF} for instance, consists of a graviton $e$, a gravitino $\psi$, and a one-form gauge field $\cA_{grav}$.
After twisting, the graviton and components of the gravitino decompose into two Dolbeault-de Rham valued fields 
\[
\alpha, \eta \in \Pi \Omega^{0,\bu}(\CC^2) \otimes \Omega^\bu(\RR) ,
\]
whose lowest components both carry odd parity. 
The one-form gauge field $\cA_{grav}$ and the remaining components of the gravitino decompose into two more Dolbeault-de Rham valued fields
\[
A_{grav} , B_{grav} \in \Pi \Omega^{0,\bu}(\CC^2) \otimes \Omega^\bu(\RR) ,
\]
whose lowest components also both carry odd parity. 

\begin{conj}
\label{conj:5dsugra}
The twist of 5d supergravity (with nonzero Chern--Simons term) with vector multiplets valued in a Lie algebra $\fg$ and hypermultiplets valued in a representation $V$ has BV fields
\begin{itemize}
\item $\alpha, A_{grav} \in \Pi \Omega^{0,\bu}(\CC^2) \otimes \Omega^\bu(\RR)$ with conjugate BV fields $\eta, B_{grav}$,
\item $A \in \Pi \Omega^{0,\bu}(\CC^2) \otimes \Omega^\bu(\RR) \otimes \fg$ with conjugate BV field $B$,
\item $\chi \in \Omega^{0,\bu}(\CC^2) \otimes \Omega^\bu(\RR) \otimes V$ with conjugate BV field $\psi$.
\end{itemize}

The action is
\begin{multline}
\label{eqn:5daction} 
\int^\Omega_{\CC^2 \times \RR} \left(\eta \dbar \alpha + B_{grav} \dbar A_{grav} + B \dbar A + \psi \dbar \chi \right) \\
  + \int^\Omega_{\CC^2 \times \RR} \left( \frac12\eta \{\alpha, \alpha\} +  B_{grav} \{\alpha, A_{grav}\}+ B \{\alpha, A\} +  \psi \{\alpha, \chi \}\right) \\ 
+ \frac16 \int_{\CC^2 \times \RR} B_{grav} \del B_{grav} \del B_{grav} .
\end{multline}
\end{conj}

%\brian{general background on susy. Why expect 5d $\cN=1$ susy}

\parsec[-]

With this description of the twist of five-dimensional supergravity, we turn to the proof of Proposition \ref{prop:5dsugra}. 


First, we set up some notation. 
Let $\Omega_X$ be the holomorphic volume form on $X$. 
To define the $11$-dimensional theory on $X \times \CC^2 \times \RR$ we use the Calabi--Yau form $\Omega_X \wedge \d z_1 \wedge \d z_2$, where $\{z_i\}$ is a holomorphic coordinate on $\CC^2$. 
Let $\omega \in \Omega^{1,1}(X)$ be a fixed K\"ahler form on $X$.
For any $k$, let $H^k(X, \Omega^k_X)_\perp$ denote the cohomology of the primitive elements. 

\begin{proof}
Consider the 11-dimensional field $\nu_{11d}$. 
Under the equivalence
\begin{align*}
\PV^{0,\bu}(X \times \CC^2) \otimes \Omega^\bu(\RR) & \simeq H^{\bu}(X, \cO) \otimes \PV^{0,\bu}(\CC^2) \otimes \Omega^\bu(\RR) \\ & = \PV^{0,\bu}(\CC^2) \otimes \Omega^\bu(\RR) \oplus \Pi \Bar{\Omega}_X \PV^{0,\bu}(\CC^2) \otimes \Omega^\bu(\RR) 
\end{align*}
the $\nu_{11d}$ field decomposes as 
\[ 
\nu_{11d} = \nu + \Bar{\Omega}_X \til \nu .
\]
Here $\Bar{\Omega}_X$ is the complex conjugate to the holomorphic volume form on $X$. 
Notice that the zero form component of $\til \nu$ is a field with even parity. 

Next, consider the 11-dimensional field $\mu_{11d}$. 
Under the equivalence 
\begin{align*}
\Pi \PV^{1,\bu}(X \times \CC^2) \otimes \Omega^\bu(\RR) & \simeq \Pi H^{\bu}(X, \cO) \otimes \PV^{1,\bu}(\CC^2) \otimes \Omega^\bu(\RR) \\ & \oplus \Pi H^\bu(X, \T_X) \otimes \PV^{0,\bu}(\CC^2) \otimes \Omega^\bu(\RR) \\ & = \Pi \PV^{1,\bu}(\CC^2) \otimes \Omega^\bu(\RR) \oplus \Bar{\Omega}_X \PV^{1,\bu}(\CC^2) \otimes \Omega^\bu(\RR) \\ & \oplus H^{1}(X, \T_X) \otimes \PV^{0,\bu}(\CC^2) \otimes \Omega^\bu(\RR) \oplus \Pi H^{2}(X, \T_X) \otimes \PV^{0,\bu}(\CC^2) \otimes \Omega^\bu(\RR)
\end{align*}
the field $\mu_{11d}$ decomposes as 
\begin{align*}
\mu_{11d} & = \mu + \Bar{\Omega}_X \til \mu \\ 
& + e^i \chi_i + f^a A_a +  (\Omega_X^{-1} \vee \omega^2)A_{grav} .  
\end{align*} 
Here, $\{e^i\}_{i=1,\ldots, h^{2,1}}$ is a basis for $H^{1}(X, \T_X)$ and $\{f^a\}_{a=1,\ldots, h^{1,1}-1}$ is a basis for 
\[
H^2 (X, \Omega^2_X)_\perp \subset H^2(X, \Omega^2_X) \cong H^2(X, \T_X) . 
\]
Notice that the zero form part of $\til{\mu}$ is an even field, the zero form part of $\chi_i$ is an even field, the zero form part of $A_a$ is an odd field, and the zero form part of $\mu_\omega$ is an odd field. 

The decomposition for the 11d fields $\gamma_{11d}$ and $\beta_{11d}$ is similar. 
We record it here:
\begin{align*}
\beta_{11d} & = \beta + \Bar{\Omega}_X \til \beta \\ 
\gamma_{11d} & = \gamma + \Bar{\Omega}_X \til \gamma + e_i \psi^i + f_a B^a + \omega \wedge B_{grav}  . 
\end{align*}
Here, $\{e_i\}_{i=1,\ldots,h^{2,1}}$ is a basis for $H^2 (X, \Omega^1_X)$ dual to the basis $\{e^i\}$ under the Serre pairing.
Also, $\{f_a\}_{a=1,\ldots,h^{1,1}-1}$ is a basis for $H^{1}(X, \Omega^1_X)_\perp$ dual to the basis $\{f^a\}$. 

To compare most directly to the description of the twist of 5d $\cN=1$ supergravity we modestly modify the fields. 
Let $\del$ be the holomorphic de Rham differential along $\CC^2$. 
First, we introduce a potential for the fields $\mu$ and $\til \mu$. 
Let 
\[
\alpha, \chi \in \Omega^{0,\bu}(\CC^2) \otimes \Omega^\bu(\RR)
\]
be differential forms satisfying $\del \alpha = \mu \vee \Omega_{\CC^2}$ and $\del \chi = \til \mu \vee \Omega_{\CC^2}$. 
The fields $\nu, \til \nu$ are set to zero. 
Dually, we replace the fields $\gamma, \til \gamma$ their `field strengths', suitably renormalized with respect to the volume form
\[
\eta = (\d^2 z)^{-1} \vee \del \til \gamma , \quad \psi = (\d^2 z)^{-1} \vee \del \gamma \in \Omega^{0,\bu}(\CC^2) \otimes \Omega^\bu(\RR) .
\]
The roles of $\beta, \til \beta$ were as gauge symmetries implementing $\gamma \to \gamma + \del \beta$ and $\til \gamma \to \til \gamma + \del \til \beta$. 
Since we are replacing $\gamma, \til \gamma$ by their images under the operator $\del$, these gauge symmetries are set to zero. 

In summary, we are left with the following fields 
\[
\begin{array}{cccccccccc}
\alpha,A_{grav} & \in & \Pi \Omega^{0,\bu}(\CC^2) \otimes \Omega^\bu(\RR), & \eta, B_{grav} & \in & \Pi \Omega^{0,\bu}(\CC^2) \otimes \Omega^\bu(\RR) \\
\chi, \chi_i & \in & \Omega^{0,\bu}(\CC^2) \otimes \Omega^\bu(\RR),  & \psi, \psi^i & \in & \Omega^{0,\bu}(\CC^2) \otimes \Omega^\bu(\RR), & i=1,\ldots, h^{2,1}  \\
A_a & \in & \Pi \Omega^{0,\bu}(\CC^2) \otimes \Omega^\bu(\RR), & B^a & \in & \Pi \Omega^{0,\bu}(\CC^2) \otimes \Omega^\bu(\RR) , & a = 1, \ldots, h^{1,1}-1.
\end{array}
\]

Let us plug these fields in to the 11-dimensional action.
First, consider the BF term $\frac12 \int^{\Omega} \frac{1}{1-\nu_{11d}} \mu_{11d}^2 \gamma_{11d}$. 
With the field redefinitions above, this decomposes as
\begin{multline}\label{eqn:5dsugra1}
 \int_{\CC^2 \times \RR}^\Omega \left( \frac12 \del \alpha \wedge \del \alpha \wedge \eta + \del A_{grav} \wedge \del A_{grav} \wedge B_{grav} \right) \\
 + \int_{\CC^2 \times \RR}^\Omega \left( \del \alpha \wedge \del \chi \wedge \psi + \del \alpha \wedge \del \chi_i \wedge \psi^i + \del \alpha \wedge \del A_a \wedge B^a \right) .
\end{multline}
This term agrees with the second line in the 5d action \eqref{eqn:5daction}. 

Finally, consider the term in the 11-dimensional action $J(\gamma_{11d}) = \frac16 \int \gamma_{11d} \wedge \del \gamma_{11d} \wedge \del \gamma_{11d}$. 
This induces the five-dimensional Chern--Simons term 
\beqn\label{eqn:5dsugra2}
\frac16 \int_{\CC^2 \times \RR} B_{grav} \del B_{grav} \del B_{grav} .
\eeqn
This completes the proof. 
\end{proof}

\parsec[s:5dglobal]

In \S \ref{sec:global} we computed the global symmetry algebra of the 11-dimensional theory on $\CC^5 \times \RR$ and found a close relationship to the exceptional super Lie algebra $E(5,10)$. 
In this section we deduce the form of the global symmetry algebra of the five-dimensional compactified theory on $\CC^2 \times \RR$. 

Consider the full de Rham cohomology of $X$ by
\[
H^\bu (X, \Omega^\bu) = \oplus_{i,j} H^i (X, \Omega^j_X)  .
\]
This is a graded commutative algebra using the wedge product of differential forms. 
Next, consider the space of holomorphic functions $\cO(\CC^2)$ on $\CC^2$. 
The Poisson bracket $\{-,-,\}$ associated to the standard holomorphic symplectic structure on $\CC^2$ endows $\cO(\CC^2)$ with the structure of a Lie algebra. 
In particular, we can tensor $\cO(\CC^2)$ with $H^\bu (X, \Omega^\bu)$ to obtain the structure of a graded Lie algebra on 
\[
H^\bu (X, \Omega^\bu) \otimes \cO(\CC^2) .
\]
Let $[\omega] \in H^1(X, \Omega^1_X)$ be the class of the K\"ahler form on $X$.

The global symmetry algebra of the compactified theory along the Calabi--Yau three-fold $X$ is equivalent to a deformation of this graded Lie algebra. 
The deformation introduces the following Lie bracket 
\[
\big[ [\omega] \otimes f, [\omega] \otimes g\big] = [\omega^2] \otimes \{f,g\} \in H^{2,2}(X, \Omega^2) \otimes \cO(\CC^2) . 
\]

%\begin{proof}
%The computation of the global symmetry algebra of the 5-dimensional theory is very similar to the computation in the 11-dimensional theory.
%In terms of the de Rham cohomology, the generators of the linearized cohomology of the space of fields read:
%\begin{itemize}
%\item a holomorphic function $\alpha(z_1,z_2) \in H^0(X, \Omega^0) \otimes \cO(\CC^2)$.
%\item a 
%
%First, let's compute the global symmetry algebra of the twist of the pure gravity theory on $\CC^2 \times \RR$. 
%This arises from terms in the BV action \eqref{eqn:5daction} involving only $\alpha, \eta, A_{grav}$, and $B_{grav}$. 
%
%The underlying graded vector space of the resulting Lie algebra is equivalent to
%\[
%\cO(\CC^2)_\alpha \oplus \cO(\CC^2)_\eta \oplus \cO(\CC^2)_{A_{grav}} \oplus \cO(\CC^2)_{B_{grav}} .
%\]
%The non trivial Lie brackets are
%\begin{multline}
%[\alpha, \alpha'] = \{\alpha, \alpha'\} \in \cO(\CC^2)_\alpha , \quad [\alpha, \eta] = \{\alpha, \eta\} \in \cO(\CC^2)_\eta , \quad [\alpha, A_{grav}] = \{\alpha, A_{grav}\} \in \cO(\CC^2)_{A_{grav}} \\ 
%[\alpha, B_{grav}] = \{\alpha, B_{grav}\} \in \cO(\CC^2)_{B_{grav}} , \quad [B_{grav}, B'_{grav}] = \{B_{grav}, B'_{grav}\} \in \cO(\CC^2)_{A_{grav}} .
%\end{multline}
%\begin{align*}
%[\alpha, \alpha' + \eta + A_{grav} + B_{grav}] & = \{\alpha,  \alpha'\} + \{\alpha, \eta\} + \{\alpha, A_{grav}\} + \{\alpha, B_{grav}\} \\ & \in \cO(\CC^2)_\alpha \oplus \cO(\CC^2)_\eta \oplus \cO(\CC^2)_{A_{grav}} \oplus \cO(\CC^2)_{B_{grav}} \\
%[B_{grav}, B'_{grav}] & = \{B_{grav}, B'_{grav}\} \in \cO(\CC^2)_{A_{grav}} .
%\end{align*}
%Denote this Lie algebra by $\fg_{grav}$. 
%
%The contribution of the vector and hypermultiplets form a module for $\fg_{grav}$. 
%As a graded vector space, this module is
%\[
%M_X = \cO(\CC^2) \otimes H^{1}(X, \Omega^1_X)_\perp [-1] \oplus \cO(\CC^2) \otimes 
%\]
%\brian{incomplete}







\section{Twisted supergavity on AdS space}
\label{sec:ads}

So far, we have mostly given evidence for the 11-dimensional theory as a twist of supergravity in a flat background. 
We now turn to twisted versions of AdS backgrounds of 11-dimensional supergravity. 

In $M$-theory, AdS backgrounds arise from backreacting some number of branes. 
For $M2$ branes, the backreacted geometry is ${\rm AdS}_4 \times S^7$.
For the $M5$ branes, the backreacted geometry is ${\rm AdS}_7 \times S^4$. 

According to the AdS/CFT correspondence, supergravity on such backgrounds should be dual to the relevant worldvolume theory. 
In this section, we do not directly refer to the worldvolume theories on the holomorphic twists of the $M2$ and $M5$ branes.
Rather, we identify the fields sourcing the branes at the level of the twisted 11-dimensional theory.
In turn, we give a proposal for the twisted AdS background. 
We will show that the twist of the superconformal algebra is a global symmetry of this twisted background. 

\subsection{Superconformal algebras}

The complex form of the algebra of isometries for supergravity in both the ${\rm AdS}_4$ and ${\rm AdS}_7$ backgrounds is $\lie{osp}(8|2)$ (though, their real forms differ). 
This agrees with the complex form of the 6d $\cN=(2,0)$ superconformal algebra and the 3d $\cN=8$ superconformal algebra. 
The bosonic part of this algebra is isomorphic to $\lie{so}(8) \oplus \lie{sp}(2) \cong \lie{so}(8) \oplus \lie{so}(5)$. 

The minimal supercharge $Q$ acting on 11-dimensional supersymmetry algebra is an element of this superconformal algebra. 
In \cite{SWsuco2}, the second two authors show that the $Q$-cohomology is isomorphic to $\lie{osp}(6|1)$. 
This super Lie algebra will play the role of the isometries in the twisted AdS background. 

\subsection{The ${\rm AdS}_4 \times S^7$ background}

In this section we introduce the analog of the ${\rm AdS}_4 \times S^7$ background in our conjectural description of the minimal twist of 11-dimensional supergravity. 
%In the physical AdS background, the only bosonic fields which are non-zero are the metric and the \brian{finish}

\parsec[]

Decompose the 11-dimensional manifold $\CC^5 \times \RR$ as
\[
 \CC^4_w\times \CC_z \times \RR .
\]

Analogous to before, the ${\rm AdS}_4 \times S^7$ background arises from backreacting M2 branes. Consider a stack of $N$ $M2$ branes wrapping $\R\times \C_z$. A natural interaction to consider is 
\[
I_{M2}(\gamma) = N\int_{\C_z} \gamma + \cdots
\] 
which is nonzero only on the component of $\gamma$ in $\Omega^1(\R)\otimes \Omega^{1,1}(\C^5)$. Unlike the case of $M5$ branes, the coupling does not involve choosing a primitive for a field strength - it is an electric coupling.
We have only indicated the lowest order coupling, the $\cdots$ indicate higher order couplings which will be higher order in the fields of the 11d theory and explicitly involve the fields in the worldvolume theory. 

This coupling is justified by comparison with the physical theory and by dimensional reduction. 
Indeed, as discussed in section \ref{s:components}, the component of $\gamma$ which participates in the above coupling is a component of the $C$-field of eleven dimensional supergravity. Thus, the proposal mirrors electric couplings of $M2$ branes in the physical theory, which simply involves integrating the $C$-field over the worldvolume of the brane. 

Moreover, reducing on a circle transverse to the $M2$ brane yields the $SU(4)$ twist of type IIA supergravity on $\R^2\times \C_z\times \C^3$ with $N$ $D2$ branes wrapping $\R\times \C_z$. As is shown in \cite{CLsugra}, an electric coupling of D2 branes to the $SU(4)$ twist of type IIA supergravity is given by 
\[
I_{D2}(\gamma) = N \int_{\R\times\C_z} \gamma + \cdots
\] 
where $\gamma$ now denotes the 1-form field of the $SU(4)$ twist of type IIA supergravity. It is immediate that the pullback of $I_{M2}$ along the map in the proof of proposition \ref{prop:dimred} recovers $I_{D2}$. 


% Based on the discussion above, it is natural to expect that there is a field of twisted 11-dimensional supergravity which sources the twist of a stack of $N$ $M2$ branes living on the submanifold $\CC_z \times \RR \cong \{w=0\} \subset \CC^5$. 

% The differential form which sources the brane is an element
% \[
% \til{F} \in \Omega^{4,3} (\CC^4_w \, \setminus \, 0) \otimes \Omega^{0,0} (\CC_z) \otimes \Omega^{0} (\RR) \subset \Omega^{\bu} \left(\CC^5 \times \RR \, \setminus \, \{w=0\}\right) .
% \]
% Equivalently, we can think about this as a distributional valued form $\til{F} \in \Bar{\Omega}^{4,3}(\CC^5) \otimes \Omega^0 (\RR)$ which satisfies the distributional equation
% \[
% \dbar \til{F} = N \delta_{w=0} 
% \]
% where $\delta_{w=0}$ is the Dirac $\delta$-distributional for the submanifold $\{w=0\} = \CC_z \times \RR$. 

% Using the Calabi--Yau form we can identify such a differential form with a field of twisted supergravity. 
% Indeed $F = \til{F} \wedge \Omega^{-1}$ is a distributional field of type
% \[
% F \in \Bar{\PV}^{1,3}(\CC^5) \otimes \Omega^0 (\RR) .
% \]
% For $F$ to make sense as a background of twisted supergravity it must satisfies the appropriate (possibly nonlinear) equation of motion, which we now verify. 

\parsec[sec:m2backreact]

The backreacted geoemtry will be given by a solution to the equations of motion upon deforming the 11d action by the interaction $I_{M2}(\gamma)$. 
Varying the deformed action with respect to $\gamma$ 
we obtain the equation of motion
\beqn\label{eqn:ads4eom1}
\dbar \mu + \frac12 [\mu, \mu] + \partial\gamma\partial\gamma = N \Omega^{-1} \delta_{w=0} \\
\eeqn
Here $[-,-]$ is the Schouten bracket. 
Varying $\beta$ we obtain the equation of motion
\beqn\label{eqn:adseom2}
\div \mu = 0 .
\eeqn

\begin{lem}
Let
\[
 F_{M2} = \frac{6}{(2\pi i)^4} \frac{\sum_{a=1}^4 \wbar_a \d \wbar_1 \cdots \Hat{\d \wbar_a} \cdots \d \wbar_4}{\|w\|^{8}} \partial_z .
\]
Then, the background where $\mu = N F_{M2}$ and $\gamma = 0$
satisfies the above equations of motion in the presence of a stack of $N$ $M2$ branes:
\begin{align*}
\dbar (N F_{M2}) + \frac12 [N F_{M2}, N F_{M2}] & = N \Omega^{-1} \delta_{w=0} \\
\div (N F_{M2}) & = 0  .
\end{align*}
Here, we set all components of the field $\gamma$ equal to zero (as well as the fields $\nu,\beta$). 
\end{lem}

\begin{proof}
Upon specializing $\gamma = 0$, the last term in the first equation above vanishes. The equation $\dbar F_{M2} = \Omega^{-1} \delta_{w=0}$ characterizes the Bochner--Martinelli kernel representing the residue class on $\CC^4 \, \setminus \, 0$. 
It is clear that $\div F_{M2} = 0$ and 
\[
[F_{M2}, F_{M2}] = 0
\] 
by simple type reasons. 
\end{proof}

\parsec[]

To provide evidence for the claim that this is the twisted analog of the AdS geometry we will match the symmetries present in the physical theory and those in the twisted theory. 

We have recalled that the $Q$-cohomology of $\lie{osp}(8|2)$ is isomorphic to the super Lie algebra $\lie{osp}(6|1)$. 
We will define an embedding of $\lie{osp}(6|1)$ into the 11-dimensional theory on $\CC^5 \times \RR \setminus \{w=0\}$ which corresponds to the twist of the 3d superconformal algebra.
We first focus on the case where the flux $N=0$, in this case the embedding can be extended to all of $\CC^5 \times \RR$. 

\parsec[] 

The bosonic part of $\lie{osp}(6|1)$ is the direct sum Lie algebra $\lie{sl}(4) \oplus \lie{sl}(2)$. 
The Lie algebra $\lie{sl}(2)$ represents special conformal transformations in $\CC_z$; the vector fields representing these transformations are not divergence-free so must be slightly adjusted. 
The Lie algebra $\lie{sl}(4)$ represents rotations along the plane $\CC^4_w$.   

\begin{itemize}
\item The bosonic summand $\lie{sl}(2)$ is mapped to the vector fields:
\[
\frac{\del}{\del z} ,\quad z \frac{\del}{\del z} - \frac14 \sum_{a=1}^4 w_a \frac{\del}{\del w_a} , \quad z \left(z \frac{\del}{\del z} - \frac12 \sum_{a=1}^4 w_a \frac{\del}{\del w_a} \right) \in \PV^{1,0}(\CC^5) \otimes \Omega^0(\RR) .
\]
Notice that these vector fields are divergence-free and along $w=0$ reduce to the usual special conformal transformations.
\item The bosonic summand $\lie{sl}(4)$ is mapped to the $4$-dimensional rotations: 
\[
\sum_{a,b=1}^4 B_{ab} w_a \frac{\del}{\del w_b} \in \PV^{1,0}(\CC^5) \otimes \Omega^0(\RR) , \quad (B_{ab}) \in \lie{sl}(4) .
\]
\end{itemize}

The odd part of the algebra $\lie{osp}(6|1)$ is $\wedge^4 W \otimes R$ where $W$ is the fundamental $\lie{sl}(4)$ representation and $R$ is the fundamental $\lie{sl}(2)$ representation. 
It is natural to split $R = \CC_{+1} \oplus \CC_{-1}$ so that the odd part decomposes as
\[
(\wedge^2 \CC^4)_{+1} \oplus (\wedge^2 \CC^4)_{-1} .
\]

\begin{itemize}
\item 
The fermionic summand $(\wedge^2 \CC^4)_{+1}$ consists of the supertranslations. 
It is mapped to the fields: 
\[
\frac{1}{2} (w_a \d w_b - w_b \d w_a) \in \Omega^{1,0}(\CC^5) \otimes \Omega^0(\RR) , \quad a,b=1,2,3,4 .
\] 
\item The fermionic summand $(\wedge^2 \CC^4)_{-1}$ consists of the remaining superconformal transformations. 
It is mapped to the fields: 
\[
\frac{1}{2} z (w_a \d w_b - w_b \d w_a) \in \Omega^{1,0}(\CC^5) \otimes \Omega^0(\RR) , \quad a,b=1,2,3,4. 
\] 
\end{itemize}


\begin{lem}\label{lem:m2emb}
These assignments define an embedding of $\lie{osp}(6|1)$ into the linearized BRST cohomology of the fields of the 11-dimensional theory on $\CC^5 \times \RR$. 
Equivalently, it defines an embedding
\[
i_{M2} \colon \lie{osp}(6|1) \hookrightarrow E(5,10) .
\]
\end{lem} 
\begin{proof}
The second assertion follows from Theorem \ref{thm:global} that as a super Lie algebra the linearized BRST cohomology of the global symmetry algebra of the 11-dimensional theory on $\CC^5 \times \RR$ is the trivial central extension of $E(5,10)$. 
Recall that the odd part of $E(5,10)$ is precisely the module of closed two-forms on $\CC^5$. 
To explicitly describe the embedding into $E(5,10)$ we simply apply the de Rham differential to the last two formulas above.
Recall, we are using the holomorphic coordinates $(z,w_1,\ldots,w_4)$ on $\CC^5$ where $z$ is the holomorphic coordinate along the $M2$ brane. 
\begin{itemize}
\item 
The fermionic summand $(\wedge^2 \CC^4)_{+1}$ embeds into closed two-forms as
\[
\d w_a \wedge \d w_b, \quad a,b=1,2,3,4. 
\] 
\item The fermionic summand $(\wedge^2 \CC^4)_{-1}$ embeds into closed two-forms as
\[
z \d w_a  \wedge \d w_b + \frac12 \d z \wedge (w_a \d w_b - w_b \d w_a) , \quad a,b=1,2,3,4. 
\] 
\end{itemize}
\end{proof}
\parsec[]

Next, we turn on a nontrivial unit of flux $N \ne 0$. 
Since not all of the fields we wrote down above commute with the flux $N F_{M2}$, they are not compatible with the total differential $\delta^{(1)} + [N F_{M2}, -]$ acting on the fields supported on $\CC^5 \times \RR \setminus \{w=0\}$. 
Nevertheless, we have the following. 

\begin{prop}
\label{prop:brads4}
There exist $N$-dependent corrections to the fields defining the embedding of $\lie{osp}(6|1)$ summarized above which are closed for the modified BRST differential $\delta^{(1)} + [N F_{M2},-]$. 
Furthermore, these order $N$ corrections define an embedding of $\lie{osp}(6|1)$ inside the cohomology of the fields of 11-dimensional theory on $\CC^5 \times \RR \setminus \CC \times \RR$ with respect to the differential $\delta^{(1)} + [N F_{M2},-]$.
\end{prop}

\begin{proof}
Let $\cL(\CC^5 \times \RR \setminus \{w=0\})$ denote the super $L_\infty$ algebra obtained by parity shifting the fields of the 11-dimensional theory. 
We make the identification 
\[
(\CC^5 \times \RR) \setminus \{w=0\} \cong (\CC_w^4 \setminus 0) \times \CC_z \times \RR .
\]

Set $F = F_{M2}$ for notational convenience. Recall that we are viewing $F$ as an element of $\PV^{1,3}(\CC_w^4 \setminus 0) \otimes \Omega^{0,0}(\CC_z) \otimes \Omega^0(\RR)$. 
The operator $[F,-]$ acts on the fields according to two types of maps:
\begin{align*}
[F ,-] & \colon \PV^{i,\bu}(\CC^4_w \setminus 0) \otimes \PV^{j,\bu} (\CC_z) \otimes \Omega^\bu (\RR) \to \PV^{i,\bu+3}(\CC^4_w \setminus 0) \otimes \PV^{j,\bu} (\CC_z) \otimes \Omega^\bu (\RR) \\
[F,-] & \colon \Omega^{i,\bu}(\CC^4_w \setminus 0) \otimes \Omega^{j,\bu} (\CC_z) \otimes \Omega^\bu (\RR) \to \Omega^{i,\bu+3}(\CC^4_w \setminus 0) \otimes \Omega^{j,\bu} (\CC_z) \otimes \Omega^\bu (\RR).
\end{align*}

%\brian{get the filtration straight}


The first page of the spectral sequence is the cohomology with respect to the original linearized BRST differential $\delta^{(1)}$. 
Recall that the linearized BRST differential decomposes as
\[
\delta^{(1)} = \dbar + \d_{\RR} + \div |_{\mu \to \nu} + \del |_{\beta \to \gamma}  .
\]
To compute this page, we use an auxiliary spectral sequence which simply filters by the holomorphic form and polyvector field type. 
This first page of this auxiliary spectral sequence is simply given by the cohomology with respect to $\dbar + \d_{\RR}$. 
This cohomology is given by
\begin{equation}
  \label{eqn:ads4ss} 
  \begin{tikzcd}[row sep = 1 ex]
    + & - \\ \hline
H^\bu(\CC^4\setminus 0, \T) \otimes H^\bu(\CC, \cO) & H^\bu(\CC^4 \setminus 0, \cO) \otimes H^\bu(\CC, \cO) \\
H^\bu(\CC^4\setminus 0, \cO) \otimes H^\bu(\CC, \T) \\
H^\bu(\CC^4\setminus 0, \cO) \otimes H^\bu(\CC, \cO) & H^\bu(\CC^4\setminus 0, \cO) \otimes H^\bu(\CC, \Omega^1) \\ & H^\bu(\CC^4\setminus 0, \Omega^1) \otimes H^\bu(\CC, \cO)  
\end{tikzcd}
\end{equation}
where $\T$ denotes the holomorphic tangent sheaf, $\Omega^1$ denotes the sheaf of holomorphic one-forms, and $\cO$ is the sheaf of holomorphic functions.

The cohomology of $\CC$ is concentrated in degree zero and there is a dense embedding
\[
\CC[z] \hookrightarrow H^\bu(\CC, \cF) 
\]
for $\cF = \cO, \T$, or $\Omega^1$. 

For $\cF = \cO, \T$, or $\Omega^1$, the cohomology $H^\bu(\CC^4 \setminus 0, \cF)$ is concentrated in degrees $0$ and $3$. 
There are the following dense embeddings 
\begin{align*}
\CC[w_1,\ldots, w_4] & \hookrightarrow H^0(\CC^4 \setminus 0, \cO) \\ 
\CC[w_1,\ldots, w_4] \{\partial_{w_i}\} & \hookrightarrow H^0(\CC^4 \setminus 0, \T) \\
\CC[w_1,\ldots, w_4] \{\d w_i\} & \hookrightarrow H^0(\CC^4 \setminus 0, \Omega^1) 
\end{align*}
and
\begin{align*}
(w_1\cdots w_4)^{-1} \CC[w_1^{-1},\ldots, w_4^{-1}] & \hookrightarrow H^3(\CC^4 \setminus 0, \cO) \\ 
(w_1\cdots w_4)^{-1} \CC[w_1^{-1},\ldots, w_4^{-1}] \{\partial_{w_i}\} & \hookrightarrow H^3(\CC^4 \setminus 0, \T) \\
(w_1\cdots w_4)^{-1} \CC[w_1^{-1},\ldots, w_4^{-1}] \{\d w_i\} & \hookrightarrow H^3(\CC^4 \setminus 0, \Omega^1) .
\end{align*}

It follows that (up to completion) the cohomology 
\[
H^\bu(\cL(\CC^5 \times \RR \setminus \{w=0\}) , \dbar)
\]
is the direct sum of $H^\bu(\cL(\CC^5 \times \RR), \dbar)$ with 
\begin{equation}
  \label{eqn:ads4ss2} 
  \begin{tikzcd}[row sep = 1 ex]
    - & + \\ \hline
H^3(\CC^4\setminus 0, \cO)[z] \{\partial_{w_i}\}  \ar[r, dotted, "\div"] & H^3(\CC^4 \setminus 0, \cO) [z] \\
H^3(\CC^4\setminus 0, \cO) [z] \partial_z \ar[ur, dotted, "\div"'] \\
H^3(\CC^4\setminus 0, \cO) [z] \ar[r, dotted, "\del"] \ar[dr, dotted, "\del"'] & H^3(\CC^4\setminus 0, \cO)[z] \d z \\ & H^3(\CC^4\setminus 0, \Omega^1)[z] \{\d w_i\} .
\end{tikzcd}
\end{equation}
The remaining piece of the original BRST operator is drawn in dotted lines. 
The first page of the spectral sequence converging to the cohomology with respect to $\delta^{(1)} + [N F, -]$ is given by the cohomology of the global symmetry algebra on $\CC^5 \times \RR$, which we computed in \S \ref{sec:global}, plus the cohomology of the above complex with respect to dotted line operators. 
In this description the image of the flux $F$ at this page in the spectral sequence corresponds to the following class 
\[
[F] = (w_1 \cdots w_4)^{-1} \partial_z \in H^3(\CC^4\setminus 0, \cO) [z] \partial_z .
\]

The next page of the spectral sequence is given by computing the cohomology with respect to the operator $[N F,-]$. 
As observed above, this operator maps Dolbeault degree zero elements to Dolbeault degree three elements. 
For degree reasons, there are no further differentials and the spectral sequence collapses after the second page. 

The embedding of $\lie{osp}(6|1)$ we wrote down in lemma \ref{lem:m2emb} lands in the kernel of the original BRST operator $\delta^{(1)}$. 
To see that it this embedding can be lifted to the full cohomology we need to check that any element in the image of the original embedding is annihilated by $\big[ N [F] , - \big]$. 
This is a direct calculation. 
For instance, recall that an element in the image of the odd summand $(\wedge^2 \CC^2)_{-1}$ (which corresponds to a superconformal transformation) is of the form $z w_a \wedge \d w_b = z(w_a \d w_b - w_b \d w_a)$. 
We have
\[
\big[[F] , z(w_a \d w_b - w_b \d w_a) \big] = (w_1\cdots w_4)^{-1} (w_a \d w_b - w_b \d w_a) = 0
\]
since the class $(w_1\cdots w_4)^{-1}$ is in the kernel of the operator given by multiplication by $w_a$ for any $a = 1,\ldots 4$. 
\end{proof}

\subsection{The ${\rm AdS}_7 \times S^4$ background}

In this section we introduce the analog of the ${\rm AdS}_7 \times S^4$ background in our description of the minimal twist of 11-dimensional supergravity. Decompose the eleven dimensional spacetime as $\C^3_z\times \C^2_w\times \R$.

%In the physical AdS background, the only bosonic fields which are non-zero are the metric and the \brian{finish}

\parsec[sec:m5coupling]

Analogous to the physical theory, the ${\rm AdS}_7 \times S^4$ background in the holomorphic twist will arise by backreacting $M5$ branes. To this effect, we begin by discussing how the 11d theory couples to M5 branes. 
Consider a stack of $N M5$ branes wrapping 
\[
\{w_1=w_2=t=0\} \subset \C^3_z\times \C^2_w\times \R 
\] 

It is natural to consider the nonlocal interaction 
\[
I_{M5} = N\int_{\C^3_z} \div^{-1}\mu \vee \Omega +\cdots 
\]
Note that this expression is only nonzero on the component of $\mu$ in $\PV^{1,3}$. 
We argue that this coupling is consistent with expectations from the physical theory and from dimensional reduction. 

The twisted field $\mu^{1,3}$ is a component of the Hodge star of the $G$-flux in the physical theory \ref{s:components}. 
In the physical theory, M5 branes magnetically couple to the $C$-field; the coupling involves choosing a primitive for the hodge star of the $G$-flux and integrating it over the $M5$ worldvolume. Our twist contains no fields corresponding to components of such a primitive; hence such a magnetic coupling is reflected in the appearance of $\div^{-1}$. 

\parsec[]

We obtain a deeper justification for this coupling through dimensional reduction to type IIA supergravity. 
Reducing on the circle along the directions the $M5$ branes wrap yields the $SU(4)$ invariant twist of type IIA supergravity on $\CC^4 \times \RR^2$ with $N$ $D4$ branes wrapping $\CC^2 \times \RR$. 

In \cite{CLsugra}, it is shown that the magnetic coupling of $D4$ branes to the $SU(4)$ twist of IIA is of the form
\[
N \int _{\C^2 \times \RR} \div^{-1} \mu \vee \Omega_{\C^4} + \cdots .
\]
Again, we have only explicitly indicated the first-order piece of the coupling. 

\parsec[s:m5backreact]

The backreacted geometry will be given by a solution to the equations of motion upon deforming the 11-dimensional action by the interaction $I_{M5}(\mu)$. 

Varying the potential $\div^{-1} \mu$, we obtain the following equation of motion involving the field $\gamma$:
\beqn\label{eqn:m5eom1}
\dbar \del \gamma + \div \left(\frac{1}{1-\nu} \mu\right) \wedge \del \gamma = N \delta_{w_1=w_2=t=0} .
\eeqn
Notice that there is an extra derivative compared to the equation of motion arising from varying the field $\mu$. 
This equation only depends on $\gamma$ through its field strength $\del \gamma$. 

Varying $\gamma$ we obtain the equation of motion 
\beqn\label{eqn:m5eom2}
(\dbar + \d_\RR) \mu + \del \gamma \del \gamma = 0 .
\eeqn 
Again, this only depends on $\gamma$ through its field strength $\del \gamma$.


\begin{lem}
\label{lem:ads7flux}
Let
\[
F_{M5} = \frac{1}{(2\pi i)^3} \frac{\wbar_1 \d \wbar_2 \wedge \d t - \wbar_2 \d \wbar_1 \wedge \d t + t \d \wbar_1 \wedge \d \wbar_2}{(\|w\|^2 + t^2)^{5/2}} \wedge \d w_1 \wedge \d w_2
\]
%and suppose 
%\[
%\til F \in \Bar{\Omega}^{1,2} (\CC^5) \otimes \Bar{\Omega}^1(\RR)
%\]
%satisfies $\del \til F = F$. 
Then, $\del\gamma = N F_{M5}$, $\mu = 0$, and $\nu = 0$ satisfies the equations of motion in the presence of a stack of $N$ $M5$ branes sourced by the term $N \delta_{w_1=w_2=t=0}$:
\begin{align*}
\dbar (NF_{M5}) + \d_{\RR} (NF_{M5}) & = N \delta_{w_1=w_2=t=0}  \\ 
(NF_{M5}) \wedge (NF_{M5}) & = 0 .
\end{align*}
Here, we set all components of the field $\mu$ equal to zero (as well as the fields $\nu,\beta$). 
\end{lem}

\begin{proof}
The first equation equation 
\[
\dbar F + \d_{\RR} F = N \delta_{w_1=w_2=t=0}
\]
characterizes the kernel representing $N$ times the residue class for a $4$-sphere in 
\[
(\CC^2 \times \RR) \setminus 0 \simeq S^4 \times \RR .
\] 
That is
\[
\oint_{S^4} N F = N 
\]
for any $4$-sphere centered at $0 \in \CC^2 \times \RR$.

The second equation $F \wedge F = 0$ follows by simple type reasons. 
\end{proof}

\parsec[s:m5embedding]

To provide evidence for the claim that this is the twisted analog of the AdS geometry we will match the symmetries present in the physical theory on ${\rm AdS}_7 \times S^4$ and those in the twisted theory. 

We have recalled that the $Q$-cohomology of $\lie{osp}(8|2)$ is isomorphic to the super Lie algebra $\lie{osp}(6|1)$. 
We will define an embedding of $\lie{osp}(6|1)$ into the 11-dimensional theory on $\CC^5 \times \RR \setminus \{w_1=w_2=t=0\}$ which corresponds to the twist of the 6d superconformal algebra.

We first focus on the case where the flux $N=0$.
In this case the embedding can be extended to all of $\CC^5 \times \RR$. 



Recall that we have chosen coordinates of the form
\[
\CC^5 \times \RR = \CC_z^3 \times \CC_w^2 \times \RR_t
\]
with $z_i, i=1,2,3$ and $w_a, a=1,2$.
The stack of $M5$ branes wrap $w_1=w_2=t=0$. 

The embedding of the bosonic piece of $\lie{osp}(6|1)$ can be described as follows. Recall that the bosonic part of $\lie{osp}(6|1)$ is the direct sum Lie algebra
\[
\lie{sl}(4) \oplus \lie{sl}(2) .
\]
which we write as $\lie{sl}(W) \oplus \lie{sl}(R)$ with $W,R$ complex four, two dimensional complex vector spaces. The roles of the $\lie{sl}(4)$ and $\lie{sl}(2)$ summands are interchanged compared to the case of the $M2$ brane. 
The Lie algebra $\lie{sl}(4)$ represents conformal transformations along the plane $\CC^3_z$.
Since not all such infinitesimal transformations are divergence-free, there precise formulas must be adjusted.   
The Lie algebra $\lie{sl}(2)$ represents rotations in $\CC^2_w$; the vector fields representing these transformations are automatically divergence free.
In more detail, the embedding of the bosonic piece can be given by the following explicit formulas. 

\begin{itemize}

\item
The bosonic abelian subalgebra $\CC^3 \subset \lie{sl}(4)$ is mapped to the translations 
\[
\frac{\del}{\del z_i} \in \PV^{1,0}(\CC^5) \otimes \Omega^0(\RR) , \quad i=1,2,3.
\]

\item
The bosonic subalgebra $\lie{sl}(3) \subset \lie{sl}(4)$ is mapped to the 
rotations
\[
A_{ij} z_i \frac{\del}{\del z_j} \in \PV^{1,0}(\CC^5)\otimes \Omega^0(\RR) , \quad (A_{ij}) \in \lie{sl}(3) .
\]

\item
The bosonic subalgebra $\CC \subset \lie{sl}(4)$ is mapped to the element
\[
\sum_{i=1}^3 z_i \frac{\del}{\del z_i} - \frac32 \sum_{a=1}^2 w_a \frac{\del}{\del w_a} \in \PV^{1,0}(\CC^5) \otimes \Omega^0(\RR)  .
\] 
Notice that these vector fields are divergence-free and restrict to the ordinary dilation along $w=0$. 
\item 
The bosonic subalgebra of $\lie{sl}(4)$ describing special conformal transformations on $\CC^3$ is mapped to the elements 
\[
z_j \left(\sum_{i=1}^3 z_i \frac{\del}{\del z_i} - 2 \sum_{a=1}^2 w_a \frac{\del}{\del w_a} \right) \in \PV^{1,0}(\CC^5) \otimes \Omega^0(\RR) .
\] 
Notice that these vector fields are divergence-free and restrict to the ordinary special conformal transformations along $w=0$. 
\item 
The bosonic summand $\lie{sl}(2)$ is mapped to the triple
\[
w_1 \frac{\del}{\del w_2}, w_2 \frac{\del}{\del w_1}, \frac12 \left(w_1 \frac{\del}{\del w_1} - w_2 \frac{\del}{\del w_2}\right) \in \PV^{1,0}(\CC^5) \otimes \Omega^0(\RR) .
\]
\end{itemize}

The odd part of the algebra $\lie{osp}(6|1)$ is $\wedge^4 W \otimes R$ where $W$ is the fundamental $\lie{sl}(4)$ representation and $R$ is the fundamental $\lie{sl}(2)$ representation. 
It is natural to split $W = L \oplus \CC$ with $L = \CC^3$ the fundamental $\lie{sl}(3) \subset \lie{sl}(4)$ representation. 
Then the odd part decomposes as
\[
L \otimes R \oplus \wedge^2 L \otimes R \cong \CC^3 \otimes \CC^2 \oplus \wedge^2 \CC^3 \otimes \CC .
\]

\begin{itemize} 
\item The summand $L \otimes R$ consists of the remaining 6d superstranlsations. 
It is mapped to the fields 
\[
z_i \d w_a \in \Omega^{1,0}(\CC^5) \otimes \Omega^0(\RR) ,\quad a=1,2, \quad i =1,2,3.
\] 
\item The summand $\wedge^2 L \otimes R$ consists of the remaining 6d superconformal transformations. 
It is mapped to the fields
\[
\frac12 w_a (z_i \d z_j - z_j \d z_i) \in \Omega^{1,0}(\CC^5)\otimes \Omega^0(\RR) , \quad a = 1,2, \quad k = 1,2,3. 
\]
\end{itemize}

\begin{lem}
These assignments define an embedding of $\lie{osp}(6|1)$ into the linearized BRST cohomology of the fields of the 11-dimensional theory on $\CC^5 \times \RR$. 
Equivalently, it defines an embedding
\[
i_{M5} \colon \lie{osp}(6|1) \hookrightarrow E(5,10) .
\]
\end{lem} 

\begin{proof}
To explicitly describe the embedding into $E(5,10)$ we simply apply the de Rham differential to the last two formulas above.
Recall, we are using the holomorphic coordinates $(z_1,z_2,z_3, w_1,w_2)$ on $\CC^5$ where $z_i$ are the holomorphic coordinates along the $M5$ brane. 
\begin{itemize}
\item 
The fermionic summand $L \otimes R$ embeds into closed two-forms as
\[
\d z_i \wedge \d w_a, \quad i=1,2,3, \quad a=1,2. 
\] 
\item 
The fermionic summand $\wedge^2 L \otimes R$ embeds into closed two-forms as
\[
w_a \d z_i  \wedge \d z_j + \frac12 \d w_a \wedge (z_i \d z_j - z_j \d z_i) , \quad i,j=1,2,3, \quad a=1,2. 
\] 
\end{itemize}
\end{proof}
\parsec[]

Next, we turn on a nontrivial unit of flux $N \ne 0$. 
Since not all of the fields we wrote down above commute with the flux $N F$, they are not compatible with the total differential $\delta^{(1)} + [N F, -]$ acting on the fields supported on $\CC^5 \times \RR \setminus \{w_1=w_2=t=0\}$. 
Nevertheless, we have the following.

\begin{prop}
\label{prop:brads7}
There exists $N$-dependent corrections to the embedding $i_{M5}$ which is compatible with the modified BRST differential $\delta^{(1)} + [N F_{M5},-]$. 
Furthermore, these order $N$ corrections define an embedding of $\lie{osp}(6|1)$ inside the cohomology of the fields of 11-dimensional theory on $\CC^5 \times \RR \setminus \CC \times \RR$ with respect to the differential $\delta^{(1)} + [N  F_{M5},-]$.
\end{prop}

\parsec[s:thfcohomology]

The proof of the above proposition follows from another indirect cohomological argument. 
Before getting to the proof, we introduce the relevant cohomology. 

The 11-dimensional theory is built from fields which live in the following tensor product of complexes 
\[
\Omega^{0,\bu}(\CC^5) \otimes \Omega^\bu(\RR).
\]
Precisely, this is where the $\beta,\nu$ fields live. 
The $\mu,\gamma$ fields live in versions of this complex where we take Dolbeault forms with coefficients in the holomorphic tangent and cotangent bundles, respectively. 

Another way to think about this complex is to first consider the full de Rham complex $\Omega^\bu(\CC^5 \times \RR)$, which includes both holomorphic and anti-holomorphic forms in the $\CC^5$ direction. 
The dg algebra of all de Rham forms on $\CC^5 \times \RR$ has an ideal generated by the holomorphic one forms $\{\d z_i\}_{i=1,\ldots,5}$.
There is an isomorphism of dg algebras
\[
\Omega^{0,\bu}(\CC^5) \otimes \Omega^\bu(\RR) \cong \Omega^\bu(\CC^5 \times \RR) \, / \, (\d z_1,\ldots, \d z_5) .
\]
The advantage of this presentation is that we can define a complex associated to more general manifolds that are not products of a complex manifold with a smooth manifold.\footnote{More generally, we are describing the cohomology of a manifold equipped with a topological holomorphic foliation.}

For the M5 brane it was convenient to relabel the holomorphic coordinates on $\CC^5$ by $z_1,z_2,z_3,w_1,w_2$. 
At the twisted level, the geometry arising from backreacting $M5$ branes is based on the manifold 
\[
\CC^5 \times \RR \setminus \CC^3 \cong \CC_z^3 \times (\CC^2_w \times \RR \setminus 0) .
\]
The $\beta,\nu$ fields of 11-dimensional theory on this submanifold of $\CC^5 \times \RR$ live in the complex 
\[
\Omega^\bu\bigg(\CC^5 \times \RR \setminus \CC^3\bigg) \, / \, (\d z_1,\d z_2,\d z_3, \d w_1, \d w_2)  .
\]
The $\mu,\gamma$ fields live in similar complexes where we introduce a (trivial) vector bundle on $\CC^5 \times \RR \setminus \CC^3$. 

Since the $\CC^3$ wraps $w_1=w_2=t=0$ we can apply a version of the K\"unneth formula to identify this complex with 
\[
\Omega^{0,\bu}(\CC^3_z) \otimes \bigg( \Omega^\bu\left(\CC^2_w \times \RR \setminus 0 \right) \, / \, (\d w_1, \d w_2) \bigg).
\]

The cohomology of the Dolbeault complex of $\CC^3_z$ is easy to compute. 
The cohomology of the bit in parentheses is concentrated in degrees zero and two. 
In degree zero, there is a dense embedding
\[
\CC[w_1,w_2] \hookrightarrow H^0 \bigg( \Omega^\bu\left(\CC^2_w \times \RR \setminus 0 \right) \, / \, (\d w_1, \d w_2) \bigg)
\]
In degree two, there is a dense embedding
\[
w_{1}^{-1} w_2^{-1} \CC[w_1,w_2] \hookrightarrow H^2 \bigg( \Omega^\bu\left(\CC^2_w \times \RR \setminus 0 \right) \, / \, (\d w_1, \d w_2) \bigg).
\]

It will be useful to explain this last embedding in more detail. 
Consider the homogenous element $w_1^{-1} w_2^{-1}$. 
This represents the class of the Dolbeault-de Rham $2$-form
\[
\frac{\wbar_1 \d \wbar_2 \wedge \d t - \wbar_2 \d \wbar_1 \wedge \d t + t \d \wbar_1 \wedge \d \wbar_2}{(\|w\|^2 + t^2)^{5/2}} .
\]
Notice that if we wedge with the volume form $\d w_1 \d w_2$ this is the unit $N=1$ flux introduced in Lemma \ref{lem:ads7flux}. 
The homogenous element $w_1^{-n-1} w_2^{-m-1}$ represents the class of the holomorphic derivatives $\partial_{w_1}^n \partial_{w_2}^{m}$ applied to this $2$-form. 

Observe that when restricted to $\CC^5 \times \RR \setminus \CC^3$ the holomorphic tangent bundle along $\CC^5$ is still trivializable. 

\parsec[]

Let's turn to the proof of Proposition \ref{prop:brads7}.
We proceed completely analogously to the case of backreacted $M2$ branes as in the proof of Proposition \ref{prop:brads4}. 

\begin{proof}[Proof of Proposition \ref{prop:brads7}]
Let $\cL (\CC^5 \times \RR \setminus \{w_1=w_2=t=0\})$ denote the super $L_\infty$ algebra obtained by parity shifting the fields of the 11-dimensional theory on $\CC^5 \times \RR \setminus \{w_1=w_2=t=0\}$. 

There is a spectral sequence which converges to the cohomology of the fields with respect to the deformed linear BRST differential $\delta^{(1)} + [N F_{M5},-]$ whose first page
is the cohomology with respect to the original linearized BRST differential $\delta^{(1)}$. 
Recall that the linearized BRST differential decomposes as
\[
\delta^{(1)} = \dbar + \d_{\RR} + \div |_{\mu \to \nu} + \del |_{\beta \to \gamma}  .
\]
To compute this page, we use an auxiliary spectral sequence which simply filters by the holomorphic form and polyvector field type. 
This first page of this auxiliary spectral sequence is simply given by the cohomology of the fields supported on 
\[
\CC^5 \times \RR \setminus \{w_1=w_2=t=0\} \cong \CC_z^3 \times (\CC^2_w \times \RR \setminus 0)
\]
with respect to $\dbar + \d_{\RR}$. 

To compute this cohomology we follow the discussion in \S \ref{s:thfcohomology}.
Just as in the case of the $M2$ brane, we see that the $\dbar + \d_{\RR}$ cohomology is (up to completions) is the direct sum of the cohomology on flat space $H^\bu(\cL(\CC^5 \times \RR), \dbar)$ with
\begin{equation}
  \label{eqn:ads7ss2} 
  \begin{tikzcd}[row sep = 1 ex]
    + & - \\ \hline
w_1^{-1} w_2^{-1} \CC[w_1^{-1}, w_2^{-1}][z_1,z_2,z_3] \{\partial_{w_i}\}  \ar[r, dotted, "\div"] & w_1^{-1} w_2^{-1} \CC[w_1^{-1}, w_2^{-1}] [z_1,z_2,z_3] \\
w_1^{-1} w_2^{-1} \CC[w_1^{-1}, w_2^{-1}] [z_1,z_2,z_3] \{\del_{z_i}\} \ar[ur, dotted, "\div"'] \\
w_1^{-1} w_2^{-1} \CC[w_1^{-1}, w_2^{-1}] [z_1,z_2,z_3] \ar[r, dotted, "\del"] \ar[dr, dotted, "\del"'] & w_1^{-1} w_2^{-1} \CC[w_1^{-1}, w_2^{-1}][z_1,z_2,z_3] \{\d z_i\} \\ & w_1^{-1} w_2^{-1} \CC[w_1^{-1}, w_2^{-1}][z_1,z_2,z_3] \{\d w_i\} .
\end{tikzcd}
\end{equation}

Recall that the flux $F$ was defined as the image under $\del$ of some $\gamma$-type field. 
Therefore, the class $[F]$ does not live inside this page of the spectral sequence, but the operator $[[F], -]$ does act on this page nevertheless. 
For instance, if $f^i(z,w) \d z_i$ is a one-form living in $H^0(\CC^5, \Omega^1) \otimes H^0(\RR)$, then
\[
[ [F] , f^i (z,w) \d z_i ] = \ep_{ijk} w_1^{-1} w_2^{-1} \partial_{z_j} f^i(z,w) \del_{z_k} 
\]
which is an element in 
\[
\CC[w_1^{-1}, w_2^{-1}][z_1,z_2,z_3] \{\del_{z_i}\} \subset H^0(\CC^3, \T) \otimes H^2 \big(\Omega^\bu(\CC^2 \times \RR \setminus 0) / (\d w_1 , \d w_2) \big) .
\]

The first page of the spectral sequence converging to the cohomology with respect to $\delta^{(1)} + [N F, -]$ is given by the cohomology of the global symmetry algebra on $\CC^5 \times \RR$, which we computed in \S \ref{sec:global}, plus the cohomology with respect to dotted line operators in \eqref{eqn:ads7ss2}. 

The next page of the spectral sequence is given by computing the cohomology with respect to the operator $[N F,-]$. 
This operator maps Dolbeault-de Rham degree zero elements to Dolbeault-de Rham degree two elements. 
For degree reasons, there are no further differentials and the spectral sequence collapses after the second page. 

The embedding of $\lie{osp}(6|1)$ for $N=0$ lives in the kernel of the original BRST operator $\delta^{(1)}$. 
To see that it this embedding can be lifted to the full cohomology we need to check that any element in the image of the original embedding is annihilated by $\big[ N [F] , - \big]$. 
This is a direct calculation. 
For instance, recall that an element in the image of the odd summand $\wedge^2 L \otimes R = \wedge^2 \CC^3 \otimes \CC^2$ (which corresponds to a superconformal transformation) is of the form $w_a (z_i \d z_j - z_j \d z_i)$, $a=1,2, i,j=1,2,3$. 
We have
\[
\big[[F] , w_a (z_i \d z_j - z_j \d z_i)\big] = 2 \ep_{ijk} (w_1^{-1} w_2^{-1}) \cdot w_a \del_{z_k} = 0
\]
since the class $w_1^{-1} w_2^{-1}$ is in the kernel of the operator given by multiplication by $w_a$ for $a=1,2$.
Verifying that the remaining elements in the image of $i_{M5}$ are in the kernel of $\big[ [F], -\big]$ is similar.
This completes the proof.
\end{proof}



%%\documentclass[11pt]{amsart}
%
%%\usepackage{../macros-master}
%\usepackage{macros-fivebrane}
%
%\begin{document}

\section{Twisted supergravity states}
\label{sec:states}

The first entry of the AdS/CFT dictionary in traditional treatments is a matching between \textit{supergravity states} and local operators in the CFT. 
The goal of this section is to provide constructions of spaces of twisted supergravity states in our eleven-dimensional model, via geometric quantization. The state spaces on ${\rm AdS}_{7}\times S^{4}$ and ${\rm AdS}_{4}\times S^{7}$ have a remarkable property---they are naturally modules for certain infinite-dimensional exceptional super Lie algebras. We conclude the section by computing characters for these modules and comparing them with large $N$ indices for fivebranes and membranes in the literature.

Before proceeding with the construction, let us first give some feel for the situation we hope to describe. Suppose we consider a gravitational theory on $AdS_{d+1}\times S^{d^{\prime}}$, which we compactify to view as a theory on $AdS_{d+1}$ with all Kaluza-Klein harmonics included. Let $M^{d}$ denote the conformal boundary of $AdS_{d+1}$. A supergravity state is traditionally defined to be a solution to linearized equations of motion with a given boundary value \cite{}. Typically, this definition is made in situations where the relevant boundary value problem has a unique solution, in which case one may label states by the corresponding boundary values. Moreover, one may think of such boundary values as arising from modifications of a vacuum boundary condition at a point.


\subsection{Twisted Backreactions}
We begin by describing the relevant backgrounds. In eleven-dimensional supergravity, the $AdS_7 \times S^4$ and $AdS_{4}\times S^{7}$ backgrounds are obtained by backreacting a number of fivebranes and membranes respectively in flat space \cite{Maldacena:1997re,WittenAdS}.
In \cite{RSW} we gave descriptions of twisted versions of these backgrounds. We will recall this construction, adapted to a slightly more global situation than is considered in \cite{RSW}.

We will consider the eleven-dimensional theory on eleven-manifolds that arise as total spaces of vector bundles. Placing the theory in the backreacted geometry is a 3-step procedure:

\begin{itemize}
  \item Place the eleven-dimensional theory on the complement of the zero section. To do so, we will wish to describe the complement of the zero-section in a way that facilitates natural operations on holomorphic-topological local $L_{\infty}$-algebras.

  \item Deform the theory on the complement of the zero section by a certain Maurer--Cartan element.
  The Maurer--Cartan element is thought of as the flux sourced by branes wrapping the zero section.
\end{itemize}

\parsec[s:brkevin]
As a way to highlight the key aspects of the construction, we detail the ingredients in the simplified model of Costello's twisted $M$ theory. The relevant local calculation can be found in the appendix of \cite{}; our goal here is to simply identify the salient global features that allow one to reduce to said local calculation.

We consider the theory on $X = \text{Tot} (\R\oplus K_{C})$, with some number of twisted `fivebranes' wrapping the zero section
\[
0 \times C \subset \R \times \T^* C .
\]
Denote by $t$ the real coordinate and by $w$ the fiber coordinate in $\T^* C$. We wish to describe the complement of the zero section $M = X - 0 \times C$.

Note that the bundle $\R\oplus K_{C}$ is equipped with a partially flat connection - this data equips the total space $X$ with the data of a transversely holomorphic foliation (THF) \cite{DuchampKalka}. 

If we choose a fiberwise partially hermitian metric on the bundle $\R \oplus K_C$ we obtain a projection $p: \R \times \T^*C \to \R_{+} \times C$ which combines the fiberwise norm with the natural bundle projection. The restriction $p|M$ equips $M$ with the structure of an $S^{2}$-bundle over $\R_{>0}\times C$. Moreover, the partial flat connection on $\R\oplus K_{C}$ induces a partially flat connection on $M$. As part of this data, each of the fiber spheres is equipped with a complex foliation of rank 1.

Compactification amounts to pushing forward a local $L_{\infty}$-algebra along $p|M$. The result is a theory with infinitely many Kaluza--Klein modes along the fiber spheres. In the holomorphic-topological setting, the Kaluza-Klein modes will be modeled by a variant of Cauchy-Riemann cohomology.

Moreover, including the flux sourced by the brane deforms this structure. The lowest lying Kaluza-Klein modes in the deformed theory are equivalent to 3d Chern-Simons.

For sake of analogy, we think of the resulting deformation as being a twisted version of $AdS_3 \times S^2$. \footnote{It is an interesting question if this corresponds the actual twist of a five--dimensional supersymmetric background of this form.}
We proceed to describe the twisted version of states at the boundary of this version of $AdS$.
We first proceed before turning on the flux sourced by the brane.

The theory admits a natural `vacuum' boundary condition at $r=0$.
In local coordinates, these are fields $\alpha(t,z,w)$ on the complement to the brane which extend to regular functions along the brane.

The `supergravity states' are, by definition, fields which satisfy the linearized equations of motion and satisfy the vacuum boundary condition except at a single point.
The linearized equations of motion are simply $(\d_{dR} + \dbar) \alpha = 0$.
Thus, up to equivalence, all solutions to the linearized equations of motion are constant in the real variable $t$, and holomorphic in $z,w$.

Modifications of the boundary condition at the point~$z = 0$ on the boundary take the form
\[
\alpha = f(w) \delta^{(r)}_{z=0}
\]
where $f$ is some holomorphic function.
Here $\delta^{(r)}_{z=0}$ denotes the $r$th derivative of the $\delta$-function at $z=0$.
It is convenient to parameterize such boundary modifications algebraically by expressions of the form
\[
\alpha_{k,r} = w^k \delta^{(r)}_{z=0} .
\]
Linear combinations of such states form a dense subspace of all possible modifications at the boundary.

The reason that the boundary modifications take this form can be seen by understanding in more explicit terms the vacuum boundary condition.
The phase space at the boundary $C$ can be identified with the following cohomology
\[
\Omega^{0,\bu}(C) \otimes \cA^{0;\bu}(\R \times \C - 0) [1]
\]
where $\cA^{0;\bu}$ denotes the mixed de Rham--Dolbeault cohomology of $\R \times \C - 0$ as a manifold equipped with a transversely holomorphic foliation \cite{DuchampKalka}.
We refer to the section below for a reminder on this geometric structure.

The phase space is equipped with a natural symplectic form given by
\[
\int_C \d z \oint_{S^2} \d w \, \alpha \wedge \alpha' .
\]
There is a natural Lagrangian inside of the phase space which consists of linear combinations of elements $\alpha(z) \otimes f(t,w)$ where $\alpha(z) \in \Omega^{0,\bu}(C)$ and $f(t,w)$ is a smooth function on $\R \times \C - 0$ which extends to zero.
The linearized equations of motion simply say that $\alpha$ is holomorphic, $f$ is independent of $t$ and depends holomorphically on $w$

\parsec[s:brfive]

We now consider the situation of backreacting some number of (twisted) fivebranes in our eleven-dimensional model.
Let $Z$ be a three-fold that the fivebranes wrap.
We also fix a rank 2 holomorphic vector bundle $V\to Z$ such that $\wedge^{2} V \cong K_{Z}$;
this condition ensures that the total space of $V$ is a Calabi-Yau five-fold. In the main body of the paper we will choose $V$ to be the bundle $K_{Z}^{1/2}\otimes \C^{2}$.

Consider the bundle $\R\oplus V$; this bundle has a canonical partially flat connection. We wish to consider our eleven dimensional model on $X = Tot (\R\oplus V)$ which is the total space of the \textit{real} rank five bundle $\R\oplus V$ over $Z$. The partially flat connection on $\R\oplus V$ equips $X$ with a canonical THF structure $F_{X}\subset T_{X}$.

We place a stack of $N$ fivebranes wrapping the zero section in $\R\oplus V$.
Denote the complement of the zero section by
\[
M_V = \text{Tot}(\R\oplus V) - 0(Z).
\]
Notice that in \S \ref{s:Lsugra} we have only defined the sheaf of $L_\infty$ algebras $\cL_{sugra}$ on a product of a smooth one-manifold times a Calabi--Yau five-fold.
The eleven-manifold $M_V$ is not of this form, nevertheless there is a generalization of $\cL_{sugra}$ which one can define using the natural geometric structure present in our situation.

A transversely holomorphic foliation (THF) on a smooth manifold $M$ is an integrable subbundle $F \subset \T_M \otimes \C$ such that $F + \Bar{F} = \T_M \otimes \C$.
We will say that $F$ equips $M$ with the a THF structure.
%Suppose $M$ is a manifold equipped with a THF structure and let $\cF$ be the corresponding foliation of even codimension.
The product $M = S \times X$, where $X$ is a complex manifold and $S$ is a smooth manifold has a natural THF structure with $F$ the restriction of the tangent bundle of $N$ along the projection.
Locally, any THF manifold is split of the form $\R^d \times \C^n$, whose coordinates we will denote by $(x_i ;  z_j)$.
The bundle $F$ is locally spanned by the vector fields $\partial / \partial x_i$'s and $\del/\del \zbar_j$'s.
(Notice that when $F \cap \Bar{F} = 0$ we are just describing an ordinary complex structure on $M$.)

Any submanifold of a THF manifold is itself a THF manifold.
We are most interested in the submanifold $M_V \subset {\rm Tot}(\R \oplus V) = \R \times X$ where we equip $\R \times X$ with its standard split THF structure.

We have expressed the fields of the eleven-dimensional theory in terms of a mixed type of de Rham and Dolbeault cohomology.
Let us focus on the fields $\beta,\gamma$ which on $\R \times X$ combine to form the complex
\beqn\label{eqn:drdol}
\Omega^{\bu}(\R) \otimes \Omega^{0,\bu}(X) \xto{1 \otimes \del} \Omega^{\bu}(\R) \otimes \Omega^{1,\bu}(X) .
\eeqn
As usual, we leave the $\d_{dR}$ and $\dbar$ operators implicit.
More generally, there is a natural cohomology associated to a THF structure.
Suppose $(M,F)$ is a THF structure and
denote by $Q$ the (complex) quotient bundle $\T_\C M / F$.
For each $p,q$ denote by $\cA^{p;q}$ smooth sections of the bundle $\wedge^p Q^\vee \otimes \wedge^q F^\vee$.
The derivative along the leaves of the foliation defined by $V$ defines a map
\[
\thfd \colon \cA^{p;q} \to \cA^{p;q+1}  .
\]
By integrability one has $\thfd^2 = \thfd \circ \thfd = 0$ and so $\thfd$ equips $\cA^{p;\bu} = \oplus_q \cA^{p;q}[-q]$ with the structure of a cochain complex for each $p$.
Locally in a split THF structure the operator $D$ is of the form $\d_{dR} + \dbar$ where $\d_{dR}$ is the de Rham differential along $\R^d$ and $\dbar$ is the Dolbeault operator along $\C^n$.
There is also an analog of the holomorphic $\del$ operator which takes the form $\thfdel \colon \cA^{p;q} \to \cA^{p+1;q}$.
The obvious exterior product $\cA^{p;q} \times \cA^{r;s} \to \cA^{p+r;q+s}$ further endows
\[
\left(\cA^{\bu;\bu} (M), \thfd + \thfdel\right) = \left(\oplus_p \cA^{p;\bu}[-p] , \thfd + \thfdel \right)
\]
with the structure of a bigraded commutative dg algebra.
This complex is simply isomorphic to the de Rham complex of $M$, but this presentation lends itself to more interesting quotient complexes.
For example, the forms of type $(p,\bu)$ with $p \geq 2$ form an ideal inside of this dg algebra; hence we get a quotient dg algebra
\beqn\label{thfcoh1}
\left(\cA^{\leq 1;\bu}(M), \thfd + \thfdel\right) = \quad \cA^{0;\bu} \xto{\thfdel} \cA^{1;\bu} .
\eeqn
We leave the $\thfd$ operator implicit in the presentation on the right hand side.
When $M = M_V$, it is this complex that is the THF generalization of the truncated de Rham--Dolbeault complex in \eqref{eqn:drdol}---it is easy to see that it agrees with this complex in the case of a split THF manifold.
There is a similar THF description for the fields $\mu,\nu$ in the eleven-dimensional theory.

%Note that the eleven-manifold $\R \times X$ is equipped with a natural transverseley--holomorphic foliation (THF)---the complexified tangent bundle decomposes as $T_{\R}\oplus T_{Z}\oplus \Bar{T}_{Z}$.
With this THF cohomological description of the eleven-dimensional theory in place we proceed to describe the boundary condition obtained by removing the location of the branes.
We may choose fiber coordinates of the bundle $t, w_{1}, w_{2}$ of $\R \oplus V$ over $Z$ and a fiberwise partially hermitian metric.
Explicitly, the corresponding norm defines a map
\begin{align*}
 h \colon  X & \to \R_{+} \\
  (t, w_{i}, \bar{w_{i}}, p)& \mapsto t^{2} + |w_{1}|^{2}+|w_{2}|^{2}
\end{align*}
Letting $\pi \colon X \to Z$ be the natural projection, we obtain the $S^{4}$ bundle
\[
p \define (h,\pi) \colon \R \times X \to \R_{+}\times Z
\]
which restricts to an $S^4$ bundle $p|M \colon M \to \R_{>0} \times Z$.
These embeddings and projections fit inside of the following commutative diagram
\[
\begin{tikzcd}
M \ar[d,"p|M"'] \ar[r,hook] & X \ar[d,"p"] & \ar[l,hook',"0"'] Z \ar[d,"="] \\
\R_{>0} \times Z \ar[r,hook] & \R_{+} \times Z & \ar[l,hook',"0 \times \id"] Z.
\end{tikzcd}
\]
The inclusions on the left are the natural embeddings.
The top right inclusion is the zero section of ${\rm Tot}(\R \oplus V) = X$ and the bottom right inclusion is the embedding at radius $r = 0$.

As we just elaborated, the eleven-dimensional theory is defined on the THF manifold $M$---in the BV formalism this is encoded, in part, by the sheaf of $L_\infty$ algebras $\cL_{sugra}$ on $M$.
Compactification of this theory along the $S^4$ link corresponds to pushing forward this sheaf along $p|M$.
The resulting sheaf of $L_\infty$ algebras $(p|M)_*\cL_{sugra}$ describes, in the BV formalism, the compactified theory on the seven-manifold $\R_{>0} \times Z$.

The theory on $M$ extends to a theory on the manifold obtained by filling in the zero section of $\R \times V$; in other words, we know that the theory is defined on the entire space $\R \times X$.
This means that there is a natural way to extend the theory on $\R_{>0} \times Z$ to the seven-manifold with boundary $\R_{+} \times Z$.
The restriction of this theory to the six-dimensional boundary plays the most important role for us.

\brian{trying to incorporate below}
%To do this we will make use of the natural foliated geometric structures which we have around.

Recall that we have a THF structure on $X$ induced from a partially flat conneciton on $\R\oplus V$; this is codified by saying that there is a splitting of the exact sequence

\[
0 \to \ker \to F_{X}\to \pi^{*}T^{1,0}_{Z}\to 0
.\]

Both $F_{X}$ and $\pi^{*}T^{1,0}_{Z}$ are involutive, and the flatness of the connection implies that that the splitting preserves the lie brackets on sections.

Consider the relevant tangent sequence of the map


The fibers of the composition $V_{M}\to M\to \R_{+}\times Z$ are copies of the tangent bundle of $S^{4}$, and the corresponding fibers of $V_{M}$ are subbundles of $TS^{4}$ that equip the fiber 4-spheres with a generalized Cauchy-Riemann structure. \surya{CITE}

 The underlying sheaf of cochain complexes is given by

\[
\Omega^{\bullet}(\R_{+})\otimes \left ( \begin{tikzcd}
\ul{\rm even} & \ul{\rm odd} \\
\PV^{1,\bu}(Z) \ar[r, "\del_{\Omega}"] & \PV^{0,\bu}(Z)\\
\Omega^{0,\bu}(Z) & \Omega^{1,\bu}(Z) \ar[l, "\del"]
\end{tikzcd}
 \right ) \otimes CR (S^{4}).
\]


Here $CR (S^{4})$ denotes the cohomology of the tangential Cauchy-Riemann complex of $S^{4}$ \surya{CITE}, equipped with the above Cauchy-Riemann structure. Its computation is facilitated by the following lemma:

\begin{lem}
  Let $\R^{d}\times \C^{n}$ be an affine THF manifold, and choose a partial hermitian metric. Let $S^{d+2n-1}$ denote the corresponding unit sphere, equipped with its standard generalized Cauchy-Riemann structure. Then there is a quasi-isomorphism

  \[CR (S^{n+2d-1})\cong \cA^{\bu;\bu}\left ( (\R^{d}\times \C^{n})\setminus 0 \right )\]

  where the right-hand-side denotes the Dolbeault-deRham complex.
\end{lem}

The cohomology of the Dolbeault-deRham complex of $\R\times \C^{2}$ is easy to describe.


It was argued in \cite{RSW} that to leading order the coupling of a stack of twisted fivebranes to the eleven-dimensional theory is given by the nonlocal interaction
\beqn\label{eqn:br1}
I_{M5} = N\int_{Z} \div^{-1}\mu \vee \Omega +\cdots
\eeqn
where $\mu \in \Omega^0 (\R) \hotimes \PV^{1,3}(X)$ is a component of a field in the eleven-dimensional theory which satisfies $\div \mu = 0$.

\parsec
Let $C$ be a curve, and let $V\to C$ be a rank 4-holomorphic vector bundle over $C$ such that $\wedge^{4} V = K_{C}$. This condition again ensures that $X = {\rm Tot} V$ is a Calabi-Yau five-fold - in the main body of the paper, we will take $V = K^{1/2}_{C}\otimes \C^{4}$. Abusively letting $V$ also denote its pullback along the canonical projection $\R\times C \to C$, we may view $\R\times X$ as the total space of $V$ on $\R\times C$. As before we will consider wrapping a stack of $N$ membranes along the zero section.

Since $V$ is a complex vector bundle, we may choose a fiberwise hermitian metric, and as before, we may view $\R\times X \setminus \R\times C$ as an $S^{7}$-bundle over $\R_{>0}\times \R\times C$.



%\parsec[s:sugraops]
%
%By the usual methods of the BV formalism the action functional $S_{sugra}$ described above endows the parity shift of the fields $\cL_{sugra} = \Pi \cF_{sugra}$ with the structure of a holomorphic-topological local $\Z/2$ graded $L_\infty$ algebra. 
%
%On $\C^5 \times \R$ we can describe this super Lie algebra structure explicitly. 
%First, by the Dolbeault and de Rham Poincar\'e lemmas it is easy that the even part of the super Lie algebra $\cL(\C^5 \times \R)$ is equivalent to a one-dimensional central summand $\C$ plus the Lie algebra of divergence-free vector fields on $\C^5$:
%\[
%\Vect_0 (\C^5) = \{X \in \Vect(\C^5) \; | \; \div X = 0\} .
%\]
%The odd part of the super Lie algebra $\cL(\C^5 \times \R)$ is equivalent to the space of holomorphic one-forms on $\C^5$ modulo exact one-forms
%\[
%\Omega^{1,hol}(\C^5) / {\rm Im}(\del) 
%\]
%which is, of course, equivalent to the space of closed holomorphic two-forms $\Omega^{2,hol}_{cl}(\C^5)$. 
%
%\begin{thm}[\cite{RSW}[Theorem 2.1]]
%The Taylor expansion map determines a map of $\Z/2$ graded $L_\infty$ algebras
%\[
%j_\infty \colon \cL_{sugra}(\C^5 \times \R) \to L_{sugra} .
%\]
%Furthermore, $L_{sugra}$ is equivalent as a $\Z/2$ graded $L_\infty$ algebra to $\Hat{E(5|10)}$. 
%\end{thm} 
%
%As an immediate corollary of this result we obtain by Lemma \ref{lem:localops} the following.
%
%\begin{cor}
%\label{cor:sugraops}
%Let $\Obs_{sugra}$ be the factorization algebra on $\C^5 \times \R$ of classical observables of the minimal twist of eleven-dimensional supergravity.
%There is a quasi-isomorphism of commutative dg algebras
%\[
%\Obs_{sugra} (0) \simeq \clie^\bu \left( \Hat{E(5|10)} \right) .
%\]
%\end{cor}

\subsection{Global symmetry for twisted $AdS$}
\label{s:global1}

After complexification, the~six-dimensional and three-dimensional superconformal algebras are isomorphic to $\lie{osp}(8|4)$.
The even part of this algebra is $\lie{so}(8) \times \lie{sp}(4)$.
This algebra contains the six-dimensional $\cN=(2,0)$ supersymmetry algebra whose odd part is four copies of $S^{6d}_+$, the positive irreducible complex spin representation of $\lie{so}(6)$.
It also contains the three-dimensional $\cN=8$ supersymmetry algebra whose odd part is eight copies of $S^{3d}$, the irreducible complex spin representation of $\lie{so}(3)$. 

In the six-dimensional case, the holomorphic supercharge is a supertranslation 
\[
Q \in \Pi S^{6d}_+ \otimes \C^4 \subset \lie{osp}(8|4)
\]
which is characterized (up to equivalence) by the properties that $Q^2 = 0$ and that its image
\[
{\rm Im}\left(Q|_{\Pi S_+ \otimes \C^4} \right) \subset \R^6 \otimes_\R \C \cong \C^6
\]
is three-dimensional (spanned by the anti-holomorphic translations). 
The supercharge $Q$ acts on $\lie{osp}(8|4)$ by commutator and the resulting cohomology will automatically act on the holomorphic twist of any six-dimensional superconformal field theory. 
This cohomology can readily be identified with the subalgebra $\lie{osp}(6|2)$, see \cite{SWe36}. 

Similarly, the minimal twisting supercharge in the three-dimensional $\cN=8$ supersymmetry algebra is an element $Q \in \Pi S^{3d} \otimes \C^8$ which is characterized (up to equivalence) by the property that $Q^2 = 0$ and that the image of $[Q,-]$ is two-dimensional. 
The cohomology of $\lie{osp}(8|4)$ with respect to this supercharge is also isomorphic to~$\lie{osp}(6|2)$. 

In the untwisted situation, the symmetry of solutions to eleven-dimensional supergravity in the presence of fivebranes wrapping a six-dimensional affine subspace is exactly the superconformal algebra $\lie{osp}(8|4)$ (after complexification). 
Geometrically, this background is $AdS_7 \times S^4$. 
Similarly, for backreacting membranes the background is $AdS_4 \times S^7$. 
In \cite{RSW} we have proposed a twisted analog of the $AdS$ background and have shown that solutions to equations of motion of our eleven-dimensional theory in the presence of twisted fivebranes and membranes contains the symmetry algebra $\lie{osp}(6|2)$---which is precisely the twists of the superconformal algebras we just discussed.

We will count the single-particle gravitational states in our eleven-dimensional model for the geometry resulting from backreacting twisted fivebranes and membranes.
As recalled above, such states have a symmetry by the twisted superconformal algebra $\lie{osp}(6|2)$.
We will enumerate states via choosing a Cartan in the bosonic subalgebra of the twisted superconformal algebra. Let us recall how the bosonic subalgebra embeds as symmetries of the eleven-dimensional theory in this twisted background.
This manifests as an embedding of this bosonic subalgebra into the ghosts of the eleven-dimensional theory. 
The embedding is distinct for fivebranes and membranes.
We turn first to the fivebrane case. 

\subsection{Supergravity states for twisted $AdS_7$}

For convenience we choose coordinates on the eleven manifold as
\[
\R \times \C^5 = \R_t \times \C^2_w \times \C_z^3 
\]
with $z = (z_i), i=1,2,3$ and $w = (w_a), a=1,2$.
The stack of fivebranes wrap $w_1=w_2=t=0$. 
Important for us is to recall that part of the ghost system for our eleven-dimensional theory consists of divergence-free vector fields on $\C^5$ which are locally constant along $\R$. 

\begin{itemize}
\item
The subalgebra $\lie{sl}(3)$ embeds as vector fields
\beqn
\sum_{ij} A_{ij} z_i \frac{\del}{\del z_j} \in \PV^{1,0}(\C^5)\otimes \Omega^0(\R) , \quad (A_{ij}) \in \lie{sl}(3) .
\eeqn
By definition, these vector fields are automatically divergence-free.

\item
        The generator of the subalgebra $\lie{gl}(1)$ is mapped to the element
        \beqn
        Y = \sum_{i=1}^3 z_i\frac{\del}{\del z_i} - \frac 32\sum_{a=1}^2 w_a\frac{\del}{del w_a}\in \PV^{1,0}(\C^5)\otimes \Omega^0 (\R).
        \eeqn
    Notice that this vector field is divergence-free and restricts to the Euler vector field along $t=w_{a} = 0$.
\item 
The subalgebra $\lie{sl}(2)$ ($R$-symmetry) is mapped to the triple
\beqn
 w_1 \frac{\del}{\del w_2}, w_2 \frac{\del}{\del 1}, \frac{1}{2}\left (w_1\frac{\del}{\del w_1}-w_2\frac{\del}{\del w_2}) \in \PV^{1,0}(\C^5) \otimes \Omega^0(\R) .
\eeqn
\end{itemize}

%\begin{rmk}
%In the classification of simple super Lie algebras, Kac makes use of a weight grading $\oplus_{j \geq -2} \fg_j$ of the exceptional Lie algebra $E(3|6)$ for which the finite-dimensional subalgebra above is the weight zero piece
%\cite{KacClass}.
%We will make use of this grading in \S \ref{s:kr}.
%\end{rmk}

The dimension of a Cartan subalgebra of $\lie{sl}(3) \times \lie{sl}(2) \times \lie{gl}(1)$ is four and accordingly, the equivariant character we study has four fugacities.
We choose these explicitly as follows:
\begin{itemize}
  \item $t_{1}, t_{2}$ denote generators for the Cartan of $\lie{sl}(3)$ which is generated by the vector fields
  \beqn
  h_1 = z_1 \frac{\del}{\del {z_1}} - z_2 \frac{\del}{\del{z_2}} , \quad h_2 = z_2 \frac{\del}{\del{z_2}} - z_3 \frac{\del}{\del{z_3}}.
  \eeqn
   \item $q$ denotes a generator for the Cartan of the~$\lie{gl}(1)$ which is generated by the element $Y$ from equation~$\eqref{eqn:Y}$. 
  \item $r$ denotes a generator for the Cartan of a $\lie{sl}(2)$ which is generated by the element 
  \beqn
  h = \frac12 \left(w_1 \frac{\del}{\del w_1} - w_2 \frac{\del}{\del w_2}\right) .
  \eeqn
\end{itemize}

The twisted supergravity states $\cH_{sugra}^{6d}$ form a representation for $\lie{osp}(6|2)$. 
The weights of twisted supergravity states with respect to the generators of the Cartan subalgebra above are completely determined by the weights of the holomorphic coordinates on $\C^2_w \times \C^3_z$.
These are summarized in table \ref{tbl:sugraM5}.

\begin{table}
\begin{center}
\begin{tabular}{c c c c c c}
  & $z_{1}$ & $z_{2}$ & $z_{3}$ & $w_{1}$ & $w_{2}$ \\
  \hline
  $t_{1}$ & $1$ & 0 & $-1$ & 0 & 0 \\
  $t_{2}$ & 0 & 1 & $-1$ & 0 & 0 \\
  $r$ & 0 & 0 & 0 & 1 & $-1$ \\
  $q$ & $-1$ & $-1$ & $-1$ & $\frac{3}{2}$ & $\frac{3}{2}$
\end{tabular}
\caption{Fugacities for the fields of the holomorphic twist of eleven-dimensional supergravity for the geometry $\R \times \C^5 \setminus \C^3$.}
\label{tbl:sugraM5}
\end{center}
\end{table}

We enumerate single particle supergravity states via computing the super trace of the operator $q^Y t_1^{h_1} t_2^{h_2} r^h$ acting on $\cH^{6d}_{sugra}$:
\beqn
f^{6d}_{sugra}(q,t_1,t_2,r) = \Tr_{\cH_{sugra}^{6d}} (-1)^F q^Y t_1^{h_1} t_2^{h_2} r^h .
\eeqn
The super trace means that there is an extra factor of $(-1)^F$, where $F$ is parity (fermion number), when computing the ordinary trace. 
That is, we compute the expression


\begin{prop}
The single particle index of the space of twisted supergravity states $\cH_{sugra}^{6d}$ is given by the following expression
\beqn
f_{sugra}^{6d} (q, t_{1}, t_{2}, r) = \frac{q^4(t_1^{-1}+t_1t_2^{-1}+t_2)-q^2(t_1+t_1^{-1}t_2+t_2^{-1})+(q^{3/2}-q^{9/2})(r+r^{-1})}{(1-t_{1}^{-1}q)(1-t_{2}q)(1-t_{1}t_{2}^{-1}q)(1-rq^{3/2})(1-r^{-1}q^{3/2})}.
\eeqn
\end{prop}

We record a few specializations of this index which we will remark on further in \S \ref{s:??}.
\parsec 
The specialization of this index $q=r^2, t_2=1$ in \eqref{eqn:special1} yields the plethystic exponential of the following single particle index
\[
f_{sugra}^{6d}(q, t_1, t_2=1, r = q^{1/2}) = \frac{q}{(1-q)^2}
\]

This plethystic exponential yields the Macmahon function, which is the character of the vacuum module of the $W_{1+\infty}$-algebra.

\parsec

The specialization $t_1=t_2=r=1$ yields the single particle index
\[
f_{sugra}^{6d} (q, t_1=t_2=r=1) = \frac{3 q^4 - 3 q^2 + 2 q^{3/2} - 2 q^{9/2}}{(1-q)^3 (1-q^{3/2})^2} .
\]

\parsec The same change of variables in \eqref{eqn:special2} agrees with previously computed indices for single particle states for supergravity on $AdS_{7}\times S^{4}$ \surya{...} \brian{not sure where this was supposed to go?}

\subsection{Supergravity states for twisted $AdS_4$}
For convenience we choose coordinates on the eleven manifold as
\[
\R \times \C^5 = \R_t \times \C^4_w \times \C_z^
\]
with $w = (w_a), a=1,2,3,4$.
The stack of membranes wrap $w_1=w_2=w_{3}=w_{4} = 0$.

\begin{itemize}
\item
The subalgebra $\lie{sl}(4)$ ($R$-symmetry) embeds as vector fields
\beqn
\sum_{ab} A_{ab} w_a \frac{\del}{\del w_b} \in \PV^{1,0}(\C^5)\otimes \Omega^0(\R) , \quad (A_{ab}) \in \lie{sl}(4) .
\eeqn
By definition, these vector fields are automatically divergence-free.

\item
The subalgebra $\lie{sl}(2)$ ($R$-symmetry) is mapped to the triple
\beqn
 \frac{\del}{\del z}, z \frac{\del}{\del z}-\frac{1}{4}\sum_{a=1}^4 w_a\frac{\del}{\del w_a}, \frac12 \left(w_1 \frac{\del}{\del w_1} - w_2 \frac{\del}{\del w_2}\right) \in \PV^{1,0}(\C^5) \otimes \Omega^0(\R) .
\eeqn
\end{itemize}

%\begin{rmk}
%In the classification of simple super Lie algebras, Kac makes use of a weight grading $\oplus_{j \geq -2} \fg_j$ of the exceptional Lie algebra $E(3|6)$ for which the finite-dimensional subalgebra above is the weight zero piece
%\cite{KacClass}.
%We will make use of this grading in \S \ref{s:kr}.
%\end{rmk}
The equivariant character again has four fugacities, which we explicitly choose as follows:
\begin{itemize}
  \item $t_{1}, t_{2}, t_{3}$ denote generators for the Cartan of $\lie{sl}(4)$ which is generated by the vector fields
  \beqn
  h_1 = w_1 \frac{\del}{\del {w_1}} - w_4 \frac{\del}{\del{w_4}} , \quad h_2 = w_2 \frac{\del}{\del{w_2}} - w_4 \frac{\del}{\del{w_4}} , \quad h_3 = w_3\frac{\del}{\del w_3}-w_4\frac{\del}{\del w_4}.
  \eeqn
  \item $q$ denotes a generator for the Cartan of the $\lie{sl}(2)$ which is generated by the vector field
        \beqn
        T =  z \frac{\del}{\del z}-\frac{1}{4}\sum_{a=1}^4 w_a\frac{\del}{\del w_a}
        \eeqn
\end{itemize}


Once again, the twisted spergravity states $\cH_{sugra}^{3d}$ form a representation for $\lie{osp}(6|2)$.
The weights of twisted supergravity states with respect to the generators of the Cartan subalgebra above are completely determined by the weights of the holomorphic coordinates on $\C^4_w \times \C_z$.
These are summarized in table \ref{tbl:sugraM5}.

\begin{table}
\begin{center}
\begin{tabular}{c c c c c c}
  & $z$ & $w_{1}$ & $w_{2}$ & $w_{3}$ & $w_{4}$ \\
  \hline
  $t_{1}$ & 0 & 1 & 0 & 0 & $-1$ \\
  $t_{2}$ & 0 & 0 & 1 & 0 & $-1$ \\
  $t_{3}$ & 0 & 0 & 0 & 1 & $-1$ \\
  $q$ & $-1$ & $\frac 14$ & $\frac 14$ & $\frac{1}{4}$ & $\frac{1}{4}$
\end{tabular}
\caption{Fugacities for the fields of the holomorphic twist of eleven-dimensional supergravity for the geometry $\R \times \C^5 \setminus \R\times \C$.}
\label{tbl:sugraM5}
\end{center}
\end{table}

We enumerate single particle supergravity states via computing the super trace of the operator $q^T t_1^{h_1} t_2^{h_2} r^h$ acting on $\cH^{3d}_{sugra}$:
\beqn
f^{3d}_{sugra}(q,t_1,t_2,t_3) = \Tr_{\cH_{sugra}^{3d}} (-1)^F q^T t_1^{h_1} t_2^{h_2} t_3^{h_3} .
\eeqn
The super trace means that there is an extra factor of $(-1)^F$, where $F$ is parity (fermion number), when computing the ordinary trace.
That is, we compute the expression


\begin{prop}
The single particle index of the space of twisted supergravity states $\cH_{sugra}^{3d}$ is given by the following expression
\beqn
f_{sugra}^{3d} (q, t_{1}, t_{2}, t_3) = \frac{\left(\begin{aligned}
        & -q^{-7/4}(t_{1}^{-1}+t_{2}^{-1}+t_{3}^{-1}+t_{1}t_{2}t_{3}) +q^{1/4}(t_{1}+t_{2}+t_{3}+t_{1}^{-1}t_{2}^{-1}t_{3}^{-1}) \\
        & +q^{1/2}(1-q)(t_{1}t_{2}+t_{1}t_{3}+t_{2}t_{3}+t_{1}^{-1}t_{2}^{-1}+t_{1}^{-1}t_{3}^{-1}+t_{2}^{-1}t_{3}^{-1})\end{aligned}\right)}{(1-q)(1-t_{1}q^{1/4})(1-t_{2}q^{1/4})(1-t_{3}q^{1/4})(1-t_{1}^{-1}t_{2}^{-1}t_{3}^{-1}q^{1/4})}.
\eeqn
\end{prop}

\parsec Upon performing the change of variables

\beqn
q= x^{2} , \quad t_1 = (y_{2}y_{3})^{1/2}/y_1^{1/2} , \quad t_2 = (y_{1}y_3)^{1/2}/ y_2^{1/2} , \quad t_3 = (y_1 y_2)^{1/2}/y_{3}^{1/2}
\eeqn

the result agrees with previously computed indices for single particle states for supergravity on $AdS_{4}\times S^{7}$ \cite{}




\printbibliography

\end{document}



















\appendix 

\section{An alternative description of the 11-dimensional theory} 
In \S \ref{s:dfn} we have described a family of BV theories on $X \times L$ where $X$ is an odd-dimensional Calabi--Yau manifold and $L$ is an odd-dimensional smooth manifold. 
The space of fields \eqref{eq:sympfields} is equipped with an odd symplectic pairing. 


There is a related, alternative description of this formal moduli space which is almost equivalent. 
Consider the resolution of holomorphic closed two-forms 
\beqn\label{eqn:twoform}
\Omega_{cl}^{2,hol}(X) \simeq \Omega^{2,\bu}(X) \xto{\del} \Omega^{3,\bu}(X)[-1] \to \cdots 
\eeqn
where we have left the $\dbar$-differential implicit, as always. 

There is a map from the two-term complex 
\[
\Omega^{0,\bu}(X)[1] \to \Omega^{1,\bu}(X)
\]
to the resolution \eqref{eqn:twoform} defined by the holomorphic de Rham operator $\del$. 
This map is almost a quasi-isomorphism of sheaves; it differs by copy of constant functions $\CC[1]$ in cohomological degree $-1$. 

Similarly, we can tensor with the dg algebra of de Rham forms on $L$ to obtain a map of $\ZZ/2$-graded cochain complexes
\[
\bigg(\Pi \Omega^{0,\bu}(X;L) \xto{\del} \Omega^{1,\bu}(X;L) \bigg) \xto{\del} \bigg(\Omega^{2,\bu}(X;L) \xto{\del} \Pi \Omega^{3,\bu}(X;L) \to \cdots \bigg) .
\]
Again, this map is a quasi-isomorphism up to a copy of $\Pi \CC$.
The left-hand side contains the fields $(\beta, \gamma)$. 
This map has the effect of sending $\beta \mapsto 0$ and $\gamma \mapsto \partial \gamma \in \Omega^{2,\bu}(X;L)$. 

The basic idea is to replace the complex where $(\beta,\gamma)$ live by this resolution of closed two-forms. 
The new space of fields is 
\begin{equation}
  \label{eq:poissfields} 
  \begin{tikzcd}[row sep = 1 ex]
     - & + & - & + \\ \hline
     \PV^{1,\bu}(X; L) \ar[r, "\div"] & \PV^{0,\bu}(X; L) \\
     & \Omega^{2,\bu}(X;L) \ar[r, "\del"] & \Omega^{3,\bu}(X;L) \ar[r] & \cdots 
\end{tikzcd}
\end{equation}

It may seem that this description of the fields of the 11-dimensional theory is not much different than the original formulation---at the level of the free theory they only differ by a copy of constant functions in odd cohomological degree. 
We'd like to point out two main differences:
\begin{itemize}
\item This new description does not have the structure of a BV theory in the usual sense; there is no odd symplectic pairing on the fields. 
Nevertheless, there is still an odd Poisson bracket acting on functionals of the fields which we will describe momentarily. 
For this reason, we will refer to the theory as a ``Poisson BV theory''. 
\item With the existence of the odd Poisson bracket one might be optimistic to formulate the CME for this Poisson BV theory.
However, there is no local interaction which is consistent with the local interaction present in the original BV theory. 
Nevertheless, the (shift of the) fields is equipped with an $L_\infty$ structure which is compatible with the odd Poisson bracket. 
\end{itemize}

\begin{prop}
The complex \eqref{eq:poissfields} is equipped with the structure of an interacting Poisson BV theory.
Let $\cG = \cG(X \times L)$ be the resulting $L_\infty$ algebra...
\end{prop}

Heuristically, we can write the action in the following non-local form
\[
\frac{1}{1-\nu} \mu^2 \eta + (\del^{-1} \eta) \eta^2 .
\]
\brian{are there more terms involving higher forms?}

The first few nonzero brackets are
\begin{align*}
[\mu]_1 & = \dbar \mu + \div \mu \\
[\eta]_1 & = \dbar \eta + \del \eta \\ 
[\mu_1,\mu_2]_2 & = \div (\mu_1 \wedge \mu_2) \\
[\mu, \eta]_2 & = \mu \vee \del \eta \\
[\eta_1,\eta_2]_2 & = \# \Omega^{-1} \vee (\eta_1 \wedge \eta_2)  \\
[\nu, \mu_1,\mu_2]_3 & = \div(\nu \mu_1 \mu_2) \\
[\nu, \mu,\eta]_3 & = \nu \mu \del \eta
\end{align*}
\brian{coefficients}
For $k \geq 2$ the general formula for the $k$-ary bracket is 
\begin{align*}
[\eta_1,\eta_2]_2 & = \# \Omega^{-1} \vee (\eta_1 \wedge \eta_2)  \\
[\nu_1,\nu_2, \ldots, \nu_{k-2}, \mu_1,\mu_2]_{k} & = \# \div(\nu_1 \cdots \nu_{k-2} \mu_1 \wedge \mu_2) \\ 
[\nu_1,\nu_2, \ldots, \nu_{k-2}, \mu,\eta]_{k} & = \# \nu_1 \cdots \nu_{k-2} (\mu \vee \del \eta) .
\end{align*}

\parsec[]
Consider the theory on $\CC^3 \times \RR$. 
There is a quasi-isomorphism of $L_\infty$ Lie algebras $\CC \simeq \cG (\CC^3 \times \RR)$. 

\parsec[]
Consider the theory on $\CC^5 \times \RR$. 

\begin{prop}
There is a quasi-isomorphism of super $L_\infty$ algebras 
\[
E(5,10) \xto{\simeq} \cG(\CC^5 \times \RR) .
\]
This lifts to map of super $L_\infty$ algebras $E(5,10) \to \cL(\CC^5 \times \RR)$ whose kernel is $\CC$.
\end{prop}


\section{BCOV stuff}

\subsection{Relation to BCOV theory}

\parsec[sec:pv] 
\ingmar{a section defining polyvector fields; not sure where this belong}
Equipped with this structure, the complex we have written maps in an obvious way into the complex of polyvector fields on~$X$. Recall that one defines
\deq{
  \PV^{i,j}(X) = \Omega^{0,j}(X, \wedge^i \T_X).
}
\ingmar{Wedges look super fucked up}
The complex $\PV^{\bu,\bu}$ is equipped with two natural differentials: the Dolbeault operator $\dbar$, of $(i,j)$-degree $(0,1)$, and the holomorphic divergence operator $\div$, which carries $(i,j)$-degree $(-1,0)$. Assigning total degree $- i + j$ to $\PV^{i,j}$ thus gives the total polyvector fields the structure of a bicomplex. We will use the shorthand notation $\PV^i = (\PV^{i,\bu}, \dbar )$ for the Dolbeault resolution of holomorphic $i$-polyvector fields. 

\subsubsection{}
It will be convenient for us to cast this dg Lie algebra in a slightly different way.
We construct a different model for the Lie algebra of divergence-free vector fields whose underlying cochain complex is the same as \eqref{eqn:cplx1}. 
The distinguishing feature is that this model is no longer a dg Lie algebra, but an $L_\infty$ algebra. 

The model we use is motivated by the topological string, specifically the description of the closed topological string in terms of Kodaira--Spencer theory \cite{BCOV}.
We recall the requisite background, but refer to \cite{CLbcov1,CLbcov2,CLtypeI} for more details. 

Suppose $X$ be a Calabi--Yau manifold of dimension $d$. 
Let $\PV^{k,\bu}(X)$ denote the Dolbeault complex of the holomorphic vector bundle $\wedge^k \T_X$; we will omit $X$ in this section and write $\PV^{k,\bu}$ for simplicity.
In particular, $\PV^{k,j}$ is the space of smooth sections of the bundle $\wedge^j \T_X^* \otimes \wedge^k \T_X$. 
With this notation, the dg Lie algebra in \eqref{eqn:cplx1} is $\PV^{1,\bu} \oplus \PV^{0,\bu}[-1]$ with differential $\dbar + \div$. 

Introduce a formal parameter $u$ of cohomological degree $+2$ and consider the complex 
\beqn\label{eqn:pv1}
\PV^{\bu,\bu} [[u]] [1]
\eeqn
with differential $\dbar + u \div$ which we will often denote just by $Q_{KS}$ and refer to as the linear BRST differential. 
With our conventions, notice that $u^l \PV^{k,j}$ sits in cohomological degree $2l +k + j - 1$. 

The Schouten--Nijenhuis bracket 
\[
[-,-] \colon \PV^{k,j} \times \PV^{p,q} \to \PV^{p+k-1,j+q} 
\]
extends $u$-linearly to endow \eqref{eqn:pv1} with the structure of a dg Lie algebra.
We refer to this as the Kodaira--Spencer, or strict, dg Lie algebra structure.
We refer to the further cohomological shift 
\[
\cE_{KS} = \PV^{\bu,\bu} [[u]] [2]
\]
as the space of fields of Kodaira--Spencer theory.

Notice that the complex resolving divergence-free vector fields \eqref{eqn:cplx1} sits inside $\cE_{KS} [-1]$ as a sub dg Lie algebra as $\PV^{1,\bu} \oplus u \PV^{0,\bu}[1]$. 

\subsubsection{}

\brian{bv bracket}

\subsubsection{}

Together with the linear BRST differential $Q_{KS}$, the bracket $\{-,-\}_{KS}$ induces the structure of a dg Lie algebra the cohomological shift of the space of local functionals $\oloc(\cE_{KS})[-2d+5]$. 

\begin{thm}[\cite{CLbcov1}]
The BCOV action 
\[
I_{BCOV} \in \oloc \left(\cE_{KS} \right) 
\]
satisfies the classical master equation 
\[
Q_{KS} I_{BCOV} + \frac12 \left\{I_{BCOV}, I_{BCOV}\right\}_{KS} = 0 .
\]
In other words, $I_{BCOV}$ determines a Maurer--Cartan element in the dg Lie algebra $\oloc(\cE_{KS})[-2d+5]$.
\end{thm}

This action induces the square-zero operator 
\[
\delta_{BCOV} = Q_{KS} + \{I_{BCOV}, -\} 
\]
acting on observables of the Kodaira--Spencer fields. 
In other words, it defines a non-linear BRST operator; in turn it
induces an $L_\infty$ structure $\{[-]_k\}_{k =1,2,3,\ldots}$ on the complex \eqref{eqn:pv1} whose linear operation is $[-]_1 = Q_{KS} = \dbar + u\div$. 

This $L_\infty$ structure is clearly not identical as the strict dg Lie algebra structure (it has operations of arbitrary high order). 
Nevertheless, it is equivalent to the strict dg Lie model: there is an $L_\infty$ automorphism which exchanges the two structures.

It is easiest to describe this automorphism at the level of observables.
Use the notation $\Sigma$ for a linear observable of the Kodaira--Spencer fields $\cE_{KS}$. 
Then, the Kodaira--Spencer dg Lie algebra structure induces the non-linear BRST operator $\delta_{KS}$ given by
\[
\delta_{KS} (\Sigma) = Q_{KS} \Sigma + \frac12 [\Sigma,\Sigma] .
\]

The non-linear change of coordinates relating the two structures is defined by
\[
\Psi \colon \Sigma \mapsto \left[u (e^{\Sigma/u} -1)\right]_+
\]
where $[-]_+$ projects onto the non-negative powers of $u$.  

\parsec[]

The complex resolving the (shift of) divergence-free vector fields 
\[
\left(\PV^{1,\bu}[1] \oplus u \PV^{0,\bu}[2] \, , \, \dbar + u \div\right) 
\]
is a subcomplex of $\cE_{KS}$.
If $d = \dim_\CC(X)=3$ then the shifted Poisson bivector $\Pi_{BCOV}$ restricts to a shifted Poisson bivector on this subcomplex. 
In particular, the action $I_{BCOV}$ restricts to a solution to the classical master equation for this subcomplex. 

If $d \ne 3$ then there is the following subcomplex $\til{\cE}_{KS}\subset \cE_{KS}$ defined by:
\beqn\label{eqn:tilks}
\begin{tikzcd}
\ul{-1} & \ul{0} & \cdots & \ul{d-4} & \ul{d-3} & \cdots    \\
\PV^{1,\bu} \ar[r] & u\PV^{0,\bu} & & & & &  \\
& & & \PV^{d-2, \bu} \ar[r] & u \PV^{d-3, \bu} \ar[r] & \cdots &   .
\end{tikzcd}
\eeqn
And one can check that the shifted Poisson bivector restricts to a shifted Poisson bivector on this subcomplex. 
Thus, we have the following.

\begin{prop}
\label{prop:tilbcov}
The BCOV action $I_{BCOV}$ restricts to a solution of the classical master equation on $\til{\cE}_{KS}$ that we denote by $\til{I}_{BCOV}$. 
\end{prop}

\subsubsection{}

Define the action by the group $\CC^\times$ on the complex $\til{\cE}_{KS}$ as follows.
Declare the first line in \eqref{eqn:tilks} is weight zero and the second line is weight $+1$. 
%Declare $\mu, \nu$ have $\CC^\times$ weight zero and $\beta,\gamma$ have $\CC^\times$ weight $+1$. 
The linear BRST operator is clearly weight zero and the shifted Poisson structure $\Pi_{KS}$ is of weight $+1$. 
Thus, the shifted Poisson bracket $\{-,-\}_{KS}$ is of $\CC^\times$ weight $+1$.

\begin{lem}
The restricted action $\til{I}_{BCOV}$ has $\CC^\times$ weight $-1$. 
In particular, the non-linear BRST operator 
\beqn\label{eqn:KSbrst}
\delta_{BCOV} = Q_{KS} + \{\til{I}_{BCOV},-\}_{KS}
\eeqn
is $\CC^\times$ weight zero.
\end{lem}
\begin{proof}
%The map $\Phi \colon \cE_{pot} \to \til{\cE}_{KS}$ preserves $\CC^\times$ weight, so it suffices to prove that $\til{I}_{BCOV}$ is weight $+1$. 
We will make use of two gradings on $\til{\cE}_{KS}$. 
The first is holomorphic polyvector field type, and the second is descendant degree.
The summand $u^l \PV^{k,\bu}$ is polyvector type $k$ and descendant degree $l$. 

We introduce notation in this proof that won't be used later on. 
Let $\alpha$ denote a super field living in the first line of \eqref{eqn:tilks} and let $\beta$ denote a super field living in the second line of \eqref{eqn:tilks}.
If a field of type $\alpha$ has polyvector type $k$ then it has descendant degree $1-k$, $k=0,1$. 
If a field of type $\beta$ has polyvector type $l$ then it has descendant degree $d-2-l$.

A homogenous polynomial degree term in the BCOV action $I_{BCOV}$ is a linear combination of functionals of the form
\[
 \alpha_1 \wedge \cdots \wedge \alpha_m \wedge \beta_1 \wedge \cdots \wedge \beta_n  .
\]
Let $k_i$ be the polyvector field type of $\alpha_i$ and let $l_j$ be the polyvector field type of $\beta_j$. 
In order for this expression to contribute nontrivially to the BCOV action two constraints must hold:
\begin{align*}
\sum_{i=1}^m k_i + \sum_{j=1}^n l_j & = d \\
\sum_{i=1}^m (1-k_i) + \sum_{j=1}^n (d-2-l_j) & = m+n-3 .
\end{align*}
The first constraint ensures that the integrand is of top polyvector degree. 
The second constraint is on the descendant degree, see \brian{ref above}. 

Simplifying these two equations, one finds the condition
\[
m+n-3 = m+(d-2)n - d 
\]
which implies $n=1$, as desired. 
\end{proof}

Consider the cochain complex $\left(\oloc(\til{\cE}_{KS}) \, , \, \delta_{KS} \right)$ of all local functionals equipped with the non-linear BRST differential \eqref{eqn:KSbrst}.  
The $\CC^\times$ weight zero subcomplex is 
\[
\big(\oloc(\til\cE_{KS})^{(0)} \, , \, \delta_{KS} \big) = \big(\oloc(\cE_0) \, , \, \delta_{KS} \big) 
\]
where $\cE_0 = \PV^{1,\bu}[1] \oplus u \PV^{0,\bu}$ is the subcomplex of the fields $\til\cE_{KS}$ comprising the first line of \eqref{eqn:tilks}.

\begin{prop}
\label{prop:Linfty}
The differential $\delta_{KS}$ acting on the weight zero fields induces a local $L_\infty$ algebra structure on the complex of vector bundles 
\[
\cE_0 [-1] = \PV^{1,\bu} \oplus \PV^{0,\bu}[-1]
\]
with $[-]_1 = \dbar + \div$. 
Furthermore, this $L_\infty$ algebra structure is equivalent to the strict dg Lie algebra structure on the resolution of divergence-free vector fields defined in \S \ref{sec:divfree}. 
\end{prop}

%Applying this to the $k$th exterior power of the holomorphic tangent bundle $V = \wedge^k \T_X$, we obtain a resolution of the sheaf of $k$-linear polyvector fields which we denote by $\PV^{k,\bu} (X) = \Omega^{0,\bu}(X, \wedge^k \T_X)$. 

%\subsubsection{}
%
%Consider the following cochain complex $\cE_{pot}$:
%\beqn\label{eqn:E}
%\begin{tikzcd}
%\ul{-1} & \ul{0} & \cdots & \ul{d-5} & \ul{d-4} &   \\
%\PV^{1,\bu} \ar[r,"\div"] & \PV^{0,\bu} & \cdots & & \\
%& & & \Omega^{0, \bu} \ar[r,"\del"] & \Omega^{1, \bu} 
%\end{tikzcd}
%\eeqn
%We will refer to the components of the fields using the notation
%\begin{align*}
%(\mu, \nu) & \in \Pi \PV^{1,\bu}(X)[1] \oplus \PV^{0,\bu}(X) \\
%(\beta, \gamma) & \in \Omega^{0,\bu}(X)[d-5] \oplus \Omega^{1,\bu}(X) [d-4].
%\end{align*}
%The expressions $\mu,\nu,\beta,\gamma$ are still super fields in the sense that they have components in all anti-holomorphic Dolbeault degree. 
%
%Let 
%\[
%\begin{array}{rccc}
% \colon \Omega_c^{0,\bu}(X) & \to & \CC [??] \\
%\alpha & \mapsto & \int_{X} \Omega \wedge \alpha 
%\end{array}
%\]
%be integration along the holomorphic volume form. 
%Define the pairing 
%\[
%\omega \colon \cE_{pot,c} \times \cE_{pot} \to \CC [??] 
%\]
%by the formula $ \gamma \vee \mu +  \beta \nu.$
%
%\brian{defines Poisson bivector $\Pi$.}
%
%\subsubsection{}

%Define the map of cochain complexes $\Phi \colon \cE_{pot} \to \til{\cE}_{KS}$
%by the formulas
%\[
%\Phi (\mu,\nu) = \mu + u \, \nu,\qquad \Phi(\beta, \gamma) = \div \left( \gamma \vee \Omega^{-1} \right)  .
%\]
%In the last expression we have used the isomorphism $\Omega^{-1} \colon \Omega^{1,\bu} \xto{\cong} \PV^{d-1,\bu}$
%so that $\div(\gamma \vee \Omega^{-1}) \in \PV^{d-2,\bu} \subset \til{\cE}_{KS}$.
%
%\begin{prop}
%\label{prop:pot}
%The map $\Phi \colon  \cE_{pot} \to \til{\cE}_{KS}$ intertwines the linear BRST differentials and the shifted Poisson bivectors $\Phi_*\Pi = \til{\Pi}_{KS}$. 
%In particular, it defines a map of dg Lie algebras
%\[
%\Phi^* \colon \big(\oloc(\cE_{KS}),\{-,-\}_c, Q_{KS} \big) \to \big(\oloc(\cE_{pot}),\{-,-\}, Q \big)
%\]
%\end{prop}
%
%\begin{proof}
%prove this
%\end{proof}
%
%As a corollary of Proposition \ref{prop:tilbcov} and Proposition \ref{prop:pot} we have the following result. 
%
%\begin{cor}
%The BCOV action $\til{I}_{BCOV}$ restricts along $\Phi$ to a solution of the classical master equation for $\cE_{pot}$:
%\begin{align*}
%I_{pot} & = \Phi^* \til{I}_{BCOV} \\
%Q I_{pot} + \frac12 \{I_{pot},I_{pot}\} & = 0 .
%\end{align*} 
%\end{cor} 


\end{document}
