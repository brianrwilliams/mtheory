\documentclass[11pt]{amsart}

\usepackage{macros-mtheory,amsaddr,tikz-feynman}

\addbibresource{references.bib}

%\linespread{1.2} %for editing
%\usepackage{mathpazo}

\begin{document}

\title{Twisted 11-dimensional supergravity}
\author{Surya Raghavendran}
\address{Perimeter Institute for Theoretical Physics \\ 31 Caroline Street North \\ 
Waterloo, Ontario N2L 2Y5\\ Canada}
\email{??}
\author{Ingmar Saberi}
\address{Ludwig-Maximilians-Universit\"at M\"unchen \\ Fakult\"at f\"ur Physik \\ Theresienstra\ss{}e 37 \\ 80333 M\"unchen \\ Deutschland}
\email{i.saberi@physik.uni-muenchen.de}
\author{Brian R. Williams}
\address{School of Mathematics \\ University of Edinburgh \\ Edinburgh EH9 3FD \\ Scotland}
\email{brian.williams@ed.ac.uk}
\begin{abstract}
to do
\end{abstract}
\maketitle

\tableofcontents

\newpage 

\documentclass[11pt]{amsart}

%\usepackage{../macros-master}
\usepackage{macros-fivebrane}

\begin{document}

\section{Introduction}
Superconformal field theories admit a plethora of exactly computable, protected quantities, giving them a distinguished role in supersymmetric physics. 
Since it is believed that all supersymmetric field theories flow to superconformal fixed points, such quantities provide robust invariants of supersymmetric field theories. 
Examples of such a quantity is the superconformal index, which is a generating function for some collection of $R$-charges of certain BPS local operators. 
Crucial steps towards a more microscopic understanding of superconformal indices were taken in \brian{surya fill in this empty citation} \cite{}, articulated through the construction of twisting.

Introduced by Witten \cite{WittenTwist} and further developed by Costello \cite{CostelloHol}, twisting refers to a localization, or fixed-point, construction for theories equipped with an action of a supersymmetry algebra. 
Operationally, one modifies the BRST differential of the theory by a nilpotent supercharge---the result is a theory for which the infinitesimal translations in the image of the nilpotent supercharge act homotopically trivially. 
When a twist exists (which is almost always) a supersymmetric field theory admits a so-called minimal twist; the resulting theory is a holomorphic-topological field theory that is holomorphic in the maximal number of spacetime directions.

One of the insights of \brian{surya fill in this empty citation} \cite{} was that superconformal indices count exactly local operators in the minimal twist---accordingly, we may think of the algebra of local operators in the minimal twist as categorifying the index. 
Moreover, the algebra of local operators is part of the richer structure of a \textit{factorization algebra}. 
Whilst the former governs the behavior of observables supported at points, the latter organizes observables that are supported on any open set (for example, non-local operators obtained from local ones via a descent procedure). 

A principal goal of the present paper is to initiate the study of factorization algebras associated to the minimal twists of the six-dimensional~$\cN=(2,0)$ superconformal field theories.
%and three-dimensional $\cN=8$ superconformal field theories. 
As is well-known, the six-dimensional~$\cN=(2,0)$ supersymmetric theory is quite elusive---it admits no known Lagrangian formulation, and outside of the abelian case, is not known to admit a field theoretic realization. 
In light of this, we propose to access the minimal twist and its local operators via the proposal of \textit{twisted holography}.

\subsection{Twisted Holography}
Introduced by Costello and Li in \cite{CLsugra}, the twisted holography proposal posits an avatar of the AdS/CFT correspondence that holds at the level of supersymmetric twists. 
Fundamentally, it conjectures a duality between the algebra of observables of the (twisted) gravitational theory and the algebra of observables of the gauge theory.
There is an exciting body of work being developed around this program including tests of this proposal from both the gravitational and gauge theory side.
The type of twisted holography which is at play in this paper is a twisted version of the $AdS_7/CFT_6$ duality involving fivebranes inside of eleven-dimensional supergravity.  

We mentioned that the precise mathematical structure modeling observables of a quantum field theory is that of a factorization algebra.
The predicted type of duality between the factorization algebras associated to a gravitational theory and to the worldvolume theory of a number of branes is a general version of \textit{Koszul duality}.
Ordinary Koszul duality for associative algebras (so quantum mechanical systems) associates to a (augmented) algebra $A$ a dual algebra $A^!$ whose appropriate derived category of modules is the same as that of $A$. 
Following the work of \cite{CLsugra,CP1} (see also the review in \cite{PWkoszul}) there is a simple physical interpretation of Koszul duality.
If $A$ is the algebra of operators of some bulk quantum field theory (perturbatively we can even consider a theory of gravity) then $A^!$ is the algebra of operators on the universal topological line defect.
Universal here means that algebra of operators on any other line defect which couples to the bulk system admits a unique map of algebras from~$A^!$. 

The general theory of Koszul duality for factorization algebras has not been developed, and we do not do so in this paper.
This sort of duality would allow one to make sense of universality statements as above for higher dimensional, possibly non-topological, defects in an arbitrary bulk quantum field theory.
Nevertheless, we can make the following ansatz for the Koszul dual $\cF^!$, which we refer to as the $!$-dual in this paper to avoid confusion, of a factorization algebra~$\cF$ of observables of some bulk quantum field theory.
It is the universal factorization algebra, along a specified defect, which couples to the bulk quantum field theory.
While this heuristic definition sounds natural, it does not lend itself to an explicit description.
However, for factorization algebras of bulk quantum field theories which can be expressed in a particular way we can refer to a local version of Noether's theorem in order to get our hands on it, see~\S\ref{s:noether}.

Let us now make a more concrete, yet slightly informal, statement of twisted holography which fits into the approach of this paper.
Let $X$ be a smooth manifold, and let $\Obs_{grav}$ denote a factorization algebra on $X$ that we view as the observables of the bulk gravitational theory. 
Suppose we have, in addition, a stack of $N$ branes, wrapping a closed submanifold $Y\hookrightarrow X$ whose worldvolume theory has a factorization algebra of observables $\Obs_{brane}(N)$.
In the context of branes, it is natural to posit that the universal theory along the brane is given by the large $N$ limit of the factorization algebra $\Obs_{brane}(N)$.

Note that $\Obs_{grav}$ is a factorization algebra on $X$, while $\Obs_{brane}(N)$ is a factorization algebra on the closed submanifold $Y$ so we cannot yet compare them.
We can, however, restrict $\Obs_{grav}$ to a factorization algebra just on $Y$, which we denote by $\Obs_{grav}|_Y$.\footnote{In general this is given by some limit construction like for sheaves, but we will use a particularly nice model in the context of holomorphic-topological factorization algebras in this paper.}

\begin{expect}[Twisted holographic principle following \cite{CLsugra}]
There is a map of factorization algebras
\[
  (\Obs_{grav}|_{Y})^{!}\to \Obs_{brane}(N)
\]
that becomes an equivalence in the large $N$ limit.
\end{expect}

From the point of view of the bulk gravitational theory, one can understand $(\Obs_{grav}|_{Y})^{!}$ as a factorization algebra enhancement of the space of (twisted) multi-particle states of the gravitational theory on the manifold $X - Y$ obtained by subtracting the brane. 
In most examples the above principle is too naive as we are forgetting a key ingredient, the so-called backreaction of the branes. 
On one hand, this deforms the geometric structure used to define the gravitational theory on $X - Y$---concretely, it will correspond to a certain deformation of the factorization algebra $(\Obs_{grav}|_{Y})^{!}$. 

The above expectation can be tested in instances where both sides of the duality admit explicit descriptions. 
This has been carried out in many examples including:
\begin{itemize}
\item $D1$ branes in the topological string on the deformed conifold \cite{costello2021twisted}.
Here, the duality is formulated in terms of vertex algebras which are avatars of holomorphic factorization algebras on a Riemann surface.
This is also understood as a deformed version of the physical duality between 4d $\cN=4$ supersymmetric Yang--Mills and type IIB superstring theory on $AdS_5$. 
\item A system of twisted $D1-D5$ branes in type IIB string theory on twisted $AdS_3 \times S^3 \times T^4$ \cite{CP1}. 
Upon compactifying along $T^4$ this can be understood as a variant of the above example where the deformed conifold is replaced by a supersymmetric version.
\item Membranes and fivebranes in twisted $\Omega$-deformed $M$-theory on Taub-NUT space \cite{CostelloM5,CostelloM2}.
In the particular $\Omega$-background membranes are localized to a topological quantum mechanical system where the duality can be phrased in terms of associative algebras (and hence ordinary Koszul duality). 
The $\Omega$-background localizes fivebranes to a complex plane and the duality can be formulated in terms of vertex algebras (one-dimensional holomorphic factorization algebras). 
\end{itemize}

A feature of the above examples is that via some method one can understand, independently, the algebra of observables on both sides of the duality. 
For fivebranes in $M$-theory, before turning on the $\Omega$-background, there is no known description which is robust enough to fit inside of the factorization algebra formulation of a quantum field theory.
So, rather than prove, or test, a version of the twisted holographic principle, the objective of this paper is to \textit{use} the twisted holographic principal to obtain an explicit description of the factorization algebra associated to the worldvolume theory on a stack of fivebranes (after performing the minimal, holomorphic, twist). 

Rather surprisingly, we will argue that in the classical limit the factorization algebra associated to the worldvolume theory on a finite number of twisted fivebranes is a particular piece of the $!$-dual of the factorization algebra associated to the bulk gravitational theory. 
After taking a certain classical limit, our conjectural description arises from a presentation of the $!$-dual as
\beqn
(\Obs_{grav}|_{Y})^{!} \simeq \cF_{-1} \otimes \cF_0 \otimes \cdots \cF_1 \otimes \cdots 
\eeqn
for some factorization algebras $\cF_{-1},\cF_0, \cF_{1},\ldots$ which are defined using a certain weight decomposition of the bulk gravitational theory.
The strange indexing conventions will be explained in \S\ref{eqn:fact}. 

Using this presentation, our conjecture for the factorization algebra associated to the worldvolume theory on a stack of $N$ twisted fivebranes, after taking the same classical limit, has the form
\beqn\label{eqn:finiteTensor}
\Obs_{brane} (N) = \cF_{-1} \otimes \cF_2 \otimes \cdots \otimes \cF_{N-2} .
\eeqn
In other words, we simply truncate the tensor decomposition at order $N$.

A related theory is the minimal twist of the six-dimensional superconformal theory associated to a Lie algebra of type $A_{N-1}$.
We denote the corresponding factorization algebra by $\Obs_{A_{N-1}}$ for now. 
This is obtained from the worldvolume theory simply by throwing away the modes propagating transverse to the brane, which in our presentation above corresponds to stripping off the first factor $\cF_{-1}$. 
Thus, at the level of factorization algebras we have a similar decomposition
\beqn
\Obs_{A_{N-1}} = \cF_0 \otimes \cF_1 \otimes \cdots \otimes \cA_{N-1} .
\eeqn

We want to emphasize that these expressions only hold after taking this classical limit.
This decomposition as a tensor product of factorization algebras will break down at the quantum level.
Nevertheless, the description provides a very effective way to compute certain protected quantities like the superconformal index.

As the formalism of Koszul duality and factorization algebras is rather abstract, it is useful to first indicate what our approach looks like at the level of the superconformal index. 
On one hand, we have the index of multi-particle states of the supergravity theory on the manifold obtained by removing the location of the brane; we denote this by $\chi_{grav}({\bf x})$.
It is a function of some number of fugacities ${\bf x}$ which we will discuss explicitly in our examples momentarily. 
Typically, this multi-particle index can be obtained from the single-particle index $f_{grav}({\bf x})$ of gravitational states through the plethystic exponential. 
Since the index is only really sensitive to the minimal twist, it is reasonable that we can compute the index using our formulation of twisted eleven-dimensional supergravity.
We will obtain known formulas for the index of eleven-dimensional supergravity states on $AdS_7$ in \S\ref{s:states} using our description of twisted supergravity. 

While we don't need to know the precise relationship at the moment, we can point out that the data of a factorization algebra encodes, in particular, the algebra of local operators of a system.
This is convenient from the point of view of the superconformal index since as we have already pointed out we can compute it as the index of local operators of the twisted theory.
We will see how the descriptions above lead to a presentation for the superconformal index $\chi_N({\bf x})$ of the worldvolume theory on a finite stack of $N$ fivebranes. 

If we are working simply on the worldvolume $\C^3 \cong \R^6$, our description in \eqref{eqn:finiteTensor} posits that after taking the classical limit the local operators on a stack of $N$ twisted fivebranes is
\beqn
\cF_{-1}(0) \otimes \cdots \otimes \cF_{N-2}(0) ,
\eeqn
where $\cF_k(0)$ stands for the algebra of local operators associated to the factorization algebra $\cF_k$.
Forgetting about the differentials for the time being, the local operators $\cF_k(0)$ are given as the free algebra on some vector space $V_k$ of linear local opeartors.
Thus, to compute the index of local operators of $\Obs_{brane}(N)$ it suffices to compute the index of all of the $V_k$---this is the single particle index---and then apply the plethystic exponential.
For now, denote by $g_k({\bf x})$ the index of the vector space $V_k$.

Putting all of this together, our approach is based on the observation that we can express the single particle supergravity index as
\beqn
f_{grav}({\bf x}) = \sum_{k=-1}^\infty g_k({\bf x}) .
\eeqn
Then, the avatar of our conjectural description in \eqref{eqn:finiteTensor} at the level of the superconformal index is
\beqn
f_N ({\bf x}) = \sum_{k = -1}^{N-2} g_k({\bf x}) .
\eeqn
To obtain the superconformal index we simply apply the plethystic exponential
\beqn
\chi_N({\bf x}) = \prod_{k=-1}^{N-2} {\rm PExp}\left[g_k({\bf x})\right] .
\eeqn
This is a formula for the index on a finite stack of $N$ fivebranes. 
A related quantity is the index of the superconformal field theory associated to the Lie algebra $A_{N-1}$. 
To obtain this we simply through away contributions coming from a single fivebrane, which results in the expression
\beqn
\chi_{A_{N-1}}({\bf x}) = \prod_{k=0}^{N-2} {\rm PExp}\left[g_k({\bf x})\right] .
\eeqn

From these formulas it is manifest that the holographic relation $\chi_{grav} = \lim_{N \to \infty} \chi_N$ holds. 
The next thing we turn to is evidence for our claim that $\chi_N({\bf x})$ really is the superconformal index associated to fivebranes. 
In ~\S\ref{s:finite} we will show that for small values of $N$ our closed of formulas, when appropriately expanded, agree with approximations which have been found in the literature.
Even stronger evidence would involve the factorization algebras $\Obs_{brane}(N)$, and we will briefly sketch how this should go in later sections but leave a full treatment to future work.

To get an even better handle on our conjectural description we focus on the case~$N=2$. 

\subsection{Infinite dimensional symmetry enhancement by exceptional simple super Lie algebras}

The first ingredient we need 
In \S\ref{s:twisted} we will recall a description of the minimal twist of eleven-dimensional supergravity following \cite{RSW} which is 

\subsection{Outline}

The paper is organized as follows. The first section is the present introduction.

In section 2 we begin by recalling descriptions of minimal twists of eleven dimensional supergravity \cite{} and the abelian six-dimensional $\cN=(2,0)$ \cite{} and three-dimensional $\cN=8$ \cite{} superconformal field theories.

\end{document}
 

\section{The minimal twist of eleven-dimensional supergravity} 
\label{s:dfn}

In this section we define the central theory of study within the Batalin--Vilkovisky formalism.
The theory will be defined on any eleven-dimensional manifold of the form $X \times L$, where $X$ is a Calabi--Yau five-fold and $L$ is a smooth oriented one-manifold.

\subsection{Divergence-free vector fields} 

\subsubsection{}
\label{sec:divfree}
We set up some notations and conventions in the context of complex geometry. 
Let $V$ be a holomorphic vector bundle on a complex manifold $X$. 
If $j$ is an integer, we let $\Omega^{0,j}(X, V)$ denote the space of anti-holomorphic Dolbeault forms of type $j$ on~$X$ with values in $V$.
The $\dbar$ operator $\dbar \colon \Omega^{0,j}(X, V)\to \Omega^{0,j+1}(X,V)$ defines the Dolbeault complex of $V$:
\[
  \Omega^{0,\bu}(X, V) = \left(\Omega^{0,j}(X, V)[-j] , \; \dbar\right) .
\]
This is a free (smooth) resolution for the sheaf of holomorphic sections of $V$.

Suppose $X$ is a Calabi--Yau manifold with holomorphic volume form $\Omega$.
The divergence $\div(\mu)$ of a holomorphic vector field $\mu$ is defined by the formula
\[
\div (\mu) \wedge \Omega = L_\mu (\Omega)
\]
where, on the right hand side, we mean the Lie derivative of $\Omega$ with respect to $\mu$.

Let $\T_X$ denote the holomorphic tangent bundle and consider its Dolbeault complex $\Omega^{0,\bu}(X , \T_X)$ resolving the sheaf of holomorphic vector fields. 
The divergence operator extends to the Dolbeault complex to yield a map of cochain complexes 
\[
\div \colon \Omega^{0,\bu}(X , \T_X) \to \Omega^{0,\bu}(X) .
\]
The resulting complex of sheaves
\beqn\label{eqn:cplx1}
\begin{tikzcd}
\ul{0} & \ul{1} \\
\Omega^{0,\bu}(X , \T_X) \ar[r, "\div"] & \Omega^{0,\bu}(X) ,
\end{tikzcd}
\eeqn
resolves the sheaf of holomorphic divergence-free vector fields $\Vect_0 (X)$.
The anti-holomorphic Dolbeault degrees and the $\dbar$ operator are left implicit. 
%By the holomorphic Poincar\'e lemma, the embedding of the sheaf $\Vect_0(X)$ into the degree zero piece of this complex is a quasi-isomorphism. 

There is a direct way to extend the Lie bracket of vector fields to the complex \eqref{eqn:cplx1}. 
Denote by $\mu$ an element of $\Omega^{0,\bu}(X , \T_X)$ and $\nu$ an element of $\Omega^{0,\bu}(X)$ (for simplicity in notation, we will not expand the anti-holomorphic dependence). 
The Lie bracket defined by the formulas
\begin{align*}
[\mu, \mu'] & = L_\mu \mu' \\
[\mu, \nu] & = L_\mu \nu 
\end{align*}
is compatible with $\div$ and endows \eqref{eqn:cplx1} with the structure of a sheaf of dg Lie algebras.
We will refer to this sheaf by the symbol $\cL_0(X)$, or just $\cL_0$ if $X$ is understood. 

The sheaf $\cL_0$ has the structure of a {\em local} dg Lie algebra~\cite[\S 3.1.3]{CG2}.
This means that, as a graded sheaf, $\cL_0$ is the smooth sections of a graded vector bundle, and its differential and Lie bracket are given by differential and bidifferential operators respectively.


\parsec[sec:Linfty]

Recall that an $L_\infty$ algebra is a $\ZZ$-graded vector space $\cL$ together with the data of a square-zero, degree $+1$ derivation $\delta_\cL$ of the free commutative graded algebra $\Sym\left(\cL^\vee [-1] \right)$. 
The Chevalley--Eilenberg cochain complex is 
\[
\left(\Sym\left(\cL^\vee [-1] \right), \delta_\cL\right) .
\]
The Taylor components of $\delta_\cL$ define higher brackets $\{[-]_k\}_{k=1,2,\ldots}$ where $[-]_k \colon \cL^{\times k} \to \cL[2-k]$. 
The condition that the differential $\delta_\cL$ is square-zero is equivalent to the higher Jacobi relations.

An $L_\infty$ morphism $\Phi: \cL \rightsquigarrow \cL'$ is the same datum as a map of commutative dg algebras 
\deq{
  \Phi^*: \clie^\bu(\cL') \to \clie^\bu(\cL)
}
between their respective Lie algebra cochains. It follows from this that \emph{any} automorphism $\Phi$ of the free commutative algebra on $\cL^\vee[-1]$ defines a new model of the $L_\infty$ algebra $\cL$, for which the Chevalley--Eilenberg differential is obtained by conjugating $\delta_\cL$ by~$\Phi$, and where $\Phi$ itself defines the $L_\infty$ isomorphism.

$\cL_0$ is the sheaf~\eqref{eqn:cplx1} resolving divergence-free vector fields equipped with the dg Lie algebra structure constructed in the previous section.
We consider the following automorphism of~$\Sym(\cL_0^\vee[-1])$, defined by its action on generators:
\deq[eq:newbase]{
    \Psi_\infty: \nu \mapsto 1 - e^{-\nu}, \quad \mu \mapsto e^{-\nu} \mu.
}
This map defines a new model of the $L_\infty$ algebra with underlying graded vector space the same as \eqref{eqn:cplx1}, which we will call $\cL_\infty$.\footnote{We are being slightly abusive and using the symbols $\nu,\mu$ dually as coordinates, or operators, on the graded linear space $\cL[1]$.}
The formulas for the automorphism above clearly arise from maps of vector bundles and hence endow $\cL_\infty$ with the structure of a local $L_\infty$ algebra, meaning all operations are given by polydifferential operators.  

The notation refers to the fact that this new model has nonvanishing $L_\infty$ brackets of every order. 
It is this new model that we will use to define the eleven-dimensional theory of twisted supergravity (as well as the family of analogous formal moduli problems on products of odd Calabi--Yau manifolds with~$\R$). 


%The previous proposition characterizes the $L_\infty$ model for divergence-free vector fields that we will use to define the 11-dimensional theory of twisted supergravity. 
%Hereon, we denote by $\cL_0 = \cE_0[-1]$ this local $L_\infty$ algebra. 

%We can unpack Proposition \ref{prop:Linfty} to describe the $L_\infty$ structure on $\cL_0$ explicitly. 
We can describe the $L_\infty$ structure on our new model $\cL_\infty$ explicitly.
Recall that we have two types of elements: $\mu \in \PV^{1,\bu}$ and $\nu \in \PV^{0,\bu}[-1]$. (Here, and in what follows, we will use the symbol $\PV^{i,\bu}$ for the Dolbeault resolution of \emph{holomorphic polyvector fields;} by definition, this is the complex $\Omega^{0,\bu}(X,\wedge^i \T_X)$.)
The first few nonzero brackets are
\begin{align*}
[\mu]_1 & = \dbar \mu + \div \mu \\
[\mu_1,\mu_2]_2 & = \div (\mu_1 \wedge \mu_2) \\
[\nu, \mu_1,\mu_2]_3 & = \div(\nu \mu_1 \wedge \mu_2) 
%\\
%[\nu_1,\nu_2, \mu_1,\mu_2]_4 & = \# \div(\nu_1 \nu_2 \mu_1 \mu_2).
\end{align*}
For $k \geq 2$ the general formula for the $k$-ary brackets are 
\begin{align*}
[\nu_1, \ldots, \nu_{k-2}, \mu_1,\mu_2]_{k} & = \div(\nu_1 \cdots \nu_k \mu_1 \wedge \mu_2) \\
[\nu_1,\ldots, \nu_{k-3}, \mu_1,\mu_2,\gamma]_k & = \nu_1 \cdots \nu_{k-3} (\mu \wedge \mu') \vee \del \gamma .\\
[\nu_1,\ldots,\nu_{k-2}, \mu, \gamma]_k & = \nu_1 \cdots \nu_{k-2} \mu \vee \del \gamma .
\end{align*}

%We describe the explicit $L_\infty$ automorphism $\Psi \colon (\cL_0)^{L_\infty} \rightsquigarrow (\cL_0)^{strict}$ intertwining the strict dg Lie structure on $\cL_0$ and this $L_\infty$ structure.
%The linear term $\Psi^{(1)} = \id$ is the identity map. 
%The higher terms $\Psi^{(n)}$ are defined by 
%\brian{someone check me}
%\begin{align*}
%\Psi^{(n)} (\nu_1,\ldots, \nu_{n-k},\mu_1,\ldots, \mu_k) & = \delta_{k=1} \nu_1 \cdots \nu_{n-1} \mu_1 . \\
%\end{align*}


%\parsec
%There is yet another $L_\infty$ model for divergence-free vector fields that we remark on here.
%\ingmar{I want to write the other automorphism later, I think; otherwise we have to write the composition, since we presented it on bcov}


\subsection{Theories of BF type}

\parsec
Suppose that $\cL$ is an $L_\infty$ algebra with $L_\infty$ operations $\{[-]^\cL_k\}_{k=1,2,\ldots}$ and that $(\cA, \d_\cA)$ is a commutative dg algebra. 
The graded vector space $\cL \otimes \cA$ is equipped with the natural structure of an $L_\infty$ algebra with operations $\{[-]_k\}_{k=1,2,\ldots}$ defined by
\begin{align*}
[x \otimes a]_1 & = [x]^\cL_1 \otimes a + (-1)^{|x|} x \otimes \d_\cA a \\
[x_1 \otimes a_1, \ldots , x_k \otimes a_k]_k & = [x_1,\ldots,x_k]^\cL_k \otimes (a_1 \cdots a_k), \qquad k \geq 2 .
\end{align*}

We apply this construction, taking $\cL$ to be the sheaf resolving divergence-free holomorphic vector fields on a Calabi--Yau manifold $X$ equipped with either the strict dg Lie algebra structure $\cL_0(X)$ or its non-strict $L_\infty$ structure $\cL_\infty (X)$. 
The algebra $\cA$ will be the smooth de Rham complex $(\Omega^\bu(S) , \d_S)$ where $S$ is a smooth manifold (we will specialize the dimension of this smooth manifold shortly, but the constructions in this section make sense in any dimension). 

We thus obtain the structure of an dg Lie algebra on $\cL_0(X) \otimes \Omega^{\bu}(S)$ or an $L_\infty$ algebra $\cL_\infty(X) \otimes \Omega^\bu(S)$.
These define equivalent local $L_\infty$ algebras on the product manifold~$X \times S$. 

\parsec[s:bf]

Associated to any local $L_\infty$ algebra is a classical field theory in the BV formalism.
Let $\cL$ be a local $L_\infty$ algebra on some manifold $M$, it is the sheaf of sections of some graded vector bundle $L$. 
For a section $A \in \cL$, introduce the `higher curvature map' defined by the formula
\[
\mathsf{F}_A = [A]_1 + \frac12 [A,A]_2 + \frac{1}{3!} [A,A,A]_3 + \cdots .
\]

The fields of the associated BV theory are pairs
\[
  (A, B) \in \cL[1] \oplus \cL^{!}[-2] .
\]
Here $\cL^!$ denotes the sheaf of sections of the bundle $L^* \otimes {\rm Dens}$, where ${\rm Dens}$ is the bundle of densities. 
The shifted symplectic BV pairing is the obvious integration pairing between $A$ and $B$. 

The action functional reads $S_{\rm BF} = \int_M B \, \mathsf{F}_{A}$ which leads to the equations of motion $\mathsf{F}_{A} = 0$ and $\mathsf{D}_A B= 0$ where $\mathsf{D}_A$ is the higher covariant derivative along $A$. 
We refer to this as the ``BF theory'' associated to $\cL$.

We thus obtain a theory in the BV formalism on the product manifold $X \times S$ associated to both local $L_\infty$ algebras $\cL_0(X) \otimes \Omega^{\bu}(S)$ and $\cL_\infty(X) \otimes \Omega^\bu(S)$.

\parsec
For concreteness, we spell out the fields of the theories we have constructed on $X \times S$.
In both cases, the space of fields equipped with the linear BRST operator is
\begin{equation}
  \label{eq:sympfields} 
  \begin{tikzcd}[row sep = 1 ex]
    -n & -n + 1 & -1 & 0 \\ \hline
    \Omega^{0}(X;S) \ar[r, "\del"] & \Omega^{1}(X;S) & 
     \PV^{1}(X; S) \ar[r, "\div"] & \PV^{0}(X; S).
\end{tikzcd}
\end{equation}
We denote the fields $(\beta,\gamma,\mu,\nu)$ respectively.
We are using the shorthand notation
\begin{align*}
\Omega^{i}(X;S) & = \Omega^{i , \bu;\bu}(X;S) \\
 & = \oplus_{j,k} \PV^{i,j}(X) \otimes \Omega^k(S) [-j-k] .
\end{align*}
which is equipped with the $\dbar + \d_S$ operator and similarly for $\PV^{i}(X;S)$. 

The natural pairing between $\PV^i(X;S)$ and~$\Omega^i(X;S)$ is of degree $-\dim_\C(X) -\dim_\R(S)$. 
As such, the $\Z$-grading indicated in~\eqref{eq:sympfields} equips the sheaf of fields with a degree $(-1)$ pairing, provided that we choose the shift to be given by
\deq{
  n = \dim_\C(X) + \dim_\R(S) - 1.
}
The pairing is defined by the formula 
\[
\int^\Omega_{X \times S} \mu \vee \gamma + \int^\Omega_{X \times S} \nu \beta 
\]
where $\int^\Omega_{X \times S} \alpha = \int_{X \times S} \alpha \wedge \Omega$. 

We have constructed two equivalent descriptions of the BF theory which share the linear BRST complex \eqref{eq:sympfields}.
Explicitly, the action functional for BF theory associated to the local dg Lie algebra $\cL_0(X) \otimes \Omega^{\bu}(S)$ is
\deq{
  S_{BF,0} =  \int^\Omega \bigg[\beta \wedge (\dbar + \d_S) \nu +  \gamma \wedge (\dbar + \d_S) \mu +  \beta \wedge \partial_\Omega \mu +  \frac{1}{2} [\mu,\mu] \vee \gamma +  [\mu,\nu] \beta \bigg] .
}
As in the Lie algebra structure of this strict model, notice that the Schouten--Nijenhuis bracket appears explicitly. 

The action functional of BF theory associated to $\cL_\infty(X) \otimes \Omega^{\bu}(S)$ is non polynomial. 
In fact, it is related to the BCOV action functional via dimensional reduction (see \S \ref{sec:dimred}).
Explicitly, this action functional is
\deq{
  S_{BF,\infty} =  \int^\Omega \bigg[ \beta \wedge (\dbar + \d_S) \nu +  \gamma \wedge (\dbar + \d_S) \mu +   \beta \wedge \partial_\Omega \mu + \frac12 \frac{1}{1-\nu} \mu^2 \vee \del \gamma \bigg] .
}

We demonstrated above that the two local $L_\infty$ algebras on which these BF theories are based are equivalent. As such, the BF theories are also equivalent; the map~\eqref{eq:newbase} extends uniquely to an automorphism of BV theories.
Explicitly, the automorphism is
\begin{equation}\label{eqn:auto1}
  \mu \mapsto e^{-\nu} \mu, \qquad \nu \mapsto 1-e^{-\nu} , \qquad
  \beta \mapsto (\beta - \mu \vee \gamma) e^{\nu},\qquad \gamma \mapsto e^{\nu} \gamma .
\end{equation}

\parsec[]

In what follows, we specialize to the case that $X$ is a Calabi--Yau five-fold and that $S$ is a one-dimensional smooth orientable manifold. 
In this case, with $n = 5 + 1 - 1 = 5$ the theories described in this section are $\ZZ$-graded in the BV formalism.
Momentarily, we consider a new term in the action which will break this grading; as such, this integer shift will not play an essential role.

\subsection{A deformation of BF theory} 

Let $X$ be a Calabi--Yau five-fold and $S$ be a smooth oriented one-dimensional real manifold. 
We will break the $\ZZ$-grading present in BF theory discussed in the previous section to a $\ZZ/2$ grading. 
For reference, this means that the linear cochain complex of fields of the model now takes the following form. 

\begin{equation}
  \label{eq:sympfields} 
  \begin{tikzcd}[row sep = 1 ex]
    {\rm odd} & {\rm even} & {\rm odd} & {\rm even} \\ \hline
    \Omega^{0}(X;S)_\beta \ar[r, "\del"] & \Omega^{1}(X;S)_\gamma & 
     \PV^{1}(X; S)_\mu \ar[r, "\div"] & \PV^{0}(X; S)_\nu.
\end{tikzcd}
\end{equation}

\parsec
%check CME

To define our classical field theory on $X \times S$, we consider  a deformation of BF theory $S_{BF}$ (this refers to either the presentation as $S_{BF,0}$ or $S_{BF,\infty}$). 
Such deformations are governed by the classical master equation: the parameterized family of actions 
\beqn\label{eqn:defaction}
S_{BF} + g J
\eeqn
defines a consistent theory in the BV formalism if and only if
\deq{
  \{S_{BF} + g J, S_{BF} + g J \} = 0.
}
Since this must hold for all $g$, and since the undeformed action $S$ is already a solution to the classical master equation, this reduces to the pair of conditions
\deq[eq:2cond]{
  \{S_{BF},J\} =  \{J,J\} = 0.
}

The form of $J$ depends on which presentation we use for BF theory.
To begin, we will use the presentation of BF theory $S_{BF, \infty}$ which uses the the non-strict $L_\infty$ structure on divergence-free holomorphic vector fields.
The deformation $J$ does not make reference to the Calabi--Yau structure explicitly, but it does involve the holomorphic de Rham operator $\del$ on $X$. 

The main result of this section is the following. 

\begin{thm}
\label{thm:dfn}
Let $X$ be a Calabi--Yau five-fold and $S$ a smooth one-dimensional manifold, and consider the BV theory $(\cE, S_{BF,\infty})$ on $X \times S$ defined above. The local functional 
  \deq{
    J = \frac16 \gamma \wedge \del \gamma \wedge \del \gamma ,
  }
  where $\gamma \in \Omega^{1,\bu}(X;S)$, defines a deformation of~$(\cE,S_{BF,\infty})$ as a $\Z/2$-graded BV theory.
\end{thm}

\parsec[]

First off, we remark on grading issues. 
In the original $\Z$-grading on the BF theory given in \eqref{eq:sympfields} with $n=5$, the component 
\[
\gamma^{1,i;j} \in \Omega^{1,i}(X) \otimes \Omega^j(S) 
\]
sits in degree $-4+i+j$. 
Thus, we see that in the original $\Z$-grading on BF theory one has
  \deq{
    \deg(J) = 6.
  }
Thus $S_{BF} + J$ is not of homogenous $\ZZ$ grading (although it is even).

This is completely reasonable from the point of view of twisting supersymmetry in eleven dimensions. 
Indeed, the $R$-symmetry group is trivial, and there is not a way to regrade the fields of the twisted theory using twisting data~\label{CosHol,ESW}. 
Nevertheless, if we break to the obvious $\ZZ/2$ grading, the functional $S_{BF} + g J$ defines an even action functional.
Unless otherwise stated, we will work with this $\ZZ/2$ grading for the remainder of this section.

\parsec[]
We proceed to show that $S_{BF,\infty} + g J$ solves the classical master equation.
For notational simplicity we will omit the integral symbol $\int^\Omega$.

%By apparent field type reasons, the equation \eqref{eq:2cond} is equivalent to the following two equations
%\[
%\{S_{BF,\infty}, J\} = 0, \qquad \{J,J\} = 0.
%\] 

It is immediate from the form of the BV bracket that $\{J,J\} = 0$, since $J$ depends only on the $\gamma$ field. 
It remains to check that $\{S_{BF,\infty},J\} = 0$. 
For the quadratic term in the BF action, we note that 
  \deq{
    \{\beta \wedge \div\mu, J\} = \frac12 \del\beta \wedge \del \gamma \wedge \del \gamma = 0,
  }
  because total derivatives are equivalent to zero as local functionals. 
  
The contribution from the remaining BF action takes the form
\[
    \left\{ \frac12 \frac{1}{1-\nu} \del\gamma \vee \mu^2, \frac16 \gamma \wedge \del\gamma \wedge \del \gamma \right\} =\frac12 (\mu \vee \del \gamma) \wedge \del \gamma \wedge \del \gamma .
\]
    %\frac12 \frac{1}{1-\nu} \bigg[ (\del\gamma\vee\mu) \wedge \del\gamma \wedge \del \gamma \pm \del\gamma \wedge (\del\gamma\vee\mu) \wedge \del \gamma \pm \del\gamma \wedge \del \gamma \wedge (\del\gamma\vee\mu) \bigg].
This expression is zero for symmetry reasons. 
Recall that $\del\gamma$ is a two-form, and that the expression must be a totally symmetric local functional which is cubic in this two-form. We can ask whether such a  contraction exists just at the level of $\lie{sl}(5)$ representation theory. Let $\ydiagram{1}$ denote the fundamental representation of~$\lie{sl}(5)$, which we identify with constant one-forms. Since the term must be a scalar, the contraction $(\partial\gamma^3$ must sit in the fundamental representation again, since it is dual to a vector field. Computing the decomposition of the tensor cube of the two-form, we find
\deq{
  \Sym^3 \left( \ydiagram{1,1} \right) \cong \ydiagram{3,3} \oplus \ydiagram{2,2,1,1}\,.
}
(In fact, the absence  of the relevant irreducible representation does not  even depend on the parity of the  field $\gamma$, since 
  \deq{
    \wedge^3\left( \ydiagram{1,1} \right) \cong \ydiagram{3,1,1,1}  \oplus \ydiagram{2,2,2}\, ;
  }
the  fundamental representation has symmetry type $\ydiagram{2,1}$.) 

\parsec[s:coupling]

We make note of the dependence on the coupling constant $g$ in the definition of the deformed action $S_{BF,\infty} + g J$. 

When $g = 0$ we return to BF theory for the $L_\infty$ algebra $\cL_\infty(\CC^5) \otimes \Omega^\bu(\RR)$. 
For any $g \ne 0$ the theories are essentially equivalent in perturbation theory. 
Indeed, if $g \ne 0$ we can make the following field redefinition 
\[
\gamma \mapsto \sqrt{g} \gamma, \quad \beta \mapsto \sqrt{g} \beta 
\]
to write the action as 
\[
\frac{1}{\sqrt{g}} \left(S_{BF,\infty} + J \right)  .
\]

In perturbation theory, this has the affect of modifying the quantization parameter $\hbar$ to $\hbar / \sqrt{g}$.
Thus, after modifying $\hbar$ and making the above field redefinition, the perturbative expansion of any theory is equivalent to the one with $g = 1$. 

\parsec[s:altdfn]

We remark on an alternative, equivalent, description of the deformed theory which involves the strict dg Lie algebra structure on divergence-free holomorphic vector fields.

We can replace $S_{BF,\infty}$ by $S_{BF,0}$ via applying the field automorphism \eqref{eqn:auto1}.
Doing this we see that $J$ becomes 
\[
\til{J} = \frac16 e^\nu \gamma \wedge \del (e^\nu \gamma) \wedge \del(e^\nu \gamma) .
\]
Since this automorphism preserves the odd BV bracket, the actions $S_{BF,\infty} + g J$ and $S_{BF, 0} + g \til{J}$ are both solutions to the classical master equation, and are equivalent as~$\ZZ/2$ graded BV theories.

\subsection{Equations of motion of the component fields} \label{s:components}

Soon, we will provide a series of justifications for the assertion that the deformed theory $S_{BF, \infty} + g J$ is the minimal twist of eleven-dimensional supergravity on flat space $X \times S = \CC^5 \times \RR$ where $\CC^5$ is equipped with its flat Calabi--Yau form. 
For the moment, we briefly read off the equations of motion of the general theory on $X \times S$.
Let $\Omega$ denote the Calabi--Yau form on $X$. 

We consider the action $S_{BF, \infty} + gJ$.
The equation of motion obtained by varying $\beta$ is especially simple---in fact linear---since $\beta$ only appears in the action via a quadratic term. 
It is
\beqn\label{eqn:eombeta}
\dbar \nu + \d_S \nu + \div \mu = 0 .
\eeqn
Varying $\gamma$ we obtain the equation of motion
\beqn\label{eqn:eomgamma}
\dbar \mu + \d_S \mu + \frac12 \frac{1}{1-\nu} \div (\mu^2) + \frac12 (\del \gamma \wedge \del \gamma) \vee (g \Omega^{-1}) = 0 .
\eeqn
The last term represents the contraction of an element of $\Omega^{4,\bu}(X;S)$ with the nonvanishing section $\Omega^{-1} \in \PV^{5,\bu}(X;S)$ to yield an element of $\PV^{1,\bu}(X;S)$. 
If we vary the $\mu$ we obtain 
\beqn\label{eqn:eommu}
(\dbar + \d_S) \gamma + \del \beta + \frac{1}{1-\nu} (\mu \vee \del \gamma) = 0 .
\eeqn
Finally, if we vary $\nu$ we obtain
\beqn\label{eqn:eomnu}
(\dbar + \d_S) \beta + \frac12 \frac{1}{(1-\nu)^2} \mu^2 \vee \del \gamma = 0 .
\eeqn

The equation of motion must hold for any inhomogenous superfields.
We can get a better sense of the equations if we expand in components of these fields. 
The component fields of the eleven-dimensional theory on $X \times S$ have the following form: 
\begin{itemize}
\item $\mu = \sum_{i,j} \mu^{i;j}$ is a superfield where
\[
\mu^{i;j} \in \PV^{1,i}(X) \otimes \Omega^j(\RR) ,\quad i=0,\ldots, 5, \quad j=0,1.
\]
The component $\mu^{i;j}$ has parity $i+j+1 \pmod 2$. 
\item $\nu = \sum_{i,j} \nu^{i;j}$ is a superfield where
\[
\nu^{i;j} \in \PV^{0,i}(X, \T_X) \otimes \Omega^j(\RR) ,\quad i=0,\ldots, 5, \quad j=0,1.
\]
The component $\nu^{i;j}$ has parity $i+j \pmod 2$. 
%The linear equations of motion state that $\mu$ is constant along $\RR$ and holomorphic divergence-free as a vector field on $\CC^5$. 
\item 
$\gamma = \sum_{i,j} \gamma^{i;j}$ is a superfield where
\[
\gamma^{i;j} \in \Omega^{1,i}(X) \otimes \Omega^j(\RR) ,\quad i=0,\ldots, 5, \quad j=0,1.
\]
The component $\gamma^{i;j}$ has parity $i+j\pmod 2$. 
\item 
\item $\beta = \sum_{i,j} \beta^{i;j}$ is a superfield where
\[
\beta^{i;j} \in \Omega^{0,i}(X) \otimes \Omega^j(\RR) ,\quad i=0,\ldots, 5, \quad j=0,1.
\]
The component $\beta^{i;j}$ has parity $i+j+1 \pmod 2$. 
\end{itemize}

We look closely at the geometric meaning of \eqref{eqn:eombeta}. 
Let's make the simplifying assumption that all components of $\mu$ are divergence-free, and further that all fields are locally constant along $S$: that is, $\div \mu = 0$ and $\d_S \mu = \d_S \gamma = 0$.
Then $\nu = 0$ is a solution to \eqref{eqn:eombeta} and we can assume that all fields are functions, or zero-forms, along $S$. 
Then, there is a component of \eqref{eqn:eomgamma} which can be written as 
\beqn\label{eqn:eomgamma1}
\dbar \mu^{1;0} + \frac12 [\mu^{1;0},\mu^{1;0}] + \left(\frac12 \del \gamma^{1;0} \wedge \del \gamma^{1;0} + \del \gamma^{2;0} \wedge \del \gamma^{0;0}\right) \vee (g \Omega^{-1}) = 0 
\eeqn
where now $[-,-]$ stands for the Schouten bracket.

To further simplify \eqref{eqn:eomgamma1}, we can look for a solutions where  $\gamma^{1;0}$, the $(0,1)$ Dolbeault part of $\gamma$, is zero. 
Then, up to the term involving 
\[
\alpha \define \del \gamma^{0;0},
\]
we find precisely the integrability equation for the complex structure determined by Beltrami differential $\mu^{1;0} \in \PV^{1,1} \otimes \Omega^0$. 
If $\dbar \alpha = 0$, the holomorphic two-form $\alpha \in \Omega^{2,hol}(X)$ defines a map of sheaves
\[
\Omega^{2,hol}_X \xto{\wedge \alpha} \Omega^{4,hol}_X \cong_\Omega \cT^{hol}_X
\]
where $\cT^{hol}_X$ denotes the sheaf of holomorphic vector fields and the last isomorphism uses the Calabi--Yau form $\Omega$ on $X$.  
The image of $\Omega^{2,hol}_X$ defines a subsheaf $\cF_{\alpha} \subset \cT^{hol}_X$. 
Since $\del \alpha = 0$, this subsheaf is automatically integrable and hence determines a foliation. 

Summarizing, see that there is a field configuration where the Beltrami diffrential $\xi = \mu^{1;0} \in \Omega^{0,1}(X, \T_X)$ satisfies the modified integrability condition
\[
\dbar \xi + \frac12  [\xi , \xi] = \alpha \vee \rho
\]
for some $\rho \in \PV^{2,2} (X)$. 
In other words, $\xi$ defines an integrable complex structure deformation along the leaf space associated to the foliation $\cF_\alpha$. 
We leave a more complete exploration of the moduli space of solutions of the equations of motion for future work. 

In~\cite{SWspinor}, the second two authors showed that the free limit of the minimal twist of eleven-dimensional supergravity agrees with the free limit of the eleven-dimensional theory that we have introduced here. 
Given this result, we can recognize many fields in the twisted theory as components of the physical fields of supergravity which remain after we twist. 

\begin{itemize}
\item 
The components 
\begin{align*}
\mu^{1;0} & = \mu^j_i(z,\zbar,t) \d \zbar_j \partial_{z_i} \\
\mu^{0;1} & = \mu^t_i (z,\zbar,t) \d t \partial_{z_i}
\end{align*}
of $\mu$ comprise components of the metric which remain after the twist. 
The components 
\[
\mu^{0;0} = \mu_i (z,\zbar,t) \partial_{z_i} 
\]
comprise the ghosts for infinitesimal (holomorphic) diffeomorphisms. 
\item 
The three-form fields
\begin{multline}
\beta^{3;0} = \beta^{ijk} (z,\zbar,t) \d \zbar_i \d \zbar_j \d \zbar_k , \quad \beta^{2;1} = \beta^{ij}_t (z,\zbar,t) \d \zbar_i \d \zbar_j \d t \\
\gamma^{2;0} = \gamma^{ijk} (z,\zbar,t) \d z_i \d \zbar_j \d \zbar_k , \quad \gamma^{1;1} = \gamma^{ij}_t (z,\zbar,t) \d z_i \d \zbar_j \d t .
\end{multline} 
comprise components of the supergravity $C$-field which remain after the twist. 
The two-form fields $\beta^{2;0}, \beta^{1;1}, \gamma^{1;0}, \gamma^{0;1}$, the one-form fields $\beta^{1;0}, \beta^{0;1}$, and the zero-form field $\beta^{0;0}$ is what remains of the ghost system (ghosts, ghosts for ghosts, etc.) for the supergravity $C$-field. 
\end{itemize}
%The most geometrically relevant component is the case where 
%\[
%\mu^{1;0} \in \PV^{1,1}(X) \otimes \Omega^0(S) 
%\]
%which is an even field in our $\ZZ/2$ graded BV theory.
%The superscript denotes anti-holomorphic; de Rham form type. 
%For this component of $\mu$, the only components of $\gamma$ which appear in the above equations of motion are $\gamma^{0;0}$, $\gamma^{1;0}$, and $\gamma^{2;0}$.
%Expanding these components out, we obtain
%\beqn\label{eqn:eomgamma1}
%\dbar \mu^{1;0} + \frac12 [\mu^{1;0},\mu^{1;0}] + (\del \gamma^{0;0} \wedge \del \gamma^{2;0}) \vee (g \Omega^{-1}) + \frac12 (\del \gamma^{1;0} \wedge \del \gamma^{1;0}) \vee (g \Omega^{-1}) = 0 .
%\eeqn

\subsection{Local character}\label{sec:locchar}

We consider the eleven-dimensional theory on the manifold $\CC^5 \times \RR$, where $\CC^5$ is equipped with its standard Calabi--Yau structure. 
On this background, the theory is manifestly $SU(5)$ invariant. 
In this section, we compute the corresponding character of the local operators at the origin. 

The local character is only sensitive to the free limit of the theory.
Furthermore, the linear BRST operator is an $SU(5)$-invariant deformation of the $(\dbar + \d_{\RR})$ operator. 
Therefore, to compute the character it suffices to compute the $SU(5)$-equivariant character of the $\dbar$ cohomology. 

The solutions to the $(\dbar + \d_{\RR})$-equations of motion simply say that all fields are holomorphic along $\CC^5$ and constant along $\RR$. 
Thus, the solutions can be identified with 
\begin{align*}
\mu^{i}\partial_{z_i} & \in \Vect(\CC^5) \cong \cO(\C^5)\partial_{z_i},\quad 
\nu \in \cO (\C^5) \\
\beta & \in \cO (\C^5), \quad \gamma^{i} \d z_i \in \Omega^{1}(\CC^5) \cong \cO (\C^5)\d z_i 
\end{align*}
where $z_i$ is a holomorphic coordinate on $\CC^5$. 

Corresponding to each of the above, we have a tower of linear local operators labeled by $(m_j) = (m_1, m_2, m_3, m_4, m_5)\in \Z^5_{\geq 0}$; these are given by
\begin{align*}
 \mu^{i}_{(m_j)} &: \mu^{i}\mapsto \partial_{z_1}^{m_1}\partial_{z_2}^{m_2}\partial_{z_3}^{m_3}\partial_{z_4}^{m_4}\partial_{z_5}^{m_5}\mu^{i} (0) \\
\nu_{(m_j)} &: \nu\mapsto \partial_{z_1}^{m_1}\partial_{z_2}^{m_2}\partial_{z_3}^{m_3}\partial_{z_4}^{m_4}\partial_{z_5}^{m_5}\nu (0) \\
\gamma^{i}_{(m_j)} &: \gamma^{i}\mapsto \partial_{z_1}^{m_1}\partial_{z_2}^{m_2}\partial_{z_3}^{m_3}\partial_{z_4}^{m_4}\partial_{z_5}^{m_5}\gamma^{i} (0) \\
 \beta_{(m_j)} &: \beta\mapsto \partial_{z_1}^{m_1}\partial_{z_2}^{m_2}\partial_{z_3}^{m_3}\partial_{z_4}^{m_4}\partial_{z_5}^{m_5}\beta (0) \\
\end{align*}

\iffalse
We choose generators of the Cartan subgroup for the $SU(5)$ action with the following weights:

\[\begin{array}{|c|c|c|c|c|c|}
& z_1 & z_2 & z_3 & z_4 & z_5 \\
\hline
q & & & & 1 & -1 \\
t_1 & 1 & -1 & & & \\
t_2 & & 1 & -1 & & \\
y & -1 & -1 & -1 &\frac 3 2 & \frac 3 2
\end{array}\]
We are choosing the weights in this way as a matter of convenience for the later sections. 
For instance, upon performing the further twist of the eleven-dimensional theory, the $SU(5)$ symmetry is broken to an $SU(3)\times SU(2)$ symmetry; the Cartan of the unbroken symmetries corresponds to the fugacities $q, t_1, t_2$ in the above table. 

We can now readily compute the characters.
\begin{prop}
The $SU(5)$ local character of the holomorphic twist of the eleven-dimensional theory on flat space is
\begin{multline}
\chi_{SU(5)} = 
\prod_ {(m_j)\in \Z^5_{\geq 0}} \frac{1- t_1^{-m_1+m_2}t_2^{-m_2+m_3}q^{-m_4+m_5}y^{m_1+m_2+m_3-\frac 3 2 (m_4+m_5)}}{1- t_1^{-m_1+m_2}t_2^{-m_2+m_3}q^{-m_4+m_5}y^{m_1+m_2+m_3-\frac 3 2 (m_4+m_5)} }
\\ 
\times \frac{t_1^{-1}y + t_1t_2^{-1}y + t_2y + q^{-1}y^{-\frac 3 2} + qy^{-\frac 3 2}}{t_1y^{-1} + t_1^{-1}t_2y^{-1} + t_2^{-1}y^{-1} + q^{-1}y^{\frac 3 2} + qy^{\frac 3 2}}.
\end{multline}
\end{prop}
\begin{proof}
We compute the local character as the plethystic exponential of the character of linear local operators. 

Note that the linear local operators $ \nu_{(m_j)}$ and $\beta_{(m_j)}$ are of the same weight but opposite parity so contribute to the character with opposite sign. These contributions therefore cancel. Next, there is a summand of the linear local operators of the form 
\[
\bigoplus _{(m_j)\in \Z^5_{\geq 0 }} \Pi \left ( \C\mu^{z_1}_{(m_j)}\partial_{z_1}\oplus \C\mu^{z_2}_{(m_j)}\partial_{z_2}\oplus \C\mu^{z_3}_{(m_j)}\partial_{z_3}\oplus\C\mu^{w_1}_{(m_j)}\partial_{w_1}\oplus \C\mu^{w_2}_{(m_j)}\partial_{w_2}\right).
\] 
This contributes 
\begin{multline}
\sum_{(m_j)\in \Z^5_{\geq 0 }}- t_1^{-m_1+m_2}t_2^{-m_2+m_3}q^{-m_4+m_5}y^{m_1+m_2+m_3-\frac 3 2 (m_4+m_5)} \times \\
\left (t_1^{-1}y + t_1t_2^{-1}y + t_2y + q^{-1}y^{-\frac 3 2} + qy^{-\frac 3 2} \right).
\end{multline}
Finally, there is a summand of the form 
\[
\bigoplus _{(m_i;n_j)\in \Z^5_{\geq 0 }} \Pi \left ( \C\gamma^{z_1}_{(m_i; n_j)}d{z_1}\oplus \C\gamma^{z_2}_{(m_i; n_j)}d{z_2}\oplus\C\gamma^{z_3}_{(m_i; n_j)}d{z_3}\oplus\C\gamma^{w_1}_{(m_i; n_j)}d{w_1}\oplus \C\gamma^{w_2}_{(m_i; n_j)}d {w_2}\right).
\] 
This likewise contributes 
\[
\sum_{(m_j)\in \Z^5_{\geq 0 }}t_1^{-m_1+m_2}t_2^{-m_2+m_3}q^{-n_1+n_2}y^{m_1+m_2+m_3-\frac 3 2 (m_4+m_5)}\left (t_1y^{-1} + t_1^{-1}t_2y^{-1} + t_2^{-1}y^{-1} + q^{-1}y^{\frac 3 2} + qy^{\frac 3 2} \right).
\] 
In sum, the character of linear local operators is the geometric series 
\[
\sum_{(m_j)\in \Z^5_{\geq 0 }}-t_1^{-m_1+m_2}t_2^{-m_2+m_3}q^{-n_1+n_2}y^{m_1+m_2+m_3-\frac 3 2 (m_4+m_5)}\left (\begin{aligned}t_1y^{-1} + t_1^{-1}t_2y^{-1} + t_2^{-1}y^{-1} + q^{-1}y^{\frac 3 2} + qy^{\frac 3 2} \\  - (t_1^{-1}y + t_1t_2^{-1}y + t_2y + q^{-1}y^{-\frac 3 2} + qy^{-\frac 3 2})\end{aligned}\right).
\] 
The plethystic exponential returns the desired expression.

\end{proof}
\fi

It is easiest to label the Cartan subgroup of $SU(5)$ by fugacities $q_1,\ldots, q_5$ subject to the constraint that $\prod_{i=1}^5 q_i = 1$. 
We first compute the single particle index.
This is the $SU(5)$ character of the space of linear local operators.

\begin{lem}
The single particle index is 
\[
i(q_1,\ldots,q_5) = \frac{\sum_{i=1}^5 q_i}{\prod_{i=1}^5 (1-q_i)} + \frac{\sum_{i=1}^5 q_i^{-1}}{\prod_{i=1}^5 (1-q_i^{-1})}
\]
where the fugacities satisfy the constraint $\prod_{i=1} q_i = 1$. 
\end{lem}
\begin{proof}
The linear local operators $ \nu_{(m_j)}$ and $\beta_{(m_j)}$ are of the same $q$-weight but opposite parity.
Thus, they do not contribute to the single particle index.

The $q$-weight of the odd local operator $\mu_{(m_j)}^i$ is 
\[
q_1^{m_1+1} \cdots q_i^{m_i} \cdots q_5^{m_5+1} .
\]
The $q$-weight of the even local operator $\gamma_{(m_j)}^i$ is 
\[
q_1^{m_1} \cdots q_i^{m_i + 1} \cdots q_5^{m_5} .
\]

Thus we find that the single particle index is given by the infinite series
\beqn\label{infseriesindex}
\sum_{i=1}^5\left ( \sum_{(m_i)\in \Z^5} q_1^{m_1} \cdots q_i^{m_i + 1} \cdots q_5^{m_5} - \sum_{(m_i)\in \Z^5} q_1^{m_1+1} \cdots q_i^{m_i} \cdots q_5^{m_5+1} \right)
\eeqn

which sums to the expression
\beqn\label{singleparticleindex}
- \frac{\sum_{i=1}^5 q_1 \cdots \Hat{q_i} \cdots q_5}{\prod_{i=1}^5 (1-q_i)} + \frac{\sum_{i=1}^5 q_i}{\prod_{i=1}^5 (1-q_i)} .
\eeqn

This simplifies to the stated expression.
\end{proof}

This single particle index for our space of local operators agrees with the one computed in \cite{NekrasovInstanton}. 
To obtain the full index of local operators we apply the plethystic exponential ${\rm PE}[f(x)] = \exp\left(\sum_n \frac1n f(x^n)\right)$. 

\begin{prop}\label{prop:locchar}
The character of local operators of the eleven-dimensional theory on $\CC^5 \times \RR$ is 
\[
\prod_{i=1}^{5} \prod_{(m_i)\in \Z^5} \frac{1-q_1^{m_1+1}\cdots q_i^{m_i}\cdots q_5^{m_5+1}}{1-q_1^{m_1}\cdots q_i^{m_i+1}\cdots q_5^{m_5}}
\]
\end{prop}
\begin{proof}
Recall that the plethystic exponential takes sums to products and monomials to geometric series. Apply this to the infinite series \eqref{infseriesindex}.
\end{proof}
%\[
%\prod_ {(m_j)\in \Z^5_{\geq 0}}\frac{1- t_1^{-m_1+m_2}t_2^{-m_2+m_3}q^{-n_1+n_2}y^{m_1+m_2+m_3-\frac 3 2 (m_4+m_5)}\left (t_1^{-1}y + t_1t_2^{-1}y + t_2y + q^{-1}y^{-\frac 3 2} + qy^{-\frac 3 2} \right)}{1- t_1^{-m_1+m_2}t_2^{-m_2+m_3}q^{-m_4+m_5}y^{m_1+m_2+m_3-\frac 3 2 (m_4+m_5)}\left (t_1y^{-1} + t_1^{-1}t_2y^{-1} + t_2^{-1}y^{-1} + q^{-1}y^{\frac 3 2} + qy^{\frac 3 2} \right)}.
%\]


%The linear equations of motion for the superfield $\mu$ read 
%\[
%\dbar \mu^{i;j} + \d_{S} \mu^{i+1, j-1} = 0, \quad \div \mu^{i,j} = 0
%\]
%for all $i,j$. 
%The operator $\dbar$ is the anti-holomorphic Dolbeault operator acting on $\CC^5$, $\d_S$ is the de Rham operator on $S$, and $\div$ is the divergence with respect to the fixed holomorphic volume form on $\CC^5$. 
%In particular, this states that $\mu^{i;0}$ is locally constant along $S$ and holomorphic divergence-free as a Dolbeault valued vector field on $\CC^5$.

\subsection{One-loop quantization}

In \cite{GRWthf} an existence result for one-loop quantizations of mixed topological-holomorphic theories was established. 
We apply this to the eleven-dimensional model at hand. 

The eleven-dimensional theory is a mixed topological-holomorphic theory.
On flat space $\CC^5_z \times \RR_t$, this means that the theory is translation invariant and that the following act homotopically trivially:
\begin{itemize}
\item the vector fields $\del_{\zbar_1}, \ldots, \del_{\zbar_{5}}$ corresponding to infinitesimal anti-holomorphic translations,
\item the vector field $\partial_t$ corresponding to infinitesimal translations in the $\RR_t$ direction. 
\end{itemize}

Recall that the action functional of the eleven-dimensional theory is $S_{BF, \infty} + c J$. 
Since the cubic and higher interactions only involve holomorphic derivatives, we obtain the following directly from the main result of \cite{GRWthf}. 

\begin{thm}
There exists a gauge fixing condition for the eleven-dimensional theory on $\CC^5 \times \RR$ which renders its one-loop quantization finite and anomaly-free. 
\end{thm} 

When $g=0$, this result is actually exact,
since there are no Feynman diagrams present past one-loop order in this case. 
When $g \ne 0$, on the other hand, this result does not immediately imply the existence of a gauge-invariant perturbative quantization to higher orders in $\hbar$. 
The presence of the functional $J = \frac16 \int \gamma \del \gamma \del \gamma$ allows one to construct Feynman graphs at arbitrary loop order. 


%see \S \ref{s:omega},
In \cite{CostelloM5}, Costello argues that, upon performing the $\Omega$-background, the theory localizes to a five-dimensional theory on $\CC^2 \times \RR$. 
Via a cohomological argument, it is shown that this effective five-dimensional theory exhibits an essentially unique quantization in perturbation theory. 
We will return to the existence and uniqueness of a higher order quantization of the eleven-dimensional theory in future work. 


\section{Twisted matrix model}

\subsection{Global symmetry algebra}
\label{sec:global}

In any field theory, the cohomology classes of states of odd ghost number have the structure of a Lie algebra. 
More generally, after shifting the cohomological degree by one the full cohomology of states with respect to the linear BRST operator results in a graded Lie algebra. 
If we forget the grading to a $\ZZ/2$ grading then the global symmetry algebra has the structure of a super Lie algebra. 

In general, taking cohomology loses information. 
If the dg Lie, or $L_\infty$ algebra, we start with is not formal then there exists higher order operations present in the linearized BRST cohomology. 
We will refer to this as the global symmetry algebra of the theory.

Before taking cohomology with respect to the linear BRST operator, we described the super $L_\infty$ structure on the parity shift of the 11-dimensional fields in the previous section. 
This is encoded by the full BV action of the 11-dimensional theory.
The cubic component of the full BV action induces the super Lie algebra structure present in the linearized BRST cohomology. 

Our main result is to relate the global symmetry algebra of the minimal twist of 11-dimensional supergravity on $\CC^5 \times \RR$ to a certain infinite-dimensional exceptional super Lie algebra studied by Kac \cite{KacClass,KR} called $E(5,10)$.
We recall the definition below. 

\begin{thm}\label{thm:global}
Let $\Pi\cE(\CC^5 \times \RR)$ be the parity shift of the fields of 11-dimensional supergravity on $\CC^5 \times \RR$ and denote by $\delta^{(1)}$ the linearized BRST operator. 
\begin{enumerate}
\item 
As a super Lie algebra, the $\delta^{(1)}$-cohomology of $\Pi\cE(\CC^5 \times \RR)$ is isomorphic to the trivial one-dimensional central extension of the super Lie algebra $E(5,10)$.
\item 
The global symmetry algebra is equivalent, as a super $L_\infty$ algebra, to the non-trivial central extension of $E(5,10)$ determined by the even cocycle \eqref{eqn:cocycle}. 
\end{enumerate}
\end{thm}

This result implies that the action functional $S_{BF, \infty} + J$ of the 11-dimensional theory is invariant for infinite-dimensional Lie algebra $E(5,10)$. 

%defined by the even cocycle
%\begin{align*}
%E(5,10) \times E(5,10) \times E(5,10) & \to \CC \\
%(\mu,\mu',\alpha) & \mapsto \<\mu \wedge \mu', \alpha\>|_{z=0} .
%\end{align*} 

%The parity of the functional in the theorem is odd, but it is also trilinear. 
%Thus, as a cocycle in the Chevalley--Eilenberg complex of $E(5,10)$ it is of total even parity.

%The super Lie algebra $E(5,10)$ is very closely related to the super Lie algebra $E(5,10)$ studied by Kac; there is a dense embedding of super Lie algebras $E(5,10) \hookrightarrow E(5,10)$. 

\subsection{Linearized BRST cohomology} 

We compute the linearized BRST cohomology of 11-dimensional supergravity.
Then, we will describe the induced structure of a super Lie algebra present in the parity shift of the cohomology thus proving part (1) of Theorem \ref{thm:global}.

\parsec[]

First, we recall the definition of the exceptional simple super Lie algebra $E(5,10)$. 
Recall that $\Vect_0 (\CC^5)$ is the Lie algebra of divergence-free holomorphic vector fields on $\CC^5$.
Let $\Omega^{2}_{cl} (\CC^5)$ be the module of holomoprhic $2$-forms which are closed for the holomorphic de Rham operator $\del$.

The even part of the super Lie algebra $E(5,10)$ is the Lie algebra of divergence-free vector fields on $\CC^5$
\[
E(5,10)_+ = \Vect_0(\CC^5) ,
\]
whose elements we continue to denote by $\mu$. 
The odd piece is the module 
\[
E(5,10)_- = \Omega^{2}_{cl} (\CC^5) 
\]
whose elements we denote by $\alpha$. 
Besides the natural module structure, there is an odd $\times$ odd $\to$ even bracket. 
The bracket uses the isomorphism $\Omega^{-1} \vee (-) \colon \Omega^{4} \cong \Vect (\CC^5)$ induced by the standard Calabi--Yau form $\d^5 z$ and is defined by
\beqn\label{eqn:e510}
[\alpha, \alpha'] = \Omega^{-1} \vee (\alpha \wedge \alpha') .
\eeqn
One immediately checks that since both $\alpha, \alpha'$ are closed two-forms that the resulting vector field on the right hand side is divergence-free. 
In coordinates, if $f^{ij} \d z_i \wedge \d z_j$, $g^{kl} \d z_k \wedge \d z_l$ are two closed two-forms, their bracket is the vector field $\ep_{ijklm} f^{ij}g^{kl} \partial_{z_m}$. 

To be precise, Kac studied a more algebraic version of the algebra we have just introduced where holomorphic functions are replaced by holomorphic polynomials.
Thus, the simple super Lie algebra classified in \cite{KacBible} is a dense sub Lie algebra of what we call $E(5,10)$ consisting of those vector fields and two-forms which have polynomial coefficients.

\parsec[]

If $\cE$ is the space of fields of any theory in the BV or BRST formalism, the shift $\cL = \cE[-1]$ has the structure of a Lie, possibly $L_\infty$ algebra. 
In the $\ZZ/2$ graded world, the parity shifted object $\cL = \Pi \cE$ has the structure of a super $L_\infty$ algebra. 

In this section, we use the description of the 11-dimensional theory as the deformation of the BF action $S_{BF,\infty}$ by the functional $J$ of Theorem \ref{thm:dfn}. 
We set the coupling $g = 1$, for any other nonzero value of $g$ we will obtain an isomorphic super $L_\infty$ algebra.

We will obtain the exact same results if we use the equivalent model of the 11-dimensional theory explained in \S \ref{s:altdfn}. 
We remark on this below in \S \ref{s:altglobal}. 

The full BRST operator is given by the BV bracket with the BV action. 
For us, this~is 
\[
\delta = \{S_{BF,\infty} + J, -\} .
\]
The linear BRST operator comes only from the quadratic summands in $S_{BF,\infty}$ and is of the form
\beqn\label{eqn:linearBRST}
\delta^{(1)} = \dbar + \d_{\RR} + \div |_{\mu \to \nu} + \del |_{\beta \to \gamma} .
\eeqn

To compute the cohomology with respect to $\delta^{(1)}$ we can use a spectral sequence, first taking the cohomology with respect to $\dbar + \d_{\RR}$ and then with respect to $\div$. 
By the $\dbar$ and de Rham Poincar\'e lemmas, the cohomology of the space of fields of the 11-dimensional theory on $\CC^5 \times \RR$ with respect $\dbar + \d_{\RR}$ results in the cochain complex
\begin{equation}
  \label{eq:lin1} 
  \begin{tikzcd}[row sep = 1 ex]
    - & + \\ \hline
    \Vect(\CC^5) \ar[r, "\div"] & \cO(\CC^5) \\ 
     \cO(\CC^5) \ar[r, "\del"] & \Omega^{1}(\CC^5).
\end{tikzcd}
\end{equation}
Recall that $\Vect(\CC^5), \cO(\CC^5)$, and $\Omega^1(\CC^5)$ denote the space of holomorphic vector fields, functions, and one-forms, respectively.

The cohomology with respect to the remaining linearized BRST operator consists of the space of triples $(\mu, [\gamma], b)$ where:
\begin{itemize}
\item $\mu$ is a divergence-free holomorphic vector field on $\CC^5$, which is constant along $\RR$
\[
\mu = \mu \otimes 1 \in \Pi \Vect_0(\CC^5) \otimes \Omega^0(\RR) .
\]
Note that $\mu$ is a ghost in the $\ZZ/2$ graded theory. 
\item $[\gamma]$ is an equivalence class of a holomorphic one-form modulo exact holomorphic one-forms along $\CC^5$, which are also constant along $\RR$
\[
[\gamma] = [\gamma] \otimes 1 \in \left(\Omega^{1}(\CC^5) / \d \cO(\CC^5) \right) \otimes \Omega^0(\RR) .
\]
\item A constant function $b \in \Pi \CC$ on $\CC^5 \times \RR$.
This is a $\beta$-type field in the 11-dimensional theory, any constant function is closed for the de Rham differential. 
This element is also a ghost in the $\ZZ/2$-graded theory. 
\end{itemize}

\parsec[]

After parity shifting, we've identified the solutions to the linear equations of motion with triples
\[
(\mu, [\gamma], b) \in \Vect_0(\CC^5) \oplus \Pi \Omega^{1}(\CC^5) / \del \cO(\CC^5) \oplus \CC .
\]
The bracket induced by the cubic component of $S_{BF, \infty}$ in the classical BV action is the usual bracket on divergence-free vector fields together with the module structure on holomorphic one-forms by Lie derivative.
Notice that the Lie derivative commutes with the $\del$ operator, so this action descends to equivalence classes as above. 
The elements $b$ are central. 

The final term in the BV action $J = \frac16\int \gamma \wedge \del \gamma \wedge \del \gamma$ induces the following Lie bracket on the solutions to the linearized equations of motion
\beqn\label{eqn:eqb}
\big[[\gamma], [\gamma'] \big] = \Omega^{-1} \vee (\del \gamma \wedge \del \gamma') \in \Vect_0(\CC^5) .
\eeqn
where $\Omega^{-1}$ denotes the section of $\PV^{5,hol}(\CC^5)$ which is inverse to the Calabi--Yau form $\Omega$ on $\CC^5$. 
Notice that this bracket is well-defined as it does not depend on the particular equivalence classes and that the resulting vector field is automatically divergence-free.

\parsec[]

We complete the proof of the first part of Theorem \ref{thm:global}. 
For this we write down an explicit map between the cohomology computed above and the algebra $E(5,10)$. 

The relationship of the $\mu$-elements in $E(5,10)$ and the 11-dimensional theory is apparent.

Next, we need to relate the equivalence classes $[\gamma]$ with the closed two-forms $\alpha$ in $E(5,10)$. 
On flat space, any closed differential form is exact (this is a holomorphic version of the Poincar\'e lemma). 
In other words, there is an isomorphism
\[
\del \colon \Omega^1 (\CC^5) / \d \cO(\CC^5) \xto{\cong} \Omega^{2}_{cl}(\CC^5)
\]
induced by the holomorphic de Rham differential.
This gives the relationship between the equivalence class $[\gamma]$ in the 11-dimensional theory and a closed two-form in $E(5,10)$ by $\alpha = \del \gamma$. 
It is clear from Equations \eqref{eqn:e510} and \eqref{eqn:eqb} that this assignment intertwines the Lie brackets in $E(5,10)$ and the twist of 11-dimensional supergravity. 
This completes the proof of part (1) of Theorem \ref{thm:global}.

\subsection{(Non)formality and homotopy transfer} 
\label{s:ht}

We produce the following homotopy data:
\begin{equation}
\begin{tikzcd}
\arrow[loop left]{l}{K}(\Pi \cE , \delta^{(1)})\arrow[r, shift left, "q"] &(E(5,10) \oplus \CC_b \, , \, 0)\arrow[l, shift left, "i"] \: ,
\end{tikzcd}
\end{equation}

\begin{itemize}
\item On the $\nu$'s we take $K$ be any operator $K \colon \cO \to \Vect$ such that $\div K \nu = \nu$. 
On the $\gamma$'s we take $K$ be any operator $K \colon \Omega^1 \to \Omega^0$ such that $\del K(\gamma) = \gamma$. 
Also, introduce the auxiliary operator $\til{K} \colon \Omega^2_{cl} \to \Omega^1$ which satisfies the homotopy relation
\beqn\label{eqn:htpy1}
\til{K} \del \gamma + \del K \gamma = \gamma . 
\eeqn
The precise form of each of these operators will not be needed.
The existence of such operators is guaranteed by the holomorphic Poincar\'e lemma.
The operator $K$ annihilates fields $\beta$ and $\mu$. 
\item 
The map $q$ is described as follows. 
First $q(\mu) = \mu - K \div (\mu)$.
Notice that $q(\mu)$ is automatically divergence-free.
Next, $q(\gamma) = [\gamma]$, the equivalence class in $\Omega^1 / \d$. 
If $\beta$ is a holomorphic function, then $q(\beta) = \beta (z=0)$.
\item 
The map $i$ embeds $\mu$ and $b$ in the obvious way.
On the equivalence class $[\gamma] \in \Omega^1 / \d$ we define $i([\gamma]) = \gamma - \til{K} \del \gamma$. 
Notice that this is independent of the choice of representative $\gamma$. 
\end{itemize}

It is straightforward to check that this comprises well-defined homotopy data, the only nontrivial thing to check is the relation $\id - i \circ q = \delta^{(1)} K - K \delta^{(1)}$. 
Plugging in the field $\gamma$ we see that we must check that
\[
\gamma - \til{K} \del \gamma = \del K \gamma 
\]
which is precisely \eqref{eqn:htpy1}. 

Given this homotopy data, we can compute the homotopy transferred $L_\infty$ structure on the linearized BRST cohomology. 
Since $\nu$ does not survive to cohomology and the fact that there are no nontrivial Lie brackets involving the field $\beta$, this transferred structure is easy to compute. 

There is a single diagram which contributes to the transferred structure, it is given by
\begin{equation}
\begin{tikzpicture}
\begin{feynman}
%\vertex at (-2,0) {$\mu'_3 \ = $};
\vertex(a) at (-1,1) {$i(\mu)$};
\vertex(b) at (-1,0) {$i([\gamma])$};
\vertex(c) at (-1,-1) {$i(\mu')$};
\vertex(d) at (0,0.5);
\vertex(e) at (1,0);
\vertex(f) at (2,0) {$q$};
\diagram* {(a)--(d), (b)--(d), (d)--[edge label = $K$](e), (c)--(e), (f)--(e)};
\end{feynman}
\end{tikzpicture}
\end{equation}
together with a similar diagram with the $\mu$ and $\mu'$ flipped. 

This diagram leads to a new $3$-ary bracket on $E(5,10) \oplus \CC_b$
\[
\big[\mu,\mu',[\gamma]\big]_3 = \varphi(\mu,\mu',[\gamma])
\]
where $\varphi \in \clie^{even} (E(5,10))$ is the even Lie algebra cocycle defined by the formula
\beqn
\begin{array}{rclr}
\varphi \colon E(5,10) \times E(5,10) \times E(5,10) & \to & \CC_b \\
\varphi(\mu,\mu',\alpha) & = & \<\mu \wedge \mu', \alpha\>|_{z=0} .
\label{eqn:cocycle}
\end{array}
\eeqn
Since $b$ is central, this cocycle defines a central extension of $E(5,10)$. 

\parsec[]
We briefly remark on Lie algebra cohomology for super Lie algebras.
The Lie algebra cohomology $\clie^{\bu,\bu}(\cL)$ of any super Lie algebra $\cL$ is graded by $\ZZ \times \ZZ/2$. 
The first grading is by the symmetric degree in the Chevalley--Eilenberg complex.
The second grading is the internal parity of the super Lie algebra $\cL$. 
The Chevalley--Eilenberg differential is degree $(1,+)$. 

The cocycle $\varphi$ has homogenous bigrading $(3,-)$.
In the above discussion we forgot the bigrading to a totalized $\ZZ/2$ grading where 
\begin{align*}
\clie^{even} (\cL) & = \clie^{2\bu , +} (\cL) \oplus \clie^{2\bu+1, -}(\cL) \\
\clie^{odd} (\cL) & = \clie^{2\bu , -} (\cL) \oplus \clie^{2\bu+1, +}(\cL) .
\end{align*}
With this totalization, $\varphi$ is an even cocycle and hence determines a super $L_\infty$ central extension by the one-dimensional even vector space $\CC$. 

\parsec[s:altglobal]

In \S \ref{s:altdfn} we gave an equivalent description of the 11-dimensional theory as a deformation of the BF action $S_{BF,0}$ by the functional $\til J$. 

\subsection{Twisted matrix model}

\section{Residual supersymmetry} 
\label{sec:susy}
%m2brane

In this section we consider the minimal twist of 11-dimensional supersymmetry explicitly. 
We compute the residual supersymmetry algebra given by taking the cohomology of the 11-dimensional supersymmetry algebra with respect to the minimal twisting supercharge. 
In order for this to be a symmetry of the 11-dimensional theory it is necessary to perform a central extension of the 11-dimensional supersymmetry algebra by the $M2$ brane.
We will see how this central extension is compatible, upon twisting by the minimal supercharge, with the central extension of $E(5,10)$ we found as the global symmetry algebra in the previous section. 

\subsection{Supersymmetry in 11 dimensions}
\label{sec:11dsusy}

The (complexified) eleven-dimensional supertranslation algebra is a complex super Lie algebra of the form
\[
  \ft_{11d} = V \oplus \Pi S
\]
where $S$ is the (unique) spin representation and $V \cong \CC^{11}$ the complex vector representation, of~$\lie{so}(11, \CC)$. 
The bracket is the unique surjective $\lie{so}(11,\CC)$-equivariant map from the symmetric square of~$S$ to~$V$;
this decomposes into three irreducibles, 
\beqn\label{eqn:decomp}
  \Sym^2(S) \cong V \oplus \wedge^2 V \oplus \wedge^5 V.
\eeqn
Denote by $\Gamma_{\wedge^1}, \Gamma_{\wedge^2}, \Gamma_{\wedge^5}$ the projections onto each of the summands above. 
The bracket in $\ft_{11d}$ is defined using the first projection
\[
[\psi, \psi'] = \Gamma_{\wedge^1} (\psi, \psi') .
\]
The super Poincar\'{e} algebra is
\[
  \lie{siso}_{11d} = \lie{so}(11 , \CC) \ltimes \ft_{11d} .
\]
The $R$-symmetry is trivial in 11-dimensional supersymmetry. 

\subsection{Central extensions of the supersymmetry algebra} 
\label{sec:m2brane}

Extensions of the supersymmetry algebra correspond to the existence of extended objects, such as branes, in the supergravity theory.
In 11-dimensional supersymmetry, there are two such extensions corresponding to the $M2$ brane and the $M5$ brane.
We begin by describing a less standard dg Lie algebra model for the $M2$ brane algebra.
In the next section we will explain the relationship to other descriptions in terms of $L_\infty$ algebras \cite{Basu_2005,Bagger_2007,fiorenza2015super}. 

Our model for the $M2$ brane algebra is a dg Lie algebra extension of the super Poincar\'e algebra $\lie{siso}_{11d}$.
 
Introduce the cochain complex $\Omega^{\bu}(\RR^{11})$ of (complex valued) differential forms on $\RR^{11}$ equipped with the de Rham differential $\d$.
The $M2$ brane algebra arises as a central extension of $\lie{siso}_{11d}$ by $\Omega^\bu(\RR^{11})[2]$ and is defined by a cocycle
\[
    c_{M2} \in \clie^{2,+} \left(\lie{siso}_{11d} \; ; \; \Omega^\bu (\RR^{11})[2]\right) .
\]
The formula is
  \[c_{M2} (\psi, \psi') = \Gamma_{\wedge^2}(\psi, \psi') \in \Omega^2(\RR^{11})\]
  where $\Gamma_{\wedge^2}$ is the projection onto $\wedge^2 V$, thought of as the space of constant coefficient two-forms, as in the decomposition \eqref{eqn:decomp}.
  
Here, we are using a bigrading by $\ZZ \times \ZZ/2$. 
The super Poincar\'e algebra is concentrated in zero integer grading and carries is natural $\ZZ/2$ grading as a super Lie algebra.
The complex $\Omega^{\bu}(\RR^{11})[2]$ is concentrated in integer degrees $[-2,9]$ and has even parity.

\begin{dfn}
The algebra $\m2$ is the $\ZZ \times \ZZ/2$-graded dg Lie algebra defined by the central extension of $\lie{siso}_{11d}$ by the cocycle $c_{M2}$.  
\end{dfn}

The bracket in $\m2$ is bidegree $(0,+)$ and the differential is bidegree $(1,+)$.

\subsection{The minimal twist}
\label{sec:mintwist}

Fix a rank one supercharge $Q \in S$ satisfying $Q^2 = 0$.
Such a supercharge has a six-dimensional image in the space of (complexified) translations on $\RR^{11}$ and defines the minimal twist of 11-dimensional supersymmetry \cite{SWspinor}. 
We characterize the cohomology of the algebra $\m2$ with respect to this supercharge. 

$Q$ defines a maximal isotropic subspace $L \subset V$. 
In turn, we will decompose the super Poincar\'e algebra into $\lie{sl}(L) = \lie{sl}(5)$ representations.
First, the defining and spinor representations decompose as
\deq{
  V = L \oplus L^\vee \oplus \CC_t, \qquad S = \wedge^\bu L^\vee.
}
In the expression for $S$, we are omitting factors of $\det(L)^{\frac12}$ for simplicity. 
Also, $\lie{so}(11, \CC)$ decomposes as
\[
\lie{sl}(5) \oplus \wedge^2 L \oplus \wedge^2 L^\vee \oplus L \oplus L^\vee \oplus \C .
\]
Furthermore, the spinorial representation can be identified with
\[
S = \wedge^\bu (L^\vee) = \CC \oplus L^\vee \oplus \wedge^2 L^\vee \oplus \wedge^3 L^\vee \oplus \wedge^4 L^\vee \oplus \wedge^5 L^\vee .
\]
The element $Q$ lives in the first summand.
Let ${\rm Stab}(Q) \subset \lie{so}(11,\CC)$ be the stabilizer of $Q$. 
This is a parabolic subalgebra whose Levi factor is $\lie{sl}(5)$.

\subsection{$Q$-cohomology of $\m2$}
\label{sec:m2branetwist}

Any element $Q \in S$ satisfying $Q^2 = 0$ determines a deformation of the dg Lie algebra $\m2$.
To deform $\d$ by $Q$ we must break the $\ZZ \times \ZZ/2$ bigrading.
The supercharge $Q$ is odd and of cohomological degree zero.
Recall, the original differential on $\m2$ is the de Rham differential $\d$ which just acts on the central summand and is even of cohomological degree $+1$.
Thus, only the totalized $\ZZ/2$ grading makes the differential $\d + [Q,-]$ homogenous. 

\begin{dfn}
The $Q$-twist $\m2^Q$ of $\m2$ is the super dg Lie algebra whose differential is $\d + [Q,-]$.
The bracket is unchanged.
\end{dfn}

We now assume that $Q$ is a rank one, or minimal, supercharge satisfying $Q^2 = 0$. 

\begin{prop}\label{prop:susycoh}
As a $\ZZ/2$ graded space, the cohomology of the $Q$-twist $\m2^Q$ is
\beqn\label{eqn:susycoh}
L \oplus {\rm Stab}(Q) \oplus \Pi \left(\wedge^2 L^\vee\right) \oplus \CC
\eeqn
whose elements we denote by $(v, m, \psi, c)$.

\begin{enumerate}
\item As a super Lie algebra, the cohomology of $\m2^Q$ is the natural extension of ${\rm Stab}(Q)$ together with the bracket
\beqn\label{eqn:susy2bra}
[\psi, \psi']_2 = \psi \wedge \psi' \in \wedge^4 L^\vee \cong L_v \\
\eeqn
\item 
$\m2^Q$ is not formal as a super dg Lie algebra.
As a super $L_\infty$ algebra, the $Q$-twist is equivalent to \eqref{eqn:susycoh} with $2$-brackets described in (1) where we additionally introduce the $3$-ary bracket 
\beqn\label{eqn:susy3bra}
[v, v', \psi]_3 = 4 \<v \wedge v', \psi\> \in \CC_b .
\eeqn
\end{enumerate}
\end{prop}

It will be useful to list the formulas for the brackets in terms of coordinates. 
Let $\{z_i\}$ denote a basis for $L$, which we will also think of as a linear coordinate on $\CC^5$. 
Let $\{\partial_{z_i}\}$ be a dual basis for $L^\vee$, which we will also think of as translation invariant vector fields.
The $2$-ary bracket above is 
\[
[z_i \wedge z_j, z_k \wedge z_l]_2 = \ep_{ijklm} \partial_{z_m} 
\]
and the $3$-ary bracket is
\[
[\partial_{z_i}, \partial_{z_j}, z_{k} \wedge z_{\ell}]_3 = 4 (\delta^i_k \delta^j_\ell - \delta^i_\ell \delta^j_k) .
\] 

\parsec[]

One outcome of this proposition is that the dg Lie algebra $\m2^Q$ is {\em not} formal; there is a 3-ary $L_\infty$ bracket present in cohomology. 
One way to prove the proposition above is to use homotopy transfer directly to $\m2^Q$, just as we did in \S \ref{s:ht} to deduce the form of the $3$-ary bracket. 
Instead, we will use the following minimal model for $\m2^Q$ to prove Proposition \ref{prop:susycoh}.
This minimal model also has the advantage of being more directly related to the 11-dimensional supergravity theory.

\begin{lem}
\label{lem:gmodel}
Let $\fg$ denote the following $\ZZ/2$ graded dg Lie algebra which as a cochain complex is
\[
H^\bu(\m2^Q) \oplus (L^\vee \xto{\id} \Pi L^\vee)  .
\]
Denote the elements of the second summand by $(\lambda, \til\lambda)$. 
The Lie structure extends the one on $H^\bu(\m2^Q)$ described in (1) of Proposition \ref{prop:susycoh} together with the brackets
\begin{align*}
[v,\lambda] & = \<v, \lambda\> \in \CC_b \\ 
[v,\psi] & = \<v, \psi\> \in \Pi L^\vee_{\Tilde{\lambda}}.
\end{align*}

There is an $L_\infty$ map 
\[
\fg \rightsquigarrow \m2^Q
\] 
which is a quasi-isomorphism of cochain complexes.  
\end{lem}
\begin{proof}

The cohomology of the non-centrally extended algebra was computed in \cite{SWspinor}, we briefly recall the result. 
The element $Q$ only acts nontrivially on the summands $\wedge^4 L$ and $\wedge^5 L$ in $S$. 
The image of $\wedge^4 L \cong L^\vee$ trivializes the antiholomorphic translations while the image of $\wedge^5 L$ trivializes the time translation.
So, of the translations, only the holomorphic ones, which live in $L$, survive.
The map 
\[
[Q,-] \colon \lie{so}(11,\CC) \to S 
\] 
is the projection onto $\wedge^0 L \oplus \wedge^1 L \oplus \wedge^2 L$. 
The kernel of $[Q,-]$ is the stabilizer~${\rm Stab}(Q)$.

In summary, the space of odd translations which survive cohomology is $\wedge^3 L \cong \wedge^2 L^\vee$.
This completes the calculation of the cohomology. 

We embed $\fg$ into $\m2^Q$ in the following way: ${\rm Stab}(Q)$ and $L$ sit inside in the evident way.
The central element maps to $c \mapsto - 1 \in \Omega^0(\RR^{11})$.
The summand $L_\lambda$ is mapped to the linear functions in $\Omega^0(\RR^{11})$ and $\Pi L_{\Tilde{\lambda}}$ is sent to the constant coefficient one-forms in $\Pi \Omega^1(\RR^{11})$. 
It remains to define where $\psi \in \wedge^2 L$ is mapped.

Notice that naively, $\psi \in \wedge^2 L$ is not $Q$-closed due to the presence of the central extension. 
To embed $\wedge^2 L$ we introduce the following operator
\[
H \colon \Omega^2 (\RR^{11}) \to \Omega^1(\RR^{11})
\]
which sends a two-form $\alpha$ to the one-form $H \alpha$ defined by the formula $(H \alpha) (x) = \int_0^x \alpha$
where we integrate over a straight line path from $0$ to $x$.

Notice that if $\alpha$ is $\d$-closed then $\d (H \alpha) = \alpha$. 
It follows that any element $\psi \in \wedge^2 L \subset S$ can be lifted to a closed element at the cochain level in $\m2^Q$ by the formula
\[
\Tilde{\psi} = \psi - H \Gamma_{\wedge^2} (Q, \psi) \in \Pi S \oplus \Pi \Omega^1 .
\]
Thus, sending $\psi \mapsto \Tilde{\psi}$ defines a cochain map $\fg \to \m2^Q$. 

The Lie bracket $[\Tilde{\psi}, \Tilde{\psi}']$ agrees with $[\psi, \psi']$. 
On the other hand, in $\m2^Q$ there is the Lie bracket 
\[
[v,\Tilde{\psi}] = - L_v (H \Gamma_{\wedge^2} (Q, \psi)) = -\<v, \Gamma_{\wedge^2}(Q, \psi)\> - \d \<v, H \Gamma_{\wedge^2}(Q, \psi)\> .
\]
The first term agrees with the bracket $[v, \psi]_{\fg}$ in $\fg$. 
The other term is exact in $\m2^Q$ and can hence be corrected by the following bilinear  
\[
v \otimes \psi \mapsto \<v, H \Gamma_{\wedge^2} (Q,\psi) \> \in L_\lambda .
\] 
Together with the cochain map described above, this bilinear term prescribes the desired $L_\infty$ map. 

\end{proof}

\parsec[]

Using the model $\fg$ the first part of Proposition \ref{prop:susycoh} follows immediately. 
We deduce the second part using homotopy transfer. 

Recall that we described the cohomology of $\m2^Q$ in \eqref{eqn:susycoh}.
Let $\delta$ denote the differential on $\fg$ which simply maps $\Pi L$ to $L$ by the identity map. 
We produce the homotopy data
\begin{equation}
\begin{tikzcd}
\arrow[loop left]{l}{K}(\fg , \delta)\arrow[r, shift left, "q"] &(H^\bu(\m2^Q) \, , \, 0)\arrow[l, shift left, "i"] \: ,
\end{tikzcd}
\end{equation}
as follows.
\begin{itemize}
\item The operator $K$ annihilates $H^\bu(\m2^Q)$ and is the identity map~$K \colon \Pi L_{\til \lambda} \to L_\lambda$. 
\item The map $q$ is the identity on $H^\bu(\m2^Q)$ and annihilates the summand~$L \to \Pi L$. 
\item The map $i$ embeds $H^\bu(\m2^Q)$ in the obvious way. 
\end{itemize}

It is immediate to verify this data prescribes valid homotopy data.
There is only a single term in the $L_\infty$ structure generated by homotopy transfer. 
It is determined by the following tree diagram
\begin{equation}
\begin{tikzpicture}
\begin{feynman}
%\vertex at (-2,0) {$\mu'_3 \ = $};
\vertex(a) at (-1,1) {$i(v)$};
\vertex(b) at (-1,0) {$i(\psi)$};
\vertex(c) at (-1,-1) {$i(v)$};
\vertex(d) at (0,0.5);
\vertex(e) at (1,0);
\vertex(f) at (2,0) {$q$};
\diagram* {(a)--(d), (b)--(d), (d)--[edge label = $K$](e), (c)--(e), (f)--(e)};
\end{feynman}
\end{tikzpicture}
\end{equation}
together with a similar diagram with the $v$ and $v'$ reversed. 
It is an immediate calculation to show that these trees recover the formula in (2) of Proposition \ref{prop:susycoh}.

\subsection{Embedding supersymmetry into the 11-dimensional theory} \label{s:residual}

Consider now the super $L_\infty$ algebra $\cL$ underlying the eleven-dimensional theory on $\CC^5 \times \RR$. 

\begin{prop}
Endow the cohomology of $\m2^Q$ with the $L_\infty$ structure of Proposition \ref{prop:susycoh} and let $\cL(\CC^5 \times \RR)$ be the super $L_\infty$ algebra underlying 11-dimensional supergravity on $\CC^5 \times \RR$. 
There is a map of super $L_\infty$ algebras 
\[
H^\bu(\m2^Q) \rightsquigarrow \cL (\CC^5 \times \RR)
\]
%where $\fg$ is the model for the $Q$-cohomology of the super Lie algebra $\m2$ from Proposition \ref{lem:gmodel}. 
In particular, the $Q$-twisted algebra $\m2^Q$ is a symmetry of 11-dimensional theory on $\CC^5 \times \RR$. 
\end{prop}
\begin{proof}
Recall the cohomology of $\m2^Q$ takes the following form
\beqn 
\begin{tikzcd}
\ul{even} & \ul{odd} & \ul{even} \\
 L^\vee & \wedge^2 L^\vee_2 & L \\
\wedge^2 L_1^\vee & & \\
\lie{sl}(5) && \CC_b  \\
%L_2 \ar[r, "\id"] & L_3 \\ 
 .
\end{tikzcd}
\eeqn
The lefthand column is ${\rm Stab}(Q)$. 
The subscripts in $\wedge^2 L_1, \wedge^2 L_2$ are used to distinguish between the two copies of $\wedge^2 L$.

The $L_\infty$ map from the dg Lie model $\fg$ to the fields of the twisted $11$-dimensional supergravity theory has a linear $\Phi^{(1)}$ and quadratic $\Phi^{(2)}$ piece.

Define the linear map $\Phi^{(1)} \colon \fg \to \cL$ as follows:
\begin{align*}
 L^\vee & \mapsto 0 \\
 \wedge^2 L_1^\vee  & \mapsto 0 \\
z_i \wedge z_j \in \wedge^2 L^\vee_2 & \mapsto \frac12 (z_i \d z_j - z_j \d z_i) \in \Omega^{1,0} (\CC^5) \hotimes \Omega^0 (\RR) \\
A_{ij} \in \lie{sl}(5) & \mapsto \sum_{ij} A_{ij} z_i \partial_{z_j} \in \PV^{1,0}(\CC^5) \hotimes \Omega^0(\RR) \\ \partial_{z_j} \in L & \mapsto
\partial_{z_i} \in \PV^{1,0} (\CC^5) \hotimes \Omega^0 (\RR^5) \\ %z_i \in %L_2 & \mapsto z_i \in \Omega^{0,0}(\CC^5) \hotimes \Omega^0 (\RR) \\
%z_i \in L_3 & \mapsto \d z_i \in \Omega^{1,0}(\CC^5) \hotimes \Omega^0 (\RR) \\
1 \in \CC_b & \mapsto 1 \in \Omega^{0,0}(\CC^5) \hotimes \Omega^0 (\RR) .
\end{align*}

It is immediate to check that this is a map of cochain complexes since all elements in the image of this map lie in the kernel of the linearized BRST operator \eqref{eqn:linearBRST}. 

This map also preserves the bracket between odd elements in $\wedge^2 L_2^\vee$. 
In the cohomology of $\m2^Q$ we have the bracket
\[
[z_i\wedge z_j , z_k \wedge z_l] = \ep_{ijklm} \partial_{z_m}
\]
which is precisely the bracket induced by the cubic term in the action $J = \frac16 \in \gamma \del \gamma \del \gamma$. 

This map does not preserve all of the brackets, however. 
Indeed, in the 11-dimensional theory $\cL(\CC^5 \times \RR)$ there is the bracket 
\[
\left[\partial_{z_i}, z_j \d z_k - z_k \d z_j\right] = \delta^i_j \d z_k - \delta^i_k \d z_j 
\]
arising from the cubic term in $\frac12 \int \frac{1}{1-\nu} \mu^2 \del \gamma$. 
To remedy the failure for $\Phi^{(1)}$ to preserve the brackets, we introduce the odd bilinear map $\Phi^{(2)} \colon \fg \times \fg \to \Pi \cL$ defined by 
\beqn\label{eqn:phi2}
\Phi^{(2)} \left(\partial_{z_i} , z_j \wedge z_k\right) = \frac12 (\delta^i_j z_k - \delta^i_k z_j) .
\eeqn
Notice that the field on the right hand side is of type $\beta$. 

The bilinear map $\Phi^{(2)}$ provides a homotopy trivialization for the failure for $\Phi^{(1)}$ to preserve the $2$-ary bracket: 
\[
[\Phi^{(1)} (\partial_{z_i}) , \Phi^{(1)}(z_j \wedge z_k)] = \del \Phi^{(2)}\left(\partial_{z_i} , z_j \wedge z_k\right).
\]
The lefthand side is $\frac12 (\delta_j^i \d z_k - \delta_k^i \d z_j)$ which is precisely the de Rham differential applied to \eqref{eqn:phi2}.

To define an $L_\infty$ morphism $\Phi^{(1)} + \Phi^{(2)}$ must satisfy additional higher relations. 
There is a single nontrivial cubic relation to verify:
\begin{multline} \label{eqn:cubicrln}
\Phi^{(1)}\left[\partial_{z_i}, \partial_{z_j}, z_k \wedge z_l\right]_3 = [\Phi^{(1)}(\partial_{z_i}), \Phi^{(1)}(\partial_{z_i}), \Phi^{(1)}(z_k \wedge z_l)]_3 \\ + [\del_{z_i}, \Phi^{(2)}(\partial_{z_j}, z_k \wedge z_l)] + [\del_{z_j}, \Phi^{(2)}(\del_{z_i}, z_k \wedge z_l)]
\end{multline}
where $[-]_3$ on the left hand side is the $3$-ary bracket defined in Proposition \ref{prop:susycoh} and $[-]_3$ on the right hand side is the $3$-ary bracket defined by the quartic part of the action $\frac12 \int \frac{1}{1-\nu} \mu^2 \vee \del \gamma$. 
The two terms in the second line of \eqref{eqn:cubicrln} cancel for symmetry reasons and the quartic term in the BV action induces precisely the correct $3$-ary bracket. 

%$[\Phi^{(1)}(z_i \wedge z_j), \Phi^{(1)}(z_k \wedge z_l), \Phi^{(1)} (z_m \wedge z_n)]_3 = \Phi^{(2)} ([z_i \wedge z_j, z_k\wedge z_l)], z_m \wedge z_n) + $permutations, where $[-]_3$ is the $3$-ary bracket arising from the 
\end{proof}

\parsec[]

Because this map preserves differentials, it descends to a map in cohomology. 
We have already computed the cohomology of $\cL$ on $\CC^5 \times \RR$, it is the trivial one-dimensional central extension of $E (5,10)$. 
The Lie algebra structure present in the cohomology of $\m2^Q$ is described in part (1) of Proposition \ref{prop:susycoh}. 
The map
\[
L \oplus {\rm Stab}(Q) \oplus \Pi \left(\wedge^3 L \right) \oplus \CC_b \to E (5,10) \oplus \CC_{b'}
\]
is defined by very similar formulas as above
\begin{align*}
 L^\vee_1 & \mapsto 0 \\
 \wedge^2 L^\vee_1 & \mapsto 0 \\
z_i \wedge z_j \in \wedge^2 L_2 & \mapsto \d z_i \wedge \d z_j \in \Omega^{2}_{cl} (\CC^5) \\
A_{ij} \in \lie{sl}(5) & \mapsto \sum_{ij} A_{ij} z_i \partial_{z_j} \in \Vect_0(\CC^5) \\ \partial_{z_i} \in L & \mapsto
\partial_{z_i} \in \Vect_0(\CC^5) \\
b \in \CC_b & \mapsto b \in \CC_{b'} .
\end{align*}

The relationship between the transferred $L_\infty$ structures can be described as follows. 
Recall, that the linear BRST cohomology of the parity shift of the fields of the 11-dimensional theory is equivalent to the super $L_\infty$ 
algebra $\Hat{E(5,10)}$ which is a central extension $E(5,10)$ by the cocycle \eqref{eqn:cocycle}.
Also, we described the $L_\infty$ structure present in the cohomology of $\m2^Q$ in part (2) of Proposition \ref{prop:susycoh}. 
Each of these $L_\infty$ structure involved introducing a single new $3$-ary bracket, which are easily seen to be compatible. 

\parsec[]

In this section we compare to another description of the $M2$ brane algebra given as a one-dimensional $L_\infty$ central extension of the super Poincar\'e algebra.
Such central extensions were studied in the work \cite{BHsusyII, SSS, FSS} following \cite{CDF}. 

In these references, the algebra $\m2$ is defined as an $L_\infty$  
central extension of $\lie{siso}_{11d}$. 
Recall that given two spinors $\psi, \psi' \in S$ we can form the constant coefficient two-form $\Gamma_{\wedge^2} (\psi, \psi')$. 
Using this two-form we can define the following four-linear expression
\[
\mu_2 (\psi, \psi',v,v') = \<v \wedge v', \Gamma(\psi, \psi')\> .
\]
This expression is symmetric on the spinors and anti-symmetric on the vectors, therefore it defines an element in $\clie^4(\lie{siso}_{11d})$. 
This expression defines a nontrivial class in $H^4(\lie{siso}_{11d})$ so defines a one dimensional central extension of $\lie{siso}_{11d}$ as a Lie 3-algebra. 
Instead of working with a one-dimensional central extension by $\CC[2]$, we work with a central extension by the resolution $\Omega^\bullet(\RR^{11})[2]$. There is a quasi-isomorphism $\clie^4(\lie{siso}_{11d})\to \clie^4(\lie{siso}_{11d}, \Omega^\bullet (\R^{11}))$ such that the induced map on cohomology identifies $\mu_2$ with the $\Omega^\bullet(\RR^{11})[2]$-valued two-cocycle we use. 

%For clarity we adjust notation for $\lie{sl}(5)$-representations. Denote by $V^{1,0} = L^\vee$ the space of holomorphic translations on $\CC^5$ and $V^{\vee 1,0}$ the translation invariant holomorphic one-forms on $\CC^5$. 

\def\im{{\rm i}}

\section{Dimensional reduction and 10-dimensional supergravity}

In this section we demonstrate that our proposal for the action of minimally twisted 11-dimensional supergravity agrees with conjectural descriptions of twisted type IIA and type I supergravities due to Costello and Li. 

The original motivation for $M$-theory was as the strong coupling limit for type IIA string theory.
Roughly, the radius of the $M$-theory circle plays the role of this coupling constant. 
Additionally, at low energies $M$-theory is expected to be approximated by 11-dimensional supergravity in the same way that the low energy limit of type IIA/IIB string theory is type IIA/IIB supergravity. 
Combining these two pictures, various checks have been made that the dimensional reduction of 11-dimensional supergravity along the $M$-theory circle is type IIA supergravity. 

Motivated by the topological string, Costello and Li have laid out a series of conjectures for twists of type IIA/IIB supergravity \cite{CLsugra} and type I supergravity \cite{CLtypeI}. 
Their description was inspired by the model of the open and closed $B$-model topological string on a Calabi--Yau manifold. 
The open sector is holomorphic Chern--Simons theory \cite{WittenOpen} and the closed sector is called Kodaira--Spencer theory \cite{BCOV}. 
There are a few different versions of Kodaira--Spencer theory, but the shared characteristic is that they are all `gravitational' in nature; they describe fluctuations of the Calabi--Yau structure. 
From this point of view, Kodaira--Spencer theory is at the heart of the formulation of the various flavors of twisted 10-dimensional supergravity.

We begin by introducing certain variants of Kodaira--Spencer theory which will feature in the descriptions of twists of type IIA and type I supergravity.

\subsection{Kodaira--Spencer theory}

Let $X$ be a Calabi--Yau manifold; for now it can be of arbitrary complex dimension. 
Define
\deq{
  \PV^{i,j}(X) = \Omega^{0,j}(X, \wedge^i \T_X).
}
We will consider the graded space $\PV^{\bu,\bu}(X) = \oplus_{i,j} \PV^{i,j}(X)[-i-j]$ where the piece of type $(i,j)$ sits in degree $i+j$. 

For each fixed $i$, while we let $j$ vary, the $\dbar$ operator defines a cochain complex $\PV^{i,\bu}(X) = (\oplus_j\PV^{i,j}(X) [-j], \dbar)$ which provides a resolution for the sheaf of holomorphic polyvector fields of type $i$. 
The divergence operator extends to an operator of the form
\[
\div \colon \PV^{i,\bu}(X) \to \PV^{i-1,\bu}(X) .
\]

Motivated by the states of the topological $B$-model, one defines the fields of Kodaira--Spencer gravity on $X$ to be the cochain complex
\beqn\label{eqn:ks1}
\left(\PV^{\bu,\bu} (X)[[u]] [2] \, , \, \dbar + u \div\right) .
\eeqn 
Here, $u$ is a parameter of cohomological degree $+2$, which turns $\delta_{KS}^{(1)} = \dbar + u \div$ into an operator of homogenous degree $+1$. 
We also have performed an overall cohomological shift by $2$ so that $u^k \PV^{i,j}$ sits in degree $i+j+2k-2$. 
More precisely, this is a model for the $S^1$-equivariant cohomology of the states of the $B$-model on a closed disk. 
We refer to \cite{CLtypeI, CLsugra} for detailed justification for this ansatz. 

\brian{surya, could fill these parsecs in?}

\parsec[s:poisson]

%describe odd poisson structure

\parsec[s:ksaction] 

%recall genus zero BCOV action. 

\parsec[s:minimalks]

%define minimal ks

\subsection{The $SU(4)$ twist of type IIA supergravity}

\parsec[sec:SU(4)twist]

Let $X$ be a Calabi-Yau manifold of complex dimension four. 
The complex of fields of minimal Kodaira--Spencer theory on $X$ takes the form
\beqn
\begin{tikzcd}
                        &                          &                                     & {\PV^{0,\bu}}    \\
                        &                          & {\PV^{1,\bu}} \arrow[r, "u\div"]    & {u\PV^{0,\bu}}   \\
                        & {\PV^{2,\bu}} \arrow[r, "u\div"]  & {u\PV^{1,\bu}} \arrow[r, "u\div"]   & {u^2\PV^{0,\bu}} \\
{\PV^{3,\bu}} \arrow[r, "u\div"] & {u\PV^{2,\bu}} \arrow[r, "u\div"] & {u^2\PV^{1,\bu}} \arrow[r, "u\div"] & {u^3\PV^{0,\bu}}
\end{tikzcd}.
\eeqn
Denote this complex by $\cE_{mKS}(X)$. 
The classical BCOV action $I_{BCOV}$ follows from the general formula we gave above. 

\brian{I've commented this out below, since we should just introduce minimal BCOV in general up above}

%Clearly, the inclusion is a Poisson map, so restricting $I_{BCOV}$ to this subcomplex defines another $\Z/2$ graded Poisson BV theory referred to as minimal Kodaira-Spencer theory. Under the identification of Kodaira-Spencer theory as the closed string sector of the B-model, minimal Kodaira-Spencer theory should be thought of as consisting of those closed string fields present in the supergravity approximation.

With this in hand the conjecture of \cite{CLSugra} takes the following form.

\begin{conj}
The $SU(4)$-invariant twist of type IIA supergravity on $\RR^2\times \CC^4$ is the $\Z/2$-graded Poisson BV theory with fields \[
\alpha = \sum_n \alpha_n u^n \in \cE_{mKS}(\CC^4) \otimes \Omega^{\bu}(\RR^2).\] 
The classical interaction takes the form $I_{IIA} = \int_{\CC^4 \times \RR^2} \alpha_0^3 + {\rm higher\;order\;terms}.$
%The $L_{\infty}$ structure is the natural one on the tensor product of the cdga $\Omega^{\bu}(M)$ with the $L_{\infty}$-algebra $\cE_{mKS}$.
\end{conj}

We will need a more detailed description of the classical action. 
For the moment, let us introduce some notations for the fields of this IIA model, as always we leave the internal Dolbeault degree implicit:
\begin{multline}
\eta \in \PV^{0,\bu}(\CC^4) \otimes \Omega^\bu (\RR^2), \quad \mu + u \nu \in \PV^{1, \bu}(\CC^4) \otimes \Omega^\bu (\RR^2) \oplus u \PV^{0,\bu} (\CC^4) \otimes \Omega^\bu (\RR^2) \\
\Pi \in \PV^{3,\bu}(\CC^4) \otimes \Omega^\bu(\RR^2), \quad \sigma \in \PV^{3,\bu}(\CC^4) \otimes \Omega^\bu (\RR^2) .
\end{multline}
We will not need an explicit notation for the remaining descendant fields. 

With this notation in hand, we have the more precise form of the action appearing in the conjecture:
\beqn\label{eqn:IIAaction}
I_{IIA} = \frac12 {\rm Tr}_{\CC^4 \times \RR^2} \frac{1}{1-\nu} \mu^2 \wedge \Pi + {\rm Tr}_{\CC^4 \times \RR^2} \frac{1}{1-\nu} \eta \wedge \mu \wedge \sigma + \frac12 {\rm Tr}_{\CC^4 \times \RR^2} \frac{1}{1-\nu} \eta \wedge \Pi^2 + \cdots 
\eeqn
where the $\cdots$ denote terms involving higher order descendants. 

\parsec[sec:IIApot]

Our goal is to compare the dimensional reduction of our 11-dimensional theory on $\CC^5 \times \RR$
%where $X$ is a Calabi-Yau 4-fold 
with the $SU(4)$ invariant twist of type IIA on $\R^{2}\times \CC^4$. 
Doing so will require some additional modifications to the above conjectural description. 

Recall that in the physical theory, the components of the $C$-field in 11d that are not supported along the M-theory circle become the components of the Ramond--Ramond 2-form of type IIA. However, as noted in \cite{CLSugra} components of Ramond--Ramond fields do not appear as fields in Kodaira--Spencer theory; rather it is components of their field strengths that appear. Since components of the $C$-field become components of $\gamma$ in $\cE$ \surya{hopefully we can see this in the component fields section}, this suggests that we must modify our description of the twist of type IIA to include potentials for certain fields.

Explicitly, to define this theory we introduce a potential for both the $\Pi$ and $\sigma$ fields. 
First, we introduce a field $\gamma \in \Omega^{1,\bu}(\CC^4) \otimes \Omega^\bu(\RR^2)$ (not to be confused, yet, with the $\gamma$ field in our 11-dimensional theory) which satisfies $\Pi \vee \Omega = \del \gamma$ where $\Omega$ is the Calabi--Yau form on $\CC^4$. 
This condition does not uniquely fix $\gamma$. 
There is a new linear gauge symmetry determined by $\gamma \to \gamma + \div \beta$ where $\beta$ is a ghost that we must also introduce. 
Similarly, we introduce a field $\theta \in \Omega^{0,\bu}(\CC^4) \otimes \Omega^\bu(\RR^2)$ which satisfies $\sigma \vee \Omega = \del \theta$, there is no extra gauge symmetry present in this condition.\footnote{Using the Calabi--Yau form we have normalized the potential fields $\gamma, \beta,\theta$ to be written as differential forms instead of polyvector fields.}

In diagrammatic detail, the potential theory we are considering has underlying cochain complex of fields
\beqn\label{eqn:IIApot}
\begin{tikzcd}
- & + \\ \hline
& {\PV^{0,\bu} (\CC^4) \otimes \Omega^\bu (\RR^2) }_\eta  \\
{\PV^{1,\bu} (\CC^4) \otimes \Omega^\bu (\RR^2)}_\mu \arrow[r, "u\div"] & u{\PV^{0,\bu} (\CC^4) \otimes \Omega^\bu (\RR^2)}_\nu \\
u^{-1}{\Omega^{0,\bu} (\CC^4) \otimes \Omega^\bu (\RR^2)}_\beta \arrow[r, "u\div"] & {\Omega^{1,\bu} (\CC^4) \otimes \Omega^\bu (\RR^2)}_\gamma  \\
{\Omega^{0,\bu} (\CC^4) \otimes \Omega^\bu (\RR^2)}_\theta &
\end{tikzcd}
\eeqn.

Recall that the original fields of the IIA supergravity model on $\CC^4 \times \RR^2$ was equipped with an odd Poisson bivector which was degenerate.
In other words, it did not define a theory in the conventional BV formalism. 
One of the key features of this new complex of fields, after we have taken these potentials, is that it is equipped with an odd non-degenerate pairing thus equipping it with the structure of a theory in the conventional BV formalism. 

The pairing is $\Res_u \frac{\d u}{u} \int^\Omega_{\CC^4 \times \RR^2} \alpha \vee \alpha'$ where $\alpha, \alpha'$ are two general fields in this potential theory on $\CC^4 \times \RR^2$. 
Explicitly, in the description of the fields in \eqref{eqn:IIApot} the pairing is 
\[
\int^\Omega_{\CC^4 \times \RR^2} \eta \theta + \int^\Omega_{\CC^4 \times \RR^2} \mu \vee \gamma + \int^\Omega_{\CC^4 \times \RR^2} \nu \beta .
\]
This pairing is compatible with the odd Poisson bracket present in the original theory on $\CC^4 \times \RR^2$. \brian{finish}

The type IIA action completely determines the action of this theory with potentials. 
One simply takes the \eqref{eqn:IIAaction} and replaces all appearances of $\Pi$ with $\div \gamma$ and all appearances of $\sigma$ with $\div \theta$. 
This yields the interaction of the potential theory
\beqn\label{eqn:IIAactionpot}
\til I_{IIA} = \frac12 \int^\Omega_{\CC^4 \times \RR^2} \frac{1}{1-\nu} \mu^2 \vee \del \gamma + \int^\Omega_{\CC^4 \times \RR^2} \frac{1}{1-\nu} (\eta \wedge \mu) \vee \del \theta + \frac12 \int_{\CC^4 \times \RR^2} \frac{1}{1-\nu} \eta \wedge \del \gamma \wedge \del \gamma 
\eeqn
Notice that the terms involving higher descendants vanishes since these fields are set to zero in the potential theory.

%Note that there is a natural map of cochain complexes $\partial \cE_{pot}\to \cE_{mKS}$ given by the dotted arrows below:
%\beqn
%\begin{tikzcd}
%                                       & {\PV^{0,\bu}} \arrow[rrr, dotted, "\id"] &                          &                                     & {\PV^{0,\bu}}    \\
%{\PV^{1,\bu}} \arrow[r, "u\div"]       & {u\PV^{0,\bu}} \arrow[rr, dotted, "\id"] &                          & {\PV^{1,\bu}} \arrow[r, "u\div"]    & {u\PV^{0,\bu}}   \\
%{u^{-1}\PV^{4,\bu}} \arrow[r, "u\div"] & {\PV^{3,\bu}} \arrow[r, dotted, "\div"]  & {\PV^{2,\bu}} \arrow[r]  & {u\PV^{1,\bu}} \arrow[r, "u\div"]   & {u^2\PV^{0,\bu}} \\
%{\PV^{4,\bu}} \arrow[r, dotted, "\div"]        & {\PV^{3,\bu}} \arrow[r]          & {u\PV^{2,\bu}} \arrow[r] & {u^2\PV^{1,\bu}} \arrow[r, "u\div"] & {u^3\PV^{0,\bu}}
%\end{tikzcd}.
%\eeqn
%This is easily seen to be a Poisson map. We have that $I_{pot} = \partial^{*}I_{BCOV}$, so we see that $I_{pot}$ satisfies the classical master equation. Therefore, $I_{pot}$ determines an $L_{\infty}$ structure on $\cE_{pot}$, and hence one on $\Omega^{\bu}(\R^{2})\otimes \cE_{pot}$. In what follows, the $SU(4)$-invariant twist of IIA will refer to the BV theory $\Omega^{\bu}(\R^{2})\otimes \cE_{pot}$.

\parsec[sec:dimred]

We turn to the proof of the main result of this section that the dimensional reduction of our 11-dimensional theory agrees with the twist of IIA supergravity just introduced. 

We recall the notion of dimensional along a holomorphic direction following \cite{ESW}. 
Suppose that $V_\RR$ is a real vector space and denote by $V$ its complexification. 
We consider a field theory defined on $M \times V$, which is holomorphic along $V$ (in particular, this means that the theory is translation invariant along $V$).  
We consider the dimensional reduction along the projection 
\beqn\label{eqn:dimred}
M \times V \to M \times V_\RR
\eeqn
induced by ${\rm Re} \colon V \to V_\RR$.
Most relevant for us is the case when $V = \CC$ and $M$ is $\CC^4 \times \RR$, but the explicit form of the theory along $M$ is not important at the moment.

In fact, we might as well assume that $M$ is a point and that the space of fields is of the form $\Omega^{0,\bu}(V) \otimes W$ for $W$ some graded vector space. 
As properly formulated in \cite{ESW}, it is shown that the dimensional reduction along $V \to V_\RR$ is equivalent to the theory whose fields are $\Omega^\bu(V_\RR) \otimes W$. 
In other words, the dimensional reduction of the holomorphic theory on $V$ is a topological theory on $V_\RR$. 

If we put $M$ back in, the result is similar. 
Suppose the original theory is of the form $\cE(M) \otimes \Omega^{0,\bu}(V) \otimes W$.
Then, the dimensional reduction along \eqref{eqn:dimred} is the theory whose space of fields is $\cE(M) \otimes \Omega^\bu(V_\RR) \otimes W$.

An explicit model for this reduction can be described as follows. 
Suppose $V \cong \CC^n$ and place the theory on $(\CC^\times)^{\times n} \subset \CC^n$. 
The dimensional reduction along $\CC^n \to \RR^n$ agrees with the compactification of the theory along $S^1 \times \cdots \times S^1$ where one throws away all nonzero winding modes around each circle.

\begin{prop}\label{prop:dimred}
The $SU(4)$ invariant twist of type IIA on $\CC^4 \times \RR^2$ is equivalent to the dimensional reduction of the 11-dimensional along  
\[
\CC^4 \times \CC \times \RR_t \to \CC^4 \times \RR_x \times \RR_t \cong \CC^4 \times \RR^2 .
\]
\end{prop}
\begin{proof}
Let us denote the holomorphic coordinate we are reducing along by $z_5 = x + \im y$. 
We first read off the dimensional reduction of each component field of the 11-dimensional theory. 
Per the above discussion, this is obtained by taking all fields to be independent of $y$ and replacing $\d \zbar_5$ by $\d x$. 
To not confuse the notations of fields in 10 and 11 dimensions, we use the notation $\alpha_{11d}$ to denote an 11-dimensional field.

The reductions of the 11d fields $\nu_{11d}, \beta_{11d}$ are easy to describe. 
Recall that 
\[
\nu_{11d} \in \PV^{0,\bu}(\CC^5) \otimes \Omega^\bu(\RR) .
\]
The reduction of this field is a 10d $\nu$ field
\[
\nu (z_i,x,t) = \nu_{11d} (z_i, x, y=0, t) |_{\d \zbar_5 = \d x}  .
\]
Similarly, the reduction of $\beta_{11d}$ is a 10d $\beta$ field
\[
\beta (z_i,x,t) = \beta_{11d} (z_i, x, y=0, t) |_{\d \zbar_5 = \d x}  .
\]

The reduction of the 11d fields $\mu_{11d}$ and $\gamma_{11d}$ require a bit of massaging. 
We break the $SU(5)$ symmetry to $SU(4)$ to write
\[
\mu_{11d} = \mu^0_{11d} + \theta_{11d} \partial_{z_5} 
\]
where
\begin{align*}
\mu^0_{11d} & \in \PV^{1,\bu}(\CC^4) \otimes \Omega^{0,\bu}(\CC_{z_5}) \otimes \Omega^\bu(\RR_t) \\
\theta_{11d} & \in \Omega^{0,\bu}(\CC^4) \otimes \Omega^{0,\bu}(\CC_{z_5}) \otimes \Omega^\bu(\RR_t) .
\end{align*}
The dimensional reduction of $\mu^0_{11d}$ is a 10d $\mu$ field
\[
\mu(z_i,x,t) = \mu_{11d}^0 (z_i, x,y=0,t)|_{\d \zbar_5 = \d x} .
\]
The dimensional reduction of $\theta_{11d}$ is a $\theta$ field
\[
\theta(z_i,x,t) = \theta_{11d} (z_i, x,y=0,t)|_{\d \zbar_5 = \d x} .
\]

Finally, write the 11d field $\gamma_{11d}$ as
\[
\gamma_{11d} = \gamma_{11d}^0 + \eta_{11d} \d z_5
\]
where
\begin{align*}
\gamma^0_{11d} & \in \Omega^{1,\bu}(\CC^4) \otimes \Omega^{0,\bu}(\CC_{z_5}) \otimes \Omega^\bu(\RR_t) \\
\eta_{11d} & \in \PV^{0,\bu}(\CC^4) \otimes \Omega^{0,\bu}(\CC_{z_5}) \otimes \Omega^\bu(\RR_t) .
\end{align*}
The dimensional reduction of $\gamma^0_{11d}$ is a 10d $\gamma$ field
\[
\gamma(z_i,x,t) = \gamma_{11d}^0 (z_i, x,y=0,t)|_{\d \zbar_5 = \d x} .
\]
The dimensional reduction of $\eta_{11d}$ is an $\eta$ field
\[
\eta(z_i,x,t) = \eta_{11d} (z_i, x,y=0,t)|_{\d \zbar_5 = \d x} .
\]

Next, we read off the dimensional reduction of the 11d action. 
Let us first focus on the term present in BF theory which is
$\int^\Omega \frac{1}{1-\nu_{11d}} \mu_{11d}^2 \vee \del \gamma_{11d}$.
Upon reduction, this becomes 
\beqn\label{eqn:bfred}
\int^{\Omega_{\CC^4}}_{\CC^4 \times \RR^2} \frac{1}{1-\nu} \mu^2 \vee \del \gamma + \int^{\Omega_{\CC^4}}_{\CC^4 \times \RR^2} \frac{1}{1-\nu} (\theta \wedge \mu) \vee  \del \eta 
\eeqn

Next, consider the cubic term in the 11d action $J = \frac16 \int \gamma_{11d} \wedge \del \gamma_{11d} \wedge \del \gamma_{11d}$. 
Upon reduction, this becomes 
\beqn\label{eqn:jred}
\int_{\CC^4 \times \RR^2} \eta \wedge \del \gamma \wedge \del \gamma .
\eeqn

The sum of the action functionals \eqref{eqn:bfred} and \eqref{eqn:jred} does not precisely agree with the IIA action $\til I_{IIA}$. 
To relate the two actions we must make the following field redefinition:
\[
\til \theta = \frac{1}{1-\nu} \theta, \quad \til \eta = (1- \nu) \eta .
\]
Notice that this change of coordinates is compatible with the odd symplectic pairing on the fields. 
Under this field redefinition the total dimensionally reduced action can be written as
\begin{multline}
\int^{\Omega_{\CC^4}}_{\CC^4 \times \RR^2} \frac{1}{1-\nu} \mu^2 \vee \del \gamma + \int_{\CC^4 \times \RR^2} \frac{1}{1-\nu} \til\eta \wedge \del \gamma \wedge \del \gamma \\ + \int^{\Omega_{\CC^4}}_{\CC^4 \times \RR^2} (\til\theta \wedge \mu) \vee  \del \left(\frac{1}{1-\nu} \til\eta\right) .
\end{multline}
The first line agrees with the first and third terms in \eqref{eqn:IIAactionpot}. 

The second line is very close to the second term in \eqref{eqn:IIAactionpot}, but they are still not quite the same. 
They are, however, cohomologous. 
Integrating by parts and applying the BV relation, we can write the second line as
\[
\int^{\Omega_{\CC^4}}_{\CC^4 \times \RR^2} \left(\frac{1}{1-\nu} \til\eta\right)\wedge \div (\til\theta \wedge \mu)  = \int^{\Omega_{\CC^4}}_{\CC^4 \times \RR^2} \frac{1}{1-\nu} (\til\eta \wedge \mu) \vee \partial \til \theta + \int^{\Omega_{\CC^4}}_{\CC^4 \times \RR^2} \frac{1}{1-\nu} \til \eta \wedge \div \mu \wedge \til \theta .
\]
The first term in this equation agrees precisely with the second term in \eqref{eqn:IIAactionpot}. 
The final term is cohomological trivial via the odd Lagrangian $\int^\Omega \log(1-\nu) \wedge \til \eta \wedge \til \theta$.
%
%Let $\pi : \R\times \C^{\times}\times X \to \R^{2}\times X$ be the projection with fiber $S^{1}\subset \C^{\times}$. Note that there is an isomorphism \[\int: \pi^{*}(\Omega^{\bullet}(\R^{2})\otimes \cE_{pot})\to \cE\] given by \[(\eta, \mu, \nu, \beta,\gamma,\theta)\mapsto (\nu = \nu, \mu = \mu + \theta \vee \Omega_{X} \wedge\del_{z}, \beta = \beta, \gamma = \gamma\vee \Omega_{X} \eta dz ).\] It is clear that this isomorphism presrves the BV pairings; we need only check that $\int^{*} I$ is cohomologous to $I_{pot}$ in the deformation-obstruction complex of the free limit of $\Omega^{\bu}(\R^{2})\otimes \cE_{pot}$.
%
%We readily compute:
%\[\int^{*}I = \]
\end{proof}


\section{Twisted supergravity on AdS space}
\label{sec:ads}

So far, we have mostly given evidence for the eleven-dimensional theory as a twist of supergravity in a flat background. 
We now turn to twisted versions of AdS backgrounds of eleven-dimensional supergravity. 

In M-theory, AdS backgrounds arise from backreacting some number $N$ of branes. 
For M2 branes, the backreacted geometry is ${\rm AdS}_4 \times S^7$.
For the M5 branes, the backreacted geometry is ${\rm AdS}_7 \times S^4$. 

According to the AdS/CFT correspondence, supergravity on such backgrounds should be dual to the relevant worldvolume theory in the large-$N$ limit. 
In this section, we do not directly refer to the worldvolume theories on the holomorphic twists of the M2 and M5 branes.
Rather, we identify the fields sourcing the branes at the level of the twisted eleven-dimensional theory.
In turn, we give a proposal for the twisted AdS background. 
We will show that the twist of the superconformal algebra is a global symmetry of this twisted background. 

\subsection{Superconformal algebras}

The complex form of the algebra of isometries for supergravity in both the ${\rm AdS}_4$ and ${\rm AdS}_7$ backgrounds is $\lie{osp}(8|2)$ (though, their real forms differ). 
This agrees with the complex form of the 6d $\cN=(2,0)$ superconformal algebra and the 3d $\cN=8$ superconformal algebra. 
The bosonic part of this algebra is isomorphic to $\lie{so}(8) \oplus \lie{sp}(2) \cong \lie{so}(8) \oplus \lie{so}(5)$. 

The minimal supercharge $Q$ acting on eleven-dimensional supersymmetry algebra is an element of this superconformal algebra. 
Its $Q$-cohomology is isomorphic to $\lie{osp}(6|1)$. (Twisted superconformal symmetry in six dimensions is studied in detail by the second  two authors  in~\cite{SWsuco2}.)
This super Lie algebra will play the role of the isometries in the twisted AdS background. 

\subsection{The ${\rm AdS}_4 \times S^7$ background}

In this section we introduce the analog of the ${\rm AdS}_4 \times S^7$ background in our conjectural description of the minimal twist of eleven-dimensional supergravity. 

\parsec[]

Decompose the eleven-dimensional manifold $\CC^5 \times \RR$ as
\[
 \CC^4_w\times \CC_z \times \RR .
\]

Analogous to before, the ${\rm AdS}_4 \times S^7$ background arises from backreacting M2 branes. Consider a stack of $N$ M2 branes wrapping $\R\times \C_z$. A natural interaction to consider is 
\[
I_{M2}(\gamma) = N\int_{\C_z} \gamma + \cdots
\] 
which is nonzero only on the component of $\gamma$ in $\Omega^1(\R)\otimes \Omega^{1,1}(\C^5)$. Unlike the case of M5 branes, the coupling does not involve choosing a primitive for a field strength---it is an electric coupling.
We have only indicated the lowest order coupling, the $\cdots$ indicate higher-order couplings which will be higher order in the fields of the eleven-dimensional theory and explicitly involve the fields in the worldvolume theory. 

This coupling is justified by comparison with the physical theory and by dimensional reduction. 
Indeed, as discussed in~\S\ref{s:components}, the component of $\gamma$ which participates in the above coupling is a component of the $C$-field of eleven dimensional supergravity. Thus, the proposal mirrors electric couplings of M2 branes in the physical theory, which simply involves integrating the $C$-field over the worldvolume of the brane. 

Moreover, reducing on a circle transverse to the M2 brane yields the $SU(4)$ twist of type IIA supergravity on $\R^2\times \C_z\times \C^3$ with $N$ $D2$ branes wrapping $\R\times \C_z$. As is shown in \cite{CLsugra}, an electric coupling of D2 branes to the $SU(4)$ twist of type IIA supergravity is given by 
\[
I_{D2}(\gamma) = N \int_{\R\times\C_z} \gamma + \cdots
\] 
where $\gamma$ now denotes the 1-form field of the $SU(4)$ twist of type IIA supergravity. It is immediate that the pullback of $I_{M2}$ along the map in the proof of proposition \ref{prop:dimred} recovers $I_{D2}$. 

\parsec[sec:m2backreact]

The backreacted geometry will be given by a solution to the equations of motion upon deforming the eleven-dimensional action by the interaction $I_{M2}(\gamma)$. 
Varying the deformed action with respect to $\gamma$,
we obtain the equation of motion
\beqn\label{eqn:ads4eom1}
\dbar \mu + \frac12 [\mu, \mu] + \partial\gamma\partial\gamma = N \Omega^{-1} \delta_{w=0} .
\eeqn
Here $[-,-]$ is the Schouten bracket. 
Varying $\beta$, we obtain the equation of motion
\beqn\label{eqn:adseom2}
\div \mu = 0 .
\eeqn

\begin{lem}
Let
\[
 F_{M2} = \frac{6}{(2\pi i)^4} \frac{\sum_{a=1}^4 \wbar_a \d \wbar_1 \cdots \Hat{\d \wbar_a} \cdots \d \wbar_4}{\|w\|^{8}} \partial_z .
\]
Then the background where $\mu = N F_{M2}$ and $\gamma = 0$
satisfies the above equations of motion in the presence of a stack of $N$ M2 branes:
\begin{align*}
\dbar (N F_{M2}) + \frac12 [N F_{M2}, N F_{M2}] & = N \Omega^{-1} \delta_{w=0} \\
\div (N F_{M2}) & = 0  .
\end{align*}
Here we set all components of the field $\gamma$ equal to zero (as well as the fields $\nu,\beta$). 
\end{lem}

\begin{proof}
Upon specializing $\gamma = 0$, the last term in the first equation above vanishes. The equation $\dbar F_{M2} = \Omega^{-1} \delta_{w=0}$ characterizes the Bochner--Martinelli kernel representing the residue class on $\CC^4 \, \setminus \, 0$. 
It is clear that $\div F_{M2} = 0$ and 
\[
[F_{M2}, F_{M2}] = 0
\] 
by simple type reasons. 
\end{proof}

\parsec[]

To provide evidence for the claim that this is the twisted analog of the AdS geometry, we will show that the twist of the symmetries present in the physical theory are witnessed in the twisted theory in this background. 

We have recalled that the $Q$-cohomology of $\lie{osp}(8|2)$ is isomorphic to the super Lie algebra $\lie{osp}(6|1)$. 
We will define an embedding of $\lie{osp}(6|1)$ into the eleven-dimensional theory on $\CC^5 \times \RR \setminus \{w=0\}$ which corresponds to the twist of the 3d superconformal algebra.
We first focus on the case where the flux $N=0$, for which the embedding can be extended to all of $\CC^5 \times \RR$. 

\parsec[] 

The bosonic part of $\lie{osp}(6|1)$ is the direct sum Lie algebra $\lie{sl}(4) \oplus \lie{sl}(2)$. 
The Lie algebra $\lie{sl}(2)$ represents (holomorphic) conformal transformations in $\CC_z$, which are inherited from the natural M\"obius group action on~$P^1(\C)$; the vector fields representing these transformations are not all divergence-free, and as such must be slightly adjusted. 
The Lie algebra $\lie{sl}(4)$ represents rotations along the plane $\CC^4_w$.   

\begin{itemize}[leftmargin=\parindent]
\item The bosonic summand $\lie{sl}(2)$ is mapped to the vector fields:
\[
\frac{\del}{\del z} ,\quad z \frac{\del}{\del z} - \frac14 \sum_{a=1}^4 w_a \frac{\del}{\del w_a} , \quad z \left(z \frac{\del}{\del z} - \frac12 \sum_{a=1}^4 w_a \frac{\del}{\del w_a} \right) \in \PV^{1,0}(\CC^5) \otimes \Omega^0(\RR) .
\]
Notice that these vector fields are divergence-free and reduce to the usual holomorphic conformal transformations along $w=0$.
\item The bosonic summand $\lie{sl}(4)$ is mapped to four-dimensional rotations: 
\[
\sum_{a,b=1}^4 B_{ab} w_a \frac{\del}{\del w_b} \in \PV^{1,0}(\CC^5) \otimes \Omega^0(\RR) , \quad (B_{ab}) \in \lie{sl}(4) .
\]
\end{itemize}

The odd part of the algebra $\lie{osp}(6|1)$ is $\wedge^4 W \otimes R$ where $W$ is the fundamental $\lie{sl}(4)$ representation and $R$ is the fundamental $\lie{sl}(2)$ representation. 
It is natural to split $R = \CC_{+1} \oplus \CC_{-1}$, so that the odd part decomposes as
\[
(\wedge^2 \CC^4)_{+1} \oplus (\wedge^2 \CC^4)_{-1} .
\]

\begin{itemize}[leftmargin=\parindent]
\item 
The fermionic summand $(\wedge^2 \CC^4)_{+1}$ consists of the supertranslations. 
It is mapped to the fields: 
\[
\frac{1}{2} (w_a \d w_b - w_b \d w_a) \in \Omega^{1,0}(\CC^5) \otimes \Omega^0(\RR) , \quad a,b=1,2,3,4 .
\] 
\item The fermionic summand $(\wedge^2 \CC^4)_{-1}$ consists of the remaining superconformal transformations. 
It is mapped to the fields: 
\[
\frac{1}{2} z (w_a \d w_b - w_b \d w_a) \in \Omega^{1,0}(\CC^5) \otimes \Omega^0(\RR) , \quad a,b=1,2,3,4. 
\] 
\end{itemize}


\begin{lem}\label{lem:m2emb}
These assignments define an embedding of $\lie{osp}(6|1)$ into the linearized BRST cohomology of the fields of the eleven-dimensional theory on $\CC^5 \times \RR$. 
Equivalently, it defines an embedding
\[
i_{M2} \colon \lie{osp}(6|1) \hookrightarrow E(5,10) .
\]
\end{lem} 
\begin{proof}
The second assertion follows from Theorem \ref{thm:global}, which shows that, as a super Lie algebra, the linearized BRST cohomology of the global symmetry algebra of the eleven-dimensional theory on $\CC^5 \times \RR$ is the trivial central extension of $E(5,10)$. 
Recall that the odd part of $E(5,10)$ is precisely the module of closed two-forms on $\CC^5$. 
To explicitly describe the embedding into $E(5,10)$ we simply apply the de Rham differential to the last two formulas above.
Recall, we are using the holomorphic coordinates $(z,w_1,\ldots,w_4)$ on $\CC^5$ where $z$ is the holomorphic coordinate along the M2 brane. 
\begin{itemize}[leftmargin=\parindent]
\item 
The fermionic summand $(\wedge^2 \CC^4)_{+1}$ embeds into closed two-forms as
\[
\d w_a \wedge \d w_b, \quad a,b=1,2,3,4. 
\] 
\item The fermionic summand $(\wedge^2 \CC^4)_{-1}$ embeds into closed two-forms as
\[
z \d w_a  \wedge \d w_b + \frac12 \d z \wedge (w_a \d w_b - w_b \d w_a) , \quad a,b=1,2,3,4. 
\] 
\end{itemize}
\end{proof}
\parsec[]

Next, we turn on $N \ne 0$ units of nontrivial flux. 
Since not all of the fields we wrote down above commute with the flux $N F_{M2}$, they are not compatible with the total differential $\delta^{(1)} + [N F_{M2}, -]$ acting on the fields supported on $\CC^5 \times \RR \setminus \{w=0\}$. 
Nevertheless, we have the following. 

\begin{prop}
\label{prop:brads4}
There exist $N$-dependent corrections to the fields defining the embedding of $\lie{osp}(6|1)$ summarized above which are closed for the modified BRST differential $\delta^{(1)} + [N F_{M2},-]$. 
Furthermore, these order $N$ corrections define an embedding of $\lie{osp}(6|1)$ inside the cohomology of the fields of eleven-dimensional theory on $\CC^5 \times \RR \setminus \CC \times \RR$ with respect to the differential $\delta^{(1)} + [N F_{M2},-]$.
\end{prop}

\begin{proof}
Let $\cL(\CC^5 \times \RR \setminus \{w=0\})$ denote the super $L_\infty$ algebra obtained by parity shifting the fields of the eleven-dimensional theory. 
We make the identification 
\[
(\CC^5 \times \RR) \setminus \{w=0\} \cong (\CC_w^4 \setminus 0) \times \CC_z \times \RR .
\]

Set $F = F_{M2}$ for notational convenience. Recall that we are viewing $F$ as an element of $\PV^{1,3}(\CC_w^4 \setminus 0) \otimes \Omega^{0,0}(\CC_z) \otimes \Omega^0(\RR)$. 
The operator $[F,-]$ acts on the fields according to two types of maps:
\begin{align*}
[F ,-] & \colon \PV^{i,\bu}(\CC^4_w \setminus 0) \otimes \PV^{j,\bu} (\CC_z) \otimes \Omega^\bu (\RR) \to \PV^{i,\bu+3}(\CC^4_w \setminus 0) \otimes \PV^{j,\bu} (\CC_z) \otimes \Omega^\bu (\RR) \\
[F,-] & \colon \Omega^{i,\bu}(\CC^4_w \setminus 0) \otimes \Omega^{j,\bu} (\CC_z) \otimes \Omega^\bu (\RR) \to \Omega^{i,\bu+3}(\CC^4_w \setminus 0) \otimes \Omega^{j,\bu} (\CC_z) \otimes \Omega^\bu (\RR).
\end{align*}

%\brian{get the filtration straight}


The first page of the spectral sequence is the cohomology with respect to the original linearized BRST differential $\delta^{(1)}$. 
Recall that the linearized BRST differential decomposes as
\[
\delta^{(1)} = \dbar + \d_{\RR} + \div |_{\mu \to \nu} + \del |_{\beta \to \gamma}  .
\]
To compute this page, we use an auxiliary spectral sequence which simply filters by the holomorphic form and polyvector field type. 
This first page of this auxiliary spectral sequence is simply given by the cohomology with respect to $\dbar + \d_{\RR}$. 
This cohomology is given by
\begin{equation}
  \label{eqn:ads4ss} 
  \begin{tikzcd}[row sep = 1 ex]
    + & - \\ \hline
H^\bu(\CC^4\setminus 0, \T) \otimes H^\bu(\CC, \cO) & H^\bu(\CC^4 \setminus 0, \cO) \otimes H^\bu(\CC, \cO) \\
H^\bu(\CC^4\setminus 0, \cO) \otimes H^\bu(\CC, \T) \\
H^\bu(\CC^4\setminus 0, \cO) \otimes H^\bu(\CC, \cO) & H^\bu(\CC^4\setminus 0, \cO) \otimes H^\bu(\CC, \Omega^1) \\ & H^\bu(\CC^4\setminus 0, \Omega^1) \otimes H^\bu(\CC, \cO)  
\end{tikzcd}
\end{equation}
where $\T$ denotes the holomorphic tangent sheaf, $\Omega^1$ denotes the sheaf of holomorphic one-forms, and $\cO$ is the sheaf of holomorphic functions.

The cohomology of $\CC$ is concentrated in degree zero and there is a dense embedding
\[
\CC[z] \hookrightarrow H^\bu(\CC, \cF) 
\]
for $\cF = \cO, \T$, or $\Omega^1$. 

For $\cF = \cO, \T$, or $\Omega^1$, the cohomology $H^\bu(\CC^4 \setminus 0, \cF)$ is concentrated in degrees $0$ and $3$. 
There are the following dense embeddings 
\begin{align*}
\CC[w_1,\ldots, w_4] & \hookrightarrow H^0(\CC^4 \setminus 0, \cO) \\ 
\CC[w_1,\ldots, w_4] \{\partial_{w_i}\} & \hookrightarrow H^0(\CC^4 \setminus 0, \T) \\
\CC[w_1,\ldots, w_4] \{\d w_i\} & \hookrightarrow H^0(\CC^4 \setminus 0, \Omega^1) 
\end{align*}
and
\begin{align*}
(w_1\cdots w_4)^{-1} \CC[w_1^{-1},\ldots, w_4^{-1}] & \hookrightarrow H^3(\CC^4 \setminus 0, \cO) \\ 
(w_1\cdots w_4)^{-1} \CC[w_1^{-1},\ldots, w_4^{-1}] \{\partial_{w_i}\} & \hookrightarrow H^3(\CC^4 \setminus 0, \T) \\
(w_1\cdots w_4)^{-1} \CC[w_1^{-1},\ldots, w_4^{-1}] \{\d w_i\} & \hookrightarrow H^3(\CC^4 \setminus 0, \Omega^1) .
\end{align*}

It follows that (up to completion) the cohomology 
\[
H^\bu(\cL(\CC^5 \times \RR \setminus \{w=0\}) , \dbar)
\]
is the direct sum of $H^\bu(\cL(\CC^5 \times \RR), \dbar)$ with 
\begin{equation}
  \label{eqn:ads4ss2} 
  \begin{tikzcd}[row sep = 1 ex]
    - & + \\ \hline
H^3(\CC^4\setminus 0, \cO)[z] \{\partial_{w_i}\}  \ar[r, dotted, "\div"] & H^3(\CC^4 \setminus 0, \cO) [z] \\
H^3(\CC^4\setminus 0, \cO) [z] \partial_z \ar[ur, dotted, "\div"'] \\
H^3(\CC^4\setminus 0, \cO) [z] \ar[r, dotted, "\del"] \ar[dr, dotted, "\del"'] & H^3(\CC^4\setminus 0, \cO)[z] \d z \\ & H^3(\CC^4\setminus 0, \Omega^1)[z] \{\d w_i\} .
\end{tikzcd}
\end{equation}
The remaining piece of the original BRST operator is drawn in dotted lines. 
The first page of the spectral sequence converging to the cohomology with respect to $\delta^{(1)} + [N F, -]$ is given by the cohomology of the global symmetry algebra on $\CC^5 \times \RR$, which we computed in \S \ref{sec:global}, plus the cohomology of the above complex with respect to the dotted-line operators. 
In this description, the image of the flux $F$ at this page in the spectral sequence corresponds to the class 
\[
[F] = (w_1 \cdots w_4)^{-1} \partial_z \in H^3(\CC^4\setminus 0, \cO) [z] \partial_z .
\]

The next page of the spectral sequence is given by computing the cohomology with respect to the operator $[N F,-]$. 
As observed above, this operator maps Dolbeault degree zero elements to Dolbeault degree three elements. 
For degree reasons, there are no further differentials and the spectral sequence collapses after the second page. 

The embedding of $\lie{osp}(6|1)$ we wrote down in lemma \ref{lem:m2emb} lands in the kernel of the original BRST operator $\delta^{(1)}$. 
To see that it this embedding can be lifted to the full cohomology we need to check that any element in the image of the original embedding is annihilated by $\big[ N [F] , - \big]$. 
This is a direct calculation. 
For instance, recall that an element in the image of the odd summand $(\wedge^2 \CC^2)_{-1}$ (which corresponds to a superconformal transformation) is of the form $z w_a \wedge \d w_b = z(w_a \d w_b - w_b \d w_a)$. 
We have
\[
\big[[F] , z(w_a \d w_b - w_b \d w_a) \big] = (w_1\cdots w_4)^{-1} (w_a \d w_b - w_b \d w_a) = 0
\]
since the class $(w_1\cdots w_4)^{-1}$ is in the kernel of the operator given by multiplication by $w_a$ for any $a = 1,\ldots 4$. 
\end{proof}

\subsection{The ${\rm AdS}_7 \times S^4$ background}

In this section we introduce the analog of the ${\rm AdS}_7 \times S^4$ background in our description of the minimal twist of eleven-dimensional supergravity. Decompose the eleven dimensional spacetime as $\C^3_z\times \C^2_w\times \R$.

\parsec[sec:m5coupling]

Analogous to the physical theory, the ${\rm AdS}_7 \times S^4$ background in the holomorphic twist will arise by backreacting M5 branes. To this effect, we begin by discussing how the eleven-dimensional theory couples to M5 branes. 
Consider a stack of $N$ M5 branes wrapping 
\[
\{w_1=w_2=t=0\} \subset \C^3_z\times \C^2_w\times \R.
\] 
It is natural to consider the nonlocal interaction 
\[
I_{M5} = N\int_{\C^3_z} \div^{-1}\mu \vee \Omega +\cdots 
\]
Note that this expression is only nonzero on the component of $\mu$ in $\PV^{1,3}$. 
We argue that this coupling is consistent with expectations from the physical theory and from dimensional reduction. 

The twisted field $\mu^{1,3}$ is a component of the Hodge star of the $G$-flux in the physical theory (\S\ref{s:components}). 
In the physical theory, M5 branes magnetically couple to the $C$-field; the coupling involves choosing a primitive for the Hodge star of the $G$-flux and integrating it over the M5 worldvolume. Our twist contains no fields corresponding to components of such a primitive; hence such a magnetic coupling is reflected in the appearance of $\div^{-1}$. 

\parsec[]

We obtain a deeper justification for this coupling through dimensional reduction to type IIA supergravity. 
Reducing on the circle along the directions the M5 branes wrap yields the $SU(4)$ invariant twist of type IIA supergravity on $\CC^4 \times \RR^2$ with $N$ $D4$ branes wrapping $\CC^2 \times \RR$. 

In \cite{CLsugra}, it is shown that the magnetic coupling of $D4$ branes to the $SU(4)$ twist of IIA is of the form
\[
N \int _{\C^2 \times \RR} \div^{-1} \mu \vee \Omega_{\C^4} + \cdots .
\]
Again, we have only explicitly indicated the first-order piece of the coupling. 

\parsec[s:m5backreact]

The backreacted geometry will be given by a solution to the equations of motion upon deforming the eleven-dimensional action by the interaction $I_{M5}(\mu)$. 

Varying the potential $\div^{-1} \mu$, we obtain the following equation of motion involving the field $\gamma$:
\beqn\label{eqn:m5eom1}
\dbar \del \gamma + \div \left(\frac{1}{1-\nu} \mu\right) \wedge \del \gamma = N \delta_{w_1=w_2=t=0} .
\eeqn
Notice that there is an extra derivative compared to the equation of motion arising from varying the field $\mu$. 
This equation only depends on $\gamma$ through its field strength $\del \gamma$. 

Varying $\gamma$ we obtain the equation of motion 
\beqn\label{eqn:m5eom2}
(\dbar + \d_\RR) \mu + \del \gamma \del \gamma = 0 .
\eeqn 
Again, this only depends on $\gamma$ through its field strength $\del \gamma$.


\begin{lem}
\label{lem:ads7flux}
Let
\[
F_{M5} = \frac{1}{(2\pi i)^3} \frac{\wbar_1 \d \wbar_2 \wedge \d t - \wbar_2 \d \wbar_1 \wedge \d t + t \d \wbar_1 \wedge \d \wbar_2}{(\|w\|^2 + t^2)^{5/2}} \wedge \d w_1 \wedge \d w_2
\]
Then, $\del\gamma = N F_{M5}$, $\mu = 0$, and $\nu = 0$ satisfies the equations of motion in the presence of a stack of $N$ M5 branes sourced by the term $N \delta_{w_1=w_2=t=0}$:
\begin{align*}
\dbar (NF_{M5}) + \d_{\RR} (NF_{M5}) & = N \delta_{w_1=w_2=t=0}  \\ 
(NF_{M5}) \wedge (NF_{M5}) & = 0 .
\end{align*}
Here, we set all components of the field $\mu$ equal to zero (as well as the fields $\nu,\beta$). 
\end{lem}

\begin{proof}
The first equation,
\[
\dbar F + \d_{\RR} F = N \delta_{w_1=w_2=t=0},
\]
characterizes the kernel representing $N$ times the residue class for a four-sphere in 
\[
(\CC^2 \times \RR) \setminus 0 \simeq S^4 \times \RR .
\] 
That is
\[
\oint_{S^4} N F = N 
\]
for any four-sphere centered at $0 \in \CC^2 \times \RR$.

The second equation $F \wedge F = 0$ follows by simple type reasons. 
\end{proof}

\parsec[s:m5embedding]

To provide evidence for the claim that this is the twisted analog of the AdS geometry, we will again show that the twist of the symmetries present in the physical theory on ${\rm AdS}_7 \times S^4$ appear in the twisted theory on this background. 

We have recalled that the $Q$-cohomology of $\lie{osp}(8|2)$ is isomorphic to the super Lie algebra $\lie{osp}(6|1)$. 
We will define an embedding of $\lie{osp}(6|1)$ into the eleven-dimensional theory on $\CC^5 \times \RR \setminus \{w_1=w_2=t=0\}$ which corresponds to the twist of the 6d superconformal algebra.

We first focus on the case where the flux $N=0$.
In this case the embedding can be extended to all of $\CC^5 \times \RR$. 



Recall that we have chosen coordinates of the form
\[
\CC^5 \times \RR = \CC_z^3 \times \CC_w^2 \times \RR_t
\]
with $z_i, i=1,2,3$ and $w_a, a=1,2$.
The stack of M5 branes wrap $w_1=w_2=t=0$. 

The embedding of the bosonic piece of $\lie{osp}(6|1)$ can be described as follows. Recall that the bosonic part of $\lie{osp}(6|1)$ is the direct sum Lie algebra
\[
\lie{sl}(4) \oplus \lie{sl}(2) .
\]
which we write as $\lie{sl}(W) \oplus \lie{sl}(R)$ with $W,R$ complex four, two dimensional complex vector spaces. The roles of the $\lie{sl}(4)$ and $\lie{sl}(2)$ summands are interchanged compared to the case of the M2 brane. 
The Lie algebra $\lie{sl}(4)$ represents (holomorphic) conformal transformations along the plane $\CC^3_z$, again coming from the natural action on $P^3(\C)$.
Since not all such infinitesimal transformations are divergence-free, the precise vector fields must be adjusted.   
The Lie algebra $\lie{sl}(2)$ represents rotations in $\CC^2_w$; the vector fields representing these transformations are automatically divergence-free.
In more detail, the embedding of the bosonic piece can be given by the following explicit formulas. 

\begin{itemize}[leftmargin=\parindent]
\item
The bosonic abelian subalgebra $\CC^3 \subset \lie{sl}(4)$ of translations is mapped to the obvious vector fields 
\[
\frac{\del}{\del z_i} \in \PV^{1,0}(\CC^5) \otimes \Omega^0(\RR) , \quad i=1,2,3.
\]

\item
The bosonic subalgebra $\lie{sl}(3) \subset \lie{sl}(4)$ is mapped to the 
rotations
\[
A_{ij} z_i \frac{\del}{\del z_j} \in \PV^{1,0}(\CC^5)\otimes \Omega^0(\RR) , \quad (A_{ij}) \in \lie{sl}(3) .
\]

\item
The bosonic subalgebra $\CC \subset \lie{sl}(4)$ corresponding to rescaling $\C^3$ is mapped to the element
\[
\sum_{i=1}^3 z_i \frac{\del}{\del z_i} - \frac32 \sum_{a=1}^2 w_a \frac{\del}{\del w_a} \in \PV^{1,0}(\CC^5) \otimes \Omega^0(\RR)  .
\] 
Notice that this vector field are divergence-free and restricts to the ordinary dilation (Euler vector field) along $w=0$. 
\item 
The bosonic subalgebra of $\lie{sl}(4)$ describing special conformal transformations on $\CC^3$ is mapped to the elements 
\[
z_j \left(\sum_{i=1}^3 z_i \frac{\del}{\del z_i} - 2 \sum_{a=1}^2 w_a \frac{\del}{\del w_a} \right) \in \PV^{1,0}(\CC^5) \otimes \Omega^0(\RR) .
\] 
Notice that these vector fields are divergence-free and restrict to the ordinary special conformal transformations along $w=0$. 
\item 
The bosonic summand $\lie{sl}(2)$ ($R$-symmetry) is mapped to the triple
\[
w_1 \frac{\del}{\del w_2}, w_2 \frac{\del}{\del w_1}, \frac12 \left(w_1 \frac{\del}{\del w_1} - w_2 \frac{\del}{\del w_2}\right) \in \PV^{1,0}(\CC^5) \otimes \Omega^0(\RR) .
\]
\end{itemize}

The odd part of the algebra $\lie{osp}(6|1)$ is $\wedge^4 W \otimes R$ where $W$ is the fundamental $\lie{sl}(4)$ representation and $R$ is the fundamental $\lie{sl}(2)$ representation. 
It is natural to split $W = L \oplus \CC$ with $L = \CC^3$ the fundamental $\lie{sl}(3) \subset \lie{sl}(4)$ representation. 
Then the odd part decomposes as
\[
L \otimes R \oplus \wedge^2 L \otimes R \cong \CC^3 \otimes \CC^2 \oplus \wedge^2 \CC^3 \otimes \CC .
\]

\begin{itemize} 
\item The summand $L \otimes R$ consists of the remaining 6d supertranslations. 
It is mapped to the fields 
\[
z_i \d w_a \in \Omega^{1,0}(\CC^5) \otimes \Omega^0(\RR) ,\quad a=1,2, \quad i =1,2,3.
\] 
\item The summand $\wedge^2 L \otimes R$ consists of the remaining 6d superconformal transformations. 
It is mapped to the fields
\[
\frac12 w_a (z_i \d z_j - z_j \d z_i) \in \Omega^{1,0}(\CC^5)\otimes \Omega^0(\RR) , \quad a = 1,2, \quad k = 1,2,3. 
\]
\end{itemize}

\begin{lem}
These assignments define an embedding of $\lie{osp}(6|1)$ into the linearized BRST cohomology of the fields of the eleven-dimensional theory on $\CC^5 \times \RR$. 
Equivalently, it defines an embedding
\[
i_{M5} \colon \lie{osp}(6|1) \hookrightarrow E(5,10) .
\]
\end{lem} 

\begin{proof}
To explicitly describe the embedding into $E(5,10)$ we simply apply the de Rham differential to the last two formulas above.
Recall, we are using the holomorphic coordinates $(z_1,z_2,z_3, w_1,w_2)$ on $\CC^5$ where $z_i$ are the holomorphic coordinates along the M5 brane. 
\begin{itemize}
\item 
The fermionic summand $L \otimes R$ embeds into closed two-forms as
\[
\d z_i \wedge \d w_a, \quad i=1,2,3, \quad a=1,2. 
\] 
\item 
The fermionic summand $\wedge^2 L \otimes R$ embeds into closed two-forms as
\[
w_a \d z_i  \wedge \d z_j + \frac12 \d w_a \wedge (z_i \d z_j - z_j \d z_i) , \quad i,j=1,2,3, \quad a=1,2. 
\] 
\end{itemize}
\end{proof}
\parsec[]

Next, we turn on $N \ne 0$  units of nontrivial flux. 
Since not all of the fields we wrote down above commute with the flux $N F$, they are not compatible with the total differential $\delta^{(1)} + [N F, -]$ acting on the fields supported on $\CC^5 \times \RR \setminus \{w_1=w_2=t=0\}$. 
Nevertheless, we have the following.

\begin{prop}
\label{prop:brads7}
There exist $N$-dependent corrections to the embedding $i_{M5}$ which are compatible with the modified BRST differential $\delta^{(1)} + [N F_{M5},-]$. 
Furthermore, these $N$-dependent corrections define an embedding of $\lie{osp}(6|1)$ inside the cohomology of the fields of eleven-dimensional theory on $\CC^5 \times \RR \setminus \CC \times \RR$ with respect to the differential $\delta^{(1)} + [N  F_{M5},-]$.
\end{prop}

\parsec[s:thfcohomology]

The proof of the above proposition follows from another indirect cohomological argument. 
Before getting to the proof, we introduce the relevant cohomology. 

The eleven-dimensional theory is built from fields which live in the tensor product of complexes 
\[
\Omega^{0,\bu}(\CC^5) \otimes \Omega^\bu(\RR).
\]
More precisely, this is where the  fields $\beta$ and~$\nu$ live. 
The $\mu$ and~$\gamma$ fields live in versions of this complex where we take Dolbeault forms with coefficients in the holomorphic tangent and cotangent bundles, respectively. 

Another way to think about this complex is to first consider the full de Rham complex $\Omega^\bu(\CC^5 \times \RR)$, which includes both holomorphic and anti-holomorphic forms in the $\CC^5$ direction. 
The dg algebra of all de Rham forms on $\CC^5 \times \RR$ has an ideal generated by the holomorphic one forms $\{\d z_i\}_{i=1,\ldots,5}$.
There is an isomorphism of dg algebras
\[
\Omega^{0,\bu}(\CC^5) \otimes \Omega^\bu(\RR) \cong \Omega^\bu(\CC^5 \times \RR) \, / \, (\d z_1,\ldots, \d z_5) .
\]
The advantage of this presentation is that we can define a complex associated to more general manifolds that are not products of a complex manifold with a smooth manifold.\footnote{More generally, we are describing the cohomology of a manifold equipped with a topological holomorphic foliation.}

For the M5 brane, it was convenient to rename the holomorphic coordinates on $\CC^5$ to $z_1,z_2,z_3,w_1,w_2$. 
At the twisted level, the geometry arising from backreacting M5 branes is based on the manifold 
\[
\CC^5 \times \RR \setminus \CC^3 \cong \CC_z^3 \times (\CC^2_w \times \RR \setminus 0) .
\]
The $\beta$ and~$\nu$ fields of the eleven-dimensional theory on this submanifold of $\CC^5 \times \RR$ live in the complex 
\[
\Omega^\bu\bigg(\CC^5 \times \RR \setminus \CC^3\bigg) \, / \, (\d z_1,\d z_2,\d z_3, \d w_1, \d w_2)  .
\]
The $\mu$ and~$\gamma$ fields live in similar complexes, where we introduce a (trivial) vector bundle on $\CC^5 \times \RR \setminus \CC^3$. 

Since the $\CC^3$ wraps $w_1=w_2=t=0$ we can apply a version of the K\"unneth formula to identify this complex with 
\[
\Omega^{0,\bu}(\CC^3_z) \otimes \bigg( \Omega^\bu\left(\CC^2_w \times \RR \setminus 0 \right) \, / \, (\d w_1, \d w_2) \bigg).
\]

The cohomology of the Dolbeault complex of $\CC^3_z$ is easy to compute. 
The cohomology of the bit in parentheses is concentrated in degrees zero and two. 
In degree zero, there is a dense embedding
\[
\CC[w_1,w_2] \hookrightarrow H^0 \bigg( \Omega^\bu\left(\CC^2_w \times \RR \setminus 0 \right) \, / \, (\d w_1, \d w_2) \bigg)
\]
In degree two, there is a dense embedding
\[
w_{1}^{-1} w_2^{-1} \CC[w_1,w_2] \hookrightarrow H^2 \bigg( \Omega^\bu\left(\CC^2_w \times \RR \setminus 0 \right) \, / \, (\d w_1, \d w_2) \bigg).
\]

It will be useful to explain this last embedding in more detail. 
Consider the homogenous element $w_1^{-1} w_2^{-1}$. 
This represents the class of the Dolbeault-de Rham two-form
\[
\frac{\wbar_1 \d \wbar_2 \wedge \d t - \wbar_2 \d \wbar_1 \wedge \d t + t \d \wbar_1 \wedge \d \wbar_2}{(\|w\|^2 + t^2)^{5/2}} .
\]
Notice that, if we wedge with the volume form $\d w_1 \d w_2$, this is the unit  flux ($N=1$) introduced in Lemma \ref{lem:ads7flux}. 
The homogenous element $w_1^{-n-1} w_2^{-m-1}$ represents the class of the holomorphic derivatives $\partial_{w_1}^n \partial_{w_2}^{m}$ applied to this two-form. 

Observe that, when restricted to $\CC^5 \times \RR \setminus \CC^3$, the holomorphic tangent bundle along $\CC^5$ is still trivializable. 

\parsec[]

Let's turn to the proof of Proposition~\ref{prop:brads7}.
We proceed completely analogously to the case of backreacted M2 branes as in the proof of Proposition \ref{prop:brads4}. 

\begin{proof}[Proof of Proposition \ref{prop:brads7}]
Let $\cL (\CC^5 \times \RR \setminus \{w_1=w_2=t=0\})$ denote the super $L_\infty$ algebra obtained by parity shifting the fields of the eleven-dimensional theory on $\CC^5 \times \RR \setminus \{w_1=w_2=t=0\}$. 

There is a spectral sequence which converges to the cohomology of the fields with respect to the deformed linear BRST differential $\delta^{(1)} + [N F_{M5},-]$ whose first page
is the cohomology with respect to the original linearized BRST differential $\delta^{(1)}$. 
Recall that the linearized BRST differential decomposes as
\[
\delta^{(1)} = \dbar + \d_{\RR} + \div |_{\mu \to \nu} + \del |_{\beta \to \gamma}  .
\]
To compute this page, we use an auxiliary spectral sequence which simply filters by the holomorphic form and polyvector field type. 
This first page of this auxiliary spectral sequence is simply given by the cohomology of the fields supported on 
\[
\CC^5 \times \RR \setminus \{w_1=w_2=t=0\} \cong \CC_z^3 \times (\CC^2_w \times \RR \setminus 0)
\]
with respect to $\dbar + \d_{\RR}$. 

To compute this cohomology we follow the discussion in \S \ref{s:thfcohomology}.
Just as in the case of the M2 brane, we see that the $\dbar + \d_{\RR}$ cohomology is (up to completions) is the direct sum of the cohomology on flat space $H^\bu(\cL(\CC^5 \times \RR), \dbar)$ with
\begin{equation}
  \label{eqn:ads7ss2} 
  \begin{tikzcd}[row sep = 1 ex]
    + & - \\ \hline
w_1^{-1} w_2^{-1} \CC[w_1^{-1}, w_2^{-1}][z_1,z_2,z_3] \{\partial_{w_i}\}  \ar[r, dotted, "\div"] & w_1^{-1} w_2^{-1} \CC[w_1^{-1}, w_2^{-1}] [z_1,z_2,z_3] \\
w_1^{-1} w_2^{-1} \CC[w_1^{-1}, w_2^{-1}] [z_1,z_2,z_3] \{\del_{z_i}\} \ar[ur, dotted, "\div"'] \\
w_1^{-1} w_2^{-1} \CC[w_1^{-1}, w_2^{-1}] [z_1,z_2,z_3] \ar[r, dotted, "\del"] \ar[dr, dotted, "\del"'] & w_1^{-1} w_2^{-1} \CC[w_1^{-1}, w_2^{-1}][z_1,z_2,z_3] \{\d z_i\} \\ & w_1^{-1} w_2^{-1} \CC[w_1^{-1}, w_2^{-1}][z_1,z_2,z_3] \{\d w_i\} .
\end{tikzcd}
\end{equation}

Recall that the flux $F$ was defined as the image under $\del$ of some $\gamma$-type field. 
Therefore, the class $[F]$ does not live inside this page of the spectral sequence, but the operator $[[F], -]$ does act on this page nevertheless. 
For instance, if $f^i(z,w) \d z_i$ is a one-form living in $H^0(\CC^5, \Omega^1) \otimes H^0(\RR)$, then
\[
[ [F] , f^i (z,w) \d z_i ] = \ep_{ijk} w_1^{-1} w_2^{-1} \partial_{z_j} f^i(z,w) \del_{z_k} 
\]
which is an element in 
\[
\CC[w_1^{-1}, w_2^{-1}][z_1,z_2,z_3] \{\del_{z_i}\} \subset H^0(\CC^3, \T) \otimes H^2 \big(\Omega^\bu(\CC^2 \times \RR \setminus 0) / (\d w_1 , \d w_2) \big) .
\]

The first page of the spectral sequence converging to the cohomology with respect to $\delta^{(1)} + [N F, -]$ is given by the cohomology of the global symmetry algebra on $\CC^5 \times \RR$, which we computed in \S \ref{sec:global}, plus the cohomology with respect to the dotted-line operators in~\eqref{eqn:ads7ss2}. 

The next page of the spectral sequence is given by computing the cohomology with respect to the operator $[N F,-]$. 
This operator maps Dolbeault-de Rham degree zero elements to Dolbeault-de Rham degree two elements. 
For degree reasons, there are no further differentials and the spectral sequence collapses after the second page. 

The embedding of $\lie{osp}(6|1)$ for $N=0$ lives in the kernel of the original BRST operator $\delta^{(1)}$. 
To see that it this embedding can be lifted to the full cohomology we need to check that any element in the image of the original embedding is annihilated by $\big[ N [F] , - \big]$. 
This is a direct calculation. 
For instance, recall that an element in the image of the odd summand $\wedge^2 L \otimes R = \wedge^2 \CC^3 \otimes \CC^2$ (which corresponds to a superconformal transformation) is of the form $w_a (z_i \d z_j - z_j \d z_i)$, $a=1,2, i,j=1,2,3$. 
We have
\[
\big[[F] , w_a (z_i \d z_j - z_j \d z_i)\big] = 2 \ep_{ijk} (w_1^{-1} w_2^{-1}) \cdot w_a \del_{z_k} = 0
\]
since the class $w_1^{-1} w_2^{-1}$ is in the kernel of the operator given by multiplication by $w_a$ for $a=1,2$.
Verifying that the remaining elements in the image of $i_{M5}$ are in the kernel of $\big[ [F], -\big]$ is similar.
This completes the proof.
\end{proof}



%%\documentclass[11pt]{amsart}
%
%%\usepackage{../macros-master}
%\usepackage{macros-fivebrane}
%
%\begin{document}

\section{Twisted supergravity states}
\label{sec:states}

The first entry of the AdS/CFT dictionary in traditional treatments is a matching between \textit{supergravity states} and local operators in the CFT.
The goal of this section is to provide constructions of spaces of twisted supergravity states in our eleven-dimensional model, via geometric quantization. The state spaces on ${\rm AdS}_{7}\times S^{4}$ and ${\rm AdS}_{4}\times S^{7}$ have a remarkable property---they are naturally modules for certain infinite-dimensional exceptional super Lie algebras. We conclude the section by computing characters for these modules and comparing them with large $N$ indices for fivebranes and membranes in the literature.

Before proceeding with the construction, let us first give some feel for the situation we hope to describe. Suppose we consider a gravitational theory on $AdS_{d+1}\times S^{d^{\prime}}$, which we compactify to view as a theory on $AdS_{d+1}$ with all Kaluza-Klein harmonics included. Let $M^{d}$ denote the conformal boundary of $AdS_{d+1}$. A supergravity state is traditionally defined to be a solution to linearized equations of motion with a given boundary value \cite{}. Typically, this definition is made in situations where the relevant boundary value problem has a unique solution, in which case one may label states by the corresponding boundary values. Moreover, one may think of such boundary values as arising from modifications of a vacuum boundary condition at a point.


\subsection{Twisted Backreactions}
We begin by describing the relevant backgrounds. In eleven-dimensional supergravity, the $AdS_7 \times S^4$ background is obtained by backreacting a number of fivebranes in flat space \cite{Maldacena:1997re,WittenAdS}.
In \cite{RSW} we gave descriptions of twisted versions of this background. We will recall this construction, adapted to a slightly more global situation than considered previously.

We will consider the eleven-dimensional theory on eleven-manifolds that arise as total spaces of vector bundles. Placing the theory in the backreacted geometry is a 2-step procedure:

\begin{itemize}
  \item Place the eleven-dimensional theory on the complement of the zero section. To do so, we will wish to describe the complement of the zero-section in a way that facilitates natural operations on holomorphic-topological local $L_{\infty}$-algebras.

  \item Deform the theory on the complement of the zero section by a certain Maurer--Cartan element.
  The Maurer--Cartan element is thought of as the flux sourced by branes wrapping the zero section.
\end{itemize}

\parsec[s:brkevin]
As a way to highlight the key aspects of the construction, we detail the ingredients in the simplified model of Costello's twisted $M$ theory. The relevant local calculation can be found in the appendix of \cite{}; our goal here is to simply identify the salient global features that allow one to reduce to said local calculation.

We consider the theory on $M = \text{Tot} (\R\oplus K_{C})$, with some number of twisted `fivebranes' wrapping the zero section
\[
0 \times C \subset \R \times \T^* C .
\]
Denote by $t$ the real coordinate and by $w$ the fiber coordinate in $\T^* C$. We wish to describe the complement of the zero section $\mathring M = M - 0 \times C$.

Note that the bundle $\R\oplus K_{C}$ is equipped with a partially flat connection - this data equips the total space $M$ with the data of a transversely holomorphic foliation (THF) \cite{DuchampKalka}.

If we choose a fiberwise partially hermitian metric on the bundle $\R \oplus K_C$ we obtain a projection $p: \R \times \T^*C \to \R_{+} \times C$ which combines the fiberwise norm with the natural bundle projection. The restriction $p| \mathring M$ equips $\mathring M$ with the structure of an $S^{2}$-bundle over $\R_{>0}\times C$. Moreover, the partial flat connection on $\R\oplus K_{C}$ induces a partially flat connection on $\mathring M$. As part of this data, each of the fiber spheres is equipped with a complex foliation of rank 1.

Compactification amounts to pushing forward a local $L_{\infty}$-algebra along $p| \mathring M$. The result is a theory with infinitely many Kaluza--Klein modes along the fiber spheres. In the holomorphic-topological setting, the Kaluza-Klein modes will be modeled by a variant of Cauchy-Riemann cohomology.

Moreover, including the flux sourced by the brane deforms this structure. The lowest lying Kaluza-Klein modes in the deformed theory are equivalent to three-dimensional Chern-Simons theory.

For sake of analogy, we think of the resulting deformation as being a twisted version of $AdS_3 \times S^2$. \footnote{It is an interesting question if this corresponds the actual twist of a five--dimensional supersymmetric background of this form.}
We proceed to describe the twisted version of states at the boundary of this version of $AdS$.
We first proceed before turning on the flux sourced by the brane.

The theory admits a natural `vacuum' boundary condition at $r=0$.
In local coordinates, these are fields $\alpha(t,z,w)$ on the complement to the brane which extend to regular functions along the brane.

The `supergravity states' are, by definition, fields which satisfy the linearized equations of motion and satisfy the vacuum boundary condition except at a single point.
The linearized equations of motion are simply $(\d_{dR} + \dbar) \alpha = 0$.
Thus, up to equivalence, all solutions to the linearized equations of motion are constant in the real variable $t$, and holomorphic in $z,w$.

Modifications of the boundary condition at the point~$z = 0$ on the boundary take the form
\[
\alpha = f(w) \delta^{(r)}_{z=0}
\]
where $f$ is some holomorphic function.
Here $\delta^{(r)}_{z=0}$ denotes the $r$th derivative of the $\delta$-function at $z=0$.
It is convenient to parameterize such boundary modifications algebraically by expressions of the form
\[
\alpha_{k,r} = w^k \delta^{(r)}_{z=0} .
\]
Linear combinations of such states form a dense subspace of all possible modifications at the boundary.

The reason that the boundary modifications take this form can be seen by understanding in more explicit terms the vacuum boundary condition.
The phase space at the boundary $C$ can be identified with the following cohomology
\[
\Omega^{0,\bu}(C) \otimes \cA^{0;\bu}(\R \times \C - 0) [1]
\]
where $\cA^{0;\bu}$ denotes the mixed de Rham--Dolbeault cohomology of $\R \times \C - 0$ as a manifold equipped with a transversely holomorphic foliation \cite{DuchampKalka}.
We refer to the section below for a reminder on this geometric structure.

The phase space is equipped with a natural symplectic form given by
\[
\int_C \d z \oint_{S^2} \d w \, \alpha \wedge \alpha' .
\]
There is a natural Lagrangian inside of the phase space which consists of linear combinations of elements $\alpha(z) \otimes f(t,w)$ where $\alpha(z) \in \Omega^{0,\bu}(C)$ and $f(t,w)$ is a smooth function on $\R \times \C - 0$ which extends to zero.
The linearized equations of motion simply say that $\alpha$ is holomorphic, $f$ is independent of $t$ and depends holomorphically on $w$

\parsec[s:brfive]

We now consider the situation of backreacting some number of (twisted) fivebranes in our eleven-dimensional model.
Let $Z$ be a three-fold that the fivebranes wrap.
We also fix a rank 2 holomorphic vector bundle $W\to Z$ such that $\wedge^{2} W \cong K_{Z}$;
this condition ensures that the total space of $W$ is a Calabi-Yau five-fold. In the main body of the paper we will choose $W$ to be the bundle $K_{Z}^{1/2}\otimes \C^{2}$.

Consider the bundle $\R\oplus W$; this bundle has a canonical partially flat connection. We wish to consider our eleven dimensional model on $M = Tot (\R\oplus W)$ which is the total space of the \textit{real} rank five bundle $\R\oplus W$ over $Z$. The partially flat connection on $\R\oplus W$ equips $M$ with a canonical THF structure $F_{M}\subset T_{M}^{\C}$.

We place a stack of $N$ fivebranes wrapping the zero section in $\R\oplus W$.
Denote the complement of the zero section by
\[
\mathring{M} = \text{Tot}(\R\oplus W) - 0(Z).
\]

We may choose fiber coordinates of the bundle $t, w_{1}, w_{2}$ of $\R \oplus W$ over $Z$ and a fiberwise partially hermitian metric.

Explicitly, the corresponding norm defines a map
\begin{align*}
 h \colon  M & \to \R_{+} \\
  (t, w_{i}, \bar{w_{i}}, p)& \mapsto t^{2} + |w_{1}|^{2}+|w_{2}|^{2}
\end{align*}
Letting $\pi \colon M \to Z$ be the natural projection, we obtain the projection
\[
p \define (h,\pi) \colon M \to \R_{+}\times Z
\]
which restricts to an $S^4$ bundle $p|\mathring {M} \colon \mathring{M} \to \R_{>0} \times Z$.
These embeddings and projections fit inside of the following commutative diagram
\[
\begin{tikzcd}
\mathring{M} \ar[d,"p|\mathring M"'] \ar[r,hook] & M \ar[d,"p"] & \ar[l,hook',"0"'] Z \ar[d,"="] \\
\R_{>0} \times Z \ar[r,hook] & \R_{+} \times Z & \ar[l,hook',"0 \times \id"] Z.
\end{tikzcd}
\]
The inclusions on the left are the natural embeddings.
The top right inclusion is the zero section of $M = {\rm Tot}(\R \oplus V)$ and the bottom right inclusion is the embedding at radius $r = 0$.

As we elaborated in section \ref{s:thfmflds}, the eleven-dimensional theory is defined on the THF manifold $\mathring M$---in the BV formalism this is encoded, in part, by the sheaf of $L_\infty$ algebras $\cL_{sugra}$ on $\mathring M$. Compactification of this theory along the $S^4$ link corresponds to pushing forward this sheaf along $p|\mathring M$. The resulting sheaf of $L_\infty$ algebras $(p|\mathring M)_*\cL_{sugra}$ describes, in the BV formalism, the compactified theory on the seven-manifold $\R_{>0} \times Z$.

We will compute the pushforward using the prescription for pushing forward a $\cA^{\bu}_{F}$-module along a compatible submersion outlined in \cite{KormanThesis}. Recall that we have a THF structure on $M$, given by an involutive subbundle $F_{M}\subset T^{\C}_{M}$, induced from a partially flat conneciton on $\R\oplus W$. The situation is summarized in the diagram below.
\[
  \begin{tikzcd}
    0 \ar[r] & F_{M}\cap T_{M/Z}^{\C}\ar[r]\ar[d] & F_{M}\ar[r]\ar[d]  & \pi^{*}T^{0,1}_{Z}\ar[r]\ar[d] & 0 \\
    0 \ar[r] & T_{M/Z}^{\C}\ar[r] & T_{M}^{\C}\ar[r] & T_{Z}^{\C}\ar[r] & 0.
  \end{tikzcd}
\]

The existence of a partially flat connection amounts to a choice of splitting \[\sigma: \pi^{*}T^{0,1}_{Z}\to F_{M}\] of the top row. Both $F_{M}$ and $\pi^{*}T^{1,0}_{Z}$ are involutive, and the flatness of the connection implies that $\sigma$ preserves the lie brackets on sections.

We have a similar diagram expressing that the sphere bundle $\mathring M \to \R_{>0}\times Z$ is a map of THF manifolds when both the domain and codomain are equipped with the induced THF structures.
\[
  \begin{tikzcd}
    0 \ar[r] & F_{\mathring M}\cap T_{\mathring M/\R_{>0}\times Z}^{\C}\ar[r]\ar[d] & F_{\mathring M}\ar[r]\ar[d]  & h^{*}T_{\R_{>0}}\oplus\pi^{*}T^{0,1}_{Z}\ar[r]\ar[d] & 0 \\
    0 \ar[r] & T_{\mathring M/\R_{>0}\times Z}^{\C}\ar[r] & T_{\mathring M}^{\C}\ar[r] & h^{*}T_{\R_{>o}}\oplus T_{Z}^{\C}\ar[r] & 0.
  \end{tikzcd}
\]

\surya{conditions for top left to be a vector bundle}

We claim that the splitting $\sigma$ induces a splitting $\mathring \sigma : h^{*}T_{\R_{>0}}\oplus \pi^{*}T^{0,1}_{Z}\to F_{\mathring M}$. \surya{is this obvious?} It is clear that $\mathring \sigma$ preserves lie brackets.

We wish to describe $\cA^{\bu}_{F}$-modules, for $F$ an involutive subbundle appearing in the top row of the above diagram. To make the burden less cumbersome, we will drop mention of the foliation and instead refer to the directions in which the leaves lie. Explicitly, let $\cA_{\mathring M/\R_{>0}\times Z} = \cA_{F_{\mathring M}\cap T^{\C}_{\mathring M/ \R_{>0}\times Z}}$, $F_{\R_{>0}\times Z} = T_{\R_{>0}}\oplus T^{0,1}_{Z}$ and $\cA_{\R_{>0}\times Z} = \cA_{\pi^{*} F _{\R_{>0}\times Z}}$. Our goal is to describe the pushforward $(p|\mathring M)_{*}\cL_{sugra}$ as a $\cA^{\bu}_{\R_{>0}\times Z}$-module.

The splitting affords an isomorphism of $C^{\infty}_{\mathring M}$-modules
\[
  \cA^{\bu}_{\mathring M}(L)\cong \cA^{\bu}_{\R_{>0}\times Z}\otimes \cA^{\bu}_{{\mathring M/ \R_{>0}\times Z}}\otimes \cL
\]

Note that since we have a complex of locally free $C^{\infty}_{\mathring M}$-modules, the (a priori derived) functor $(p|\mathring M)_{*}$ has several nice properties. We may commute it past the tensor product above. Moreover, we may apply the functor term-wise. The result is that the pushforward is given by \surya{finish}

In \cite{KormanThesis}, the THF differential $\thfD$ on $\cA^{\bu}_{\mathring M}(L)$ is interpreted as defining a superconnection on the pushforward $\pi_{*}\cA^{\bu}_{\mathring M/R_{>0}\times Z}$ viewed as the sheaf of sections of a complex of pro-vector bundles on $\R_{+}\times Z$.

\surya{finish}

In sum, we have proven the following:

\begin{prop}
  The pushforward $(p| \mathring M)_{*}\cL_{sugra}$ is described by the following complex of sheaves
  \beqn
  \cA^{\bu}_{F_{\R_{>0}\times Z}}\left(H^\bu(\cA^\bu_{F_{\mathring M/ R_{>0}\times Z}}(\cL), \thfd)\right)
  \eeqn

  The BV pairing is given by
  \beqn
  \int_{}
  \eeqn

\end{prop}

The model for the pushforward in the previous proposition appears naturally on the $E_{1}$-page of a Leray-Serre-type spectral sequence \cite{KormanThesis} that converges to the cohomology with respect to the totalized differential $D+\thfd$ $H^{\bullet}(\cA_{\mathring M}(\cL), D+\thfd)$. In \cite{RSW}, this spectral sequence was implicitly used to give a concrete description of the pushforward $(p|\mathring M)_{*}\cL_{sugra}$ in the case where $M = Tot (\R\oplus \C^{2}\to \C^{3})$. We recall the result. Fix holomorphic coordinates $z_{i}, i=1, 2, 3$ on $\C^{3}$ and holomorphic fiber coordinates $w_{a}, a= 1, 2$ on $\C^{2}$.

\begin{prop}
Up to completion, the cohomology $H^{\bu}(\Gamma(\cA_{\mathring M}^{\bu}(\cL)), \thfD)$ is a direct sum of the cohomology on flat space $H^{\bu}(\gamma(\cA)$
\end{prop}

\parsec[s:flux]
The second step in the construction of the backreacted geometry involves deforming the pushforward $(p| \mathring{M})_{*}\cL_{sugra}$ by a certain Maurer-Cartan element. The Maurer-Cartan element is determined by the lowest order term in the coupling between the eleven dimensional theory and the stack of fivebranes. It was argued in \cite{RSW} that the relevant coupling is given by the nonlocal interaction
\beqn\label{eqn:br1}
I_{fivebrane} = N\int_{Z} D_\Omega^{-1}\mu \vee \Omega +\cdots
\eeqn
where $\mu \in \Omega^0 (\R) \hotimes \PV^{1,3}(X)\subset \cA^{\bullet}_{F}(\vartheta_{F})$ is a component of a field in the eleven-dimensional theory which satisfies $\thfd_{\Omega} \mu = 0$.

The relevant Maurer-Cartan element is a solution to the equations of motion for the eleven-dimensional theory deformed by this interaction. After deforming with this interaction, varying with respect to $D_{\Omega}^{-1}\mu$ yields the following equation for $\gamma$
\beqn
\thfd D \gamma + D_\Omega\left (\frac{\mu}{1-\nu}\right )\wedge D \gamma = N \delta_Z.
\eeqn

%\parsec
%Let $C$ be a curve, and let $V\to C$ be a rank 4-holomorphic vector bundle over $C$ such that $\wedge^{4} V = K_{C}$. This condition again ensures that $X = {\rm Tot} V$ is a Calabi-Yau five-fold - in the main body of the paper, we will take $V = K^{1/2}_{C}\otimes \C^{4}$. Abusively letting $V$ also denote its pullback along the canonical projection $\R\times C \to C$, we may view $\R\times X$ as the total space of $V$ on $\R\times C$. As before we will consider wrapping a stack of $N$ membranes along the zero section.

%Since $V$ is a complex vector bundle, we may choose a fiberwise hermitian metric, and as before, we may view $\R\times X \setminus \R\times C$ as an $S^{7}$-bundle over $\R_{>0}\times \R\times C$.



%\parsec[s:sugraops]
%
%By the usual methods of the BV formalism the action functional $S_{sugra}$ described above endows the parity shift of the fields $\cL_{sugra} = \Pi \cF_{sugra}$ with the structure of a holomorphic-topological local $\Z/2$ graded $L_\infty$ algebra. 
%
%On $\C^5 \times \R$ we can describe this super Lie algebra structure explicitly. 
%First, by the Dolbeault and de Rham Poincar\'e lemmas it is easy that the even part of the super Lie algebra $\cL(\C^5 \times \R)$ is equivalent to a one-dimensional central summand $\C$ plus the Lie algebra of divergence-free vector fields on $\C^5$:
%\[
%\Vect_0 (\C^5) = \{X \in \Vect(\C^5) \; | \; \div X = 0\} .
%\]
%The odd part of the super Lie algebra $\cL(\C^5 \times \R)$ is equivalent to the space of holomorphic one-forms on $\C^5$ modulo exact one-forms
%\[
%\Omega^{1,hol}(\C^5) / {\rm Im}(\del) 
%\]
%which is, of course, equivalent to the space of closed holomorphic two-forms $\Omega^{2,hol}_{cl}(\C^5)$. 
%
%\begin{thm}[\cite{RSW}[Theorem 2.1]]
%The Taylor expansion map determines a map of $\Z/2$ graded $L_\infty$ algebras
%\[
%j_\infty \colon \cL_{sugra}(\C^5 \times \R) \to L_{sugra} .
%\]
%Furthermore, $L_{sugra}$ is equivalent as a $\Z/2$ graded $L_\infty$ algebra to $\Hat{E(5|10)}$. 
%\end{thm} 
%
%As an immediate corollary of this result we obtain by Lemma \ref{lem:localops} the following.
%
%\begin{cor}
%\label{cor:sugraops}
%Let $\Obs_{sugra}$ be the factorization algebra on $\C^5 \times \R$ of classical observables of the minimal twist of eleven-dimensional supergravity.
%There is a quasi-isomorphism of commutative dg algebras
%\[
%\Obs_{sugra} (0) \simeq \clie^\bu \left( \Hat{E(5|10)} \right) .
%\]
%\end{cor}
\parsec[s:hilbertspace]
The description of the $S^{4}$ compactification of eleven dimensional supergravity obtained in the previous subsection facilitates a straightforward construciton of the Hilbert space. The compactification is described by the pro-local Lie algebra $(p|\mathring M)_{*} \cL_{sugra}$ on the seven-manifold $\R_{>0}\times Z$. We will construct the Hilbert space by applying geometric quantization ansatz to the phase space at $\infty \in \R_{>0}$.

Our ansatz for geometric quantization will avoid discussion of subtler aspects such as the metaplectic correction. We define the geometric quantization to be given by functions that are constant along the leaves of a lagrangian foliation of the phase space.

The phase space at $\infty\in \R_{>0}$ is computed by restricting to a neighborhood $(a, \infty)\subset \R_{>0}$:
\beqn
(p| \mathring M)_* \cL_{sugra}|_{\infty \times Z}=
\eeqn

There is a symplectic pairing given by
\beqn
\eeqn



\subsection{Global symmetry for twisted $AdS$}
\label{s:global1}

After complexification, the~six-dimensional superconformal algebra is isomorphic to $\lie{osp}(8|4)$.
The even part of this algebra is $\lie{so}(8) \times \lie{sp}(4)$.
This algebra contains the six-dimensional $\cN=(2,0)$ supersymmetry algebra whose odd part is four copies of $S^{6d}_+$, the positive irreducible complex spin representation of $\lie{so}(6)$.
%It also contains the three-dimensional $\cN=8$ supersymmetry algebra whose odd part is eight copies of $S^{3d}$, the irreducible complex spin representation of $\lie{so}(3)$.

The holomorphic supercharge is a supertranslation
\[
Q \in \Pi S^{6d}_+ \otimes \C^4 \subset \lie{osp}(8|4)
\]
which is characterized (up to equivalence) by the properties that $Q^2 = 0$ and that its image
\[
{\rm Im}\left(Q|_{\Pi S_+ \otimes \C^4} \right) \subset \R^6 \otimes_\R \C \cong \C^6
\]
is three-dimensional (spanned by the anti-holomorphic translations). 
The supercharge $Q$ acts on $\lie{osp}(8|4)$ by commutator and the resulting cohomology will automatically act on the holomorphic twist of any six-dimensional superconformal field theory. 
This cohomology can readily be identified with the subalgebra $\lie{osp}(6|2)$, see \cite{SWe36}. 

%Similarly, the minimal twisting supercharge in the three-dimensional $\cN=8$ supersymmetry algebra is an element $Q \in \Pi S^{3d} \otimes \C^8$ which is characterized (up to equivalence) by the property that $Q^2 = 0$ and that the image of $[Q,-]$ is two-dimensional. The cohomology of $\lie{osp}(8|4)$ with respect to this supercharge is also isomorphic to~$\lie{osp}(6|2)$.

In \cite{RSW} we have shown that solutions to equations of motion of our eleven-dimensional theory in the twisted $AdS_{7}\times S^{4}$ background recalled in the previous subsection contains the symmetry algebra $\lie{osp}(6|2)$. This is precisely the twist of the superconformal algebra we just discussed, so this statement may be interpreted as the twisted avatar of the fact that $lie{osp}(8|4)$ acts as isometries on $AdS_{7}\times S^{4}.$ An easy consequence of the result of \cite{RSW} and the definition of $\cH_{sugra}$ in \ref{hilbert} is that the twisted symmetry algebra $\lie{osp}(6|2)$ acts on $\cH_{sugra}$.

We will enumerate states via their weights a Cartan in the bosonic subalgebra of the twisted superconformal algebra.
We recall below how the bosonic piece of the algebra
\beqn
\label{eqn:gut}
\lie{sl}(4) \times \lie{sl}(2) \subset \lie{osp}(6|2) 
\eeqn
embeds as symmetries, or ghosts, of the eleven-dimensional theory in this twisted background.

\brian{enhance to E(3|6)}.
However, we first note that the hilbert space $\cH_{sugra}$ in fact enjoys an action of a much larger symmetry algebra.



\subsection{Mass Spectroscopy on twisted $AdS_7\times S^{4}$}
\label{s:ads7}

For convenience we choose coordinates on the eleven manifold as
\[
\R \times \C^5 = \R_t \times \C^2_w \times \C_z^3 
\]
with $z = (z_i), i=1,2,3$ and $w = (w_a), a=1,2$.
The stack of fivebranes wrap 
\beqn
w_1=w_2=t=0 .
\eeqn
Important for us is to recall that part of the ghost system for our eleven-dimensional theory consists of divergence-free vector fields on $\C^5$ which are locally constant along $\R$. 

The subalgebra \eqref{eqn:gut} of the twisted superconformal algebra $\lie{osp}(6|2)$ embeds as ghosts in our eleven-dimensional model as follows.
\begin{itemize}
\item
The subalgebra $\lie{sl}(3) \subset \lie{sl}(4)$ embeds as vector fields rotating the plane $\C^3_z$
\beqn
\sum_{ij} A_{ij} z_i \frac{\del}{\del z_j} \in \PV^{1,0}(\C^5)\otimes \Omega^0(\R) , \quad (A_{ij}) \in \lie{sl}(3) .
\eeqn
By definition, these vector fields are automatically divergence-free.
Notice that this vector field is divergence-free and restricts to the Euler vector field along $t=w_{a} = 0$.
\item 
The subalgebra $\lie{sl}(2)$ ($R$-symmetry) is mapped to the triple
\beqn
 w_1 \frac{\del}{\del w_2}, \quad w_2 \frac{\del}{\del w_1}, \quad \frac{1}{2}\left (w_1\frac{\del}{\del w_1}-w_2\frac{\del}{\del w_2}\right) \in \PV^{1,0}(\C^5) \otimes \Omega^0(\R) .
\eeqn
\item Scaling on the plane $\C^3$ embeds as the vector field
\beqn\label{eqn:Delta}
        \Delta = \sum_{i=1}^3 z_i\frac{\del}{\del z_i} - \frac 32\sum_{a=1}^2 w_a\frac{\del}{\del w_a}\in \PV^{1,0}(\C^5)\otimes \Omega^0 (\R).
\eeqn
\end{itemize}

%\begin{rmk}
%In the classification of simple super Lie algebras, Kac makes use of a weight grading $\oplus_{j \geq -2} \fg_j$ of the exceptional Lie algebra $E(3|6)$ for which the finite-dimensional subalgebra above is the weight zero piece
%\cite{KacClass}.
%We will make use of this grading in \S \ref{s:kr}.
%\end{rmk}
This describes an embedding of the algebra 
\beqn\label{eqn:gut2}
\lie{sl}(3) \times \lie{sl}(2) \times \lie{gl}(1) \subset \lie{sl}(4) \times \lie{sl}(2) 
\eeqn
into the ghosts of our model.
The dimension of a Cartan subalgebra of $\lie{sl}(3) \times \lie{sl}(2) \times \lie{gl}(1)$ is four and accordingly, the equivariant character we study has four fugacities.
We choose these explicitly as follows:
\begin{itemize}
  \item $t_{1}, t_{2}$ denote generators for the Cartan of $\lie{sl}(3)$ which is spanned by the vector fields
  \beqn
  h_1 = z_1 \frac{\del}{\del {z_1}} - z_2 \frac{\del}{\del{z_2}} , \quad h_2 = z_2 \frac{\del}{\del{z_2}} - z_3 \frac{\del}{\del{z_3}}.
  \eeqn
  \item $r$ denotes a generator for the Cartan of a $\lie{sl}(2)$ which is generated by the element 
  \beqn
  \label{eqn:hCartan}
  h = \frac12 \left(w_1 \frac{\del}{\del w_1} - w_2 \frac{\del}{\del w_2}\right) .
  \eeqn
\item $q$ denotes a generator for the Cartan of the~$\lie{gl}(1)$ which is generated by the element $\Delta$ from equation~$\eqref{eqn:Delta}$. 
\end{itemize}

The twisted supergravity states $\cH_{sugra}^{6d}$ form a representation for $\lie{osp}(6|2)$. 
The weights of twisted supergravity states with respect to the generators of the Cartan subalgebra above are completely determined by the weights of the holomorphic coordinates on $\C^2_w \times \C^3_z$.
These are summarized in table \ref{tbl:sugraM5}.

\begin{table}
\begin{center}
\begin{tabular}{c c c c c c}
  & $z_{1}$ & $z_{2}$ & $z_{3}$ & $w_{1}$ & $w_{2}$ \\
  \hline
  $t_{1}$ & $1$ & 0 & $-1$ & 0 & 0 \\
  $t_{2}$ & 0 & 1 & $-1$ & 0 & 0 \\
  $r$ & 0 & 0 & 0 & 1 & $-1$ \\
  $q$ & $-1$ & $-1$ & $-1$ & $\frac{3}{2}$ & $\frac{3}{2}$
\end{tabular}
\caption{Fugacities for the fields of the holomorphic twist of eleven-dimensional supergravity for the geometry $\R \times \C^5 \setminus \C^3$.}
\label{tbl:sugraM5}
\end{center}
\end{table}

We enumerate single particle supergravity states via computing the super trace of the operator $q^Y t_1^{h_1} t_2^{h_2} r^h$ acting on $\cH^{6d}_{sugra}$:
\beqn
f^{6d}_{sugra}(t_1,t_2,r,q) = \Tr_{\cH_{sugra}^{6d}} (-1)^F t_1^{h_1} t_2^{h_2} r^h q^\Delta .
\eeqn
The super trace means that there is an extra factor of $(-1)^F$, where $F$ is parity (fermion number), when computing the ordinary trace. 
That is, we compute the expression


\begin{prop}
\label{prop:sugraindex1}
The single particle index of the space of twisted supergravity states~$\cH_{sugra}^{6d}$ is given by the following expression
\beqn
\label{eqn:sugra_index}
f_{sugra}^{6d} (t_1,t_2, r, q) = \frac{q^4(t_1^{-1}+t_1t_2^{-1}+t_2)-q^2(t_1+t_1^{-1}t_2+t_2^{-1})+(q^{3/2}-q^{9/2})(r+r^{-1})}{(1-t_{1}^{-1}q)(1-t_{2}q)(1-t_{1}t_{2}^{-1}q)(1-rq^{3/2})(1-r^{-1}q^{3/2})}.
\eeqn
The full (multiparticle) index is defined to be the plethystic exponential 
\beqn
{\rm PExp}\left[f_{sugra}^{6d}(t_1,t_2,r,q)\right] .
\eeqn
\end{prop}

To simplify the form of this index we can introduce a different parametrization of the Cartan of $\lie{sl}(3) \times \lie{sl}(2) \times \lie{gl}(1)$.
First, we can parameterize the Cartan of $\lie{sl}(3)$ by the vector fields
  \beqn\label{eqn:ys}
  -(\log y_1) z_1 \frac{\del}{\del {z_1}} - (\log y_2) z_2 \frac{\del}{\del{z_2}} - (\log y_3) z_3 \frac{\del}{\del{z_3}} .
  \eeqn
where $y_1,y_2,y_3$ are parameters which satisfy the single constraint
\beqn
y_1 y_2 y_3 = 1 .
\eeqn
In terms of the variables $t_1,t_2$ used above we have
\beqn
y_1 = t_1^{-1},\quad y_2 = t_1 t_2^{-1}, \quad y_3 = t_2 .
\eeqn
Second, we can parametrize the Cartan of the remaining subalgebra $\lie{sl}(2) \times \lie{gl}(1)$ by the two vector fields
\beqn
\til{h} = h + \frac12 \Delta \quad \text{and} \quad \Delta
\eeqn
where $\Delta$ is as in equation \eqref{eqn:Delta} and $h$ is as in \eqref{eqn:hCartan}.
We denote by $y$ the generator of the Cartan corresponding to the vector field $\til{h}$ and by $q$ (as above) the generator corresponding to~$\Delta$.
In terms of the variable~$r$ used above we have 
\beqn
y = q^{1/2} r .
\eeqn

Using the paramterization fo the Cartan given by the variables $y_i,y,\Delta$ we obtain the equivalent expression for the index \eqref{eqn:sugra_index} as 
\beqn
\label{eqn:Kim_sugra}
f_{sugra}^{6d} (y_i, y, q) = \frac{q^4(y_1+y_2+y_3)-q^2(y_1^{-1} + y_2^{-1} + y_3^{-1})+(1-q^3)(yq + y^{-1} q^2)}{(1-y_1 q)(1-y_2 q)(1-y_3 q)(1-yq)(1-y^{-1} q^2)},
\eeqn
We note that this matches exactly with the index computed in \cite[Eq. (3.23)]{Kim:2013nva} with the change of variables.
%\beqn
%t_1 = y_1^{-1} , \quad t_2 = y_3, \quad r = q^{-1/2} y 
%\eeqn 

Our formula \eqref{eqn:sugra_index} also matches with \cite[Eq. (3.24)]{Bhattacharya:2008zy} where we use the change of variables
\beqn
q = x^4, \quad t_1 = y_2, \quad t_2 = y_1, \quad r^2 = z .
\eeqn
(Notice the variables $y_1,y_2$ used in \cite{Bhattacharya:2008zy} differ from the variables we introduced in \eqref{eqn:ys}.)
We record a few specializations of this index which we will remark on further in~\S\ref{s:??}.

\parsec 
The specialization of this index $q=r^2, t_2=1$ in \eqref{eqn:special1} yields the plethystic exponential of the following single particle index
\[
f_{sugra}^{6d}(q, t_1, t_2=1, r = q^{1/2}) = \frac{q}{(1-q)^2}
\]

This plethystic exponential yields the Macmahon function, which is the character of the vacuum module of the $W_{1+\infty}$-algebra.

\parsec

The specialization $t_1=t_2=r=1$ yields the single particle index
\[
f_{sugra}^{6d} (q, t_1=t_2=r=1) = \frac{3 q^4 - 3 q^2 + 2 q^{3/2} - 2 q^{9/2}}{(1-q)^3 (1-q^{3/2})^2} .
\]

\parsec The same change of variables in \eqref{eqn:special2} agrees with previously computed indices for single particle states for supergravity on $AdS_{7}\times S^{4}$


\section{The non-minimal twist} 

We have provided numerous consistency checks that the 11-dimensional theory defined on a manifold with $SU(5)$ holonomy is a twist of supergravity. 
We have referred to this theory as minimal as it depends on the complex structure in the maximal number of directions. 
In this section we characterize a further twist of 11-dimensional supergravity from the lens of the holomorphic theory.  
This further twist is invariant for the group $G_2 \times SU(2)$ and is topological along seven directions, as opposed to just a single direction as in the $SU(5)$ twist. 

On flat space, the further twist is essentially unique and renders the theory topological in seven directions, rather than just one as in the holomorphic twist. 
We will show that it is equivalent to a theory on $\CC^2 \times \RR^7$ that we call ``Poisson'' Chern--Simons theory. 
In the BV formalism, the theory $\ZZ/2$ graded and has fields given by
\[
A \in \Pi \Omega^{0,\bu}(\CC^2) \hotimes \Omega^\bu(\RR^7) ,
\]
where $\Pi$, as always, denotes parity shift.
The equations of motion are of the form
\[
\dbar A + \d_{\RR^7} A + \partial_{z_1} A \wedge \partial_{z_2} A = 0 .
\]
The action functional depends on the holomorphic symplectic structure on $\CC^2$ through the Poisson bracket on the algebra of holomorphic functions.
We give a precise definition below. 

The main result of this section is the following.

\begin{thm}
\label{thm:nonmin}
The non-minimal twist of the 11-dimensional theory is equivalent to Poisson Chern--Simons theory on 
\[
\CC^2 \times \RR^7 .
\]
\end{thm}

\subsection{Index matching}
\label{sec:indexcheck}

As a first consistency check, we can compare deformation invariants attached to the holomorphic twist and the $G_2\times SU(2)$ twist. %\brian{If we are going to say G2 we need to explain how G2 arises. I've been calling it the non minimal twist.}
We will find that the local character of the latter agrees with a specialization of the local character computed in proposition \ref{prop:locchar}

\begin{prop}
The  local character of the $G_2\times SU(2)$-invariant twist of 11d supergravity on flat space is given by
\[
\chi_{SU(3)\times SU(2)}(\Obs_0^{\cE_{G2\times SU(2)}}) = \prod _{(n_1,n_2)\in \Z^2_{\geq 0}} \frac{1}{1-q^{-n_1+n_2}}.
\] 
This agrees with the specializaiton of the local character computed in proposition \ref{prop:locchar} at $y=1$.
\end{prop}
\begin{proof}
The space of solutions to linearized equations of motion is parametrized by a holomorphic function $A$ on $\C^2_{w_j}$. The corresponding linear local operators are labeled by $(n_1,n_2)\in \Z^2_{\geq 0}$  and are given by 
\[
A_{(n_1,n_2)} : A \mapsto \partial_{w_1}^{n_1}\partial_{w_2}^{n_2} A (0).
\]

The character of the linear span of these is given by the geometric series
\[
\sum _{(n_1,n_2)\in \Z^2_{\geq 0}} q^{-n_1+n_2}
\] 
with plethystic exponential given by 
\[
\prod _{(n_1,n_2)\in \Z^2_{\geq 0}} \frac{1}{1-q^{-n_1+n_2}}.
\]

For the last part, it suffices to show that the characters of linear local operators agree upon evaluating $y=1$. Summing the geometric series giving the character of linear local operators in the holomorphic twist yields the rational function \[\frac{\left (\begin{aligned}t_1y^{-1} + t_1^{-1}t_2y^{-1} + t_2^{-1}y^{-1} + q^{-1}y^{\frac 3 2} + qy^{\frac 3 2} \\  - (t_1^{-1}y + t_1t_2^{-1}y + t_2y &+ q^{-1}y^{-\frac 3 2} + qy^{-\frac 3 2})\end{aligned}\right)}{(1-t_1^{-1}y)(1-t_1t_2^{-1}y)(1-t_2y)(1-q^{-1}y^{-3/2})(1-qy^{-3/2})}.\] Specializing $y=1$, the numerator cancels with the factors in the denominator involving the $t_i$. The result is \[\frac{1}{(1-q^{-1})(1-q)}\] which is exactly the sum of the geometric series we found

\end{proof}

%Notice that changing the values of $i,j$ just has the affect of permuting the holomorphic copies of $\CC^2$ leftover in the further twist. 

%The Lie algebra of gauge symmetries of this model on flat space is 
%\[
%\Omega^{0,\bu} (\CC_i \times \CC_j) \hotimes \Omega^\bu(\RR^7) 
%\]
%which is quasi-isomorphic to $\cO^{hol}(\CC^2)$ equipped with the Poisson bracket $\{-,-\}_{pb}$. 
%
%More generally, the twist can be defined \brian{finish}

\subsection{The non-minimal supercharge}

From the point of view of the untwisted theory, the non-minimal twist is defined by working in a background where the fermionic ghost in the physical theory is equal to a supertranslation of the form
\[
Q + Q_{nm} 
\]
where $Q$ is the supertranslation which defines the minimal twist, see \S \ref{sec:mintwist}.
The minimal twist of supergravity is obtained by setting a fermionic ghost equal to $Q$. 

In the language of the minimal twist, the supercharge $Q_{nm}$ determines a square-zero element in the $Q$-cohomology of the original supersymmetry algebra (which we will denote by the same letter). 
The characterization of this cohomology in Proposition \ref{prop:susycoh} implies that $Q_{nm}$ is an element 
\[
Q_{nm} \in \wedge^2 L 
\]
where $L \cong \CC^5$ is the defining $SU(5)$ representation. 
In other words, $Q$ is a translation invariant holomorphic two-form on $\CC^5$. 
The condition that $[Q_{nm}, Q_{nm}] = 0$ simply says that $Q_{nm}\wedge Q_{nm} = 0$ as a translation invariant four-form on $\CC^5$. 
By a linear change of coordinates, all such two-forms $Q$ are of the form $Q_{nm} = \d z_i \wedge \d z_j$ where $i,j=1,\ldots, 5$.

From hereon in this section we will rename coordinates by
\[
\CC^5 \times \RR = \CC^2_{z_i} \times \CC^3_{w_a} \times \RR
\]
which is most natural from the point of view of the non-minimal twist. 
We will fix the non-minimal supercharge 
\[
Q_{nm} = \d z_1 \wedge \d z_2 .
\]
Notice that this choice of supercharge breaks the holonomy of the 11-dimensional theory from $SU(5)$ to $SU(2)$. 

We constructed an embedding of the $Q$-cohomology of the supersymmetry algebra into the fields of our 11-dimensional theory on $\CC^5 \times \RR$. 
The further twist is obtained by working in a background where a certain field on $\CC^5 \times \RR$ takes nonzero value $Q_{nm}$. 
Explicitly, the element $Q_{nm} \in \wedge^2 L$ corresponds to the image under $\del$ of a $\gamma$-field of type $\Omega^{1,0}(\CC^5) \otimes \Omega^0(\RR)$. 
According to the embedding in \S \ref{s:residual} this is the $\gamma$-field 
\beqn\label{eqn:gammanm}
\gamma_{nm} = \frac12 (z_1 \d z_2 - z_2 \d z_1) \in \Omega^{1,0}(\CC^5) \otimes \Omega^0(\RR) 
\eeqn
Notice that $\del \gamma_{nm} = \d z_1 \wedge \d z_2$ as desired.

\parsec[sec:nmsymmetry]
Before proceeding to the proof of the theorem above, we perform a simple calculation of the global symmetry algebra present in the $Q_{nm}$-twisted theory. 

Recall that up to a copy of constant functions, the global symmetry algebra of the holomorphic twist of the 11-dimensional theory is the super Lie algebra $E(5,10)$.
From this point of view, the global symmetry algebra of the $Q_{nm}$-twisted theory is given by deformation of this super Lie algebra by the Maurer--Cartan element 
\[
\d z_1 \wedge \d z_2 \in \Omega^{2}_{cl}(\CC^5) .
\]
We recall that the space of closed two-forms on $\CC^5$ is precisely the odd part of the super Lie algebra $E(5,10)$. 

We compute the cohomology of $E(5,10)$ with respect to the differential which is bracketing with this Maurer--Cartan element. 
Recall that we are using the holomorphic coordinates $(z_1,z_2,w_1,w_2,w_3)$ on $\CC^5$. 

There are the following brackets in the super Lie algebra $E(5,10)$ 
\begin{align*}
[f_l \partial_{z_l} , \d z_1 \wedge \d z_2] & = \del f_i \wedge \d z_j - \del f_j \wedge \d z_i \\
[g_a \partial_{w_a} , \d z_1 \wedge \d z_2] & = 0 \\
[h^{ab} \d w_a \wedge \d w_b , \d z_1 \wedge \d z_2 ] & = \ep_{abc} h^{ab} \partial_{w_c} .
\end{align*}
where $f_l \partial_{z_l}$, $g_a \partial_{w_a}$ are divergence-free vector fields on $\CC^5$ and $h^{ab} \d w_a \wedge \d w_b$ is a closed two-form. 

From these relations, we see that the following elements are in the kernel of $[\d z_1 \wedge \d z_2, -]$:
\begin{itemize}
\item $f(z_i) \partial_{z_1} + g(z_i) \partial_{z_2}$ for holomorphic functions $f,g$ on $\CC_{z_1} \times \CC_{z_2}$ which satisfy 
\[
\del_{z_1} f + \del_{z_2} g = 0 .
\]
In other words, this is a divergence-free vector field on $\CC_{z_1} \times \CC_{z_2}$. 
\item $f_b(z_i, w_a) \partial_{w_b}$ for $f_b$ a holomorphic function on $\CC^5$ where $b=1,2,3$. 
\end{itemize}
It is immediate to check that these are the only nonzero elements in the kernel. 
Further, any element of the second type is clearly exact by the closed two-form $\ep^{ijklm} f \d z_l \d z_m$. 

Thus, the cohomology is the (purely bosonic) Lie algebra of divergence-free vector fields on $\CC^2 = \CC_i \times \CC_j$
\[
H^\bu\big(E(5,10), [\d z_1 \wedge \d z_2, -] \big) \simeq \Vect_0(\CC^2) .
\]

We proved in Theorem \ref{thm:global} that the global symmetry algebra of the 11-dimensional theory on $\CC^5 \times \RR$ is equivalent to a central extension $\Hat{E(5,10)}$ of the super Lie algebra~$E(5,10)$. 

The Lie algebra of divergence-free vector fields on $\CC^2$ also admits a central extension:
\beqn\label{eqn:centralvect}
0 \to \CC \to \cO (\CC^2) \to \Vect_0 (\CC^2) \to 0
\eeqn
where $\cO(\CC^2)$ is equipped with the Poisson bracket with respect to the symplectic form~$\d z_1 \wedge \d z_2$.
These two central extension are compatible. 

%Thus, the Lie algebra of gauge symmetries of Poisson Chern--Simons theory on $\CC_i \times \CC_j \times \RR^7$ is a trivial one-dimensional central extension of the cohomology of $E(5,10)$. 

\begin{prop}
Let $\Hat{E(5,10)}$ be the central extension of $E(5,10)$ which is equivalent to the global symmetry algebra of the 11-dimensional theory on $\CC^5 \times \RR$. 
Then, there is an isomorphism of Lie algebras 
\[
H^\bu \big(\Hat{E(5,10)} , [\d z_1 \wedge \d z_2, -] \big) \simeq \cO(\CC^2) .
\]
\end{prop}
\begin{proof}
The only thing to check is that, in cohomology, the cocycle defining the central extension of $E(5,10)$ is the cocycle exhibiting $\cO(\CC^2)$ as the central extension of divergence-free vector fields. 
Recall that the formula \eqref{eqn:cocycle} for the cocycle is 
\[
\varphi(\mu, \mu', \alpha) = \<\mu \wedge \mu', \alpha\>|_{z=0}.
\]

In cohomology, we obtain the cocycle for divergence-free vector fields by plugging in $\alpha = \d z_1 \wedge \d z_2$. 
This gives the cocycle on $\Vect_0(\CC^2)$ 
\[
(f_i \del_{z_i}, g_j \del_{z_j}) \mapsto (f_1 g_2 - f_2 g_1)(z_1=z_2=0) .
\]
This is the cocycle defining \eqref{eqn:centralvect}, as desired. 
\end{proof}.

This proposition implies that the global symmetry algebra of the non-minimal twist of 11-dimensional supergravity is the Lie algebra $\cO(\CC^2)$. 
We will see that this is compatible with the calculation of the non-minimal twist of the full BV theory. 

\subsection{The non-minimal twist of the 11-dimensional theory}

Now, we turn to deducing the action function of the non-minimal twist and hence the proof of Theorem \ref{thm:nonmin}. 
We will show that the eleven-dimensional theory on $\CC^5 \times \RR$ placed in the background where the $(1,0)$ component of $\gamma$ takes the value $\gamma_{nm}$ \eqref{eqn:gammanm} is equivalent to a theory with a purely Chern--Simons-like action functional that we referred to in the introduction to this section. 

Poisson Chern--Siimons theory is defined on any manifold of the form
\[
Z \times M
\]
where $Z$ is a hyper K\"ahler surface and $M$ is a smooth manifold of real dimension seven. 
The fundamental field of the theory is  
\[
\alpha \in \Pi \Omega^{0,\bu}(Z) \; \Hat{\otimes} \; \Omega^\bu(M)  .
\]
Just in our original 11-dimensional theory, this theory is also only $\ZZ/2$ graded. 

The holomorphic symplectic form $\omega_Z^{2,0}$ on $Z$ induces a Poisson bracket define on all Dolbeault forms $\Omega^{0,\bu}(Z)$ which we denote by $\{-,-\}_{pb}$. 
In local Darboux coordinates $(z_1,z_2)$, this bracket reads
\[
\{\alpha^I (z,\zbar) \d \zbar_I , \alpha'^J (z,\zbar) \d \zbar_J\}_{pb} = (\partial_{z_1} \alpha^I \partial_{z_2} \alpha^J \pm \partial_{z_2} \alpha^I \partial_{z_1} \alpha^J) \d \zbar_I \wedge \d \zbar_J . 
\]
The action functional of Poisson Chern--Simons theory is 
\beqn\label{eqn:pcsaction}
    \frac12 \int_{Z \times M} (\alpha \wedge \d\alpha) \wedge \omega^{2,0}_Z  + \frac16 \int_{Z\times M} \alpha \wedge \{\alpha, \alpha\}_{pb} \wedge \omega^{2,0}_Z
\eeqn
where $\{-,-\}$ is the Poisson bracket induced from the symplectic form $\omega_Z$ on $Z$. 

For simplicity, we will work only on flat space $\CC^5 \times \RR = \CC^2_z \times (\CC^3_w \times \RR)$, where we view $Z = \CC^2_z$ as a hyper K\"ahler manifold with its standard holomorphic symplectic form $\omega^{2,0} = \d^2 z$.

We will decompose the fields according to these coordinates. 
For example, we decompose the $\mu$-field as $\mu = \mu_z + \mu_w$ where
\begin{align*}
\mu_z  &\in \PV^{1,\bu}(\CC^2_z) \otimes \PV^{0,\bu}(\CC^3_w) \otimes \Omega^\bu(\RR) \\
\mu_w & \in \PV^{0,\bu}(\CC^2_z) \otimes \PV^{1,\bu}(\CC^3_w) \otimes \Omega^\bu(\RR)  .
\end{align*}
and similarly $\gamma = \gamma_z + \gamma_w$. 
We will also use the notation $\del^z$ for the holomorphic de Rham differential along $\CC_z^2$ and similarly $\del^w$ for the holomorphic de Rham differential along $\CC^3_w$. 

To twist, we expand near the background where the field $\gamma_z$ takes value $\gamma_{nm}$ as in \eqref{eqn:gammanm}. 
This will generate new kinetic and interacting terms. 
%which we can extract by inserting a formal parameter $\delta$ and expressing the action functional in terms of the deformed field $\Tilde{\gamma} = \gamma + \delta \gamma^{1,0}$.

There are two types of interactions in the original theory.
The first is
\begin{equation}\label{eqn:int1}
  \frac12 \int_{\CC^2 \times \CC^3 \times \RR} \frac{1}{1-\nu} \left(\del \gamma \vee \mu^2 \right) \wedge (\d^2 z \wedge \d^3 w)
\end{equation}
and the second is
\begin{equation} \label{eqn:int2}
  \frac16\int_{\CC^2 \times \CC^3 \times \RR} \gamma \partial \gamma \partial \gamma .
\end{equation}

%We can integrate Equation (\ref{eqn:int1}) by parts to put it in the form $\frac12 \int_{X \times Z \times L} \left[(\partial \gamma) \vee (\mu \wedge \mu) \right]$ where $\mu \wedge \mu$ is the wedge product of polvector fields.\brian{there might be some factors I'm being sloppy with here}
Expanding \eqref{eqn:int1} around the background where $\gamma$ takes value $\gamma_{nm}$, we obtain,
%\[
%  \frac12 \int \frac{1}{1-\nu} \left(\del \gamma \vee \mu^2 \right) \wedge (\omega_Z \wedge \Omega_W) + \frac{\delta}{2} \int \frac{1}{1-\nu} \left(\omega_Z \vee \mu^2 \right) \wedge (\omega_Z \wedge \Omega_W) .
%\]
%Here, we have used the equation of motion $\partial \gamma^{1,0} = \Omega_Z$.
%It will be convenient to further expand this into the components $\mu_W, \mu_Y, \gamma_W, \gamma_Y$:
\begin{multline}
 \int \frac{1}{1-\nu} \left(\frac12 \del^w \gamma_w \vee \mu_w^2  + \del^z \gamma_w \vee \mu_w \mu_z + \del^w \gamma_z \vee \mu_w\mu_z + \frac12 \del^z \gamma_z \vee \mu_z^2 \right) \wedge (\d^2 z \wedge \d^3 w) 
 \\
  + \frac{1}{2} \int \frac{1}{1-\nu} \left(\d^2 z \vee \mu_z^2 \right) \wedge (\d^2 z \wedge \d^3 w) .
  \label{eqn:delta1}
\end{multline}

We similarly expand (\ref{eqn:int2}),
%\[
%  \frac16 \int \gamma \partial \gamma \partial \gamma + \frac{\delta}{2} \int \left(\gamma \partial \gamma\right) \wedge \omega_Z .
%\]
%Notice that there are no $\delta^2$ terms since $\partial \gamma^{1,0} \partial \gamma^{1,0} = 0$.
%Again, we further expand this into holomorphic components along $X,Z$:
\beqn
\frac16 \int \left(\gamma_w \partial^z \gamma_w \partial^z \gamma_w +\gamma_w \partial^w \gamma_w \partial^z \gamma_z +  \gamma_w \partial^w \gamma_z \partial^w \gamma_z \right) + \frac{1}{2} \int \left(\gamma_w \partial^w \gamma_w \right) \wedge \d^2 z
\label{eqn:delta2}
\eeqn

The new terms in the non-minimally twisted linearized BRST differential arise from the quadratic terms in the action in Equations \eqref{eqn:delta1} and \eqref{eqn:delta2}:
\begin{equation}\label{eqn:newterms}
  \frac{1}{2} \int (\d^2 z \vee \mu_z^2) \wedge (\d^2 z \wedge \d^3 w) + \frac{1}{2} \int \left(\gamma_w \wedge \partial^w \gamma_w \right) \wedge \d^2 z .
\end{equation}
The non-minimally twisted linear BRST complex thus takes the form
\[
  \begin{tikzcd}
  & \PV^{1,\bu}_Z \hotimes \PV^{0,\bu}_W \ar[dr, "\div^z"] \ar[dashed, rounded corners, to path={ -- ([yshift=-2ex]\tikztostart.west) |- ([xshift=-1.5ex]\tikztotarget.west) -- (\tikztotarget)}, dddddr]\\
  & & \PV^{0,\bu}_Z \hotimes \PV^{0,\bu}_W \\
 & \PV^{0,\bu}_Z \hotimes \PV^{1,\bu}_W \ar[ur, "\div^w"'] & \\
\;_{\cong}  & & \Omega^{0,\bu}_Z \hotimes \Omega^{1,\bu}_W \ar[ul, dashed, bend left = 10, "\Omega^{-1}_W \partial^w"]\\
 & \Omega^{0,\bu}_Z \hotimes \Omega^{0,\bu}_W \ar[ur, "\partial^w"] \ar[dr,"\partial^z"'] \\
  & & \Omega^{1,\bu}_Z \hotimes \Omega^{0,\bu}_W
  %\ar[uuuuul, start anchor =  {[yshift = 0ex, xshift=0ex]}, end anchor = {[yshift=1.0ex, xshift=-5ex]}, bend left = 90, dotted] .
  \end{tikzcd}
\]
Here, we write $Z = \CC^2_z$ and $X = \CC^3_w$ for notational simplicity. 

Here, the dashed arrow along the outside of the diagram corresponds to the BV antibracket with the first term in (\ref{eqn:newterms}).
It is given by the isomorphism 
\[
\Omega^{1,\bu}_Z \hotimes \Omega^{0,\bu}_W \xto{\omega^{2,0}_Z \otimes \id} \PV^{1,\bu}_Z \hotimes \PV^{0,\bu}_W
\]
induced holomorphic symplectic form on $Z$. 
The other dashed arrow corresponds to the BV antibracket with the second term in (\ref{eqn:newterms}).
It is given by the composition
\[
\Omega^{0,\bu}_Z \hotimes \Omega^{1,\bu}_W \xto{\id \otimes \del^w} \Omega^{0,\bu}_Z \hotimes \Omega^{2,\bu}_W \xto{\id \otimes \Omega_W} \PV^{0,\bu}_Z \hotimes \PV^{1,\bu}_W
\]
given by applying the holomorphic de Rham operator along $X$ followed by contracting with the inverse holomorphic volume form along $X$. 
%\brian{introduce PCS}
%Explicitly, if $f,g$ are holomorphic functions on $\CC^2$ then
%\[
%\{f(z_1,z_2) , g(z_1,z_2)\}_{pb} = \partial_{z_i} f \partial_{z_j} g - \partial_{z_j} f \partial_{z_i} g .
%\]

We replace this linear BRST complex, up to quasi-isomorphism, with a smaller BRST complex. 
Consider the complex
\beqn
\Omega^{0,\bu}_Z \hotimes \Omega^{\bu,\bu}_X \hotimes \Omega^\bu_L = \oplus_{k =0}^3 \Omega^{0,\bu}_Z \hotimes \Omega^{k,\bu}_X \hotimes \Omega^\bu_L 
%[-k] 
\eeqn
which is equipped with the differential $\dbar^z + \dbar^w + \del^w + \d_{\RR}$. 
Write $\alpha = \alpha^0 + \cdots + \alpha^3$ for a field in this complex, using the decomposition on the right hand side. 

There is a map of linear BRST complexes from this one to the original one defined by the following equations 
\begin{multline}
\mu_z = (\del_{z_1} \wedge \del_{z_2}) \vee \del^z \alpha^0, \quad \mu_w = (\del_{w_1} \wedge \del_{w_2} \wedge \del_{w_3}) \vee \alpha^2, \quad \nu = \til{\alpha}^3 \\
\beta = \alpha^0 , \quad \gamma_w = \alpha^1 , \quad \gamma_z = 0 .
\label{eqn:g2map}
\end{multline}
In the above equation we have introduced the notation $\til{\alpha}^3 = \Omega_X^{-1} \vee \alpha^3$. 

The restriction of the kinetic terms $\int \gamma (\dbar + \d_{\RR}) \mu + \beta (\dbar + \d_{\RR}) \nu$ along \eqref{eqn:g2map} is
\beqn\label{eqn:kin1}
\int \sum_{k=0}^3 \alpha^k (\dbar + \d_{\RR}) \alpha^{3-k} 
\eeqn
The restriction of the kinetic term $\int \beta \div \mu$ along \eqref{eqn:g2map} is
\beqn\label{eqn:kin2}
\int \alpha^0 \del^w \alpha^2 . 
\eeqn
Finally, the restriction of the kinetic term $\gamma \del^w \gamma$ along \eqref{eqn:g2map} is 
\beqn\label{eqn:kin3} 
\int \frac12 \alpha^1 \del^w \alpha^1 . 
\eeqn
Together, \eqref{eqn:kin1}--\eqref{eqn:kin3} give the kinetic term in Poisson Chern--Simons theory. 

This shows that \eqref{eqn:g2map} is a map of linear BRST complexes.
Applying the obvious contracting homotopy, we see that this map is a quasi-isomorphism.
We will show that the full non-linear map intertwines the action functionals up to cohomologically exact terms, and hence defines an equivalence of BV theories.

%
%Restricting the action \eqref{eqn:delta1} along the map \eqref{eqn:g2map} we obtain
%\begin{multline}
%\frac12 \alpha^1 (\alpha^2)^2 \del^X \til{\alpha}^3 + \frac12 \del^X \alpha^1 (\alpha^2)^2 
%\end{multline}
%
%\begin{multline}
%\int \frac{\omega_Z}{1-\Omega^{-1}_X \alpha^3} \left(\frac12 \del^X \alpha^1  [ \alpha^2 \vee (\alpha^2 \vee \Omega_X^{-1})] + \alpha^2 \del^Z \alpha^1 \del^Z \alpha^0\right) \\
%+ \frac16 \int \frac{\omega_Z}{1-\Omega^{-1}_X \alpha^3} \alpha^1 \partial^Z \alpha^1 \partial^Z \alpha^1 \\ + \frac{\delta}{2} \int \frac{\omega_Z}{1-\Omega^{-1}_X \alpha^3} \alpha^3 \partial^Z \alpha^0 \partial^Z \alpha^0 + \frac{\delta}{2} \int \frac{\omega_Z}{1-\Omega^{-1}_X \alpha^3} \partial^X \alpha^1 (\alpha^1 \Omega^{-1}_X \alpha^3) .
%\end{multline}
%
%More invariantly, we can write the total BV action as
%\[
%\Tilde{I} = I_{CS} + I' 
%\]
%where $I_{CS} = \frac16 \int \alpha\{\alpha,\alpha\}$ and 
%\beqn
%I' = \frac16 \int \frac{\omega_Z}{1-\Omega^{-1}_X \alpha} \partial^X \alpha [\alpha \vee (\alpha \vee \Omega_X^{-1})] + \frac16 \int \frac{\omega_Z}{1-\Omega^{-1}_X \alpha} [\alpha \vee (\alpha \vee \Omega^{-1}_Z)] \{\alpha,\alpha\} .
%\label{eqn:Iprime}
%\eeqn
%
%Consider the odd functional
%\[
%K = \frac16 \int \frac{\omega_Z}{1-\Omega^{-1}_X \alpha} \alpha \wedge [\alpha \vee (\alpha \vee \Omega_X^{-1})] .
%\]
%
%\begin{lem}
%$Q K + \{I_{CS}, K\} = I'$ .
%\end{lem}
%
%The linear differential applied to $K$ is 
%\[
%\frac16 \int \frac{\omega_Z}{1-\Omega^{-1}_X \alpha} \partial^X \alpha \wedge [\alpha \vee (\alpha \vee \Omega_X^{-1})] \omega_Z .
%\]
%This is the first term in \eqref{eqn:Iprime}. 
%
%Next, we compute
%\[
%\{I_{CS}, K\} = \frac16 \int \frac{\omega_Z}{1-\Omega^{-1}_X \alpha} [\alpha \vee (\alpha \vee \Omega^{-1}_X)] \{\alpha,\alpha\} 
%\]
%which is the second term in \eqref{eqn:Iprime}. 


\end{document}

\appendix 

\section{An alternative description of the 11-dimensional theory} 
In \S \ref{s:dfn} we have described a family of BV theories on $X \times L$ where $X$ is an odd-dimensional Calabi--Yau manifold and $L$ is an odd-dimensional smooth manifold. 
The space of fields \eqref{eq:sympfields} is equipped with an odd symplectic pairing. 


There is a related, alternative description of this formal moduli space which is almost equivalent. 
Consider the resolution of holomorphic closed two-forms 
\beqn\label{eqn:twoform}
\Omega_{cl}^{2,hol}(X) \simeq \Omega^{2,\bu}(X) \xto{\del} \Omega^{3,\bu}(X)[-1] \to \cdots 
\eeqn
where we have left the $\dbar$-differential implicit, as always. 

There is a map from the two-term complex 
\[
\Omega^{0,\bu}(X)[1] \to \Omega^{1,\bu}(X)
\]
to the resolution \eqref{eqn:twoform} defined by the holomorphic de Rham operator $\del$. 
This map is almost a quasi-isomorphism of sheaves; it differs by copy of constant functions $\CC[1]$ in cohomological degree $-1$. 

Similarly, we can tensor with the dg algebra of de Rham forms on $L$ to obtain a map of $\ZZ/2$-graded cochain complexes
\[
\bigg(\Pi \Omega^{0,\bu}(X;L) \xto{\del} \Omega^{1,\bu}(X;L) \bigg) \xto{\del} \bigg(\Omega^{2,\bu}(X;L) \xto{\del} \Pi \Omega^{3,\bu}(X;L) \to \cdots \bigg) .
\]
Again, this map is a quasi-isomorphism up to a copy of $\Pi \CC$.
The left-hand side contains the fields $(\beta, \gamma)$. 
This map has the effect of sending $\beta \mapsto 0$ and $\gamma \mapsto \partial \gamma \in \Omega^{2,\bu}(X;L)$. 

The basic idea is to replace the complex where $(\beta,\gamma)$ live by this resolution of closed two-forms. 
The new space of fields is 
\begin{equation}
  \label{eq:poissfields} 
  \begin{tikzcd}[row sep = 1 ex]
     - & + & - & + \\ \hline
     \PV^{1,\bu}(X; L) \ar[r, "\div"] & \PV^{0,\bu}(X; L) \\
     & \Omega^{2,\bu}(X;L) \ar[r, "\del"] & \Omega^{3,\bu}(X;L) \ar[r] & \cdots 
\end{tikzcd}
\end{equation}

It may seem that this description of the fields of the 11-dimensional theory is not much different than the original formulation---at the level of the free theory they only differ by a copy of constant functions in odd cohomological degree. 
We'd like to point out two main differences:
\begin{itemize}
\item This new description does not have the structure of a BV theory in the usual sense; there is no odd symplectic pairing on the fields. 
Nevertheless, there is still an odd Poisson bracket acting on functionals of the fields which we will describe momentarily. 
For this reason, we will refer to the theory as a ``Poisson BV theory''. 
\item With the existence of the odd Poisson bracket one might be optimistic to formulate the CME for this Poisson BV theory.
However, there is no local interaction which is consistent with the local interaction present in the original BV theory. 
Nevertheless, the (shift of the) fields is equipped with an $L_\infty$ structure which is compatible with the odd Poisson bracket. 
\end{itemize}

\begin{prop}
The complex \eqref{eq:poissfields} is equipped with the structure of an interacting Poisson BV theory.
Let $\cG = \cG(X \times L)$ be the resulting $L_\infty$ algebra...
\end{prop}

Heuristically, we can write the action in the following non-local form
\[
\frac{1}{1-\nu} \mu^2 \eta + (\del^{-1} \eta) \eta^2 .
\]
\brian{are there more terms involving higher forms?}

The first few nonzero brackets are
\begin{align*}
[\mu]_1 & = \dbar \mu + \div \mu \\
[\eta]_1 & = \dbar \eta + \del \eta \\ 
[\mu_1,\mu_2]_2 & = \div (\mu_1 \wedge \mu_2) \\
[\mu, \eta]_2 & = \mu \vee \del \eta \\
[\eta_1,\eta_2]_2 & = \# \Omega^{-1} \vee (\eta_1 \wedge \eta_2)  \\
[\nu, \mu_1,\mu_2]_3 & = \div(\nu \mu_1 \mu_2) \\
[\nu, \mu,\eta]_3 & = \nu \mu \del \eta
\end{align*}
\brian{coefficients}
For $k \geq 2$ the general formula for the $k$-ary bracket is 
\begin{align*}
[\eta_1,\eta_2]_2 & = \# \Omega^{-1} \vee (\eta_1 \wedge \eta_2)  \\
[\nu_1,\nu_2, \ldots, \nu_{k-2}, \mu_1,\mu_2]_{k} & = \# \div(\nu_1 \cdots \nu_{k-2} \mu_1 \wedge \mu_2) \\ 
[\nu_1,\nu_2, \ldots, \nu_{k-2}, \mu,\eta]_{k} & = \# \nu_1 \cdots \nu_{k-2} (\mu \vee \del \eta) .
\end{align*}

\parsec[]
Consider the theory on $\CC^3 \times \RR$. 
There is a quasi-isomorphism of $L_\infty$ Lie algebras $\CC \simeq \cG (\CC^3 \times \RR)$. 

\parsec[]
Consider the theory on $\CC^5 \times \RR$. 

\begin{prop}
There is a quasi-isomorphism of super $L_\infty$ algebras 
\[
E(5,10) \xto{\simeq} \cG(\CC^5 \times \RR) .
\]
This lifts to map of super $L_\infty$ algebras $E(5,10) \to \cL(\CC^5 \times \RR)$ whose kernel is $\CC$.
\end{prop}


\section{BCOV stuff}

\subsection{Relation to BCOV theory}

\parsec[sec:pv] 
\ingmar{a section defining polyvector fields; not sure where this belong}
Equipped with this structure, the complex we have written maps in an obvious way into the complex of polyvector fields on~$X$. Recall that one defines
\deq{
  \PV^{i,j}(X) = \Omega^{0,j}(X, \wedge^i \T_X).
}
\ingmar{Wedges look super fucked up}
The complex $\PV^{\bu,\bu}$ is equipped with two natural differentials: the Dolbeault operator $\dbar$, of $(i,j)$-degree $(0,1)$, and the holomorphic divergence operator $\div$, which carries $(i,j)$-degree $(-1,0)$. Assigning total degree $- i + j$ to $\PV^{i,j}$ thus gives the total polyvector fields the structure of a bicomplex. We will use the shorthand notation $\PV^i = (\PV^{i,\bu}, \dbar )$ for the Dolbeault resolution of holomorphic $i$-polyvector fields. 

\subsubsection{}
It will be convenient for us to cast this dg Lie algebra in a slightly different way.
We construct a different model for the Lie algebra of divergence-free vector fields whose underlying cochain complex is the same as \eqref{eqn:cplx1}. 
The distinguishing feature is that this model is no longer a dg Lie algebra, but an $L_\infty$ algebra. 

The model we use is motivated by the topological string, specifically the description of the closed topological string in terms of Kodaira--Spencer theory \cite{BCOV}.
We recall the requisite background, but refer to \cite{CLbcov1,CLbcov2,CLtypeI} for more details. 

Suppose $X$ be a Calabi--Yau manifold of dimension $d$. 
Let $\PV^{k,\bu}(X)$ denote the Dolbeault complex of the holomorphic vector bundle $\wedge^k \T_X$; we will omit $X$ in this section and write $\PV^{k,\bu}$ for simplicity.
In particular, $\PV^{k,j}$ is the space of smooth sections of the bundle $\wedge^j \T_X^* \otimes \wedge^k \T_X$. 
With this notation, the dg Lie algebra in \eqref{eqn:cplx1} is $\PV^{1,\bu} \oplus \PV^{0,\bu}[-1]$ with differential $\dbar + \div$. 

Introduce a formal parameter $u$ of cohomological degree $+2$ and consider the complex 
\beqn\label{eqn:pv1}
\PV^{\bu,\bu} [[u]] [1]
\eeqn
with differential $\dbar + u \div$ which we will often denote just by $Q_{KS}$ and refer to as the linear BRST differential. 
With our conventions, notice that $u^l \PV^{k,j}$ sits in cohomological degree $2l +k + j - 1$. 

The Schouten--Nijenhuis bracket 
\[
[-,-] \colon \PV^{k,j} \times \PV^{p,q} \to \PV^{p+k-1,j+q} 
\]
extends $u$-linearly to endow \eqref{eqn:pv1} with the structure of a dg Lie algebra.
We refer to this as the Kodaira--Spencer, or strict, dg Lie algebra structure.
We refer to the further cohomological shift 
\[
\cE_{KS} = \PV^{\bu,\bu} [[u]] [2]
\]
as the space of fields of Kodaira--Spencer theory.

Notice that the complex resolving divergence-free vector fields \eqref{eqn:cplx1} sits inside $\cE_{KS} [-1]$ as a sub dg Lie algebra as $\PV^{1,\bu} \oplus u \PV^{0,\bu}[1]$. 

\subsubsection{}

\brian{bv bracket}

\subsubsection{}

Together with the linear BRST differential $Q_{KS}$, the bracket $\{-,-\}_{KS}$ induces the structure of a dg Lie algebra the cohomological shift of the space of local functionals $\oloc(\cE_{KS})[-2d+5]$. 

\begin{thm}[\cite{CLbcov1}]
The BCOV action 
\[
I_{BCOV} \in \oloc \left(\cE_{KS} \right) 
\]
satisfies the classical master equation 
\[
Q_{KS} I_{BCOV} + \frac12 \left\{I_{BCOV}, I_{BCOV}\right\}_{KS} = 0 .
\]
In other words, $I_{BCOV}$ determines a Maurer--Cartan element in the dg Lie algebra $\oloc(\cE_{KS})[-2d+5]$.
\end{thm}

This action induces the square-zero operator 
\[
\delta_{BCOV} = Q_{KS} + \{I_{BCOV}, -\} 
\]
acting on observables of the Kodaira--Spencer fields. 
In other words, it defines a non-linear BRST operator; in turn it
induces an $L_\infty$ structure $\{[-]_k\}_{k =1,2,3,\ldots}$ on the complex \eqref{eqn:pv1} whose linear operation is $[-]_1 = Q_{KS} = \dbar + u\div$. 

This $L_\infty$ structure is clearly not identical as the strict dg Lie algebra structure (it has operations of arbitrary high order). 
Nevertheless, it is equivalent to the strict dg Lie model: there is an $L_\infty$ automorphism which exchanges the two structures.

It is easiest to describe this automorphism at the level of observables.
Use the notation $\Sigma$ for a linear observable of the Kodaira--Spencer fields $\cE_{KS}$. 
Then, the Kodaira--Spencer dg Lie algebra structure induces the non-linear BRST operator $\delta_{KS}$ given by
\[
\delta_{KS} (\Sigma) = Q_{KS} \Sigma + \frac12 [\Sigma,\Sigma] .
\]

The non-linear change of coordinates relating the two structures is defined by
\[
\Psi \colon \Sigma \mapsto \left[u (e^{\Sigma/u} -1)\right]_+
\]
where $[-]_+$ projects onto the non-negative powers of $u$.  

\parsec[]

The complex resolving the (shift of) divergence-free vector fields 
\[
\left(\PV^{1,\bu}[1] \oplus u \PV^{0,\bu}[2] \, , \, \dbar + u \div\right) 
\]
is a subcomplex of $\cE_{KS}$.
If $d = \dim_\CC(X)=3$ then the shifted Poisson bivector $\Pi_{BCOV}$ restricts to a shifted Poisson bivector on this subcomplex. 
In particular, the action $I_{BCOV}$ restricts to a solution to the classical master equation for this subcomplex. 

If $d \ne 3$ then there is the following subcomplex $\til{\cE}_{KS}\subset \cE_{KS}$ defined by:
\beqn\label{eqn:tilks}
\begin{tikzcd}
\ul{-1} & \ul{0} & \cdots & \ul{d-4} & \ul{d-3} & \cdots    \\
\PV^{1,\bu} \ar[r] & u\PV^{0,\bu} & & & & &  \\
& & & \PV^{d-2, \bu} \ar[r] & u \PV^{d-3, \bu} \ar[r] & \cdots &   .
\end{tikzcd}
\eeqn
And one can check that the shifted Poisson bivector restricts to a shifted Poisson bivector on this subcomplex. 
Thus, we have the following.

\begin{prop}
\label{prop:tilbcov}
The BCOV action $I_{BCOV}$ restricts to a solution of the classical master equation on $\til{\cE}_{KS}$ that we denote by $\til{I}_{BCOV}$. 
\end{prop}

\subsubsection{}

Define the action by the group $\CC^\times$ on the complex $\til{\cE}_{KS}$ as follows.
Declare the first line in \eqref{eqn:tilks} is weight zero and the second line is weight $+1$. 
%Declare $\mu, \nu$ have $\CC^\times$ weight zero and $\beta,\gamma$ have $\CC^\times$ weight $+1$. 
The linear BRST operator is clearly weight zero and the shifted Poisson structure $\Pi_{KS}$ is of weight $+1$. 
Thus, the shifted Poisson bracket $\{-,-\}_{KS}$ is of $\CC^\times$ weight $+1$.

\begin{lem}
The restricted action $\til{I}_{BCOV}$ has $\CC^\times$ weight $-1$. 
In particular, the non-linear BRST operator 
\beqn\label{eqn:KSbrst}
\delta_{BCOV} = Q_{KS} + \{\til{I}_{BCOV},-\}_{KS}
\eeqn
is $\CC^\times$ weight zero.
\end{lem}
\begin{proof}
%The map $\Phi \colon \cE_{pot} \to \til{\cE}_{KS}$ preserves $\CC^\times$ weight, so it suffices to prove that $\til{I}_{BCOV}$ is weight $+1$. 
We will make use of two gradings on $\til{\cE}_{KS}$. 
The first is holomorphic polyvector field type, and the second is descendant degree.
The summand $u^l \PV^{k,\bu}$ is polyvector type $k$ and descendant degree $l$. 

We introduce notation in this proof that won't be used later on. 
Let $\alpha$ denote a super field living in the first line of \eqref{eqn:tilks} and let $\beta$ denote a super field living in the second line of \eqref{eqn:tilks}.
If a field of type $\alpha$ has polyvector type $k$ then it has descendant degree $1-k$, $k=0,1$. 
If a field of type $\beta$ has polyvector type $l$ then it has descendant degree $d-2-l$.

A homogenous polynomial degree term in the BCOV action $I_{BCOV}$ is a linear combination of functionals of the form
\[
 \alpha_1 \wedge \cdots \wedge \alpha_m \wedge \beta_1 \wedge \cdots \wedge \beta_n  .
\]
Let $k_i$ be the polyvector field type of $\alpha_i$ and let $l_j$ be the polyvector field type of $\beta_j$. 
In order for this expression to contribute nontrivially to the BCOV action two constraints must hold:
\begin{align*}
\sum_{i=1}^m k_i + \sum_{j=1}^n l_j & = d \\
\sum_{i=1}^m (1-k_i) + \sum_{j=1}^n (d-2-l_j) & = m+n-3 .
\end{align*}
The first constraint ensures that the integrand is of top polyvector degree. 
The second constraint is on the descendant degree, see \brian{ref above}. 

Simplifying these two equations, one finds the condition
\[
m+n-3 = m+(d-2)n - d 
\]
which implies $n=1$, as desired. 
\end{proof}

Consider the cochain complex $\left(\oloc(\til{\cE}_{KS}) \, , \, \delta_{KS} \right)$ of all local functionals equipped with the non-linear BRST differential \eqref{eqn:KSbrst}.  
The $\CC^\times$ weight zero subcomplex is 
\[
\big(\oloc(\til\cE_{KS})^{(0)} \, , \, \delta_{KS} \big) = \big(\oloc(\cE_0) \, , \, \delta_{KS} \big) 
\]
where $\cE_0 = \PV^{1,\bu}[1] \oplus u \PV^{0,\bu}$ is the subcomplex of the fields $\til\cE_{KS}$ comprising the first line of \eqref{eqn:tilks}.

\begin{prop}
\label{prop:Linfty}
The differential $\delta_{KS}$ acting on the weight zero fields induces a local $L_\infty$ algebra structure on the complex of vector bundles 
\[
\cE_0 [-1] = \PV^{1,\bu} \oplus \PV^{0,\bu}[-1]
\]
with $[-]_1 = \dbar + \div$. 
Furthermore, this $L_\infty$ algebra structure is equivalent to the strict dg Lie algebra structure on the resolution of divergence-free vector fields defined in \S \ref{sec:divfree}. 
\end{prop}

%Applying this to the $k$th exterior power of the holomorphic tangent bundle $V = \wedge^k \T_X$, we obtain a resolution of the sheaf of $k$-linear polyvector fields which we denote by $\PV^{k,\bu} (X) = \Omega^{0,\bu}(X, \wedge^k \T_X)$. 

%\subsubsection{}
%
%Consider the following cochain complex $\cE_{pot}$:
%\beqn\label{eqn:E}
%\begin{tikzcd}
%\ul{-1} & \ul{0} & \cdots & \ul{d-5} & \ul{d-4} &   \\
%\PV^{1,\bu} \ar[r,"\div"] & \PV^{0,\bu} & \cdots & & \\
%& & & \Omega^{0, \bu} \ar[r,"\del"] & \Omega^{1, \bu} 
%\end{tikzcd}
%\eeqn
%We will refer to the components of the fields using the notation
%\begin{align*}
%(\mu, \nu) & \in \Pi \PV^{1,\bu}(X)[1] \oplus \PV^{0,\bu}(X) \\
%(\beta, \gamma) & \in \Omega^{0,\bu}(X)[d-5] \oplus \Omega^{1,\bu}(X) [d-4].
%\end{align*}
%The expressions $\mu,\nu,\beta,\gamma$ are still super fields in the sense that they have components in all anti-holomorphic Dolbeault degree. 
%
%Let 
%\[
%\begin{array}{rccc}
% \colon \Omega_c^{0,\bu}(X) & \to & \CC [??] \\
%\alpha & \mapsto & \int_{X} \Omega \wedge \alpha 
%\end{array}
%\]
%be integration along the holomorphic volume form. 
%Define the pairing 
%\[
%\omega \colon \cE_{pot,c} \times \cE_{pot} \to \CC [??] 
%\]
%by the formula $ \gamma \vee \mu +  \beta \nu.$
%
%\brian{defines Poisson bivector $\Pi$.}
%
%\subsubsection{}

%Define the map of cochain complexes $\Phi \colon \cE_{pot} \to \til{\cE}_{KS}$
%by the formulas
%\[
%\Phi (\mu,\nu) = \mu + u \, \nu,\qquad \Phi(\beta, \gamma) = \div \left( \gamma \vee \Omega^{-1} \right)  .
%\]
%In the last expression we have used the isomorphism $\Omega^{-1} \colon \Omega^{1,\bu} \xto{\cong} \PV^{d-1,\bu}$
%so that $\div(\gamma \vee \Omega^{-1}) \in \PV^{d-2,\bu} \subset \til{\cE}_{KS}$.
%
%\begin{prop}
%\label{prop:pot}
%The map $\Phi \colon  \cE_{pot} \to \til{\cE}_{KS}$ intertwines the linear BRST differentials and the shifted Poisson bivectors $\Phi_*\Pi = \til{\Pi}_{KS}$. 
%In particular, it defines a map of dg Lie algebras
%\[
%\Phi^* \colon \big(\oloc(\cE_{KS}),\{-,-\}_c, Q_{KS} \big) \to \big(\oloc(\cE_{pot}),\{-,-\}, Q \big)
%\]
%\end{prop}
%
%\begin{proof}
%prove this
%\end{proof}
%
%As a corollary of Proposition \ref{prop:tilbcov} and Proposition \ref{prop:pot} we have the following result. 
%
%\begin{cor}
%The BCOV action $\til{I}_{BCOV}$ restricts along $\Phi$ to a solution of the classical master equation for $\cE_{pot}$:
%\begin{align*}
%I_{pot} & = \Phi^* \til{I}_{BCOV} \\
%Q I_{pot} + \frac12 \{I_{pot},I_{pot}\} & = 0 .
%\end{align*} 
%\end{cor} 


\end{document}
