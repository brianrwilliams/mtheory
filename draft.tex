\documentclass[11pt]{amsart}

\usepackage{macros-mtheory,amsaddr}

\addbibresource{cfs.bib}

%\linespread{1.2} %for editing
%\usepackage{mathpazo}

\begin{document}

\title{Twisted 11-dimensional supergravity, I}
\author{Surya Raghavendran}
\address{Perimeter Institute}
\email{??}
\author{Ingmar Saberi}
\address{Ludwig-Maximilians-Universit\"at M\"unchen \\ Fakult\"at f\"ur Physik \\ Theresienstra\ss{}e 37 \\ 80333 M\"unchen \\ Deutschland}
\email{i.saberi@physik.uni-muenchen.de}
\author{Brian R. Williams}
\address{School of Mathematics, University of Edinburgh, Edinburgh, UK}
\email{brian.williams@ed.ac.uk}
\begin{abstract}
to do
\end{abstract}
\maketitle

\section{The 11-dimensional theory} 

In this section we define the central theory of study, within the Batalin--Vilkovisky formalism.
The theory will be defined on any eleven-dimensional manifold of the form $X \times L$ where $X$ is a Calabi--Yau five-fold and $L$ is a smooth oriented one-manifold.

\subsection{Divergence-free vector fields} 

\subsubsection{}
We set up some notations and conventions in the context of complex geometry. 
Let $V$ be a holomorphic vector bundle on a complex manifold $X$. 
If $j$ is an integer, we let $\Omega^{0,j}(X, V)$ denote the space of anti-holomorphic Dolbeault forms of type $j$ on with values in $V$.
The $\dbar$ operator for $V$ is $\dbar \colon \Omega^{0,j}(X, V)\to \Omega^{0,j+1}(X)$ defines the Dolbeault complex of $V$
\[
  \Omega^{0,\bu}(X, V) = \left(\Omega^{0,j}(X, V)[-j] \; , \; \dbar\right) .
\]
This is a free resolution for the sheaf of holomorphic sections of $V$.

Suppose $X$ is a Calabi--Yau manifold with holomorphic volume form $\Omega$.
The divergence $\div(\mu)$ of a holomorphic vector field $\mu$ is defined by the formula
\[
\div (\mu) \wedge \Omega = L_\mu (\Omega)
\]
where, on the right hand side, we mean the Lie derivative of $\Omega$ with respect to $\mu$.

Let $\T_X$ denote the holomorphic tangent bundle and consider its Dolbeault complex $\Omega^{0,\bu}(X , \T_X)$ resolving the sheaf of holomorphic vector fields. 
The divergence operator extends to the Dolbeault complex to yield a map of cochain complexes 
\[
\div \colon \Omega^{0,\bu}(X , \T_X) \to \Omega^{0,\bu}(X) .
\]

The kernel of this map resolves the sheaf of holomorphic divergence-free vector fields $\Vect_0^{hol}(X)$.
In fact, we can form the complex 
\beqn\label{eqn:cplx1}
\begin{tikzcd}
\ul{0} & \ul{1} \\
\Omega^{0,\bu}(X , \T_X) \ar[r, "\div"] & \Omega^{0,\bu}(X) .
\end{tikzcd}
\eeqn
The $\dbar$ operator, as always, is left implicit. 
By the holomorphic Poincar\'e lemma, the embedding of the sheaf $\Vect^{hol}_0(X)$ into the degree zero piece of this complex is a quasi-isomorphism. 

There is a direct way to extend the Lie bracket of vector fields to the complex \eqref{eqn:cplx1}. 
Denote by $\mu$ an element of $\Omega^{0,\bu}(X , \T_X)$ and $\nu$ an element of $\Omega^{0,\bu}(X)$ (for simplicity in notation, we will not expand the anti-holomorphic dependence). 
The Lie bracket defined by the formulas
\begin{align*}
[\mu, \mu'] & = L_\mu \mu' \\
[\mu, \nu] & = L_\mu \nu 
\end{align*}
is compatible with $\div$ and endows \eqref{eqn:cplx1} with the structure of a sheaf of dg Lie algebras.\footnote{This is a local dg Lie algebra, see \cite[??]{CG2}.}

\subsubsection{}
It will be convenient for us to cast this dg Lie algebra in a slightly different way.
We construct a different model for the Lie algebra of divergence-free vector fields whose underlying cochain complex is the same as \eqref{eqn:cplx1}. 
The distinguishing feature is that this model is no longer a dg Lie algebra, but an $L_\infty$ algebra. 

The model we use is motivated by the topological string, specifically the description of the closed topological string in terms of Kodaira--Spencer theory \cite{BCOV}.
We recall the requisite background, but refer to \cite{CLbcov1,CLbcov2,CLtypeI} for more details. 

Let $X$ be a Calabi--Yau manifold of dimension $d$. 
Let $\PV^{k,\bu}(X)$ denote the Dolbeault complex of the holomorphic vector bundle $\wedge^k \T_X$. 
In particular, $\PV^{k,j}(X)$ is the space of smooth sections of the bundle $\wedge^j \T_X^* \otimes \wedge^k \T_X$. 
With this notation, the dg Lie algebra in \eqref{eqn:cplx1} is $\PV^{1,\bu}(X) \oplus \PV^{0,\bu}(X)[-1]$ with differential $\dbar + \div$. 

Introduce a formal parameter $u$ of cohomological degree $+2$ and consider the complex 
\beqn\label{eqn:pv1}
\PV^{\bu,\bu} (X) [[u]] [1]
\eeqn
with differential $\dbar + u \div$ which we will often denote just by $Q_{KS}$ and refer to as the linear BRST differential. 
With our conventions, notice that $u^l \PV^{k,j}(X)$ sits in cohomological degree $2l +k + j - 1$. 

The Schouten--Nijenhuis bracket 
\[
[-,-]_{NS} \colon \PV^{k,j} (X) \times \PV^{p,q} (X) \to \PV^{p+k-1,j+q} (X) 
\]
extends $u$-linearly to endow \eqref{eqn:pv1} with the structure of a dg Lie algebra.
We refer to this as the Kodaira--Spencer, or strict, dg Lie algebra structure.
We refer to the further cohomological shift 
\[
\cE_{KS} = \PV^{\bu,\bu} (X) [[u]] [2]
\]
as the space of fields of Kodaira--Spencer theory.

Notice that the complex resolving divergence-free vector fields \eqref{eqn:cplx1} sits inside $\cE_{KS} [-1]$ as a sub dg Lie algebra as $\PV^{1,\bu}(X) \oplus u \PV^{0,\bu}(X)[1]$. 

\subsubsection{}

\brian{bv bracket}

\subsubsection{}

Together with the linear BRST differential $Q_{KS}$, the bracket $\{-,-\}_c$ induces the structure of a dg Lie algebra the cohomological shift of the space of local functionals $\oloc(\cE_{KS})[-2d+5]$. 

\begin{thm}[\cite{CLbcov1}]
The BCOV action 
\[
I_{BCOV} \in \oloc \left(\cE_{KS} \right) 
\]
satisfies the classical master equation 
\[
Q_{KS} I_{BCOV} + \frac12 \left\{I_{BCOV}, I_{BCOV}\right\}_c = 0 .
\]
In other words, $I_{BCOV}$ determines a Maurer--Cartan element in the dg Lie algebra $\oloc(\cE)[-2d+5]$.
\end{thm}

This action induces the square-zero operator 
\[
\delta_{BCOV} = Q_{KS} + \{I_{BCOV}, -\} 
\]
acting on observables of the Kodaira--Spencer fields. 
In other words, it defines a non-linear BRST operator; in turn it
induces an $L_\infty$ structure $\{[-]_k\}_{k =1,2,3,\ldots}$ on the complex \eqref{eqn:pv1} whose linear operation is $[-]_1 = Q = \dbar + u\div$. 

This $L_\infty$ structure is clearly not identical as the strict dg Lie algebra structure (it has operations of arbitrary high order). 
Nevertheless, it is equivalent to the strict dg Lie model: there is an $L_\infty$ automorphism which exchanges the two structures.

It is easiest to describe this automorphism at the level of observables.
Use the notation $\Sigma$ for a linear observable of the Kodaira--Spencer fields $\cE_{KS}$. 
Then, the Kodaira--Spencer dg Lie algebra structure induces the non-linear BRST operator $\delta_{KS}$ given by
\[
\delta_{KS} (\Sigma) = Q_{KS} \Sigma + \frac12 [\Sigma,\Sigma]_{NS} .
\]

The non-linear change of coordinates relating the two structures is defined by
\[
\Psi \colon \Sigma \mapsto \left[u (e^{\Sigma/u} -1)\right]_+
\]
where $[-]_+$ projects onto the non-negative powers of $u$.  

\subsubsection{}

The complex resolving the (shift of) divergence-free vector fields 
\[
\left(\PV^{1,\bu}[1] \oplus u \PV^{0,\bu}[2] \, , \, \dbar + u \div\right) 
\]
is a subcomplex of $\cE_{KS}$.
If $d = \dim_\CC(X)=3$ then the shifted Poisson bivector $\Pi_{BCOV}$ restricts to a shifted Poisson bivector on this subcomplex. 
In particular, the action $I_{BCOV}$ restricts to a solution to the classical master equation for this subcomplex. 

If $d \ne 3$ then there is the following subcomplex $\til{\cE}_{KS}\subset \cE_{\rm KS}$ defined by:
\[
\begin{tikzcd}
\ul{-1} & \ul{0} & \cdots & \ul{d-4} & \ul{d-3} & \cdots    \\
\PV^{1,\bu} \ar[r] & u\PV^{0,\bu} & & & & &  \\
& & & \PV^{d-2, \bu} \ar[r] & u \PV^{d-3, \bu} \ar[r] & \cdots &   .
\end{tikzcd}
\]
And one can check that the shifted Poisson bivector restricts to a shifted Poisson bivector on this subcomplex. 
Thus, we have the following.

\begin{prop}
\label{prop:tilbcov}
The BCOV action $I_{BCOV}$ restricts to a solution of the classical master equation on $\til{\cE}_{KS}$ that we denote by $\til{I}_{BCOV}$. 
\end{prop}
%Applying this to the $k$th exterior power of the holomorphic tangent bundle $V = \wedge^k \T_X$, we obtain a resolution of the sheaf of $k$-linear polyvector fields which we denote by $\PV^{k,\bu} (X) = \Omega^{0,\bu}(X, \wedge^k \T_X)$. 

\subsubsection{}

Consider the following cochain complex $\cE$:
\beqn\label{eqn:E}
\begin{tikzcd}
\ul{-1} & \ul{0} & \cdots & \ul{d-5} & \ul{d-4} &   \\
\PV^{1,\bu} \ar[r,"\div"] & \PV^{0,\bu} & \cdots & & \\
& & & \Omega^{0, \bu} \ar[r,"\del"] & \Omega^{1, \bu} 
\end{tikzcd}
\eeqn
We will write the fields using the notation
\begin{align*}
(\mu, \nu) & \in \Pi \PV^{1,\bu}(X)[1] \oplus \PV^{0,\bu}(X) \\
(\beta, \gamma) & \in \Omega^{0,\bu}(X)[d-5] \oplus \Omega^{1,\bu}(X) [d-4].
\end{align*}

Let 
\[
\begin{array}{rccc}
\int^\Omega \colon \Omega^{0,\bu}(X) & \to & \CC [6] \\
\alpha & \mapsto & \int_{X} \Omega \wedge \alpha 
\end{array}
\]
be integration along the holomorphic volume form. 

Define the pairing 
\[
\omega \colon \cE_c \times \cE \to \cE [??] 
\]
by the formula $\int^\Omega \gamma \vee \mu + \int^\Omega \beta \nu.$

\brian{defines Poisson bivector $\Pi$.}

\subsubsection{}

Define the map of cochain complexes
\[
\Phi \colon \cE \to \til{\cE}_{KS}
\]
by the formulas
\[
\Phi (\mu,\nu) = \mu + u \nu,\qquad \Phi(\beta, \gamma) = \div \left( \gamma \vee \Omega^{-1} \right)  .
\]
In the last expression we have used the isomorphism
\[
(-) \vee \Omega^{-1} \colon \Omega^{1,\bu} \xto{\cong} \PV^{d-1,\bu} 
\]
so that $\div(\gamma \vee \Omega^{-1}) \in \PV^{d-2,\bu} \subset \til{\cE}$.

\begin{prop}
\label{prop:pot}
The map $\Phi \colon  \cE \to \til{\cE}_{KS}$ intertwines the linear BRST differentials and the shifted Poisson bivectors $\Phi_*\Pi = \til{\Pi}_{KS}$. 
In particular, it defines a map of dg Lie algebras
\[
\Phi^* \colon \big(\oloc(\cE_{KS}),\{-,-\}_c, Q_{KS} \big) \to \big(\oloc(\cE),\{-,-\}, Q \big)
\]
\end{prop}

\begin{proof}
prove this
\end{proof}

As a corollary of Proposition \ref{prop:tilbcov} and Proposition \ref{prop:pot} we have the following result. 

\begin{cor}
The BCOV action $\til{I}_{BCOV}$ restricts along $\Phi$ to a solution of the classical master equation for $\cE$:
\begin{align*}
I & = \Phi^* \til{I}_{BCOV} \\
Q I + \frac12 \{I,I\} & = 0 .
\end{align*} 
\end{cor} 

\subsubsection{}



\subsection{A family of deformations} 

%check CME

\subsection{Equations of motion}

%whatever we can say about the formal moduli space

\subsection{Boundary conditions}

%foreshadow the Horava--Witten picture

\subsection{One-loop quantization} 

%cite result on thf quantization.

\section{Relationship to physical supergravity}

\subsection{Component fields}

%Can we try to roughly say where each of our twisted fields come from in the untwisted theory? I imagine this section being pretty chill and mostly to orient physicists. 

\subsection{Residual supersymmetry} 

%m2brane

\subsection{Superconformal algebras}

%6d suco 
%what about 3d N=8 suco?

\section{Dimensional reduction and compactifications} 

\subsection{Relationship to the topological string}

%conjecture for minimal twist of IIA
%check the SU(4) twist

\subsection{The theory on a three-fold}

%count hypers and vectors in 5d N=1 sugra relate to Kevin and my work with Chris

\section{The twisted $\Omega$-backround} 

\section{The non-minimal twist} 


\end{document}