\documentclass[11pt]{amsart}

\usepackage{macros-mtheory,amsaddr}

\addbibresource{cfs.bib}

%\linespread{1.2} %for editing
%\usepackage{mathpazo}

\begin{document}

\title{Twisted 11-dimensional supergravity, I}
\author{Surya Raghavendran}
\address{Perimeter Institute for Theoretical Physics \\ 31 Caroline Street North \\ 
Waterloo, Ontario N2L 2Y5\\ Canada}
\email{??}
\author{Ingmar Saberi}
\address{Ludwig-Maximilians-Universit\"at M\"unchen \\ Fakult\"at f\"ur Physik \\ Theresienstra\ss{}e 37 \\ 80333 M\"unchen \\ Deutschland}
\email{i.saberi@physik.uni-muenchen.de}
\author{Brian R. Williams}
\address{School of Mathematics \\ University of Edinburgh \\ Edinburgh EH9 3FD \\ Scotland}
\email{brian.williams@ed.ac.uk}
\begin{abstract}
to do
\end{abstract}
\maketitle

\newpage 

\section{A family of moduli spaces} 
\label{s:dfn}

In this section we define the central theory of study, within the Batalin--Vilkovisky formalism.
The theory will be defined on any eleven-dimensional manifold of the form $X \times L$ where $X$ is a Calabi--Yau five-fold and $L$ is a smooth oriented one-manifold.

\subsection{Divergence-free vector fields} 

\subsubsection{}
\label{sec:divfree}
We set up some notations and conventions in the context of complex geometry. 
Let $V$ be a holomorphic vector bundle on a complex manifold $X$. 
If $j$ is an integer, we let $\Omega^{0,j}(X, V)$ denote the space of anti-holomorphic Dolbeault forms of type $j$ on with values in $V$.
The $\dbar$ operator for $V$ is $\dbar \colon \Omega^{0,j}(X, V)\to \Omega^{0,j+1}(X)$ defines the Dolbeault complex of $V$
\[
  \Omega^{0,\bu}(X, V) = \left(\Omega^{0,j}(X, V)[-j] \; , \; \dbar\right) .
\]
This is a free resolution for the sheaf of holomorphic sections of $V$.

Suppose $X$ is a Calabi--Yau manifold with holomorphic volume form $\Omega$.
The divergence $\div(\mu)$ of a holomorphic vector field $\mu$ is defined by the formula
\[
\div (\mu) \wedge \Omega = L_\mu (\Omega)
\]
where, on the right hand side, we mean the Lie derivative of $\Omega$ with respect to $\mu$.

Let $\T_X$ denote the holomorphic tangent bundle and consider its Dolbeault complex $\Omega^{0,\bu}(X , \T_X)$ resolving the sheaf of holomorphic vector fields. 
The divergence operator extends to the Dolbeault complex to yield a map of cochain complexes 
\[
\div \colon \Omega^{0,\bu}(X , \T_X) \to \Omega^{0,\bu}(X) .
\]

The kernel of this map resolves the sheaf of holomorphic divergence-free vector fields $\Vect_0^{hol}(X)$.
In fact, we can form the complex 
\beqn\label{eqn:cplx1}
\begin{tikzcd}
\ul{0} & \ul{1} \\
\Omega^{0,\bu}(X , \T_X) \ar[r, "\div"] & \Omega^{0,\bu}(X) .
\end{tikzcd}
\eeqn
The $\dbar$ operator, as always, is left implicit. 
By the holomorphic Poincar\'e lemma, the embedding of the sheaf $\Vect^{hol}_0(X)$ into the degree zero piece of this complex is a quasi-isomorphism. 

There is a direct way to extend the Lie bracket of vector fields to the complex \eqref{eqn:cplx1}. 
Denote by $\mu$ an element of $\Omega^{0,\bu}(X , \T_X)$ and $\nu$ an element of $\Omega^{0,\bu}(X)$ (for simplicity in notation, we will not expand the anti-holomorphic dependence). 
The Lie bracket defined by the formulas
\begin{align*}
[\mu, \mu'] & = L_\mu \mu' \\
[\mu, \nu] & = L_\mu \nu 
\end{align*}
is compatible with $\div$ and endows \eqref{eqn:cplx1} with the structure of a sheaf of dg Lie algebras.
We will refer to this sheaf by the symbol $\cL_0(X)$, or just $\cL_0$ if $X$ is understood. 

The sheaf $\cL_0$ has the structure of a {\em local} dg Lie algebra, see see \cite[??]{CG2}.
This means that as a graded sheaf, $\cL_0$ is the smooth sections of a graded vector bundle and the differential and Lie bracket are given by differential and bidifferential operators, respectively.


\parsec[sec:Linfty]

Recall that an $L_\infty$ algebra is a $\ZZ$-graded vector space $\cL$ together with the data of a square-zero, degree $+1$ derivation $\delta$ of the free commutative graded algebra $\Sym\left(\cL^\vee [-1] \right)$. 
The Chevalley--Eilenberg cochain complex is 
\[
\left(\Sym\left(\cL^\vee [-1] \right), \delta\right) .
\]
The Taylor components of $\delta$ define higher brackets $\{[-]_k\}_{k=1,2,\ldots}$ where $[-]_k \colon \cL^{\times k} \to \cL[2-k]$. 

An $L_\infty$ morphism $\Phi: \cL \rightsquigarrow \cL'$ is the same datum as a map of commutative dg algebras 
\deq{
  \Phi^*: \clie^\bu(\cL') \to \clie^\bu(\cL)
}
between their respective Lie algebra cochains. It follows from this that \emph{any} automorphism $\Phi$ of the free commutative algebra on $\cL^\vee[-1]$ defines a new model of the $L_\infty$ algebra $\cL$, for which the Chevalley--Eilenberg differential is obtained by conjugating $\delta_\cL$ by~$\Phi$, and where $\Phi$ itself defines the $L_\infty$ isomorphism.

$\cL_0$ is the sheaf~\eqref{eqn:cplx1} resolving divergence-free vector fields equipped with the dg Lie algebra structure constructed in the previous section.
We consider the following automorphism of~$\Sym(\cL_0^\vee[-1])$, defined by its action on generators:\ingmar{notation for operators vs fields}
\deq[eq:newbase]{
%    \Psi_\infty:  \nu &\mapsto -\frac{\nu}{1-\nu}, \quad \mu \mapsto -\mu \\
    \Psi_\infty: \nu \mapsto 1 - e^{-\nu}, \quad \mu \mapsto e^{-\nu} \mu.
}
This map defines a new model of the $L_\infty$ algebra with underlying graded vector space the same as \eqref{eqn:cplx1}, which we will call $\cL_\infty$. 
The formulas for the automorphism above clearly arise from maps of vector bundles and hence endow $\cL_\infty$ with the structure of a local $L_\infty$ algebra, meaning all operations are given by polydifferential operators.  

The notation refers to the fact that this new model has nonvanishing $L_\infty$ brackets of every order. 
It is this new model that we will use to define the eleven-dimensional theory of twisted supergravity (as well as the family of analogous formal moduli problems on products of odd Calabi--Yau manifolds with~$\R$). 


%The previous proposition characterizes the $L_\infty$ model for divergence-free vector fields that we will use to define the 11-dimensional theory of twisted supergravity. 
%Hereon, we denote by $\cL_0 = \cE_0[-1]$ this local $L_\infty$ algebra. 

%We can unpack Proposition \ref{prop:Linfty} to describe the $L_\infty$ structure on $\cL_0$ explicitly. 
We can describe the $L_\infty$ structure on our new model $\cL_\infty$ more explicitly.
Recall that we have two types of elements: $\mu \in \PV^{1,\bu}$ and $\nu \in \PV^{0,\bu}[-1]$. 
The first few nonzero brackets are
\begin{align*}
[\mu]_1 & = \dbar \mu + \div \mu \\
[\mu_1,\mu_2]_2 & = \div (\mu_1 \mu_2) \\
[\nu, \mu_1,\mu_2]_3 & = \div(\nu \mu_1 \mu_2) \\
[\nu_1,\nu_2, \mu_1,\mu_2]_4 & = \# \div(\nu_1 \nu_2 \mu_1 \mu_2).
\end{align*}
\brian{coefficients}
For $k \geq 2$ the general formula for the $k$-ary bracket is 
\[
[\nu_1,\nu_2, \ldots, \nu_{k-2}, \mu_1,\mu_2]_{k} = \# \div(\nu_1 \cdots \nu_k \mu_1 \mu_2) .
\]

%We describe the explicit $L_\infty$ automorphism $\Psi \colon (\cL_0)^{L_\infty} \rightsquigarrow (\cL_0)^{strict}$ intertwining the strict dg Lie structure on $\cL_0$ and this $L_\infty$ structure.
%The linear term $\Psi^{(1)} = \id$ is the identity map. 
%The higher terms $\Psi^{(n)}$ are defined by 
%\brian{someone check me}
%\begin{align*}
%\Psi^{(n)} (\nu_1,\ldots, \nu_{n-k},\mu_1,\ldots, \mu_k) & = \delta_{k=1} \nu_1 \cdots \nu_{n-1} \mu_1 . \\
%\end{align*}


\parsec
There is yet another $L_\infty$ model for divergence-free vector fields that we remark on here.
\ingmar{I want to write the other automorphism later, I think; otherwise we have to write the composition, since we presented it on bcov}


\subsection{Theories of BF type}

\parsec
Suppose that $\cL$ is an $L_\infty$ algebra with $L_\infty$ operations $\{[-]^\cL_k\}_{k=1,2,\ldots}$ and that $(A, \d_A)$ is a commutative dg algebra. 
The graded vector space $\cL \otimes A$ is equipped with the natural structure of an $L_\infty$ algebra with operations $\{[-]_k\}_{k=1,2,\ldots}$ defined by
\begin{align*}
[x \otimes a]_1 & = [x]^\cL_1 \otimes a + (-1)^{|x|} x \otimes \d_A a \\
[x_1 \otimes a_1, \ldots , x_k \otimes a_k]_k & = [x_1,\ldots,x_k]^\cL_k \otimes (a_1 \cdots a_k), \qquad k \geq 2 .
\end{align*}

We apply this construction, taking $\cL$ to be the sheaf resolving divergence-free holomorphic vector fields on a Calabi--Yau manifold $X$ equipped with either the strict dg Lie algebra structure $\cL_0(X)$ or its non-strict $L_\infty$ structure $\cL_\infty (X)$. 
The algebra $A$ will be the smooth de Rham complex $(\Omega^\bu(L) , \d_L)$ where $L$ is any smooth manifold. 

We thus obtain the structure of an dg Lie algebra on $\cL_0(X) \otimes \Omega^{\bu}(L)$ or an $L_\infty$ algebra $\cL_\infty(X) \otimes \Omega^\bu(L)$.
These define equivalent local $L_\infty$ algebras on the product manifold $X \times L$. 

\subsubsection{}

Associated to any local $L_\infty$ algebra is a classical field theory in the BV formalism that one refers to as BF theory.
We recall the construction. 
Let $\cL$ be any local $L_\infty$ algebra on some manifold $M$, it is the sheaf of sections of some graded vector bundle $L$. 
For a section $A \in \cL$, introduce the `higher curvature map' defined by the formula
\[
\mathsf{F}_A = [A]_1 + \frac12 [A,A]_2 + \frac{1}{3!} [A,A,A]_3 + \cdots .
\]

The fields of the associated BV theory are pairs
\[
  (A, B) \in \cL[1] \oplus \cL^{!}[-2] .
\]
Here $\cL^!$ denotes the sheaf of sections of the bundle $L^* \otimes {\rm Dens}$, where ${\rm Dens}$ is the bundle of densities. 
The shifted symplectic BV pairing is the obvious integration pairing between $A$ and $B$. 

The action functional reads $S_{\rm BF} = \int_M B \, \mathsf{F}_{A}$ which leads to the equations of motion $\mathsf{F}_{A} = 0$ and $\mathsf{D}_A B= 0$ where $\mathsf{D}_A$ is the higher covariant derivative along $A$. 

We thus obtain a theory in the BV formalism on the product manifold $X \times L$ associated to both local $L_\infty$ algebras $\cL_0(X) \otimes \Omega^{\bu}(L)$ and $\cL_\infty(X) \otimes \Omega^\bu(L)$.

\parsec
For concreteness, we spell out the fields of the theories we have constructed on $X \times L$.
In both cases, the space of fields equipped with the linear BRST operator is
\begin{equation}
  \label{eq:sympfields} 
  \begin{tikzcd}[row sep = 1 ex]
    -n & -n + 1 & -1 & 0 \\ \hline
    \Omega^{0,\bu}(X;L) \ar[r, "\del"] & \Omega^{1,\bu}(X;L) & 
     \PV^{1,\bu}(X; L) \ar[r, "\div"] & \PV^{0,\bu}(X; L).
\end{tikzcd}
\end{equation}
We denote the fields $(\beta,\gamma,\mu,\nu)$ respectively.
We are using the shorthand notation $\PV^{i}(X;L) = (\PV^{i,\bu}(X)\otimes \Omega^\bu(L), \dbar + \d_{L}))$.

The natural pairing between $\PV^i(X;L)$ and~$\Omega^i(X;L)$ is of degree $-\dim_\C(X) -\dim_\R(L)$. 
As such, the $\Z$-grading indicated in~\eqref{eq:sympfields} equips the sheaf of fields with a $(-1)$-shifted symplectic structure, provided that we choose the shift to be given by
\deq{
  n = \dim_\C(X) + \dim_\R(L) - 1.
}

We have constructed two equivalent descriptions of the BF theory which share the linear BRST complex \eqref{eq:sympfields}.

Explicitly, the action functional for BF theory associated to the local dg Lie algebra $\cL_0(X) \otimes \Omega^{\bu}(L)$ is
\deq{
  S_0 =  \beta (\dbar + \d_L) \nu +  \gamma (\dbar + \d_L) \mu +  \beta\partial_\Omega \mu + \frac{1}{2}  \gamma [\mu,\mu] +  \beta[\mu,\nu].
}
As in the Lie algebra structure of this strict model, notice that the Schouten--Nijenhuis bracket appears explicitly. 

The action functional of BF theory associated to $\cL_\infty(X) \otimes \Omega^{\bu}(L)$ is non polynomial.
\brian{I think the phrasing ``infinite order'' can mean a lot of different things. I've changed it to ``non polynomial''.} 
(In fact, it is related to the BCOV action functional via a procedure we outline below.)\ingmar{or somewhere} 
Explicitly, this action functional is
\deq{
  S_\infty =  \beta (\dbar + \d_L) \nu +  \gamma (\dbar + \d_L) \mu +   \beta \partial_\Omega \mu + \frac12 \frac{1}{1-\nu} (\partial \gamma) \mu^2.
}

We demonstrated above that the two local $L_\infty$ algebras on which these BF theories are based are equivalent. As such, the BF theories are also equivalent; the map~\eqref{eq:newbase} extends uniquely to an automorphism of BV theories.
Explicitly, the automorphism is
\begin{multline}
  \mu \mapsto e^{-\nu} \mu, \qquad \nu \mapsto 1-e^{-\nu} \\
  \beta \mapsto (\beta - \mu \gamma) e^{\nu},\qquad \gamma \mapsto e^{\nu} \gamma .
\end{multline}
\subsection{A family of deformations} 

\parsec
%check CME
We now specialize to the case where $X$ is an odd Calabi--Yau manifold and $L$ an odd-dimensional smooth manifold. 
In this case, there exists a consistent deformation of the BF theory we have constructed by a further term which is not of BF type. That is, we cannot define this deformation at the level of the local $L_\infty$ algebra resolving divergence-free holomorphic vector fields; it is defined only at the level of the full BF theory.

Recall that such deformations are governed by the classical master equation: the parameterized family of actions $S + g J$ is sensible only if 
\deq{
  \{S + g J, S + g J \} = 0.
}
Since this must hold for all $g$, and since the undeformed action $S$ is already a solution to the classical master equation, this reduces to the two conditions 
\deq[eq:2cond]{
  \{S,J\} = \{J,J\} = 0.
}
\begin{thm}
  Let $X$ be a Calabi--Yau manifold of complex dimension $2k+1$ and $L$ a smooth manifold of real dimension $2\ell + 1$, and consider the BV theory $(\cE, S_\infty)$ on $X \times L$ constructed above. The local functional 
  \deq{
    J = \gamma (\del \gamma)^k,
  }
  where $\gamma \in \Omega^1(X;L)$, defines a deformation of~$(\cE,S_\infty)$ as a $\Z/2\Z$-graded BV theory.
\end{thm}
\begin{proof}
  First off, we consider grading issues. In the $\Z$-grading on the free theory, the component $\gamma^{1,i;j}$ sits in degree $-2k-2\ell -2+i+j$. The local functional $J$ includes terms that are degree-$(k+1)$ polynomials in the components of $\gamma$, for which $\sum i = 2k+1$ and $\sum j = 2\ell + 1$. As such, we have
  \deq{
    \deg(J) =-  (2k+ 2\ell+ 2)(k+1) + 2k+2\ell + 2 = - k(2k+2\ell+2).
  }
  Thus $J$ defines an even local functional; of course, it does not define a local functional of degree zero.

  We then consider the classical master equation, in the form of the conditions~\eqref{eq:2cond}. It is immediate from the form of the shifted Poisson bracket that $\{J,J\} = 0$, since $J$ depends only on the $\gamma$ field. It remains to check that $\{I_\infty,J\} = 0$. For the quadratic term, we note that 
  \deq{
    \{\beta\div\mu, J\} = (k+1) \del\beta (\del \gamma)^k = 0,
  }
  because total derivatives are equivalent to zero as local functionals. The cubic term takes the form
  \deq{
    \left\{ \frac{1}{1-\nu} \del\gamma \mu^2, \gamma(\del\gamma)^k \right\} = \frac{2(k+1)}{1-\nu} (\del\gamma\vee\mu) (\del\gamma)^k.
}
This expression is zero for symmetry reasons. Recall that $\del\gamma$ is a two-form, and that the expression must be a totally symmetric local functional in $(k+1)$ copies of this two-form. We can ask whether such a  contraction exists just at the level of $\lie{sl}(2k+1)$ representation theory by computing the decomposition of the two-form  
\end{proof}

\subsection{Global symmetry algebra}

\subsection{Further equivalences}

\subsection{Equations of motion}

%whatever we can say about the formal moduli space

\subsection{Boundary conditions}

%foreshadow the Horava--Witten picture

\subsection{One-loop quantization} 

%cite result on thf quantization.


\section{Symmetries of the 11-dimensional theory}

\subsection{Component fields}

%Can we try to roughly say where each of our twisted fields come from in the untwisted theory? I imagine this section being pretty chill and mostly to orient physicists. 

\subsection{Residual supersymmetry} 

%m2brane

\parsec[sec:susy]

The (complexified) eleven-dimensional supertranslation algebra is a complex super Lie algebra of the form
\[
  \ft_{11d} = V \oplus \Pi S
\]
where $S$ is the (unique) spin representation and $V \cong \CC^{11}$ the complex vector representation, of~$\lie{so}(11, \CC)$. 
The bracket is the unique surjective $\lie{so}(11,\CC)$-equivariant map from the symmetric square of~$S$ to~$V$;
this decomposes into three irreducibles, 
\beqn\label{eqn:decomp}
  \Sym^2(S) \cong V \oplus \wedge^2 V \oplus \wedge^5 V.
\eeqn
Denote by $\Gamma_{\wedge^1}, \Gamma_{\wedge^2}, \Gamma_{\wedge^5}$ the projections onto each of the summands above. 
The bracket in $\ft_{11d}$ is defined using the first projection
\[
[\psi, \psi'] = \Gamma_{\wedge^1} (\psi, \psi') .
\]
The super Poincar\'{e} algebra is
\[
  \lie{siso}_{11d} = \lie{so}(11 , \CC) \ltimes \ft_{11d} .
\]
The $R$-symmetry is trivial in 11-dimensional supersymmetry. 


\parsec[sec:m2brane]

Extensions of the supersymmetry algebra correspond to the existence of branes in the original theory of supergravity \brian{good reference?}. 
In 11d, there are two such extensions corresponding to the M2 brane and the M5 brane.
We begin by describing a dg Lie algebra model for the M2 brane algebra, then we will discuss its relationship to other descriptions as a Lie 3-algebra \cite{Basu_2005,Bagger_2007,fiorenza2015super}. 

The M2 brane algebra is dg Lie algebra extension of the super Poincar\'e algebra $\lie{siso}_{11d}$.
Introduce the cochain complex $\Omega^{\bu}(\RR^{11})$ of (complex valued) differential forms on $\RR^{11}$ equipped with the de Rham differential $\d$.  
 
The super dg Lie algebra $\m2$ is the extension of $\lie{siso}_{11d}$ by the cocycle
  \[
    c_{M2} \in \clie^2\left(\lie{siso}_{11d} \; ; \; \Omega^\bu (\RR^{11})[2]\right)
  \]
  defined by the formula 
  \[c_{M2} (\psi, \psi') = \Gamma_{\wedge^2}(\psi, \psi') \in \Omega^2(\RR^{11})\]
  where $\Gamma_{\wedge^2}$ is the projection onto $\wedge^2 V$, thought of as the space of constant coefficient two-forms, as in the decomposition \eqref{eqn:decomp}.

We view $\m2$ as a $\ZZ \times \ZZ/2$-graded dg Lie algebra where the bracket is bidegree $(0,+)$ and the differential is bidegree $(1,+)$.

\parsec[sec:m2branetwist]

Fix a rank one supercharge $Q \in S$ satisfying $Q^2 = 0$.
This supercharge defines the holomorphic twist of 11-dimensional supergravity. 
\brian{cite \cite{SWspinor}}
We characterize the cohomology of the algebra $\m2$ with respect to this supercharge. 

Such a supercharge defines a maximal isotropic subspace $L \subset V$. 
We can decompose the algebra into $\lie{sl}(L) = \lie{sl}(5)$ representations by
\deq{
  V = L \oplus L^\vee \oplus \CC_t, \qquad S = \wedge^\bu L^\vee.
}
In the expression for $S$, we are omitting factors of $\det(L)^{\frac12}$ for simplicity. 
Also, $\lie{so}(11, \CC)$ decomposes as
\[
\lie{sl}(5) \oplus \wedge^2 L \oplus \wedge^2 L^\vee \oplus L \oplus L^\vee .
\]
Furthermore, the spinorial representation can be identified with
\[
S = \wedge^\bu (L^\vee) = \CC \oplus L^\vee \oplus \wedge^2 L^\vee \oplus \wedge^3 L^\vee \oplus \wedge^4 L^\vee \oplus \wedge^5 L^\vee .
\]
The element $Q$ lives in the first summand.
Let ${\rm Stab}(Q) \subset \lie{so}(11,\CC)$ be the stabilizer of $Q$. 
This is a parabolic subalgebra whose Levi factor is $\lie{sl}(5)$.

\begin{prop}\label{prop:model}
Fix a holomorphic supercharge $Q \in S$ and let $\m2^Q$ be the dg Lie algebra with bracket that on $\m2$ and with the differential $\d + [Q,-]$. 
As a $\ZZ/2$ graded space, the cohomology $H^\bu(\m2^Q)$ is
\[
    L^\vee \oplus {\rm Stab}(Q) \oplus \Pi \left(\wedge^3 L^\vee\right) \oplus \CC
  \]
whose elements we denote by $(v, m, \psi, c)$.

The transferred $L_\infty$ structure on $H^\bu(\fg) \cong H^\bu(\m2^Q)$ is a Lie-3 algebra given the obvious extension of ${\rm Stab}(Q)$ together with the brackets
\begin{align*}
[\psi, \psi']_2 & = \psi \wedge \psi' \in \wedge^4 L \cong L^\vee_v \\
[v, v', \psi]_3 & = \<v \wedge v', \psi\> \in \CC_c .
\end{align*}
\end{prop}

\parsec[]

As a consequence of the above proposition, we see that the dg Lie algebra $(\m2,Q)$ is {\em not} formal; there is a 3-ary $L_\infty$ bracket present in cohomology. 
Instead of this $L_\infty$ description, there is the following minimal model of this dg Lie algebra, which we will use to relate to twist of 11d supergravity. 

\begin{prop}
\label{prop:gmodel}
Let $\fg$ denote the following $\ZZ/2$ graded dg Lie algebra which as a cochain complex is
\[
H^\bu(\m2^Q) \oplus (\Pi L \xto{\id} L)  .
\]
Denote the elements of the second summand by $(\tilde{\lambda}, \lambda)$. 
The nontrivial Lie brackets are
\begin{align*}
[v,\lambda] & = \<v, \til\lambda\> \in \CC_c \\ 
[v,\psi] & = \<v, \psi\> \in \Pi L_{\Tilde{\lambda}} \\
[\psi, \psi'] & = \psi \wedge \psi' \in L^\vee_v  .
\end{align*}
There is an $L_\infty$ map 
\[
\fg \rightsquigarrow \m2^Q
\] 
which is a quasi-isomorphism of complexes.  
\end{prop}
\begin{proof}
The supercharge $Q$ is odd and of cohomological degree zero.
Since the differential on $\m2$ is even of cohomological degree $+1$, only a totalized $\ZZ/2$ grading makes the differential $\d + [Q,-]$ homogenous. 

The cohomology of the non-centrally extended algebra was computed in \cite{SWspinor}, we briefly recall the result. 
The element $Q$ only acts nontrivially on the summands $\wedge^4 L^\vee$ and $\wedge^5 L^\vee$ in $S$. 
The image of $\wedge^4 L^\vee \cong L$ trivializes the antiholomorphic translations while the image of $\wedge^5 L^\vee$ trivializes the time translation.
So, of the translations, only the holomorphic ones $L^\vee$ survive.
The map 
\[
[Q,-] \colon \lie{so}(11,\CC) \to S 
\] 
is the projection onto $\wedge^0 L^\vee \oplus \wedge^1 L^\vee \oplus \wedge^2 L^\vee$. 
The kernel of $[Q,-]$ is the stabilizer ${\rm Stab}(Q)$.

In summary, the space of odd translations which survive cohomology is $\wedge^3 L^\vee \cong \wedge^2 L$.
This completes the calculation of the cohomology. 

We embed $\fg$ into $\m2^Q$ in the following way: ${\rm Stab}(Q)$ and $L^\vee$ sit inside in the evident way.
The central element maps to $c \mapsto \pm 1 \in \Omega^0(\RR^{11})$.
The summand $L_\lambda$ is mapped to the linear functions in $\Omega^0(\RR^{11})$ and $\Pi L_{\Tilde{\lambda}}$ is sent to the constant coefficient one-forms in $\Pi \Omega^1(\RR^{11})$. 
It remains to declare where $\psi \in \wedge^2 L$ is mapped.

Define the map
\[
H \colon \Omega^2 (\RR^{11}) \to \Omega^1(\RR^{11})
\]
which sends a two-form $\alpha$ to the one-form $H \alpha$ defined by the formula
\[
(H \alpha) (x) = \int_0^x \alpha .
\]

Notice that if $\alpha$ is $\d$-closed then $\d (H \alpha) = \alpha$. 
It follows that any element $\psi \in \wedge^2 L \subset S$ can be lifted to a closed element at the cochain level in $\m2^Q$ by the formula
\[
\Tilde{\psi} = \psi - H \Gamma_{\wedge^2} (Q, \psi) \in \Pi S \oplus \Pi \Omega^1 .
\]
Thus, sending $\psi \mapsto \Tilde{\psi}$ defines a cochain map $\fg \to \m2^Q$. 

The Lie bracket $[\Tilde{\psi}, \Tilde{\psi}']$ agrees with $[\psi, \psi']$. 
On the other hand, in $\m2^Q$ there is the Lie bracket 
\[
[v,\Tilde{\psi}] = - L_v (H \Gamma_{\wedge^2} (Q, \psi)) = -\<v, \Gamma_{\wedge^2}(Q, \psi) - \d \<v, H \Gamma_{\wedge^2}(Q, \psi) .
\]
The first term agrees with the bracket $[v, \psi]_{\fg}$ in $\fg$. 
The other term is exact in $\m2^Q$ and can hence be corrected by the following bilinear  
\[
v \otimes \psi \mapsto \<v, H \Gamma_{\wedge^2} \> \in L_\lambda .
\] 
Together with the cochain map described above, this bilinear term prescribes the desired $L_\infty$ map. 

\end{proof}

\parsec[s:residual]
Consider now the $\ZZ/2$ graded dg Lie algebra $\cL$ underlying the eleven-dimensional theory on $\CC^5 \times \RR$. 

\begin{prop}
There is a map of $\ZZ/2$-graded dg Lie algebras 
\[
\fg \to \cL 
\]
where $\fg$ is the model for the $Q$-cohomology of the super Lie algebra $\m2$ from Proposition \ref{prop:gmodel}. 
In particular, the $Q$-twisted algebra $\m2^Q \simeq \fg$ is a symmetry of eleven-dimensional theory on $\CC^5 \times \RR$. 
\end{prop}
\begin{proof}
Recall the model $\fg$ takes the following form
\beqn 
\begin{tikzcd}
\ul{even} & \ul{odd} & \ul{even} \\
 L_a & \wedge^2 L_b & L^\vee \\
\wedge^2 L_a & & \\
\lie{sl}(5) \\
L_b \ar[r, "\id"] & L_c .
\end{tikzcd}
\eeqn
The subscripts in $L_a, L_b, L_c$ are used to distinguish between the various copies of $L$.

The map from the dg model $\fg$ to the fields of the twisted $11$-dimensional supergravity theory is as follows. 
\begin{align*}
 L_a & \mapsto 0 \\
 \wedge^2 L_a & \mapsto 0 \\
z_i \wedge z_j \in \wedge^2 L_b & \mapsto \frac12 (z_i \d z_j - z_j \d z_i) \in \Omega^{1,0} (\CC^5) \hotimes \Omega^0 (\RR) \\
A_{ij} \in \lie{sl}(5) & \mapsto \sum_{ij} A_{ij} z_i \in \PV^{1,0}(\CC^5) \hotimes \Omega^0(\RR) \\\frac{\partial}{\partial z_j} \in L^\vee & \mapsto
\frac{\partial}{\partial z_i} \in \PV^{1,0} (\CC^5) \hotimes \Omega^0 (\RR^5) \\ z_i \in L_b & \mapsto z_i \in \Omega^{0,0}(\CC^5) \hotimes \Omega^0 (\RR) \\
z_i \in L_c & \mapsto \d z_i \in \Omega^{1,0}(\CC^5) \hotimes \Omega^0 (\RR)
\end{align*}
One immediately verifies that this is a map of dg Lie algebras. 
\end{proof}

\parsec[]

In this section we compare to the work of \brian{Linfty refs}.
In these references, the algebra $\m2$ is defined as an $L_\infty$  
central extension of $\lie{siso}_{11d}$. 

Recall that given two spinors $\psi, \psi' \in S$ we can form the constant coefficient two-form $\Gamma_{\wedge^2} (\psi, \psi')$. 
Using this two-form we can define the following four-linear expression
\[
\mu_2 (\psi, \psi',v,v') = \<v \wedge v', \Gamma(\psi, \psi')\> .
\]
This expression is symmetric on the spinors and anti-symmetric on the vectors, therefore it defines an element in $\clie^4(\ft_{11d})$. 
In \cite{??} it is shown that this cocycle $\mu_2$ is 

%For clarity we adjust notation for $\lie{sl}(5)$-representations. Denote by $V^{1,0} = L^\vee$ the space of holomorphic translations on $\CC^5$ and $V^{\vee 1,0}$ the translation invariant holomorphic one-forms on $\CC^5$. 


\section{Twisted supergavity on AdS space}

\subsection{Superconformal algebras}

\begin{prop}
The $Q$-cohomology of the 6d $\cN=(2,0)$ superconformal algebra and the 3d $\cN=8$ superconformal algebra is isomorphic to the super Lie algebra $\lie{osp}(6|1)$. 
\end{prop} 


\subsection{The ${\rm AdS}_7 \times S^4$ background}

\parsec[]

Choose a splitting 
\[
\CC^5 \times \RR = \CC_z^3 \times \CC_w^2 \times \RR
\]
and denote coordinates $z_i, i=1,2,3$ and $w_a, a=1,2$.

The bosonic part of $\lie{osp}(6|1)$ is the direct sum Lie algebra
\[
\lie{sl}(4) \oplus \lie{sl}(2) .
\]
which we write as $\lie{sl}(W) \oplus \lie{sl}(R)$ with $W,R$ complex four, two dimensional complex vector spaces. 

The embedding of the bosonic piece can roughly be described as follows. 
The Lie algebra $\lie{sl}(4)$ represents conformal transformations along the plane $\CC^3_z$.
Since not all such infinitesimal transformations are divergence-free, there precise formulas must be adjusted.   
The Lie algebra $\lie{sl}(2)$ represents rotations in $\CC^2_w$; the vector fields representing these transformations are automatically divergence free.


\begin{itemize}

\item
The bosonic abelian subalgebra $\CC^3 \subset \lie{sl}(4)$ is mapped to the translations 
\[
\frac{\del}{\del z_i} \in \PV^{1,0}(\CC^5) \otimes \Omega^0(\RR) , \quad i=1,2,3.
\]

\item
The bosonic subalgebra $\lie{sl}(3) \subset \lie{sl}(4)$ is mapped to the 
rotations
\[
A_{ij} z_i \frac{\del}{\del z_j} \in \PV^{1,0}(\CC^5)\otimes \Omega^0(\RR) , \quad (A_{ij}) \in \lie{sl}(3) .
\]

\item
The bosonic subalgebra $\CC \subset \lie{sl}(4)$ is mapped to the element
\[
\sum_{i=1}^3 z_i \frac{\del}{\del z_i} - \frac32 \sum_{a=1}^2 w_a \frac{\del}{\del w_a} \in \PV^{1,0}(\CC^5) \otimes \Omega^0(\RR)  .
\] 
Notice that these vector fields are divergence-free and restrict to the ordinary dilation along $w=0$. 
\item 
The bosonic subalgebra of $\lie{sl}(4)$ describing special conformal transformations on $\CC^3$ is mapped to the elements 
\[
z_i \left(\sum_{i=1}^3 z_i \frac{\del}{\del z_i} - 2 \sum_{a=1}^2 w_a \frac{\del}{\del w_a} \right) \in \PV^{1,0}(\CC^5) \otimes \Omega^0(\RR) .
\] 
Notice that these vector fields are divergence-free and restrict to the ordinary special conformal transformations along $w=0$. 
\item 
The bosonic summand $\lie{sl}(2)$ is mapped to the triple
\[
w_1 \frac{\del}{\del w_2}, w_2 \frac{\del}{\del w_1}, \frac12 \left(w_1 \frac{\del}{\del w_1} - w_2 \frac{\del}{\del w_2}\right) \in \PV^{1,0}(\CC^5) \otimes \Omega^0(\RR) .
\]
\end{itemize}

The odd part of the algebra $\lie{osp}(6|1)$ is $\wedge^4 W \otimes R$ where $W$ is the fundamental $\lie{sl}(4)$ representation and $R$ is the fundamental $\lie{sl}(2)$ representation. 
It is natural to split $W = L \oplus \CC$ with $L = \CC^3$ the fundamental $\lie{sl}(3) \subset \lie{sl}(4)$ representation. 
Then the odd part decomposes as
\[
L \otimes R \oplus \wedge^2 L \otimes R \cong \CC^3 \otimes \CC^2 \oplus \wedge^2 \CC^3 \otimes \CC .
\]

\begin{itemize} 
\item The summand $L \otimes R$ consists of the remaining 6d superstranlsations. 
It is mapped to the fields 
\[
z_i \d w_a \in \Omega^{1,0}(\CC^5) \otimes \Omega^0(\RR) ,\quad a=1,2, \quad i =1,2,3.
\] 
\item The summand $\wedge^2 L \otimes R$ consists of the remaining 6d superconformal transformations. 
It is mapped to the fields
\[
\ep^{ijk} z_i w_a \d z_j \in \Omega^{1,0}(\CC^5)\otimes \Omega^0(\RR) , \quad a = 1,2, \quad k = 1,2,3. 
\]
\end{itemize}

\subsection{The ${\rm AdS}_4 \times S^7$ background}

\parsec[]
\brian{discuss coupling}

\parsec[]

We decompose the 11-dimensional manifold $\CC^5 \times \RR$ as
\[
\CC_z \times \CC^4_w \times \RR .
\]
Based on the discussion above, it is natural to expect that there is a field of twisted 11-dimensional supergravity which sources the twist of a stack of $N$ $M2$ branes living on the submanifold $\CC_z \times \RR \cong \{w=0\}\subset \CC^5$. 

The differential form which sources the brane is an element
\[
\til{F} \in \Omega^{4,3} (\CC^4_w \, \setminus \, 0) \otimes \Omega^{0,0} (\CC_z) \otimes \Omega^{0} (\RR) \subset \Dol\left(\CC^5 \times \RR \, \setminus \, \{w=0\}\right) .
\]
Equivalently, we can think about this as a distributional valued form $\til{F} \in \Bar{\Omega}^{4,3}(\CC^5) \otimes \Omega^0 (\RR)$ which satisfies the distributional equation
\[
\dbar \til{F} = N \delta_{w=0} 
\]
where $\delta_{w=0}$ is the Dirac $\delta$-distributional for the submanifold $\{w=0\} = \CC_z \times \RR$. 

Using the Calabi--Yau form we can identify such a differential form with a field of twisted supergravity. 
Indeed $F = \til{F} \wedge \Omega^{-1}$ is a distributional field of type
\[
F \in \Bar{\PV}^{1,4}(\CC^5) \otimes \Omega^0 (\RR) .
\]
For $F$ to make sense as a background of twisted supergravity it must satisfies the appropriate (possibly nonlinear) equation of motion, which we now verify. 

\begin{lem}
The field 
\[
F = \# N \frac{\sum_{a=1}^4 \wbar_a \d \wbar_1 \cdots \Hat{\d \wbar_a} \cdots \d \wbar_4}{\|w\|^{8}} \partial_z .
\]
satisfies the $\mu$-equation of motion in the presence of a stack of $N$ $M2$ branes sourced by the term $N \Omega^{-1} \delta_{w=0}$:
\[
\dbar F + \div F + \frac12 [F, F] = N \Omega^{-1} \delta_{w=0} 
\]
where we set the field in the component $\gamma \in \Omega^{1,\bu}(\CC^5) \otimes \Omega^\bu(\RR)$ equal to zero. 
\end{lem}

\begin{proof}
The equation $\dbar F = N \Omega^{-1} \delta_{w=0}$ characterizes the Bochner--Martinelli kernel representing the residue class on $\CC^4 \, \setminus \, 0$. 
It is clear that $\div F = 0$ and $[F, F] = 0$ by simple type reasons. 
\end{proof}

\parsec[]

Recall the bosonic part of $\lie{osp}(6|1)$ is the direct sum Lie algebra $\lie{sl}(4) \oplus \lie{sl}(2)$. 
For the embedding of the bosonic subalgebra, the roles of $\lie{sl}(4)$ and $\lie{sl}(2)$ are somewhat reversed for the M2 brane as compared to the M5 brane. 
The Lie algebra $\lie{sl}(2)$ represents special conformal transformations in $\CC_z$; the vector fields representing these transformations are not divergence-free so must be slightly adjusted. 
The Lie algebra $\lie{sl}(4)$ represents rotations along the plane $\CC^4_w$.   


\begin{itemize}
\item The bosonic summand $\lie{sl}(2)$ is mapped to the vector fields:
\[
\frac{\del}{\del z} ,\quad z \frac{\del}{\del z} - \frac14 \sum_{a=1}^4 w_a \frac{\del}{\del w_a} , \quad z \left(z \frac{\del}{\del z} - \frac12 \sum_{a=1}^4 w_a \frac{\del}{\del w_a} \right) \in \PV^{1,0}(\CC^5) \otimes \Omega^0(\RR) .
\]
Notice that these vector fields are divergence-free and along $w=0$ reduce to the usual special conformal transformations.
\item The bosonic summand $\lie{sl}(4)$ is mapped to the $4$-dimensional rotations: 
\[
\sum_{a,b=1}^4 B_{ab} w_a \frac{\del}{\del w_b} \in \PV^{1,0}(\CC^5) \otimes \Omega^0(\RR) , \quad (B_{ab}) \in \lie{sl}(4) .
\]
\end{itemize}

The odd part of the algebra $\lie{osp}(6|1)$ is $\wedge^4 W \otimes R$ where $W$ is the fundamental $\lie{sl}(4)$ representation and $R$ is the fundamental $\lie{sl}(2)$ representation. 
It is natural to split $R = \CC_{+1} \oplus \CC_{-1}$ so that the odd part decomposes as
\[
(\wedge^2 \CC^4)_{+1} \otimes (\wedge^2 \CC^4)_{-1} .
\]

\begin{itemize}
\item 
The fermionic summand $(\wedge^2 \CC^4)_{+1}$ consists of the supertranslations. 
It is mapped to the fields: 
\[
\ep^{abcd} w_c \d w_d \in \Omega^{1,0}(\CC^5) \otimes \Omega^0(\RR) , \quad a,b=1,2,3,4. 
\] 
\item The fermionic summand $(\wedge^2 \CC^4)_{-1}$ consists of the remaining superconformal transformations. 
It is mapped to the fields: 
\[
\ep^{abcd} z w_c \d w_d \in \Omega^{1,0}(\CC^5) \otimes \Omega^0(\RR) , \quad a,b=1,2,3,4. 
\] 
\end{itemize}

\subsection{Supersymmetric indices}

\brian{SU(5) and SU(2) character calculations}

\section{Dimensional reduction and compactifications} 
\def\im{{\rm i}}

\section{Dimensional reduction and 10-dimensional supergravity}
\label{sec:dimred}

In this section we demonstrate that our proposal for the action of minimally twisted 11-dimensional supergravity agrees with conjectural descriptions of twisted type IIA and type I supergravities due to Costello and Li. 

The original motivation for $M$-theory was as the strong coupling limit for type IIA string theory.
Roughly, the radius of the $M$-theory circle plays the role of this coupling constant. 
Additionally, at low energies $M$-theory is expected to be approximated by 11-dimensional supergravity in the same way that the low energy limit of type IIA/IIB string theory is type IIA/IIB supergravity. 
Combining these two pictures, various checks have been made that the dimensional reduction of 11-dimensional supergravity along the $M$-theory circle is type IIA supergravity. 

Motivated by the topological string, Costello and Li have laid out a series of conjectures for twists of type IIA/IIB supergravity \cite{CLsugra} and type I supergravity \cite{CLtypeI}. 
Their description was inspired by the model of the open and closed $B$-model topological string on a Calabi--Yau manifold. 
The open sector is holomorphic Chern--Simons theory \cite{WittenOpen} and the closed sector is called Kodaira--Spencer theory \cite{BCOV}. 
There are a few different versions of Kodaira--Spencer theory, but the shared characteristic is that they are all `gravitational' in nature; they describe fluctuations of the Calabi--Yau structure. 
From this point of view, Kodaira--Spencer theory is at the heart of the formulation of the various flavors of twisted 10-dimensional supergravity.

We begin by introducing certain variants of Kodaira--Spencer theory which will feature in the descriptions of twists of type IIA and type I supergravity.

\subsection{Kodaira--Spencer theory}
\label{s:BCOV}

Let $X$ be a Calabi--Yau manifold; for now it can be of arbitrary complex dimension $d$. 
Define
\deq{
  \PV^{i,j}(X) = \Omega^{0,j}(X, \wedge^i \T_X).
}
We will consider the graded space $\PV^{\bu,\bu}(X) = \oplus_{i,j} \PV^{i,j}(X)[-i-j]$ where the piece of type $(i,j)$ sits in degree $i+j$. 

For each fixed $i$, while we let $j$ vary, the $\dbar$ operator defines a cochain complex $\PV^{i,\bu}(X) = (\oplus_j\PV^{i,j}(X) [-j], \dbar)$ which provides a resolution for the sheaf of holomorphic polyvector fields of type $i$. 
The divergence operator extends to an operator of the form
\[
\div \colon \PV^{i,\bu}(X) \to \PV^{i-1,\bu}(X) .
\]

Motivated by the states of the topological $B$-model, one defines the fields of Kodaira--Spencer gravity on $X$ to be the cochain complex
\beqn\label{eqn:ks1}
\left(\PV^{\bu,\bu} (X)[[u]] [2] \, , \, \dbar + u \div\right) .
\eeqn 
Here, $u$ is a parameter of cohomological degree $+2$, which turns $\delta_{KS}^{(1)} = \dbar + u \div$ into an operator of homogenous degree $+1$. 
We also have performed an overall cohomological shift by $2$ so that $u^k \PV^{i,j}$ sits in degree $i+j+2k-2$. 
More precisely, this is a model for the $S^1$-equivariant cohomology of the states of the $B$-model on a closed disk. 
We refer to \cite{CLtypeI, CLsugra} for detailed justification for this ansatz. 

\parsec[s:poisson]
The original action for Kodaira-Spencer theory posited by \cite{BCOV} has a nonlocal kinetic term. In the BV formalism, this is codified by stipulating that the BV pairing is a degenerate odd Poisson tensor rather than an odd symplectic form. 
The Poisson kernel is given by the expression 
\[
(\div\otimes 1)\delta_{\Delta \subset X\times X} \in \left[\PV^{\bu,\bu}(X)\right]^{\hotimes 2} ,
\]
see \cite[{\S 1.4}]{CLbcov1}. 
Here, we view the $\delta$-distribution as a polyvector field using the Calabi--Yau form. 
Notice that the shifted Poisson tensor does not involve the parameter $u$ at all. 
For this reason, only the duals of a small number of fields pair nontrivially under the resulting odd BV bracket. 
%Explicitly, the above kernel pairs against the dual to 
%\[
%\Sigma^1 = \sum u^k\mu^1_k, \Sigma^2 = \sum u^k\mu^2_k\in \PV^{\bu,\bu}%(X)[[u]][2]
%\]
% by the formula 
%\[
%\int \overline{\mu}_0^1\div \overline{\mu}_0^2.
%\] 
%Here, the overlines denote the dual field in 
%\[
%(\PV^{\bu,\bu}(X)[[u]][2])^*=\PV_c^{d-\bu,d-\bu}(X)[[u]][2].
%\] 
%That is, further expanding the fields as 
%\[
%\mu_k = \sum_{i,j=1}^d\mu_k^{(i,j)}\in \PV^{i,j}
%\] 
%the expression pairs $\overline{\mu}_0^{(i,j)}$ with $\overline{\mu}_0^{(d-1-i, d-j)}$.

\parsec[s:ksaction] 

There is a natural local interaction which equips the complex \eqref{eqn:ks1} with the structure of $\Z/2$ graded Poisson BV theory. Explicitly, it is given by 
\beqn
I_{BCOV}(\Sigma) = {\rm Tr}_X \, \langle \exp \Sigma\rangle_0 = \sum_{n\geq 0} {\rm Tr}_X \, \langle\Sigma^{\otimes n}\rangle_0
\eeqn
where ${\rm Tr}_X \, \Phi = \int_X (\Phi \vee \Omega) \wedge \Omega$ and where $\langle - \rangle_0$ denotes the genus zero Gromov-Witten invariant with marked points
\beqn
\langle u^{k_1}\mu_1 \otimes \cdots \otimes u^{k_m}\mu_m\rangle_0 := \left (\int _{\overline {\cM}_{0,m}} \psi_1^{k_1}\cdots \psi_m^{k_m}\right ) \mu_1\cdots \mu_m = \binom{m-3}{ k_1,\cdots, k_m}  \mu_1\cdots \mu_m.
\eeqn

This interaction is extremely natural from the point of view of string field theory. Indeed, the B-model localizes to the space of constant maps into $X$, which factors as a product of $\overline{\cM}_{0,m}\times X$. This is in keeping with finding an interaction that factors as an integral over $X$ times an integral over $\overline{\cM}_{0,m}$. 

In \cite{BCOV} the authors show that the above interaction satisfies the classical master equation. Moreover, they show that the $L_\infty$ structure determined by the above action is equivalent to a natural dgla structure on the complex of fields with Lie bracket given by the Schouten bracket. Explicitly, the equivalence is given by the transcendental automorphism 
\[
\Sigma \mapsto [u(\exp (\Sigma/u)-1)]_+
\]
where $[-]_+$ denotes projection onto positive powers of $u$.

\parsec[s:minimalks]

We pointed out in \S\ref{s:poisson} that the majority of fields pair to zero under the Poisson tensor. Physically these correspond to closed string fields that do not propogate. In the supergravity approximation, the fields that survive are those closed string fields that propogate. In terms of our description of closed string field theory in terms of Kodaira-Spencer theory, this motivates us to consider the smallest cochain complex containing those fields thathave nonzero pairing under the Poisson tensor. This is referred to as minimal Kodaira-Spencer theory.

The fields of minimal Kodaira-Spencer theory are given by the subcomplex of \label{eqn:ks1}
\beqn
\left (\bigoplus_{i+j\leq d -1}u^i\PV^{j,\bu}(X)[2], \dbar + u \div\right).
\eeqn
We observe that the original odd Poisson tensor lives in this subcomplex. 
%here is a natural odd Poisson tensor on the above cochain complex such that the inclusion of this subcomplex into the fields of Kodaira-Spencer theory is a Poisson map. 
There is a natural action functional given by restricting $I_{BCOV}$ to this space.

\subsection{The $SU(4)$ twist of type IIA supergravity}\label{sec:SU(4)twist}

We recall the description of the $SU(4)$ twist of type IIA supergravity conjectured in \cite{CLsugra}. 
In principal, there is also a minimal, $SU(5)$ invariant, twist of type IIA supergravity but so far no description, even conjecturally, exists.
We turn to this in \S \ref{s:su5IIA}. 

Let $X$ be a Calabi-Yau manifold of complex dimension four. 
The $\ZZ/2$ graded complex of fields of minimal Kodaira--Spencer theory on $X$ takes the form
\beqn
\begin{tikzcd}
- & + & - & + \\ \hline
                        &                          &                                     & {\PV^{0,\bu}}    \\
                        &                          & {\PV^{1,\bu}} \arrow[r, "u\div"]    & {u\PV^{0,\bu}}   \\
                        & {\PV^{2,\bu}} \arrow[r, "u\div"]  & {u\PV^{1,\bu}} \arrow[r, "u\div"]   & {u^2\PV^{0,\bu}} \\
{\PV^{3,\bu}} \arrow[r, "u\div"] & {u\PV^{2,\bu}} \arrow[r, "u\div"] & {u^2\PV^{1,\bu}} \arrow[r, "u\div"] & {u^3\PV^{0,\bu}}
\end{tikzcd}.
\eeqn
Denote this complex by $\cE_{mKS}(X)$. 
Here, $u^\ell\PV^{k,i}$ is placed in parity $k + i -1 \mod 2$. 
%Denote this complex by $\cE_{mKS}(X)$. 
The classical BCOV action $I_{BCOV}$ follows from the general formula we gave above. 

With this in hand the conjecture of \cite{CLsugra} takes the following form.

\begin{conj}
The $SU(4)$-invariant twist of type IIA supergravity on $\RR^2\times \CC^4$ is the $\Z/2$-graded Poisson BV theory with fields 
\beqn\label{eqn:IIAfields}
\alpha = \sum_n \alpha_n u^n \in \cE_{mKS}(\CC^4) \otimes \Omega^{\bu}(\RR^2).
\eeqn
The classical interaction takes the form \[I_{IIA} = \int_{\CC^4 \times \RR^2} \alpha_0^3 + \cdots\]
%The $L_{\infty}$ structure is the natural one on the tensor product of the cdga $\Omega^{\bu}(M)$ with the $L_{\infty}$-algebra $\cE_{mKS}$.
\end{conj}

We will need a more detailed description of the classical action. 
For the moment, let us introduce some notations for the fields of this IIA model, as always we leave the internal Dolbeault degree implicit:
\begin{multline}
\eta \in \PV^{0,\bu}(\CC^4) \otimes \Omega^\bu (\RR^2), \quad \mu + u \nu \in \PV^{1, \bu}(\CC^4) \otimes \Omega^\bu (\RR^2) \oplus u \PV^{0,\bu} (\CC^4) \otimes \Omega^\bu (\RR^2) \\
\Pi \in \PV^{3,\bu}(\CC^4) \otimes \Omega^\bu(\RR^2), \quad \sigma \in \PV^{3,\bu}(\CC^4) \otimes \Omega^\bu (\RR^2) .
\end{multline}
We will not need an explicit notation for the remaining descendant fields. 

With this notation in hand, we have the more precise form of the action appearing in the conjecture:
\beqn\label{eqn:IIAaction}
I_{IIA} = \frac12 {\rm Tr}_{\CC^4 \times \RR^2} \frac{1}{1-\nu} \mu^2 \wedge \Pi + {\rm Tr}_{\CC^4 \times \RR^2} \frac{1}{1-\nu} \eta \wedge \mu \wedge \sigma + \frac12 {\rm Tr}_{\CC^4 \times \RR^2} \frac{1}{1-\nu} \eta \wedge \Pi^2 + \cdots 
\eeqn
where the $\cdots$ denote terms involving higher order descendants. 

\subsection{Reduction to IIA supergravity}
\label{s:su4red}

We now turn back to our 11-dimensional theory. 
The first goal is to compare the dimensional reduction of our 11-dimensional theory on $\CC^5 \times \RR$
%where $X$ is a Calabi-Yau 4-fold 
with the $SU(4)$ invariant twist of type IIA on $\R^{2}\times \CC^4$. 
Doing so will require a slight modification to the description of the $SU(4)$ twist of IIA supergravity recollected in \S \ref{sec:SU(4)twist}. 

\parsec[sec:IIApot]

Recall that in the physical theory, the components of the $C$-field in 11d that are not supported along the M-theory circle become the components of the Ramond--Ramond 2-form of type IIA. However, as noted in \cite{CLsugra} components of Ramond--Ramond fields do not appear as fields in Kodaira--Spencer theory; rather it is components of their field strengths that appear. 
We recalled in \S \ref{s:components} that components of the $C$-field become components of $\gamma_{11d}$ in $\cE$.
This suggests that we must modify our description of the twist of type IIA to include potentials for certain fields.

The fundamental fields of the $SU(4)$ twist of IIA supergravity were given in \eqref{eqn:IIAfields}. 
We modify the space of fields by introducing potentials for both the $\Pi$ and $\sigma$ fields. 
First, we introduce a field $\gamma \in \Omega^{1,\bu}(\CC^4) \otimes \Omega^\bu(\RR^2)$ (not to be confused, yet, with the $\gamma$ field in our 11-dimensional theory) which satisfies $\Pi \vee \Omega = \del \gamma$ where $\Omega$ is the Calabi--Yau form on $\CC^4$. 
This condition does not uniquely fix $\gamma$. 
There is a new linear gauge symmetry determined by $\gamma \to \gamma + \div \beta$ where $\beta$ is a ghost that we must also introduce. 
Similarly, we introduce a field $\theta \in \Omega^{0,\bu}(\CC^4) \otimes \Omega^\bu(\RR^2)$ which satisfies $\sigma \vee \Omega = \del \theta$, there is no extra gauge symmetry present in this condition.\footnote{Using the Calabi--Yau form we have normalized the potential fields $\gamma, \beta,\theta$ to be written as differential forms instead of polyvector fields.}

In diagrammatic detail, the potential theory we are considering has underlying cochain complex of fields
\beqn\label{eqn:IIApot}
\begin{tikzcd}
- & + \\ \hline
& {\PV^{0,\bu} (\CC^4) \otimes \Omega^\bu (\RR^2) }_\eta  \\
{\PV^{1,\bu} (\CC^4) \otimes \Omega^\bu (\RR^2)}_\mu \arrow[r, "u\div"] & u{\PV^{0,\bu} (\CC^4) \otimes \Omega^\bu (\RR^2)}_\nu \\
u^{-1}{\Omega^{0,\bu} (\CC^4) \otimes \Omega^\bu (\RR^2)}_\beta \arrow[r, "u\del"] & {\Omega^{1,\bu} (\CC^4) \otimes \Omega^\bu (\RR^2)}_\gamma  \\
{\Omega^{0,\bu} (\CC^4) \otimes \Omega^\bu (\RR^2)}_\theta &
\end{tikzcd}
\eeqn.

The original space of fields of the twist of IIA supergravity on $\CC^4 \times \RR^2$ was equipped with an odd Poisson bivector which was degenerate.
In other words, it did not define a theory in the conventional BV formalism. 
One of the key features of this new complex of fields, after we have taken these potentials, is that it is equipped with an odd non-degenerate pairing thus equipping it with the structure of a theory in the conventional BV formalism. 

The pairing is $\Res_u \frac{\d u}{u} \int^\Omega_{\CC^4 \times \RR^2} \alpha \vee \alpha'$ where $\alpha, \alpha'$ are two general fields in this potential theory on $\CC^4 \times \RR^2$. 
Explicitly, in the description of the fields in \eqref{eqn:IIApot} the pairing is 
\[
\int^\Omega_{\CC^4 \times \RR^2} \eta \theta + \int^\Omega_{\CC^4 \times \RR^2} \mu \vee \gamma + \int^\Omega_{\CC^4 \times \RR^2} \nu \beta .
\]
This pairing is compatible with the odd Poisson bracket present in the original theory on $\CC^4 \times \RR^2$.

The type IIA action completely determines the action of this theory with potentials. 
One simply takes the \eqref{eqn:IIAaction} and replaces all appearances of $\Pi$ with $\div \gamma$ and all appearances of $\sigma$ with $\div \theta$. 
This yields the interaction of the potential theory
\beqn\label{eqn:IIAactionpot}
\til I_{IIA} = \frac12 \int^\Omega_{\CC^4 \times \RR^2} \frac{1}{1-\nu} \mu^2 \vee \del \gamma + \int^\Omega_{\CC^4 \times \RR^2} \frac{1}{1-\nu} (\eta \wedge \mu) \vee \del \theta + \frac12 \int_{\CC^4 \times \RR^2} \frac{1}{1-\nu} \eta \wedge \del \gamma \wedge \del \gamma 
\eeqn
Notice that the terms involving higher descendants vanishes since these fields are set to zero in the potential theory.

%Note that there is a natural map of cochain complexes $\partial \cE_{pot}\to \cE_{mKS}$ given by the dotted arrows below:
%\beqn
%\begin{tikzcd}
%                                       & {\PV^{0,\bu}} \arrow[rrr, dotted, "\id"] &                          &                                     & {\PV^{0,\bu}}    \\
%{\PV^{1,\bu}} \arrow[r, "u\div"]       & {u\PV^{0,\bu}} \arrow[rr, dotted, "\id"] &                          & {\PV^{1,\bu}} \arrow[r, "u\div"]    & {u\PV^{0,\bu}}   \\
%{u^{-1}\PV^{4,\bu}} \arrow[r, "u\div"] & {\PV^{3,\bu}} \arrow[r, dotted, "\div"]  & {\PV^{2,\bu}} \arrow[r]  & {u\PV^{1,\bu}} \arrow[r, "u\div"]   & {u^2\PV^{0,\bu}} \\
%{\PV^{4,\bu}} \arrow[r, dotted, "\div"]        & {\PV^{3,\bu}} \arrow[r]          & {u\PV^{2,\bu}} \arrow[r] & {u^2\PV^{1,\bu}} \arrow[r, "u\div"] & {u^3\PV^{0,\bu}}
%\end{tikzcd}.
%\eeqn
%This is easily seen to be a Poisson map. We have that $I_{pot} = \partial^{*}I_{BCOV}$, so we see that $I_{pot}$ satisfies the classical master equation. Therefore, $I_{pot}$ determines an $L_{\infty}$ structure on $\cE_{pot}$, and hence one on $\Omega^{\bu}(\R^{2})\otimes \cE_{pot}$. In what follows, the $SU(4)$-invariant twist of IIA will refer to the BV theory $\Omega^{\bu}(\R^{2})\otimes \cE_{pot}$.

\parsec[-]

We turn to the proof of the main result of this section that the dimensional reduction of our 11-dimensional theory agrees with the twist of IIA supergravity just introduced. 

We recall the notion of dimensional along a holomorphic direction following \cite{ESW}. 
Suppose that $V_\RR$ is a real vector space and denote by $V$ its complexification. 
We consider a field theory defined on $M \times V$, which is holomorphic along $V$ (in particular, this means that the theory is translation invariant along $V$).  
We consider the dimensional reduction along the projection 
\beqn\label{eqn:dimred}
M \times V \to M \times V_\RR
\eeqn
induced by ${\rm Re} \colon V \to V_\RR$.
Most relevant for us is the case when $V = \CC$ and $M$ is $\CC^4 \times \RR$, but the explicit form of the theory along $M$ is not important at the moment.

For illustrative purposes, let us first assume that $M$ is a point and that the space of fields is of the form $\Omega^{0,\bu}(V) \otimes W$ for $W$ some graded vector space. 
As properly formulated in \cite{ESW}, it is shown that the dimensional reduction along $V \to V_\RR$ is equivalent to the theory whose fields are $\Omega^\bu(V_\RR) \otimes W$. 
In other words, the dimensional reduction of the holomorphic theory on $V$ is a topological theory on $V_\RR$. 

If we put $M$ back in, the result is similar. 
Suppose the original theory is of the form $\cE(M) \otimes \Omega^{0,\bu}(V) \otimes W$.
Then, the dimensional reduction along \eqref{eqn:dimred} is the theory whose space of fields is $\cE(M) \otimes \Omega^\bu(V_\RR) \otimes W$.

An explicit model for this reduction can be described as follows. 
Suppose $V \cong \CC^n$ and place the theory on $(\CC^\times)^{\times n} \subset \CC^n$. 
The dimensional reduction along $\CC^n \to \RR^n$ agrees with the compactification of the theory along $S^1 \times \cdots \times S^1$ where one throws away all nonzero winding modes around each circle.

\begin{prop}\label{prop:dimred}
The $SU(4)$ invariant twist of type IIA on $\CC^4 \times \RR^2$ is the dimensional reduction of the 11-dimensional theory along  
\[
\CC^4 \times \CC \times \RR_t \to \CC^4 \times \RR_x \times \RR_t \cong \CC^4 \times \RR^2 .
\]
\end{prop}
\begin{proof}
Let us denote the holomorphic coordinate we are reducing along by $z_5 = x + \im y$. 
We first read off the dimensional reduction of each component field of the 11-dimensional theory. 
Per the above discussion, this is obtained by taking all fields to be independent of $y$ and replacing $\d \zbar_5$ by $\d x$. 
To not confuse the notations of fields in 10 and 11 dimensions, we use the notation $\alpha_{11d}$ to denote an 11-dimensional field.

The reductions of the 11d fields $\nu_{11d}, \beta_{11d}$ are easy to describe. 
Recall that 
\[
\nu_{11d} \in \PV^{0,\bu}(\CC^5) \otimes \Omega^\bu(\RR) .
\]
The reduction of this field is a 10d $\nu$ field
\[
\nu (z_i,x,t) = \nu_{11d} (z_i, x, y=0, t) |_{\d \zbar_5 = \d x}  .
\]
Similarly, the reduction of $\beta_{11d}$ is a 10d $\beta$ field
\[
\beta (z_i,x,t) = \beta_{11d} (z_i, x, y=0, t) |_{\d \zbar_5 = \d x}  .
\]

The reduction of the 11d fields $\mu_{11d}$ and $\gamma_{11d}$ require a bit of massaging. 
We break the $SU(5)$ symmetry to $SU(4)$ to write
\[
\mu_{11d} = \mu^0_{11d} + \theta_{11d} \partial_{z_5} 
\]
where
\begin{align*}
\mu^0_{11d} & \in \PV^{1,\bu}(\CC^4) \otimes \Omega^{0,\bu}(\CC_{z_5}) \otimes \Omega^\bu(\RR_t) \\
\theta_{11d} & \in \Omega^{0,\bu}(\CC^4) \otimes \Omega^{0,\bu}(\CC_{z_5}) \otimes \Omega^\bu(\RR_t) .
\end{align*}
The dimensional reduction of $\mu^0_{11d}$ is a 10d $\mu$ field
\[
\mu(z_i,x,t) = \mu_{11d}^0 (z_i, x,y=0,t)|_{\d \zbar_5 = \d x} .
\]
The dimensional reduction of $\theta_{11d}$ is a $\theta$ field
\[
\theta(z_i,x,t) = \theta_{11d} (z_i, x,y=0,t)|_{\d \zbar_5 = \d x} .
\]

Finally, write the 11d field $\gamma_{11d}$ as
\[
\gamma_{11d} = \gamma_{11d}^0 + \eta_{11d} \d z_5
\]
where
\begin{align*}
\gamma^0_{11d} & \in \Omega^{1,\bu}(\CC^4) \otimes \Omega^{0,\bu}(\CC_{z_5}) \otimes \Omega^\bu(\RR_t) \\
\eta_{11d} & \in \PV^{0,\bu}(\CC^4) \otimes \Omega^{0,\bu}(\CC_{z_5}) \otimes \Omega^\bu(\RR_t) .
\end{align*}
The dimensional reduction of $\gamma^0_{11d}$ is a 10d $\gamma$ field
\[
\gamma(z_i,x,t) = \gamma_{11d}^0 (z_i, x,y=0,t)|_{\d \zbar_5 = \d x} .
\]
The dimensional reduction of $\eta_{11d}$ is an $\eta$ field
\[
\eta(z_i,x,t) = \eta_{11d} (z_i, x,y=0,t)|_{\d \zbar_5 = \d x} .
\]

Next, we read off the dimensional reduction of the 11d action. 
Let us first focus on the term present in BF theory which is
$\int^\Omega \frac{1}{1-\nu_{11d}} \mu_{11d}^2 \vee \del \gamma_{11d}$.
Upon reduction, this becomes 
\beqn\label{eqn:bfred}
\int^{\Omega_{\CC^4}}_{\CC^4 \times \RR^2} \frac{1}{1-\nu} \mu^2 \vee \del \gamma + \int^{\Omega_{\CC^4}}_{\CC^4 \times \RR^2} \frac{1}{1-\nu} (\theta \wedge \mu) \vee  \del \eta 
\eeqn

Next, consider the cubic term in the 11d action $J = \frac16 \int \gamma_{11d} \wedge \del \gamma_{11d} \wedge \del \gamma_{11d}$. 
Upon reduction, this becomes 
\beqn\label{eqn:jred}
\int_{\CC^4 \times \RR^2} \eta \wedge \del \gamma \wedge \del \gamma .
\eeqn

The sum of the action functionals \eqref{eqn:bfred} and \eqref{eqn:jred} does not precisely agree with the IIA action $\til I_{IIA}$. 
To relate the two actions we must make the following field redefinition:
\[
\til \theta = \frac{1}{1-\nu} \theta, \quad \til \eta = (1- \nu) \eta, \quad \til \beta = \beta + \frac{1}{1-\nu} \eta \wedge \theta .
\]
Notice that this change of coordinates is compatible with the odd symplectic pairing on the fields. 
Under this field redefinition the total dimensionally reduced action can be written as
\begin{multline}
\int^{\Omega_{\CC^4}}_{\CC^4 \times \RR^2} \frac{1}{1-\nu} \mu^2 \vee \del \gamma + \in^{\Omega_{\CC^4}}t_{\CC^4 \times \RR^2} \frac{1}{1-\nu} \til\eta \wedge \del \gamma \wedge \del \gamma + \int^{\Omega_{\CC^4}}_{\CC^4 \times \RR^2} (\til\theta \wedge \mu) \vee  \del \left(\frac{1}{1-\nu} \til\eta\right) \\ + \int_{\CC^4 \times \RR^2}^{\Omega_{\CC^4}} \frac{1}{1-\nu} (\til \eta \wedge \til \theta) \div \mu  .
\end{multline}
The first line comes from plugging in the new fields into the interactions \eqref{eqn:bfred} and \eqref{eqn:jred}.
The second line comes from plugging in the new fields into the kinetic term $\int \beta \div \mu$, which because of the non-linear change of coordinates now contributes to the interaction. 
We observe that the first two terms agree with the first and third terms in \eqref{eqn:IIAactionpot}. 

After integrating by parts, the remaining terms can be written as 
\[
- \int^{\Omega_{\CC^4}}_{\CC^4 \times \RR^2} \left(\frac{1}{1-\nu} \til\eta\right) \div (\til\theta \mu) + \int_{\CC^4 \times \RR^2}^{\Omega_{\CC^4}} \left(\frac{1}{1-\nu} \til \eta\right) \til \theta \div \mu .
\]
Applying the identity $\div (\til \theta \mu) = \til \theta \div \mu + \del (\til \theta) \vee \mu$, we see that this agrees exactly with the second term in \eqref{eqn:IIAactionpot}.
% but they are still not quite the same. 
%They are, however, cohomologous. 
%Integrating by parts and applying the BV relation, we can write the second line as
%\[
%\int^{\Omega_{\CC^4}}_{\CC^4 \times \RR^2} \left(\frac{1}{1-\nu} \til\eta\right)\wedge \div (\til\theta \wedge \mu)  = \int^{\Omega_{\CC^4}}_{\CC^4 \times \RR^2} \frac{1}{1-\nu} (\til\eta \wedge \mu) \vee \partial \til \theta + \int^{\Omega_{\CC^4}}_{\CC^4 \times \RR^2} \frac{1}{1-\nu} \til \eta \wedge \div \mu \wedge \til \theta .
%\]
%The first term in this equation agrees precisely with the second term in \eqref{eqn:IIAactionpot}. 
%The final term is cohomologically trivial via the odd Lagrangian $\int^\Omega \log(1-\nu) \wedge \til \eta \wedge \til \theta$.
%
%Let $\pi : \R\times \C^{\times}\times X \to \R^{2}\times X$ be the projection with fiber $S^{1}\subset \C^{\times}$. Note that there is an isomorphism \[\int: \pi^{*}(\Omega^{\bullet}(\R^{2})\otimes \cE_{pot})\to \cE\] given by \[(\eta, \mu, \nu, \beta,\gamma,\theta)\mapsto (\nu = \nu, \mu = \mu + \theta \vee \Omega_{X} \wedge\del_{z}, \beta = \beta, \gamma = \gamma\vee \Omega_{X} \eta dz ).\] It is clear that this isomorphism presrves the BV pairings; we need only check that $\int^{*} I$ is cohomologous to $I_{pot}$ in the deformation-obstruction complex of the free limit of $\Omega^{\bu}(\R^{2})\otimes \cE_{pot}$.
%
%We readily compute:
%\[\int^{*}I = \]
\end{proof}

\subsection{The twist of type I supergravity}

We now turn to a different type of redution of the 11-dimensional theory, this time involving type I supergravity. 
We begin by briefly recalling the description of type I supergravity following \cite{CLtypeI} which was motivated by the unoriented $B$-model. 
In \cite{SWspinor}, the second two authors verified the conjectural description of the space of fields recalled below using the pure spinor formalism. 
Unlike type IIA supergravity, there only exists an $SU(5)$ invariant twist of type I supergravity and it is holomorphic in the maximal number of dimensions.

Concretely, the space of fields of the $SU(5)$ twist of type I supergravity is a subspace of minimal Kodaira--Spencer theory on $\CC^5$. 
The $\ZZ/2$ graded space of field equipped with its linear BRST operator is 
\beqn\label{eqn:IIApot}
\begin{tikzcd}
- & + & - & +  \\ \hline
{\PV^{1,\bu}}(\CC^5) \arrow[r, "u\div"]    & {u\PV^{0,\bu}}(\CC^5) \\
{\PV^{3,\bu}} (\CC^5)\arrow[r, "u\div"] & {u\PV^{2,\bu}} (\CC^5)\arrow[r, "u\div"] & {u^2\PV^{1,\bu}}(\CC^5) \arrow[r, "u\div"] & {u^3\PV^{0,\bu}}(\CC^5)
\end{tikzcd}.
\eeqn

Let us give a description of the classical action. 
Introduce notations for the fields of this type I model:
\beqn\label{eqn:Ifields}
\mu + u \nu \in \PV^{1, \bu}(\CC^5) \oplus u \PV^{0,\bu} (\CC^5), \quad \sigma \in \PV^{3,\bu}(\CC^5) .
\eeqn
We will not need an explicit notation for the remaining descendant fields. 

\begin{conj}
The twist of type I supergravity on $\CC^5$ is the $\Z/2$-graded theory with fields $\mu+u\nu, \sigma$ as above and with classical action
\beqn\label{eqn:typeIaction}
I_{{\rm type\, I}} = {\rm Tr}_{\CC^5} \frac{1}{1-\nu} \mu^2 \vee \sigma + \cdots
\eeqn
where the $\cdots$ stands for terms involving the higher descendant fields. 
%The $L_{\infty}$ structure is the natural one on the tensor product of the cdga $\Omega^{\bu}(M)$ with the $L_{\infty}$-algebra $\cE_{mKS}$.
\end{conj}

\parsec[s:typeIpot]

Like in the type IIA discussion, there is a slight modification of the type I model above which is most directly related to 11-dimensional supergravity. 

This modification involves replacing the field $\sigma$ above by a potential $\til \gamma \in \Omega^{1,\bu}(\CC^5)$ which satisfies $\Omega \vee \sigma = \del \til \gamma$. 
This condition does not fix $\til \gamma$ uniquely, there is a gauge symmetry of the form $\til \gamma \to \til \gamma + \del \til \beta$. 

In detail, this potential theory we are considering has underlying cochain complex of fields
\begin{equation}
  \label{eq:Ipot} 
  \begin{tikzcd}[row sep = 1 ex]
    - & + & -\\ \hline
    \PV^{1,\bu}(\CC^5)_\mu \ar[r, "\div"] & \PV^{0,\bu}(\CC^5)_\nu  \\
         & \Omega^{0,\bu}(\CC^5)_{\til\beta} \ar[r, "\del"] & \Omega^{1,\bu}(\CC^5)_{\til\gamma} .
\end{tikzcd}
\end{equation} 
This space of fields is equipped with an odd non-degenerate pairing.
Like the eleven-dimensional theory, it is a classical BV theory in the $\ZZ/2$-graded sense. 

The type I action \eqref{eqn:typeIaction} completely determines the action of this theory with potentials. 
One simply takes the action and replaces all appearances of $\sigma$ with $\Omega^{-1} \vee \del \til\gamma$. 
This yields the interaction of the potential theory
\beqn\label{eqn:Iactionpot}
\til I_{\text{type I}} = \frac12 \int^\Omega_{\CC^5} \frac{1}{1-\nu} \mu^2 \vee \del \til\gamma .
\eeqn
Notice that the terms involving higher descendants vanishes since these fields are set to zero in the potential theory.

\subsection{Slab compactification}\label{s:Ired}

We consider placing twisted 11-dimensional supergravity on the manifold $\CC^5 \times [0,1]$. 
In order to do this, we must choose appropriate boundary conditions at $t=0$ and $t=1$.
Our 11-dimensional theory on such manifolds fits nicely into the formalism of \cite{BY,Eugene} in that it is topological in the direction transverse to the boundary.  

The phase space of the theory at $t=0$ or $t=1$ is 
\begin{equation}
  \label{eq:lin1} 
  \begin{tikzcd}[row sep = 1 ex]
    - & + \\ \hline
    \PV^{1,\bu}(\CC^5)_\mu \ar[r, "\div"] & \PV^{0,\bu}(\CC^5)_\nu \\ 
     \Omega^{0,\bu}(\CC^5)_\beta \ar[r, "\del"] & \Omega^{1,\bu}(\CC^5)_\gamma.
\end{tikzcd}
\end{equation}
The wedge an integrate pairing between the top and bottom lines induces an {\em even} symplectic structure on the phase space. 
Denote this phase space by $\cE_{\del}$ for the moment.

The phase space is equipped with the restriction of the linear BRST operator of the full 11-dimensional theory. 
There is also a non linear BRST operator, just like in the bulk theory.
The BV action induces a $L_\infty$ structure on the parity shift $\Pi\cE_{\del}$ whose cohomology is still a trivial central extension of $E(5,10)$. 

A boundary condition is given by a Lagrangian subspace of $\cE_{\del}$ with respect to this even symplectic structure. 
To make sense of the theory on $\CC^5 \times [0,1]$ we must make the choice of two separate boundary conditions 
\[
\cM_{t=0} , \cM_{t=1} \subset \cE_{\del} .
\]
Moreover, these boundary conditions carry non linear BRST operators endowing their parity shifts $\Pi \cM_{t=0} , \Pi\cM_{t=1}$ with the structures of $L_\infty$ algebras. 
These $L_\infty$ structures must be compatible with the one on the phase space.
In fact, in our context these boundary conditions are abstractly isomorphic. 
We will explain the explicit boundary conditions momentarily. 

An important thing to note is that the fields of the theory compactified on the slab is computed by the {\em derived} intersection of the two Lagrangians:
\[
\cM_{t=0} \overset{\LL}{\underset{\cE_{\del}}{\times}}\cM_{t=1}.
\]
To compute this derived intersection we must suitably resolve the boundary conditions.


\parsec[s:boundary]
At $t=0$, the boundary condition of the 11-dimensional theory is determined by declaring 
\[
\cM_{t=0}: \quad \gamma|_{t=0} = \beta|_{t=0} = 0 .
\]
We will place the theory on $\CC^5 \times [0,1]$ by imposing the same boundary condition at $t=1$:
\[
\cM_{t=1}: \quad \gamma|_{t=1} = \beta|_{t=1} = 0 .
\]

\begin{prop}
With these boundary conditions for the classical 11-dimensional theory on $\CC^5 \times [0,1]$, the dimensional reduction along 
\[
\CC^5 \times [0,1] \to \CC^5
\]
is equivalent to the twist of type I supergravity on $\CC^5$. 
\end{prop}
\begin{proof}
Notice that both $\cM_{t=0}$ and $\cM_{t=0}$ are abstractly isomorphic to the complex resolving divergence-free vector fields
\begin{equation}
  \label{eq:lin2} 
  \begin{tikzcd}[row sep = 1 ex]
    - & + \\ \hline
    \PV^{1,\bu}(\CC^5)_\mu \ar[r, "\div"] & \PV^{0,\bu}(\CC^5)_\nu  .
\end{tikzcd}
\end{equation}

To compute the derived intersection between the two Lagrangians at $t=0$ and $t=1$ we replace the Lagrangian morphism $\cM_{t=0} \hookrightarrow \cE_{\del}$. 
Consider the cochain complex $\til \cM_{t=0}$ defined by
\begin{equation}
  \label{eq:lin3} 
  \begin{tikzcd}[row sep = 1 ex]
    - & + & - \\ \hline
    \PV^{1,\bu}(\CC^5)_\mu \ar[r, "\div"] & \PV^{0,\bu}(\CC^5)_\nu \\ 
     \Omega^{0,\bu}(\CC^5)_\beta \ar[dr,dotted,"\id"]\ar[r, "\del"] & \Omega^{1,\bu}(\CC^5)_\gamma \ar[dr,dotted,"\id"] \\
     & \Omega^{0,\bu}(\CC^5)_{\til\beta} \ar[r, "\del"] & \Omega^{1,\bu}(\CC^5)_{\til\gamma} .
\end{tikzcd}
\end{equation}
Notice that as a graded vector space, this complex is of the form $\cE_{\del} \oplus (\Omega^{0,\bu} \oplus \Pi \Omega^{1,\bu})$. 
The $L_\infty$ structure on $\Pi \til \cM_{t=0}$ extends the one on $\cE_{\del}$ coming from the bulk BV action. 
Notice that the obvious embedding $\cM_{t=0} \hookrightarrow \til \cM_{t=0}$ is a quasi-isomorphism.

The projection map $\til \cM_{t=0} \twoheadrightarrow \cE_{\del}$ factors the original Lagrangian inclusion as
\[
\cM_{t=0} \hookrightarrow \til \cM_{t=0} \twoheadrightarrow \cE_{\del} .
\]
To compute the derived intersection of $\cM_{t=0}$ and $\cM_{t=1}$ we can compute the ordinary intersection of $\til \cM_{t=0}$ and $\cM_{t=1}$. 

Let $\mu_{t=1}$ and $\nu_{t=1}$ denote the fields present in the other boundary condition $\cM_{t=1}$. 
The intersection $\til \cM_{t=0} \times_{\cE_{\del}} \cM_{t=1}$ is computed by setting the fields $\beta, \gamma$ to zero and $\mu=\mu_{t=1}$, $\nu = \nu_{t=1}$. 
Thus, we are left with
\begin{equation}
  \label{eq:lin3} 
  \begin{tikzcd}[row sep = 1 ex]
    - & + & -\\ \hline
    \PV^{1,\bu}(\CC^5)_\mu \ar[r, "\div"] & \PV^{0,\bu}(\CC^5)_\nu  \\
         & \Omega^{0,\bu}(\CC^5)_{\til\beta} \ar[r, "\del"] & \Omega^{1,\bu}(\CC^5)_{\til\gamma} 
\end{tikzcd}
\end{equation}
This is precisely the underlying cochain complex of fields for the type I model with potentials. 
The odd non-degenerate pairing on this complex agrees with the one on this particular potential theory for the twist of type I supergravity. 
The $L_\infty$ structure on the parity shift of this complex is compatible with the one induced from the BV action in \eqref{eqn:Iactionpot}.
\end{proof}

\subsection{The $SU(5)$ twist of type IIA supergravity}
\label{s:su5IIA}

%\brian{Ingmar can you argue why this is the SU(5) twist on susy grounds?}

Thus, given that our 11-dimensional theory correctly describes the $SU(5)$-invariant twist of supergravity on $\CC^5 \times \RR$, to obtain the $SU(5)$ twist of type IIA supergravity we should reduce along the topological $\RR$ direction. 
This results in a $SU(5)$ invariant, holomorphic, theory on $\CC^5$. 

Let us briefly spell out the fields present in this dimensional reduction. 
The reduction is obtained by replacing $\Omega^\bu(\RR)$ with its translation invariant subalgebra $\CC[\ep] = \CC[\d t]$. 
Here, $\ep$ is an odd parameter playing the role of the translation invariant one-form $\d t \in \Omega^1(\RR)$. 
Equivalently, we are compactifying the theory along 
\[
\CC^5 \times S^1 \to \CC^5 .
\]

The 11-dimensional the field $\mu_{11d}$ is replaced by the field 
\[
\mu + \ep \mu' \in \Pi \PV^{1,\bu}(\CC^5) [\ep] .
\]
Notice that the lowest component of $\mu$ is odd (just like $\mu_{11d})$, but the lowest component of $\mu'$ is now even. 
Completely similarly, the remaining fields reduce as $\nu + \ep \nu'$, $\gamma + \ep \gamma'$, and $\beta + \ep \beta'$. 

In summary, the linear complex of fields of the dimensionally reduced theory on $\CC^5$ is
\begin{equation}
  \label{eqn:IIAsu5} 
  \begin{tikzcd}[row sep = 1 ex]
    {\rm odd} & {\rm even} & {\rm even} & {\rm odd} \\ \hline
    \PV^{1,\bu}(\CC^5)_{\mu} \ar[r, "\del"] & \PV^{0,\bu}(\CC^5)_\nu & \\ 
     & \ep\Omega^{0,\bu}(\CC^5)_{\beta'} \ar[r, "\div"] & \ep\Omega^{1,\bu}(\CC^5)_{\gamma'} . \\
     &  \ep \PV^{1,\bu}(\CC^5)_{\mu'} \ar[r, "\div"] & \ep \PV^{0,\bu}(\CC^5)_{\nu'} \\
     & & \Omega^{0,\bu}(\CC^5)_\beta \ar[r, "\div"] & \Omega^{1,\bu}(\CC^5)_\gamma.
\end{tikzcd}
\end{equation}

We can compute the dimensional reduction of the 11-dimensional action $S_{BF,\infty} + J$ in a similar way to how we have done in the past few sections. 
We arrive at the action functional described below. 

\begin{conj}
\label{conj:IIAsu5}
The $SU(5)$ twist of type IIA supergravity on $\CC^5$ is equivalent to the theory whose linear BRST complex of fields is displayed in \eqref{eqn:IIAsu5}. 
The full action functional is 
\begin{multline}
\label{eqn:su5action}
\int^\Omega_{\CC^5}\bigg(\beta' \wedge \dbar \nu + \beta \wedge \dbar \nu' + \gamma' \wedge \dbar \mu + \gamma \wedge \dbar \mu' +  \beta' \wedge \div \mu + \beta \wedge \div \mu' \bigg) \\
+ \int^\Omega_{\CC^5} \bigg( \frac12 \frac{1}{1-\nu} \mu^2 \vee \del \gamma' +  \frac{1}{1-\nu} (\mu \wedge \mu') \vee \del \gamma' + \frac12 \frac{\nu'}{(1-\nu)^2} \mu^2 \vee \del \gamma \bigg) \\
+ \frac12 \int_{\CC^5} \gamma' \wedge \del \gamma \wedge \del \gamma .
\end{multline} 
\end{conj} 

The first two lines in \eqref{eqn:su5action} arise from the reduction of the BF action $S_{BF,\infty}$. 
The final line arises from the reduction of $J = \frac16 \int \gamma_{11d} \del \gamma_{11d} \del \gamma_{11d}$. 

%\parsec[]
%
%
%We obtain further evidence that this is the $SU(5)$ invariant twist of type IIA supergravity by checking that it has the expected residual supersymmetries. 
%The type IIA superstring algebra $\lie{string}_{IIA}$ is a central extension of the 10d $\cN=(1,1)$ super Poincar\'e algebra, for a definition we refer to \cite{BH,FSS}. 
%
%A minimal twisting supercharge $Q_{hol}$ in the $\cN=(1,1)$ super Poincar\'e algebra determines a maximal isotropic $L \subset \CC^{10}$. 
%In a completely analogous way to the calculation for the twist of the algebra $\m2$ or the $SU(4)$ twist of the as in \cite{CLsugra} one can prove the following. 
%
%\begin{prop}
%As an $\lie{sl}(5)$ representation the $Q_{hol}$-cohomology of $\lie{string}_{IIA}$ is equivalent
%\end{prop}

\parsec[]\label{s:orbifold}

The slab compactification of the previous section was one way to implement the $S^1 / \ZZ/2$ reduction of the 11-dimensional theory. 
We offer another point of view of this $S^1 / \ZZ/2$ reduction. 

First off, there is the following $\ZZ/2$ action on the 11-dimensional theory on $\CC^5 \times S^1$ before compactifying. 
We obtain it by the following tensor product of $\ZZ/2$ actions. 
First, $\ZZ/2$ acts on $\Omega^\bu(S^1)$ by orientation reversing diffeomorphisms. 
Second, we declare that the eigenvalue of the $\ZZ/2$ action on $\PV^{k,\bu}(\CC^5)$, for $k=0,1$ is $+1$ and the eigenvalue of the $\ZZ/2$ action on $\Omega^{k,\bu}(\CC^5)$ for $k=0,1$ is $-1$. 
This determines a $\ZZ/2$ action on the full space of fields of the 11-dimensional theory. 

Upon $S^1$ compactification the $\ZZ/2$ action is easy to describe: $\mu,\nu$ both have eigenvalue $+1$, $\mu',\nu'$ both have eigenvalue $-1$, $\gamma,\beta$ both have eigenvalue $-1$, and $\gamma',\beta'$ both have eigenvalue $+1$. 
In particular, we see that the $\ZZ/2$ fixed points simply pick out the $\mu, \nu, \gamma', \beta'$ fields; this comprises the first two lines of \eqref{eqn:IIAsu5}. 

The fields match precisely with the fields in the twist of type I supergravity that we recalled in \S \ref{s:typeIpot} (Under the relabeling $\gamma' \leftrightarrow \til \gamma, \beta' \leftrightarrow \til \beta$). 
Furthermore, restricting the action in the above conjecture agrees precisely with the action of this twisted type I model. 

\subsection{Compactification along a CY3}\label{s:CY3}

In the first section we saw that the 11-dimensional theory can be defined on any manifold that is a product of a Calabi--Yau five-fold with a smooth oriented one-manifold. 
In this section, we investigate an important compactification of the 11-dimensional theory which involves the Calabi--Yau manifold $X \times \CC^2$ where $X$ is a simply connected compact Calabi--Yau three-fold.

The compactification of the theory along the three-fold $X$ 
\[
X \times \CC^2 \times \RR \to \CC^2 \times \RR
\]
yields an effective five-dimensional theory which is holomorphic along $\CC^2$ and topological along $\RR$. 
Upon compactification, we will find a match with a description of the twist of five-dimensional minimally supersymmetric supergravity. 

\begin{prop}
\label{prop:5dsugra}
The compactification of the 11-dimensional theory along a Calabi--Yau three-fold $X$ is equivalent to the twist of 5d $\cN=1$ supergravity with $h^{1,1}(X)-1$ vector multiplets and $h^{1,2}(X) + 1$ hyper multiplets. 
\end{prop}

\parsec[s:5dsugra]

We give a conjectural capitulation of the twist of 5d $\cN=1$ supergravity. 
Before twisting, a general 5d $\cN=1$ supergravity contains a gravity multiplet coupled to some number of vector and hyper multiplets. 
The twist of the vector and hyper multiplet has been computed in \cite{ESW}, and we recall it below. 
The twist of the gravity multiplet is less clear. 
A thorough computation of the twist has yet to appear, though some checks have been established by Elliott and the last author in \cite{EWpoisson}. 
We give a description of the twist now, but leave a detailed computation from first principles to future work. 

The gravity multiplet, see \cite{CCDF} for instance, consists of a graviton $e$, a gravitino $\psi$, and a one-form gauge field $\cA_{grav}$.
After twisting, the graviton and components of the gravitino decompose into two Dolbeault-de Rham valued fields 
\[
\alpha, \eta \in \Pi \Omega^{0,\bu}(\CC^2) \otimes \Omega^\bu(\RR) ,
\]
whose lowest components both carry odd parity. 
The one-form gauge field $\cA_{grav}$ and the remaining components of the gravitino decompose into two more Dolbeault-de Rham valued fields
\[
A_{grav} , B_{grav} \in \Pi \Omega^{0,\bu}(\CC^2) \otimes \Omega^\bu(\RR) ,
\]
whose lowest components also both carry odd parity. 

\begin{conj}
\label{conj:5dsugra}
The twist of 5d supergravity (with nonzero Chern--Simons term) with vector multiplets valued in a Lie algebra $\fg$ and hypermultiplets valued in a representation $V$ has BV fields
\begin{itemize}
\item $\alpha, A_{grav} \in \Pi \Omega^{0,\bu}(\CC^2) \otimes \Omega^\bu(\RR)$ with conjugate BV fields $\eta, B_{grav}$,
\item $A \in \Pi \Omega^{0,\bu}(\CC^2) \otimes \Omega^\bu(\RR) \otimes \fg$ with conjugate BV field $B$,
\item $\chi \in \Omega^{0,\bu}(\CC^2) \otimes \Omega^\bu(\RR) \otimes V$ with conjugate BV field $\psi$.
\end{itemize}

The action is
\begin{multline}
\label{eqn:5daction} 
\int^\Omega_{\CC^2 \times \RR} \left(\eta \dbar \alpha + B_{grav} \dbar A_{grav} + B \dbar A + \psi \dbar \chi \right) \\
  + \int^\Omega_{\CC^2 \times \RR} \left( \frac12\eta \{\alpha, \alpha\} +  B_{grav} \{\alpha, A_{grav}\}+ B \{\alpha, A\} +  \psi \{\alpha, \chi \}\right) \\ 
+ \frac16 \int_{\CC^2 \times \RR} B_{grav} \del B_{grav} \del B_{grav} .
\end{multline}
\end{conj}

%\brian{general background on susy. Why expect 5d $\cN=1$ susy}

\parsec[-]

With this description of the twist of five-dimensional supergravity, we turn to the proof of Proposition \ref{prop:5dsugra}. 


First, we set up some notation. 
Let $\Omega_X$ be the holomorphic volume form on $X$. 
To define the $11$-dimensional theory on $X \times \CC^2 \times \RR$ we use the Calabi--Yau form $\Omega_X \wedge \d z_1 \wedge \d z_2$, where $\{z_i\}$ is a holomorphic coordinate on $\CC^2$. 
Let $\omega \in \Omega^{1,1}(X)$ be a fixed K\"ahler form on $X$.
For any $k$, let $H^k(X, \Omega^k_X)_\perp$ denote the cohomology of the primitive elements. 

\begin{proof}
Consider the 11-dimensional field $\nu_{11d}$. 
Under the equivalence
\begin{align*}
\PV^{0,\bu}(X \times \CC^2) \otimes \Omega^\bu(\RR) & \simeq H^{\bu}(X, \cO) \otimes \PV^{0,\bu}(\CC^2) \otimes \Omega^\bu(\RR) \\ & = \PV^{0,\bu}(\CC^2) \otimes \Omega^\bu(\RR) \oplus \Pi \Bar{\Omega}_X \PV^{0,\bu}(\CC^2) \otimes \Omega^\bu(\RR) 
\end{align*}
the $\nu_{11d}$ field decomposes as 
\[ 
\nu_{11d} = \nu + \Bar{\Omega}_X \til \nu .
\]
Here $\Bar{\Omega}_X$ is the complex conjugate to the holomorphic volume form on $X$. 
Notice that the zero form component of $\til \nu$ is a field with even parity. 

Next, consider the 11-dimensional field $\mu_{11d}$. 
Under the equivalence 
\begin{align*}
\Pi \PV^{1,\bu}(X \times \CC^2) \otimes \Omega^\bu(\RR) & \simeq \Pi H^{\bu}(X, \cO) \otimes \PV^{1,\bu}(\CC^2) \otimes \Omega^\bu(\RR) \\ & \oplus \Pi H^\bu(X, \T_X) \otimes \PV^{0,\bu}(\CC^2) \otimes \Omega^\bu(\RR) \\ & = \Pi \PV^{1,\bu}(\CC^2) \otimes \Omega^\bu(\RR) \oplus \Bar{\Omega}_X \PV^{1,\bu}(\CC^2) \otimes \Omega^\bu(\RR) \\ & \oplus H^{1}(X, \T_X) \otimes \PV^{0,\bu}(\CC^2) \otimes \Omega^\bu(\RR) \oplus \Pi H^{2}(X, \T_X) \otimes \PV^{0,\bu}(\CC^2) \otimes \Omega^\bu(\RR)
\end{align*}
the field $\mu_{11d}$ decomposes as 
\begin{align*}
\mu_{11d} & = \mu + \Bar{\Omega}_X \til \mu \\ 
& + e^i \chi_i + f^a A_a +  (\Omega_X^{-1} \vee \omega^2)A_{grav} .  
\end{align*} 
Here, $\{e^i\}_{i=1,\ldots, h^{2,1}}$ is a basis for $H^{1}(X, \T_X)$ and $\{f^a\}_{a=1,\ldots, h^{1,1}-1}$ is a basis for 
\[
H^2 (X, \Omega^2_X)_\perp \subset H^2(X, \Omega^2_X) \cong H^2(X, \T_X) . 
\]
Notice that the zero form part of $\til{\mu}$ is an even field, the zero form part of $\chi_i$ is an even field, the zero form part of $A_a$ is an odd field, and the zero form part of $\mu_\omega$ is an odd field. 

The decomposition for the 11d fields $\gamma_{11d}$ and $\beta_{11d}$ is similar. 
We record it here:
\begin{align*}
\beta_{11d} & = \beta + \Bar{\Omega}_X \til \beta \\ 
\gamma_{11d} & = \gamma + \Bar{\Omega}_X \til \gamma + e_i \psi^i + f_a B^a + \omega \wedge B_{grav}  . 
\end{align*}
Here, $\{e_i\}_{i=1,\ldots,h^{2,1}}$ is a basis for $H^2 (X, \Omega^1_X)$ dual to the basis $\{e^i\}$ under the Serre pairing.
Also, $\{f_a\}_{a=1,\ldots,h^{1,1}-1}$ is a basis for $H^{1}(X, \Omega^1_X)_\perp$ dual to the basis $\{f^a\}$. 

To compare most directly to the description of the twist of 5d $\cN=1$ supergravity we modestly modify the fields. 
Let $\del$ be the holomorphic de Rham differential along $\CC^2$. 
First, we introduce a potential for the fields $\mu$ and $\til \mu$. 
Let 
\[
\alpha, \chi \in \Omega^{0,\bu}(\CC^2) \otimes \Omega^\bu(\RR)
\]
be differential forms satisfying $\del \alpha = \mu \vee \Omega_{\CC^2}$ and $\del \chi = \til \mu \vee \Omega_{\CC^2}$. 
The fields $\nu, \til \nu$ are set to zero. 
Dually, we replace the fields $\gamma, \til \gamma$ their `field strengths', suitably renormalized with respect to the volume form
\[
\eta = (\d^2 z)^{-1} \vee \del \til \gamma , \quad \psi = (\d^2 z)^{-1} \vee \del \gamma \in \Omega^{0,\bu}(\CC^2) \otimes \Omega^\bu(\RR) .
\]
The roles of $\beta, \til \beta$ were as gauge symmetries implementing $\gamma \to \gamma + \del \beta$ and $\til \gamma \to \til \gamma + \del \til \beta$. 
Since we are replacing $\gamma, \til \gamma$ by their images under the operator $\del$, these gauge symmetries are set to zero. 

In summary, we are left with the following fields 
\[
\begin{array}{cccccccccc}
\alpha,A_{grav} & \in & \Pi \Omega^{0,\bu}(\CC^2) \otimes \Omega^\bu(\RR), & \eta, B_{grav} & \in & \Pi \Omega^{0,\bu}(\CC^2) \otimes \Omega^\bu(\RR) \\
\chi, \chi_i & \in & \Omega^{0,\bu}(\CC^2) \otimes \Omega^\bu(\RR),  & \psi, \psi^i & \in & \Omega^{0,\bu}(\CC^2) \otimes \Omega^\bu(\RR), & i=1,\ldots, h^{2,1}  \\
A_a & \in & \Pi \Omega^{0,\bu}(\CC^2) \otimes \Omega^\bu(\RR), & B^a & \in & \Pi \Omega^{0,\bu}(\CC^2) \otimes \Omega^\bu(\RR) , & a = 1, \ldots, h^{1,1}-1.
\end{array}
\]

Let us plug these fields in to the 11-dimensional action.
First, consider the BF term $\frac12 \int^{\Omega} \frac{1}{1-\nu_{11d}} \mu_{11d}^2 \gamma_{11d}$. 
With the field redefinitions above, this decomposes as
\begin{multline}\label{eqn:5dsugra1}
 \int_{\CC^2 \times \RR}^\Omega \left( \frac12 \del \alpha \wedge \del \alpha \wedge \eta + \del A_{grav} \wedge \del A_{grav} \wedge B_{grav} \right) \\
 + \int_{\CC^2 \times \RR}^\Omega \left( \del \alpha \wedge \del \chi \wedge \psi + \del \alpha \wedge \del \chi_i \wedge \psi^i + \del \alpha \wedge \del A_a \wedge B^a \right) .
\end{multline}
This term agrees with the second line in the 5d action \eqref{eqn:5daction}. 

Finally, consider the term in the 11-dimensional action $J(\gamma_{11d}) = \frac16 \int \gamma_{11d} \wedge \del \gamma_{11d} \wedge \del \gamma_{11d}$. 
This induces the five-dimensional Chern--Simons term 
\beqn\label{eqn:5dsugra2}
\frac16 \int_{\CC^2 \times \RR} B_{grav} \del B_{grav} \del B_{grav} .
\eeqn
This completes the proof. 
\end{proof}

\parsec[s:5dglobal]

In \S \ref{sec:global} we computed the global symmetry algebra of the 11-dimensional theory on $\CC^5 \times \RR$ and found a close relationship to the exceptional super Lie algebra $E(5,10)$. 
In this section we deduce the form of the global symmetry algebra of the five-dimensional compactified theory on $\CC^2 \times \RR$. 

Consider the full de Rham cohomology of $X$ by
\[
H^\bu (X, \Omega^\bu) = \oplus_{i,j} H^i (X, \Omega^j_X)  .
\]
This is a graded commutative algebra using the wedge product of differential forms. 
Next, consider the space of holomorphic functions $\cO(\CC^2)$ on $\CC^2$. 
The Poisson bracket $\{-,-,\}$ associated to the standard holomorphic symplectic structure on $\CC^2$ endows $\cO(\CC^2)$ with the structure of a Lie algebra. 
In particular, we can tensor $\cO(\CC^2)$ with $H^\bu (X, \Omega^\bu)$ to obtain the structure of a graded Lie algebra on 
\[
H^\bu (X, \Omega^\bu) \otimes \cO(\CC^2) .
\]
Let $[\omega] \in H^1(X, \Omega^1_X)$ be the class of the K\"ahler form on $X$.

The global symmetry algebra of the compactified theory along the Calabi--Yau three-fold $X$ is equivalent to a deformation of this graded Lie algebra. 
The deformation introduces the following Lie bracket 
\[
\big[ [\omega] \otimes f, [\omega] \otimes g\big] = [\omega^2] \otimes \{f,g\} \in H^{2,2}(X, \Omega^2) \otimes \cO(\CC^2) . 
\]

%\begin{proof}
%The computation of the global symmetry algebra of the 5-dimensional theory is very similar to the computation in the 11-dimensional theory.
%In terms of the de Rham cohomology, the generators of the linearized cohomology of the space of fields read:
%\begin{itemize}
%\item a holomorphic function $\alpha(z_1,z_2) \in H^0(X, \Omega^0) \otimes \cO(\CC^2)$.
%\item a 
%
%First, let's compute the global symmetry algebra of the twist of the pure gravity theory on $\CC^2 \times \RR$. 
%This arises from terms in the BV action \eqref{eqn:5daction} involving only $\alpha, \eta, A_{grav}$, and $B_{grav}$. 
%
%The underlying graded vector space of the resulting Lie algebra is equivalent to
%\[
%\cO(\CC^2)_\alpha \oplus \cO(\CC^2)_\eta \oplus \cO(\CC^2)_{A_{grav}} \oplus \cO(\CC^2)_{B_{grav}} .
%\]
%The non trivial Lie brackets are
%\begin{multline}
%[\alpha, \alpha'] = \{\alpha, \alpha'\} \in \cO(\CC^2)_\alpha , \quad [\alpha, \eta] = \{\alpha, \eta\} \in \cO(\CC^2)_\eta , \quad [\alpha, A_{grav}] = \{\alpha, A_{grav}\} \in \cO(\CC^2)_{A_{grav}} \\ 
%[\alpha, B_{grav}] = \{\alpha, B_{grav}\} \in \cO(\CC^2)_{B_{grav}} , \quad [B_{grav}, B'_{grav}] = \{B_{grav}, B'_{grav}\} \in \cO(\CC^2)_{A_{grav}} .
%\end{multline}
%\begin{align*}
%[\alpha, \alpha' + \eta + A_{grav} + B_{grav}] & = \{\alpha,  \alpha'\} + \{\alpha, \eta\} + \{\alpha, A_{grav}\} + \{\alpha, B_{grav}\} \\ & \in \cO(\CC^2)_\alpha \oplus \cO(\CC^2)_\eta \oplus \cO(\CC^2)_{A_{grav}} \oplus \cO(\CC^2)_{B_{grav}} \\
%[B_{grav}, B'_{grav}] & = \{B_{grav}, B'_{grav}\} \in \cO(\CC^2)_{A_{grav}} .
%\end{align*}
%Denote this Lie algebra by $\fg_{grav}$. 
%
%The contribution of the vector and hypermultiplets form a module for $\fg_{grav}$. 
%As a graded vector space, this module is
%\[
%M_X = \cO(\CC^2) \otimes H^{1}(X, \Omega^1_X)_\perp [-1] \oplus \cO(\CC^2) \otimes 
%\]
%\brian{incomplete}







\subsection{The theory on a three-fold}

%count hypers and vectors in 5d N=1 sugra relate to Kevin and my work with Chris

\section{The non-minimal twist} 

We have provided numerous consistency checks that the 11-dimensional theory defined on $\CC^5 \times \RR$ is a twist of supergravity. 
We have referred to this theory as holomorphic as it depends on the complex structure in the maximal number of directions. 
In this section we characterize a further twist of 11-dimensional supergravity from the lens of the holomorphic theory. 

On flat space, the further twist is essentially unique and renders the theory topological in seven directions, rather than just one in the holomorphic twist. 
We will show that it is equivalent to a theory on $\CC^2 \times \RR^7$ that we call ``Poisson'' Chern--Simons theory. 
In the BV formalism, the theory $\ZZ/2$ graded and has fields given by
\[
A \in \Pi \Omega^{0,\bu}(\CC^2) \hotimes \Omega^\bu(\RR^7) ,
\]
where $\Pi$, as always, denotes parity shift.
The equations of motion are of the form
\[
\dbar A + \d_{\RR^7} A + \partial_{z} A \wedge \partial_{w} A = 0 .
\]
The action functional depends on the holomorphic symplectic structure on $\CC^2$ through the Poisson bracket on the algebra of holomorphic functions.
We give a precise definition below. 

More generally, the twist can be defined \brian{finish}


\subsection{The non-minimal supercharge}

Let $Q^{hol}$ denote the fixed holomorphic supercharge. 
We have argued in \S \ref{s:residual} that the $Q^{hol}$-cohomology of the 11-dimensional supersymmetry algebra is a symmetry of our proposed holomorphic twist of 11-dimensional supergravity. 

There is a twist of supergravity which can be witnessed as a further deformation of the $Q^{hol}$-twist. 
\brian{fix this}

Recall that the odd component of this residual supersymmetry algebra is
\[
\wedge^2 L \cong \wedge^2(\CC^5) 
\]
which we can identify with constant coefficient holomorphic $(2,0)$ forms on $\CC^5 \times \RR$ (they are zero-forms along the $\RR$-direction). 
Because these forms are constant coefficient, they are also closed, and hence lift to $(1,0)$ forms on $\CC^5 \times \RR$ in a way that is unique up to $\del$-exact terms. 

Using this observation, we constructed an embedding of $\wedge^2(L)$ into the fields of supergravity. 
Explicitly, the element $z_i \wedge z_j \in \wedge^2 L$ was mapped to the $\gamma$-type field
\[
\Psi_{ij} = \frac12 (z_i \d z_j - z_j \d z_i) \in \Omega^{1,0}(\CC^5) \hotimes \Omega^0(\RR) ,
\]
where $i,j=1,\ldots, 5$ and $i \ne j$.  

\parsec[-] 

The first main result of this section is to characterize the twist of our 11-dimensional theory on $\CC^5 \times \RR$ by the element $\Psi_{ij}$. 

\begin{thm}
The $\Psi_{ij}$-twist of the 11-dimensional theory is equivalent to Poisson Chern--Simons theory on 
\[
\CC_{z_i} \times \CC_{z_j} \times \RR^7 .
\]
\end{thm}

Notice that changing the values of $i,j$ just has the affect of permuting the holomorphic copies of $\CC^2$ leftover in the further twist. 

\parsec[sec:nmsymmetry]
Before proceeding to the proof of the theorem above, we perform a very simple calculation of the global symmetry algebra present in the $\Psi_{ij}$-twisted theory. 

We start with the global symmetry algebra $E(5,10)$ of the holomorphic twist of the 11-dimensional theory. 
From this point of view, the global symmetry algebra of the $\Psi_{ij}$-twisted theory is given by deformation of this super Lie algebra by the Maurer--Cartan element 
\[
\partial \Psi_{ij} = \d z_i \wedge \d z_j \in \Omega^{2,hol}_{cl}(\CC^5) .
\]
We recall that the space of closed two-forms on $\CC^5$ is precisely the odd part of the super Lie algebra $E(5,10)$. 

Notice that there are the following brackets in the super Lie algebra $E(5,10)$ 
\begin{align*}
[f_l \partial_{z_l} , \d z_i \wedge \d z_j] & = \del f_i \wedge \d z_j - \del f_j \wedge \d z_i \\
[g^{kl} \d z_k \wedge \d z_l , \d z_i \wedge \d z_j ] & = \ep_{ijklm} g^{kl} \partial_{z_m} .
\end{align*}
where $f_l \partial_{z_l}$ is a divergence-free vector field on $\CC^5$ and $g^{kl} \d z_k \wedge \d z_l$ is a closed two-form. 

From these relations, it is easy to see that any divergence-free vector field of the form:
\begin{itemize}
\item $f(z_i, z_j) \partial_{z_i} + g(z_i, z_j) \partial_{z_j}$ for holomorphic functions $f,g$ on $\CC_i \times \CC_j$ or 
\item $f(z_1,\ldots,z_5) \partial_{z_k}$ for $k \ne i$ or $j$
\end{itemize}
are closed for the differential $\d = [\d z_i \wedge \d z_j  , -]$, and that these are the only nonzero closed elements. 
Any element of the second type is clearly exact by the closed two-form $\ep^{ijklm} f \d z_l \d z_m$. 

Thus, the cohomology is the (purely bosonic) Lie algebra of divergence-free vector fields on $\CC^2 = \CC_i \times \CC_j$
\[
H^\bu\big(E(5,10), [\partial \Psi_{ij}, -] \big) \cong \Vect^{hol}_0(\CC_i \times \CC_j) .
\]
Notice that on $\CC^2$, there is an exactly sequence of Lie algebras
\[
0 \to \ul\CC \to \cO^{hol}(\CC^2) \to \Vect_0^{hol} (\CC^2) \to 0
\]
where $\cO^{hol}(\CC^2)$ is equipped with the Poisson bracket with respect to the symplectic form $\d z_i \wedge \d z_j$:
\[
[f(z_1,z_2) , g(z_1,z_2)] = \partial_{z_i} f \partial_{z_j} g - \partial_{z_j} f \partial_{z_i} g .
\]

\section{The twisted $\Omega$-background} 

\subsection{$\Omega$-deformed symmetry algebra}

On flat space, we have 



\subsection{M2 branes in the $\Omega$-background} 

\parsec[]

In the twisted $\Omega$-background a stack of $N$ M2 branes wraps 
\beqn\label{eqn:m2omega}
\{0\} \times \RR \subset \CC^2 \times \RR .
\eeqn

The field which sources this brane is a distributional differential form $F$.
The linear equation of motion for $F$ is the distributional equation 
\[
\d^2 z \wedge \dbar F = N \delta_{z=0} 
\]
where $\delta_{z=0}$ is the $\delta$-distribution for the submanifold \eqref{eqn:m2omega}.  
In what follows we denote $\til{\delta}_{z=0}$ the distributional $(0,2)$ form defined by contracting $\delta_{z=0}$ with the Calabi--Yau form~$\d^2 z$. 

The field $F$ is a smooth Dolbeault form of type $(0,1)$ away from the locus of the brane $\CC^2 \times \RR_t \setminus \{0\} \times \RR \cong (\CC^2 \setminus 0) \times \RR_t$. 
Explicitly, we can characterize it as follows. 

\begin{prop}
The field $F \in \Omega^{0,1} (\CC^2 \setminus 0) \otimes \Omega^{0} (\RR) $ defined by 
\[
F = \# N \frac{\zbar_1 \d \zbar_2 - \zbar_2 \d \zbar_1}{\|z\|^4} 
\] 
satisfies the non-linear equation of motion 
\[
\dbar F + \frac12 F \star_c F = N  \til{\delta}_{z=0} .
\]
\end{prop}
\begin{proof}
The equation $\dbar F = N  \til{\delta}_{z=0}$ characterizes the Bochner--Martinelli kernel representing the residue class in $\CC^2 \setminus 0$. 
Since $F$ is a one-form $F \star_c F = 0$ is satisfied by anti-symmetry. 
\end{proof}

\parsec[]

We can interpret $F$ geometrically in the following way. 
Let's first consider just the complex geometric situation, forgetting about the $\RR$-direction. 
 
On any hyper K\"ahler manifold $X$ there is an exact sequence of sheaves of Lie algebras
\[
0 \to \ul\CC \to \cO^{hol}(X) \to \Vect_0^{hol} (X) \to 0
\]
where $\Vect^{hol}_0(X)$ is the Lie algebra of holomorphic divergence-free vector fields and $\cO^{hol}(X)$ is equipped with the Poisson bracket induced by the holomorphic symplectic form. 
The map $\cO^{hol}(X) \to \Vect_0(X)$ is given by the holomorphic de Rham operator $\del \colon \cO^{hol}(X) \to \Omega^{1,hol}_{cl}(X)$ followed by the isomorphism 
$\Omega^{1,hol}_{cl} (X) \cong \Vect_0(X)$ given by the holomorphic symplectic form. 

Extending this exact sequence to the level of Dolbeault resolutions, we see that we can view $F_0 \in \Omega^{0,1}(\CC^2 \setminus 0)$ as a $(0,1)$ Dolbeault valued divergence-free vector field $\til{F}_0$ on $\CC^2 \setminus 0$. 
Explicitly, if $F_0 = f^i (z,\zbar) \d \zbar_i$ then this $(0,1)$-valued vector field is
\[
\til{F}_0 = \ep_{jk} \partial_{z_j} f^i (z,\zbar) \d \zbar_i \partial_{z_k} .
\]

In the case at hand, $F \in \Omega^{0,1}(\CC^2 \setminus 0) \otimes \Omega^0(\RR)$ is given by the restriction along 
\[
(\CC^2 \setminus 0) \times \RR \to \CC^2 \setminus 0
\]
of the $(0,1)$-form
\[
F_0 = \# N \frac{\zbar_1 \d \zbar_2 - \zbar_2 \d \zbar_1}{\|z\|^4} .
\] 
The corresponding $(0,1)$-valued vector field on $\CC^2 \setminus 0$ is
\[
\til{F}_0 = \#' N \frac{\zbar_1 \d \zbar_2 - \zbar_2 \d \zbar_1}{\|z\|^6} \left(\zbar_1 \del_{z_2} - \zbar_2 \del_{z_1}\right) .
\] 

\appendix 

\section{An alternative description of the 11-dimensional theory} 
In \S \ref{s:dfn} we have described a family of BV theories on $X \times L$ where $X$ is an odd-dimensional Calabi--Yau manifold and $L$ is an odd-dimensional smooth manifold. 
The space of fields \eqref{eq:sympfields} is equipped with an odd symplectic pairing. 


There is a related, alternative description of this formal moduli space which is almost equivalent. 
Consider the resolution of holomorphic closed two-forms 
\beqn\label{eqn:twoform}
\Omega_{cl}^{2,hol}(X) \simeq \Omega^{2,\bu}(X) \xto{\del} \Omega^{3,\bu}(X)[-1] \to \cdots 
\eeqn
where we have left the $\dbar$-differential implicit, as always. 

There is a map from the two-term complex 
\[
\Omega^{0,\bu}(X)[1] \to \Omega^{1,\bu}(X)
\]
to the resolution \eqref{eqn:twoform} defined by the holomorphic de Rham operator $\del$. 
This map is almost a quasi-isomorphism of sheaves; it differs by copy of constant functions $\CC[1]$ in cohomological degree $-1$. 

Similarly, we can tensor with the dg algebra of de Rham forms on $L$ to obtain a map of $\ZZ/2$-graded cochain complexes
\[
\bigg(\Pi \Omega^{0,\bu}(X;L) \xto{\del} \Omega^{1,\bu}(X;L) \bigg) \xto{\del} \bigg(\Omega^{2,\bu}(X;L) \xto{\del} \Pi \Omega^{3,\bu}(X;L) \to \cdots \bigg) .
\]
Again, this map is a quasi-isomorphism up to a copy of $\Pi \CC$.
The left-hand side contains the fields $(\beta, \gamma)$. 
This map has the effect of sending $\beta \mapsto 0$ and $\gamma \mapsto \partial \gamma \in \Omega^{2,\bu}(X;L)$. 

The basic idea is to replace the complex where $(\beta,\gamma)$ live by this resolution of closed two-forms. 
The new space of fields is 
\begin{equation}
  \label{eq:poissfields} 
  \begin{tikzcd}[row sep = 1 ex]
     - & + & - & + \\ \hline
     \PV^{1,\bu}(X; L) \ar[r, "\div"] & \PV^{0,\bu}(X; L) \\
     & \Omega^{2,\bu}(X;L) \ar[r, "\del"] & \Omega^{3,\bu}(X;L) \ar[r] & \cdots 
\end{tikzcd}
\end{equation}

It may seem that this description of the fields of the 11-dimensional theory is not much different than the original formulation---at the level of the free theory they only differ by a copy of constant functions in odd cohomological degree. 
We'd like to point out two main differences:
\begin{itemize}
\item This new description does not have the structure of a BV theory in the usual sense; there is no odd symplectic pairing on the fields. 
Nevertheless, there is still an odd Poisson bracket acting on functionals of the fields which we will describe momentarily. 
For this reason, we will refer to the theory as a ``Poisson BV theory''. 
\item With the existence of the odd Poisson bracket one might be optimistic to formulate the CME for this Poisson BV theory.
However, there is no local interaction which is consistent with the local interaction present in the original BV theory. 
Nevertheless, the (shift of the) fields is equipped with an $L_\infty$ structure which is compatible with the odd Poisson bracket. 
\end{itemize}

\begin{prop}
The complex \eqref{eq:poissfields} is equipped with the structure of an interacting Poisson BV theory.
Let $\cG = \cG(X \times L)$ be the resulting $L_\infty$ algebra...
\end{prop}

Heuristically, we can write the action in the following non-local form
\[
\frac{1}{1-\nu} \mu^2 \eta + (\del^{-1} \eta) \eta^2 .
\]
\brian{are there more terms involving higher forms?}

The first few nonzero brackets are
\begin{align*}
[\mu]_1 & = \dbar \mu + \div \mu \\
[\eta]_1 & = \dbar \eta + \del \eta \\ 
[\mu_1,\mu_2]_2 & = \div (\mu_1 \wedge \mu_2) \\
[\mu, \eta]_2 & = \mu \vee \del \eta \\
[\eta_1,\eta_2]_2 & = \# \Omega^{-1} \vee (\eta_1 \wedge \eta_2)  \\
[\nu, \mu_1,\mu_2]_3 & = \div(\nu \mu_1 \mu_2) \\
[\nu, \mu,\eta]_3 & = \nu \mu \del \eta
\end{align*}
\brian{coefficients}
For $k \geq 2$ the general formula for the $k$-ary bracket is 
\begin{align*}
[\eta_1,\eta_2]_2 & = \# \Omega^{-1} \vee (\eta_1 \wedge \eta_2)  \\
[\nu_1,\nu_2, \ldots, \nu_{k-2}, \mu_1,\mu_2]_{k} & = \# \div(\nu_1 \cdots \nu_{k-2} \mu_1 \wedge \mu_2) \\ 
[\nu_1,\nu_2, \ldots, \nu_{k-2}, \mu,\eta]_{k} & = \# \nu_1 \cdots \nu_{k-2} (\mu \vee \del \eta) .
\end{align*}

\parsec[]
Consider the theory on $\CC^3 \times \RR$. 
There is a quasi-isomorphism of $L_\infty$ Lie algebras $\CC \simeq \cG (\CC^3 \times \RR)$. 

\parsec[]
Consider the theory on $\CC^5 \times \RR$. 

\begin{prop}
There is a quasi-isomorphism of super $L_\infty$ algebras 
\[
E(5,10) \xto{\simeq} \cG(\CC^5 \times \RR) .
\]
This lifts to map of super $L_\infty$ algebras $E(5,10) \to \cL(\CC^5 \times \RR)$ whose kernel is $\CC$.
\end{prop}


\section{BCOV stuff}

\subsection{Relation to BCOV theory}

\parsec[sec:pv] 
\ingmar{a section defining polyvector fields; not sure where this belong}
Equipped with this structure, the complex we have written maps in an obvious way into the complex of polyvector fields on~$X$. Recall that one defines
\deq{
  \PV^{i,j}(X) = \Omega^{0,j}(X, \wedge^i \T_X).
}
\ingmar{Wedges look super fucked up}
The complex $\PV^{\bu,\bu}$ is equipped with two natural differentials: the Dolbeault operator $\dbar$, of $(i,j)$-degree $(0,1)$, and the holomorphic divergence operator $\div$, which carries $(i,j)$-degree $(-1,0)$. Assigning total degree $- i + j$ to $\PV^{i,j}$ thus gives the total polyvector fields the structure of a bicomplex. We will use the shorthand notation $\PV^i = (\PV^{i,\bu}, \dbar )$ for the Dolbeault resolution of holomorphic $i$-polyvector fields. 

\subsubsection{}
It will be convenient for us to cast this dg Lie algebra in a slightly different way.
We construct a different model for the Lie algebra of divergence-free vector fields whose underlying cochain complex is the same as \eqref{eqn:cplx1}. 
The distinguishing feature is that this model is no longer a dg Lie algebra, but an $L_\infty$ algebra. 

The model we use is motivated by the topological string, specifically the description of the closed topological string in terms of Kodaira--Spencer theory \cite{BCOV}.
We recall the requisite background, but refer to \cite{CLbcov1,CLbcov2,CLtypeI} for more details. 

Suppose $X$ be a Calabi--Yau manifold of dimension $d$. 
Let $\PV^{k,\bu}(X)$ denote the Dolbeault complex of the holomorphic vector bundle $\wedge^k \T_X$; we will omit $X$ in this section and write $\PV^{k,\bu}$ for simplicity.
In particular, $\PV^{k,j}$ is the space of smooth sections of the bundle $\wedge^j \T_X^* \otimes \wedge^k \T_X$. 
With this notation, the dg Lie algebra in \eqref{eqn:cplx1} is $\PV^{1,\bu} \oplus \PV^{0,\bu}[-1]$ with differential $\dbar + \div$. 

Introduce a formal parameter $u$ of cohomological degree $+2$ and consider the complex 
\beqn\label{eqn:pv1}
\PV^{\bu,\bu} [[u]] [1]
\eeqn
with differential $\dbar + u \div$ which we will often denote just by $Q_{KS}$ and refer to as the linear BRST differential. 
With our conventions, notice that $u^l \PV^{k,j}$ sits in cohomological degree $2l +k + j - 1$. 

The Schouten--Nijenhuis bracket 
\[
[-,-] \colon \PV^{k,j} \times \PV^{p,q} \to \PV^{p+k-1,j+q} 
\]
extends $u$-linearly to endow \eqref{eqn:pv1} with the structure of a dg Lie algebra.
We refer to this as the Kodaira--Spencer, or strict, dg Lie algebra structure.
We refer to the further cohomological shift 
\[
\cE_{KS} = \PV^{\bu,\bu} [[u]] [2]
\]
as the space of fields of Kodaira--Spencer theory.

Notice that the complex resolving divergence-free vector fields \eqref{eqn:cplx1} sits inside $\cE_{KS} [-1]$ as a sub dg Lie algebra as $\PV^{1,\bu} \oplus u \PV^{0,\bu}[1]$. 

\subsubsection{}

\brian{bv bracket}

\subsubsection{}

Together with the linear BRST differential $Q_{KS}$, the bracket $\{-,-\}_{KS}$ induces the structure of a dg Lie algebra the cohomological shift of the space of local functionals $\oloc(\cE_{KS})[-2d+5]$. 

\begin{thm}[\cite{CLbcov1}]
The BCOV action 
\[
I_{BCOV} \in \oloc \left(\cE_{KS} \right) 
\]
satisfies the classical master equation 
\[
Q_{KS} I_{BCOV} + \frac12 \left\{I_{BCOV}, I_{BCOV}\right\}_{KS} = 0 .
\]
In other words, $I_{BCOV}$ determines a Maurer--Cartan element in the dg Lie algebra $\oloc(\cE_{KS})[-2d+5]$.
\end{thm}

This action induces the square-zero operator 
\[
\delta_{BCOV} = Q_{KS} + \{I_{BCOV}, -\} 
\]
acting on observables of the Kodaira--Spencer fields. 
In other words, it defines a non-linear BRST operator; in turn it
induces an $L_\infty$ structure $\{[-]_k\}_{k =1,2,3,\ldots}$ on the complex \eqref{eqn:pv1} whose linear operation is $[-]_1 = Q_{KS} = \dbar + u\div$. 

This $L_\infty$ structure is clearly not identical as the strict dg Lie algebra structure (it has operations of arbitrary high order). 
Nevertheless, it is equivalent to the strict dg Lie model: there is an $L_\infty$ automorphism which exchanges the two structures.

It is easiest to describe this automorphism at the level of observables.
Use the notation $\Sigma$ for a linear observable of the Kodaira--Spencer fields $\cE_{KS}$. 
Then, the Kodaira--Spencer dg Lie algebra structure induces the non-linear BRST operator $\delta_{KS}$ given by
\[
\delta_{KS} (\Sigma) = Q_{KS} \Sigma + \frac12 [\Sigma,\Sigma] .
\]

The non-linear change of coordinates relating the two structures is defined by
\[
\Psi \colon \Sigma \mapsto \left[u (e^{\Sigma/u} -1)\right]_+
\]
where $[-]_+$ projects onto the non-negative powers of $u$.  

\parsec[]

The complex resolving the (shift of) divergence-free vector fields 
\[
\left(\PV^{1,\bu}[1] \oplus u \PV^{0,\bu}[2] \, , \, \dbar + u \div\right) 
\]
is a subcomplex of $\cE_{KS}$.
If $d = \dim_\CC(X)=3$ then the shifted Poisson bivector $\Pi_{BCOV}$ restricts to a shifted Poisson bivector on this subcomplex. 
In particular, the action $I_{BCOV}$ restricts to a solution to the classical master equation for this subcomplex. 

If $d \ne 3$ then there is the following subcomplex $\til{\cE}_{KS}\subset \cE_{KS}$ defined by:
\beqn\label{eqn:tilks}
\begin{tikzcd}
\ul{-1} & \ul{0} & \cdots & \ul{d-4} & \ul{d-3} & \cdots    \\
\PV^{1,\bu} \ar[r] & u\PV^{0,\bu} & & & & &  \\
& & & \PV^{d-2, \bu} \ar[r] & u \PV^{d-3, \bu} \ar[r] & \cdots &   .
\end{tikzcd}
\eeqn
And one can check that the shifted Poisson bivector restricts to a shifted Poisson bivector on this subcomplex. 
Thus, we have the following.

\begin{prop}
\label{prop:tilbcov}
The BCOV action $I_{BCOV}$ restricts to a solution of the classical master equation on $\til{\cE}_{KS}$ that we denote by $\til{I}_{BCOV}$. 
\end{prop}

\subsubsection{}

Define the action by the group $\CC^\times$ on the complex $\til{\cE}_{KS}$ as follows.
Declare the first line in \eqref{eqn:tilks} is weight zero and the second line is weight $+1$. 
%Declare $\mu, \nu$ have $\CC^\times$ weight zero and $\beta,\gamma$ have $\CC^\times$ weight $+1$. 
The linear BRST operator is clearly weight zero and the shifted Poisson structure $\Pi_{KS}$ is of weight $+1$. 
Thus, the shifted Poisson bracket $\{-,-\}_{KS}$ is of $\CC^\times$ weight $+1$.

\begin{lem}
The restricted action $\til{I}_{BCOV}$ has $\CC^\times$ weight $-1$. 
In particular, the non-linear BRST operator 
\beqn\label{eqn:KSbrst}
\delta_{BCOV} = Q_{KS} + \{\til{I}_{BCOV},-\}_{KS}
\eeqn
is $\CC^\times$ weight zero.
\end{lem}
\begin{proof}
%The map $\Phi \colon \cE_{pot} \to \til{\cE}_{KS}$ preserves $\CC^\times$ weight, so it suffices to prove that $\til{I}_{BCOV}$ is weight $+1$. 
We will make use of two gradings on $\til{\cE}_{KS}$. 
The first is holomorphic polyvector field type, and the second is descendant degree.
The summand $u^l \PV^{k,\bu}$ is polyvector type $k$ and descendant degree $l$. 

We introduce notation in this proof that won't be used later on. 
Let $\alpha$ denote a super field living in the first line of \eqref{eqn:tilks} and let $\beta$ denote a super field living in the second line of \eqref{eqn:tilks}.
If a field of type $\alpha$ has polyvector type $k$ then it has descendant degree $1-k$, $k=0,1$. 
If a field of type $\beta$ has polyvector type $l$ then it has descendant degree $d-2-l$.

A homogenous polynomial degree term in the BCOV action $I_{BCOV}$ is a linear combination of functionals of the form
\[
 \alpha_1 \wedge \cdots \wedge \alpha_m \wedge \beta_1 \wedge \cdots \wedge \beta_n  .
\]
Let $k_i$ be the polyvector field type of $\alpha_i$ and let $l_j$ be the polyvector field type of $\beta_j$. 
In order for this expression to contribute nontrivially to the BCOV action two constraints must hold:
\begin{align*}
\sum_{i=1}^m k_i + \sum_{j=1}^n l_j & = d \\
\sum_{i=1}^m (1-k_i) + \sum_{j=1}^n (d-2-l_j) & = m+n-3 .
\end{align*}
The first constraint ensures that the integrand is of top polyvector degree. 
The second constraint is on the descendant degree, see \brian{ref above}. 

Simplifying these two equations, one finds the condition
\[
m+n-3 = m+(d-2)n - d 
\]
which implies $n=1$, as desired. 
\end{proof}

Consider the cochain complex $\left(\oloc(\til{\cE}_{KS}) \, , \, \delta_{KS} \right)$ of all local functionals equipped with the non-linear BRST differential \eqref{eqn:KSbrst}.  
The $\CC^\times$ weight zero subcomplex is 
\[
\big(\oloc(\til\cE_{KS})^{(0)} \, , \, \delta_{KS} \big) = \big(\oloc(\cE_0) \, , \, \delta_{KS} \big) 
\]
where $\cE_0 = \PV^{1,\bu}[1] \oplus u \PV^{0,\bu}$ is the subcomplex of the fields $\til\cE_{KS}$ comprising the first line of \eqref{eqn:tilks}.

\begin{prop}
\label{prop:Linfty}
The differential $\delta_{KS}$ acting on the weight zero fields induces a local $L_\infty$ algebra structure on the complex of vector bundles 
\[
\cE_0 [-1] = \PV^{1,\bu} \oplus \PV^{0,\bu}[-1]
\]
with $[-]_1 = \dbar + \div$. 
Furthermore, this $L_\infty$ algebra structure is equivalent to the strict dg Lie algebra structure on the resolution of divergence-free vector fields defined in \S \ref{sec:divfree}. 
\end{prop}

%Applying this to the $k$th exterior power of the holomorphic tangent bundle $V = \wedge^k \T_X$, we obtain a resolution of the sheaf of $k$-linear polyvector fields which we denote by $\PV^{k,\bu} (X) = \Omega^{0,\bu}(X, \wedge^k \T_X)$. 

%\subsubsection{}
%
%Consider the following cochain complex $\cE_{pot}$:
%\beqn\label{eqn:E}
%\begin{tikzcd}
%\ul{-1} & \ul{0} & \cdots & \ul{d-5} & \ul{d-4} &   \\
%\PV^{1,\bu} \ar[r,"\div"] & \PV^{0,\bu} & \cdots & & \\
%& & & \Omega^{0, \bu} \ar[r,"\del"] & \Omega^{1, \bu} 
%\end{tikzcd}
%\eeqn
%We will refer to the components of the fields using the notation
%\begin{align*}
%(\mu, \nu) & \in \Pi \PV^{1,\bu}(X)[1] \oplus \PV^{0,\bu}(X) \\
%(\beta, \gamma) & \in \Omega^{0,\bu}(X)[d-5] \oplus \Omega^{1,\bu}(X) [d-4].
%\end{align*}
%The expressions $\mu,\nu,\beta,\gamma$ are still super fields in the sense that they have components in all anti-holomorphic Dolbeault degree. 
%
%Let 
%\[
%\begin{array}{rccc}
% \colon \Omega_c^{0,\bu}(X) & \to & \CC [??] \\
%\alpha & \mapsto & \int_{X} \Omega \wedge \alpha 
%\end{array}
%\]
%be integration along the holomorphic volume form. 
%Define the pairing 
%\[
%\omega \colon \cE_{pot,c} \times \cE_{pot} \to \CC [??] 
%\]
%by the formula $ \gamma \vee \mu +  \beta \nu.$
%
%\brian{defines Poisson bivector $\Pi$.}
%
%\subsubsection{}

%Define the map of cochain complexes $\Phi \colon \cE_{pot} \to \til{\cE}_{KS}$
%by the formulas
%\[
%\Phi (\mu,\nu) = \mu + u \, \nu,\qquad \Phi(\beta, \gamma) = \div \left( \gamma \vee \Omega^{-1} \right)  .
%\]
%In the last expression we have used the isomorphism $\Omega^{-1} \colon \Omega^{1,\bu} \xto{\cong} \PV^{d-1,\bu}$
%so that $\div(\gamma \vee \Omega^{-1}) \in \PV^{d-2,\bu} \subset \til{\cE}_{KS}$.
%
%\begin{prop}
%\label{prop:pot}
%The map $\Phi \colon  \cE_{pot} \to \til{\cE}_{KS}$ intertwines the linear BRST differentials and the shifted Poisson bivectors $\Phi_*\Pi = \til{\Pi}_{KS}$. 
%In particular, it defines a map of dg Lie algebras
%\[
%\Phi^* \colon \big(\oloc(\cE_{KS}),\{-,-\}_c, Q_{KS} \big) \to \big(\oloc(\cE_{pot}),\{-,-\}, Q \big)
%\]
%\end{prop}
%
%\begin{proof}
%prove this
%\end{proof}
%
%As a corollary of Proposition \ref{prop:tilbcov} and Proposition \ref{prop:pot} we have the following result. 
%
%\begin{cor}
%The BCOV action $\til{I}_{BCOV}$ restricts along $\Phi$ to a solution of the classical master equation for $\cE_{pot}$:
%\begin{align*}
%I_{pot} & = \Phi^* \til{I}_{BCOV} \\
%Q I_{pot} + \frac12 \{I_{pot},I_{pot}\} & = 0 .
%\end{align*} 
%\end{cor} 


\end{document}
