\documentclass[11pt]{amsart}

\usepackage{macros-mtheory,amsaddr}

\addbibresource{cfs.bib}

%\linespread{1.2} %for editing
%\usepackage{mathpazo}

\begin{document}

\title{Twisted 11-dimensional supergravity, I}
\author{Surya Raghavendran}
\address{Perimeter Institute}
\email{??}
\author{Ingmar Saberi}
\address{Ludwig-Maximilians-Universit\"at M\"unchen \\ Fakult\"at f\"ur Physik \\ Theresienstra\ss{}e 37 \\ 80333 M\"unchen \\ Deutschland}
\email{i.saberi@physik.uni-muenchen.de}
\author{Brian R. Williams}
\address{School of Mathematics, University of Edinburgh, Edinburgh, UK}
\email{brian.williams@ed.ac.uk}
\begin{abstract}
to do
\end{abstract}
\maketitle

\section{The 11-dimensional theory} 

\subsection{Divergence-free vector fields} 

\subsection{A family of deformations} 

%check CME

\subsection{Equations of motion}

%whatever we can say about the formal moduli space

\subsection{Boundary conditions}

%foreshadow the Horava--Witten picture

\subsection{One-loop quantization} 

%cite result on thf quantization.

\section{Relationship to physical supergravity}

\subsection{Component fields}

%Can we try to roughly say where each of our twisted fields come from in the untwisted theory? I imagine this section being pretty chill and mostly to orient physicists. 

\subsection{Residual supersymmetry} 

%m2brane

\subsection{Superconformal algebras}

%6d suco 
%what about 3d N=8 suco?

\section{Dimensional reduction and compactifications} 

\subsection{Relationship to the topological string}

%conjecture for minimal twist of IIA
%check the SU(4) twist

\subsection{The theory on a three-fold}

%count hypers and vectors in 5d N=1 sugra relate to Kevin and my work with Chris

\section{The non-minimal twist} 


\end{document}