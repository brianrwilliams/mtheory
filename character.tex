\documentclass[11pt]{amsart}

\usepackage{macros-mtheory,amsaddr,tikz-feynman}

\addbibresource{references.bib}

%\linespread{1.2} %for editing
%\usepackage{mathpazo}

\begin{document}

\subsection{Local character}

We consider the 11-dimensional theory on the manifold $\CC^5 \times \RR$ where $\CC^5$ is equipped with its standard Calabi--Yau structure. 
On this background, the theory is manifestly $SU(5)$ invariant. 
In this section, we compute the corresponding character of the local operators at $0 \in \CC^5 \times \RR$. 

The character is only sensitive to the free limit of the theory.
Furthermore, the linear BRST operator is an $SU(5)$-invariant deformation of the $(\dbar + \d_{\RR})$ operator. 
Therefore, to compute the character it suffices to compute the $SU(5)$-equivariant character of the $\dbar$-cohomology. 

The solutions to the $(\dbar + \d_{\RR})$-equations of motion simply say that all fields are holomorphic along $\CC^5$ and constant along $\RR$. 
Thus, the solutions can be identified with 
\begin{align*}
\mu^{i}\partial_{z_i} & \in \Vect(\CC^5) \cong \oplus_i \cO(\C^5)\partial_{z_i},\quad 
\nu \in \cO (\C^5) \\
\beta & \in \cO (\C^5), \quad \gamma^{i} \d z_i \in \Omega^{1}(\CC^5) \cong \oplus_i \cO (\C^5)\d z_i 
\end{align*}
where $z_i$ is a holomorphic coordinate on $\CC^5$. 

Corresponding to each of the above, we have a tower of linear local operators labeled by $(m_j) = (m_1, m_2, m_3, m_4, m_5)\in \Z^5_{\geq 0}$; these are given by
\begin{align*}
 \mu^{i}_{(m_j)} &: \mu^{i}\mapsto \partial_{z_1}^{m_1}\partial_{z_2}^{m_2}\partial_{z_3}^{m_3}\partial_{z_4}^{m_4}\partial_{z_5}^{m_5}\mu^{i} (0) \\
\nu_{(m_j)} &: \nu\mapsto \partial_{z_1}^{m_1}\partial_{z_2}^{m_2}\partial_{z_3}^{m_3}\partial_{z_4}^{m_4}\partial_{z_5}^{m_5}\nu (0) \\
\gamma^{i}_{(m_j)} &: \gamma^{i}\mapsto \partial_{z_1}^{m_1}\partial_{z_2}^{m_2}\partial_{z_3}^{m_3}\partial_{z_4}^{m_4}\partial_{z_5}^{m_5}\gamma^{i} (0) \\
 \beta_{(m_j)} &: \beta\mapsto \partial_{z_1}^{m_1}\partial_{z_2}^{m_2}\partial_{z_3}^{m_3}\partial_{z_4}^{m_4}\partial_{z_5}^{m_5}\beta (0) \\
\end{align*}

It is easiest to label the Cartan subgroup of $SU(5)$ by fugacities $q_1,\ldots, q_5$ subject to the constraint that $\prod_{i=1}^5 q_i = 1$. 
The single particle index is the $SU(5)$ character of the space of linear local operators.

\begin{lem}
The single particle index is 
\[
i(q_1,\ldots,q_5) = \frac{\sum_{i=1}^5 q_i}{\prod_{i=1}^5 (1-q_i)} + \frac{\sum_{i=1}^5 q_i^{-1}}{\prod_{i=1}^5 (1-q_i^{-1})}
\]
where the fugacities satisfy the constraint $\prod_{i=1} q_i = 1$. 
\end{lem}
\begin{proof}
The linear local operators $ \nu_{(m_j)}$ and $\beta_{(m_j)}$ are of the same $q$-weight but opposite parity.
Thus, they do not contribute to the single particle index.

The $q$-weight of the odd local operator $\mu_{(m_j)}^i$ is 
\[
q_1^{m_1+1} \cdots q_i^{m_i} \cdots q_5^{m_5+1} .
\]
The $q$-weight of the even local operator $\gamma_{(m_j)}^i$ is 
\[
q_1^{m_1} \cdots q_i^{m_i + 1} \cdots q_5^{m_5} .
\]

Thus we find that the single particle index is given by the infinite series
\beqn\label{infseriesindex}
\sum_{i=1}^5\left ( \sum_{(m_i)\in \Z^5} q_1^{m_1} \cdots q_i^{m_i + 1} \cdots q_5^{m_5} - \sum_{(m_i)\in \Z^5} q_1^{m_1+1} \cdots q_i^{m_i} \cdots q_5^{m_5+1} \right)
\eeqn

which sums to the expression
\beqn\label{singleparticleindex}
- \frac{\sum_{i=1}^5 q_1 \cdots \Hat{q_i} \cdots q_5}{\prod_{i=1}^5 (1-q_i)} + \frac{\sum_{i=1}^5 q_i}{\prod_{i=1}^5 (1-q_i)} .
\eeqn

This simplifies to the stated expression.
\end{proof}

This single particle index for our space of local operators agrees with the one computed in \cite{NekrasovInstanton}. 
To obtain the full index of local operators we apply the plethystic exponential ${\rm PE}[i(x)] = \exp\left(\sum_n \frac1n f(x^n)\right)$. 

\begin{prop}
The character of local operators of the 11-dimensional theory on $\CC^5 \times \RR$ is 
\[
\prod_{i=1}^{5} \prod_{(m_i)\in \Z^5} \frac{1-q_1^{m_1+1}\cdots q_i^{m_i}\cdots q_5^{m_5+1}}{1-q_1^{m_1}\cdots q_i^{m_i+1}\cdots q_5^{m_5}}
\]
\end{prop}
\begin{proof}
Recall that the plethystic exponential takes sums to products and monomials to geometric series. Apply this to the infinite series \ref{infseriesindex}.
\end{proof}

In terms of these fugacities, deforming the theory in a way that breaks the $SU(5)$ symmetry to a $SU(3)\times SU(2)$ subgroup can be described by requiring that the fugacities satisfy the additional constraints \[q_1q_2q_3 = 1, \ \ \ q_4q_5=1.\]
\begin{prop}
After imposing the above constraints, the single particle index is
\[
i(q) = \frac{1}{(1-q)(1-q^{-1})}
\]
where $q = q_4 = q_5^{-1}$. This matches with the single particle index of the nonminimal twist.
\end{prop}
\end{document}