\section{The minimal twist of eleven-dimensional supergravity} 
\label{s:dfn}

In this section we define the central theory of study within the Batalin--Vilkovisky formalism.
The theory will be defined on any eleven-dimensional manifold of the form $X \times L$, where $X$ is a Calabi--Yau five-fold and $L$ is a smooth oriented one-manifold.

\subsection{Divergence-free vector fields} 

\subsubsection{}
\label{sec:divfree}
We set up some notations and conventions in the context of complex geometry. 
Let $V$ be a holomorphic vector bundle on a complex manifold $X$. 
If $j$ is an integer, we let $\Omega^{0,j}(X, V)$ denote the space of anti-holomorphic Dolbeault forms of type $j$ on with values in $V$.
The $\dbar$ operator for $V$ is $\dbar \colon \Omega^{0,j}(X, V)\to \Omega^{0,j+1}(X)$ defines the Dolbeault complex of $V$
\[
  \Omega^{0,\bu}(X, V) = \left(\Omega^{0,j}(X, V)[-j] , \; \dbar\right) .
\]
This is a free resolution for the sheaf of holomorphic sections of $V$.

Suppose $X$ is a Calabi--Yau manifold with holomorphic volume form $\Omega$.
The divergence $\div(\mu)$ of a holomorphic vector field $\mu$ is defined by the formula
\[
\div (\mu) \wedge \Omega = L_\mu (\Omega)
\]
where, on the right hand side, we mean the Lie derivative of $\Omega$ with respect to $\mu$.

Let $\T_X$ denote the holomorphic tangent bundle and consider its Dolbeault complex $\Omega^{0,\bu}(X , \T_X)$ resolving the sheaf of holomorphic vector fields. 
The divergence operator extends to the Dolbeault complex to yield a map of cochain complexes 
\[
\div \colon \Omega^{0,\bu}(X , \T_X) \to \Omega^{0,\bu}(X) .
\]
The resulting complex of sheaves
\beqn\label{eqn:cplx1}
\begin{tikzcd}
\ul{0} & \ul{1} \\
\Omega^{0,\bu}(X , \T_X) \ar[r, "\div"] & \Omega^{0,\bu}(X) ,
\end{tikzcd}
\eeqn
resolves the sheaf of holomorphic divergence-free vector fields $\Vect_0 (X)$.
The anti-holomorphic Dolbeault degrees and the $\dbar$ operator are left implicit. 
%By the holomorphic Poincar\'e lemma, the embedding of the sheaf $\Vect_0(X)$ into the degree zero piece of this complex is a quasi-isomorphism. 

There is a direct way to extend the Lie bracket of vector fields to the complex \eqref{eqn:cplx1}. 
Denote by $\mu$ an element of $\Omega^{0,\bu}(X , \T_X)$ and $\nu$ an element of $\Omega^{0,\bu}(X)$ (for simplicity in notation, we will not expand the anti-holomorphic dependence). 
The Lie bracket defined by the formulas
\begin{align*}
[\mu, \mu'] & = L_\mu \mu' \\
[\mu, \nu] & = L_\mu \nu 
\end{align*}
is compatible with $\div$ and endows \eqref{eqn:cplx1} with the structure of a sheaf of dg Lie algebras.
We will refer to this sheaf by the symbol $\cL_0(X)$, or just $\cL_0$ if $X$ is understood. 

The sheaf $\cL_0$ has the structure of a {\em local} dg Lie algebra, see see \cite[\S 3.1.3]{CG2}.
This means that as a graded sheaf, $\cL_0$ is the smooth sections of a graded vector bundle and the differential and Lie bracket are given by differential and bidifferential operators, respectively.


\parsec[sec:Linfty]

Recall that an $L_\infty$ algebra is a $\ZZ$-graded vector space $\cL$ together with the data of a square-zero, degree $+1$ derivation $\delta_\cL$ of the free commutative graded algebra $\Sym\left(\cL^\vee [-1] \right)$. 
The Chevalley--Eilenberg cochain complex is 
\[
\left(\Sym\left(\cL^\vee [-1] \right), \delta_\cL\right) .
\]
The Taylor components of $\delta_\cL$ define higher brackets $\{[-]_k\}_{k=1,2,\ldots}$ where $[-]_k \colon \cL^{\times k} \to \cL[2-k]$. 
The condition that the differential is square-zero $\delta_\cL \circ \delta_\cL = 0$ is equivalent to the higher Jacobi relations.

An $L_\infty$ morphism $\Phi: \cL \rightsquigarrow \cL'$ is the same datum as a map of commutative dg algebras 
\deq{
  \Phi^*: \clie^\bu(\cL') \to \clie^\bu(\cL)
}
between their respective Lie algebra cochains. It follows from this that \emph{any} automorphism $\Phi$ of the free commutative algebra on $\cL^\vee[-1]$ defines a new model of the $L_\infty$ algebra $\cL$, for which the Chevalley--Eilenberg differential is obtained by conjugating $\delta_\cL$ by~$\Phi$, and where $\Phi$ itself defines the $L_\infty$ isomorphism.

$\cL_0$ is the sheaf~\eqref{eqn:cplx1} resolving divergence-free vector fields equipped with the dg Lie algebra structure constructed in the previous section.
We consider the following automorphism of~$\Sym(\cL_0^\vee[-1])$, defined by its action on generators:
\deq[eq:newbase]{
%    \Psi_\infty:  \nu &\mapsto -\frac{\nu}{1-\nu}, \quad \mu \mapsto -\mu \\
    \Psi_\infty: \nu \mapsto 1 - e^{-\nu}, \quad \mu \mapsto e^{-\nu} \mu.
}
This map defines a new model of the $L_\infty$ algebra with underlying graded vector space the same as \eqref{eqn:cplx1}, which we will call $\cL_\infty$.\footnote{We are being slightly abusive and using the symbols $\nu,\mu$ dually as coordinates, or operators, on the graded linear space $\cL[1]$.}
The formulas for the automorphism above clearly arise from maps of vector bundles and hence endow $\cL_\infty$ with the structure of a local $L_\infty$ algebra, meaning all operations are given by polydifferential operators.  

The notation refers to the fact that this new model has nonvanishing $L_\infty$ brackets of every order. 
It is this new model that we will use to define the eleven-dimensional theory of twisted supergravity (as well as the family of analogous formal moduli problems on products of odd Calabi--Yau manifolds with~$\R$). 


%The previous proposition characterizes the $L_\infty$ model for divergence-free vector fields that we will use to define the 11-dimensional theory of twisted supergravity. 
%Hereon, we denote by $\cL_0 = \cE_0[-1]$ this local $L_\infty$ algebra. 

%We can unpack Proposition \ref{prop:Linfty} to describe the $L_\infty$ structure on $\cL_0$ explicitly. 
We can describe the $L_\infty$ structure on our new model $\cL_\infty$ explicitly.
Recall that we have two types of elements: $\mu \in \PV^{1,\bu}$ and $\nu \in \PV^{0,\bu}[-1]$. 
The first few nonzero brackets are
\begin{align*}
[\mu]_1 & = \dbar \mu + \div \mu \\
[\mu_1,\mu_2]_2 & = \div (\mu_1 \wedge \mu_2) \\
[\nu, \mu_1,\mu_2]_3 & = \div(\nu \mu_1 \wedge \mu_2) 
%\\
%[\nu_1,\nu_2, \mu_1,\mu_2]_4 & = \# \div(\nu_1 \nu_2 \mu_1 \mu_2).
\end{align*}
For $k \geq 2$ the general formula for the $k$-ary brackets are 
\begin{align*}
[\nu_1, \ldots, \nu_{k-2}, \mu_1,\mu_2]_{k} & = \div(\nu_1 \cdots \nu_k \mu_1 \wedge \mu_2) \\
[\nu_1,\ldots, \nu_{k-3}, \mu_1,\mu_2,\gamma]_k & = \nu_1 \cdots \nu_{k-3} (\mu \wedge \mu') \vee \del \gamma .\\
[\nu_1,\ldots,\nu_{k-2}, \mu, \gamma]_k & = \nu_1 \cdots \nu_{k-2} \mu \vee \del \gamma .
\end{align*}

%We describe the explicit $L_\infty$ automorphism $\Psi \colon (\cL_0)^{L_\infty} \rightsquigarrow (\cL_0)^{strict}$ intertwining the strict dg Lie structure on $\cL_0$ and this $L_\infty$ structure.
%The linear term $\Psi^{(1)} = \id$ is the identity map. 
%The higher terms $\Psi^{(n)}$ are defined by 
%\brian{someone check me}
%\begin{align*}
%\Psi^{(n)} (\nu_1,\ldots, \nu_{n-k},\mu_1,\ldots, \mu_k) & = \delta_{k=1} \nu_1 \cdots \nu_{n-1} \mu_1 . \\
%\end{align*}


%\parsec
%There is yet another $L_\infty$ model for divergence-free vector fields that we remark on here.
%\ingmar{I want to write the other automorphism later, I think; otherwise we have to write the composition, since we presented it on bcov}


\subsection{Theories of BF type}

\parsec
Suppose that $\cL$ is an $L_\infty$ algebra with $L_\infty$ operations $\{[-]^\cL_k\}_{k=1,2,\ldots}$ and that $(\cA, \d_\cA)$ is a commutative dg algebra. 
The graded vector space $\cL \otimes \cA$ is equipped with the natural structure of an $L_\infty$ algebra with operations $\{[-]_k\}_{k=1,2,\ldots}$ defined by
\begin{align*}
[x \otimes a]_1 & = [x]^\cL_1 \otimes a + (-1)^{|x|} x \otimes \d_\cA a \\
[x_1 \otimes a_1, \ldots , x_k \otimes a_k]_k & = [x_1,\ldots,x_k]^\cL_k \otimes (a_1 \cdots a_k), \qquad k \geq 2 .
\end{align*}

We apply this construction, taking $\cL$ to be the sheaf resolving divergence-free holomorphic vector fields on a Calabi--Yau manifold $X$ equipped with either the strict dg Lie algebra structure $\cL_0(X)$ or its non-strict $L_\infty$ structure $\cL_\infty (X)$. 
The algebra $\cA$ will be the smooth de Rham complex $(\Omega^\bu(S) , \d_S)$ where $S$ is a smooth manifold (we will specialize the dimension of this smooth manifold shortly, but the constructions in this section make sense in any dimension). 

We thus obtain the structure of an dg Lie algebra on $\cL_0(X) \otimes \Omega^{\bu}(S)$ or an $L_\infty$ algebra $\cL_\infty(X) \otimes \Omega^\bu(S)$.
These define equivalent local $L_\infty$ algebras on the product manifold~$X \times S$. 

\parsec[s:bf]

Associated to any local $L_\infty$ algebra is a classical field theory in the BV formalism.
Let $\cL$ be a local $L_\infty$ algebra on some manifold $M$, it is the sheaf of sections of some graded vector bundle $L$. 
For a section $A \in \cL$, introduce the `higher curvature map' defined by the formula
\[
\mathsf{F}_A = [A]_1 + \frac12 [A,A]_2 + \frac{1}{3!} [A,A,A]_3 + \cdots .
\]

The fields of the associated BV theory are pairs
\[
  (A, B) \in \cL[1] \oplus \cL^{!}[-2] .
\]
Here $\cL^!$ denotes the sheaf of sections of the bundle $L^* \otimes {\rm Dens}$, where ${\rm Dens}$ is the bundle of densities. 
The shifted symplectic BV pairing is the obvious integration pairing between $A$ and $B$. 

The action functional reads $S_{\rm BF} = \int_M B \, \mathsf{F}_{A}$ which leads to the equations of motion $\mathsf{F}_{A} = 0$ and $\mathsf{D}_A B= 0$ where $\mathsf{D}_A$ is the higher covariant derivative along $A$. 
We refer to this as the ``BF theory'' associated to $\cL$.

We thus obtain a theory in the BV formalism on the product manifold $X \times S$ associated to both local $L_\infty$ algebras $\cL_0(X) \otimes \Omega^{\bu}(S)$ and $\cL_\infty(X) \otimes \Omega^\bu(S)$.

\parsec
For concreteness, we spell out the fields of the theories we have constructed on $X \times S$.
In both cases, the space of fields equipped with the linear BRST operator is
\begin{equation}
  \label{eq:sympfields} 
  \begin{tikzcd}[row sep = 1 ex]
    -n & -n + 1 & -1 & 0 \\ \hline
    \Omega^{0}(X;S) \ar[r, "\del"] & \Omega^{1}(X;S) & 
     \PV^{1}(X; S) \ar[r, "\div"] & \PV^{0}(X; S).
\end{tikzcd}
\end{equation}
We denote the fields $(\beta,\gamma,\mu,\nu)$ respectively.
We are using the shorthand notation
\begin{align*}
\Omega^{i}(X;S) & = \Omega^{i , \bu;\bu}(X;S) \\
 & = \oplus_{j,k} \PV^{i,j}(X) \otimes \Omega^k(S) [-j-k] .
\end{align*}
which is equipped with the $\dbar + \d_S$ operator and similarly for $\PV^{i}(X;S)$. 

The natural pairing between $\PV^i(X;S)$ and~$\Omega^i(X;S)$ is of degree $-\dim_\C(X) -\dim_\R(S)$. 
As such, the $\Z$-grading indicated in~\eqref{eq:sympfields} equips the sheaf of fields with a degree $(-1)$ pairing, provided that we choose the shift to be given by
\deq{
  n = \dim_\C(X) + \dim_\R(S) - 1.
}
The pairing is defined by the formula 
\[
\int^\Omega_{X \times S} \mu \vee \gamma + \int^\Omega_{X \times S} \nu \beta 
\]
where $\int^\Omega_{X \times S} \alpha = \int_{X \times S} \alpha \wedge \Omega$. 

We have constructed two equivalent descriptions of the BF theory which share the linear BRST complex \eqref{eq:sympfields}.
Explicitly, the action functional for BF theory associated to the local dg Lie algebra $\cL_0(X) \otimes \Omega^{\bu}(S)$ is
\deq{
  S_{BF,0} =  \int^\Omega \bigg[\beta \wedge (\dbar + \d_S) \nu +  \gamma \wedge (\dbar + \d_S) \mu +  \beta \wedge \partial_\Omega \mu +  \frac{1}{2} [\mu,\mu] \vee \gamma +  [\mu,\nu] \beta \bigg] .
}
As in the Lie algebra structure of this strict model, notice that the Schouten--Nijenhuis bracket appears explicitly. 

The action functional of BF theory associated to $\cL_\infty(X) \otimes \Omega^{\bu}(S)$ is non polynomial. 
In fact, it is related to the BCOV action functional via dimensional reduction, see \S \ref{sec:dimred}.
Explicitly, this action functional is
\deq{
  S_{BF,\infty} =  \int^\Omega \bigg[ \beta \wedge (\dbar + \d_S) \nu +  \gamma \wedge (\dbar + \d_S) \mu +   \beta \wedge \partial_\Omega \mu + \frac12 \frac{1}{1-\nu} \mu^2 \vee \del \gamma \bigg] .
}

We demonstrated above that the two local $L_\infty$ algebras on which these BF theories are based are equivalent. As such, the BF theories are also equivalent; the map~\eqref{eq:newbase} extends uniquely to an automorphism of BV theories.
Explicitly, the automorphism is
\begin{multline}\label{eqn:auto1}
  \mu \mapsto e^{-\nu} \mu, \qquad \nu \mapsto 1-e^{-\nu} \\
  \beta \mapsto (\beta - \mu \vee \gamma) e^{\nu},\qquad \gamma \mapsto e^{\nu} \gamma .
\end{multline}

\parsec[]

In what follows, we specialize to the case that $X$ is a Calabi--Yau five-fold and that $S$ is a one-dimensional smooth orientable manifold. 
In this case, with $n = 5 + 1 - 1 = 5$ the theories described in this section are $\ZZ$-graded in the BV formalism.
Momentarily, we consider a new term in the action which will break this grading, so this integer shift will not play an essential role.

\subsection{A deformation of BF theory} 

Let $X$ be a Calabi--Yau five-fold and $S$ be a smooth oriented one-dimensional real manifold. 
We will break the $\ZZ$-grading present in BF theory discussed in the previous section to a $\ZZ/2$ grading. 
For reference, this means that linear BRST complex of fields of the model now take the following form. 

\begin{equation}
  \label{eq:sympfields} 
  \begin{tikzcd}[row sep = 1 ex]
    {\rm odd} & {\rm even} & {\rm odd} & {\rm even} \\ \hline
    \Omega^{0}(X;S)_\beta \ar[r, "\del"] & \Omega^{1}(X;S)_\gamma & 
     \PV^{1}(X; S)_\mu \ar[r, "\div"] & \PV^{0}(X; S)_\nu.
\end{tikzcd}
\end{equation}

\parsec
%check CME

To define our classical field theory on $X \times S$ we consider  a deformation of BF theory $S_{BF}$ (this refers to either the presentation as $S_{BF,0}$ or $S_{BF,\infty}$). 
Such deformations are governed by the classical master equation: the parameterized family of actions 
\beqn\label{eqn:defaction}
S_{BF} + g J
\eeqn
defines a consistent theory in the BV formalism if and only if
\deq{
  \{S_{BF} + g J, S_{BF} + g J \} = 0.
}
Since this must hold for all $g$, and since the undeformed action $S$ is already a solution to the classical master equation, this reduces to the pair of conditions
\deq[eq:2cond]{
  \{S_{BF},J\} =  \{J,J\} = 0.
}

The form of $J$ depends on which presentation we use for BF theory.
To begin, we will use the presentation of BF theory $S_{BF, \infty}$ which uses the the non-strict $L_\infty$ structure on divergence-free holomorphic vector fields.
The deformation $J$ does not make reference to the Calabi--Yau structure explicitly, but it does involve the holomorphic de Rham operator $\del$ on $X$. 

The main result of this section is the following. 

\begin{thm}
\label{thm:dfn}
Let $X$ be a Calabi--Yau five-fold and $S$ a smooth one-dimensional manifold, and consider the BV theory $(\cE, S_{BF,\infty})$ on $X \times S$ defined above. The local functional 
  \deq{
    J = \frac16 \gamma \wedge \del \gamma \wedge \del \gamma ,
  }
  where $\gamma \in \Omega^{1,\bu}(X;S)$, defines a deformation of~$(\cE,S_{BF,\infty})$ as a $\Z/2$-graded BV theory.
\end{thm}

\parsec[]

First off, we remark on grading issues. 
In the original $\Z$-grading on the BF theory given in \eqref{eq:sympfields} with $n=5$, the component 
\[
\gamma^{1,i;j} \in \Omega^{1,i}(X) \otimes \Omega^j(S) 
\]
sits in degree $-4+i+j$. 
Thus, we see that in the original $\Z$-grading on BF theory one has
  \deq{
    \deg(J) = 6.
  }
Thus $S_{BF} + J$ is not of homogenous $\ZZ$ grading.

This is completely reasonable from the point of view of twisting supersymmetry in eleven dimensions. 
Indeed, the $R$-symmetry group is trivial, and there is not a way to regrade the fields of the twisted theory using twisting data \label{CosHol,ESW}. 

Nevertheless, if we break to the obvious $\ZZ/2$ grading, the functional $S_{BF} + g J$ defines an even action functional.
Unless otherwise stated, we will work with this $\ZZ/2$ grading for the remainder of this section.

\parsec[]
We proceed to show that $S_{BF,\infty} + g J$ solves the classical master equation.
For notational simplicity we will omit the integral symbol $\int^\Omega$.

%By apparent field type reasons, the equation \eqref{eq:2cond} is equivalent to the following two equations
%\[
%\{S_{BF,\infty}, J\} = 0, \qquad \{J,J\} = 0.
%\] 

It is immediate from the form of the BV bracket that $\{J,J\} = 0$, since $J$ depends only on the $\gamma$ field. 
It remains to check that $\{S_{BF,\infty},J\} = 0$. 
For the quadratic term in the BF action, we note that 
  \deq{
    \{\beta \wedge \div\mu, J\} = \frac12 \del\beta \wedge \del \gamma \wedge \del \gamma = 0,
  }
  because total derivatives are equivalent to zero as local functionals. 
  
The contribution from the remaining BF action takes the form
\[
    \left\{ \frac12 \frac{1}{1-\nu} \del\gamma \vee \mu^2, \frac16 \gamma \wedge \del\gamma \wedge \del \gamma \right\} =\frac12 (\mu \vee \del \gamma) \wedge \del \gamma \wedge \del \gamma .
\]
    %\frac12 \frac{1}{1-\nu} \bigg[ (\del\gamma\vee\mu) \wedge \del\gamma \wedge \del \gamma \pm \del\gamma \wedge (\del\gamma\vee\mu) \wedge \del \gamma \pm \del\gamma \wedge \del \gamma \wedge (\del\gamma\vee\mu) \bigg].
This expression is zero for symmetry reasons. 
Recall that $\del\gamma$ is a two-form, and that the expression must be a totally symmetric local functional which is cubic in this two-form. We can ask whether such a  contraction exists just at the level of $\lie{sl}(5)$ representation theory. Let $\ydiagram{1}$ denote the fundamental representation of~$\lie{sl}(5)$, which we identify with constant one-forms. Since the term must be a scalar, the contraction $(\partial\gamma^3$ must sit in the fundamental representation again, since it is dual to a vector field. Computing the decomposition of the tensor cube of the two-form, we find
\deq{
  \Sym^3 \left( \ydiagram{1,1} \right) \cong \ydiagram{3,3} \oplus \ydiagram{2,2,1,1}.
}
(In fact, the absence  of the relevant irreducible representation does not  even depend on the parity of the  field $\gamma$, since 
  \deq{
    \wedge^3\left( \ydiagram{1,1} \right) \cong \ydiagram{3,1,1,1}  \oplus \ydiagram{2,2,2};
  }
the  fundamental representation has symmetry type $\ydiagram{2,1}$.) 

\parsec[s:coupling]

We make note of the dependence on the coupling constant $g$ in the definition of the deformed action $S_{BF,\infty} + g J$. 

When $g = 0$ we return to BF theory for the $L_\infty$ algebra $\cL_\infty(\CC^5) \otimes \Omega^\bu(\RR)$. 
For any $g \ne 0$ the theories are essentially equivalent in perturbation theory. 
Indeed, if $g \ne 0$ we can make the following field redefinition 
\[
\gamma \mapsto \sqrt{g} \gamma, \quad \beta \mapsto \sqrt{g} \beta 
\]
to write the action as 
\[
\frac{1}{\sqrt{g}} \left(S_{BF,\infty} + J \right)  .
\]

In perturbation theory, this has the affect of modifying the quantization parameter $\hbar$ to $\hbar / \sqrt{g}$.
Thus, after modifying $\hbar$ and making the above field redefinition, the perturbative expansion of any theory is equivalent to the one with $g = 1$. 

\parsec[s:altdfn]

We remark on an alternative, equivalent, description of the deformed theory which involves the strict dg Lie algebra structure on divergence-free holomorphic vector fields.

We can replace $S_{BF,\infty}$ by $S_{BF,0}$ via applying the field automorphism \eqref{eqn:auto1}.
Doing this we see that $J$ becomes 
\[
\til{J} = \frac16 e^\nu \gamma \wedge \del (e^\nu \gamma) \wedge \del(e^\nu \gamma) .
\]
Since this automorphism preserves the odd BV bracket, the actions $S_{BF,\infty} + g J$ and $S_{BF, 0} + g \til{J}$ are both solutions to the classical master equations and are equivalent as~$\ZZ/2$ graded BV theories.

\subsection{Equations of motion of the component fields} \label{s:components}

Soon, we will provide a series of justifications for the assertion that the deformed theory $S_{BF, \infty} + g J$ is the minimal twist of eleven-dimensional supergravity on flat space $X \times S = \CC^5 \times \RR$ where $\CC^5$ is equipped with its flat Calabi--Yau form. 
For the moment, we briefly read off the equations of motion of the general theory on $X \times S$.
Let $\Omega$ denote the Calabi--Yau form on $X$. 

We consider the action $S_{BF, \infty} + gJ$.
The equation of motion obtained by varying $\beta$ is especially simple, in fact linear, since it only appears in the action via a quadratic term. 
It is
\beqn\label{eqn:eombeta}
\dbar \nu + \d_S \nu + \div \mu = 0 .
\eeqn
Varying $\gamma$ we obtain the equation of motion
\beqn\label{eqn:eomgamma}
\dbar \mu + \d_S \mu + \frac12 \frac{1}{1-\nu} \div (\mu^2) + \frac12 (\del \gamma \wedge \del \gamma) \vee (g \Omega^{-1}) = 0 .
\eeqn
The last term represents the contraction of an element of $\Omega^{4,\bu}(X;S)$ with the nonvanishing section $\Omega^{-1} \in \PV^{5,\bu}(X;S)$ to yield an element of $\PV^{1,\bu}(X;S)$. 
If we vary the $\mu$ we obtain 
\beqn\label{eqn:eommu}
(\dbar + \d_S) \gamma + \del \beta + \frac{1}{1-\nu} (\mu \vee \del \gamma) = 0 .
\eeqn
Finally, if we vary $\nu$ we obtain
\beqn\label{eqn:eomnu}
(\dbar + \d_S) \beta + \frac12 \frac{1}{(1-\nu)^2} \mu^2 \vee \del \gamma = 0 .
\eeqn

The equation of motion must hold for any inhomogenous superfields.
We can get a better sense of the equations if we expand in components of these fields. 
The component fields of the eleven-dimensional theory on $X \times S$ have the following form: 
\begin{itemize}
\item $\mu = \sum_{i,j} \mu^{i;j}$ is a superfield where
\[
\mu^{i;j} \in \PV^{1,i}(X) \otimes \Omega^j(\RR) ,\quad i=0,\ldots, 5, \quad j=0,1.
\]
The component $\mu^{i;j}$ has parity $i+j+1 \mod 2$. 
\item $\nu = \sum_{i,j} \nu^{i;j}$ is a superfield where
\[
\nu^{i;j} \in \PV^{0,i}(X, \T_X) \otimes \Omega^j(\RR) ,\quad i=0,\ldots, 5, \quad j=0,1.
\]
The component $\nu^{i;j}$ has parity $i+j \mod 2$. 
%The linear equations of motion state that $\mu$ is constant along $\RR$ and holomorphic divergence-free as a vector field on $\CC^5$. 
\item 
$\gamma = \sum_{i,j} \gamma^{i;j}$ is a superfield where
\[
\gamma^{i;j} \in \Omega^{1,i}(X) \otimes \Omega^j(\RR) ,\quad i=0,\ldots, 5, \quad j=0,1.
\]
The component $\gamma^{i;j}$ has parity $i+j$. 
\item 
\item $\beta = \sum_{i,j} \beta^{i;j}$ is a superfield where
\[
\beta^{i;j} \in \Omega^{0,i}(X) \otimes \Omega^j(\RR) ,\quad i=0,\ldots, 5, \quad j=0,1.
\]
The component $\beta^{i;j}$ has parity $i+j+1 \mod 2$. 
\end{itemize}

We look closely at the geometric meaning of \eqref{eqn:eombeta}. 
Let's make the simplifying assumption that all components of $\mu$ are divergence-free that all fields are locally constant along $S$: $\div \mu = 0$ and $\d_S \mu = \d_S \gamma = 0$.
Then, $\nu = 0$ is a solution to \eqref{eqn:eombeta} and we can assume that all fields are functions, or zero-forms, along $S$. 
Then, there is a component of \eqref{eqn:eomgamma} which can be written as 
\beqn\label{eqn:eomgamma1}
\dbar \mu^{1;0} + \frac12 [\mu^{1;0},\mu^{1;0}] + \left(\frac12 \del \gamma^{1;0} \wedge \del \gamma^{1;0} + \del \gamma^{2;0} \wedge \del \gamma^{0;0}\right) \vee (g \Omega^{-1}) = 0 
\eeqn
where now $[-,-]$ stands for the Schouten bracket.

To further simplify \eqref{eqn:eomgamma1}, we can look for a solutions where the $(0,1)$ Dolbeault part of $\gamma$ is zero $\gamma^{1;0} = 0$. 
Then, up the term involving 
\[
\alpha \define \del \gamma^{0;0},
\]
we find precisely the integrability equation for the complex structure determined by Beltrami differential $\mu^{1;0} \in \PV^{1,1} \otimes \Omega^0$. 
If $\dbar \alpha = 0$, the holomorphic two-form $\alpha \in \Omega^{2,hol}(X)$ defines a map of sheaves
\[
\Omega^{2,hol}_X \xto{\wedge \alpha} \Omega^{4,hol}_X \cong_\Omega \cT^{hol}_X
\]
where $\cT^{hol}_X$ denotes the sheaf of holomorphic vector fields and the last isomorphism uses the Calabi--Yau form $\Omega$ on $X$.  
The image of $\Omega^{2,hol}_X$ defines a subsheaf $\cF_{\alpha} \subset \cT^{hol}_X$. 
Since $\del \alpha = 0$, this subsheaf is automatically integrable and hence determines a foliation. 

Summarizing, see that there is a field configuration where the Beltrami diffrential $\xi = \mu^{1;0} \in \Omega^{0,1}(X, \T_X)$ satisfies the modified integrability condition
\[
\dbar \xi + \frac12  [\xi , \xi] = \alpha \vee \rho
\]
for some $\rho \in \PV^{2,2} (X)$. 
In other words, $\xi$ defines an integrable complex structure deformation along the leaf space associated to the foliation $\cF_\alpha$. 
We leave a more complete exploration of the moduli space of solutions of the equations of motion for future work. 

In \cite{SWspinor}, the second two authors showed that the free limit of the minimal twist of eleven-dimensional supergravity agrees with the free limit of the eleven-dimensional theory that we have introduced here. 
Given this result, we can recognize many fields in the twisted theory as components of the physical fields of supergravity which remain after we twist. 

\begin{itemize}
\item 
The following components of $\mu$:
\begin{align*}
\mu^{1;0} & = \mu^j_i(z,\zbar,t) \d \zbar_j \partial_{z_i} \\
\mu^{0;1} & = \mu^t_i (z,\zbar,t) \d t \partial_{z_i}
\end{align*}
comprise components of the metric which remain after the twist. 
The components 
\[
\mu^{0;0} = \mu_i (z,\zbar,t) \partial_{z_i} 
\]
comprise the ghosts for infinitesimal diffeomorphisms. 
\item 
The following three-form fields
\begin{multline}
\beta^{3;0} = \beta^{ijk} (z,\zbar,t) \d \zbar_i \d \zbar_j \d \zbar_k , \quad \beta^{2;1} = \beta^{ij}_t (z,\zbar,t) \d \zbar_i \d \zbar_j \d t \\
\gamma^{2;0} = \gamma^{ijk} (z,\zbar,t) \d z_i \d \zbar_j \d \zbar_k , \quad \gamma^{1;1} = \gamma^{ij}_t (z,\zbar,t) \d z_i \d \zbar_j \d t .
\end{multline} 
comprise components of the supergravity $C$-field which remain after the twist. 
The two-form fields $\beta^{2;0}, \beta^{1;1}, \gamma^{1;0}, \gamma^{0;1}$, the one-form fields $\beta^{1;0}, \beta^{0;1}$, and the zero-form field $\beta^{0;0}$ is what remains of the ghost system (ghosts, ghosts for ghosts, etc.) for the supergravity $C$-field. 
\end{itemize}
%The most geometrically relevant component is the case where 
%\[
%\mu^{1;0} \in \PV^{1,1}(X) \otimes \Omega^0(S) 
%\]
%which is an even field in our $\ZZ/2$ graded BV theory.
%The superscript denotes anti-holomorphic; de Rham form type. 
%For this component of $\mu$, the only components of $\gamma$ which appear in the above equations of motion are $\gamma^{0;0}$, $\gamma^{1;0}$, and $\gamma^{2;0}$.
%Expanding these components out, we obtain
%\beqn\label{eqn:eomgamma1}
%\dbar \mu^{1;0} + \frac12 [\mu^{1;0},\mu^{1;0}] + (\del \gamma^{0;0} \wedge \del \gamma^{2;0}) \vee (g \Omega^{-1}) + \frac12 (\del \gamma^{1;0} \wedge \del \gamma^{1;0}) \vee (g \Omega^{-1}) = 0 .
%\eeqn

\subsection{Local character}\label{sec:locchar}

We consider the eleven-dimensional theory on the manifold $\CC^5 \times \RR$ where $\CC^5$ is equipped with its standard Calabi--Yau structure. 
On this background, the theory is manifestly $SU(5)$ invariant. 
In this section, we compute the corresponding character of the local operators at the origin. 

The local character is only sensitive to the free limit of the theory.
Furthermore, the linear BRST operator is an $SU(5)$-invariant deformation of the $(\dbar + \d_{\RR})$ operator. 
Therefore, to compute the character it suffices to compute the $SU(5)$-equivariant character of the $\dbar$ cohomology. 

The solutions to the $(\dbar + \d_{\RR})$-equations of motion simply say that all fields are holomorphic along $\CC^5$ and constant along $\RR$. 
Thus, the solutions can be identified with 
\begin{align*}
\mu^{i}\partial_{z_i} & \in \Vect(\CC^5) \cong \cO(\C^5)\partial_{z_i},\quad 
\nu \in \cO (\C^5) \\
\beta & \in \cO (\C^5), \quad \gamma^{i} \d z_i \in \Omega^{1}(\CC^5) \cong \cO (\C^5)\d z_i 
\end{align*}
where $z_i$ is a holomorphic coordinate on $\CC^5$. 

Corresponding to each of the above, we have a tower of linear local operators labeled by $(m_j) = (m_1, m_2, m_3, m_4, m_5)\in \Z^5_{\geq 0}$; these are given by
\begin{align*}
 \mu^{i}_{(m_j)} &: \mu^{i}\mapsto \partial_{z_1}^{m_1}\partial_{z_2}^{m_2}\partial_{z_3}^{m_3}\partial_{z_4}^{m_4}\partial_{z_5}^{m_5}\mu^{i} (0) \\
\nu_{(m_j)} &: \nu\mapsto \partial_{z_1}^{m_1}\partial_{z_2}^{m_2}\partial_{z_3}^{m_3}\partial_{z_4}^{m_4}\partial_{z_5}^{m_5}\nu (0) \\
\gamma^{i}_{(m_j)} &: \gamma^{i}\mapsto \partial_{z_1}^{m_1}\partial_{z_2}^{m_2}\partial_{z_3}^{m_3}\partial_{z_4}^{m_4}\partial_{z_5}^{m_5}\gamma^{i} (0) \\
 \beta_{(m_j)} &: \beta\mapsto \partial_{z_1}^{m_1}\partial_{z_2}^{m_2}\partial_{z_3}^{m_3}\partial_{z_4}^{m_4}\partial_{z_5}^{m_5}\beta (0) \\
\end{align*}

\iffalse
We choose generators of the Cartan subgroup for the $SU(5)$ action with the following weights:

\[\begin{array}{|c|c|c|c|c|c|}
& z_1 & z_2 & z_3 & z_4 & z_5 \\
\hline
q & & & & 1 & -1 \\
t_1 & 1 & -1 & & & \\
t_2 & & 1 & -1 & & \\
y & -1 & -1 & -1 &\frac 3 2 & \frac 3 2
\end{array}\]
We are choosing the weights in this way as a matter of convenience for the later sections. 
For instance, upon performing the further twist of the eleven-dimensional theory, the $SU(5)$ symmetry is broken to an $SU(3)\times SU(2)$ symmetry; the Cartan of the unbroken symmetries corresponds to the fugacities $q, t_1, t_2$ in the above table. 

We can now readily compute the characters.
\begin{prop}
The $SU(5)$ local character of the holomorphic twist of the eleven-dimensional theory on flat space is
\begin{multline}
\chi_{SU(5)} = 
\prod_ {(m_j)\in \Z^5_{\geq 0}} \frac{1- t_1^{-m_1+m_2}t_2^{-m_2+m_3}q^{-m_4+m_5}y^{m_1+m_2+m_3-\frac 3 2 (m_4+m_5)}}{1- t_1^{-m_1+m_2}t_2^{-m_2+m_3}q^{-m_4+m_5}y^{m_1+m_2+m_3-\frac 3 2 (m_4+m_5)} }
\\ 
\times \frac{t_1^{-1}y + t_1t_2^{-1}y + t_2y + q^{-1}y^{-\frac 3 2} + qy^{-\frac 3 2}}{t_1y^{-1} + t_1^{-1}t_2y^{-1} + t_2^{-1}y^{-1} + q^{-1}y^{\frac 3 2} + qy^{\frac 3 2}}.
\end{multline}
\end{prop}
\begin{proof}
We compute the local character as the plethystic exponential of the character of linear local operators. 

Note that the linear local operators $ \nu_{(m_j)}$ and $\beta_{(m_j)}$ are of the same weight but opposite parity so contribute to the character with opposite sign. These contributions therefore cancel. Next, there is a summand of the linear local operators of the form 
\[
\bigoplus _{(m_j)\in \Z^5_{\geq 0 }} \Pi \left ( \C\mu^{z_1}_{(m_j)}\partial_{z_1}\oplus \C\mu^{z_2}_{(m_j)}\partial_{z_2}\oplus \C\mu^{z_3}_{(m_j)}\partial_{z_3}\oplus\C\mu^{w_1}_{(m_j)}\partial_{w_1}\oplus \C\mu^{w_2}_{(m_j)}\partial_{w_2}\right).
\] 
This contributes 
\[
\sum_{(m_j)\in \Z^5_{\geq 0 }}- t_1^{-m_1+m_2}t_2^{-m_2+m_3}q^{-m_4+m_5}y^{m_1+m_2+m_3-\frac 3 2 (m_4+m_5)}\left (t_1^{-1}y + t_1t_2^{-1}y + t_2y + q^{-1}y^{-\frac 3 2} + qy^{-\frac 3 2} \right).
\] 
Finally, there is a summand of the form 
\[
\bigoplus _{(m_i;n_j)\in \Z^5_{\geq 0 }} \Pi \left ( \C\gamma^{z_1}_{(m_i; n_j)}d{z_1}\oplus \C\gamma^{z_2}_{(m_i; n_j)}d{z_2}\oplus\C\gamma^{z_3}_{(m_i; n_j)}d{z_3}\oplus\C\gamma^{w_1}_{(m_i; n_j)}d{w_1}\oplus \C\gamma^{w_2}_{(m_i; n_j)}d {w_2}\right).
\] 
This likewise contributes 
\[
\sum_{(m_j)\in \Z^5_{\geq 0 }}t_1^{-m_1+m_2}t_2^{-m_2+m_3}q^{-n_1+n_2}y^{m_1+m_2+m_3-\frac 3 2 (m_4+m_5)}\left (t_1y^{-1} + t_1^{-1}t_2y^{-1} + t_2^{-1}y^{-1} + q^{-1}y^{\frac 3 2} + qy^{\frac 3 2} \right).
\] 
In sum, the character of linear local operators is the geometric series 
\[
\sum_{(m_j)\in \Z^5_{\geq 0 }}-t_1^{-m_1+m_2}t_2^{-m_2+m_3}q^{-n_1+n_2}y^{m_1+m_2+m_3-\frac 3 2 (m_4+m_5)}\left (\begin{aligned}t_1y^{-1} + t_1^{-1}t_2y^{-1} + t_2^{-1}y^{-1} + q^{-1}y^{\frac 3 2} + qy^{\frac 3 2} \\  - (t_1^{-1}y + t_1t_2^{-1}y + t_2y + q^{-1}y^{-\frac 3 2} + qy^{-\frac 3 2})\end{aligned}\right).
\] 
The plethystic exponential returns the desired expression.

\end{proof}
\fi

It is easiest to label the Cartan subgroup of $SU(5)$ by fugacities $q_1,\ldots, q_5$ subject to the constraint that $\prod_{i=1}^5 q_i = 1$. 
We first compute the single particle index.
This is the $SU(5)$ character of the space of linear local operators.

\begin{lem}
The single particle index is 
\[
i(q_1,\ldots,q_5) = \frac{\sum_{i=1}^5 q_i}{\prod_{i=1}^5 (1-q_i)} + \frac{\sum_{i=1}^5 q_i^{-1}}{\prod_{i=1}^5 (1-q_i^{-1})}
\]
where the fugacities satisfy the constraint $\prod_{i=1} q_i = 1$. 
\end{lem}
\begin{proof}
The linear local operators $ \nu_{(m_j)}$ and $\beta_{(m_j)}$ are of the same $q$-weight but opposite parity.
Thus, they do not contribute to the single particle index.

The $q$-weight of the odd local operator $\mu_{(m_j)}^i$ is 
\[
q_1^{m_1+1} \cdots q_i^{m_i} \cdots q_5^{m_5+1} .
\]
The $q$-weight of the even local operator $\gamma_{(m_j)}^i$ is 
\[
q_1^{m_1} \cdots q_i^{m_i + 1} \cdots q_5^{m_5} .
\]

Thus we find that the single particle index is given by the infinite series
\beqn\label{infseriesindex}
\sum_{i=1}^5\left ( \sum_{(m_i)\in \Z^5} q_1^{m_1} \cdots q_i^{m_i + 1} \cdots q_5^{m_5} - \sum_{(m_i)\in \Z^5} q_1^{m_1+1} \cdots q_i^{m_i} \cdots q_5^{m_5+1} \right)
\eeqn

which sums to the expression
\beqn\label{singleparticleindex}
- \frac{\sum_{i=1}^5 q_1 \cdots \Hat{q_i} \cdots q_5}{\prod_{i=1}^5 (1-q_i)} + \frac{\sum_{i=1}^5 q_i}{\prod_{i=1}^5 (1-q_i)} .
\eeqn

This simplifies to the stated expression.
\end{proof}

This single particle index for our space of local operators agrees with the one computed in \cite{NekrasovInstanton}. 
To obtain the full index of local operators we apply the plethystic exponential ${\rm PE}[f(x)] = \exp\left(\sum_n \frac1n f(x^n)\right)$. 

\begin{prop}\label{prop:locchar}
The character of local operators of the eleven-dimensional theory on $\CC^5 \times \RR$ is 
\[
\prod_{i=1}^{5} \prod_{(m_i)\in \Z^5} \frac{1-q_1^{m_1+1}\cdots q_i^{m_i}\cdots q_5^{m_5+1}}{1-q_1^{m_1}\cdots q_i^{m_i+1}\cdots q_5^{m_5}}
\]
\end{prop}
\begin{proof}
Recall that the plethystic exponential takes sums to products and monomials to geometric series. Apply this to the infinite series \eqref{infseriesindex}.
\end{proof}
%\[
%\prod_ {(m_j)\in \Z^5_{\geq 0}}\frac{1- t_1^{-m_1+m_2}t_2^{-m_2+m_3}q^{-n_1+n_2}y^{m_1+m_2+m_3-\frac 3 2 (m_4+m_5)}\left (t_1^{-1}y + t_1t_2^{-1}y + t_2y + q^{-1}y^{-\frac 3 2} + qy^{-\frac 3 2} \right)}{1- t_1^{-m_1+m_2}t_2^{-m_2+m_3}q^{-m_4+m_5}y^{m_1+m_2+m_3-\frac 3 2 (m_4+m_5)}\left (t_1y^{-1} + t_1^{-1}t_2y^{-1} + t_2^{-1}y^{-1} + q^{-1}y^{\frac 3 2} + qy^{\frac 3 2} \right)}.
%\]


%The linear equations of motion for the superfield $\mu$ read 
%\[
%\dbar \mu^{i;j} + \d_{S} \mu^{i+1, j-1} = 0, \quad \div \mu^{i,j} = 0
%\]
%for all $i,j$. 
%The operator $\dbar$ is the anti-holomorphic Dolbeault operator acting on $\CC^5$, $\d_S$ is the de Rham operator on $S$, and $\div$ is the divergence with respect to the fixed holomorphic volume form on $\CC^5$. 
%In particular, this states that $\mu^{i;0}$ is locally constant along $S$ and holomorphic divergence-free as a Dolbeault valued vector field on $\CC^5$.

\subsection{One-loop quantization}

In \cite{GRWthf} an existence result for one-loop quantizations of mixed topological-holomorphic theories was established. 
We apply this to the eleven-dimensional model at hand. 

The eleven-dimensional theory is a mixed topological-holomorphic theory.
On flat space $\CC^5_z \times \RR_t$, this means that the theory is translation invariant and that the following act homotopically trivially:
\begin{itemize}
\item the vector fields $\del_{\zbar_1}, \ldots, \del_{\zbar_{5}}$ corresponding to infinitesimal anti-holomorphic translations,
\item the vector field $\partial_t$ corresponding to infinitesimal translations in the $\RR_t$ direction. 
\end{itemize}

Recall that the action functional of the eleven-dimensional theory is $S_{BF, \infty} + c J$. 
Since the cubic and higher interactions only involve holomorphic derivatives, we obtain the following directly from the main result of \cite{GRWthf}. 

\begin{thm}
There exists a gauge fixing condition for the eleven-dimensional theory on $\CC^5 \times \RR$ which renders its one-loop quantization finite and anomaly-free. 
\end{thm} 

When $g=0$, this result is actually exact
since there are no Feynman diagrams present past one-loop in this case. 
When $g \ne 0$, on the other hand, this result does not immediately imply the existence of a gauge invariant perturbative quantization to higher orders in $\hbar$. 
The presence of the functional $J = \frac16 \int \gamma \del \gamma \del \gamma$ allows one to construct Feynman graphs at arbitrary loop order. 

Upon performing the $\Omega$-background,
%see \S \ref{s:omega},
In \cite{CostelloM5}, Costello argues that the theory localizes to a five-dimensional theory on $\CC^2 \times \RR$. 
Via a cohomological argument, it is shown that this effective five-dimensional theory exhibits an essentially unique quantization in perturbation theory. 
We will return to the existence and uniqueness of a higher order quantization of the eleven-dimensional theory in future work. 
